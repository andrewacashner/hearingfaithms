% Voces-versions.tex
% TODO redo with original orthography?
\documentclass{aac-table}
\newcommand{\str}{\hspace{1em}}
\begin{document}
\begin{longtable}{lll}
    \toprule
    1642 Lisbon (Santiago?) & 
    1647 Seville (Jalón?) & 
    1657 Puebla (Padilla) \\
    \midrule \endhead
    \bottomrule\endfoot

    \str{} Voces las de la capilla,    &
    \str{} Cantores \uline{de la Capilla,} &
    \str{} \uline{Voces las de la capilla,} \\

    cuenta con lo que se canta, &
    \uline{cuenta con lo que se canta,} &
    \uline{cuenta con lo que se canta} \\

    que es músico el Rey, y nota &
    \uline{que es Músico el} Niño, \uline{y nota} &
    \uline{que es músico el Rey, y nota} \\

    las más leves disonancias. &
    \uline{las más leves disonancias.} &
    \uline{las más leves disonancias} \\

    & 
    \str{} La música que componó 
    & \\

    & 
    de vozes altas y bajas, 
    & \\

    & 
    a compás mayor las rige, 
    & \\

    & 
    y es proporción abreviada. 
    & \\

 
    \str{} A lo de Jesús infante & 
    \str{} Una \uline{clave} con \uline{tres} tiempos &
    \uline{a lo de Jesús infante} \\

    y a lo de David monarca, & 
    \uline{pone} con \uline{destreza} tanta, &
    \uline{y a lo de David monarca}. \\

    puntos ponen a sus letras,
    & que el pasado y el futuro &
    \str{} \uline{Puntos ponen a sus letras} \\

    los siglos de sus hazañas.
    & al \uline{compás} presente iguala. &
    \uline{los siglos de sus hazañas.} \\

    \str{} La clave, que sobre el hombre
    & \str{} Un Coro errado enmendó &
    \uline{La clave que sobre el} hombro \\

    para el treinta y tres se guarda,
    & con un \uline{medio}, que a la entrada &
    \uline{para el treinta y tres se aguarda.} \\

    años antes la divisa,
    & puso, y una espiración & 
    \str{} \uline{Años antes la divisa,} \\

    la destreza, en la esperanza.
    & que \uline{para el} Calvario \uline{guarda.} & 
    \uline{la destreza en la esperanza} \\

    \str{} Por sol comienza una gloria. & 
    (cf. Seville copla 4) & 
    \uline{por sol comienza una gloria,} \\

    por mi se canta una gracia, &
    &
    \uline{por mi se canta una gracia,} \\

    y a medio compás la noche, &
    &
    \uline{y a medio compás la noche} \\

    remeda quiebros del alba. &
    &
    \uline{remeda quiebros del alba.} \\
  
    & &
    \str{} Puntos ponen a sus letras \\
   
    & & 
    los siglos de sus hazañas. \\

    & & 
    La clave que sobre el hombre \\

    & & 
    para el treinta y tres se aguarda. \\

    \addlinespace

    \textsc{Estribo} &
    \textsc{Estribo} &
    \\

    \str{} Y a trechos las estancias, &
    &
    \str{} \uline{Y a trechos las distancias} \\

    en uno, y otro coro, &
    &
    \uline{en uno y otro coro,} \\

    grave, suave, sonoro, &
    &
    \uline{grave, suave y sonoro,} \\

    hombres, y brutos, y Dios. &
    &
    \uline{hombres y brutos y Dios,} \\

    tres a tres, y dos a dos &
    &
    \uline{tres a tres y dos a dos,} \\

    uno a uno, &
    &
    \uline{uno a uno,} \\
    
    & 
    O que lindamente suenan! 
    & \\
    
    & 
    o que dulcemente cantan 
    & \\
    
    & 
    al compás que lleva \uline{el Infante,} 
    & \\
    
    & 
    Serafines que cruzan y passan! 
    & \\
    
    & 
    y de sus gemidos 
    & \\
    
    & 
    aprended trinados y sustenidos, 
    & \\
    
    & 
    y con mil primores 
    & \\ 
    
    & 
    responden los Reyes y los Pastores, 
    & \\
 
    & 
    después que \uline{aguardaron uno} 
    & \\

    y aguarda tiempo oportuno, &
    que llegó a \uline{tiempo oportuno}, &
    y aguardan \uline{tiempo oportuno,} \\ 
    
    quién antes del tiempo fue: &
    \uline{quien antes del tiempo fue} &
    \uline{quién antes del tiempo fue}. \\

    por el signo a la mi re, &
    \uline{por el signo a la mi re,} &
    \uline{Por el signo a la mi re,} \\

    puestos los ojos en mi &
    \uline{puestos los ojos en mi,} &
    \uline{puestos los ojos en mi,} \\

    a la voz del padre oí &
    con que mil maravillas vi &
    \uline{a la voz del padre oí} \\

    cantan por puntos de llanto, &
    &
    cantar \uline{por puntos de llanto.} \\

    O qué canto &
    &
    \uline{O qué canto} \\

    tan de oír, y de admirar! &
    \uline{tan de ir} [sic] \uline{y de admirar,} &
    \uline{tan de oír y de admirar,} \\

    & 
    que si lo acierto a dezir, &
    tan de admirar y de oír. \\

    todo en el hombre es subir, &
    \uline{todo en el hombre es subir,} &
    \uline{Todo en el hombre es subir} \\

    y todo en Dios es bajar, &
    \uline{y todo en Dios es bajar.} &
    \uline{y todo en Dios es bajar.} \\

    y con el favor usanos, &
    & \\

    los corazones humanos, &
    & \\

    vuelven guecos. &
    & \\


    \str{} \speaker{Eco} Ecos, &
    & \\

    pues zagalejos: &
    & \\

    \str{} \speaker{Eco} Lejos, &
    & \\

    desvalidos, &
    & \\

    \str{} \speaker{Eco} Idos, &
    & \\

    distraídos, &
    & \\

    \str{} \speaker{Eco} Traídos, &
    & \\

    son sustenidos, &
    & \\

    \str{} \speaker{Eco} tenidos, &
    & \\

    del niño hermoso, &
    & \\

    suena blando el arrullo, &
    & \\

    y en la tropa el orgullo &
    & \\

    travieso del viento &
    & \\

    por aquí por allí contento, &
    & \\

    inquieto, y bullicioso, &
    & \\

    se penetra blando, &
    & \\

    y discurre airoso.
    & \\

    \addlinespace
    \newpage
    &
    &
    \textsc{Coplas} \\


    \str{} A suspensiones del cielo. &
    \str{} \uline{A Suspensiones} \uline{el Cielo} & 
    \\

    hacen sus esferas pausas, &
    \uline{hace} \uline{en sus esferas pausas,} &
    \\

    porque el Ángel, que las mueve &
    \uline{porque el Ángel que} los \uline{mueve} &
    \\

    cuenta le admira, se pasma. &
    \uline{cuanto le admira} le \uline{pasma.} &
    \\

    \str{} Los compases son del tiempo, &
    \str{} \uline{Los compases son del tiempo,} &
    \\

    que ya sus compases guarda, &
    \uline{que ya sus compases guarda} &
    \\

    quien al tiempo lleva siglos, &
    \uline{quien al tiempo lleva siglos,} &
    \\

    quien lleva al siglo distancias. &
    \uline{quien lleva al siglo distancias.} &
    \\

    \str{} Toda la solfa, la cifran, &
    & \\

    relieves de nieve, y grana, &
    & \\

    en dos labios que rubrica, &
    & \\

    y en dos mejillas que escarcha. &
    & \\

    \str{} Los puntos son cuantas perlas &
    & \\

    dos vivos diamantes passan, &
    & \\

    de si mismos, que las vierten &
    & \\

    a un pesebre que las guarda. 
    & \\

    & 
    \str{} Tambien se canta a ternario, 
    & \\

    & 
    pues entran, caben y passan 
    & \\

    & 
    tres Reyes en un compas, 
    & \\
    
    &
    de \uline{brutos} \uline{breve} morada. 
    & \\


    & 
    \str{} \uline{Por sol comienza una gloria,}  &
    \\

    & 
    \uline{por mi se canta una gracia,} & 
    \\
    
    & \uline{y a medio compás la Noche} & 
    \\

    & \uline{remeda quiebros del Alba.} &
    \\

    & 
    &
    \str{} Daba un niño peregrino \\
    
    &
    &
    tono al hombre y subió tanto \\
    
    &
    &
    que en sustenidos de llanto \\
    
    &
    &
    dió octava arriba en un trino. \\
    
    &
    &
    \str{} Hizo alto en lo divino \\
    
    &
    &
    y de la máxima y breve \\
    
    &
    &
    composición en que pruebe \\
    
    &
    &
    de un hombre y Dios consonancias. \\

    \str{} \emph{Y a trechos \&c.}
    & \str{} \emph{O que lindamente \&c.}
    & \str{} \emph{Y a trechos \&c.} \\

\end{longtable}
\end{document}
