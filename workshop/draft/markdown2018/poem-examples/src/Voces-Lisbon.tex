% Voces las de la capills
% Lisbon imprint 1642, no. 1

% Courtesy Alvaro Torrente, need to ask Rui Cabral Lopes for signature and
% permission

% 2014/04/09

\documentclass{aac-poem}
\begin{document}
\numberlinetrue
    \begin{poemoriginal}
        \StanzaSection{4}[\add{Introducción}]
        Voces las de la capilla, &
        cuenta con lo que se canta, &
        que es músico el Rey, y nota &
        las más leves disonancias. \&

        \Stanza{4}
        A lo de Jesús infante &
        y a lo de David Monarcha, &
        puntos ponen a sus letras, &
        los siglos de sus hazañas. \&

        \Stanza{4}
        La clave, que sobre el hombre &
        para el trinta y tres se guarda, &
        años antes la divisa, &
        la destreza, en la esperanza. \&

        \Stanza{4}
        Por sol comienza una gloria. &
        por mi se canta una gracia, &
        y a medio compás la noche, &
        remeda quiebros del alba.
        \SectionBreak

        \StanzaSection{19}[Estribo]
        Y atrechos las estancias, &
        en uno, y otro coro, &
        grave, suave, sonoro, &
        hombres, y brutos, y Dios. &
        tres a tres, y dos a dos &
        uno a uno, &
        y a guarda tiempo oportuno, &
        quien antes del tiempo fue: &
        por el signo a la mi re, &
        puestos los ojos en mí &
        a la voz del padre oí &
        cantan por puntos de llanto, &
        o que canto &
        tan de oír, y de admirar! &
        todo en el hombre es subir, &
        y todo en Dios es bajar, &
        y con el favor usanos, &
        los corazones humanos, &
        vuelven guecos. \&
        \Stanza{2}
        \speaker{Eco} Ecos, &
        pues zagalejos: \&
        \Stanza{2}
        \speaker{Eco} Lejos, &
        desvalidos, \&
        \Stanza{2}
        \speaker{Eco} Idos, &
        distraídos, \&
        \Stanza{2}
        \speaker{Eco} Traídos, &
        son sustenidos, \&
        \Stanza{9}
        \speaker{Eco} tenidos, &
        del niño hermoso, &
        suena blando el arrullo, &
        y en la tropa el orgullo &
        travieso del viento &
        por aquí por allí contento, &
        inquieto, y bullicioso, &
        se penetra blando, &
        y discurre airoso.
        \SectionBreak

        \StanzaSection{4}[\add{Coplas}]
        A suspensiones del cielo. &
        hacen sus esferas pausas, &
        porque el Ángel, que las mueve &
        cuenta le admira, se pasma. \&

        \Stanza{4}
        Los compases son del tiempo, &
        que ya sus compases guarda, &
        quien al tiempo lleva siglos, &
        quien lleva al siglo distancias. \&

        \Stanza{4}
        Toda la solfa, la cifran, &
        relieves de nieve, y grana, &
        en dos labios que rubrica, &
        y en dos mejillas que escarcha. \&

        \Stanza{4}
        Los puntos son cuantas perlas &
        dos vivos diamantes passan, &
        de si mismos, que las vierten &
        a un pesebre que las guarda. 
        \SectionBreak[\add{Repeat estribo}]
    \end{poemoriginal}
\end{document}
