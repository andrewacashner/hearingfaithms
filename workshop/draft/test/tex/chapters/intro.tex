% Andrew Cashner, Book on Villancicos, Intro chapter
% 2015-08-27 begun

\chapter{Villancicos as Musical Theology}

\epigraph{%
\foreign{ergo fides ex auditu/ auditus autem per verbum Christi}

Thus faith [comes] from hearing; and hearing, by the word of Christ.%
}{St.~Paul, Romans 10:17}


St.~Paul taught that faith came by means of hearing, and one of the distinctive effects of the sixteenth-century reformations of Western Christianity was that Christians discovered new ways to make their faith audible.
Voices raised in acrid contention or pious devotion boomed from pulpits, clamored in public squares, and were echoed in homes and schools.
In new forms of vernacular music, the voices of the newly distinct communities united to articulate their own vision of Christian faith.
Catholic reformers and missionaries enlisted music in their campaigns to educate, evangelize, and build Christian civilization, both in an increasingly divided Europe and in the new domains of the Spanish crown across the globe. 
In these efforts to make \enquote{the word of Christ} to be heard and believed, then, what was the role of music?
What kind of power did Catholics believe music had over the dynamics of hearing and faith?

This book is a study of how Christians in early modern Spain and Spanish America enacted religious beliefs about music through the medium of music itself.
It focuses on villancicos, a widespread genre of devotional poetry and musical performance, for two primary reasons.
First, these pieces were actively employed by the Spanish church and state as tools for propagating faith.
By the seventeenth century villancicos had grown in to a complex, large-scale form of vocal and instrumental music based on poetic texts in the vernacular, and they were performed in and around liturgical celebrations on all the major feast days, across the Spanish world.

In their poetic themes and in their musical content, villancicos combined elements of elite and common culture. 
In subject matter as well as in the places and occasions of their performance, villancicos stood on the threshold between the world in and outside of church (which is not quite the same as a modern divide between sacred and secular). 
The chorus of Puebla Cathedral in colonial Mexico, singing a villancico by chapelmaster Juan Gutiérrez de Padilla on Christmas Eve 1653, expressed the boundary-crossing nature of villancicos through a food metaphor (\poemref{poem:A_la_jacara_jacarilla}).
Sets of villancicos featured dramatic, often comic texts reminiscent of Spanish minor theater (\term{teatro menor}) alongside cultivated and even arcanely sophisticated theological reflections.
The music for villancicos covered a wide stylistic range from old-style polyphonic techniques to highly rhythmic music drawing on dance traditions, as in \worktitle{A la jácara, jacarilla} (\musicref{mus:Padilla-A_la_jacara_jacarilla}).

%********************
\begin{poemexample}
\inputpoemexample{A_la_jacara_jacarilla}
\caption{\worktitle{A la jácara, jacarilla}, poem as set by Juan Gutiérrez de Padilla}
\label{poem:A_la_jacara_jacarilla}
\end{poemexample}
%********************

%********************
\begin{musicexample}
%\includescore{Padilla-A_la_jacara_jacarilla}
\caption{\worktitle{A la jácara, jacarilla}, setting by Padilla (Puebla, 1653), opening}
\label{mus:Padilla-A_la_jacara_jacarilla}
\end{musicexample}
%********************

Villancicos are valuable, then, for assessing the interaction of these distinctive elements that meet within the genre.
Were villancicos a form of top-down \socalled{propaganda} intended to indoctrinate and control, as some have claimed of post-Tridentine religious arts?
Or were they a grassroots expression of popular devotion? 
Did they work on multiple levels, even contradictory ones?

