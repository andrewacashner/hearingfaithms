% FLOATS
% Create new environments for music examples, poem examples, diagrams
\RequirePackage{graphicx}
\RequirePackage{quoting}
\RequirePackage{longtable}

\RequirePackage{newfloat}
\RequirePackage{tocloft}

% Command to create all needed elements for a new float type
% #1 environment command name
% #2 file extension for storing float info
% #3 name of float to be used in labels
% #4 string used in title of list of floats
\NewDocumentCommand{\@NewFloatAndList}{ m m m m }{%
    \DeclareFloatingEnvironment[
        fileext=#2,
        name=#3,
        listname={List of #4},
        within=chapter
    ]{#1}%
}

% Create the float types
\@NewFloatAndList{musicexample} {mus}  {Music example} {Music Examples}
\@NewFloatAndList{poemexample}  {poem} {Poem example}  {Poem Examples}
\@NewFloatAndList{diagram}      {dia}  {Diagram}       {Diagrams}
\@NewFloatAndList{map}          {map}  {Map}           {Maps}

% Command to create reference labels, uppercase and lowercase
% = substitute for cleveref
% #1 lowercase abbreviation, used in command: `fig` -> `\figref`
% #2 lowercase label
% #3 uppercase abbreviation 
% #4 uppercase label
\NewDocumentCommand{\@NewRefCommand}{ m m m m }{%
    \expandafter\NewDocumentCommand\csname#1ref\endcsname{ m }{#2~\ref{##1}}
    \expandafter\NewDocumentCommand\csname#3ref\endcsname{ m }{#4~\ref{##1}}
}

% Create the labels
\@NewRefCommand{part}   {part}          {Part}  {Part}
\@NewRefCommand{chap}   {chapter}       {Chap}  {Chapter}
\@NewRefCommand{sec}    {section}       {Sec}   {Section}
\@NewRefCommand{note}   {note}          {Note}  {Note}
\@NewRefCommand{fig}    {figure}        {Fig}   {Figure}
\@NewRefCommand{tab}    {table}         {Tab}   {Table}
\@NewRefCommand{mus}    {music example} {Mus}   {Music example}
\@NewRefCommand{poem}   {poem example}  {Poem}  {Poem example}
\@NewRefCommand{dia}    {diagram}       {Dia}   {Diagram}
\@NewRefCommand{map}    {map}           {Map}   {Map}

% Set defaults for graphics commands used in above command
% Graphics size: Width and height can be as large as \textwidth and the
% \textheight minus enough room for the caption and some text at the bottom of
% the page
\newlength{\figureheight}
\setlength{\figureheight}{\dimexpr\textheight-10\baselineskip}
\graphicspath{
    {build/figures/}
    {build/music-examples/}
    {build/tables/}
    {build/diagrams/}
    {build/poem-examples/}
    {build/maps/}
}

\RequirePackage{adjustbox}
\NewDocumentCommand{\includefloat}{ m }{%
        \adjincludegraphics[
                max width=\textwidth,
                max height=\figureheight,
                keepaspectratio
            ]{#1}%
}

% FLOAT FORMATTING AND FILE INCLUSION
% Command to create commands for inserting TeX-generated floats 
% #1 Command to insert float type
% #2 name of float type
% #3 abbreviation used in label prefix
% #4 directory to input from
\NewDocumentCommand{\@NewIMGfloatCmd}{ m m m m }{%
    \expandafter\NewDocumentCommand\csname#1\endcsname{ m m }{%
        % ##1 file basename = label basename
        % ##2 caption text
        \begin{#2}[!t]
            \caption{##2}
            \label{#3:##1}
            \includefloat{#1}%
            \centering
        \end{#2}%
    }%
}

% Now create the float-insertion commands
% TeX floats
\@NewIMGfloatCmd{insertTable}   {table}        {tab}  {tables}
\@NewIMGfloatCmd{insertPoem}    {poemexample}  {poem} {poem-examples}
\@NewIMGfloatCmd{insertDiagram} {diagram}      {dia}  {diagrams}

% Image floats
\@NewIMGfloatCmd{insertFigure}  {figure}       {fig}  {build/figures}
\@NewIMGfloatCmd{insertMusic}   {musicexample} {mus}  {build/music-examples}
\@NewIMGfloatCmd{insertMap}     {map}          {map}  {build/maps}


