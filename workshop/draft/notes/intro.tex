% for preface?
It seems appropriate in a study of historical theology to be frank about my own
personal relationship to the subject matter.
I am a Christian.
This means that I believe that the God who created the universe has redeemed me
from sin and death by taking on human flesh in the historic person of Jesus
Christ, suffering death on a cross and then rising again to new life on the
third day, and that the Spirit of God now lives in me and all other believers
and is working to give us ever-new and eternal life together with God; and that
the purpose of all of for all to share in the love that is God's nature.
I hope that most of the theologically educated people whose writings and
creations I have studied in this book would agree with at least most of the
preceding statement.
I am not a Roman Catholic, though I respect the Roman church and have learned
much from its liturgical and intellectual heritage.
In any case I am certainly not a person of the seventeenth-century Spanish
Empire: I do not believe in the authority of the pope or in purgatory any more
than I believe (as the people I study here did) that the earth is the center of
the universe or that the planets are arranged according to musical ratios and
influence the body through a balance of four humors.
As a Christian I may be inclined to take a more positive view of theology
generally, and perhaps as a non-Catholic I am actually less critical of the
Roman Church than I might be otherwise, as I have no skin in that game.
But for the same reason my own faith actually makes it clearer to me where I
would depart from these historical sources.
I am ashamed of the church's hypocrisy and corruption, especially its history
of justifying the subjugation of indigenous peoples and the enslavement of any
people, two problems which continue today.

% % %
Though I myself am a Christian believer I am not a Roman Catholic (I would say
I am an evangelical Methodist with some Lutheran leanings and a love for both
liturgical and contemporary forms of worship).

% % %
For all its engagement with theology, this book is not a work of constructive
or normative theology. 
Though I myself am a Christian believer I am not a Roman Catholic, and I
certainly do not subscribe to the worldview of seventeenth-century Spanish
subjects.
But this book takes no position on the truth claims of any religious tradition.
Understanding the religious aspect of historical listening practices in a
particular time and place may lead to a more nuanced understanding of
contemporary theology, just as it might hopefully enable more sensitive and
meaningful performances of historic repertoire; but that is up to the reader.
I also would not claim that the theological approach is the only valid way to
study villancicos, and indeed there is much more to the genre than the aspect
that I present here, for which I refer readers to my articles and to a growing
body of studies by scholars with different perspectives and priorities.

Nevertheless, the religious element of early modern Spanish culture is so
overwhelmingly evident to anyone who has visited Mexico or Spain or read any of
its literature, that it can hardly be justified if we overlook it or insist on
interpreting it through an anti-religious or anti-Catholic lens.
Instead, this book is for anyone who has gazed upwards in a Spanish or Mexican
church and wondered why there are so many images of angel musicians with harps,
\term{vihuelas}, \term{bajones}, and organs; or who has read a play by
Calderón, a poem by Sor Juana, or a devotional book by Ignatius of Loyola or
Saint John of the Cross, and has observed how often these writers use musical
metaphors; or who wonders what people thought was happening when they listened
to music in church and how they believed this connected them to God, to each
other, and to the cosmos.

These examples all show that theology was a major intellectual pursuit of the
Spanish and New Spanish elite.

\begin{Footnote}
    Whether in the official Latin translation or the original Greek, the word for
    hearing (Latin \term{auditus}, \Greek \term{akoē}) can mean the faculty of
    hearing, the act of hearing, the hearing organ, or that which is heard.
    The New Revised Standard Version translates this \quoted{So faith comes from
    what is heard, and what is heard comes through the word of Christ}.
    \Autocites{Weber:Vulgate}{Aland:GNT4}[\sv{akoē}]{BDAG}.
\end{Footnote}


% opening example:
% Padilla, solfa villancico from Puebla MS
% possibly also/or "En la gloria de un portalillo" as in diss

\begin{Footnote}
    The major studies of the villancico as a musical and poetic genre are, in
    chronological order,
    \autocites{Rubio:Forma}{Laird:VC}{Torrente:PhD}{Tenorio:SorJuana}
    {CaberoPueyo:PhD}{Illari:Polychoral}{Knighton-Torrente:VCs}
    {Davies:Guadalupe}
    {Cashner:Cards}{Cashner:PhD}
    {LopezLorenzo:VC-Sevillano}{Swadley:VillancicoPhD}{Torrente:Historia17C}
    {ChavezBarcenas:PhD}.
    For musical editions, see \autocite{Cashner:WLSCM32} and the other sources
    cited there.
\end{Footnote}

% # other scholarship
% # outline of parts and chapters

% # sources
% ## non-musical sources for each chapter
% - ch1
%   + VCs: By Juan Gutiérrez de Padilla, Juan Hidalgo, Cristóbal Galán, Mateo
%   de Villavieja, Joan Cererols, Gaspar Fernández; texts by Sor Juana, Manuel
%   de León Marchante, Vicente Sánchez, etc.

% - ch2: 
%   + villancicos: *Si los sentidos* settings by Miguel de Irízar, Jerónimo de
%   Carrión; *sordos* by Juan Gutiérrez de Padilla, Matíás Ruíz; *Oigan todos
%   del ave* by  Cristóbal Galán 
% + doctrinal, catechetical, missionary, physiological literature,
%   chronicles of festivals, religious drama + (doctrinal/systematic theology,
%   ecclesiology, theological anthropology)

% - ch3: 
%   + VC: Juan Gutiérrez de Padilla, *Voces las de la capilla*; and texts of
%   related VCs from Seville and Lisbon
%   + exegetical, homiletical literature, liturgy, painting/architecture
%   + (exegetical theology, homiletics, liturgy and liturgical theology,
%   Christology, sacramental theology)

% - ch4: 
%   + VC: Joan Cererols, *Suspended, cielos, vuestro dulce canto*; and texts of
%   related VCs from Barcelona, Madrid, Zaragoza, Toledo, Seville; fragment of
%   music from Ibarra, Ecuador
%   + speculative music theory, scientific (astronomical) literature
%   + (natural/empirical theology (?), eschatology)

% - ch5: 
%   + VCs: *Suban las voces al cielo* by Pablo Bruna and Miguel Ambiela; *Qué
%   música divina* by José de Cáseda
%   + emblem books, painting, scientific literature, mystical theology, music
%   theory and performance literature
%   + (mystical/devotional theology)

% ## sources not emphasized
% - cathedral chapter acts, personnel records, inventories, payment records
% - municipal financial or other administrative records
% - diaries, personal accounts (mostly not extant?)
% - primary focus is on villancicos c1650-1700 with surviving music

% # problems, methodology

%{{{1 lit review
\section{%
Musicological Context of the Study
}

This dissertation contributes to a growing musicological conversation about music's relation to power and faith, though for the most part this has dealt with other musics and other places.
Regarding music's power, Margaret Murata has shown how \quoted{singing about singing} in certain Italian secular chamber cantatas of the mid-seventeenth century was a way for composers to question or even mock the power and truth of musical representation.%
	%
	\autocite{Murata:Singing}
	%
These composers deliberately drew attention to musical representation in order to comment ironically on the \term{seconda prattica} and conventions of operatic music.
The case studies in part~\ref{part:Singing} build on Murata's notion that \soCalled{singing about singing} highlights the artifice of music and allows for commentary on music within musical performance.

This dissertation's concern with changing understandings of senses, affects, and cosmology intersects with a large literature on these topics in other fields. 
Penelope Gouk, Gary Tomlinson, Linda Austern, Lorraine Daston and Katharine Park, Martha Feldman, and others have begun to bring these discourses---primarily from the history of science and philosophy but also from studies of magic and medicine---into historical musicology.%
	%
	\autocites{Gouk:MusicScienceMagic}{Gouk:Harmonics}{Gouk:Sciences}{Gouk:RepresentingEmotions}{Tomlinson:Magic}{Austern:Nature}{Daston:Wonders}{Feldman:Passions}
	%
As pathbreaking as this work is, much more still needs to be done to connect the theoretical discourses around music to specific, historical musical practices. 
Grayson Wagstaff has modeled how this might be done, exploring the role of senses and ritual in music for a Mexican funeral procession.%
	%
	\autocite{Wagstaff:Processions}	
	%
Any attempt to hear with \quoted{period ears,} limited as that enterprise must be, will have to be based on both how people made music in a particular historical moment and cultural orbit, and on how people, as far as can be determined, heard that music.%
	%
	\autocite{Burstyn:PeriodEar}
	%
For example, in an important colloquy on historical listening practice in \wtitle{Early Music}, Jeffrey Dean argued that the audience for much music before the eighteenth century should be considered to include the performers, and in many cases, the performers were the sole \quoted{listeners} to church music in particular.%
	%
	\autocite{Dean:ListeningPolyphony}
	%
Elisabeth Le Guin demonstrates how notated music may be read as a record of bodily experience, and Melanie Lowe and Richard Cullen Rath model different ways that we might begin to reconstruct historical hearing of music.
	%
	\autocites{LeGuin:BoccheriniBody}{Lowe:PleasureSymphony}{Rath:EarlyAmerica}
	%

The relationship between music and faith has been central in recent research into Lutheran music in the early modern period.
Researchers in that field have an advantage over scholars of Catholic music in that the Lutherans produced more writing about music in the form of vernacular hymns, hymnal prefaces, and polemical texts.
The work of Christian Bunners and Joyce Irwin on Lutheran theology of music is notable for drawing on some of these sources, but does not make clear enough connections between the polemics and particular musical repertoires.%
	%
	\autocites{Bunners:Kirchenmusik}{Bunners:Singende}{Irwin:VoiceHeart}
	%
Gregory Johnston, David Yearsley, and Eric Chafe have interpreted the music of Schütz, Buxtehude, and J.~S.~Bach, respectively, in the context of Lutheran theology and piety.%
	%
	\autocites{Johnston:Rhetorical}{Yearsley:Buxtehude}{Yearsley:BachThron}{Chafe:Tonal}
	%
Mary Frandsen's current work is furthering this effort of theological interpretation in historical context.%
	%
	\footnote{%
	\autocite{Frandsen:Crossing}, and a monograph in progress on Christocentric devotion through music in seventeenth-century Lutheran piety.
	}
	%

Some scholars of Roman Catholic music, in contrast, have not taken early modern sacred music seriously as a source for theological understanding, or have simply not been interested in theology. 
Lorenzo Bianconi, for example, accepts the now questioned narratives of confessionalization and secularization, and portrays Catholic theology and liturgy in the period as rigid, conformist, and unchanging.%
	%
	\autocite{Bianconi:17C}
	%
Bianconi presents Monteverdi as covertly bringing the \quoted{secular} styles of opera into the church with little concern for theology or piety, when we might just as well view the sacred and secular production of Monteverdi and his contemporaries as integrated parts of a whole.

These two narratives of confessionalization and secularization have especially plagued scholarship on villancicos: scholars who have investigated the music at all tend to either consider villancicos as an incursion of \quoted{popular,} \quoted{secular} music into the liturgy, or if they do consider the pieces' theology, they tend to see them as reiterations of preprogrammed Tridentine dogma.
Scholars with this latter perspective have tended to see Catholic music and arts in terms of twentieth-century propaganda or mass marketing.%
	%
	\footnote{%
	An especially embittered example is \autocite{Menache:Vox}.
	}
	%

Patrick Rietbergen and Jack Sage have criticized the \quoted{propaganda} approach as anachronistic, and have instead sought to avoid reductionism and take historical insider beliefs seriously.%
	%
	\autocites{Rietbergen:Power}{Sage:Instrumentum}
	%
Historical musicologists are beginning to catch up to the work in this vein by ethnomusicologists like Glenn Hinson, Jeff Todd Titon, and Monique Ingalls, that has sought to take seriously the beliefs of religious insiders regarding music's supernatural powers.%
	%
	\footnote{%
	\autocites{Hinson:Fire}{Titon:Powerhouse}{Ingalls:Awesome}.
	For perspectives from religious studies, see also the essays in \autocite{McCutcheon:InsiderOutsider}.
	}
	%
Gregory Barnett has argued that a form previously written off as \quoted{secular music in church}---the \term{sonata da chiesa}---was perceived by early modern Catholic worshippers as sacred on the basis of musical topics indexing other types of liturgical music such as choral Kyries and organ fugues.%
	%
	\autocite{Barnett:Bolognese}
	%

The old narratives of secularization and confessionalization are slowly being overturned, particularly because of increasing studies of Catholicism outside Europe.%
	%
	\footnote{%
	For example, \autocites{Bailey:Art}{Dean:Inka}{Ditchfield:Dancing}.
	}
	%
Seventeenth-century Catholicism is increasingly being presented as a much more colorful, diverse, and dynamic entity than previously thought.
Robert Kendrick's study of music for Holy Week in the early modern world demonstrates a comprehensive approach to music as a social activity with economic and political aspects as well as a religious expression with multivalent meanings to hearers of different stations.
	%
	\autocite{Kendrick:Jeremiah}
	%
Post-Tridentine Catholicism, while not the monolithic, quasi-totalitarian entity some once thought it was, nevertheless cannot be understood without considering the tensions between \quoted{top} and \quoted{bottom} strata of the Church, and positions in between.

%*******************
\subsection{%
Bringing Hispanic Music into the Conversation
}

Despite the value of all this scholarly work on questions of music, power, and faith, these discourses have not generally considered Spanish music. 
Music scholars continue to leave Spain entirely out of the grand narrative of early modern music, not to mention the Americas.
Unfortunately, on the other end, scholars Spanish music have not generally considered these larger questions.
Aside from what the many metamusical villancicos can teach us about early modern understandings of music, the villancico repertoire deserves further study in its own right. 
Numerically speaking it may well be the largest category of vernacular sacred music in the early modern world, and one of the first of any music to have a truly global spread---and yet it has received hardly any attention outside of Hispanic music studies.

Literary scholars and cultural historians have certainly contributed to our understanding of early modern worldviews. 
Francisco Rico documents the Neoplatonic notion of man as microcosm in Golden Age Spanish literature, and Frederick de Armas has shown how astrology was linked to notions of political power in the plays of Calderón, for example.%
	%
	\autocites{Rico:PequenoMundo}{DeArmas:Astraea}
	%
And anthropologists like William Christian and historians of religion like Gillian Ahlgren (among many others) have shed light on the devotional piety of Spanish Catholics in this era, in the lives of both common people and uncommon ones like Teresa of Ávila and other visionaries.%
	%
	\autocites{Christian:LocalReligion}{Christian:PersonAndGod}{Ahlgren:TeresaPolitics}
	%

But most scholars of Spanish literature and culture have not seriously considered the villancico, despite its being one of the most common forms of religious devotion and of sacred lyric poetry disseminated throughout the Hispanic world.
The one exception is the singular attention lavished on the prolific villancico poet Sor Juana Inés de la Cruz, but even music scholars such as Geoffrey Baker have written about Sor Juana's villancico texts without discussing their extant musical settings.%
	%
	\autocites{Tenorio:SorJuana}{Tenorio:Gongorismo}{Baker:EthnicVC}
	%

Miguel Querol Gavaldá and José María Díez Borque have written about the meaning and function of music in the plays of Calderón, with the latter scholar raising important questions about the public's understanding of and involvement with these plays. 
These literary scholars, though, do not connect Calderón's words about music with the actual music that survives for these plays.%
	%
	\autocites{Querol:Calderon}{DiezBorque:Publico}
	%
No one has yet done for music in the \term{auto sacramental} what Louise Stein has done for the Calderonian \term{comedia}.%
	%
	\autocite{Stein:Songs}
	%Insert Larissa brewer cite in this paragraph? \X
	%
These cultural forms demand interdisciplinary perspectives, and though individual scholars may attempt to integrate multiple approaches in a single project such as this one, the best way forward will be through real dialogue and collaboration across disciplines.

Part of the reason for this neglect of villancicos by musicological and literary scholars is that the relatively small amount of musicological research on villancicos has not generally broached themes of wider interest.
Important research by Samuel Rubio, Paul Laird, Bernat Cabero Pueyo, and others focused primarily on tracing the structural evolution of the villancico as an abstract form from the fifteenth century through the seventeenth.%
	%
	\autocites{Rubio:Forma}{Laird:VC}{CaberoPueyo:PhD}
	%
Other scholars have attempted taxonomies of villancico types, or studied particular subgenres of villancico such as the \quoted{ethnic villancico.}
Álvaro Torrente and others pushed research in a more contextual direction by investigating the liturgical function of villancicos in particular places.%
	%
	\footnote{%
	\autocite{Torrente:PhD} and the essays by Bégue, Bombi, Cabral, Hathaway, and Knighton in \autocite{Knighton-Torrente:VCs}.
	}
	%
Dianne Goldmann situates the villancico's sister genre, the Latin Responsory, in its ritual context in Mexico City Cathedral.
	%
	\autocite{Goldman:Responsory}
	%

But there are still few studies that interpret villancico poetry and music and also connect it with broader discourses.
Laird is to be commended as the first to suggest how such interpretation might proceed.
Bernardo Illari's thorough study of villancicos in eighteenth-century La Plata (Sucre, Bolivia) admirably combines both detailed local context and interpretive analysis of pieces of music.%
	%
	\autocites{Illari:Polychoral}{Illari:Popular}
	%
Illari was the first to identify the category of metamusical villancicos, which he labels \quoted{singing about singing,} but because this is not his primary question, he does not pursue the implications of these pieces far beyond that.
Illari was also one of the first to consider the theological dimension of villancicos, in distinction to other work that has focused more on political, social, or racial elements.
Like this other scholarship, though, Illari's ultimate focus is less on theological interpretation than on how the La Plata villancicos reflect and ritually enact structures of secular power.
Illari's thesis of \quoted{polychoral culture}---that music was a way of creating a harmoniously balanced, hierarchical society, a medium for ritually enforcing the \term{ancien régime}---is echoed in other recent work on Hispanic music by Baker (on Cuzco) and Irving (on Manila, who calls a similar idea \quoted{colonial counterpoint}).%
	%
	\autocites{Baker:Harmony}{Irving:Colonial}
	%
Ricardo Miranda has begun to move in this direction by contextualizing Juan Gutiérrez de Padilla's Latin-texted music with historical theology, though a deeper engagement with primary theological sources and musical manuscripts is needed.%
	%
	\autocite{Miranda:PadillaLuz}
	%
Miranda connects Padilla's Latin music for the cathedral of Puebla with seventeenth-century theological notions of light. 
Extending the study to Padilla's many villancicos based on tropes of light (such as pieces referring to the Virgin as the dawn and Christ as the sun) would yield even more fruitful results.

There is ample need, then, for an interpretive project that considers villancicos as sources of musical theology, focusing close analysis on music and poetry, and grounding interpretation in the intellectual context of specific times and places, but also with an eye to common global trends.
We will begin by focusing on the theological problem at the center of this inquiry: what kind of power did early modern Catholics believe music exerted in the relationship between faith and hearing?
%}}}1
