\subsubsection{Chapter 5: The Earthly Side of Celestial Music (Segovia, 1680s)}

The first piece of the cycle composed for Christmas 1678 at Segovia Cathedral by
its chapelmaster Miguel de Irízar is a metamusical piece about celestial music
coming down to earth: \worktitle{Qué música celestial}.
This chapter traces the tropes of heavenly music as they continue to develop in
the later seventeenth-century. 
By focusing on the composition of this Christmas cycle, the chapter
provides the first detailed study of how a complete set of villancicos was
composed, from assembling the poetic texts to drafting the musical setting and
having it copied and performed.
This study shows the earthly side of creating heavenly music by bringing
readers into Irízar's workshop as he provides for the specific needs of his
local community.

Segovia Cathedral's remarkable archive is one of the only places to preserve a 
large number of draft scores by seventeenth-century Spanish composers, rather 
than just performing parts.
These sources are even more precious because Miguel de Irízar wrote them in
makeshift notebooks assembled from his received letters, fitting music on the
backsides and margins of the letters.
The dates on the letters, then, allow for an unprecedented amount of detail in
tracking Irízar's compositional process.
Moreover, the letters are largely correspondence from other musicians regarding 
the exchange of villancico poetry.
Thus, focusing on the cycle of pieces for Christmas 1678, it is possible to 
determine exactly how Irízar obtained all of the poems for his cycle through 
his network of colleagues, and how he reworked these sources into a coherent 
cycle of his own.

The first piece in the set, \worktitle{Qué música celestial}, continues the 
traditions of metamusical villancicos using simple but ingenious means to 
create contrasts between earthly and heavenly music.
Irízar was an economical composer in every sense of the word.
His output was designed to meet local devotional needs, including a special 
local cult of St. Blaise (San Blas), for which Irízar wrote numerous 
villancicos. 
In the difficult economic environment of late seventeenth-century Spain, Irízar 
found ways to use his scarce resources to meet local demand and support the 
community both spiritually and practically.


