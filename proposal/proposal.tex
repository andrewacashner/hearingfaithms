% BOOK PROPOSAL FOR FIRST BOOK, ON VILLANCICOS
% ANDREW A. CASHNER
% 2016-06-23  Begun

\documentclass{vcbook-proposal}
\begin{document}

\begin{frontmatter}
\title{Faith, Hearing, and the Power of Music in Villancicos of the Spanish Empire\\[1ex]
Book Proposal for the University of California Press}
\author{Andrew A. Cashner}
\date{July 2016}
\maketitle

\tableofcontents
\end{frontmatter}

%*******************
\section{Profile}

Devotional writers of the seventeenth-century Spanish empire frequently cited St.~Paul's dictum from Romans~10:17 that \quoted{Faith comes through hearing, and hearing by the word of Christ}.
As Roman Catholics in missions and colonies across the globe, and in educational and evangelical efforts in Europe, sought to make faith audible in persuasive new ways, what was the role of music?
In the theological understanding of the time, what kind of power did music have to create or strengthen a link between faith and hearing?
How did church leaders in the Hispanic world use music to appeal to the sense of hearing?
For worshippers, whether the lettered elite or the classes of commoners for whom literacy was primarily a matter of hearing and remembering, how might the participants of Catholic ceremonies with music have actually listened?

There exists a rich and largely unmined vein of sources that can help us explore these questions, in the thousands of surviving villancicos in archives across the world.
Villancicos originated in the late medieval period as a genre of courtly entertainment and sometimes devotion, with elements drawn from common culture.
These were chamber songs on vernacular poems, similar to \term{virelais} or early madrigals.
In the late sixteenth cnetury, though, Spaniards began performing villancicos as an increasingly integral part of the liturgy, interpolating them among the lessons of Matins or in the Mass, especially at Christmas and Epiphany, Corpus Christi, and other high feasts like the Immaculate Conception of Mary.

The genre became a meeting place for elements of elite and common, erudite and comic, celestial and mundane elements.
In every major church from Madrid to Manila, chapelmasters were contractually obligated to compose dozens of villancicos each year, most often in sets of eight or more for Matins, such that the total global production for the seventeenth century alone must be in the hundreds of thousands.
Though huge numbers have been lost or destroyed, there still survive many hundreds of musical settings, and vastly more printed leaflets of villancico poems. 
These are not trivial popular tunes: they are long, complex polyphonic pieces for voices and instruments, often arranged for eight, twelve, or more voices in multiple choirs.
Most feature an expansive, motet-like \term{estribillo} or refrain section for the full ensemble, and strophic \term{coplas} or verses for soloists or a reduced group.

Aside from their ubiquitous presence in colonial Hispanic culture---something that not always noted even in musicological studies---and their mixture of vernacular poetry with multiple styles of music designed to appeal to listeners from the whole array of social positions, villancicos demand our attention because so many of them directly address the topic of music and hearing.
Hundreds of pieces explicitly invoke the sense of hearing with opening lines of \quoted{Listen!} \quoted{Silence!} \quoted{Pay attention}.
The repertoire abounds in representations of angelic choirs at Christmas, dancing shepherds and magi from the far corners of the earth (especially Africa), evocations of instruments, and even elaborate conceptual plays on terms from music theory and philosophy.
These pieces constitute \quoted{music about music} and as such provide as with an unparalelled historical discourse on the nature of music, through the medium of music itself.
The book's central question, then, is as follows: 
\emph{What can we learn about how seventeenth-century Hispanic Catholics understood music's power in the relationship between faith and hearing, by listening closely to villancicos that address this very subject?}

\subsection{Scholarly Context of the Study}

%***************************
\section{Outline of Chapters: Part I, Listening for Faith in Villancicos}

\subsection{Villancicos as Musical Theology}

\subsection{Making Faith Appeal to Hearing}

\subsection{Theological Functions of Villancicos}


%*******
\section{Part II, Listening for Unhearable Music}

\subsection{Christ as Singer and Song (Padilla)}

\subsection{Heavenly Dissonance (Cererols)}

\subsection{Burning Hearts and Voices (Bruna, Ambiela)}

\subsection{Christ as a \term{Vihuela} (Cáseda)}


%*******
\section{Part III, Listening for Community}

\subsection{Labor, Economy, and Devotion in Segovia}

\subsection{Building the Colonial City in Puebla}

%**************************
\section{Plans, Goals, Audience}


\printbibliography

\end{document}
