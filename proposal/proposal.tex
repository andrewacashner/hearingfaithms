% BOOK PROPOSAL FOR FIRST BOOK, ON VILLANCICOS
% ANDREW A. CASHNER
% 2016-06-23  Begun

\documentclass{vcbook-proposal}

\begin{document}

\frontmatter

\begin{titlingpage}
\title    {Faith, Hearing, and the Power of Music 
           in Villancicos of the Spanish Empire}
\subtitle {Book Proposal for the University of California Press}
\author   {Andrew A. Cashner}
\date     {July 2016}
\maketitle
\end{titlingpage}

\tableofcontents*

%*******************
\mainmatter

\section{Description}

Devotional writers of the seventeenth-century Spanish empire frequently cited St.~Paul's dictum from Romans~10:17 that \quoted{Faith comes through hearing, and hearing by the word of Christ}.
As Roman Catholics in missions and colonies across the globe, and in educational and evangelical efforts in Europe, sought to make faith audible in persuasive new ways, what was the role of music?
In the theological understanding of the time, what kind of power did music have to create or strengthen a link between faith and hearing?
How did church leaders in the Hispanic world use music to appeal to the sense of hearing?
For worshippers, whether the lettered elite or the classes of commoners for whom literacy was primarily a matter of hearing and remembering, how might the participants of Catholic ceremonies with music have actually listened?

There exists a rich and largely unmined vein of sources that can help us explore these questions, in the thousands of surviving villancicos in archives across the world.
Villancicos originated in the late medieval period as a genre of courtly entertainment and sometimes devotion, with elements drawn from common culture.
These were chamber songs on vernacular poems, similar to \term{virelais} or early madrigals.
In the late sixteenth cnetury, though, Spaniards began performing villancicos as an increasingly integral part of the liturgy, interpolating them among the lessons of Matins or in the Mass, especially at Christmas and Epiphany, Corpus Christi, and other high feasts like the Immaculate Conception of Mary.

The genre became a meeting place for elements of elite and common, erudite and comic, celestial and mundane elements.
In every major church from Madrid to Manila, chapelmasters were contractually obligated to compose dozens of villancicos each year, most often in sets of eight or more for Matins, such that the total global production for the seventeenth century alone must be in the hundreds of thousands.
Though huge numbers have been lost or destroyed, there still survive many hundreds of musical settings, and vastly more printed leaflets of villancico poems. 
These are not trivial popular tunes: they are long, complex polyphonic pieces for voices and instruments, often arranged for eight, twelve, or more voices in multiple choirs.
Most feature an expansive, motet-like \term{estribillo} or refrain section for the full ensemble, and strophic \term{coplas} or verses for soloists or a reduced group.

Aside from their ubiquitous presence in colonial Hispanic culture---something that not always noted even in musicological studies---and their mixture of vernacular poetry with multiple styles of music designed to appeal to listeners from the whole array of social positions, villancicos demand our attention because so many of them directly address the topic of music and hearing.
Hundreds of pieces explicitly invoke the sense of hearing with opening lines of \quoted{Listen!} \quoted{Silence!} \quoted{Pay attention}.
The repertoire abounds in representations of angelic choirs at Christmas, dancing shepherds and magi from the far corners of the earth (especially Africa), evocations of instruments, and even elaborate conceptual plays on terms from music theory and philosophy.
These pieces constitute \quoted{music about music} and as such provide as with an unparalelled historical discourse on the nature of music, through the medium of music itself.
The book's central question, then, is as follows: 
\emph{What can we learn about how seventeenth-century Hispanic Catholics understood music's power in the relationship between faith and hearing, by listening closely to villancicos that address this very subject?}

\subsection{Scholarly Context of the Study}

This book will be the first large-scale study to examine in detail the links between music, poetry, and theology in villancicos.
It will be the second major monograph, after Paul Laird's \worktitle{Toward a History of the Spanish Villancico}, to analyze the music of villancicos; and the first to be based on a global selection of sources and to interpret them in light of contemporary theological literature and devotional practices.

While musicologists have increasingly turned their interest to early modern Iberian music, most of the recent studies in English have focused on social functions of music, based on archival documentation and engaging with postcolonial thought and other critical theory.
This admirable work has greatly advanced our understanding of Hispanic musical culture, but at the same time it has not been sufficiently grounded in study of the actual surviving sources of musical practice.
No doubt, much has been lost, and what remains presents a skewed perspective focused on the elite music of urban cathedrals; but this book will demonstrate how much we can still learn from the copious sources we do have.

If we want to understand what early modern Spanish subjects believed about music, then we must pay close attention to the music they made as an expression of those beliefs.
Likewise, if we want to understand how their music worked even on a formal-structural level, we must seek to understand its creators' and hearers' beliefs about music.
This project seeks points of intersection between structures of music, structures of belief, and structures of society.


%***************************
\section{Outline of Chapters}

\subsection{Part I: Listening for Faith in Villancicos}

The book is organized in three parts.
The first, \quoted{Listening for Faith in Villancicos}, builds a foundation for understanding villancicos as sources of theology, as windows into historical beliefs about sensation and faith.
The second part, \quoted{Listening for Unhearable Music}, presents four case studies in interpreting specific villancico families on the subject of music, with an emphasis on the links between earthly, heavenly, and divine music.
The final part, \quoted{Listening for Community}, analyzes how processes of villancico composition contributed to the social structure of two contrasting communities, Segovia in old Castile and Puebla de los Ángeles in New Spain.
The first part focuses on tracing intellectual themes relevant to the whole seventeenth century; the second part focuses closely on individual poetic and musical texts in specific local contexts; and the third part emphasizes social, political, and economic aspects of villancicos in particular locales.

\subsubsection{Chapter 1: Villancicos as Musical Theology}

The first chapter makes the case for understanding villancicos as embodiments of theological beliefs about music---that is, as ways of hearing faith through music.
The chapter presents the results of a global survey and sampling of villancico poems and music, examining the different ways that villancicos represent or discuss music-making.
These include pieces on topics of musical instruments (for example the \term{clarín} or clarion, drums, castanets), specific songs and dances like \term{jácaras} and imitations of African and Indian dances, depictions of angelic music and music of the spheres, and learned pieces playing on solmization syllables and terms from music theory to create a double discourse and music and theology.
The chapter considers the peculiar semiotics of musical performances that point to other forms of music, or even to themselves, raising questions about how Hispanic Catholics understood music's power to represent and signify.

Along the way the chapter also establishes fundamental concepts and analytical methods that will familiarize readers with villancicos as a genre and with the specialized approaches developed for this project, since there is thus far little consensus (or even conversation) about musical structure in seventeenth-century Hispanic music.
The examples are chosen to represent fairly the gloabl reach of this genre, focusing on composers with wide influence, such as Juan Hidalgo, Cristóbal Galán, Joan Cererols, and Juan Gutiérrez de Padilla.
The chapter briefly situates the project in the context of scholarship on its broad themes---faith, sensation, music's power---in cultural studies, as well as in relation to the small but growing literature on early modern Hispanic villancicos and other music.

\subsubsection{Chapter 2: Making Faith Appeal to Hearing}

The second chapter explores the complex relationships between hearing and faith in the period after the Council of Trent, especially in the church's missionary and colonial efforts.
It looks at theological literature---devotional, dogmatic, and dramatic---together with villancicos on themes of sensation and faith, such as pieces presenting contests of the senses, and representations of deafness and sensory confusion.
The chapter explores both the theological climate in which villancicos were created and heard, and how villancicos themselves contributed to that climate.

The Roman Catechism produced by decree of the Council of Trent begins with an exposition of Romans~10:17, \quoted{Faith comes by hearing, and hearing by the word of Christ}.
The catechism argues that Christ who is the Word of God created the Church as the means through which God would speak to the world, and so the Church needed teachers to proclaim the word, and disciples to listen, so that all could be brought into communion with Christ the Word.
To do this, the catechism says, teachers must accommodate their teaching to \quoted{the sense of hearing and intelligence} of the hearers; but in the same breath the catechism says that \quoted{those whose senses have been properly trained} will be able to derive benefit from the teaching.
In other words, teachers must accommodate the sense of hearing even while training it.
Connecting this official document to a vernacular expositions of the Creed by Fray Antonio de Azevedo and to discussions of music from the missions (such as the experience of five Japanese youths on a European tour in the 1580s), I demonstrate that this way of understanding the relationship between hearing and faith created certain paradoxes and challenges, since one needed to be properly capacitated to hear faith with faith.
As missionaries were beginning to understand, both personal subjectivity (differing individual temperaments, education, ability) and cultural conditioning affected people's ability to hear the Church's teaching rightly.
While in some ways music seemed to break through these obstacles, music also placed even greater demands on individual and communal \quoted{ear training} to successfully connect people with the Church.

Literary contests of the senses help us understand how Catholics believed the senses to operate in relation to faith, and also reveal a certain amount of anxiety about the relationship.


\subsubsection{Theological Functions of Villancicos}


%*******
\subsection{Part II: Listening for Unhearable Music}

\subsubsection{Christ as Singer and Song (Padilla)}

\subsubsection{Heavenly Dissonance (Cererols)}

\subsubsection{Burning Hearts and Voices (Bruna, Ambiela)}

\subsubsection{Christ as a \term{Vihuela} (Cáseda)}


%*******
\subsection{Part III: Listening for Community}

\subsubsection{Labor, Economy, and Devotion in Segovia}

\subsubsection{Building the Colonial City in Puebla}

%**************************
\section{Plans, Goals, Audience}


\printbibliography

\end{document}
