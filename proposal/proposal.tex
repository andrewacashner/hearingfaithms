% BOOK PROPOSAL FOR FIRST BOOK, ON VILLANCICOS
% ANDREW A. CASHNER
% 2016-06-23  Begun

\documentclass[tt]{vcbook-proposal}
\newcommand{\publisher}{the University of California Press}

\begin{document}

\frontmatter

\begin{titlingpage}
\title    {Faith, Hearing, and the Power of Music 
           in Villancicos of the Spanish Empire}
\subtitle {Book Proposal for \publisher}
\author   {Andrew A. Cashner}
\date     {\today}
\maketitle
\end{titlingpage}

\tableofcontents*

%*******************
\mainmatter

\section{Description}

Devotional writers of the seventeenth-century Spanish empire frequently cited St.~Paul's dictum from Romans~10:17 that \quoted{Faith comes through hearing, and hearing by the word of Christ}.
As Roman Catholics in missions and colonies across the globe, and in educational and evangelical efforts in Europe, sought to make faith audible in persuasive new ways, what was the role of music?
In the theological understanding of the time, what kind of power did music have to create or strengthen a link between faith and hearing?
How did church leaders in the Hispanic world use music to appeal to the sense of hearing?
For worshippers, whether the lettered elite or the classes of commoners for whom literacy was primarily a matter of hearing and remembering, how might the participants of Catholic ceremonies with music have actually listened?

There exists a rich and largely unmined vein of sources that can help us explore these questions, in the thousands of surviving villancicos in archives across the world.
Villancicos originated in the late medieval period as a genre of courtly entertainment and sometimes devotion, with elements drawn from common culture.
These were chamber songs on vernacular poems, similar to \term{virelais} or early madrigals.
In the late sixteenth cnetury, though, Spaniards began performing villancicos as an increasingly integral part of the liturgy, interpolating them among the lessons of Matins or in the Mass, especially at Christmas and Epiphany, Corpus Christi, and other high feasts like the Immaculate Conception of Mary.

The genre became a meeting place for elements of elite and common, erudite and comic, celestial and mundane elements.
In every major church from Madrid to Manila, chapelmasters were contractually obligated to compose dozens of villancicos each year, most often in sets of eight or more for Matins, such that the total global production for the seventeenth century alone must be in the hundreds of thousands.
Though huge numbers have been lost or destroyed, there still survive many hundreds of musical settings, and vastly more printed leaflets of villancico poems. 
These are not trivial popular tunes: they are long, complex polyphonic pieces for voices and instruments, often arranged for eight, twelve, or more voices in multiple choirs.
Most feature an expansive, motet-like \term{estribillo} or refrain section for the full ensemble, and strophic \term{coplas} or verses for soloists or a reduced group.

Aside from their ubiquitous presence in colonial Hispanic culture---something that not always noted even in musicological studies---and their mixture of vernacular poetry with multiple styles of music designed to appeal to listeners from the whole array of social positions, villancicos demand our attention because so many of them directly address the topic of music and hearing.
Hundreds of pieces explicitly invoke the sense of hearing with opening lines of \quoted{Listen!} \quoted{Silence!} \quoted{Pay attention}.
The repertoire abounds in representations of angelic choirs at Christmas, dancing shepherds and magi from the far corners of the earth (especially Africa), evocations of instruments, and even elaborate conceptual plays on terms from music theory and philosophy.
These pieces constitute \quoted{music about music} and as such provide as with an unparalelled historical discourse on the nature of music, through the medium of music itself.
The book's central question, then, is as follows: 
\emph{What can we learn about how seventeenth-century Hispanic Catholics understood music's power in the relationship between faith and hearing, by listening closely to villancicos that address this very subject?}

\subsection{Scholarly Context of the Study}

This book will be the first large-scale study to examine in detail the links between music, poetry, and theology in villancicos.
It will be the second major monograph, after Paul Laird's \worktitle{Toward a History of the Spanish Villancico}, to analyze the music of villancicos; and the first to be based on a global selection of sources and to interpret them in light of contemporary theological literature and devotional practices.
This study of villancicos helps us understand what kind of power music had in the understanding of Hispanic Catholics of the seventeenth century.

While musicologists have increasingly turned their interest to early modern Iberian music, most of the recent studies in English have focused on social functions of music, based on archival documentation and engaging with postcolonial thought and other critical theory.
This admirable work has greatly advanced our understanding of Hispanic musical culture, but at the same time it has not been sufficiently grounded in study of the actual surviving sources of musical practice.
No doubt, much has been lost, and what remains presents a skewed perspective focused on the elite music of urban cathedrals; but this book will demonstrate how much we can still learn from the copious sources we do have.
The book's primary goal is to contribute to the development of a historically grounded interpretive framework for early modern religious music.
It is an interpretive study based primarily on poetic and musical texts, rather than a social history or anthropological study of music's functions.
This approach is valuable because there is an aspect of lived experience in any age that is only accessible through expressive culture; and particularly the realm of musical experience cannot be penetrated through descriptions or records of music-making alone. 

Instead, if we want to understand what early modern Spanish subjects believed about music, then we must pay close attention to the music they made as an expression of those beliefs.
Likewise, if we want to understand how their music worked even on a formal-structural level, we must seek to understand its creators' and hearers' beliefs about music.
This project seeks points of intersection between structures of music and structures of belief, and, to a  limited extent, structures of society.


%***************************
\section{Outline of Chapters}

\subsection{Part I: Listening for Faith in Villancicos}

The book is organized in three parts.
The first, \quoted{Listening for Faith in Villancicos}, builds a foundation for understanding villancicos as sources of theology, as windows into historical beliefs about sensation and faith.
The second part, \quoted{Listening for Unhearable Music}, presents four case studies in interpreting specific villancico families on the subject of music, with an emphasis on the links between earthly, heavenly, and divine music.
The final part, \quoted{Listening for Community}, analyzes how processes of villancico composition contributed to the specific needs of local communities, focusing on Segovia in old Castile and Puebla de los Ángeles in New Spain.
The first part focuses on tracing intellectual themes relevant to the whole seventeenth century; the second part focuses closely on the meanings of individual poetic and musical texts in specific local contexts; and the third part connects the history of musical composition in particular moments to broader social, political, and economic aspects of local communities.

\subsubsection{Chapter 1: Villancicos as Musical Theology}

The first chapter makes the case for understanding villancicos as embodiments of theological beliefs about music---that is, as ways of hearing faith through music.
The chapter presents the results of a global survey and sampling of villancico poems and music, examining the different ways that villancicos represent or discuss music-making.
These include pieces on topics of musical instruments (for example the \term{clarín} or clarion, drums, castanets), specific songs and dances like \term{jácaras} and imitations of African and Indian dances, depictions of angelic music and music of the spheres, and learned pieces playing on solmization syllables and terms from music theory to create a double discourse and music and theology.
The chapter considers the peculiar semiotics of musical performances that point to other forms of music, or even to themselves, raising questions about how Hispanic Catholics understood music's power to represent and signify.

Along the way the chapter also establishes fundamental concepts and analytical methods that will familiarize readers with villancicos as a genre and with the specialized approaches developed for this project, since there is thus far little consensus (or even conversation) about musical structure in seventeenth-century Hispanic music.
The examples are chosen to represent fairly the gloabl reach of this genre, focusing on composers with wide influence, such as Juan Hidalgo, Cristóbal Galán, Joan Cererols, and Juan Gutiérrez de Padilla.
The chapter briefly situates the project in the context of scholarship on its broad themes---faith, sensation, music's power---in cultural studies, as well as in relation to the small but growing literature on early modern Hispanic villancicos and other music.

\subsubsection{Chapter 2: Making Faith Appeal to Hearing}

The second chapter interprets villancicos as part of a broader discourse on the relationship between hearing and faith in the Spanish Catholic world.
After the Council of Trent, theologians placed a new emphasis on making faith appeal to the sense of hearing, to contribute to the educational, missionary, and colonial goals of the newly global church.
Villancicos on themes of sensation and faith include allegorical contests of the senses, representations of sensory confusion, and comic dialogues featuring deaf men.

These villancicos were heard in a theological climate of anxiety about the role of subjective sensory perception in relation to the objective faith of the church.
The Tridentine Catechism begins with Romans~10:17, \quoted{Faith comes by hearing, and hearing by the word of Christ}; and this teaching was echoed in vernacular expositions of the Creed such as that of Fray Antonio de Azevedo.
The catechism teaches that Christ who is the Word of God created the Church to make God known to the world, and that in order to fulfill this purpose the church's messengers must accomodate their teaching to \quoted{the sense of hearing and intelligence} of the hearers.
In the same breath, though, the catechism says that from this teaching spiritual fruit will be reaped by \quoted{those whose senses have been properly trained}.
In other words, teachers must accommodate the sense of hearing even while training it.

The paradox in this way of thinking, which was most clear on the missions and in colonial settings, was that people needed to learn to listen properly in order to acquire faith.
They needed a certain capacity to hear rightly and not be deceived or confused.
Missionary encounters in particular demonstrated that both personal subjectivity and cultural conditioning affected people's ability to hear the Church's teaching rightly.
Athanasius Kircher wrote in 1650 that music fitted to sacred words could move listeners beyond understanding the text intellectually, to be actually \quoted{carried away by joy} to \quoted{experience the truth of what was said}.
In other words, music connected subjective experience to objective truth. 
But Kircher takes for granted that listeners know how to hear music rightly, even as he elsewhere acknowledges that both individuals and cultural groups have differing temperaments and thus perceive differently.
For example, the five young Japanese Christians who toured Europe in the 1580s were said to have reported to their friends back home that it was only after considerable exposure that they came to perceive European music as beautiful.
While in some ways music seemed to break through obstacles of perception and understanding, music also placed even greater demands on individual and communal \quoted{ear training} to successfully connect people with the Church.

The villancicos studied in this chapter form part of the effort to make faith appeal to hearing, by addressing directly the relationship between the two.
Two pieces based on the same poem, composed in the later seventeenth century by successive chapelmasters at Segovia Cathedral Miguel de Irízar and Jerónimo de Carrión, present allegorical contests in which each sense receives a \soCalled{hearing} before Faith, who gives first prize to the sense Hearing.
This villancico tradition shows how traditional Scholastic discourses on sensation, drawing on Aristotle and Aquinas, were presented to a broader audience.

The chapter connects these pieces to a 1634 Corpus Christi play, \worktitle{El nuevo palacio del Retiro}, by court poet Pedro Calderón de la Barca.
As part of the festivities inaugurating Philip IV's new palace retreat, the Buen Retiro, Calderón staged both a contest of the senses before faith, and an extended allegory of obstacles to faith.
Faith crowns Hearing not because of his strengths but because of the sense's \quoted{incertitude}, because he humbly admits that he is \quoted{the sense most easily deceived}, that he perceives not the man but only his voice, which may be feigned or only an echo.
Calderón presents hearing as a portal to truths that go beyond normal sensation, but also seems to raise serious doubts about how much hearing can be trusted.
If hearing is so easily deceived, we may wonder from our cultural remove, how could the church effectively use auditory art forms like sung poetry and drama to propagate faith?
Even in Calderón's play there is an extended scene in which Faith explains the Eucharist to  \term{Judaísmo}, the stock character of the unbelieving Jew, but Judaism cannot comprehend.
He repeats the refrain, \quoted{For I have listened to Faith without faith}.
Recalling the Tridentine Catechism's emphasis on both accommodation and training, how was one supposed to acquire the capacity to hear \quoted{the Faith} of the church, \emph{with} faith?

The \quoted{villancicos of the deaf} by Matías Ruíz of Madrid and Juan Gutiérrez de Padilla of Puebla do not answer this question, but do demonstrate a prevalent fascination with the propblematic role of hearing in acquiring faith.
Both pieces present comic catechism lessons between hard-of-hearing men and their religious teachers, at a time when the first schools for the deaf were being founded in Spain.
The \quoted{deaf} men mishear the teaching in absurd and even impious ways.
These pieces are only in small part about actual people with hearing disabilities; rather they poke fun at ineffective and incompetent teachers, and at the spiritual deafness of all sinners.

These pieces on hearing and faith reflect a certain amount of doubt and anxiety about how much hearing could be trusted, who had the proper capacity for faithful hearing, and how the church could overcome these obstacles to promulgate its teaching and devotion.
They challenge simplistic notions of Catholic devotional music as an imposition of dogma in the manner of twentieth-century propaanda. 
Like the metamusical pieces studied in the first chapter, these villancicos do not assume passive listeners ready to be branded with Christian dogma; rather, they challenge listeners to think about the act of hearing itself.
Perhaps they even create some space for considering doubt, even if the ultimate goal is to assert confidence in the Church's teaching and devotion.

\subsubsection{Chapter 3: The Sacred Power of Music}

This chapter seeks to recover early modern understandings of music's power, in order to understand what Hispanic Catholics might have thought performing and hearing villancicos actually accomplished.
What did people in this culture hope would happen to them when they listened to devotional music?
By what processes, precisely, could music work to link faith and hearing?
How was music's use justified acccording to the prevailing theological and scientific systems of knowledge?

The villancicos on the subject of music that will be studied in part II will be our primary sources for answering these questions.
Those pieces consistently articulate a conception of music within the tradition of Christian Neoplatonism, as developed by Augustine and Boethius and reinvigorated by early modern theologians.
We will understand pieces more fully if we begin with an understanding of music's theological functions within that tradition, according to representative and influential exponents of Neoplatonic theology relevant to music.
The Dominican theologian Fray Luis de Granada was one of the most widely read authors in all genres of Spanish literature through the eighteenth century.
Reading Fray Luis along with the speculative music theory of Athanasius Kircher (whose \worktitle{Musurgia universalis} was read in Mexico and Manila), the chapter develops a historically grounded conception of music listening.

These writers understand earthly music to be an imperfect reflection of higher forms of music---beyond the harmony of the sphers to the angelic chorus and the mysterious harmonies of the triune Godhead. 
The listener's task is to discern those elements of earthly music that point upward to these higher harmonies.
The created world, they teach, presents people with a \soCalled{book of nature}, and it is humans' task to learn to read this book and thereby come to know and adore its author.
In an age in which most books were read aloud, music becomes a way to hear the book of nature read aloud.
The same proportions and ratios that were inscribed in creation as marks of the Creator's mastery were turned from potential into sounding reality through music. 
When a tree, which silently bore witness to its Creator's artifice already, was carved into a shwan, with holes drilled in accord with timeless Pythagorean ratios, and played, then the divine signatures already latent in the wood and even in the air within the tube were set in motion and made audible.

Even more so, then, when the instrument played was not wood or metal but the infinitely flexible human voice.
For people who understood \quoted{Man} as a microcosm of all creation, the human voice made audible the very structure of the universe.
And for people who believed that the divine Word had taken on flesh in Jesus Christ, the human voice could echo not only creation but also the divine nature of Christ.

In this conception, music is much more than a mere delivery system for doctrinally appropriate words, as a simplistic notion of religious arts in this period might suggest.
Rather, its appeal to hearing was as a witness of the artifice of creation and the wonders of its creator, and even further, it promised to align hearers in harmony both with God and with each other.
Music could be actively deployed, then, not just to reflect divine harmonies but actually to create harmony within the human community, accoridng to a Boethian concept of \term{musica humana}.
Thus music linked aspects of personal contemplation and devotion with collective social structures.

We will hear this theology of music echoed and chorused in seventeenth-century villancicos, particularly those that represent divine and heavenly music.
This understanding makes it possible to propose a preliminary taxonomy of how villancicos functioned theologically to connect faith and hearing. 
I identify three modes of listening to villancicos that emphasize different theological functions: mnemonic, contemplative, and affective.

In the mnemonic mode of listening, music functions as a tool for remembering language-based concepts---that is, as \term{mnemotecnía}, to use the apt Spanish loanword from Greek.
Here the words are primary and musical structures serve to help the words stick in memory.
This does not mean that all the meaning and impact comes from the words; rather, musical forms project the form and meaning of the words in a way that makes the text both appealing and memorable.
The strophic \term{coplas} of most villancicos fit within this function, as do many pieces in which the projection of poetic text is the first priority.
Also in this cateogry are villancicos that teach doctrinal concepts, or which set key verses to be memorized.
For example, \worktitle{Señor mío Jesucristo} by Joan Cererols is a setting of the standard Act of Contrition prayer used in the confessional; for the boys of the choir-school at Montserrat, where Cererols was monk and chapelmaster, the \term{estribillo} helped the singers and their hearers learn an important text, while the coplas, which expound on the doctrine of confession in verse, helped them understand its meaning.
Pieces like this are actually unusual in the villancico genre, though; few pieces really teach religious doctrine, contrary to what some scholars have assumed.

A contemplative mode of listening focuses more on the sonic structures of  music, relatively independent of words, as bearers of a meaning of their own, which in Neoplatonic fashion communicate a message beyond words about the creation and its Creator.
This type of listening requires more knowledge of musical-rhetorical devices and contrapuntal techniques. 
Villancicos that explicitly play on such elements---as when Joan Cererols sets the words \worktitle{contrapunto celestial} as an eight-voice fugue with countersubjects in inversion---overtly draw listeners' attention to musical artifice and invite contemplative listening.

Affective listening focuses on music's power to move the body and soul directly, and in the power of this movement to unite listeners together in sympathetic vibration.
Here music is not a medium for ideas or an object of contemplation, but rather an active force working on the bodies of hearers the influences of occult planetary and humoral powers.
This type of listening is at once the most direct and the most abstract, because of the role, already discussed, of individual subjectivity and cultural conditioning in hearing.
While early modern authors tend to speak of the musical modes as having objectively different effects, some like Kircher also acknowledged that music's affective powers depended on elaborate, culturally constructed symbolic systems of topics and tropes and of stylistic allusions, such as when Cererols contrasts earthly and heavenly music by juxtaposing a relatively modern, dissonance-laden, expressive style with stricter music in the older contrapuntal style of church music.

These three modes of listening could all be used for any villancico, though some subgenres and formal types seem to appeal to one mode more than the others.
Different listeners, depending on their personal temperament and cultural conditioning, their level of education and musical training, and their social position, could use these three modes in varying ways---and of course they could choose not to listen closely at all.

% doctrinal, doxological, social, and economic functions
The chapter concludes that villancicos linked faith and hearing by making audible not just propositions of belief (doctrine), but by embodying the communal celebration of shared values (doxology).
Villancicos should not be seen in purely social or economic terms; but rather, the theological nature of this music formed part of its social and economic appeal.
Villancicos celebrated the objects of faith in ways that made them appealing to the ears of listeners of varying social stations and orientations.
They bear witness to, and are products of, complex interactions between church leaders and this congregation; between tradition and novelty; between high and low, official and unofficial, elite and common; between a universal empire and church, and local deovtional and social demands.
Studying individual villancicos in depth in the next part of the book, will allow us to understand these dynamics with more specificity and refinement.
The next part of the book, then, presents detailed case studies of particular villancioc traditions in which music, faith, and hearing are prominent themes.


%*******
\subsection{Part II: Listening for Unhearable Music}

\subsubsection{Christ as Singer and Song (Padilla)}

\subsubsection{Heavenly Dissonance (Cererols)}

\subsubsection{Burning Hearts and Voices (Bruna, Ambiela)}

\subsubsection{Christ as a \term{Vihuela} (Cáseda)}


%*******
\subsection{Part III: Listening for Community}

\subsubsection{Labor, Economy, and Devotion in Segovia}

\subsubsection{Building the Colonial City in Puebla}

%**************************
\section{Timeline}
% put audience stuff in first section

\printbibliography

\end{document}
