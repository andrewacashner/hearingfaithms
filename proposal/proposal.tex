% BOOK PROPOSAL FOR FIRST BOOK, ON VILLANCICOS
% ANDREW A. CASHNER
% 2016-06-23  Begun

\documentclass{vcbook-proposal}
\newcommand{\publisher}{the University of California Press}

\begin{document}

\frontmatter

\begin{titlingpage}
\title    {Faith, Hearing, and the Power of Music 
           in Villancicos of the Spanish Empire}
\subtitle {Book Proposal for \publisher}
\author   {Andrew A. Cashner}
\date     {\today}
\maketitle
\end{titlingpage}

\tableofcontents*

%*******************
\mainmatter

\section{Description}

Devotional writers of the seventeenth-century Spanish empire frequently cited St.~Paul's dictum from Romans~10:17 that \quoted{Faith comes through hearing, and hearing by the word of Christ}.
As Roman Catholics in missions and colonies across the globe, and in educational and evangelical efforts in Europe, sought to make faith audible in persuasive new ways, what was the role of music?
In the theological understanding of the time, what kind of power did music have to create or strengthen a link between faith and hearing?
How might the participants of Catholic ceremonies with music, from a variety of social stations, have actually listened?

There exists a rich and largely unmined vein of sources that can help us explore these questions, in the thousands of surviving Spanish \term{villancicos} in archives across the world.
Villancicos originated in the late medieval period as a genre of courtly entertainment and sometimes devotion, with elements drawn from common culture.
In the late sixteenth century, Spaniards began performing villancicos as an increasingly integral part of the liturgy, interpolating them among the lessons of Matins or in the Mass, especially at Christmas and Epiphany, Corpus Christi, and other high feasts like the Immaculate Conception of Mary.

The genre became a meeting place for elements of elite and common, erudite and comic, celestial and mundane elements.
In every major church from Madrid to Manila, chapelmasters were contractually obligated to compose dozens of villancicos each year, most often in sets of eight or more for Matins.
Even after considerable loss of sources, there still survive many hundreds of musical settings, and vastly more printed leaflets of villancico poems. 
These are not trivial popular tunes: they are long, complex polyphonic pieces for voices and instruments, often arranged for eight, twelve, or more voices in multiple choirs.
Most feature an expansive, motet-like \term{estribillo} or refrain section for the full ensemble, and strophic \term{coplas} or verses for soloists or a reduced group.
Villancicos were ubiquitous in colonial Hispanic culture, and their mixture of vernacular poetry with multiple styles of music seems designed to appeal to listeners from the whole array of social positions.

Many villancicos directly address themes of faith, hearing, and the power of music.
Hundreds of pieces explicitly invoke the sense of hearing with opening lines of \quoted{Listen!} \quoted{Silence!} \quoted{Pay attention}.
The repertoire abounds in representations of angelic choirs at Christmas, dancing shepherds and magi from the far corners of the earth (especially Africa), evocations of instruments, and even elaborate conceptual plays on terms from music theory and philosophy.
These pieces constitute \quoted{music about music} and as such provide as with an unparalleled historical discourse on the nature of music, through the medium of music itself.
This book, therefore, ask the question,
What can we learn about how seventeenth-century Hispanic Catholics understood music's power in the relationship between faith and hearing, by listening closely to villancicos that address this very subject?

\subsection{Scholarly Context and Significance of the Study}

This book will be the first large-scale study to examine in detail the links between music, poetry, and theology in villancicos.
It will be the second major monograph, after Paul Laird's \worktitle{Toward a History of the Spanish Villancico}, to analyze the music of villancicos.%
  \begin{Footnote}
  \autocite{Laird:VC}. 
  The otherwise excellent collection of essays, \autocite{Knighton-Torrente:VCs}, devotes little attention to the actual music.
  Other than \autocite{Rubio:Forma} (limited by a focus on too small a repertoire), the only other works addressing musical structure in villancicos have been theses: \autocites{CaberoPueyo:PhD}{Illari:Polychoral}.
  \end{Footnote}
It will be the first to be based on a global selection of sources and to interpret them in light of contemporary theological literature and devotional practices.
This study of villancicos helps us understand what kind of power music had in the understanding of Hispanic Catholics of the seventeenth century.

While musicologists have increasingly turned their interest to early modern Iberian music, most of the recent studies in English have focused on social functions of music, based on archival documentation and engaging with postcolonial thought and other critical theory.
This admirable work has greatly advanced our understanding of Hispanic music, but at the same time it has not been sufficiently grounded in study of the actual surviving sources of musical practice---the notated music.%
  \autocites{Torrente:PhD}{Baker:Harmony}{Irving:Colonial}
  {BakerKnighton:MusicUrbanSociety}
It is certainly true that the archival sources of music do not present a full picture of music in Hispanic society, since they include primarily music used in the rituals of urban cathedrals, composed and enjoyed primarily by the elite.
But these pieces provide insight into historical ways of experiencing the world that no other source can provide; and this book will demonstrate how much we can still learn from the copious sources we do have.

Villancicos are an important genre of Spanish literature as well as a musical genre, but the only literary studies have concentrated on the published works of only a few poets---chiefly Sor Juana Inés de la Cruz---and have not considered the musical aspects of the genre.%
  \autocite{Tenorio:SorJuana}
Musical settings of villancicos are an important source of previously unstudied villancico texts that are available only scattered through the various performing parts of musical manuscripts.
These settings also provide us with contemporary \soCalled{readings} of seventeenth-century poems, since we can see exactly how composers wanted the poetry to be declaimed, and how they interpreted the text through musical symbolism and affect.

The book's primary goal is to contribute to the development of a historically grounded interpretive framework for early modern religious music.
The study connects to a growing literature on sound, sensation, and musical experience in the early modern world.%
  \begin{Footnote}
  Notable examples include \autocites{Rath:EarlyAmerica}{Feldman:Passions}
  {Austern:Nature}{Gouk:MusicScienceMagic}
  \end{Footnote}
It differs from some of these studies, though, in its greater emphasis on the interpretive act of listening to musical performative texts.
In this respect it is more similar to studies of Spanish music by Elisabeth LeGuin, on Italian opera by Martha Feldman, and in particular to the work of David Yearsley on Lutheran music, which closely links musical analysis with cultural context.%
  \autocites{LeGuin:Tonadilla}{LeGuin:BoccheriniBody}
  {Feldman:Opera}{Yearsley:BachCounterpoint}

The book argues that if we want to understand what early modern Spanish subjects believed about music, then we must pay close attention to the music they made as an expression of those beliefs.
Likewise, if we want to understand how their music worked even on a formal-structural level, we must seek to understand its creators' and hearers' beliefs about music.

The significance of this book lies in its a detailed analytical and interpretive approach to a previously unstudied musical genre, which is put to the service of a research question that should be of great relevance for anyone interested in the history and meaning of religious arts in the early modern world.
The book is based on original archival research in nine archives in Mexico and Spain.
Analysis is based on my own editions of villancicos that have in most cases never been previously published or performed, using the original partbooks, poetry imprints, and other archival sources.
I ground my interpretations as rigorously as possible in specific historical and local contexts, reading the music together with theological literature and visual art that the poets and composers would have known.
In some cases these non-musical sources have received little scholarly attention, and few have been applied to the study of music.

The book will be of interest, then, first to musicologists specializing in Ibero-American and early modern music, and to scholars of Spanish literature.
It should also interest scholars of early modern culture and of religious literature and art, as well as those interested in colonial arts.


%***************************
\section{Outline of Chapters}

\subsection{Part I: Listening for Faith in Villancicos}

The book is organized in three parts.
The first, \quoted{Listening for Faith in Villancicos}, builds a foundation for understanding villancicos as historical sources of theological beliefs about sensation and faith.
The four case studies of the second part, \quoted{Listening for Unhearable Music}, interpret specific villancico families on the subject of music, with an emphasis on the links between earthly, heavenly, and divine music.
The final part, \quoted{Listening for Community}, analyzes how processes of villancico composition contributed to the specific needs of local communities, focusing on Segovia in Castile and Puebla de los Ángeles in New Spain.

\subsubsection{Chapter 1: Villancicos as Musical Theology}

The first chapter makes the case for understanding villancicos as embodiments of theological beliefs about music---that is, as ways of hearing faith through music.
The chapter presents the results of a global survey of villancico poems and music, examining the different ways that villancicos represent or discuss music-making.
These include pieces on topics of musical instruments (for example the \term{clarín} or clarion, drums, castanets), specific songs and dances like \term{jácaras} and imitations of African and Indian dances, depictions of angelic music and music of the spheres, and learned pieces playing on solmization syllables and terms from music theory to create a double discourse and music and theology.

The chapter also establishes fundamental concepts and analytical methods that will familiarize readers with villancicos as a genre and with the specialized approaches developed for this project.
The examples are chosen to represent fairly the global reach of this genre, focusing on composers with wide influence, such as Juan Hidalgo, Cristóbal Galán, and Joan Cererols; and composers whose works are frequently performed today, chiefly Juan Gutiérrez de Padilla.
The chapter briefly situates the project in the context of scholarship on its broad themes---faith, sensation, music's power---in cultural studies, as well as in relation to the small but growing literature on early modern Hispanic villancicos and other music.

\subsubsection{Chapter 2: Making Faith Appeal to Hearing}

The second chapter interprets villancicos as part of a broader discourse on the relationship between hearing and faith in the Spanish Catholic world.
After the Council of Trent, theologians placed a new emphasis on making faith appeal to the sense of hearing, to contribute to the educational, missionary, and colonial goals of the newly global church.
Villancicos on themes of sensation and faith include allegorical contests of the senses, representations of sensory confusion, and comic dialogues featuring deaf men.
These villancicos were heard in a theological climate of anxiety about the role of subjective sensory perception in relation to the objective faith of the church.

The Tridentine Catechism begins with Romans~10:17, \quoted{Faith comes by hearing, and hearing by the word of Christ}.%
  \autocite{Catholic:Catechismus1614}
The Roman Church, it says, is the embodiment of Christ, the Word of God, in the world.
Its teachers must accommodate their teaching to \quoted{the sense of hearing and intelligence} of listeners.
But the catechism also acknowledges that listeners must be trained to hear properly.

This paradoxical challenge of appealing to the ear even while training it was most clear on the missions and in colonial settings.
Accounts like that of the five Japanese youths who toured Europe in the 1580s demonstrated that both personal subjectivity and cultural conditioning affected people's ability to hear the Church's teaching rightly.%
  \autocite{Massarella:JapaneseTravellers}
People needed to learn to listen properly in order to acquire faith and not be deceived or confused.

Athanasius Kircher wrote in 1650 that music fitted to sacred words could move listeners to \quoted{experience the truth of what was said}.%
  \autocite{Kircher:Musurgia}
But Kircher takes for granted that listeners know how to hear music rightly, even as he elsewhere acknowledges that individuals and nations perceive music differently.
While in some ways music promised to break through obstacles of perception, music also required individual and communal \quoted{ear training} to successfully connect people with the Church.

The villancicos studied in this chapter form part of the effort to make faith appeal to hearing, by addressing directly the relationship between the two.
Two pieces based on the same poem, composed in the later seventeenth century by successive chapelmasters at Segovia Cathedral Miguel de Irízar and Jerónimo de Carrión, present allegorical contests in which each sense receives a \soCalled{hearing} before Faith, who gives first prize to the sense Hearing.

The chapter connects these pieces to a 1634 Corpus Christi play, \worktitle{El nuevo palacio del Retiro}, by court poet Pedro Calderón de la Barca.%
  \autocite{Calderon:Retiro}
As part of the festivities inaugurating Philip IV's new palace retreat, the Buen Retiro, Calderón staged both a contest of the senses before faith that raises some doubts about hearing's reliability.
Faith crowns Hearing not because of his strengths but because of the sense's \quoted{incertitude}, because he humbly admits that he is \quoted{the sense most easily deceived}.
If hearing is so easily deceived (we may wonder from our cultural remove), how could the church effectively use auditory art forms like sung poetry and drama to propagate faith?
This doubt becomes even stronger when Calderón stages an extended anti-Semitic dialogue dramatizing the inability of \soCalled{Judaism} to understand what Faith is telling him.
As Judaism keeps lamenting, \quoted{I have listened to Faith without faith}.
Recalling the Tridentine Catechism's emphasis on both accommodation and training, how was one supposed to acquire the capacity to hear \quoted{the Faith} of the church, \emph{with} faith?

The \quoted{villancicos of the deaf} by Matías Ruíz of Madrid and Juan Gutiérrez de Padilla of Puebla demonstrate a prevalent fascination with the problematic role of hearing in acquiring faith.
Both pieces present comic catechism lessons between hard-of-hearing men and their religious teachers.
The \quoted{deaf} men mishear the teaching in absurd and even impious ways.
These pieces certainly mock disabled people while also poking fun at ineffective and incompetent teachers, and at the spiritual deafness of all sinners.

These pieces on hearing and faith reflect a certain amount of doubt and anxiety about how much hearing could be trusted, who had the proper capacity for faithful hearing, and how the church could overcome these obstacles to promulgate its teaching and devotion.
They challenge simplistic notions of Catholic devotional music as an imposition of doctrine in the manner of twentieth-century propaganda. 
Like the metamusical pieces studied in the first chapter, these villancicos do not assume passive listeners ready to be branded with Christian dogma; rather, they challenge listeners to think about the act of hearing itself.


\subsubsection{Chapter 3: The Sacred Power of Music}

This chapter seeks to recover early modern understandings of music's power.
What did people in this culture hope would happen to them when they listened to devotional music like villancicos?
By what processes, precisely, could music work to link faith and hearing?

The villancicos on the subject of music that will be studied in part II consistently articulate a conception of music within the tradition of Christian Neoplatonism, as developed by Augustine and Boethius and reinvigorated by early modern theologians.
We will understand pieces more fully if we begin with an understanding of music's theological functions within that tradition.
A representative exponent of Neoplatonic theology in Hispanic culture is the Dominican theologian Fray Luis de Granada, whose was one of the most widely read authors in all genres of Spanish literature through the eighteenth century.%
  \autocite{LuisdeGranada:Simbolo}
Reading Fray Luis along with the speculative music theory of Athanasius Kircher (whose \worktitle{Musurgia universalis} was read in both Mexico and Manila), the chapter develops a historically grounded conception of music listening.

These writers understand earthly music to be an imperfect reflection of higher forms of music---beyond the harmony of the spheres to the angelic chorus and the mysterious harmonies of the triune Godhead. 
The listener's task is to discern those elements of earthly music that point upward to these higher harmonies.
The created world, they teach, presents people with a \soCalled{book of nature}, and it is humans' task to learn to read this book and thereby come to know and adore its author.
In an age in which most books were read aloud, music becomes a way to hear the book of nature read aloud.
The same proportions and ratios that were inscribed in creation as marks of the Creator's mastery were turned from potential into sounding reality through music. 
For people who understood \quoted{Man} as a microcosm of all creation, the human voice made audible the very structure of the universe.
And for people who believed that the divine Word had taken on flesh in Jesus Christ, the human voice could echo not only creation but also the divine nature of Christ.

In this conception, music is much more than a mere delivery system for doctrinally appropriate words, as a simplistic notion of religious arts in this period might suggest.
Rather, its appeal to hearing was as a witness of the artifice of creation and the wonders of its creator, and even further, it promised to align hearers in harmony both with God and with each other.
Music could be actively deployed, then, not just to reflect divine harmonies but actually to create harmony within the human community, according to a Boethian concept of \term{musica humana}.

This understanding makes it possible to propose a preliminary taxonomy of how villancicos functioned theologically to connect faith and hearing. 
I identify three modes of listening to villancicos that emphasize different theological functions: mnemonic, contemplative, and affective.
In the mnemonic mode of listening, music functions as a tool for remembering language-based concepts, in which the words are primary and musical structures serve to help the words stick in memory.
A contemplative mode of listening focuses more on the sonic structures of  music, relatively independent of words, as bearers of a meaning of their own, which in Neoplatonic fashion communicate a message beyond words about the creation and its Creator.
This type of listening requires more knowledge of musical-rhetorical devices and contrapuntal techniques. 
Affective listening focuses on music's power to move the body and soul, working actively to unite a community of listeners together in sympathetic vibration.

The chapter concludes that villancicos linked faith and hearing by making audible not just propositions of belief (doctrine), but by embodying the communal celebration of shared values (doxology).
The social and economic functions of villancicos, which until now have been considered primary, should be understood as closely linked with this theological function.

%*******
\subsection{Part II: Listening for Unhearable Music}

The next part of the book presents detailed case studies of particular villancico traditions (related poems and their musical settings) that represent divine and heavenly music.
These metamusical pieces encapsulate, through their musical structures, contemporary beliefs about music's sacred power.
As the creators of villancicos developed a consistent set of theological, poetic, and musical tropes, they used this subgenre to prove their mastery and establish themselves within a tradition of metamusical representation.

\subsubsection{Chapter 4: Christ as Singer and Song (Padilla)}

This chapter and the following interpret a pair of villancico families that represent Christ as both singer and song.
The first, and the subject of this chapter, is \worktitle{Voces, las de la capilla}, set by Juan Gutiérrez de Padilla (Puebla, 1657).
The second is \worktitle{Suspended, cielos, vuestro dulce canto}, set by Joan Cererols (Montserrat, ca. 1660).
They invite hearers to listen contemplatively for the voice of Christ echoing in the voices of the chapel choir, and they proclaim that human voices can be a means through which God reorders the created world to be in harmony with the incarnate Christ.

In Padilla's \worktitle{Voces}, the chorus exhorts itself, \quoted{Voices of the chapel choir,/ keep count with what is sung,/ for the King is a musician/ and notes even the most venial dissonances/ in the manner of Christ, the infant prince [\foreign{infante}]/, as in the manner of David, the monarch}.
The piece celebrates the baby Jesus as the heir of the musician-king David, as chapelmaster, singer, and even as the song that is sung.
In a series of ingenious and cryptic conceits after the manner of the influential Baroque poet Luis de Góngora, the poem presents Christ's birth as \quoted{the sign of \term{A (la, mi, re)}}, the tuning note of the \quoted{\term{alpha} and \term{omega}}.
Christ's life, death, and resurrection as God incarnate represent a \quoted{composition} in which the divine chapelmaster could \quoted{prove the consonances of Man and God}.
Padilla projects and dramatizes the poem through musical techniques both literal (numerological puns, matching the solmization syllables in the poem by using the same notes in the voices) and figurative (evoking the song of angels, men, and beasts in Christ's stable in the style of a double-choir madrigal).

In its own gnomic way, Padilla's piece distills the most central elements of contemporary Christmas doctrine and doxology.
The chapter unlocks its rich meanings by reading it in the context of liturgy, preaching, and Biblical interpretation.
The sources for interpretation are based on what materials would have been available to Padilla himself, from the evidence of Puebla's seminary and convent libraries.
Among these sources, the exegete Cornelius à Lapide sums up a vast range of patristic and medieval sources as they were understood in Padilla's day. 
A contemporary model sermon by Fray Luis de Granada presents many of the same tropes of Christmas, often in strikingly similar language.

A central part of the trope of voice at Christmas is the patristic concept of the Christ-child as \term{Verbum infans}. 
Christmas sermons by Augustine and Bernard of Clairvaux develop this trope of Christ as the Word made flesh (Jn. 1:1) who as an infant (Latin \foreign{in-fans}) is unable to speak any words, but who is, in his body, the Word itself.
He cannot speak, but he does cry.
This villancico extends the trope by interpreting Christ's inarticulate cries as musical performance.

The relationship of Padilla's \worktitle{Voces, las de la capilla} to similar pieces suggests that Padilla used this metamusical piece to establish his compositional pedigree and demonstrate his mastery of the craft.
The chapter compares Padilla's text to a similar one (\worktitle{Cantores de la capilla}) set ten years earlier by the chapelmaster of Seville Cathedral, Luis Bernardo Jalón, and evidence for an earlier lost setting beginning \worktitle{Voces, las de la capilla} by Francisco de Santiago, Jalón's predecessor at Seville and possibly a personal acquaintance of Padilla's.
By setting the older text from Santiago rather than a simplified version by Jalón, Padilla may have been seeking to align himself with Santiago's compositional pedigree, which extended back the Hapsburg Flemish Chapel.


\subsubsection{Chapter 5: Heavenly Dissonance (Cererols)}

\worktitle{Suspended, cielos} by Joan Cererols represents a family of villancico poems set to music in various versions at least eight times between 1651 and the end of the seventeenth century.
\quoted{Suspend, O heavens, your sweet chant}, Cererols's chorus begins, addressing the typical villancico exhortation to listen, to the heavenly spheres rather that the congregation. 
The harmony of the spheres, the piece proclaims, is out of tune, and must give way to \quoted{the newest consonance}, that is, \quoted{the tender sobs of a child} that \quoted{bear the plainchant to the angels}.
Here again, Christ is \term{Verbum infans}, but his cries are the \term{cantus firmus} on which the music of a new creation will be based.
This villancico connects Christ's voice, and the choir's voices, to the music of the planetary spheres, prompting an interpretation rooted in contemporary astronomical beliefs.

The musical setting projects the musical concepts of the poem through a masterful musical-rhetorical discourse.
Cererols structures the piece with tow primary motivic subjects and two contrasting styles, which map onto a contrast between worldly music (including the spheres) and divine and angelic music. 
At one point the motive associated with divine music becomes the subject of an eight-voice fugue for the full double chorus, with fugal answers in inversion.
At the end of the piece, setting the words \quoted{bears the plainchant to the angels}, it becomes the literal cantus firmus of a section in the style of a cantus-firmus motet.

Cererols draws listeners' attention, in keeping with a contemplative listening practice, to the musical structure itself as the bearer of meaning.
Cererols highlights the artificiality, indeed the imperfection, of his own music, when he sets the phrase \quoted{listen to the newest consonance} with the last word on a prominent unprepared dissonance, which he repeats several times.
This dissonance becomes an ironic symbol that prompts the hearer to listen for a higher music, audible only with the ears of faith.

The chapter contextualizes the piece's representation of sidereal music with the musical-cosmological writings of Athanasius Kircher.
Kircher illuminates the question of the tunefulness of the music of the spheres.
Kircher explains that some planets exert consonant influences on the earth, but others are dissonant, and must be held in tension---prepared, resolved in musical terms---by the other planets. 

In an age of new astronomical discoveries, it may be that this emphasis on the untunefulness of the heavens, bears witness to a growing anxiety about the old cosmology.
But this shift in emphasis marks a different trajectory than that theorized in an earlier generation by John Hollander, in his study of English poetry on music from the same period.%
  \autocite{Hollander:Untuning}
Certainly some villancicos do fit with Hollander's argument that references to heavenly music become increasingly cliché, generic, and conventionalized.
But the traditional Roman Catholic culture of imperial Spain does not fit Hollander's broader claim that this is evidence of a \quoted{disenchantment} and secularization process.
Poets and composers continue to manifest genuine belief in the old cosmology; if anything the increasing abstraction of their references to heavenly music might be seen as a way of preserving the old cosmology in a way that made it immune from scientific attack.

The chapter traces the genealogy of the \worktitle{Suspended, cielos} family through multiple poetic versions, which also provide evidence for lost musical settings.
The imprints demonstrate that from a 1651 Royal Chapel performance in Madrid, the text traveled quickly through a network of affiliated composers across Iberia and into the New World, including a fragmentary musical setting from a convent in Ecuador.
The chapter delineates three main textual families, one of which is based on the influential villancico poet Manuel de León Marchante's elaborated version of the poem. 

\subsubsection{Chapter 6: Burning Hearts and Voices (Bruna, Ambiela)}

The pair of villancicos studied in this chapter demonstrate clearly one composer modeling a metamusical villancico on the poetry and musical setting of a more senior composer.
As Pedro Calahorra discovered, a villancico poem beginning \worktitle{Suban las voces al cielo} was set first by Pablo Bruna, the blind organist of Daroca in the region of Zaragoza, and then in a variant text by Miguel Ambiela, who had studied in Daroca just after Bruna's death and who would later go on to prestigious positions in Lérida, Zaragoza, and Toledo.%
  \autocite{Calahorra:Suban}
The similarities between the pieces make it clear that Ambiela's work is a conscious homage to Bruna's, even as the differences also demonstrate how Ambiela differentiated himself from the older generation's aesthetic. 

Ambiela's piece is a kind of offering, then, putting forth his best effort in homage to an admired older master.
This is fitting since both villancico poems represent music as a form of offering.
The chapter examines the Bruna villancico in light of Spanish mystical theology, particular the concepts of fire in the \worktitle{Flame of Living Love} by St. John of the Cross.
Iconography and epigrammatic poetry by Sebastián de Covarrubias deepen the understanding of music's relation to fire and air, explaining how music could serve as a fitting means of self-offering.

Ambiela's version keeps the specific musical references from Bruna's text, but resituates them as an act of devotion to the Blessed Virgin.
His villancico shows how some of the same musical tropes that we have seen associated with Christ's Incarnation and Eucharistic presence could be adapted to suit sanctoral devotion. 

\subsubsection{Chapter 7: Christ as a \term{Vihuela} (Cáseda)}

A piece by another composer in the region of Zaragoza, José de Cáseda (\worktitle{Qué música divina}, ca. 1700), further develops the concepts of music as an act of offering, as it meditates on Christ's passion through the conceit of Christ as a \term{vihuela}. 
The piece identifies Christ with both a \term{vihuela} and a \term{cítara}, comparing the stretching of strings over the bridge to Christ's being stretched out on the cross, the placement of tuning pins to the nails in Christ's hands, the bow to the lance that pierced his side.
It specifies that this is a seven-course vihuela, because of the seven sacraments flowing from the spear wound.
This music, it says, \quoted{is not for the ears}, for even if anyone could hear it, \quoted{as many notes as they would hear, they would perceive as false}---that is, as out of tune (or perhaps, as \term{ficta} deviations from the gamut). 

Taking the conceit literally provides insights into odd features of the musical structures: Cáseda imitates the structure and style of the vihuela through the vocal texture, such that the choir itself embodies the vihuela, and by extension, Christ's own body.
Allegorical traditions connect plucked chordophones---originally the cithara and lyre---to the body of Christ and by extension to the bodies of virgins and martyrs, as Craig Monson has shown.%
  \autocite{Monson:DivasConvent}
The villancico extends this patristic tradition by applying it to a contemporary Spanish instrument.
Similarly, the painter Manuel Correa included the vihuela along with harp, lute, lyre, and cittern among the heavenly \quoted{cithara players} of Revelation in the sacristy of Mexico City Cathedral.

To embody the poetic concept of the \quoted{music} of Christ's passion as sounding \quoted{false}, Cáseda plays with various types of musical falsehood.
These range from blatant solecisms like parallel fifths, to an enigmatic passage on the phrase \quoted{various cadences} that appears to intentionally defy \term{musica ficta} conventions, creating a kind of impossible music.
As the piece goes to rather extreme lengths to depict musical \quoted{falsehood}, it also demonstrates the limits of imitation within the metamusical villancico tradition, since it shows the pressure to outdo previous composers in highlighting musical artifice.
Because the piece only survives in a copy from a convent in Puebla, it also gives insight into the unique functions of villancicos within cloistered communities.

%*******
\subsection{Part III: Listening for Community}

Part III presents two case studies of villancico composition in its social and economic contexts, showing how villancicos connected hearing and faith within the community.
This part shifts attention from individual listening toward practices of hearing music in community.

\subsubsection{Chapter 8: Labor, Economy, and Devotion in Segovia}

Segovia Cathedral's remarkable archive is one of the only places to preserve a large number of seventeenth-century Spanish composer's draft scores, rather than just performing parts.
These sources are even more precious because Miguel de Irízar wrote those draft scores in makeshift notebooks assembled from his received letters.
Irízar filled the backsides and margins of the letters with his musical drafts.
The dates on the letters allow for an unprecedented amount of detail in tracking Irízar's compositional process.
Moreover, the letters are largely correspondence from other musicians regarding the exchange of villancico poetry.
Thus, focusing on the cycle of pieces for Christmas 1678, it is possible to determine exactly how Irízar obtained all of the poems for his cycle through his network of colleagues, and how he reworked these sources into a coherent cycle of his own.

The first piece in the set, \worktitle{Qué música celestial}, continues the traditions of metamusical villancicos using simple but ingenious means to create contrasts between earthly and heavenly music.
Irízar was an economical composer in every sense of the word.
His output was designed to meet local devotional needs, including a special local cult of St. Blaise (San Blas), for which Irízar wrote numerous villancicos. 
In the difficult economic environment of late seventeenth-century Spain, Irízar found ways to use his scarce resources to meet local demand and support the community both spiritually and practically.

\subsubsection{Chapter 9: Building the Colonial City in Puebla}

Juan Gutiérrez de Padilla faced a similar demand for villancicos to serve local devotional functions in Puebla, but the colonial setting in New Spain meant that his music was much more closely involved with the social and political projects of building Puebla's cathedral, and thereby, the colonial community itself.
This chapter situates Padilla's cycles of the 1650s as part of Bishop Juan de Palafox y Mendoza's project to build the cathedral and unite the colonial community around it.
Padilla's villancico cycles functioned like microcosms of colonial society, with a variety of styles and subgenres designed to appeal to the ears of hearers from different social stations.%
  \footnote{This chapter builds on some of the themes of my article, \autocite{Cashner:Cards}, but is based on entirely new research in different sources.}

Like Irízar, Padilla assembled his poetic sources largely from poetry imprints from peninsular churches, especially Madrid and Seville, but also incorporated texts that may be his own creations and may reflect a more local outlook.
The topics and themes of these cycles bear close correspondences to the iconography of the early cathedral furnishings, particularly the resplendent high altar paintings by Pedro García Ferrer, another Andalusian immigrant.
Both  music and art reflect the agenda of Bishop Palafox as expressed in his own pastoral writings and decrees.
The chapter also traces the functions of Padilla's villancicos in Puebla outside the cathedral, exploring possible connections with the Oratorian society, to which Padilla belonged; and to the same Conceptionist convent that later performed the Cáseda villancico in chapter 7.

This case study illustrates in detail how church leaders actively used villancicos to make faith appeal to hearing, with the ideal goal of binding a new community together in harmony.
It also considers the paradoxes and contradictions of this music in the colonial context, particularly with regard to race and class.

A special focus on two subgenres, the \term{jácara} and the \term{ensaladilla} with \term{negrilla}, consider to what extent Padilla incorporated lower-class and \soCalled{ethnic} elements into his cycles, and why.
Padilla's caricatures of black music-making in his \term{negrillas} raises numerous problems of racial representation and stereotyping.
The chapter seeks to deepen our understanding of these troubling pieces by devoting the first serious attention to their music, and by illuminating their theological context.
Padilla himself was both a priest and a slave owner, and his depictions of blacks are, paradoxically, both denigrating and idealistic.
There is a gaping divide between the kind of society Padilla represents musically and the one in which he actually lived.
Though a full examination of racial questions in villancicos must await a future study, this chapter will advance a conversation that is of great relevance to contemporary social issues.


%**************************
\section{Plans}

The book follows the general plan and content of my 2015 dissertation, but this will be a new work.%
  \autocite{Cashner:PhD}
I have reorganized the chapters and shifted their emphasis.
The last chapter will include the most new research since the dissertation.
The other chapters need to be rewritten in a more concise, argument-driven manner, and they need to be brought in ever closer dialogue with secondary scholarship, especially that which has emerged in the last several years.
For the most part this is a matter of presenting a focused selection of the dissertation material in a fresh way for a broader audience.
The chapters outlined above are intended to be relatively short (around thirty pages) and able to be read independently, such as for assignments in graduate courses.
There will be much technical material, but scholars from different disciplines will be able to follow the central arguments without every detail of the musical, poetic, and theological analyses.
I expect a total page count of around 350 pages.

The book should include a glossary of Spanish and Latin terms, and an appendix with some primary source texts.
There will be numerous musical examples throughout the book, for which I intend to secure funding for recordings.
Some chapters (particularly chapters 7 and 9) discussion visual-art sources and thus would require several full-color illustrations.

For my musical editions of the scores discussed in depth in the book, I am pursuing publication through the Web Library of Seventeenth Century Music.
The first volume of these editions is currently under review.
These editions will be available freely online as a supplement to the monograph, and I also hope to secure funding to include recordings as well to make freely available as part of the the Web Library editions online.

For reviewers, I would recommend Tess Knighton, Álvaro Torrente, Paul Laird, Bernardo Illari, Geoffrey Baker, David Irving, Jesús Ramos-Kittrell, or Elisabeth LeGuin.


\end{document}
