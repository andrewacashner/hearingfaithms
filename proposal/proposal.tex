% BOOK PROPOSAL FOR FIRST BOOK, ON VILLANCICOS
% ANDREW A. CASHNER
% 2016-06-23  Begun
% 2017-01-27  Send to editor Raina Polivka with one chapter
% 2018-06-01  Revised for second submission with two chapters 
% 2018-11     UC press declined contract, moving on

\documentclass{vcbook-proposal}
\singlespacing

\title{Hearing Faith:\\ Music as Theology in the Spanish Empire\\
\emph{Book Proposal}}
\author{Andrew A. Cashner}

\begin{document}

\maketitle

\tableofcontents
\clearpage

\section{Description}

Devotional writers of the seventeenth-century Spanish empire frequently cited 
St.~Paul's dictum from Romans~10:17 that \quoted{Faith comes through hearing, 
and hearing by the word of Christ}.
As Roman Catholics in missions and colonies across the globe, and in 
educational and evangelical efforts in Europe, sought to make faith audible in 
persuasive new ways, what was the role of music?
In the theological understanding of the time, what kind of power did music have 
to create or strengthen a link between faith and hearing?
How might the participants of Catholic ceremonies with music, from a variety of 
social stations, have actually listened?

There exists a rich and largely unmined vein of sources that can help us 
explore these questions, in the thousands of surviving Spanish 
\term{villancicos} in archives across the world.
Though the term \term{villancico} in colloquial Spanish today refers to popular
Christmas carols, in the seventeenth century the term typically denoted long,
complex polyphonic pieces for voices and instruments with vernacular words,
often arranged for eight, twelve, or more voices in multiple choirs.
Most feature an expansive, motet-like \term{estribillo} or refrain section for 
the full ensemble, and strophic \term{coplas} or verses for soloists or a 
reduced group.
Villancicos originated in the late medieval period as a genre of courtly 
entertainment and sometimes devotion, with elements drawn from common culture.
In the late sixteenth century, Spaniards began performing villancicos as an 
increasingly integral part of the liturgy, interpolating them among the lessons 
of Matins or in the Mass, especially at Christmas and Epiphany, Corpus Christi, 
and other high feasts like the Conception of Mary.

The genre became a meeting place for elements of elite and common, erudite and 
comic, celestial and mundane elements.
In every major church from Madrid to Manila, chapelmasters composed dozens of
villancicos each year, most often in sets of eight or more for Matins.
Even after considerable loss of sources, there still survive many hundreds of 
musical settings, and vastly more printed leaflets of villancico poems. 
Villancicos were ubiquitous in colonial Spanish culture, and their mixture of 
vernacular poetry with multiple styles of music seems designed to appeal to 
listeners from the whole array of social positions.

Many villancicos directly address themes of faith, hearing, and the power of 
music.
Hundreds of pieces explicitly invoke the sense of hearing with opening lines of 
\quoted{Listen!} \quoted{Silence!} \quoted{Pay attention}.
The repertoire abounds in representations of angelic choirs at Christmas, 
dancing shepherds and magi from the far corners of the earth (especially 
Africa), evocations of instruments, and even elaborate conceptual plays on 
terms from music theory and philosophy.
These pieces constitute \quoted{music about music} and as such provide us with 
an unparalleled historical discourse on the nature of music, through the medium 
of music itself.

This book argues that by listening closely to villancicos we can learn how
seventeenth-century Spanish Catholics understood music's power in the
relationship between faith and hearing. 
It contributes to current conversations in early modern cultural history and
sound studies by drawing primary evidence from close analysis and interpretation
of musical performative texts, and it expands our understanding of a fascinating
musical and poetic genre that not enough scholarship has taken seriously.

\subsection{Scholarly Context and Significance of the Study}

This book will be the first large-scale study to examine in detail the links 
between music, poetry, and theology in villancicos.
It will be the second major monograph, after Paul Laird's \worktitle{Toward a 
History of the Spanish Villancico}, to analyze the music of villancicos.%
  \begin{Footnote}
      \Autocite{Laird:VC}. 
      The otherwise excellent collection of essays, 
      \autocite{Knighton-Torrente:VCs}, devotes little attention to the actual music.
      Other than \autocite{Rubio:Forma}, which is limited by a focus on too small a 
      repertoire, the only other works addressing musical structure in villancicos 
      have been dissertations: \autocites{CaberoPueyo:PhD}{Illari:Polychoral}.
  \end{Footnote}
It will be the first to be based on a global selection of sources and to 
interpret them in light of contemporary theological literature and devotional 
practices.

While musicologists have increasingly turned their interest to early modern 
Iberian music, most of the recent studies in English have focused on social 
functions of music, based on archival documentation and engaging with 
postcolonial thought and other critical theory.
This admirable work has greatly advanced our understanding of music in the
Spanish-speaking world, but it needs to be balanced with research that focuses
on the actual surviving sources of musical practice---the notated music.%
  \Autocites{Torrente:PhD}{Baker:Harmony}{Irving:Colonial}
  {BakerKnighton:MusicUrbanSociety}
It is certainly true that the archival sources of music do not present a full 
picture of music in Spanish society, since they include primarily music used 
in the rituals of urban cathedrals, composed and enjoyed primarily by the elite.
But these pieces provide insight into historical ways of experiencing the world 
that no other source can provide; and this book will demonstrate how much we 
can still learn from the copious sources we do have.

Villancicos are an important genre of Spanish literature as well as a musical 
genre, but literary studies have concentrated on the published works 
of only a few poets like Sor Juana Inés de la Cruz, and have not 
considered the musical aspects of the genre.%
  \Autocite{Tenorio:SorJuana}
Musical settings of villancicos are an important source of previously unstudied 
villancico texts that are available only scattered through the various 
performing parts of musical manuscripts.
These settings also provide us with contemporary \soCalled{readings} of 
seventeenth-century poems, in terms of both declamation and interpretation.

The book's primary goal is to contribute to the development of a historically 
grounded interpretive framework for early modern religious music.
The study connects to a growing literature on sound, sensation, and musical
experience in the early modern world; and it advances an ongoing conversation in
sound studies regarding the ability of voice, sound, and vibration to shape our
understanding of the world and order our social relationships.%
  \begin{Footnote}
      Notable examples include \autocites{Rath:EarlyAmerica}{Feldman:Passions}
      {Austern:Nature}{Gouk:MusicScienceMagic}{Ochoa:Aurality}{Eidsheim:SensingSound}
  \end{Footnote}
This book differs from most of these studies, though, because it places a much
greater emphasis on the interpretive act of listening to musical performative
texts.
In this respect it is similar to Elisabeth LeGuin's studies of Spanish music,
Martha Feldman's work on Italian opera, and especially David Yearsley's research
on Lutheran music, all of which closely link musical analysis with cultural
context.%
    \Autocites{LeGuin:Tonadilla}{LeGuin:BoccheriniBody}
    {Feldman:Opera}{Yearsley:BachCounterpoint}

The book provides a needed complement and counterweight to Andrew
Dell'Antonio's study of musical listening in early modern Italy.%
    \Autocite{DellAntonio:Listening}
Dell'Antonio addresses similar questions but draws from a quite distinct body of
evidence in that his study focuses primarily on written discourse about music in
elite Roman circles.
Such accounts of musical listening are rare in the Spanish world of the
seventeenth century, in part because of different social contexts for musical
performance; but the large repertoire of villancicos on the subject of musical
hearing provides an alternative way of exploring these questions.
These pieces allow us to examine not only words about music, but \quoted{music
about music}.
This book, then, connects discourse about music with detailed analysis of musical
practice in social contexts that involved a wide range of listeners; it also 
concentrates in more detail on theological and music-theoretical conceptions
of how music worked.

The book argues that if we want to understand what early modern Spanish 
subjects believed about music, then we must pay close attention to the music 
they made as an expression of those beliefs.
Likewise, if we want to understand how their music worked even on a 
formal-structural level, we must seek to understand its creators' and hearers' 
beliefs about music.

This book will be significant for scholarship because it applies a detailed
analytical and interpretive approach to an overlooked musical genre in the
service of a research question of great relevance for anyone interested in the
history and meaning of religious arts in the early modern world.
The book is based on original archival research in nine archives in Mexico and 
Spain.
Analysis is based on my own editions of villancicos that have in most cases 
never been previously published or performed, drawn from the original partbooks, 
poetry imprints, and other archival sources.
I ground my interpretations as rigorously as possible in specific historical 
and local contexts, reading the music together with theological literature and 
visual art that the poets and composers would have known.
In some cases these non-musical sources have received little scholarly 
attention, and few have been applied to the study of music.

The book will be of interest, then, to scholars specializing in early
modern music, Ibero-American music, and colonial music, as well as those
interested in questions of sensation, faith, and religious experience in other
periods and places.
Scholars of Spanish literature will find the book helpful for understanding a
widespread poetic genre that should not be considered apart from music.
Historians of religion and devotional literature, as well as those interested
in sound studies and even the history of science will find many points of
intersection with their own work.
Though the book will include detailed analysis of poetry, music, and theology,
it will be written in an engaging, jargon-free style that should make it
accessible for graduate students in a variety of fields as well as upper-level
undergraduates. 

%***************************
\section{Outline of Chapters}

\subsection{Part I: Listening for Faith}

The book is organized in two parts.
The first, \quoted{Listening for Faith}, builds a foundation for 
understanding villancicos as historical sources of theological beliefs about 
sensation and faith.
The case studies of the second part, \quoted{Listening for Unhearable 
Music}, interpret specific villancico families on the subject of music, with an 
emphasis on the links between earthly, heavenly, and divine music.
These case studies focus on villancicos composed and performed across the
different regions of the empire, from Segovia in Castile to Puebla in New Spain.

\subsubsection{Chapter 1: Villancicos as Musical Theology}

The first chapter makes the case for understanding villancicos as embodiments 
of theological beliefs about music---that is, as ways of hearing faith through 
music.
The chapter presents the results of a global survey of villancico poems and 
music, examining the different ways that villancicos represent or discuss 
music-making.
These include pieces on topics of musical instruments (for example the 
\term{clarín} or clarion, drums, castanets), specific songs and dances like 
\term{jácaras} and imitations of African and Indian dances, depictions of 
angelic music and music of the spheres, and learned pieces playing on 
solmization syllables and terms from music theory to create a double discourse 
on music and theology.

The chapter also establishes fundamental concepts and analytical methods that 
will familiarize readers with villancicos as a genre and with the specialized 
approaches developed for this project.
The examples are chosen to represent fairly the global reach of this genre, 
focusing on composers with wide influence, such as Juan Hidalgo, Cristóbal 
Galán, and Joan Cererols; and composers whose works are frequently performed 
today, chiefly Juan Gutiérrez de Padilla.
The chapter briefly situates the project in the context of related scholarship
in musicology and other fields of cultural studies.

The rest of the chapter situates the musical theology of villancicos in the
context of early modern theological and theoretical literature about music to
develop a historically grounded concept of musical listening.
Villancicos on the subject of music, such as those that will be studied in part
II, consistently articulate a conception of music within the tradition of
Christian Neoplatonism, as developed by Augustine and Boethius and reinvigorated
by early modern theologians like Fray Luis de Granada and music theorists like
Athanasius Kircher.%
    \Autocites{LuisdeGranada:Simbolo}{Kircher:Musurgia}
These writers understand earthly music to be an imperfect reflection of higher 
forms of music---beyond the harmony of the spheres to the angelic chorus and 
the mysterious harmonies of the triune Godhead. 
The listener's task is to discern those elements of earthly music that point 
upward to these higher harmonies.
The created world, they teach, presents people with a \soCalled{book of 
nature}, and it is humans' task to learn to read this book and thereby come to 
know and adore its author.
In an age in which most books were read aloud, music becomes a way to hear the
book of nature read aloud.  
For people who understood \quoted{Man} as a microcosm of all creation, the 
human voice made audible the very structure of the universe.

In this conception, music is much more than a mere delivery system for 
doctrinally appropriate words, as a simplistic notion of religious arts in this 
period might suggest.
Rather, it appealed to hearing as a witness of the artifice of creation and 
the wonders of its creator, and even further, it promised to align hearers in 
harmony both with God and with each other.

\subsubsection{Chapter 2: Making Faith Appeal to Hearing}

The second chapter interprets villancicos as part of a broader discourse on the 
relationship between hearing and faith in the Spanish Catholic world.
After the Council of Trent, theologians placed a new emphasis on making faith 
appeal to the sense of hearing, to contribute to the educational, missionary, 
and colonial goals of the newly global church.
Villancicos on themes of sensation and faith include allegorical contests of 
the senses, representations of sensory confusion, and comic dialogues featuring 
deaf men.

The Tridentine Catechism begins with Romans~10:17, \quoted{Faith comes by 
hearing, and hearing by the word of Christ}.%
    \Autocite{Catholic:Catechismus1614}
The Roman Church, it says, is the embodiment of Christ, the Word of God, in the 
world.
Its teachers must accommodate their teaching to \quoted{the sense of hearing 
and intelligence} of listeners.
But the catechism also acknowledges that listeners must be trained to hear 
properly.

This paradoxical challenge of appealing to the ear even while training it was 
most clear on the missions and in colonial settings.
For example, Duarte de Sande's account of five Japanese youths who toured Europe
with the Jesuits in the 1580s demonstrates that both personal subjectivity and
cultural conditioning affected people's ability to hear the Church's teaching
rightly.%
  \Autocite{Massarella:JapaneseTravellers}
People needed to learn to listen properly in order to acquire faith and not be 
deceived or confused.

Athanasius Kircher wrote in 1650 that the right kind of music fitted to sacred
words could move listeners to \quoted{experience the truth of what was said}.%
  \Autocite{Kircher:Musurgia}
But Kircher takes for granted that listeners know how to hear music rightly, 
even as he elsewhere acknowledges that individuals and nations perceive music 
differently.
Both individual subjectivity and cultural conditioning, then, posed challenges
to the task of making faith appeal to the ear.
While in some ways music promised to break through obstacles of perception, 
music also required individual and communal \quoted{ear training} to 
successfully connect people with the Church.

Spanish literature gives evidence of widespread anxiety about hearing and faith.
A striking example is the 1634 Corpus Christi play, \worktitle{El nuevo palacio
del Retiro}, by court poet Pedro Calderón de la Barca.%
    \Autocite{Calderon:Retiro}
As part of the festivities inaugurating Philip IV's new palace retreat, the 
Buen Retiro, Calderón staged both a contest of the senses before the allegorical
character Faith.
Faith crowns Hearing not because of his strengths but because of the sense's 
\quoted{incertitude}, because he humbly admits that he is \quoted{the sense 
most easily deceived}.
If hearing is so easily deceived, how could the church effectively use auditory
art forms like sung poetry and drama to propagate faith?  
This doubt becomes even stronger when Calderón stages an extended anti-Semitic 
dialogue in which Judaism confesses his inability to believe what Faith tells
him, lamenting, \quoted{I have listened to Faith without faith}.
Recalling the Tridentine Catechism's emphasis on both accommodation and 
training, how was one supposed to acquire the capacity to hear \quoted{the 
Faith} of the church, \emph{with} faith?

The villancicos studied in this chapter form part of the effort to make faith
appeal to hearing, by addressing directly the relationship between the two.
Two pieces based on the same poem stage a contest of the senses like that of
Calderón; they were composed in the later seventeenth century by successive
chapelmasters at Segovia Cathedral Miguel de Irízar and Jerónimo de Carrión.
Each sense receives a \soCalled{hearing} before Faith, who gives first prize to
Hearing.
Two \quoted{villancicos of the deaf} by Matías Ruíz of Madrid and Juan 
Gutiérrez de Padilla of Puebla demonstrate a prevalent fascination with the 
problematic role of hearing in acquiring faith.
Both pieces present comic catechism lessons between religous teachers and 
hard-of-hearing men who mishear the teaching in absurd and even impious ways.
These pieces mock people with hearing impairments while also poking
fun at ineffective and incompetent teachers, and warning against spiritual
deafness.

Devotional music about hearing and faith reflects a certain amount of doubt and
anxiety about how much hearing could be trusted, who had the proper capacity for
faithful hearing, and how the church could overcome these obstacles to
promulgate its teaching and devotion.
Like the metamusical pieces studied in the first chapter, these villancicos do 
not assume passive listeners ready to be branded with Christian dogma; rather, 
they challenge listeners to think about the act of hearing itself.

%*******
\subsection{Part II: Listening for Unhearable Music}

The next part of the book presents detailed case studies of particular 
villancico traditions (related poems and their musical settings) that represent 
divine and heavenly music.
These metamusical pieces encapsulate, through their musical structures, 
contemporary beliefs about music's sacred power.
As the creators of villancicos developed a consistent set of theological, 
poetic, and musical tropes, they used this subgenre to prove their mastery and 
establish themselves within a tradition of metamusical representation.

\subsubsection{Chapter 3: Hearing the Christ-Child Sing in the \quoted{Voices of
the Chapel Choir} (Puebla, 1657)}

The next two chapters interprets villancico families that represent Christ as
both singer and song.
The first is \worktitle{Voces, las de la capilla}, set by Juan Gutiérrez de
Padilla (Puebla, 1657).
The second is \worktitle{Suspended, cielos, vuestro dulce canto}, set by Joan 
Cererols (Montserrat, ca. 1660).
Both pieces invite hearers to listen for the voice of Christ echoing in the
voices of the chapel choir, and they proclaim that human voices can be a means
through which God reorders the created world to be in harmony with the incarnate
Christ.

In Padilla's \worktitle{Voces}, the chorus exhorts itself, \quoted{Voices of 
the chapel choir,/ keep count with what is sung,/ for the King is a musician}.
In a series of ingenious and cryptic conceits after the manner of the
influential Baroque poet Luis de Góngora, the piece celebrates the baby Jesus as
the heir of the musician-king David, as chapelmaster, singer, and even as the
song that is sung.
Christ's life, death, and resurrection as God incarnate represent a 
\quoted{composition} in which the divine chapelmaster could \quoted{prove the 
consonances of Man and God}.
Padilla projects and dramatizes the poem through musical techniques both literal
and figurative, from punning on the numbers and solmization syllables in the
poem to evoking the song of angels, men, and beasts in Christ's stable in the
style of a double-choir madrigal.

Padilla's piece distills the central elements of Christmas devotion in his time.
The chapter unlocks its rich meanings by reading it in the context of liturgy, 
preaching, and Biblical interpretation.
The sources for interpretation are based on materials that were available to
Padilla himself, from the evidence of Puebla's seminary and convent libraries;
with particular emphasis on the exegetical writings of Cornelius à Lapide and a
contemporary model sermon of Fray Luis de Granada.

A central part of the trope of voice at Christmas is the patristic concept of 
the Christ-child as \term{Verbum infans}, found in Christmas sermons by St.
Augustine and St. Bernard of Clairvaux.
The trope reveres Christ as the Word made flesh (Jn. 1) who as an infant (Latin 
\foreign{in-fans}) is unable to speak any words, but who is, in his body, the 
Word itself.
This villancico extends the trope by interpreting Christ's inarticulate cries 
as musical performance.

The relationship of Padilla's \worktitle{Voces, las de la capilla} to similar 
pieces suggests that Padilla used this metamusical piece to establish his 
compositional pedigree and demonstrate his mastery of the craft.
The chapter compares Padilla's text to a similar one set ten years earlier by
the chapelmaster of Seville Cathedral, Luis Bernardo Jalón; there is also
evidence for an earlier lost setting by Francisco de Santiago, Jalón's
predecessor at Seville and possibly a personal acquaintance of Padilla's.

\subsubsection{Chapter 4: Heavenly Dissonance (Montserrat, 1660s)}

The next chapter focuses on another treatment of the same \emph{Verbum
infans} trope, with greater emphasis on the relationship between heavenly music
and earthly music, which includes the music of the spheres.
The villancico \worktitle{Suspended, cielos} by Joan Cererols, represents a
family of villancico poems set to music in various versions at least eight times
between 1651 and the end of the seventeenth century.
\quoted{Suspend, O heavens, your sweet chant}, Cererols's chorus begins:
the harmony of the spheres is out of tune, and must give 
way to \quoted{the newest consonance}---the voice of the Christ-child.
Here again, Christ is \term{Verbum infans}, and his cries are the
\quoted{plainchant} on which the music of a new creation will be based.

The musical setting projects the musical concepts of the poem through a 
masterful musical-rhetorical discourse.
Cererols structures the piece with two primary motivic subjects and two 
contrasting styles, which map onto a contrast between worldly music (including 
the spheres) and divine and angelic music. 
At one point the motive associated with divine music becomes the subject of an 
eight-voice fugue for the full double chorus, with fugal answers in inversion.
At the end of the piece, setting the words \quoted{bears the plainchant to the 
angels}, it becomes the literal cantus firmus of a section in the style of a 
cantus-firmus motet.

In keeping with a Neoplatonic listening practice, Cererols draws listeners'
attention to the musical structure itself as the bearer of meaning.
Cererols highlights the artificiality, indeed the imperfection, of his own 
music, when he sets the phrase \quoted{listen to the newest consonance} with 
the last word on a prominent unprepared dissonance.
This dissonance becomes an ironic symbol that prompts the hearer to listen for 
a higher music, audible only with the ears of faith.

As with \worktitle{Voces}, the chapter traces the genealogy of the
\worktitle{Suspended, cielos} family through multiple poetic versions, which
also provide evidence for lost musical settings.
The imprints demonstrate that from a 1651 Royal Chapel performance in Madrid, 
the text traveled quickly through a network of affiliated composers across 
Iberia and into the New World, including a fragmentary musical setting from a 
convent in Ecuador.
The chapter delineates three main textual families, one of which is based on 
the influential villancico poet Manuel de León Marchante's elaborated version 
of the poem. 
The different versions reflect changing attitudes toward heavenly music.

This villancico connects Christ's voice, and the choir's voices, to the music 
of the planetary spheres, prompting an interpretation rooted in contemporary 
astronomical beliefs.
This also allows for a reappraisal of John Hollander's study of themes of
heavenly music in English poetry of this period.%
  \Autocite{Hollander:Untuning}
In an age of new astronomical discoveries, it may be that this emphasis on the 
untunefulness of the heavens, bears witness to a growing anxiety about the old 
cosmology.
But rather than providing evidence for a \quoted{disenchantment} and secularization
process, villancicos on heavenly music continue to manifest genuine belief in
the old cosmology; though they do show an increasing preference for human affective
experience over abstract astronomical speculation.


\subsubsection{Chapter 5: The Earthly Side of Celestial Music (Segovia, 1680s)}

The first piece of the cycle composed for Christmas 1678 at Segovia Cathedral by
its chapelmaster Miguel de Irízar is a metamusical piece about celestial music
coming down to earth: \worktitle{Qué música celestial}.
This chapter traces the tropes of heavenly music as they continue to develop in
the later seventeenth-century. 
By focusing on the composition of this Christmas cycle, the chapter
provides the first detailed study of how a complete set of villancicos was
composed, from assembling the poetic texts to drafting the musical setting and
having it copied and performed.
This study shows the earthly side of creating heavenly music by bringing
readers into Irízar's workshop as he provides for the specific needs of his
local community.

Segovia Cathedral's remarkable archive is one of the only places to preserve a 
large number of draft scores by seventeenth-century Spanish composers, rather 
than just performing parts.
These sources are even more precious because Miguel de Irízar wrote them in
makeshift notebooks assembled from his received letters, fitting music on the
backsides and margins of the letters.
The dates on the letters, then, allow for an unprecedented amount of detail in
tracking Irízar's compositional process.
Moreover, the letters are largely correspondence from other musicians regarding 
the exchange of villancico poetry.
Thus, focusing on the cycle of pieces for Christmas 1678, it is possible to 
determine exactly how Irízar obtained all of the poems for his cycle through 
his network of colleagues, and how he reworked these sources into a coherent 
cycle of his own.

The first piece in the set, \worktitle{Qué música celestial}, continues the 
traditions of metamusical villancicos using simple but ingenious means to 
create contrasts between earthly and heavenly music.
Irízar was an economical composer in every sense of the word.
His output was designed to meet local devotional needs, including a special 
local cult of St. Blaise (San Blas), for which Irízar wrote numerous 
villancicos. 
In the difficult economic environment of late seventeenth-century Spain, Irízar 
found ways to use his scarce resources to meet local demand and support the 
community both spiritually and practically.

\subsubsection{Chapter 6: Offering and Imitation (Zaragoza and Puebla,
1670--1700)}

% Burning Hearts and Voices (Bruna, Ambiela)

The first section of this chapter compares two villancicos in which one composer
is clearly modeling a metamusical villancico on the poetry and musical setting
of a more senior composer.
As Pedro Calahorra discovered, a villancico poem beginning \worktitle{Suban las 
voces al cielo} was set first by Pablo Bruna, the blind organist of Daroca in 
the region of Zaragoza, and then in a variant text by Miguel Ambiela, who had 
studied in Daroca just after Bruna's death and who would later go on to 
prestigious positions in Lérida, Zaragoza, and Toledo.%
  \Autocite{Calahorra:Suban}
The similarities between the pieces make it clear that Ambiela's work is a 
conscious homage to Bruna's, even as the differences also demonstrate how 
Ambiela differentiated himself from the older generation's aesthetic. 

Ambiela's piece is a kind of offering, then, putting forth his best effort in 
homage to an admired older master.
This is fitting since both villancico poems represent music as a form of 
offering.
The chapter examines the Bruna villancico in light of Spanish mystical 
theology, particular the concepts of fire in the \worktitle{Flame of Living 
Love} by St. John of the Cross.
Iconography and epigrammatic poetry by Sebastián de Covarrubias deepen the
understanding of music's relation to fire and air, explaining how music could
serve as a fitting means of self-offering.

Ambiela's version keeps the specific musical references from Bruna's text, but 
resituates them as an act of devotion to the Blessed Virgin.
His villancico shows how some of the same musical tropes that we have seen 
associated with Christ's Incarnation and Eucharistic presence could be adapted 
to suit sanctoral devotion. 

% Christ as a \term{Vihuela} (Cáseda)

A piece by another composer in the region of Zaragoza, José de Cáseda 
(\worktitle{Qué música divina}, ca. 1700), further develops the concepts of 
music as an act of offering, as it meditates on Christ's passion through the 
conceit of Christ as a \term{vihuela}. 
The piece identifies Christ with both a \term{vihuela} and a \term{cítara}, 
comparing the stretching of strings over the bridge to Christ's being stretched 
out on the cross, the placement of tuning pins to the nails in Christ's hands, 
the bow to the lance that pierced his side.
It specifies that this is a seven-course vihuela, because of the seven 
sacraments flowing from the spear wound.
This music, it says, \quoted{is not for the ears}, for even if anyone could 
hear it, \quoted{as many notes as they would hear, they would perceive as 
false}---that is, as out of tune. 

Taking the conceit literally provides insights into odd features of the musical 
structures: Cáseda imitates the structure and style of the vihuela through the 
vocal texture, such that the choir itself embodies the vihuela, and by 
extension, Christ's own body.
Allegorical traditions connect plucked chordophones---originally the cithara 
and lyre---to the body of Christ and by extension to the bodies of virgins and 
martyrs, as Craig Monson has shown.%
  \Autocite{Monson:DivasConvent}
The villancico extends this patristic tradition by applying it to a 
contemporary Spanish instrument.

To embody the poetic concept of the \quoted{music} of Christ's passion as 
sounding \quoted{false}, Cáseda plays with various types of musical falsehood.
These range from blatant solecisms like parallel fifths, to an enigmatic passage
that appears to intentionally defy \term{musica ficta} conventions, creating a
kind of impossible music.
As the piece goes to rather extreme lengths to depict musical 
\quoted{falsehood}, it also demonstrates the limits of imitation within the 
metamusical villancico tradition, since it shows the pressure to outdo previous 
composers in highlighting musical artifice.
The piece only survives in a copy from a convent in Puebla, and thus it also 
gives insight into the unique functions of villancicos within cloistered 
communities and in a colonial context.

%**************************
\section{Plans}

The book follows the general plan and content of my 2015 dissertation, but this 
will be a new work.%
    \Autocite{Cashner:PhD}
The chapters are reorganized and many of the ideas are reconceived in light of a
longer and deeper engagement with the primary sources and continued dialogue
with current secondary literature.
I am rewriting the chapters completely in a more concise, argument-driven manner
for a broader audience.
The chapters should be able to stand somewhat independently so that they can be
assigned as course reading assignments.

Though the book largely focuses on close textual analysis, I am taking care to
avoid technical language wherever possible, and to frame the analysis within
broader discourses of wide interest.
The analysis is also leavened by biographical sketches and historical vignettes,
aiming to help readers imagine the full cultural context of this
music as ritual and performance, as a human activity in a particular place and
time.
A preface, glossary, and other supplemental material will provide explanation of
the more difficult concepts in music theory, literary theory, theology, and the
Spanish colonial system.
Scholars from different disciplines will be able to follow the central arguments
even without every detail of the musical, poetic, and theological analyses.

Each chapter will be about 20,000 words in manuscript, for a total of around
120,000 words.
I estimate that I will need three to four months to complete each chapter.
Chapters two and three are already finished; four chapters remain plus
conclusions and polishing the whole.
(The introduction and conclusion chapters will be shorter than the others.)
I am planning to complete the manuscript by December 1, 2019.

There will be two or three figures in each chapter of black-and-white
photographs of paintings and manuscript sources or scans of printed iconography.
All of the chapters include several tables, and several feature my own line
diagrams; the book should also include a map of the cities discussed in Spain
and New Spain.  
There will be about five or six full-page musical examples in each chapter,
which I will prepare myself.
I intend to secure funding to make recordings of all the music examples in the
book, which will make the book more accessible to all readers.
I am also capable of designing and maintaining an accompanying website with
videos, images, links to related scholarship, and other supplementary materials.

To accompany the monograph I am separately publishing full musical editions for
the scores discussed in the book through the peer-reviewed Web Library of
Seventeenth Century Music.%
    \Autocite{Cashner:WLSCM32}
These editions will be available freely online.
The first volume of these editions was published last year, and the second is in
progress.

% For reviewers, I would recommend Geoffrey Baker, Bernardo Illari, David Irving,
% Tess Knighton, Paul Laird, Elisabeth LeGuin, Álvaro Torrente, and David
% Yearsley.

\end{document}
