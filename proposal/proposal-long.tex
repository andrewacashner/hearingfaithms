% BOOK PROPOSAL FOR FIRST BOOK, ON VILLANCICOS
% ANDREW A. CASHNER
% 2016-06-23  Begun

%**************************
% LONG ZERO-DRAFT VERSION
%**************************

\documentclass[tt]{vcbook-proposal}
\newcommand{\publisher}{the University of California Press}

\begin{document}

\frontmatter

\begin{titlingpage}
\title    {Faith, Hearing, and the Power of Music 
           in Villancicos of the Spanish Empire}
\subtitle {Book Proposal for \publisher}
\author   {Andrew A. Cashner}
\date     {\today}
\maketitle
\end{titlingpage}

\tableofcontents*

%*******************
\mainmatter

\section{Description}

Devotional writers of the seventeenth-century Spanish empire frequently cited St.~Paul's dictum from Romans~10:17 that \quoted{Faith comes through hearing, and hearing by the word of Christ}.
As Roman Catholics in missions and colonies across the globe, and in educational and evangelical efforts in Europe, sought to make faith audible in persuasive new ways, what was the role of music?
In the theological understanding of the time, what kind of power did music have to create or strengthen a link between faith and hearing?
How did church leaders in the Hispanic world use music to appeal to the sense of hearing?
For worshippers, whether the lettered elite or the classes of commoners for whom literacy was primarily a matter of hearing and remembering, how might the participants of Catholic ceremonies with music have actually listened?

There exists a rich and largely unmined vein of sources that can help us explore these questions, in the thousands of surviving villancicos in archives across the world.
Villancicos originated in the late medieval period as a genre of courtly entertainment and sometimes devotion, with elements drawn from common culture.
These were chamber songs on vernacular poems, similar to \term{virelais} or early madrigals.
In the late sixteenth cnetury, though, Spaniards began performing villancicos as an increasingly integral part of the liturgy, interpolating them among the lessons of Matins or in the Mass, especially at Christmas and Epiphany, Corpus Christi, and other high feasts like the Immaculate Conception of Mary.

The genre became a meeting place for elements of elite and common, erudite and comic, celestial and mundane elements.
In every major church from Madrid to Manila, chapelmasters were contractually obligated to compose dozens of villancicos each year, most often in sets of eight or more for Matins, such that the total global production for the seventeenth century alone must be in the hundreds of thousands.
Though huge numbers have been lost or destroyed, there still survive many hundreds of musical settings, and vastly more printed leaflets of villancico poems. 
These are not trivial popular tunes: they are long, complex polyphonic pieces for voices and instruments, often arranged for eight, twelve, or more voices in multiple choirs.
Most feature an expansive, motet-like \term{estribillo} or refrain section for the full ensemble, and strophic \term{coplas} or verses for soloists or a reduced group.

Aside from their ubiquitous presence in colonial Hispanic culture---something that not always noted even in musicological studies---and their mixture of vernacular poetry with multiple styles of music designed to appeal to listeners from the whole array of social positions, villancicos demand our attention because so many of them directly address the topic of music and hearing.
Hundreds of pieces explicitly invoke the sense of hearing with opening lines of \quoted{Listen!} \quoted{Silence!} \quoted{Pay attention}.
The repertoire abounds in representations of angelic choirs at Christmas, dancing shepherds and magi from the far corners of the earth (especially Africa), evocations of instruments, and even elaborate conceptual plays on terms from music theory and philosophy.
These pieces constitute \quoted{music about music} and as such provide as with an unparalelled historical discourse on the nature of music, through the medium of music itself.
The book's central question, then, is as follows: 
\emph{What can we learn about how seventeenth-century Hispanic Catholics understood music's power in the relationship between faith and hearing, by listening closely to villancicos that address this very subject?}

\subsection{Scholarly Context of the Study}

This book will be the first large-scale study to examine in detail the links between music, poetry, and theology in villancicos.
It will be the second major monograph, after Paul Laird's \worktitle{Toward a History of the Spanish Villancico}, to analyze the music of villancicos; and the first to be based on a global selection of sources and to interpret them in light of contemporary theological literature and devotional practices.
This study of villancicos helps us understand what kind of power music had in the understanding of Hispanic Catholics of the seventeenth century.

While musicologists have increasingly turned their interest to early modern Iberian music, most of the recent studies in English have focused on social functions of music, based on archival documentation and engaging with postcolonial thought and other critical theory.
This admirable work has greatly advanced our understanding of Hispanic musical culture, but at the same time it has not been sufficiently grounded in study of the actual surviving sources of musical practice.
No doubt, much has been lost, and what remains presents a skewed perspective focused on the elite music of urban cathedrals; but this book will demonstrate how much we can still learn from the copious sources we do have.
The book's primary goal is to contribute to the development of a historically grounded interpretive framework for early modern religious music.
It is an interpretive study based primarily on poetic and musical texts, rather than a social history or anthropological study of music's functions.
This approach is valuable because there is an aspect of lived experience in any age that is only accessible through expressive culture; and particularly the realm of musical experience cannot be penetrated through descriptions or records of music-making alone. 

Instead, if we want to understand what early modern Spanish subjects believed about music, then we must pay close attention to the music they made as an expression of those beliefs.
Likewise, if we want to understand how their music worked even on a formal-structural level, we must seek to understand its creators' and hearers' beliefs about music.
This project seeks points of intersection between structures of music and structures of belief, and, to a  limited extent, structures of society.


%***************************
\section{Outline of Chapters}

\subsection{Part I: Listening for Faith in Villancicos}

The book is organized in three parts.
The first, \quoted{Listening for Faith in Villancicos}, builds a foundation for understanding villancicos as sources of theology, as windows into historical beliefs about sensation and faith.
The second part, \quoted{Listening for Unhearable Music}, presents four case studies in interpreting specific villancico families on the subject of music, with an emphasis on the links between earthly, heavenly, and divine music.
The final part, \quoted{Listening for Community}, analyzes how processes of villancico composition contributed to the specific needs of local communities, focusing on Segovia in old Castile and Puebla de los Ángeles in New Spain.
The first part focuses on tracing intellectual themes relevant to the whole seventeenth century; the second part focuses closely on the meanings of individual poetic and musical texts in specific local contexts; and the third part connects the history of musical composition in particular moments to broader social, political, and economic aspects of local communities.

\subsubsection{Chapter 1: Villancicos as Musical Theology}

The first chapter makes the case for understanding villancicos as embodiments of theological beliefs about music---that is, as ways of hearing faith through music.
The chapter presents the results of a global survey and sampling of villancico poems and music, examining the different ways that villancicos represent or discuss music-making.
These include pieces on topics of musical instruments (for example the \term{clarín} or clarion, drums, castanets), specific songs and dances like \term{jácaras} and imitations of African and Indian dances, depictions of angelic music and music of the spheres, and learned pieces playing on solmization syllables and terms from music theory to create a double discourse and music and theology.
The chapter considers the peculiar semiotics of musical performances that point to other forms of music, or even to themselves, raising questions about how Hispanic Catholics understood music's power to represent and signify.

Along the way the chapter also establishes fundamental concepts and analytical methods that will familiarize readers with villancicos as a genre and with the specialized approaches developed for this project, since there is thus far little consensus (or even conversation) about musical structure in seventeenth-century Hispanic music.
The examples are chosen to represent fairly the gloabl reach of this genre, focusing on composers with wide influence, such as Juan Hidalgo, Cristóbal Galán, Joan Cererols, and Juan Gutiérrez de Padilla.
The chapter briefly situates the project in the context of scholarship on its broad themes---faith, sensation, music's power---in cultural studies, as well as in relation to the small but growing literature on early modern Hispanic villancicos and other music.

\subsubsection{Chapter 2: Making Faith Appeal to Hearing}

The second chapter explores the complex relationships between hearing and faith in the period after the Council of Trent, especially in the church's missionary and colonial efforts.
It looks at theological literature---devotional, dogmatic, and dramatic---together with villancicos on themes of sensation and faith, such as pieces presenting contests of the senses, and representations of deafness and sensory confusion.
The chapter explores both the theological climate in which villancicos were created and heard, and how villancicos themselves contributed to that climate.

The Roman Catechism produced by decree of the Council of Trent begins with an exposition of Romans~10:17, \quoted{Faith comes by hearing, and hearing by the word of Christ}.
The catechism argues that Christ who is the Word of God created the Church as the means through which God would speak to the world, and so the Church needed teachers to proclaim the word, and disciples to listen, so that all could be brought into communion with Christ the Word.
To do this, the catechism says, teachers must accommodate their teaching to \quoted{the sense of hearing and intelligence} of the hearers; but in the same breath the catechism says that \quoted{those whose senses have been properly trained} will be able to derive benefit from the teaching.
In other words, teachers must accommodate the sense of hearing even while training it.
Connecting this official document to a vernacular expositions of the Creed by Fray Antonio de Azevedo and to discussions of music from the missions (such as the experience of five Japanese youths on a European tour in the 1580s), I demonstrate that this way of understanding the relationship between hearing and faith created certain paradoxes and challenges, since one needed to be properly capacitated to hear faith with faith.
As missionaries were beginning to understand, both personal subjectivity (differing individual temperaments, education, ability) and cultural conditioning affected people's ability to hear the Church's teaching rightly.
Athanasius Kircher wrote in 1650 that music added power to preaching such that those who heard a sacred text fitted with appropriate music would go beyond understanding the text intellectually and would in fact be \quoted{carried away by joy} to \quoted{experience the truth of what was said}.
In other words, music connected subjective experience to objective truth. 
But Kircher takes for granted that listeners know how to hear music rightly, evn as he elsewhere acknowledges that both individuals and cultural groups have differing temperaments and thus perceive differently.
While in some ways music seemed to break through obstacles of perception and understanding, music also placed even greater demands on individual and communal \quoted{ear training} to successfully connect people with the Church.

Literary contests of the senses help us understand how Catholics believed the senses to operate in relation to faith, and also reveal a certain amount of anxiety about the relationship.
In a Corpus Christi play (\term{auto sacramental}) for the festivities inaugurating Philip IV's new palace retreat, the Buen Retiro, in 1634, court poet Pedro Calderón de la Barca staged both a contest of the senses before faith, and an extended allegory of obstacles to faith.
In the contest, each of the personified senses boasts of its powers before the character Faith, except for Hearing, who admits that he is \quoted{the sense most easily deceived}.
He perceives not a man directly, but only his voice, which may be feigned or echoed, and thus can never be certain of the true object of his sensations.
Faith crowns Hearing as the most favored sense because he alone is humble and relies on faith rather than his own powers of \quoted{sense} or reason.
In another scene, the senses are confronted by the Eucharist, and the transubstantiated host confounds all the other senses.
\quoted{I smell only bread}, says Smell, for example.
But Hearing simply trusts in the priest's repetition of Christ's words which he has heard, \quoted{This is my body}.
Hearing is thus presented as a portal to truths that go beyond normal sensation.
But at the same time, Calderón so vividly dramatizes Hearing's weakness and incertitude that he seems to encourage listeners to question what they hear as much as they are to trust in it.

This seemingly paradoxical message seems designed to urge Catholics to trust the Church but nothing else, to submit their subjective experience to the Church's authority.
But despite the triumphal political context of this play, in which Philip IV and Christ are made nearly indistinguishable, it is remarkable how much room Calderón leaves on the stage for doubt and uncertainty.
Indeed, the last half of the play features a long dialogue between Faith and \foreign{Judaísmo}, the stock allegoricla character of the unbelieving Jew.
Faith tries to explain the Eucharist to Judaism, but Judaism cannot comprehend, repeating the refrain, \quoted{For I have listened to Faith without faith}.
How, then, we might ask at our historical and cultural remove, was the Church supposed to use the auditory medium of music to propagate faith, if hearing could not be trusted? 
What if some people lacked the necessary capacity to hear \quoted{the Faith} \emph{with} faith?

A series of villancicos on the theme of hearing do not answer these questions, but the do provide a richer context for understanding them, and for considering to what extent people in the seventeenth century were asking these questions themselves.
One tradition of villancicos---that is, a set of villancico poems and musical settings that are variants of a single poem---presents a contest of the senses similar to Calderón's.
Two late-seventeenth-century musical settings survive of a poem attributed to Zaragoza poet Vicente Sánchez, by successive chapelmasters at Segovia Cathedral, Miguel de Irízar and Jerónimo de Carrión.
In both pieces, each sense has a \soCalled{hearing} before Faith, but only the sense of Hearing, in the form of music, prevails.
Irízar's festival setting for a large polychoral ensemble evokes battle pieces in staging its contest, while Carrión's solo continuo song encourages a more individual contemplation of the relation between sensation and faith. 
These pieces demonstrate how traditional Scholastic discourses on the powers and relative merits of the sense, from Aristotle and Aquinas, were presented to a broader audience.

Providing more perspectives on the problems of this relationship are pieces depicting sensory confusion (like Cristóbal Galán's\X{} synesthetic \worktitle{Oigan todos del ave y mira la voz}) or impairment, like two dialogues with \gloss{sordos}{deaf or hard-of-hearing men} from mid-seventeenth century Madrid (by Matías Ruíz) and Puebla (by Juan Gutiérrez de Padilla).
These dialogues present mock catechism lessons between a friar and a man who mishears everything he is taught in a comic way.
In different ways, both pieces not only mock the deaf man's impairment (at the same time as the first schools for the deaf were being founded in Spain), but also poke fun at ineffective and incompetent teachers, and at the spiritual deafness of all Christians.
Though the sense of doubt in these villancicos is perhaps not as pressing as in Calderón's play, these pieces on hearing and faith manifest a widespread fascination with sensation and a certain level of anxiety about how much it could be trusted.
They challenge simplistic notions of Catholic devotional music as an imposition of dogma in the manner of twentieth-century propaanda. 
Like the metamusical pieces studied in the first chapter, these villancicos do not assume passive listeners ready to be branded with Christian dogma; rather, they challenge listeners to think about the act of hearing itself.
Perhaps they even create some space for considering doubt, even if the ultimate goal is to assert confidence in the Church's teaching and devotion.

\subsubsection{Theological Functions of Villancicos}


%*******
\subsection{Part II: Listening for Unhearable Music}

\subsubsection{Christ as Singer and Song (Padilla)}

\subsubsection{Heavenly Dissonance (Cererols)}

\subsubsection{Burning Hearts and Voices (Bruna, Ambiela)}

\subsubsection{Christ as a \term{Vihuela} (Cáseda)}


%*******
\subsection{Part III: Listening for Community}

\subsubsection{Labor, Economy, and Devotion in Segovia}

\subsubsection{Building the Colonial City in Puebla}

%**************************
\section{Plans, Goals, Audience}


\printbibliography

\end{document}
