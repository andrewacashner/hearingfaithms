The second chapter explores the complex relationships between hearing and faith in the period after the Council of Trent, especially in the church's missionary and colonial efforts.
It looks at theological literature---devotional, dogmatic, and dramatic---together with villancicos on themes of sensation and faith, such as pieces presenting contests of the senses, and representations of deafness and sensory confusion.
The chapter explores both the theological climate in which villancicos were created and heard, and how villancicos themselves contributed to that climate.

The Roman Catechism produced by decree of the Council of Trent begins with an exposition of Romans~10:17, \quoted{Faith comes by hearing, and hearing by the word of Christ}.
The catechism argues that Christ who is the Word of God created the Church as the means through which God would speak to the world, and so the Church needed teachers to proclaim the word, and disciples to listen, so that all could be brought into communion with Christ the Word.
To do this, the catechism says, teachers must accommodate their teaching to \quoted{the sense of hearing and intelligence} of the hearers; but in the same breath the catechism says that \quoted{those whose senses have been properly trained} will be able to derive benefit from the teaching.
In other words, teachers must accommodate the sense of hearing even while training it.
Connecting this official document to a vernacular expositions of the Creed by Fray Antonio de Azevedo and to discussions of music from the missions (such as the experience of five Japanese youths on a European tour in the 1580s), I demonstrate that this way of understanding the relationship between hearing and faith created certain paradoxes and challenges, since one needed to be properly capacitated to hear faith with faith.
As missionaries were beginning to understand, both personal subjectivity (differing individual temperaments, education, ability) and cultural conditioning affected people's ability to hear the Church's teaching rightly.
Athanasius Kircher wrote in 1650 that music added power to preaching such that those who heard a sacred text fitted with appropriate music would go beyond understanding the text intellectually and would in fact be \quoted{carried away by joy} to \quoted{experience the truth of what was said}.
In other words, music connected subjective experience to objective truth. 
But Kircher takes for granted that listeners know how to hear music rightly, evn as he elsewhere acknowledges that both individuals and cultural groups have differing temperaments and thus perceive differently.
While in some ways music seemed to break through obstacles of perception and understanding, music also placed even greater demands on individual and communal \quoted{ear training} to successfully connect people with the Church.

Literary contests of the senses help us understand how Catholics believed the senses to operate in relation to faith, and also reveal a certain amount of anxiety about the relationship.
In a Corpus Christi play (\term{auto sacramental}) for the festivities inaugurating Philip IV's new palace retreat, the Buen Retiro, in 1634, court poet Pedro Calderón de la Barca staged both a contest of the senses before faith, and an extended allegory of obstacles to faith.
In the contest, each of the personified senses boasts of its powers before the character Faith, except for Hearing, who admits that he is \quoted{the sense most easily deceived}.
He perceives not a man directly, but only his voice, which may be feigned or echoed, and thus can never be certain of the true object of his sensations.
Faith crowns Hearing as the most favored sense because he alone is humble and relies on faith rather than his own powers of \quoted{sense} or reason.
In another scene, the senses are confronted by the Eucharist, and the transubstantiated host confounds all the other senses.
\quoted{I smell only bread}, says Smell, for example.
But Hearing simply trusts in the priest's repetition of Christ's words which he has heard, \quoted{This is my body}.
Hearing is thus presented as a portal to truths that go beyond normal sensation.
But at the same time, Calderón so vividly dramatizes Hearing's weakness and incertitude that he seems to encourage listeners to question what they hear as much as they are to trust in it.

This seemingly paradoxical message seems designed to urge Catholics to trust the Church but nothing else, to submit their subjective experience to the Church's authority.
But despite the triumphal political context of this play, in which Philip IV and Christ are made nearly indistinguishable, it is remarkable how much room Calderón leaves on the stage for doubt and uncertainty.
Indeed, the last half of the play features a long dialogue between Faith and \foreign{Judaísmo}, the stock allegoricla character of the unbelieving Jew.
Faith tries to explain the Eucharist to Judaism, but Judaism cannot comprehend, repeating the refrain, \quoted{For I have listened to Faith without faith}.
How, then, we might ask at our historical and cultural remove, was the Church supposed to use the auditory medium of music to propagate faith, if hearing could not be trusted? 
What if some people lacked the necessary capacity to hear \quoted{the Faith} \emph{with} faith?

A series of villancicos on the theme of hearing do not answer these questions, but the do provide a richer context for understanding them, and for considering to what extent people in the seventeenth century were asking these questions themselves.
One tradition of villancicos---that is, a set of villancico poems and musical settings that are variants of a single poem---presents a contest of the senses similar to Calderón's.
Two late-seventeenth-century musical settings survive of a poem attributed to Zaragoza poet Vicente Sánchez, by successive chapelmasters at Segovia Cathedral, Miguel de Irízar and Jerónimo de Carrión.
In both pieces, each sense has a \soCalled{hearing} before Faith, but only the sense of Hearing, in the form of music, prevails.
Irízar's festival setting for a large polychoral ensemble evokes battle pieces in staging its contest, while Carrión's solo continuo song encourages a more individual contemplation of the relation between sensation and faith. 
These pieces demonstrate how traditional Scholastic discourses on the powers and relative merits of the sense, from Aristotle and Aquinas, were presented to a broader audience.

Providing more perspectives on the problems of this relationship are pieces depicting sensory confusion (like Cristóbal Galán's\X{} synesthetic \worktitle{Oigan todos del ave y mira la voz}) or impairment, like two dialogues with \gloss{sordos}{deaf or hard-of-hearing men} from mid-seventeenth century Madrid (by Matías Ruíz) and Puebla (by Juan Gutiérrez de Padilla).
These dialogues present mock catechism lessons between a friar and a man who mishears everything he is taught in a comic way.
In different ways, both pieces not only mock the deaf man's impairment (at the same time as the first schools for the deaf were being founded in Spain), but also poke fun at ineffective and incompetent teachers, and at the spiritual deafness of all Christians.
Though the sense of doubt in these villancicos is perhaps not as pressing as in Calderón's play, these pieces on hearing and faith manifest a widespread fascination with sensation and a certain level of anxiety about how much it could be trusted.
They challenge simplistic notions of Catholic devotional music as an imposition of dogma in the manner of twentieth-century propaanda. 
Like the metamusical pieces studied in the first chapter, these villancicos do not assume passive listeners ready to be branded with Christian dogma; rather, they challenge listeners to think about the act of hearing itself.
Perhaps they even create some space for considering doubt, even if the ultimate goal is to assert confidence in the Church's teaching and devotion.
