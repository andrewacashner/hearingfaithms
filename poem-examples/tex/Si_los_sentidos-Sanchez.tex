\documentclass{aac-poem}
\begin{document}

% Si los sentidos queja forman del Pan Divino
% Critical poetry edition
% Newly TeXed and revised, 2016/04/05

\def\srcC{C} % Carrión
\def\srcI{I} % Irízar

\begin{poemtranslation}
\begin{original}

\StanzaSection{6}[Estribillo]
Si los sentidos queja &
forman del Pan Divino, &
porque los que ellos sienten &
no es de Fe consentido, &
\critnote{hoy todos con la Fe}
    {\variant{\srcC}{todos hoy con la fe}} sean oídos. &
\critnote{No se den por sentidos los sentidos}
    {\foreign{Darse por sentido}, idiom for taking offense at something}.
\SectionBreak

\def\tagline{\emph{no se den por sentidos los sentidos.}}

\StanzaSection{7}[Coplas]

\critnote{}[Coplas] 
  {Irízar sets Sánchez coplas in the order 1, 7, 6, 4, 3, and 5; he omits 2.
   Carrión sets 1, 3, 4, 6, 7; omits 2, 5.}

1. Si en ellos va el no ver bien &
los ojos de que se admiran, &
pues mal verán lo que miran &
si no miran lo que ven, &
si su ceguedad es quien &
los tiene impedidos, &
\tagline \&

\Stanza{7}
2. Entre velos transparentes, &
no se ve Dios Encarnado, &
que el color se la ha mudado, &
y lo hazen sus accidentes, &
si en nubes rayos lucientes &
están escondidos, &
\tagline \&

\Stanza{7}
3. Toca el tacto pero 
  \critnote{yerra}
  {\variant*{\srcI}{ierra}; \variant*{\srcC}{hierra};
    probably variants of \gloss{erra}{errs, misses}.} &
\critnote{que si}{\variant{\srcC}{pues}}
  en que es pan se equivoca, &
aunque todo un Cielo toca, &
no toca en Cielo, ni en tierra, &
toca misterio, y si encierra &
portentos no oídos, &
\tagline \&

\Stanza{7}
4. Que tenga voto, no es justo, &
el gusto en este Manjar, &
que el gusto en él no ha de entrar &
aunque el Manjar entre en gusto: &
mas si les causa disgusto &
no ser admitidos, &
\tagline \&

\Stanza{7}
5. Para que el Manjar alabe &
\critnote{lleve}{\variant{\srcI}{llegue}[arrive]} el gusto con afán &
que \critnote{al que}{\variant{\srcI}{aunque}[even though]} 
    sabe que no es pan &
sabe \critnote{a más}{\variant{\srcI}{más}} de lo que sabe, &
mas si en su esfera no cabe &
y se hallan perdidos, &
\tagline \&

\Stanza{7}
6. Si el olfato se le humilla &
con Fe a entenderle la flor &
le maravilla su olor &
\critnote{porque huele}
{\variant*{\srcI}{por guele}; \variant{\srcC}{porque guele}} a maravilla &
mas si para \critnote{percibilla}
{\variant{\srcI}{a percebilla}[\sameas percibirla: notice, discern, recognize]} &
no llegan rendidos, &
\tagline \&

\Stanza{7}
7. Porque a Dios puedan gustar, &
en los puntos sus concentos, &
todos sus cinco instrumentos &
la Fe los ha de templar, & 
sino los puede ajustar &
para ser oídos, &
\tagline \&

\end{original}



\begin{translation}
\StanzaSection{6}
If the senses make &
a complaint about the Divine Bread, &
because what they sense &
is not by faith consented, &
today let them all with faith be heard. &
    \critnote{Let the senses not resent it}
    {Or, Let the senses not be considered senses.}.
\SectionBreak

\def\tagline{\emph{let the senses not resent it.}}

\StanzaSection{7}
1. If in them the eyes that admire &
cannot see well, &
since they shall see poorly what they see &
if they do not look at what they see, &
if their blindness is what &
keeps them impaired, &
\tagline \&

\Stanza{7}
2. Within transparent veils, &
God Incarnate is not seen, &
for the color has been changed, &
and its accidents are doing it. &
If in the clouds flashing rays &
are hidden, &
\tagline \&

\Stanza{7}
3. Touch touches but it errs, &
for if in what is bread it is mistaken, &
even though it touches all of Heaven, &
it touches neither Heaven nor earth, &
it touches a mystery, and if it encloses &
unheard portents, &
\tagline \&

\Stanza{7}
4. It is not fair that Taste &
should have a vote on this Morsel, &
for Taste shall not come into this, &
though the Morsel may come into Taste, &
but if it causes distaste &
that the senses are not admitted, &
\tagline \&

\Stanza{7}
5. So that he might praise the Morsel &
bring on taste eagerly, &
for of that which he knows is not bread &
he knows more than what he knows, &
but if it does not fit in his sphere &
and the senses find themselves lost, &
\tagline \&

\Stanza{7}
6. If smell humbles himself, &
by Faith to understand the flower, &
he wonders at its aroma &
because \critnote{it smells wondrous}
{Or, he smells in a wondrous/miraculous manner.}, &
but if in order to perceive it &
the senses do not submit, &
\tagline \&

\Stanza{7}
7. So that they could taste God, &
their tuneful concords on the notes, &
Faith must temper &
all their five instruments, &
moreover, Faith can adjust them &
so that they may be heard; &
\tagline \&

\end{translation}
\end{poemtranslation}
\end{document}
