% Pues la fiesta del niño es, Villancico de los sordos, Ruiz, E-E: Mus. 83-12
% ESTRIBILLO
\begin{poemtranslation}
\begin{original}
\StanzaSection{7}[Introducción Solo]
Pues la fiesta del niño es,  &
y es el día de tanto placer, &
de todo ha de haber. &
Un sordo, muy noticioso &
de letras de humanidad, &
con otro que le pregunta, &
viene a alegrar el portal. 
\SectionBreak

\StanzaSection{3}[Responsión a 8]
Vaya de sordo, &
y háblenle todos recio &
porque oiga a todos. 
\SectionBreak

\Stanza{8}
\speaker{Sordo} Éntrome de hoz, y de coz. &
\speaker{Preg.} ¿Quién llama con tanto estruendo? &
\speaker{S} Hablen alto, que no entiendo, &
sino levantan la voz. &
\speaker{P} Bajad la voz, &
que a Dios gracias no soy sordo. &
\speaker{S} ¿Dice que está el niño gordo? &
pues de eso me alegro mucho. \&

\Stanza{8}
Pues vaya de viestas &
al niño que adoro &
que está como un oro, &
y el coro sonoro &
responda veloz, &
que sordos son &
los que no escuchan &
ni entienden el son. \&
\end{original}

\begin{translation}
\StanzaSection{7}
Since it is the Christ-child's festival, &
and the day of so much enjoyment, &
there must be a little of everything. &
A deaf man, very learned &
in humanist letters, &
with another man who questions him, &
comes to liven up the stable. \&

\Stanza{8}
Hurrah, bring on the deaf man &
and let all speak loudly to him &
so that he can hear all. \&

\Stanza{8}
\speaker{Deaf Man} Here I come, like it or not. &
\speaker{Catechist} Who is calling out with such a ruckus? &
\speaker{D.} Speak up, for I don't understand &
unless you raise your voice. &
\speaker{C.} Lower your voice, &
for by God's grace I am not deaf. &
\speaker{D.} Are you saying the baby is fat? &
well, that sure makes me happy. \&

\StanzaSection{8}
So on with the festivities &
for the Christ-child I adore, &
since he is like a gold coin, &
and let the resounding choir &
respond quickly, &
for the deaf are those &
who do not listen &
nor understand the sound. \&
\end{translation}
\end{poemtranslation}
