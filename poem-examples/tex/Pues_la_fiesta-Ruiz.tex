\documentclass{aac-poem}
\begin{document}

% Pues la fiesta del niño es, Villancico de los sordos, Ruiz, E-E: Mus. 83-12
% ESTRIBILLO
\begin{poemtranslation}
\begin{original}
\StanzaSection{7}[Introducción Solo]
Pues la fiesta del niño es,  &
y es el día de tanto placer, &
de todo ha de haber. &
Un sordo, muy noticioso &
de letras de humanidad, &
con otro que le pregunta, &
viene a alegrar el portal. 
\SectionBreak

\StanzaSection{3}[Responsión a 8]
Vaya de sordo, &
y háblenle todos recio &
porque oiga a todos. 
\SectionBreak

\Stanza{8}
\speaker{Sordo} Éntrome de hoz, y de coz. &
\speaker{Preg.} ¿Quién llama con tanto estruendo? &
\speaker{S} Hablen alto, que no entiendo, &
sino levantan la voz. &
\speaker{P} Bajad la voz, &
que a Dios gracias no soy sordo. &
\speaker{S} ¿Dice que está el niño gordo? &
pues de eso me alegro mucho. \&

\Stanza{8}
Pues vaya de viestas &
al niño que adoro &
que está como un oro, &
y el coro sonoro &
responda veloz, &
que sordos son &
los que no escuchan &
ni entienden el son. \&
\end{original}

\begin{translation}
\StanzaSection{7}
Since it is the Christ-child's festival, &
and the day of so much enjoyment, &
there must be a little of everything. &
A deaf man, very learned &
in humanist letters, &
with another man who questions him, &
comes to liven up the stable. \&

\Stanza{8}
Hurrah, bring on the deaf man &
and let all speak loudly to him &
so that he can hear all. \&

\Stanza{8}
\speaker{Deaf Man} Here I come, like it or not. &
\speaker{Catechist} Who is calling out with such a ruckus? &
\speaker{D.} Speak up, for I don't understand &
unless you raise your voice. &
\speaker{C.} Lower your voice, &
for by God's grace I am not deaf. &
\speaker{D.} Are you saying the baby is fat? &
well, that sure makes me happy. \&

\StanzaSection{8}
So on with the festivities &
for the Christ-child I adore, &
since he is like a gold coin, &
and let the resounding choir &
respond quickly, &
for the deaf are those &
who do not listen &
nor understand the sound. \&
\end{translation}
\end{poemtranslation}
% Pues la fiesta del niño es, Villancico de los sordos, Ruiz, E-E: Mus. 83-12
% COPLAS
\begin{poemtranslation}
\begin{original}
\StanzaSection{4}[Coplas en diálogo, y solo]
\speaker{Preg.} 1. Di, Sordo, si Dios cumplió &
la palabra al rey profeta? &
\speaker{Sordo} No ha venido la estafeta, &
por el tiempo se tardó. \&

\Stanza{4}
\speaker{P} 2. Que llore el Omnipotente, &
nadie en el mundo lo ha oído? &
\speaker{S} Es la verdad: De este oído &
estoy un poco teniente. \&

\Stanza{4}
\speaker{P} 3. A ver al Niño, pastores &
vienen hoy con gran decoro. &
\speaker{S} No hay cosa peor que ser moro, &
di tú, Gil, lo que quisieres. \&

\Stanza{4}
\speaker{P} 4. No digo, sino que amor &
es quien traza tales medios. &
\speaker{S} Hanme dado mil remedios, &
y siempre me hallo peor. \&

\Stanza{4}
\speaker{P} 5. Entended lo que os pregunto, &
que no oyes hacia esta parte. &
\speaker{S} Y lo entiendo: que el dios Marte &
tiene cara de difunto. \&
\end{original}

\begin{translation}
\StanzaSection{4}
\speaker{Catechist} 1. Tell, Deaf Man, if God fulfilled &
the Word to the prophet-king? &
\speaker{Deaf Man} The mailman has not arrived; &
he was delayed because of the season. \&

\Stanza{4}
\speaker{C} 2. That the Omnipotent should cry, &
has anyone in the world ever heard this? &
\speaker{D} It's true: in this ear &
I am a little hard of hearing. \&

\Stanza{4}
\speaker{C} 3. To see the Child, shepherds &
are coming today with great respect. &
\speaker{D} There is nothing worse than being a Moor, &
no matter what you say, Gil. \&

\Stanza{4}
\speaker{C} 4. I say nothing, except that love &
is the one who traces such means. &
\speaker{D} They have given me a thousand remedies, &
and I always find myself worse off. \&

\Stanza{4}
\speaker{C} 5. Understand what I am asking you, &
since you haven't heard up till now. &
\speaker{D} I understand just fine: the God Mars &
has a face like the dead. \&
\end{translation}
\end{poemtranslation}
% PUes la fiesta del niño es, Villancico de los sordos, Ruiz, E-E: Mus. 83-12
% COPLAS 2
\begin{poemtranslation}
\begin{original}
\Stanza{4}
\speaker{P} 6. Lleno de danzas, y bailes, &
el portal es nuestro alivio. &
\speaker{S} Yo he leído a Tito Libio &
pero no trata de frailes. \&

\Stanza{4}
\speaker{P} 7. Cuando el Niño nace, apenas, &
duro el frío le combate. &
\speaker{S} Si él tomara chocolate, &
sintiera menos las penas. \&

\Stanza{4}
\speaker{P} 8. La Reina, al Rey de las vidas &
abriga, que tiembla, y arde. &
\speaker{S} Ésta es, por la mañana, y tarde &
la Reina de las bebidas. \&

\Stanza{4}
\speaker{P} 9. Mira en un pobre portal &
la majestad reducida. &
\speaker{S} La Virgen fue concebida &
sin pecado original. \&

\Stanza{4}
\speaker{P} 10. De nueve coros, aquí &
hacen cielo, y tierra aprecio. &
\speaker{S} No los oigo, canten recio, &
sino dicen mal de mí. \&
\end{original}

\begin{translation}
\Stanza{4}
\speaker{C} 6. Full of all kinds of dancing, &
the stable is our recreation. &
\speaker{D} I have read Titus Livy &
but he doesn't discuss friars. \&

\Stanza{4}
\speaker{C} 7. Scarcely has the Child been born, &
and the cold fights hard against him. &
\speaker{D} If he drank some chocolate, &
he wouldn't feel the hardships so much. \&

\Stanza{4}
\speaker{C} 8. The Queen bundles up the King of life, &
who is trembling, and he glows with warmth. &
\speaker{D} Indeed \add{chocolate} is, by morning or by evening, &
the Queen of the beverages. \&

\Stanza{4}
\speaker{C} 9. See, in a poor stable &
his majesty, reduced. &
\speaker{S} The Virgin was conceived & 
without original sin. \&

\Stanza{4}
\speaker{C} 10. In nine choirs, here &
heaven and earth render worship. &
\speaker{S} I don't hear them---let them sing loud, &
as long as they don't say anything bad about me. \&
\end{translation}
\end{poemtranslation}
\end{document}
