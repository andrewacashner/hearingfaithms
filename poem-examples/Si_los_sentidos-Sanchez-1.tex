% Si los sentidos queja forman del Pan Divino
% Critical poetry edition
% Newly TeXed and revised, 2016/04/05
\documentclass[poem]{vcbook-float}

\def\taglineES{\emph{no se den por sentidos los sentidos.}}
\def\taglineEN{\emph{let the senses not resent it.}}

\begin{document}

\begin{poemtranslation}
\begin{original}

\StanzaSection{6}[Estribillo]
Si los sentidos queja &
forman del Pan Divino, &
porque los que ellos sienten &
no es de Fe consentido, &
hoy todos con la Fe sean oídos. &
No se den por sentidos los sentidos.
\SectionBreak


\StanzaSection{7}[Coplas]

1. Si en ellos va el no ver bien &
los ojos de que se admiran, &
pues mal verán lo que miran &
si no miran lo que ven, &
si su ceguedad es quien &
los tiene impedidos, &
\taglineES \&

\Stanza{7}
2. Entre velos transparentes, &
no se ve Dios Encarnado, &
que el color se la ha mudado, &
y lo hazen sus accidentes, &
si en nubes rayos lucientes &
están escondidos, &
\taglineES \&

\Stanza{7}
3. Toca el tacto pero yerra &
que si en que es pan se equivoca, &
aunque todo un Cielo toca, &
no toca en Cielo, ni en tierra, &
toca misterio, y si encierra &
portentos no oídos, &
\taglineES \&
\end{original}

\begin{translation}
\StanzaSection{6}
If the senses make &
a complaint about the Divine Bread, &
because what they sense &
is not by faith consented, &
today let them all with faith be heard. &
Let the senses not resent it.
\SectionBreak


\StanzaSection{7}
1. If in them the eyes that admire &
cannot see well, &
since they shall see poorly what they see &
if they do not look at what they see, &
if their blindness is what &
keeps them impaired, &
\taglineEN \&

\Stanza{7}
2. Within transparent veils, &
God Incarnate is not seen, &
for the color has been changed, &
and its accidents are doing it. &
If in the clouds flashing rays &
are hidden, &
\taglineEN \&

\Stanza{7}
3. Touch touches but it errs, &
for if in what is bread it is mistaken, &
even though it touches all of Heaven, &
it touches neither Heaven nor earth, &
it touches a mystery, and if it encloses &
unheard portents, &
\taglineEN \&
\end{translation}
\end{poemtranslation}
\end{document}
