\documentclass{aac-poem}
\begin{document}

% Padilla, Oyeme, Toribio, 1651
\begin{poemtranslation}
\begin{original}
\StanzaSection{15}[\add{Introducción} Dúo]
1. Óyeme, Toribio. &
2. ¿Hablas me, chamorro? &
1. Gloria es todo el valle. &
2. ¿E? ¿E? que no te oigo. &
1. Ya es la tierra cielo, &
y hasta él, llanto es gozo. &
2. No oigo de ese oído. &
1. Pondréme desotro. &
2. Desotro oigo menos. & 
1. Tú eres lindo tonto, &
yo más que te escucho\dots{} &
2. ¿Si tengo bochorno? &
¿Qué es lo que me dices? & 
1. \dots{} que me vuelves loco. &
2. \add{Text missing} 
\SectionBreak

\StanzaSection{5}[\add{Estribillo} solo, Responsión a 5] 
De la aurora la risa &
serán sollozos &
si ven sus ojos, &
al nacer la palabra, &
los hombres sordos. \&
\end{original}

\begin{translation}
\StanzaSection{15}
1. Listen to me, Toribio. &
2. Are you talking to me, baldy? &
1. Glory is in all the valley. &
2. Eh? Eh? I can't hear you. &
1. Behold, the earth has become heaven, &
and toward heaven, the only crying is for joy. &
2. I can't hear from that ear. &
1. I'll try the other one. &
2. From the other one I can hear even less. &
1. You are a sheer idiot! &
the more I listen to you\dots{} &
2. Am I embarrassed? &
What is it you are telling me? &
1. \dots{} you are driving me crazy! &
2. \add{Text missing} \&

\StanzaSection{5}
The laughter of the dawn &
will become sobs &
if her eyes see, &
upon the birth of the Word, &
deaf men. \&
\end{translation}
\end{poemtranslation}
\end{document}
