% Silencio, atencion, que canta Maria by Sor Juana
\begin{poemtranslation}
    \begin{original}
        \StanzaSection{6}[Estribillo]
        ¡Silencio, atención, &
        que canta María! &
        Escuchen, atiendan, &
        que a su voz Divina, &
        los vientos se paran &
        y el Cielo se inclina.
        \SectionBreak

        \StanzaSection{4}[Coplas]
        1. Hoy la Maestra Divina, &
        de la Capilla Suprema &
        hace ostentación lucida &
        de su sin igual destreza: \&

        \Stanza{5}
        2. Desde el \term{ut} del \term{Ecce ancilla}, &
        por ser el más \term{bajo} empieza, &
        y subiendo más que el \term{Sol} &
        al \term{la} de \term{Exaltata} llega. &
        \Dots \&

        \Stanza{5}
        4. \term{Be-fa-be-mí}, que juntando &
        diversas Naturalezas, &
        unió el \term{mi} de la Divina &
        al \term{bajo fa} de la nuestra. &
        \Dots \&
    \end{original}
    \begin{translation}
        \StanzaSection{6}
        Silence, attention, &
        for Mary is singing! &
        Listen, attend, &
        for at her divine voice &
        the winds cease &
        and Heaven inclines \add{to hear}.
        \SectionBreak

        \StanzaSection{4}
        Today, she, the divine master &
        of the Supreme Chapel &
        makes a brilliant demonstration &
        of her unequalled virtuosity: \&

        \Stanza{5}
        From the \term{ut} of \quoted{Behold the handmaid}, &
        since it is the lowliest, she begins, &
        and rising more than the sun/\term{sol} &
        she arrives at the state/\term{la} of \term{\add{She is} exalted}. &
        \Dots{} \&

        \Stanza{5}
        B (\term{fa})/B (\term{mi}), since, joining &
        distinct natures, &
        \add{God} united the \quoted{me}/\term{mi} of the divine nature &
        to the \quoted{low \term{fa}} of our nature. &
        \Dots{} \&
    \end{translation}
\end{poemtranslation}
\endinput
