% Gutiérrez de Padilla Miraba el sol, second example for chapter 1
% 2017/07/17, revised 2019/07
\documentclass{tex/vcbook-float}
\usepackage{xparse}
\usepackage{graphicx}
\usepackage[normalem]{ulem}
\setlength{\parindent}{0pt}

\newlength{\solfalevel}
\setlength{\solfalevel}{6ex}
\newlength{\minim}
\setlength{\minim}{2em}

% #1 optional star for when syllable matches lyric text,
% #2 length in minims, 
% #3 solmization syllables
\NewDocumentCommand{\solfa}{ s m m }{%
    \makebox[#2\minim][l]{%
        \IfBooleanTF{#1}{\emph{#3}}{#3}%
    }\ignorespaces
}

% #1 hexachord name, typeset below rule and \BODY (solfa syllables)
\usepackage{environ}
\NewEnviron{hexachord}[1]{%
    \leavevmode\rlap{\textsc{#1}}%
    {\raisebox{\baselineskip}{\uline{\BODY}}}%
}

% Overlap next hexachord and previous by specified number of minims
\NewDocumentCommand{\mutate}{ O{1} }{%
    \kern-#1\minim\ignorespaces
}

% Shift argument up specified number of \solfalevel
\NewEnviron{levelup}[1]{%
    \raisebox{#1\solfalevel}{\BODY}\ignorespaces
}

\begin{document}
\begin{textwidthbox}
    \hspace{2\minim}
    \noindent\begin{hexachord}{mollis}
        \solfa{1}{la}
        \solfa{2}{re}
        \solfa*{1}{sol}
        \solfa*{2}{mi}
        \solfa*{3.3}{re}
        \solfa{1}{re}
    \end{hexachord}
    \mutate
    \begin{levelup}{1}
        \begin{hexachord}{ficta}
            \solfa{1}{fa}
            \solfa{2}{mi}
            \solfa{1}{fa}
        \end{hexachord}
    \end{levelup}
    \mutate
    \begin{levelup}{2}
        \begin{hexachord}{naturalis}
            \solfa*{1}{sol}
            \solfa*{1.5}{mi}
            \solfa*{1}{re}
        \end{hexachord}
    \end{levelup}
    \mutate
    \begin{hexachord}{mollis}
        \solfa{1}{la}
        \solfa{2}{re}
        \solfa*{3}{sol}
    \end{hexachord}
\end{textwidthbox}

\includegraphics[width=\textwidth]{build/music-examples/Padilla-Miraba_el_sol-hexachords-music}
\end{document}
