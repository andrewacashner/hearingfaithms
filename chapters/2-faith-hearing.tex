% Andrew Cashner
%
% Faith, Hearing, and the Power of Music 
% in Devotional Music of the Spanish Empire
%
% Chapter 2: Making Faith Appeal to Hearing
%
% 2016-12-29    Compromise new book draft based on USCB with some
%                   additions from 9/16 version and new material (50pp)
% 2016-10-13    Abridged version for USCB presentation (12pp)
% 2016-09-16    First draft based on diss., expanded (60pp)
% 2016-08-31    Revision for book begun
% 2016-03-18    Dissertation ch. 2 defended
% *******************************************

\chapter{Making Faith Appeal to Hearing}

\label{ch:faith-hearing}

\epigraph
{I cannot comprehend you nor know your enigma,\\
because I have listened to Faith without faith.}
{\quoted{Judaism}, in Calderón, \worktitle{El nuevo palacio del Retiro}}

St. Paul wrote to the Christian community in Rome, \quoted{How are they to believe if they have not heard?}, since \quoted{faith comes through hearing, and hearing, by the Word of Christ} (\scripture{Rom 10:16-17}).
Seventeen centuries later, amid the ongoing reformations of the Western Church, Catholic Christians were seeking ever new ways to make faith audible.
Poets and composers of the Spanish Empire expanded a genre of sung poetry in the vernacular---the villancico---into large-scale choral and instrumental performances that could appeal to the ears of elite and common people alike.%
\begin{Footnote}
    The major studies of the villancico as a musical and poetic genre are, in chronological order, 
    \autocites{Rubio:Forma}{Laird:VC}{Torrente:PhD}{Tenorio:SorJuana}
    {CaberoPueyo:PhD}{Illari:Polychoral}{Knighton-Torrente:VCs}
    {Cashner:Cards}{Cashner:PhD}.
%    \XXX[new caliope article]
\end{Footnote}

With the Church's active patronage, villancicos became a central activity in religious festivals throughout the year, particularly at Christmas, Corpus Christi, and the Immaculate Conception of Mary.
Church ensembles performed these pieces, with their motet-like refrains or \term{estribillos} surrounding a set of strophic verses or \term{coplas}, as an integral part of Matins and other liturgies.
Festival crowds from Madrid to Manila also heard villancicos in public processions and in conjunction with mystery plays.
Villancicos expressed widespread beliefs and attitudes while also shaping them.

A large number of villancicos begin with calls to listen---\foreign{escuchad}, \foreign{atended}, \foreign{silencio, atención}.
Because so many villancicos explicitly address concepts of music, sensation, and faith, these remarkable but understudied pieces offer us unique insights into how early modern Catholics understood the supernatural power of music.
How, then, did the people who created, performed, and listened to villancicos understand music's role in the relationship between hearing and faith?

This chapter explores the theological climate in which villancicos were created and heard, considering theological literature on faith and hearing, and villancicos that treat these themes explicitly.
I argue that villancicos provided a way for church leaders to make faith appeal to the hearing of a broad range of listeners, while also creating opportunities for listeners to contemplate the nature of faithful hearing.
Villancicos thus fulfilled one of the central prerogatives of the Catholic teaching after Trent, to accommodate the sense of hearing even while training it.

At the same time, I argue, villancicos manifest certain widespread anxieties about the role of sensory experience in faith. 
Spaniards and their colonial subjects worried about how to listen faithfully, about the role of subjective experience and cultural conditioning, and about the possibility that some listeners might lack the capacity to hear music rightly.
Understanding villancicos in their theological context, and interpreting villancicos as theological sources in themselves, reveals varied and sometimes conflicting beliefs about faith and hearing.
The discourse on faith and hearing in devotional poetry and music often raises as many questions as it answers; but this tension, I argue, is actually central to its ability to both accommodate the senses and train them.

We will begin by discussing theological and literary sources that reveal tensions in theological formulations of faith, hearing, and music.
Then we will look at a group of related villancicos, never previously edited or studied, that manifest the same tensions.
The first two pieces stage allegorical contests of the senses in which hearing is the favored sense of faith; other pieces deliberately confuse the senses to point to a higher truth that is beyond sensation.
The last two pieces are \soCalled{villancicos of the deaf}: they represent characters whose impairments of hearing render them unable to understand religious teaching.

If we listen closely to villancicos as historical sources for theological understanding, it becomes clear that villancicos functioned as much more than vehicles for simplistic religious teaching or as banal, worldly entertainment.
The villancicos in this chapter appeal to the ear in order to give listeners an opportunity to contemplate the challenges of faithful hearing.

% ***********
\section{Accommodating and Training the Ear}

Catholic theologians of the Tridentine era emphasized the need for the Church's teachers to make faith appeal to hearing, and the Church used devotional music like villancicos in that endeavor.
Catholic writers taught that since faith came through hearing, as St.\ Paul said, ministers must accommodate their teaching to the sense of hearing of their parishioners.
They did not, though, view hearing alone as completely trustworthy.
Even if faith came through hearing, one could not believe everything one heard; and thus Catholics emphasized not only accommodating hearing, but training it.

Though the Protestant reformers insisted that their teachings were a return to true biblical orthodoxy, Catholic theologians believed that the reformers had redefined faith completely.
In contrast, Catholics continued to develop theology of faith they inherited from medieval writers like Thomas Aquinas, in which faith was one of three virtues or capacities, along with hope and charity.%
    \Autocite[130--132]{Schreiner:Certainty}
Simple belief in intellectual propositions was \quoted{unformed faith} (\term{fides informe}); it was essential that every Christian believe certain things, but this was not the summit of Christian life.
The goal was that Christians would develop fully \quoted{formed faith} (\term{fides formata}), which \quoted{worked through} the two higher virtues to result in what might best be translated as \quoted{faithfulness}.
True faith was a conviction that manifested itself through ethical behavior in fidelity to God's will.%
    \Autocite[\sv{Credo in Deum Patrem}, 15--20]{Catholic:Catechismus1614}

For Catholics, then, faith encompassed beliefs, attitudes, and behaviors; and it connected individuals experience with ethical action as part of the community of the Church.
A man of faith was a man of virtue (the root of \term{virtus} being \term{vir}), and thus faith was central to the Catholic Humanist goal of building a virtuous society.
Christian disciples had to not only \quoted{hear} the Church's teaching and believe it (this would be unformed faith); they also had to \quoted{listen} in the sense of obeying.

%************************************
\subsection{Hearing Christ the Word}

A central document for understanding the faith of post-Reformation Catholics is the catechism \quoted{for parish priests} commissioned by the Council of Trent and first published in 1566.%
    \Autocites{Catholic:Catechismus1614}[\sv{catechism}]{NewCatholic}
In responding to both the challenge of Protestantism and the underlying problems that had allowed heresy to take root, the council's bishops sought to improve the education of clergy and laity in matters of faith.
They required priests and bishops to preach on all Sundays and holy days.%
    \Autocite[\sv{Trent, Council of}]{NewCatholic}

The catechism was a model for teaching the clergy how to preach and teach---a guide both to the content of Catholic faith and to the best ways of making the faith heard and understood.
Vernacular expositions of the catechism, like the Spanish versions of Antonio de Azevedo, Juan Eusebio de Nieremberg, and Juan de Palafox y Mendoza, brought this teaching down to a more colloquial level for (in Palafox's words) \quoted{laborers and simple folk}, with earthy illustrations and paraphrases of Biblical stories.%
    \Autocites{Azevedo:Catecismo}{Nieremberg:PracticaCatecismo}{Palafox:Bocados}
These texts come alive when read aloud, and indeed the primary goal of this type of literature was to prepare pastors to teach unlettered disciples through words and voice, whether those disciples were the \soCalled{Indians} of America or Asia, or the peasants of Europe whose customs were still more pagan than Christian.%
    \Autocites
    [On Europe as a mission front after Trent, see]
    [60--63]{Kamen:EarlyModernSociety}

In the Roman Catechism, hearing is central to faith because the object of faith is Christ, the Word of God, and the Church's mission is to make that Word audible.
According to the catechism's preface, God communicated his own nature to humanity by taking on human flesh in Christ.
Therefore the true Word of God was not confined to Scripture or doctrines alone---the Word was a person, Jesus Christ, to whom the prophetic scriptures testified and from whom the Church's doctrine flowed.%
    \Autocite
    [9: \quotedla{Omnis autem doctrinæ ratio, quæ fidelibus tradenda sit, verbo Dei continetur, quod in scripturas, traditionesque distributum est}.]
    {Catholic:Catechismus1614}
In the words of John's gospel, Jesus Christ himself was the \term{logos} or \term{verbum} (\scripture{Jn 1:1}), the \quoted{Word made flesh}.
Where God had spoken through the law and the prophets, now he had spoken through his Son (\scripture{Heb 1:1}).

The catechism builds its ecclesiology, or its theology of the church, on this theology of the Incarnation.
Christ, the incarnate Word, appointed apostles, chief among them St.\ Peter, to be the custodians of the true faith.
The community of the Church, then, was the sacramental means through which people would come to know Christ after his resurrection.

There was thus a close link in Catholic theology between hearing the Word of the Church's teaching, obeying that Word, and encountering Christ who was the Word.
The absolute center of the Church's teaching, according to the catechism, was found in Christ's own words in \scripture{Jn 17:3}: \quoted{This is eternal life, that you should know the only true God, and the one whom he sent, Jesus Christ}.%
   \Autocite
   [6: \quotedla{Hæc est vita æterna, vt cognoscant te solùm verum Deum, \& quem misisti, Iesum Christum}.]
   {Catholic:Catechismus1614}
All the Church's teachers should exert themselves to the end that 
    \begin{quote}
        the faithful should know and love from the heart Jesus Christ, and him crucified \add{\scripture{ICor 2:2}}; and indeed to persuade them so that they believe with the faithfulness \add{\foreign{pietas}} of their inmost heart and with cultivated devotion \add{\foreign{religio}}, that there is no other name given to people under heaven, through whom they can be saved \add{\scripture{Acts 4:12}}, since Christ is the atonement for our sins \add{\scripture{IJn 2:2}}.%
            \Autocite
            [6: \quotedla{Quamobrem in eo præcipuè Ecclesiastici doctoris opera versabitur, vt fideles scire ex animo cupiant, Iesum Christum \& hunc crucifixum: sibíque certò persuadeant, atque intima cordis pietate, \& religione credant, aliud nomen non esse datum hominibus sub cœlo, in quo oporteat nos saluos fieri: siquidem ipse propitiatio est pro peccatis nostris}.]
            {Catholic:Catechismus1614}
    \end{quote}
Finally---contrary to Protestant teaching as Catholics understood it---since \quoted{in this we know that we know him, if he keep his commandments} (\scripture{IJn 2:3}), teachers should model faithful living, \quoted{not in leisure} but \quoted{in applying diligent effort to justice, piety, faith, charity, mercy}, having been redeemed \quoted{to do good works} (\scripture{ITim 2:12}); so that \quoted{whether one sets out to believe or hope or do anything}, the love of God should be the summit of all Christian life.%
    \Autocite[6--7: \quotedla{At verò quia in hoc scimus, quoniam cognouimus eum, si mandata eius obseruemus, proximum est, \& cum eo, quod diximus, maximè coniunctum, vt simul etiam ostendat vitam à fidelibus non in otio, \& desidia degendam esse, verùm oportere, \Dots{} sectemúrque omni studio iustitiam, pietatem, fidem, charitatem, manusetudinem: dedit enim semetipsum pro nobis, vt nos redimeret ab omni iniquitate, \& mundaret sibi populum acceptabilem, sectatorem bonorum operum \Dots{}. Siue enim credendum, siue sperandum, siue agendum aliquid proponatur, ita in eo semper charitas Domini nostri commendari debet \Dots.}]
    {Catholic:Catechismus1614}

How, then, could the Church communicate the Word through the sense of hearing?
Drawing on St.\ Paul's dictum from Romans, the catechism challenges the Church's ministers to find a way to make Christ the Word audible:
\begin{quote}
    Since, therefore, faith is conceived by means of hearing, it is apparent, how necessary it is for achieving eternal life to follow the works of the legitimate teachers and ministers of the faith. \Dots{}
    Those who are called to this ministry should understand that in passing along the mysteries of faith and the precepts of life, \emph{they must accommodate the teaching to the sense of hearing and intelligence}, so that by these \add{mysteries and precepts}, \emph{those whose senses have been trained} should be filled up by spiritual food.%
        \Autocite
        [2, 8--9 (emphasis added):
        \quotedla{Cvm autem fides ex auditu concipiatur, perspicuum est, 
        quàm necessaria semper fuerit ad æternam vitam consequendam 
        doctoris legitimi fidelis opera, 
        ac ministerium \Dots\ vt videlicet intelligerent, 
        que ad hoc ministerium vocati sunt, 
        ita in tradendis fidei mysteriis, ac vitæ præceptis,
        doctrinam ad audientium sensum, atque intelligentiam accommodari oportere,
        vt cùm eorum animos, qui exercitatos sensus habent, 
        spirituali cibo expleuerint \Dots}.]
        {Catholic:Catechismus1614}
\end{quote}
Pastors should consider \quoted{the age, intelligence \add{\foreign{ingenium}}, customs, condition} of their charges, to give milk to spiritual infants and solid food to the maturing, to raise up a \quoted{perfect man \add{\foreign{virum perfectum}}, after the measure of the fullness of Christ}.%
    \Autocite
    [8: \quotedla{Obseruanda est enim audientium ætas, ingenium, mores, conditio \Dots. sed cùm alij veluti modò geniti infantes sint, alij in Christo adolescere incipiant, nonnulli verò quodammodo confirmata sint ætate, necesse est diligenter considerare, quibus lacte, quibus solidiore cibo opus sit, ac singulis ea doctrinæ alimenta præbere, quæ spiritum augeant, donec occurramus omnes in vnitatem fidei, \& agnitiones filij Dei, in virum perfectum, in mensuram ætatis plenitudinis Christi}.]
    {Catholic:Catechismus1614}

Thus the catechism teaches that for faith to come through hearing, both the teacher and the listener had to be involved.
This concept was echoed in 1589 Spanish \worktitle{Catechism of the Mysteries of the Faith} by Antonio de Azevedo.%
  \Autocite{Azevedo:Catecismo}
This catechism begins with an image taken from Pliny, which depicts a religious teacher with a pupil at his feet and a harp in his hand.
The teacher is shown, Azevedo says,
\begin{quote}
    with a musical instrument which gives pleasure to the ear, so that we should understand that Faith enters through the ear \add{\foreign{oído}}, as St.\ Paul says, and that the disciple should be like a child, simple, without malice or duplicity, without knowing even how to respond or argue, but only how to listen and learn.
    Thus this image depicts for us elegantly, what the hearer of the Faith should be like.%
        \Autocite
        [f.~1b: 
        \quotedsp{Para pintar los Romanos la Fe, lo primero que hizieron templo y altar \Dots{} Numa pompilio \Dots{} puso vn idolo:
        de forma de vn viejo cano, que tenia vna harpa en la mano, i estava enseñando vn niño echado a sus pies.
        En esta figura o geroglifica esta encerrada mucha filosofia, i aun cristiana.
        En el templo i ara denota, que la fe a de ser firme i fixa, no movediça, ni flaca, que a cada ayre de novedad se mueva
        I tambien que a menester maestros que la enseñen, i dicipulos que la oygan.
        I, que el maestro a de ser anciano, i maduro en edad y bondad.
        Porque la dotrina es grave, antigua, i de tomo i sustancia. 
        No nueua, ni de pocos años sino antigue dende los Apostoles, i con instrumento musico que da gusto al oido.
        Paraque entendamos, que la Fe entra por el oido; como dize S.\ Pablo.
        I que el dicipulo sea como niño, sencillo, sin malicia ni doblez, sin saber ni replicar, ni arguir, mas de solo oir, y deprender.
        En lo qual nos dibuxa galanamente, qual a de ser el oiente de la Fe.}]
        {Azevedo:Catecismo}
\end{quote}

The teacher's task according to Azevedo, then, is not only to make the faith heard, but to make it \quoted{appeal to the ear}, just as he says music does.
But such teaching was limited by the sensory capacity of each listener, and therefore the Roman Catechism argues that the task of listening requires the sense to be properly trained.
Thus pastors had to accommodate the ear even while training it.

How could the Church use music, then, in the effort to make Christ the Word audible?
The catechism's theology might suggest that music would contribute to propagating faith because this medium could appeal in a special way to the sense of hearing.
If teaching should appeal to the ear \emph{like} music, as Azevedo says, then combining teaching with actual music would appeal all the more.
Villancicos, as an auditory medium based on poetry in the vernacular, would seem like an ideal tool for accommodating the faith to Spanish-speaking listeners' \quoted{sense and intelligence}.
At the same time, the challenge of training their sense of hearing would seem to be multiplied with music, since the listeners must learn to understand not only spoken language but musical structures as well.
In other words, music would seem to be a way of better accommodating faith to the sense of hearing, but it also required more training to be heard in a faithful way.


%****************
\subsection{Experiencing Spiritual Truth through Music}

This tension between accommodating the ear and training it may be seen in the explanations of music's power by Athanasius Kircher.
This Jesuit polymath's compendium of musical knowledge, \worktitle{Musurgia universalis} was published in Rome in 1650 and thence disseminated throughout the Hispanic world, with copies sent to centers of colonial culture including Puebla and Manila.%
\begin{Footnote}
    On the worldwide distribution of the book as far as Manila, see \autocite[48--50]{Irving:Colonial}.
    The book may be found in historical collections in Madrid, Barcelona, Mexico City, and Puebla (two copies).
    On Kircher, see \autocites{Findlen:Kircher}{Godwin:KircherTheater}.
\end{Footnote}
Kircher discusses the power of music several times throughout his ten-volume treatise, including a detailed analysis of \quoted{whether, why, and what kind of power music might have to move people's souls, and whether the stories are true that were written about the miraculous effects of ancient music}.%
    \Autocite
    [bk.~VII, 549: \quotedla{Vtrum, cur, \& quomodo Musica uim habeat ad animos hominum commouendos, \& vtrum vera sint, quæ de mirificis Musicæ Veteris effectibus scribuntur}.]
    {Kircher:Musurgia}
Kircher's contribution to this favorite controversy of the Renaissance is to defend the superiority of modern music on the basis of, among other factors, its increased ability to move listeners through varieties of musical structure and style.

For Kircher, music added such power to words, that it could move a listener not only to understand the subject of the words, but to physically experience their truth.
According to legend the famed \term{aulos} player Timotheus aroused Alexander the Great to the furor of war through music, and Kircher said he did this by adapting his song both to the feeling of war and to the disposition of the king.%
    \begin{Footnote}
        Kircher is probably responding to Vincenzo Galilei, \worktitle{Dialogo della musica} (Florence, 1581), 90, in discussing the Classical source, Dio Chrysostom, \worktitle{Orationes} 1 (\textgreek{Περὶ Βασιλείας});  a later treatment of this subject is John Dryden, \worktitle{Alexander's Feast, or the Power of Music} (London, 1697).
    \end{Footnote}
The same music, he says, would have had a different effect on someone else; and to illustrate this contrast, Kircher goes on to paint a remarkable picture of how sacred music can move those who are disposed to it:
\begin{quote}
    If, on the other hand, \add{the musician} addressed the sort of man who was devoted to God and dedicated to meditation on heavenly things,
    and wished to move him in otherworldly affects and rapture of the mind,
    he would take up some notable theme expressed in words---a theme that would recall to the listener's memory the sweetness and mildness of heavenly things---
    and he would fittingly adapt it in the Dorian mode through cadences and intervals,
    then \add{the listener} would experience that what was said was actually true,
    those heavenly things that were made by harmonic sweetness,
    and he would suddenly be carried away beyond himself to that place where those joyful things are true.%
        \Autocite
        [bk.~7, 550:
        \quotedla{Sicuti contra, si quis Deo deuotum hominum rerumque cœlestium, meditationi deditum in exoticos affectus raptusque mentis commouere vellet is supra insigne aliquod verborum thema, quod rerum cælestium dulcedinem, \& suauitatem auditori in memoriam reuocaret, modulo dorio per clausulas interuallaque aptè adaptet, \& experietur quod dixi verum esse, statim extra se factos dulcedine harmonica eò, vbi vera sunt gaudi rapi: vidi ego nonsemel in viris ordinis nostri sanctitate illustribus huiusmodi experimenta}.]
        {Kircher:Musurgia}
\end{quote}

Kircher says that the experience of Jesuit missionaries around the world provides ample evidence of this miraculous power of music combined with preaching.
But Kircher's depiction of music's power goes well beyond the Jesuit formula of \quoted{teaching, pleasing, and persuading}.%
\begin{Footnote}
    On the Jesuit approach to religious arts, see \autocite[35--51]{Bailey:Art}.
\end{Footnote}
Music not only makes the teaching of doctrinal truth appealing and persuasive; it actually transforms listeners through affective experience.
In this conception, music links the objective truth with subjective experience through the unique ways that music affects the human body.
Through principles of sympathetic vibration, the humoral-affective properties of music could be transferred from composer and performer to listeners.

This passage in Kircher's Book VII comes in the midst of a comparison between the fabled powers of music in the ancient world versus the possibilities for moving the affections in contemporary music.
Kircher argues that music moves a listener's soul through sympathetic resonance that creates harmony between the numerical proportions of the music and the humoral composition of the listener.
The various modes and styles of modern music, which Kircher catalogues in detail, apparently have inherent affective properties: the Dorian mode, for example, is suited to grave and pious religious sentiment.
For Kircher, modern music was superior to ancient music not because it operated according to different principles, but because modern musicians had found ways to extend and amplify the same principles that made ancient music effective.

The fundamental principle that gave music its affective power was sympathy.%
\begin{Footnote}
    On the links between this interest in the occult powers of music and early scientific research, see \autocites{Gouk:Sciences}{Gouk:Harmonics}.
\end{Footnote}
Music created physical harmony---sympathetic vibration---between the affective subject of the music (that is, what the music was meant to express), and the listening subject, so that the music moved in the same way as the affections it expressed, and this in turn moved the listener to the same affections.
Music with different intervallic relationships, with the semitone placed differently in different modes, produced different effects.

As Kircher explains, 
\begin{quote}
    Since harmony is nothing other than the concord, agreement, and mutually corresponding proportion between dissimilar voices,
    this proportion, then, of numbers, sets the air in motion;
    the motion, indeed, is to be varied by the ratios of various intervals, ascending and descending;
    so that the spirits \add{i.e., \term{spiritus animales}}, or the implanted internal air (as I have just shown), should be moved according to the proportions of the motion of the external air, so that the spirits' motion are effected in various ways; and through this affections can be engendered in the person.%
        \Autocite
        [552: \quotedla{Cum harmonia nihil aliud sit, quam dissimilium vocum concordia, consensus, \& undequaque correspondens proportio; proportio autem numerorum in motu aeris elucescat; motus verò pro varia interuallorum, ascensus, descensusque ratione varius sit, spiritus quoque, siue aer internus implantatus, vti paulò ante ostensum fuit, iuxta proportionem motus aeris extrinseci moueatur, fit vt spiritus moti ope vari\c{e}, indè in homini affectiones nascantur}.]
        {Kircher:Musurgia}
\end{quote}

The structure of the music's movements must correspond to the movements of the body's humors.
Kircher theorizes four conditions that are necessary for music to achieve an effect; without any one, music will fail to move the listener in the intended way:
\begin{quote}
  The first is harmony itself.
  Second, number, and proportion.
  Third, the power and efficacy of the words to be pronounced in music itself; or, the oration.
  Fourth is the disposition of the hearers, or the subject's capacity to remember things.%
    \Autocite
    [550: \quotedla{Tertius purè naturalis est, per harmonicum, scilicet sonum, qui nisi quatuor conditiones annexas habeat, quarum vna deficiente, desideratus effectus minimé obtinebitur: Prima est ipsa harmonia. Secunda, numerus, \& proportio. Tertia, verborum in ipsa musica prouniciandorum vis, \& efficacia, siue ipsa oratio. Quarta audientis dispositio, siue subiectum memoratarum rerum capax}.]
    {Kircher:Musurgia}
\end{quote}

If there must be this kind of congruence between music and listener, then it makes sense that Kircher acknowledges that music affects different people in idfferent ways.
First, Kircher concedes that geographic and cultural factors influence music style and its effect, such that Italians and Germans are moved by different styles and therefore compose differently.
These national styles, he says, are the result of a national \quoted{genius} (that is, the special gift of that people), as well as environmental factors, such that Germans draw a grave style from living in a cold climate, contrasted with the more moderate style of Italians.
People of the Orient who visit Rome, Kircher says, do not enjoy the highly delicate music of that city, and prefer their own strident, clangorous music.
These differences of style and perception are cause by the patriotism---the inordinate love of things from one's own country, as Kircher describes it; and by what each person is accustomed to hearing, which is shaped by the traditions of each country.%
    \Autocite[543--544]{Kircher:Musurgia} % XXX translation in Strunk

Moreover, \quoted{just as different nations enjoy a different style of music, likewise within each nation, people of different temperaments appreciate different styles that conform the most to their natural inclinations}.%
    \Autocite
    [544: \quotedla{Quod quemadmodum diversæ nationes diuerso stylo musico gaudent, ita \& in vnaquaque natione diuersi temperamenti homines, diuersis stylis, vnusquisque suæ naturali inclinationi maximè conformibus afficiuntur}.]
    {Kircher:Musurgia}
What delights a person with a sanguine temperament might enrage or madden a melancholic listener; what has a strong effect on one person may have no effect on another.%
    \Autocite[550]{Kircher:Musurgia}
\quoted{Music does not just move any subject, but that with which the natural humor of the music is congruent \Dots{} for unless the spirits of the recieving subject correspond exactly, the music accomplishes nothing}.%
    \Autocite[550: \quotedla{Musica igitur vt moueat, non qualecunque subiectum vult, sed illud cuius humor naturalis musicae congruit \Dots{} quæ nisi recipientis subiecti spiritui extactè respondeant, nihil efficient}.]
    {Kircher:Musurgia}
By including the capacity of memory in his list of four conditions for effective music, Kircher suggests that not only humoral temperament but also training and intelligence are a factor in individual listening.

Despite Kircher's confidence in modern musicians' ability to make music music move people, the conditions he names may not be as easy to fulfill as he suggests.
There must be congruence, first of all, between the structure of the music and subject of that music: the music must move in the same way as the affective movements it seeks to incite.
Harmonic ratios, metrical proportions, verbal rhetoric---all of these must align, but they are still not enough without the fourth condition, the disposition of the hearer.
The listener must have a humoral temperament that is moved in the desired way by the music. 

In other words, the tension between acccommodating hearing and training it is multiplied vastly by the addition of music.
While it might seem that music would allow for greater accommodation, the number of potential obstacles is increased because content of the music, the performance, and the listeners must all be in harmony.
When Kircher compares music to preaching, he says that an effective preacher is familiar with his audience and therefore \quoted{knows which strings to pluck}---a phrase that recalls Antonio de Azevedo's image of the religious teacher with harp in hand.%
    \Autocite
    [551: \quotedla{Nouerat enim prædicator fuorum auditorum inclinationem; nouerat chordam, quam tangere debebat}.]
    {Kircher:Musurgia}
But Kircher never fully resolves the tension between the universal power of music and the variables of individual subjectivity and cultural conditioning.

Music could, as Kircher describes, move someone to experience the truth of religious teaching through affective experience, but only if the listener was \quoted{the sort of man who was devoted to God}---that is, someone who already had faith, whose temperament was already disposed to religious devotion of this kind. 
If, as Kircher acknowledges, people of different nations are moved by different kinds of music, and if individual people respond differently depending on their temperament as well as their intellectual capacity, how could the creators of sacred music be assured of its power?

Kircher's theory represents one attempt to reconcile the challenges of accommodating the ear and training it. 
But outside the erudite realm of theoretical speculation, this was a problem that faced every Catholic church leader and musician who was serious about using music to make faith appeal to hearing.
How could the Church use music to accommodate hearing, when individuals and communities did not hear the same way? 
The capacity to listen faithfully, and therefore music's power to make faith appeal to hearing, would be limited by both personal subjectivity and cultural conditioning, and these limitations created anxiety and fear about the role of hearing in faith.

% ****************
\section{Danger and Doubt}

\subsection{Subjective Experience in Faith}

Regarding the problem of individual hearing, Reformation controversies had pushed Catholics into an increasingly negative and anxious attitude toward subjective religious experience.
Polemicists like Thomas More accused Martin Luther of turning his followers away from the trustworthy institutional Church with its objectively operating sacraments, leaving them with only a subjective experience as assurance of salvation, and separating individual faith from the cultivation of a just society.%
    \Autocite[ch.~4]{Schreiner:Certainty}
But for Catholics the problem remained that some connection needed to be forged between individual people and the transcendent object of faith, namely God in Christ, and no way of defining faith could completely remove the individual element.
The Roman Catechism says that faith means more than having an opinion or conception of something---it \quoted{has the strength of the most certain agreement, such that the mind, having been opened by God to his mysteries, firmly and steadfastly gives assent}.%
    \Autocite
    [15: \quotedla{Igitvr credendi vox, hoc loco putare, existimare, opinari, non significat, sed vt docent sacræ litteræ, certissimæ assensionis vim habet, qua mens Deo sua mysteria aperienti, firmè constantérq; assentitur}.]
    {Catholic:Catechismus1614}
This theology of faith begins with God's grace opening the mind of the passive person, but ends with an act of the individual will.

The catechism preaches, in the tradition of Augustinian Neoplatonism, that God the creator is above all created things that can be perceived through the senses, and that therefore humans cannot come to the knowledge of God without the gift of God's grace.
\quoted{In order for our minds to reach God, since nothing is more sublime than God, our mind needs to be pulled away from everything that pertains to the senses---something that we, in this natural life, do not have the capacity to do}.%
    \Autocite
    [18: \quotedla{Nam vt mens nostra ad Deum, quo nihil est sublimius, perueniat, necesse est eam omnino à sensibus abstractam esse: cuius rei facultatem in hac vita naturaliter non habemus}.]
    {Catholic:Catechismus1614}
But even as the catechism stresses that sensory experience is inadequate, and human beings are incapable of knowing God by their own lights, it still speaks of the knowledge of God in subjective terms: \quoted{our minds} must be changed, our relationship to the world of the senses must be changed by divine action.
Though God is beyond sensation, sensation is the entrance to the path; the created world speaks of God's nature to those who can perceive it.
Each person is still called to a process of seeking to know God, seeking to move beyond sensory experience.

As Susan Schreiner argues, the Catholic response to early modern debates about the certainty of faith and salvation was to emphasize the work of the Holy Spirit through the institutional Church.%
    \Autocite[131--208]{Schreiner:Certainty}
Not every individual perception or experience was valid, but only those that accorded with the revelation already given through the Church.

As a result, Spanish theological writers cultivated disciplines for regulating spiritual experiences and submitting individual sensation to the Church's authority.
The Spanish Inquisition investigated Teresa of Ávila and other mystics who claimed authority only on the basis of spiritual experiences, and most were not as successful as Teresa in avoiding punishment.
    \Autocites{Ahlgren:TeresaPolitics}{Francisca:Inquisition}
Teresa's student Juan de la Cruz (St.\ John of the Cross) taught contemplatives to pursue union with God by weaning themselves of sensory experiences in the \quoted{dark night of the soul}.
The Carmelite reformer defines that union not in sensual terms but in ethical ones, as the total surrender and conformity of one's will to God.%
    \Autocite[bk.~I, ch.~5--7, 226--248]{JuandelaCruz:Subida}
Similarly, Ignatius of Loyola provided Jesuits with his \worktitle{Spiritual Exercises} to discern the validity of their religious sensations.%
    \Autocite[ch.~6]{Schreiner:Certainty}

In such a climate, music's power over the senses and affections might be used for the purposes of cultivating faith, but this power could also be dangerous.
Some Catholics such as the Jesuit missionaries, despite their founder's suspicion of music, were eager to use this power to advance the cause of the Church.
On the mission field, the Jesuits were involved with subjective experience to such a degree that they even interpreted the dreams of native people in New Spain.%
    \Autocite[40--41]{Bailey:Art}
But others saw the use of music as a potential distraction or distortion.
For example, Juan de la Cruz complains that even though churches would seem the ideal place for prayer, their decorations, ceremonies, and music so engage a person's senses that it can be impossible to worship God \quoted{in spirit and in truth} (\scripture{Jn 4:23-24}).%
    \Autocite[bk.~3, ch.~39--45, 415--424]{JuandelaCruz:Subida}
Desiring \quoted{to feel some effect on oneself} in doing elaborate ceremonies \quoted{is no less than to tempt God and provoke him gravely; so much so, that sometimes it gives license to the devil to deceive them, making them feel and understand things far removed from what is good for their soul}.% 
    \Autocite[bk.~3, ch.~43, 420:
    \quotedsp{Y lo que es peor es que algunos quieren sentire algún efecto en sí, o cumplirse lo que piden, o saber que se cumple el fin de aquellas sus oraciones ceremoniáticas; que no es menos que tentar a Dios y enojarle gravemente; tanto, que algunas veces da licencia al demonio para que los engañe, haciéndolos sentir y entender cosas harto ajenas del provecho de su alma, mereciéndolos ellos por la propiedad que llevan en sus oraciones, no deseando más que se haga lo que Dios quiere que lo que ellos pretenden.}]
    {JuandelaCruz:Subida}

In a strong counterpoint to Kircher's statement about music's power to augment preaching, Juan admonishes his readers to be wary of overly artful preaching which, like music, only serves to stimulate \quoted{the sense and understanding}---Juan uses the exact language of the Roman Catechism---but has no impact on the hearer's will to live faithfully:
\begin{quote}
    How commonly we see that \Dots{} if the preacher's life is better, greater is the fruit that he gains though his style be low and his rhetoric scanty, and his teaching common, because the living spirit infuses him with ardor;
    but the other preacher gets very little gain, no matter how much more elevated his style and doctrine may be: because, though it is true that good style and actions and elevated doctrine and good language move and create more effect when accompanied by a good spirit, without the spirit, though the sermon may give the sense and understanding much to savor and enjoy, it infuses little or no sustenance to the spirit, because commonly it remains as lax and loath as before to labor, even though marvelous things were said in marvelous ways, which only serve to delight the ear \add{\foreign{oído}}, like some polyphonic music \add{\foreign{una música concertada}} or the clanging of bells \Dots{}.
    It matters little to hear someone perform one kind of music that is better than some other, if it does not move me more than the other to do works; because, although they have spoken marvels, then they are forgotten, as they do not infuse fire into the will.%
        \Autocite[bk.~3, ch.~45, 425:
        \quotedsp{Que comúnmente vemos que ---cuanto acá podemos juzgar--- cuanto el predicador es de mejor vida, mayor es el fruto que hace por bajo que sea su estilo, y poca su retórica, y su doctrina común, porque del espíritu vivo se pega el calor;
        pero el otro muy poco provecho hará, aunque más subido sea su estilo y doctrina;
        porque, aunque es verdad que el buen estilo y acciones y subida doctrina y buen lenguaje mueven y hacen más efecto acompañado de buen espíritu;
        pero sin él, aunque da sabor y gusto el sermón al sentido y al entendimiento, muy poco o nada de jugo pega a la voluntad,
        porque comúnmente se queda tan floja y remisa como antes para obrar, aunque haya dicho maravillosas cosas maravillosamente dichas, que sólo sirven para deleitar el oído, como una música concertada o sonido de campanas; mas el espíritu, como digo, no sale de sus quicios más que antes, no teniendo la voz virtud para resucitar al muerto de su sepultura.
        Poco importa oír una música mejor que otra sonar, si no me mueve ésta más que aquélla a hacer obras; porque, aunque hayan dicho maravillas, luego se olvidan, como no pegaron fuego en la voluntad.}]
        {JuandelaCruz:Subida}
\end{quote}

This passage of warning comes at the end of Juan's encyclopedic treatment of contemplative practice and mystical experience, and it seems to deliberately echo the \soCalled{love} chapter of I Corinthians in insisting that, as Juan says elsewhere, \quoted{a single act of charity is worth more than a thousand sweet words of consolation from beyond}.%
\quoted{a single work or act of the will done in charity is more precious before God than all the visions and revelations and communications from heaven that there can be}.
    \Autocite
    [bk.~II, ch.~22, p.~306: 
    \quotedsp{\Dots{} es más preciosa delante de Dios una obra o acto de voluntad hecho en caridad que cantas visiones y revelaciones y comunicaciones pueden tener del cielo}.]
    {JuandelaCruz:Subida}
The contrast between this passage and the quotation from Kircher reflects not only a chronological change in Catholic attitudes between the late sixteenth and mid seventeenth centuries, but also a contrast between religious orders, and differing emphases on parts of the Christian life.
Juan's \worktitle{Ascent of Mount Carmel} was first published in 1618, though it was written some forty years earlier; his outlook reflects the hard ascetic extreme of Catholic Reformation attitudes toward music.
Kircher, on the other hand, writes with the confidence of the Jesuit order at the height of its global influence, when this order was still trying to engage with any aspect of culture that they saw as advancing their mission.
Kircher is interested in music's power to convert---to not only persuade catechumens of the truth but to enable them to experience truth.
Juan, on the other hand, is concerned with the limiting effect of sonic pleasures on the maturity of the already converted.
Both acknowledge music's power over the individual person's senses, but Catholics who shared Juan's ascetic bent, or his pastoral concern with spiritual growth, saw that power as a danger or even an opening for diabolical influence.

% *********************
\subsection{Cultural Conditioning in Hearing}

If reckoning with individual sensory experience was a struggle for the Church after Trent, the problem of cultural conditioning in this period presented a newer and more difficult challenge that profoundly changed the character of Catholic Christianity.
Catholics since the early age of exploration had struggled to determine which parts of Christian religion they could adapt to other cultures.
As Ines Żupanov describes Jesuit intercultural encounters in India in her book \worktitle{Missionary Tropics}, the \quoted{tropics} represented both a geographic zone and a process of inevitable \quoted{turning} or cultural transformation.%
    \Autocite{Zupanov:MissionaryTropics}
Some missionaries like the Jesuits in Japan and Brazil actively sought to accommodate local customs and music; but everywhere that missionaries brought Christian faith, the process of cultural translation inevitably transformed it into something neither they nor their converts could necessarily predict.
    \Autocites{Bailey:Art}{Waterhouse:EarliestJapaneseContacts}
    {Castagna:JesuitsConversionBrazil}

A half century after the Aztec empire fell to the Spanish, the Franciscan Bernardino de Sahagún was trying to provide Mexica natives with suitable Christian songs in their own language.%
    \Autocite{Candelaria:Psalmodia}
But at the same time New-Spanish church leaders complained that there were often no linguists sufficiently skilled in Náhuatl to verify that the songs and dances of the natives were not secretly continuing to invoke demonic powers.%
    \Autocite[637]{Candelaria:Psalmodia}
Lorenzo Candelaria has made the provocative suggestion that the Council of Trent's infamous decree that music should be preserved from anything \soCalled{lascivious or impure} may have been motivated more by anxiety about the songs in Native American languages---and I would add, syncretic musical practices across the globe---rather than any concern about styles of European church music.%
    \Autocite[637--638]{Candelaria:Psalmodia} % XXX on Trent

We know in the realm of visual art that even where Jesuits wrote to their superiors in Rome praising the natives' uncanny ability to copy European models, the actual surviving artifacts often bear strong traces of indigenous methods, aesthetics, and religious understandings.%
    \Autocites[27--29, 34]{Bailey:Art}
Likewise, given what we know about differing techniques and styles of vocal production between places like central Africa and south India today, there is every reason to think that music in mission churches and colonial cathedrals, even plainchant and polyphony imported directly from Europe, sounded quite different in actual performance from its European models.

As the Church was adapting itself to native sensibilities, or being adapted by colonial subjects in ways the Church could not fully control, how was the church to accommodate its teaching to the \quoted{sense and intelligence} of all these different people?
The church could proclaim \soCalled{the Faith}, but how could leaders know that people heard what they intended?
And even more challenging, once these processes of cultural adaptation and exchange had begun to \quoted{turn}, what parts of Christianity constituted \quoted{the Faith} that was supposed to come through hearing?
Did changing the musical style of worship, for example, mean changing the Faith as well?

If ways of hearing music were culturally conditioned, then religious ear training was required.
The cultural aspect of acquiring a \quoted{properly trained sense}, as the Roman Catechism puts it, is plainly stated in a Latin dialogue published by the leaders of the Jesuit mission in Japan in 1590.%
    \Autocite{Sande:DeMissioneLegatorum}
The missionaries had taken four Japanese noble youths on a grand tour of Spain and Italy between 1582 and 1590.
The boys practiced and performed music throughout their trip, and heard the greatest ensembles of Catholic Europe in a tour that included most of the major Iberian musicl centers discussed in this study: on the outgoing trip, Lisbon, Évora, Toledo, Madrid, and Alcalá; and on the return, Barcelona, Montserrat, Zaragoza and Daroca.
With this trip and with the subsequent publication, the leader of the Jesuit mission to Japan, Alessandro Valignano, hoped to persuade the authorities of his order and church that \quoted{European Jesuits must accommodate themselves to Japanese manners and customs}.%
    \Autocite[4]{Massarella:JapaneseTravellers}
The boys' mission was both to represent Japan to Europe and, on their return, to represent Europe to Japan.

In Colloquio XI of the dialogue, Michael, on of the \soCalled{ambassadors}, tells his friend who stayed home about European music:
\begin{quotation}
    You must remember \Dots{} how much we are swayed by longstanding custom, or on the other side, by unfamiliarity and inexperience, and the same is true of singing. 
    You are not yet used to European singing and harmony, so you do not yet appreciate how sweet and pleasant it is, whereas we, since we are now accustomed to listening to it, feel that there is nothing more agreeable to the ear.

    But if we care to avert our minds from what is customary, and to consider the thing in itself, we find that European singing is in fact composed with remarkable skill; it does not always keep to the same note for all voices, as ours does, but some notes are higher, some lower, some intermediate, and when all of these are skillfully sung together, at the same time, they produce a certain remarkable harmony \Dots{} all of which, \Dots{} together with the sounds of the musical instruments, are wonderfully pleasing to the ear of the listener. \Dots{}

    With our singing, since there is no diversity in the notes, but one and the same way of producing the voice, we don't yet have any art or discipline in which the rules of harmony are contained; whereas the Europeans, with their great variety of sounds, their skillful construction of instruments, and their remarkable quantity of books on music and note shapes, have hugely enriched this art.%
         \Autocite[155-156]{Massarella:JapaneseTravellers}
\end{quotation}
Michael's friend Linus responds with a statement that surely understates the attitude of many non-Europeans whose ears had not yet been trained for European music:
\begin{quote}
    I am sure all these things which you say are true; for the variety of the instruments and the books which you have brought back, as well as the singing and the modulation of harmony, testify to a remarkable artistic system.
    Nor do I doubt that our normal expectations in listening to singing are an impediment when it comes to appreciating the beauties of European harmony.%
         \Autocite[156]{Massarella:JapaneseTravellers}
\end{quote}


% ***********************
\subsection{Obstacles to Faith and Mistrust of Hearing}

How, then, could the Church overcome such an impediment?
What if a person simply lacked the proper disposition to hear the Word with faith?
Theologians debated, at some remove from the human beings at the center of their disputations, whether Indians and Africans truly had the capacity for faith.
% \citXXX[Pagden?]
Someone who was not completely a \term{vir}, after all, could not be expected to arrive at the fullness of \term{virtus}; but still most writers concluded that these groups could be converted from the Satanic darkness of their paganism despite severe handicaps in reason and moral sense.
Jews, on the other hand, were not accorded even this limited hope of salvation, particularly in Iberia.
% citXXX[?]
The treatment of Judaism in Spanish religious literature exposes cracks in early modern Catholics' conception of faith and allows us to consider the anxiety and doubts many people had about hearing and faith.

In a Corpus Christi mystery play (\term{auto sacramental}) by Spanish court poet Pedro Calderón de la Barca, the figure of \term{Judaísmo} becomes a vivid representation of the incapacity to acquire faith, \term{despite} the sense of hearing.
Performed in 1634 to inaugurate Philip IV's new palace, the Buen Retiro, Calderón's \worktitle{El nuevo palacio del Retiro} centers on Judaism's incapacity for faith.%
    \Autocite{Calderon:Retiro}
Judaism is forcefully excluded from the festivities celebrated within the play, which culminate with the consecration of the Eucharist.
Instead Judaism stands to the side and asks the character Faith to explain each event to him (\expoemref{expoem:Calderon-Retiro-Judaismo}).
But despite trying to connect Faith's message with what he knows of the Hebrew Sciptures, Judaism cannot accept any of these explanations.
In fact he is unable to believe what Faith has said, because, as he says in an increasingly embittered refrain, \quoted{I have listened to Faith without Faith}.

% *******
\begin{expoem}
    \caption{Calderón, \worktitle{El nuevo palacio del Retiro}, \textlinenums{1303--1336}: Judaism rejects faith}
    \label{expoem:Calderon-Retiro-Judaismo}
    \inputexpoem{Calderon-Retiro-Judaismo}
\end{expoem}
% *******

In one way the character of Judaism is perfectly consistent with Catholic theology of faith and hearing.
His problem is neither with sensory ability nor even with his intellectual knowledge of doctrine: clearly heas listened carefully to Faith's explanations and grasps their connections to Hebrew scriptures.
But something must be lacking in this character that permanently disables him from acquiring the heartfelt conviction of faith and thus membership in the community of faith.
What is lacking, according to the theological system, is divine grace, bestowed through the sacrament of baptism in the true Church.
It is impossible for Judaism to convert because grace is withheld from him permanently.
% \citXXX[theological justifications for anti-Semitism in Spain]
Simple hearing is not enough.

Judaism's eloquent confession of unbelief is immediately drowned out by music, as clarion fanfares announce a royal procession.
For Calderón's listeners, who had been taught to regard Jews as the embodiment of willful unbelief and worse, the entry of the musicians would clear away the acrid sound of Judaisms's speech.
The feeling of doubt about the senses, however, pervades the entire play.

Much of the rest of the play dramatizes a contest of the senses, in which Hearing prevails---but only after confessing to his own incertitude.
Each personified sense competes for a laurel prize awarded by Faith (\expoemref{expoem:Calderon-Retiro-Hearing}).
Each sense in turn boasts of his powers, but Faith rejects each one.
Hearing is the last sense to present himself, and in contrast to the other senses, he speaks of his weakness, and how easily he can be fooled by echos or feigned voices.
Since he cannot trust his own powers, he must rely on faith.
In response, Faith crowns Hearing precisely because of his \foreign{desconfianza}---meaning lack of confidence, mistrust, and humility.

% *****
\begin{expoem}
    \caption{Calderón, \worktitle{El nuevo palacio}, \textlinenums{593--602}: Faith crowns Hearing}
    \label{expoem:Calderon-Retiro-Hearing}
    \inputexpoem{Calderon-Retiro-Hearing}
\end{expoem}
% ******

What would it mean, then, for hearing to be the favored sense of faith not just because of its humility, but because of its actual weakness?
How could the auditory art forms of music and poetry in villancicos, then, provide a medium for propagating the faith, if hearing was so easily deceived?

Calderón presents one character, Judaism, who hears what Faith says but lacks the faith to believe it; and another, Hearing, who admits that he cannot trust his own sense but for that very reason receives Faith's favor.
In Calderón's play, the mysteries of the Eucharist are beyond physical sensation: vision, taste, touch, and smell would not lead to the truth of Christ's presence in the host, but only hearing and believing Christ's words \quoted{This is my body} in the mouth of the priest.
Fully in line with the discourse on sensation and faith in the Roman Catechism, Calderón advises his listeners to trust some senses while distrusting others; to seek God through sensation while purging themselves of reliance on the senses.

The dialectic of trust and doubt in the senses urged Catholics to rely on the Church community and not their own experience.
The senses were powerful, and thus sensory art was powerful; but this power had to be harnessed to serve the purposes of the Church.
Simply put, faithful hearing required listeners to doubt their own sensation.

% XXX footnote on Reyre and Greer (+ Sage, Rietbergen) ?


\section{Hearing the Word in Community}

Catholic devotional music provided a practical medium for both appealing to the ear and training it, though music amplified the challenges of acquiring faith through hearing.
Music could be a way to make faith pleasing to the ear, as Kircher  describes, but this only worked if the ear was trained to find such music pleasing.
Likewise, techniques that gave sense and meaning to music---structures of phrasing, contrapuntal devices, symbolic and figurative word-painting---were not universally intelligible.
Religious ear training required individual discipline to avoid the dangers of over-reliance on subjective sensory experience.
In an ideal Catholic scenario, musical training also involved the whole community as participants through performance or listening, and in missionary and colonial contexts, it required the community to change its expectations of music to fit European norms.

For this reason the emphasis of post-Reformation Catholics on the Church as the source of certainty in faith, as the living embodiment of Christ the Word in the world, inspired Catholics to create Christian communities, and music was a potent tool in that endeavor.
At the same time, the virtue of man as Neoplatonic microcosm was reflected in the broader society and in turn depended on it.
Propagating faith, then, meant trying to establish not just individual Christians, but also building a Christian society as the body of Christ.
For Roman Catholics, the Church was the gospel.

This is why the Franscican friars in New Spain and the Jesuit priests in Brazil not only started parishes, but also trained choirs.
The musical efforts of the colonizing church concretely built social relationships through musical training.
Forming choirs of boys and training ensembles of village musicians in colonial Mexico were practical means of establishing the Church and inculcating faith on individual and communal levels.

Inculcating faith through music---the kind of fully formed faith that resulted in a faithful life as part of the community of the Church---meant more than a one-on-one dialogue between \soCalled{the music} and \soCalled{the listener}.
The musical ritual of the seventeenth-century Church involved a large number of community participants, for whom performing music with the body and hearing it were inextricably linked.

In terms of the most influential medieval music theorist Boethius (\worktitle{De musica} II), whose concept of music was foundational to the major Spanish music treatises like that of Pedro Cerone, well-tuned \gloss{musica instrumentalis}{sounding, played music} could harmonize the \term{musica humana}---the harmony of the individual in body and soul, reason and passion, but also the concord of human society.%
    \Autocites
    [II: 187--189]{Boethius:Musica}
    [203--208]{Cerone:Melopeo}
    [22--31]{Baker:Harmony}
All this could better reflect the cosmic \term{musica mundana} and ultimately the harmonies of the triune God.

Catholic music, then, was not \emph{about} society; it was a form \emph{of} society.
The challenge of cultural conditioning led Catholics to experiment with, and argue about, new ways of communicating across boundaries of language and customs.
Devotional music provided one way to make faith appeal to hearing, if some way could be found to overcome the \quoted{impediment} of non-European listeners' \quoted{normal expectations in listening to singing}, as the Japanese ambassadors' dialogue puts it.
At the same time, Catholics of the seventeenth century were encouraged to doubt their senses as much as to trust them; to submit their sensory experience to the authority of the church; and to be wary of the possibility that some people may lack the capacity for faithful hearing entirely.

% ****************************************
\section{The Primacy of Hearing in Villancicos}

Villancicos manifest many of these same theological preoccupations and anxieties.
They do not answer the questions we have raised---in fact some of them intensify the problems; but they do provide evidence for a broad public discourse about sensation and faith even as these pieces of music were themselves objects to be sensed and believed.

Like most of the other sources discussed in this chapter, villancicos were created by members of the literate elite, and thus they represent first of all a discourse between those elite poets, musicians, and churchmen.
At the same time, though, villancicos were generally performed in public festivities with varied audiences.
Compared with the royal pomp and verbal virtuosity of Calderón's Corpus Christi drama, for example, villancicos were performed in settings where there could be more proximity between people of different social strata.

It is legitimate to ask, as José María Díez Borque has done for Calderonian drama, how much listeners could actually hear and understand of these complex performances.%
    \Autocite{DiezBorque:Publico}
The same question should be asked of Italian opera, German sacred concertos, and English anthems of the same period.
There are not, at present, sufficient sources to reconstruct the sociology of villancico audiences for all their varied performance contexts around the world.
But Margit Frenk argues that Spanish literature of this period was written for the ear and that reading in most cases meant one person reading out loud to an audience of illiterate family and friends, whose auditory comprehension and memory would probably astound us today.%
    \Autocite{Frenk:Voz}
And while we may not know exactly who listened to villancicos, the musical sources do tell us what they heard with a far greater degree of specificity than is possible for literature alone.

Through vernacular poetry and likely also through musical style, the creators and performers of villancicos addressed a broad public audience.
These pieces often represented complex theological concepts, but they made these concepts accessible in quite a different way than a theological treatise or even a sermon.
As individual villancicos were repeated, and as conventional villancico typese were performed at multiple festivals each year, these pieces must have shaped attitudes and beliefs in the broader community.


%**************************************
\subsection{Contests of the Senses and Early Modern Concepts of Sensation}

Two villancicos from Segovia in the later seventeenth century demonstrate how a sophisticated discourse on sensation and faith was presented through devotional music.
The contest of the senses in Calderón's play is echoed in villancicos by successive chapelmasters at Segovia Cathedral.
Miguel de Irízar was born in 1634 and served at Segovia from 1671 until his death in 1684.
Jerónimo de Carrión, born in 1660, followed after Irízar in 1684 and died in 1721.
Both chapelmasters set variants of the same villancico poem, \worktitle{Si los sentidos queja forman del Pan divino}, which was attributed to Zaragoza poet Vicente Sánchez in the posthumous publication of his works in 1688.

The Segovia Cathedral archive preserves manuscript performing parts for both settings along with Irízar's draft score for his version.
Irízar corresponded with a peninsular network of musicians, often about exchanging villancico poetry; and he drafted his scores on the backsides and in the margins of these letters in makeshift notebooks.
\begin{Footnote}
    The performing parts for Irízar's setting are in \signature{E-SE}{5/32}; those for Carrión's are in \signature{E-SE}{28/25}.
2\end{Footnote}

The estribillo (\expoemref{expoem:Si_los_sentidos-Sanchez-estribillo}) invites hearers to imagine the senses \quoted{filing a complaint} against the bread of the Eucharist because \quoted{what they sense is not by faith consented}---playing on \foreign{sentido}, the word for sense.
Each of the coplas treats a different sense (\expoemref{expoem:Si_los_sentidos-Sanchez-coplas}), following nearly the same order as in Calderón's play: 
Sight comes first, followed by Touch; next are Taste and Smell, and Hearing comes last (\tableref{table:senses-order}).

% **************
\begin{expoem}
    \caption{\worktitle{Si los sentidos queja forman del Pan divino}, \shortcite[171--172]{Sanchez:LiraPoetica}, estribillo and coplas 1--2}
    \label{expoem:Si_los_sentidos-Sanchez-estribillo}
    \inputexpoem{Si_los_sentidos-Sanchez-estribillo}
\end{expoem}
% *************
% **************
\begin{expoem}
    \caption{\worktitle{Si los sentidos queja forman del Pan divino}, conclusion of coplas}
    \label{expoem:Si_los_sentidos-Sanchez-coplas}
    \inputexpoem{Si_los_sentidos-Sanchez-coplas}
\end{expoem}
% *************


The poem's treatment of the senses reflects a common physiological model of sensation and perception, as educated Spaniards would have learned from theological treatises.
Spanish theologians always treated Vision as the first and highest of the five exterior senses, as can be seen in treatises from seminary and convent libraries in Spain and Mexico.
A typical example is the 1557 natural-philosophy textbook \worktitle{Phisica, Speculatio} by a New Spanish Augustinian friar, Alphonus à Veracruce.%
    \Autocite{Veracruce:Phisica}
Veracruce summarize the traditional Catholic teaching, which drew on Aristotle as interpreted by Thomas Aquinas.
    % \citXXX[Aristotle early modern]
Veracruce's Latin treatise accords with Fray Luis de Granada's widely read Spanish \worktitle{Introduction to the Creed} of 1583.%
    \Autocites{LuisdeGranada:Simbolo}[(part I)]{LuisdeGranada-Balcells:SimboloPtI}

% ************
\begin{table}
    \caption{The exterior senses: Order of presentation in versions of \worktitle{Si los sentidos}, correlated with Calderón and Veracruce}
    \label{table:senses-order}
    \inputtable{senses-order}
\end{table}
% ***********

\tableref{table:senses-fray-luis} shows how Fray Luis explains the relationship of the five exterior senses to the interior senses, including the affective faculty, in which the sensory stimuli interacted with the balance of bodily humors.
The five exterior senses mediated between the outside world and the interior senses by means of the ethereal \term{spiritus animales}.
The cerebrum housed the internal faculties, which \quoted{made sense} of what the external senses told them---first the \quoted{common sense}, a kind of reception area where the exterior senses met the interior faculties; and next the imagination, the cogitative faculty, and memory.
All of these exterior and interior senses were part of the \term{ánima sensitiva}, the sensing, reasoning soul.
In addition to these senses the \term{ánima sensitiva} possessed an affective faculty, in which the balance of humors in the body interacted with the interior and exterior senses to produce different \quoted{passions} or \quoted{affects} (Fray Luis uses \term{pasiones} and \term{afectos} interchangeably).
Based on fundamental dichotomy (like magnetism) between attraction and repulsion, this \quoted{concupiscible} part of the soul experienced three primary pairs of passions: love and hate, desire and fear, joy and sadness.

% ************
\begin{table}
    \caption{The senses and faculties of the sensible soul (\term{ánima sensitiva}), according to Fray Luis de Granada}
    \label{table:senses-fray-luis}
    \inputtable{senses-fray-luis}
\end{table}
% ***********

The act of sensation, then, involved the entire body and soul, in a pre-Cartesian wholistic model; but the external senses differed in how they connected the external world to the internal faculties and passions.
The hierarchy of the senses was determined by the degree of mediation between the object of sensation and the person sensing.
The most base sense was taste, because the person actually had to physically consume the object of sensation.
The most powerful sense was sight, since it enabled a person to perceive objects a great distance away without any direct contact.

Hearing stood out from the other senses because for it alone, the object of perception was not identical with the thing sensed.
As Calderón's character Hearing says, \quoted{Sight sees, without doubting/ what she sees; Smell smells/ what he smells; Touch touches/ what he touches, and Taste tastes/ what he tastes, since the object/ is proximate \add{immediate} to the action}.%
\Autocite[\textlinenums{577--582}]{Calderon:Retiro}
But Hearing hears a person's voice, not the person directly, as Calderón's text continues: \quoted{But what Hearing hears/ is only a fleeting echo,/ born of a distant voice/ without a known object}.%
\Autocite[\textlinenums{583--586}]{Calderon:Retiro}
While this feature of hearing may have made it \quoted{easily deceived}, it also gave this sense a unique capability in spiritual matters, where the object of perception was not immediately sensible at all.

The poetic contests of the senses thus rearrange the traditional scholastic hierarchy by putting Hearing at the end for a dramatic climax.
Sight comes first, but Hearing, the underdog competitor, triumphs at the last.
In the Sánchez villancico, each of the coplas highlights the failure of one of the senses to rightly perceive the sacrament.
For exapmle, the eyes \quoted{do not look at what they see}, and the Eucharist reduces Sight to \quoted{blindness} (copla 1).
The \quoted{colors} and \quoted{rays of light} through which Sight normally operates are \quoted{hidden} \quoted{beneath transparent veils} and \quoted{transformed} so that \quoted{God Incarnate is not seen} (copla 2).
Similarly, Touch may make contact with the host, but not with they \quoted{mystery} hidden within (copla 3).
Taste and Smell (coplas 5 and 6) are similarly hindered by their ability only to perceive material accidents and not spiritual substance.

Sánchez presents hearing in the last copla, through the conceit of music.
The senses are \quoted{five instruments} like a musical consort, which must be \quoted{tempered} by faith.
Without Faith, sight is actually blind, and touch, taste, and smell are fooled; but when properly attuned by Faith, the sense can be harmonized into a pleasing concord.
Here Faith is not the object of sensation, but the subject, who delights in hearing the music of properly tuned senses.

% ***************************************************
\subsection{Staging the Contest in Sound: Irízar and Carrión at Segovia Cathedral}

%%% XXX Redo all measure numbers according to new edition

The two surviving settings of \worktitle{Si los sentidos} by Irízar and Carrión stage the contest of the senses in sound.
Their contrasting styles invite different types of involvement from listeners.
In the earlier setting, for Corpus Christi 1674 at Segovia Cathedral, Miguel de Irízar creates a musical competition in grand festival style by pitting his two four-voice choirs against each other in polychoral dialogue (\exmusicref{exmusic:Irizar-Si_los_sentidos}).%
\begin{Footnote}
    \Autocite{Cashner:SingingAboutSingingI}.
    \signature{E-SE}{5/32} is the manuscript performing parts in a copyist's hand, while \signature{E-SE}{18/19} is the draft score in Irízar's hand, and includes the heading, \quoted{Fiesta del SSantissimo de este año del 1674}.
\end{Footnote}
Polychoral dialogue, of course, is typical of large-scale villancicos, but Irízar has the choirs interrupt each other in ways that create not just a dialogue, but a debate.
Like a film editor creating a fight scene, Irízar builds intensity by cutting the text into shorter phrases to be tossed back and forth between the two choirs: \foreign{no se den por sentidos} becomes \foreign{no se den} and then \foreign{no, no}.

% *************
\begin{exmusic}
    \inputexmusic{Irizar-Si_los_sentidos}
    \caption{\worktitle{Si los sentidos queja forman del pan divino}, Miguel de Irízar (\signature{E-SE}{18/19, 5/32})}
    \label{exmusic:Irizar-Si_los_sentidos}
\end{exmusic}
% *************

Irízar creates a steadily increasing sense of excitement through shifts of rhythmic motion and style.
The setting of the opening phrase suggests a tone of hushed awe: the voices sing low in their registers, with a slow harmonic rhythm, and pause for prominent breaths (\measurenums{1--9}).
The harmonies here change less frequently than in the following sections, creating a relatively static feeling for this introduction.
In \measurenum{10} Irízar has the ensemble switch to ternary meter and increases the rate of harmonic motion.
The sense of antagonism is heightened when one choir interrupts the other with exclamations of \foreign{no} on normally weak beats (\measurenums{13--14}).
When Irízar returns to duple meter in \measurenum{18}, the voices move in smaller note values (\term{corcheas}) and exchange shorter phrases, so that the tempo feels faster (and the actual tempo could certainly be increased here in performance).
Each choir's entrances become more emphatic, repeating tones in simple triads, and Irízar adds more offbeat accents and syncopations, particularly for \foreign{no se den por sentidos los sentidos} in \measurenums{24--32}.
The estribillo builds to a climactic \term{peroratio} with the voices breaking into imitative texture in descending melodic lines.

The distinguishing stylistic characteristics of the setting suggest that Irízar is evoking a musical battle topic, a style one may find in \term{batallas} for organ as well as other villancicos on military themes.%
\begin{Footnote}
    \citXXX[Cite keyboard 2ry source on batallas]
    Keyboard examples include the \term{batallas} in Martín y Coll's \worktitle{Huerto ameno de flores de música}\citXXX, and in \citXXX[Portuguese collection] and \citXXX[Bruna works].
    Another villancico in this style is Antonio de Salazar's \worktitle{Al campo, a la batalla} (\signature{MEX-Mc}{A28}).
\end{Footnote}
Battle pieces typically feature a slow, peaceful introduction followed by sections in contrasting meters and styles and a texture of dialogue between opposing groups (as in between high and low registers on the keyboard). 
Typical of the style is the reiteration of chord voicings in what we could call root position, with the bass moving by fourths and fifths, and the 3-3-2 syncopations on \foreign{no se den por sentidos los sentidos} (\measurenums{25--26, 31--32, 43--46}).

Irízar sets the coplas, by contrast, in a sober and deliberate style.
The melody moves more calmly in duple meter with melodic phrases that fit well with the rhetorical structure of the poetic strophes.
Irízar has the treble soloist sing the third and fourth lines of each strophe in short phrases, where the latter phrase repeats the former down a fifth.
This creates a feeling of a teacher saying \quoted{on the one one hand} and \quoted{on the other hand} that suits the general philosophical tone of the strophes and matches the specific poetic phrasing of these lines.
To recall the Jesuit formula, Irízar's estribillo seems more designed to delight, while the coplas provide more of an opportunity to teach.

Irízar's villancico seems to speak to a large crowd through grand, unsubtle gestures and sharp contrasts of bright colors.
By contrast, Jerónimo de Carrión's later setting of the same poem (\exmusicref{exmusic:Carrion-Si_los_sentidos}) invites a more personal reflection.%
    \Autocite{Cashner:SingingAboutSingingI}
Carrión was capable of the festival style (as in his \worktitle{Qué destemplada armoníá}, which almost takes on the dimensions of a \term{cantada}).%
    \footnote{\signature{E-SE}{20/5}.}
But this setting fits more in the subgenre of \term{tono divino} or chamber villancico, a continuo song used in more intimate settings like Eucharistic devotion.%
    \citXXX[tonos divinos]
The style similar to the \soCalled{high Baroque} music of contemporary Italy, with a tonal harmonic language, a running bass part in the accompaniment, and a single affective manner throughout.

% ************
\begin{exmusic}
    \inputexmusic{Carrion-Si_los_sentidos}
    \caption{\worktitle{Si los sentidos queja forman del pan divino}, Jerónimo de Carrión (\signature{E-SE}{28/25})}
    \label{exmusic:Carrion-Si_los_sentidos}
\end{exmusic}
% ************

The dialogue and rivalry of the poetic text happens now not through polychoral effects but through motivic exchanges between voice and accompaniment.
Instead of metrical contrasts from one section to the next, Carrión creates rhythmic contrasts between simultaneous voices.
Carrión dramatizes \foreign{queja} (\measurenum{2}) with a metrical disagreement between the two voices (normal ternary motion versus the voice's sesquialtera).
The descending pattern of leaps for \foreign{porque lo que ellos sienten} perhaps suggests the confusion and tumult of the senses, and it creates a certain amount of rhythmic confusion as it moves between voices.
Carrión creates a climax through a canon between soloist and accompaniment in \measurenums{18--20} that leads the singer to the top of his register.
The upward leaps in the last line on \foreign{no se den} (\measurenum{16}) contrast with the downward leaping motive of the opening (on \foreign{sentidos}, \measurenum{1}).

The similarities between these two settings of \worktitle{Si los sentidos} demonstrate the persistence of concerns about the hearing's role in faith.
Meanwhile the differences between versions reflect changing styles not only of composition but of devotional practice.
Irízar and Carrión take a verbal discourse on sensation and faith, in which music is the paradigm of something that pleases the ear, and bring it to life through actual music.
Thus the pieces seem designed to teach listeners how to hear music even as they are listening---they accomodate hearing while training it, as the catechism says.


%*****************
\subsection{Sensory Confusion}

While the \worktitle{Si los sentidos} villancicos may not project as much uncertainty about sensation and faith as does Calderón's \worktitle{El nuevo palacio del Retiro}, they still emphasize the need for all the senses to submit to faith, which means that listeners should not trust their senses alone.
Some villancico poets and composers go further than stating that senses can be deceitful; they use paradox to deliberately confuse the senses for pious purposes.
We have already seen in the previous chapter how many villancicos feature auditory \soCalled{special effects} like echoes, voices imitating instruments, and voices imitating birdsong.
Such pieces might be compared to the contemporary rise of \term{trompe l'oeil} effects in visual art, like the illusion of the heavens opening the \term{Transparente} of Toledo Cathedral or the false domes painting on Jesuit church ceilings from Rome to Japan.%
    \Autocite[\XXX + illusion in Baroque art]{Bailey:Art}

Villancicos with \soCalled{synesthetic} topics mismatch the sense in the spirit of paradox and enigma.%
\begin{Footnote}
    \Autocite{DoetschKraus:Sinestesia} explores connections between poetic \quoted{synthesis of the senses} in Spanish verse and the actual neurological phenomenon of synesthesia.
\end{Footnote}
For example, sight and hearing are the principal objects in the anonymous fragment \worktitle{Porque cuando las voces puedan pintarla} (If voices could only paint her).%
\footnote{\signature{E-Mn}{M3881/44}.}

Cristóbal Galán, master of the Royal Chapel (1680--1684, after serving as chapelmaster at Segovia Cathedral) juxtaposes hearing and vision in a villancico for the Conception of Mary.%
\footnote{\signature{D-Mbs}{Mus. ms. 2893}, edition in \autocite[567--568]{CaberoPueyo:PhD}.}
Galán's text exhorts listeners to \quoted{hear the bird} and \quoted{see the voice}.
The visual language in this villancico evokes the common iconography of the Holy Spirit as a dove surrounded by golden rays, such as may be seen in the Monastery of the Encarnación in Madrid, where the Royal Chapel frequently performed.%
\begin{Footnote}
    The image was painted on the ceiling of the monastery's Capilla del Cordero and when a new church building was added later, this image was incorporated as the central element atop the high altar.%
    \Autocite[69--70, 81]{Sanz:GuiaDescalzasEncarnacion}\XXX[Royal Chapel, dates]
\end{Footnote}

The poem (\expoemref{expoem:Oigan_todos_del_ave-Galan}) makes \quoted{confusion} of sight and hearing, which is projected partly through irregular poetic meter.%
\begin{Footnote}
    The division into lines is speculative, but the syllable counts and line groupings in this arrangement could be scanned as 10.6 10.8 7 6.6.6 10.10.
\end{Footnote}
In his musical setting for eleven voices in three choirs, Galán creates \term{equivocación} through rhythm, notation, and texture.
Galán juxtaposes the three voices of Chorus I against the other two choirs by having Chorus I sing primarily in a normal triple-meter motion (with dotted figures intensifying the ternary feeling), while the other two choirs interject \foreign{¡Oigan!} and \foreign{¡Miran!} in sesquialtera rhythm.

%**************
\begin{expoem}
    \caption{\worktitle{Oigan todos del ave}, setting by Cristóbal Galán, estribillo}
    \label{expoem:Oigan_todos_del_ave-Galan}
    \inputexpoem{Oigan_todos_del_ave-Galan}
\end{expoem}
%*************

To notate these rhythms, Galán must use white notes for the regular ternary motion in Chorus I, but blackened noteheads (mensural coloration) to indicate the hemiola pattern in the other choirs.
But when each voice in Chorus I sings the synesthetic phrase \quoted{the light is heard to shine}, Galán turns the lights out---on the page at least---by giving each voice a passage in all black-note sesquialtera (\figureref{fig:Galan-Oigan-coloratio}).
They return to white notation again for the following phrase, \quoted{while the voice is seen in purity}.
Any attentive listener could hear these juxtapositions and abrupt shifts in rhythmic patterns, though only the musicians themselves would likely have been in on the dark--light symbolism in the notation.

%***************
\begin{figure}
    \includeWideFigure{Galan-Oigan-coloratio}
    \caption{Galán, \worktitle{Oigan todos del ave} (\signature{D-Mbs}{Mus.\ ms.\ 2893}), Tiple I-2, end of estribillo: Ironic play of coloration}
    \label{fig:Galan-Oigan-coloratio}
\end{figure}
%***************

In texture Galán plays with \foreign{qué equivocación} literally by setting these words to a fugato on a long ascending scalar figure (starting with the Tiple II's stepwise ascent A\octave{4}--G\octave{5}).
In addition to the literal pun of \quoted{equal voices} for \foreign{equivocación}, the sudden outburst of polyphonic texture in the midst of primarily homophonic polychoral dialogue could create a more affective sense of confusion as well.
As the estribillo continues, Galán increasingly mixes up the music for \foreign{Oigan todos del ave}, the sesquialtera interjections, and the contrapuntal texture of \foreign{qué equivocación}, between the various choirs.

These pieces describe and seek to incite a condition of sensory overload, an ecstasy in which all the senses blend together in the effort to grasp something that is beyond them.%
  \begin{Footnote}
      This notion is part of the Catholic heritage that shaped the Olivier Messiaen's concept of \quoted{dazzlement}.\citXXX[Messiaen, French term]
  \end{Footnote}
Such pieces provide further evidence that Catholic religious arts could not be reduced to the function of simply teaching doctrine; these pieces train the senses by appealing to them directly.
They seem intended to provoke listeners to a higher form of sensation, to a holy dismay and wonder that would lead to true, faithful perception.

% ******************
\section{Impaired Hearers, Incompetent Teachers: \quoted{Villancicos of the Deaf}}

The problem of lacking the capacity for faithful hearing is evident in
villancicos that represent deaf characters.
The final section of this chapter presents two \term{villancicos de sordos}---villancicos of the deaf---which dramatize the limitations of hearing, and poke fun at the difficulty some religious teachers faced in making faith appeal to this sense.
Similar pieces from both sides of the Atlantic use hearing disability as a symbol of spiritual deafness: the first is by Juan Gutiérrez de Padilla from Puebla Cathedral in New Spain, and the other is by Matías Ruíz for the Royal Chapel in Madrid.
Both pieces present mock catechism scenes with a friar and a deaf hard-of-hearing man.

\subsection{Laughing at the Deaf: Juan Gutiérrez de Padilla (Puebla, 1651)}

In a piece labeled \foreign{sordo} in the partbooks, Juan Gutiérrez de Padilla creates a comic dialogue between a religious teacher and a \soCalled{deaf} man named Toribio (\expoemref{expoem:Oyeme_Toribio-Padilla}).
The villancico, which begins \worktitle{Óyeme, Toribio} and is labeled as a \quoted{Dúo con bajón}, was performed in Matins for Christmas at Puebla Cathedral in 1651.%
\footnote{\signature{MEX-Pc}{Leg. 1/2}.}
It is part of Padilla's earliest surviving Christmas cycle for the new cathedral, which had been consecrated in 1649.
Though two key partbooks are missing, including the Tenor I part who played the deaf man, but the dialogue can be reconstructed because the lyrics of the deaf man's part were written in the surviving bass part.%
\begin{Footnote}
    The friar was played by the Altus I. 
    The friar was probably accompanied by the Bassus I on \term{bajón}, but this partbook is missing.
    The deaf man was played by the Tenor I, but this partbook is missing.
    The deaf man was accompanied by the Bassus II on \term{bajón}.
    Though the Tenor I solo music is missing, the lyrics for the vocal part are written in above the accompaniment in the Bassus II part. 
    Thus we have the music for the friar without its accompaniment, and the accompaniment for the deaf man and most of the lyrics, but not the deaf man's music.
\end{Footnote}

% ***********
\begin{expoem}
    \caption{\worktitle{Óyeme, Toribio (El sordo)}, from setting by Juan Gutiérrez de Padilla, Puebla, 1651 (\signature{MEX-Pc}{Leg. 1/2}), excerpt}
    \label{expoem:Oyeme_Toribio-Padilla}
    \inputexpoem{Oyeme_Toribio-Padilla}
\end{expoem}
% ************

Playing with a conventional villancico type of a dialogue between a \term{docto} and a \term{simple}---a learned man and a simpleton---Padilla's villancico stages a parody of catechism instruction.
The friar's attempts to communicate with the \soCalled{deaf} man fail, and this prompts the five-voice chorus to warn the congregation against spiritual deafness.
Padilla dramatizes the two characters' unsuccessful attempts to communicate through disjunctions of rhythm and mode (\exmusicref{exmusic:Padilla-Sordo-intro}).
Rhythmically, Padilla gives each singer a distinct \quoted{accent} in music: the friar's musical speech is more fluent and sophisticated, while the deaf man's speech is halting and clumsy, such as his questions \quoted{Eh? eh?} on an offbeat figure.

% *******************
\begin{exmusic}
    \inputexmusic{Padilla-Sordo-intro}
    \caption{Padilla, \worktitle{Óyeme, Toribio (El sordo)}, introducción, \measurenums{1--25}, extant parts (missing Tenor I, Bassus I)}
    \label{exmusic:Padilla-Sordo-intro}
\end{exmusic}
% *******************

Padilla illustrates the men's disagreement by having them fail to concur on where to cadence.
Given the one-flat \term{cantus mollis} signature, the cadence points articulated by the extant bass part, and the final on F, we may categorize this piece as mode 11 or 12.
The friar appropriately sings an opening phrase that surely would have cadenced on F; but the deaf man responds with a phrase that cadences on C.
At one point the friar moves to a cadence on A, but the deaf man, responding that he can't hear out of that ear, cadences on D.
The friar says he will try the other ear, but no sooner has the friar moved to D, than Toribio, saying, \quoted{Out of that ear I hear even less!} moves to a cadence on C.
This pushes the friar over the edge.
He bursts out, \quoted{You are a sheer idiot!}---mimicking the deaf man's halting short-long rhythms, and returning to his own final of F.

After the dialogue in the introducción, the Altus I, who has been representing the friar, seems to turn away from the dramatic scenario and address the congregation as a preacher: \quoted{The laughter of the dawn will turn to sobs}, he says---referring to the Virgin Mary---when, having given birth to the Word Incarnate, \quoted{her eyes see deaf men}.
Padilla sets the final phrase about deaf men, with ten notes in black mensural notation that make a rhetorical \term{catabasis} as they descend in a leaping melodic sequence (\figureref{figure:Padilla-Sordo-MS-estribillo}).

When the rest of Chorus I joins in for the responsión, their repeated dotted rhythm suggests vivacious laughter and comic offbeat sobs on \foreign{sollozos}.
The catabasis figure is passed through all the voices in imitation, leading to a harmonic catabasis when the Tiple I adds E\fl{}---shifting further away from the \soCalled{natural} into the \soCalled{weak} realm of flats.%
\citXXX[chafe?]
The heavy syncopation in each voice creates rhythmic confusion that is not sorted out until the final cadential flourish on F, validating the friar's initial choice of mode.

% *******************
\begin{figure}
    \includeTallFigure{Padilla-Sordo-MS-estribillo}
    \caption{Padilla, \worktitle{Óyeme, Toribio (El sordo)}, Altus I manuscript partbook, introducción and estribillo}
    \label{figure:Padilla-Sordo-MS-estribillo}
\end{figure}
% *******************

Through a mixture of sophisticated musical technique, high-minded theology, and low caricature that is characteristic of this composer, Padilla belittles actual deaf people as deficient, undignified, and deserving of laughter.
Ironically, Padilla's ensemble made their caricature at the same time that some Spanish churchmen, like Juan Pablo Bonet, were engaged in actually ministering to deaf people.%
\begin{Footnote}
    Juan Pablo Bonet, \worktitle{Reduction de las letras y Arte para enseñar a ablar a los mudos} (\XXX, 1620), cited and discussed in \autocite{Plann:DeafEducationSpain}.
    % XXX more
\end{Footnote}
But the creators of \term{villancicos de sordos} seem more interested in exploiting the deaf to amuse and edify the hearing.

Actual deafness is used here with the main goal of pointing to spiritual deafness: the \quoted{deaf men} that will make Mary weep are all people whose ears have been stopped by sin and cannot hear the divine Word of Christ with faith.
In his 1610 Spanish dictionary, Sebastián de Covarrubias defined the \term{sordo} as \quoted{he who does not hear}, not \quoted{who cannot hear}.
He adds, \quoted{There is no worse kind of deaf man than he who is unwilling to hear}.%
\begin{Footnote}
    \Autocite[\sv{sordo}]{Covarrubias:Tesoro}:
    \quoted{SORDO, Lat. surdus, el que no oye. 
    No ay peor sordo que le que no quiere oyr.}
\end{Footnote}
In this view, if hearing is the paradigm of faith, then deafness is its opposite.
While Calderón's Judaism heard Faith without faith, the deaf men in villancicos cannot even hear Faith to begin with.

% **********
% START
\subsection{Learning from the Deaf: Matías Ruíz (Madrid, 1671)}

The \worktitle{Villancico de los sordos} by Matías Ruiz extends its parody to the catechist as well.%
\footnote{\signature{E-E}{Mús. 83-12}.}
Ruiz was chapelmaster at the Real Convento de la Encarnación in Madrid, and the poetry imprint survives from what must have been the first performance there at Christmas, 1671.%
\autocite{1671-Madrid-Enc-Nav}
Here the \term{sordo} is a hard-of-hearing man, \quoted{very learned in humane letters}---a doddering old university professor, or perhaps a street sage.
The piece mocks his impairment while contrasting true faith with the book learning of this would-be humanistic scholar.
But the biggest laughs come at the expense of the friar, as the deaf man mishears his rote teaching formulas in increasingly absurd ways (\expoemref{expoem:Pues_la_fiesta-Ruiz-estribillo}).

% ****************
\begin{expoem}
    \caption{\worktitle{Pues la fiesta del Niño es (Villancico de los sordos)}, from setting by Matías Ruiz, Madrid, 1671 (\signature{E-E}{Mús. 83-12}, \signature{E-Mn}{R/34989/1}), estribillo}
    \label{expoem:Pues_la_fiesta-Ruiz-estribillo}
    \inputexpoem{Pues_la_fiesta-Ruiz-estribillo}
\end{expoem}
% ***************

The piece begins with soloist and chorus gleefully crying \quoted{On with the deaf man!}  rather like a bunch of high-school bullies, telling everyone to speak up so he can hear.
When the catechist and the \term{sordo} enter, Ruiz gives the two characters phrases that contrast melodically, harmonically, and rhythmically, to illustrate their inability to understand each other.
The deaf man's musical speech is abrupt, uncouth, and loud, fitting with the friar's mockery of the deaf man's unmodulated voice.
The deaf man bursts on the scene with a scale from the top of his register to the bottom (F\octave{4} to G\octave{3}).
The descent across vocal \term{passaggi} would encourages the singer to bawl the phrases in a coarse tone of voice.
It is ironic that this character who hears poorly should be marked primarily by the kind of sounds he makes.

% ***********
\begin{exmusic}
    \inputexmusic{Ruiz-Sordos-dialogue}
    \caption{Ruiz, \worktitle{Pues la fiesta del niño es (Villancico de los sordos)} (\signature{E-E}{Mús. 83-12}), estribillo, \measurenums{41--55}}
    \label{exmusic:Ruiz-Sordos-dialogue}
\end{exmusic}
% ***********

Unlike in Padilla's villancico, here the deaf man has a lesson of his own to teach.
He may not be able to hear well but has come with love to adore the Christ-child. 
Acting as a kind of holy fool, and echoing Covarrubias's definition of deafness, he reminds everyone within the sound of his voice that the truly deaf are \quoted{those who neither listen nor understand the sound}.

In the parodied catechism lesson presented in the coplas, (\expoemref{expoem:Pues_la_fiesta-Ruiz-coplas-1}) the friar quizzes his pupil on key doctrines of Christmas, classic topics in both scholastic and pastoral literature:
Tell, \term{sordo}, he asks, how did God fulfill his word to the prophet-king David?
What motivated Christ to become incarnate?%
\citXXX[Christmas doctrinal literature]
But the deaf man mishears every statement: he mistakes \foreign{sordo} for \gloss{gordo}{chubby}, and \gloss{profeta}{prophet} as \gloss{estafeta}{mailman}.

% ****************
\begin{expoem}
    \caption{\worktitle{Pues la fiesta del Niño es (Villancico de los sordos)}, from setting by Matías Ruiz, coplas 1--5}
    \label{expoem:Pues_la_fiesta-Ruiz-coplas-1}
    \inputexpoem{Pues_la_fiesta-Ruiz-coplas-1}
\end{expoem}
% ***************

% ****************
\begin{expoem}
    \caption{\worktitle{Pues la fiesta del Niño es (Villancico de los sordos)}, from setting by Matías Ruiz, conclusion of coplas}
    \label{expoem:Pues_la_fiesta-Ruiz-coplas-2}
    \inputexpoem{Pues_la_fiesta-Ruiz-coplas-2}
\end{expoem}
% ***************

His supposed learning in the humanities leads only to confusion.
When the friarn lauds the \gloss{bailes}{dances} of Christmas, and says \gloss{el portal es nuestro alivio}{the stable is our remedy}, the deaf man thinks he is citing \foreign{Tito Libio}.
The humanist is puzzled: he has read the Classical historian Livy, he says, but Livy doesn't say anything about \gloss{frailes}{friars}.

Hearing that the child Jesus is shivering with cold, the deaf man suggests he drink hot \foreign{chocolate}.
The friar reassures him that \quoted{the Queen}---the Virgin Mary---is keeping the child bundled, such that he glows with warmth (\foreign{arde}).
The deaf man now seems to feel that at last he has figured out what they are talking about, and sums up with satisfaction, \gloss{Esta es, por la mañana y tarde, la Reina de las bebidas}{Chocolate is, morning and evening, the Queen of beverages}.

We can imagine that Ruiz's deaf man would invite the sympathies of listeners.
He is an earthy, common character, focused on material comforts, keenly aware as many older men are of the chill, and doting tenderly on the infant.
The deaf man's bumbling but endearing staments contrast strikingly with the friar's abstract theology and clichéd poetic language.
The text even reminds listeners of the deaf man's response to mockery.
He cannot hear the nine choirs of Christmas angels, so he asks them to sing out loudly, as long as they don't say anything bad about him.

Ruiz's characters present a contrast of types of learning: the churchman who repeats the same teaching points in every catechism class, versus an ersatz humanist who has read Livy and perhaps Ovid but may not understand them at all.
Ruiz's hard-of-hearing humanities scholar demonstrates a central tenet of Reformation-era Catholicism: that Classical learning alone is not enough to understand Christianity.
Even so, the friar's inability to teach this man demonstrates yet again the pervasive uncertainty about how faith could be made to appeal to hearing.

Ruiz's \worktitle{Villancico de los sordos}, like the other pieces we have discussed, exalts hearing music as a means to faith---one which the deaf cannot understand.
This is because the word \foreign{son} in the epigrammatic conclusion of the estribillo could mean not only sound but was also a term for a type of dance or song, according to the 1610 dictionary of Covarrubias.
The villancico is built on a distinctive harmonic and rhythmic pattern of alternating ternary and sesquialtera groupings, which are especially clear on the phrase \foreign{los que no escuchan ni entienden el son}.
This pattern bears a close resemblance to dance forms known as \term{son} today (\exmusicref{exmusic:Ruiz-Sordos-son})---most obviously, to the Mexican \term{huarache} familiar from Leonard Bernstein's \quoted{America}.%
    \Autocite[\sv{huarache}]{Grove}
Even if the reference to \term{son} is not this specific, the music certainly evokes the feeling of a social dance whose circle the deaf man is unable to join.


% **************
\begin{exmusic}
    \inputexmusic{Ruiz-Sordos-son}
    \caption{Ruiz, \worktitle{Villancico de los sordos}, conclusion of estribillo, \measurenums{76--83}: Possible evocation of \term{son} song/dance style}
    \label{exmusic:Ruiz-Sordos-son}
\end{exmusic}
% *************

The creators of villancicos of the deaf could certainly have imagined that this music was contributing to the goal of spiritual ear training.
There may be a higher theological point here about spiritual deafness, but both \term{sordo} villancicos make the point at the expense of real people with disabilities.
Whether the music really accomplished that goal, or just provided amusement and reinforced cultural stereotypes for commoners, is a question that haunts the whole repertoire.
In a pattern typical of this genre (and especially vexing in the \soCalled{ethnic} villancicos), someone on the margins of society is, through musical representation, welcomed to Christ's manger, but the way he is represented actually emphasizes his exclusion from the community.
At the same time they also demonstrate a surprising levity about the church and its incompetent teachers.


% *************
\section{Failures of Faithful Hearing}

Critiquing both the poor level of theological knowledge among the lay people, and the low quality of teaching among the clergy, were both characteristic Tridentine postures.%
    \Autocite[56-57]{Kamen:EarlyModernSociety}
Antonio de Azevedo describes real-life scenes of failed catechesis:
\begin{quotation}
    Some will say that the doctrine of the gospel has already been taught everywhere or almost everywhere (I am speaking of our Spain), and we concede;
    but there are so many parts that so badly lack anyone who could teach matters of faith,
    that indeed it is a shame to see it happen in many parts of Spain, and particularly in the  mountains,
    where there are many so unlettered \add{\foreign{bozales}} in the matters of faith,
    that if you would ask them, how many are the persons of the Holy Trinity, some would say that they are seven, and others, fifteen; and others say about twenty---of this I am a good witness.

    And a principal friar of my order, I've heard that once he was asking a woman how many \add{persons in the Trinity} there were, and she said, \quoted{Fifteen}.
    And he said, \quoted{\foreign{Ay}, is that really your answer?}
    And then she wanted to correct herself, and she said, \quoted{\foreign{Ay Señor}, I think I was wrong---I'll say there are five hundred}.%
        \Autocite[26: \quotedsp{Diran, o que ya ay dotrina del Euangelio en todas partes, o casi todas (hablo de nuestra España) concedamoslo: 
        Pero ay tanta falta en muchas, de quien enseñe las cosas de la fe; que cierto que es lastima, verlo que en muchas partes de España, y particularmente en montañas passa: a do estan muchos tan boçales en las cosas de la fe, que si les preguntays, quantas son las personas de la Santissima Trinidad, vnos dizen que son siete otros que quinze; y otros veynte desatinos, de los quales yo soy buen testigo.
        Y a un frayle principal de mi orden le oy, que preguntando el a vna muger, quantas eran, que dixo ella que quinze, y diziendole el ay, y esso aueys de dezir? y ella se quiso emendar, y dixo ay Señor, digo mi culpa, digo que son quinientas}.]
        {Azevedo:Catecismo}
\end{quotation}

Azevedo sees no humor in this lack of religious knowledge; and he faults not the illiterate laypeople but the religious orders and clerics who have failed to teach the basics of faith in a plain way, as Azevedo himself endeavors to do in his book:
\begin{quote}
    It is a shame to see the ignorance that there is in many, in things of such importance.
    \Dots{} because even though the religious orders and those who preach do declare the gospel, 
    they do not explain the ABCs \add{\foreign{b, a, ba}} of Christianity;
    they do not want to deal with giving milk because this is the task of mothers, those lordly Curates or Orators, who are responsible for this task, and what I have described is their fault.%
        \Autocite[27: \quotedsp{Es lastima ver la ignorancia que ay en muchos, en cosas de tanta importancia: Y preguntados algunos qual de las tres personas encarno, el vno dize, que el Padre otros que el Espiritu Santo: y en muy buenos pueblos lo he oydo yo, hartas vezes con mis oydos; porque dado los religiosos y los que predican declaren el Euangelio, no tratan del b,a ba de cristiandad, no tratan de dar leche porque esse es officio de madres, de los señores Curas, o Retores, a cuyo cargo esta esso; y cuya culpa es lo dicho.}]
        {Azevedo:Catecismo}
\end{quote}

The friars of Padilla's and Ruiz's villancicos seem to fit with Azevedo's description of \quoted{lordly orators} who delight in lofty language, rather than motherly teachers who spell out the fundamentals of Christian faith.
Azevedo's critique, though, is motivated by the late-sixteenth-century Tridentine agenda of reform and education.
By the mid-seventeenth century, the function of Catholic religious art under developing \soCalled{Baroque} aesthetics shifted away from the kind of \quoted{plain} instruction Azevedo models, toward more ornate, learned, and often arcane forms of expression.

Few villancicos of the seventeenth century would satisfy Azevedo's call to teach the \quoted{\foreign{b, a, ba} of Christianity}.
Even the comic villancicos depend on learned plays of language and music, like the Classical references in Ruiz's poetic text, or the play on modal cadences and black notation of Padilla's music.
This means that the depictions of imperfect hearing in the villancicos of the deaf themselves depended on the attention of listeners with well-trained ears.
The fundamental Tridentine problem, of making faith appeal to hearing by both accomodating the senses and training them, remained a challenge for Catholics, both those who would teach through speech or song and those who would listen.


