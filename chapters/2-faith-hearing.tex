% Andrew Cashner -- Making Faith Appeal to Hearing

% 2016-08-31    Revision for book begun (ch. 2)
% 2016-09-16    First draft based on diss.
% 2016-10-13    Abridged version for USCB presentation

% *******************************************

\label{ch:faith-hearing}

\epigraph
{I cannot comprehend you nor know your enigma,/ because I have listened to Faith without faith.}
{\quoted{Judaism}, in Calderón, \worktitle{El nuevo palacio del Retiro}, \textlinenums{13\XXX--\XXX}}

St. Paul wrote to the Christian community in Rome, \quoted{How are they to believe if they have not heard?}, since \quoted{faith comes through hearing, and hearing, by the Word of Christ} (\scripture{Rom 10:16-17}).
Seventeen centuries later, amid the ongoing reformations of the Western Church, Catholic Christians were seeking ever new ways to make faith audible.
Poets and composers of the Spanish Empire expanded a genre of sung poetry in the vernacular---the villancico---into large-scale choral and instrumental performances that could appeal to the ears of elite and common people alike.%
\begin{Footnote}
  For an introduction to the genre, see the entries for \quoted{villancico} in \worktitle{Grove Music Online} and the \worktitle{Diccionario de la música española e hispanoamericana}.
  The major studies of the villancico as a musical and poetic genre are, in chronological order, \autocites{Rubio:Forma}{Laird:VC}{Torrente:PhD}{Tenorio:SorJuana}
  {CaberoPueyo:PhD}{Illari:Polychoral}{Knighton-Torrente:VCs}
  {Cashner:Cards}{Cashner:PhD}.
  \XXX[new caliope article]

  Catalogs of the imprints of villancico poetry, published in coordination with festival performances of villancicos, are \autocites{BNE:VCs17C}{BNE:VCs18C}{UK:VCs}{US:VCs}. \XXX[Montserrat]
  Modern musical editions of villancicos include \autocites{Cererols:MEM-VC}
  {Stevenson:Christmas}{Ruimonte:Parnaso}{Padilla:Tello}{Ezquerro:MME55}
  {RuizSamaniego:MME63}{Ezquerro:MME59}{Ezquerro:MME65}{Fernandez:Cancionero}
  {Torrejon:VCs}{Cashner:SingingAboutSingingI}.
\end{Footnote}


With the Church's active patronage, villancicos became a central activity in religious festivals throughout the year, particularly at Christmas, Corpus Christi, and the Immaculate Conception of Mary.
Church ensembles performed these pieces, with their motet-like refrains or \term{estribillos} surrounding a set of strophic verses or \term{coplas}, as an integral part of Matins and other liturgies.
Festival crowds from Madrid to Manila also heard villancicos in public processions and in conjunction with mystery plays.
Villancicos expressed widespread beliefs and attitudes while also shaping them.

A large number of villancicos begin with calls to listen---\foreign{escuchad}, \foreign{atended}, \foreign{silencio, atención}.
Because so many villancicos explicitly address concepts of music, sensation, and faith, these remarkable but understudied pieces offer us unique insights into how early modern Catholics understood the supernatural power of music.
How, then, did the people who created, performed, and listened to villancicos understand music's role in the relationship between hearing and faith?

Villancicos on the subject of hearing and faith, I argue, reveal a certain anxiety about the possibility of making faith appeal to hearing.
Spaniards and their colonial subjects worried about how to listen faithfully, about the role of subjective experience and cultural conditioning, and about the possibility that some listeners might lack the capacity to hear music rightly.

We will begin by discussing theological and literary sources that reveal tensions in theological formulations of faith, hearing, and music.
Then we will look at two pairs of villancicos, never previously edited or studied, that manifest the same tensions.
The first two pieces stage allegorical contests of the senses, and the last two represent characters whose impairments of hearing render them unable to understand religious teaching.

If we listen closely to villancicos as historical sources for theological understanding, it becomes clear that villancicos functioned as much more than vehicles for simplistic religious teaching or as banal, worldly entertainment.
The four villancicos we will discuss today appeal to the ear in order to give listeners an opportunity to contemplate the challenges of faithful hearing.

% ***********
\section{The Challenge of Making Faith Appeal to Hearing}

A central document for understanding the faith of post-Reformation Catholics is the catechism \quoted{for parish priests} commissioned by the Council of Trent and first published in 1566.%
\Autocites{Catholic:Catechismus1614}[\sv{catechisms}]{NewCatholic}
The catechism grounds the Church's authority to preserve and teach \soCalled{the faith} in the theology of Christ's Incarnation.

According to the catechism, God communicated his own nature to humanity by taking on human flesh in Christ.
Therefore the true Word of God was not confined to Scripture or doctrines alone---the Word was a person, Jesus Christ, to whom the prophetic scriptures testified and from whom the Church's doctrine flowed.
In the words of John's gospel, Jesus Christ himself was the \term{logos} or \term{verbum} (\scripture{Jn 1:1}), the \quoted{Word made flesh}.
Where God had spoken through the law and the prophets, now he had spoken through his Son (\scripture{Heb 1:1}).

Christ founded the Church, then, to be the means through which people would come to know him after his resurrection.
The Word that was made flesh in the person of Christ was now embodied sacramentally in a community.
The church's task in promulgating faith, then, was not merely to teach rote verbal formulas to its catechumens or to persuade them to agree with certain concepts.
Rather, the church represented Christ to the world, and the catechumen's role was ultimately to enter into communion with the triune God through participation in the church.
How, then, could the church communicate Christ the Word through the sense of hearing?

Drawing on St.\ Paul's dictum from Romans, the catechism challenges the Church's ministers to find a way to make Christ the Word audible:
\begin{quote}
  Since, therefore, faith is conceived by means of hearing, it is apparent, how necessary it is for achieving eternal life to follow the works of the legitimate teachers and ministers of the faith. \Dots{}
  Those who are called to this ministry should understand that in passing along the mysteries of faith and the precepts of life, \emph{they must accommodate the teaching to the sense of hearing and intelligence}, so that by these \add{mysteries and precepts}, \emph{those who possess a well-trained sense} should be filled up by spiritual food.%
  \begin{Footnote}
    \Autocite[2, 8--9 (emphasis added)]{Catholic:Catechismus1614}: 
    \quotedla{Cvm autem fides ex auditu concipiatur, perspicuum est, 
    quàm necessaria semper fuerit ad æternam vitam consequendam 
    doctoris legitmi fidelis opera, 
    ac ministerium \Dots\ vt videlicet intelligerent, 
    que ad hoc ministerium vocati sunt, 
    ita in tradendis fidei mysteriis, ac vitæ præceptis,
    doctrinam ad audientium sensum, atque intelligentiam accomodari oportere,
    vt cùm eorum animos, qui exercitatos sensus habent, 
    spirituali cibo expleuerint \Dots.}
  \end{Footnote}
\end{quote}

Thus the catechism teaches that for faith to come through hearing, both the teacher and the listener had to be involved.
This concept was echoed in Spanish vernacular expositions of the catechism, such as a 1589 \worktitle{Catechism of the Mysteries of the Faith} by Antonio de Azevedo.%
\autocite{Azevedo:Catecismo}
This catechism begins with an image taken from Pliny, 
which depicts a religious teacher with a pupil at his feet and a harp in his hand.
The teacher is shown, Azevedo says, \quoted{with a musical instrument which gives pleasure to the ear, so that we should understand that Faith enters through the ear \add{\foreign{oído}}, as St.\ Paul says, and that the disciple should be like a child, simple, without malice or duplicity, without knowing even how to respond or argue, but only how to listen and learn.}%
\begin{Footnote}
  \Autocite[f.~1b]{Azevedo:Catecismo}:
  \quotedsp{Para pintar los Romanos la Fe, lo primero que hizieron templo y altar \Dots{} Numa pompilio \Dots{} puso vn idolo:
  de forma de vn viejo cano, que tenia vna harpa en la mano, i estava enseñando vn niño echado a sus pies.
  En esta figura o geroglifica esta encerrada mucha filosofia, i aun cristiana.
  En el templo i ara denota, que la fe a de ser firme i fixa, no movediça, ni flaca, que a cada ayre de novedad se mueva
  I tambien que a menester maestros que la enseñen, i dicipulos que la oygan.
  I, que el maestro a de ser anciano, i maduro en edad y bondad.
  Porque la dotrina es grave, antigua, i de tomo i sustancia. 
  No nueua, ni de pocos años sino antigue dende los Apostoles, i con instrumento musico que da gusto al oido.
  Paraque entendamos, que la Fe entra por el oido; como dize S.\ Pablo.
  I que el dicipulo sea como niño, sencillo, sin malicia ni doblez, sin saber ni replicar, ni arguir, mas de solo oir, y deprender.
  En lo qual nos dibuxa galanamente, qual a de ser el oiente de la Fe.}
\end{Footnote}

The teacher's task according to Azevedo, then, is not only to make the faith heard, but to make it \quoted{appeal to the ear}, just as he says music does.
But such teaching was limited by the sensory capacity of each listener, and therefore the Roman Catechism argues that the task of listening requires the sense to be properly trained.
Thus pastors had to accommodate the ear even while training it.

% word, text, person, community, performance
% challenge of faithful hearing, connection of faith and ethics and community; types of faith (informe, formata); music as both way to increase appeal to hearing as well as form of building social relationships; but cultural conditioning and subjective experience are problems for both endeavors.
% XXX

The theology of the catechism might suggest that the value of music in propagating faith would come from the medium's ability to make the faith appeal to the ears of listeners.
Villancicos, as an auditory medium based on vernacular poetry, would seem like an ideal vehicle for this project.
If teaching should appeal to the ear \emph{like} music, as Azevedo says, then combining teaching with actual music would appeal all the more.
At the same time, the challenge of training the sense of hearing would seem to be multiplied with music, since a listener must learn to understand not only spoken language but musical structures as well.

\subsection{Experiencing Spiritual Truth through Music}

This tension between accommodating the ear and training it may be seen in the explanations of music's power by Athanasius Kircher.
In his 1650 compendium \worktitle{Musurgia univeralis}, which was disseminated throughout the Hispanic world, the Jesuit writer argues that music adds an even greater power than that of preaching alone:
\begin{quote}
  If one should wish by the power of God to move a devout person to heavenly things, so that the listener is given over in meditation in otherworldly affects and raptured in his mind,
  and if one should take some notable theme expressed in words,
  which would recall to the hearer's memory the sweetness of heavenly things and their mildness,
  and then fittingly adapt that verbal theme through cadences and intervals in the Dorian mode,
  the \emph{the listener could experience that what was said is actually true}, 
  since through harmonic sweetness he could be transported beyond himself by those heavenly things,
  carried away by joy to where those things are true.%
  \begin{Footnote}
    \Autocite[bk.\~7, 550 (emphasis added)]{Kircher:Musurgia}:
    \quotedla{Si quis Deo deuotum hominum rerumque c{\oe}lestium, meditationi deditum in exoticos affectus raptusque mentis commouere vellet is supra insigne aliquod verborum thema, quod rerum c{\ae}lestium dulcedium, \& suauitatem auditori in memoriam reuocaret, modulo dorio per clausulas interuallaque aptè adaptet, \& experietur quod dixi verum esse, statim extra se factos dulcedine harmonica eò, vbi vera sunt gaudi rapi.}
  \end{Footnote}
\end{quote}

Kircher's depiction of music's power goes well beyond the Jesuit formula of \quoted{teaching, pleasing, and persuading}.\citXXX[for Jesuit formula: Bailey?]
Music not only makes the teaching of doctrinal truth appealing and persuasive; it actually transforms listeners through affective experience.
For Kircher, music links the objective truth with subjective experience through the unique ways that music affects the human body.\citXXX[Kircher discussion of affects, passions]
Through principles of sympathetic vibration, the humoral-affective properties of music could be transferred from composer and performer to listeners.

But Kircher never fully resolves the problem of subjectivity in this process. 
He does acknowledge that temperament and climate lead to differences in perception, but he does not explain how a common listener could acquire the necessary capacity to hear the musical structures he describes and derive the intended benefits.

% ****************
\subsection{The Danger of Subjective Experience in Faith}

The capacity to listen faithfully, and therefore music's power to make faith appeal to hearing, would then be limited by cultural conditioning as well as by personal subjectivity.
Reformation controversies had pushed Catholics into an increasingly negative and anxious attitude toward subjective religious experience.
Polemicists like Thomas More accused Martin Luther of turning his followers away from the trustworthy institutional Church with its objectively operating sacraments, leaving them with only a subjective experience as assurance of salvation.%
\autocite[\XXX, also More]{Schreiner:Certainty}
The Spanish Inquisition investigated Teresa of Ávila and other mystics who claimed authority only on the basis of spiritual experiences.%
\autocites[\XXX]{Ahlgren:TeresaPolitics}{Francisca:Inquisition}
Teresa's student John of the Cross taught Carmelite contemplatives to pursue union with God by weaning themselves of sensory experiences in the \quoted{dark night of the soul}.
Similarly, Ignatius of Loyola provided Jesuits with \worktitle{Spiritual Exercises} to discern the validity their religious sensations.

In such a climate, music's power over the sense and affects might be used for the purposes of cultivating faith, but this power could also be dangerous.
It had to be carefully controlled to mitigate the dangers of individual subjectivity.

% *********************
\subsection{The Need for Cultural Conditioning in Hearing}

On the cultural side, Catholics in the age of exploration struggled to determine which parts of Christian religion they could adapt to other cultures.
As Ines Zúpanov describes Jesuit intercultural encounters in India in her book \worktitle{Missionary Tropics}, the \quoted{tropics} represented both a geographic zone and a process of inevitable \quoted{turning} or cultural transformation.\citXXX[Zupanov]
Some missionaries like the Jesuits in Japan and Brazil actively sought to accommodate local customs and music; but everywhere that missionaries brought Christian faith, the process of cultural translation inevitably transformed it into something neither they nor their converts could necessarily predict.\citXXX[Japanese mission, Bailey, Castagna Brazil]
Missionaries in Mexico complained that although the Mexica claimed they were singing Christian songs in Nahuatl, there were no linguists sufficiently skilled to verify the songs' orthodoxy.
The church could proclaim \soCalled{the Faith}, but how could leaders know that people heard what they intended?

The cultural problem of religious ear training is plainly stated in a Latin dialogue published by the leaders of the Jesuit mission in Japan in 1590.
The missionaries had taken four Japanese noble youths on a grand tour of Spain and Italy between 1582 and 1590.%
\autocite{Sande:DeMissioneLegatorum}
The boys practiced and performed music throughout their trip, and heard the grandest ensembles of Catholic Europe.
In the dialogue, Michael, on of the so-called ambassadors, tells his friend who stayed home about European music:
\begin{quote}
  You must remember \Dots{} how much we are swayed by longstanding custom, or on the other side, by unfamiliarity and inexperience, and the same is true of singing. 
  You are not yet used to European singing and harmony, so you do not yet appreciate how sweet and pleasant it is, whereas we, since we are now accustomed to listening to it, feel that there is nothing more agreeable to the ear.
  \begin{Footnote}
    \autocite[109--110]{Sande:DeMissioneLegatorum}, translation from \autocite[155-156]{Massarella:JapaneseTravellers}, emphasis added.
    \end{Footnote}
\end{quote}
He contrasts the marvels of European polyphony and instruments with the simplicity of monophonic Japanese music.
The friend responds,
\begin{quote}
  I am sure all these things which you say are true; for the variety of the instruments and the books which you have brought back, as well as the singing and the modulation of harmony, testify to a remarkable artistic system.
  Nor do I doubt that our normal expectations in listening to singing are an impediment when it comes to appreciating the beauties of European harmony.%
  \footnote{Ibid.}
\end{quote}

% ***********************
\subsection{Obstacles to Faith and Mistrust of Hearing}

How, then, could the Church overcome such an impediment?
What if a person simply lacked the proper disposition to hear the Word with faith?

In a Corpus Christi mystery play (\term{auto sacramental}) by Spanish court poet Pedro Calderón de la Barca, the figure of \term{Judaísmo} becomes a vivid representation of the incapacity to acquire faith, \term{despite} the sense of hearing.
Performed in 1634 to inaugurate Philip IV's new palace, the Buen Retiro, Calderón's \worktitle{El nuevo palacio del Retiro} centers on Judaism's incapacity for faith.%
\autocite{Calderon:Retiro}
Judaism is forcefully excluded from the festivities celebrated within the play, which culminate with the consecration of the Eucharist.
Instead Judaism stands to the side and asks the character Faith to explain each event to him.
But despite trying to connect Faith's message with what he knows of the Hebrew Sciptures, Judaism cannot accept any of these explanations.
In fact he is unable to believe what Faith has said, because, as he says in an increasingly embittered refrain, \quoted{I have listened to Faith without Faith}.

% *******
\begin{expoem}
  \caption{Calderón, \worktitle{El nuevo palacio del Retiro}, \textlinenums{1303--1336}: Judaism rejects faith}
  \label{expoem:Calderon-Retiro-Judaismo}
  \inputexpoem{Calderon-Retiro-Judaismo}
\end{expoem}
% *******

Judaism's eloquent confession of unbelief is immediately drowned out by music, as clarion fanfares announce a royal procession.
For Calderón's listeners, who had been taught to regard Jews as the embodiment of willful unbelief and worse, the entry of the musicians would clear away the acrid sound of Judaisms's speech.
The feeling of doubt about the senses, however, pervades the entire play.

Much of the rest of the play dramatizes a contest of the senses, in which Hearing prevails---but only after confessing to his own incertitude.
Each personified sense competes for a laurel prize awarded by Faith (\expoemref{expoem:Calderon-Retiro-Hearing}).
Each sense in turn boasts of his powers, but Faith rejects each one.
Hearing is the last sense to present himself, and in contrast to the other senses, he speaks of his weakness, and how easily he can be fooled by echos or feigned voices.
Since he cannot trust his own powers, he must rely on faith.
In response, Faith crowns Hearing precisely because of his \foreign{desconfianza}---meaning lack of confidence, mistrust, and humility.

% *****
\begin{expoem}
  \caption{Calderón, \worktitle{El nuevo palacio}, \textlinenums{593--602}: Faith crowns Hearing}
  \label{expoem:Calderon-Retiro-Hearing}
  \inputexpoem{Calderon-Retiro-Hearing}
\end{expoem}
% ******

What would it mean, then, for hearing to be the favored sense of faith not just because of its humility, but because of its actual weakness?
How could the auditory art forms of music and poetry in villancicos, then, provide a medium for propagating the faith, if hearing was so easily deceived?

% ****************************************
\section{Contests of the Senses in Villancicos}

Villancicos manifest many of these same theological preoccupations and anxieties.
They do not answer the questions we have raised---in fact some of them intensify the problems; but they do provide evidence for a broad public discourse about sensation and faith even as these pieces of music were themselves objects to be sensed and believed.

The contest of the senses in Calderón's play is echoed in two villancicos by successive chapelmasters at Segovia Cathedral in the later seventeenth century.
Miguel de Irízar was born in 1634 and served at Segovia from 1671 until his death in 1684.
Jerónimo de Carrión, born in 1660, followed after Irízar in 1684 and died in 1721.
Both chapelmasters set variants of the same villancico poem, \worktitle{Si los sentidos queja forman del Pan divino}, which was attributed to Zaragoza poet Vicente Sánchez in the posthumous publication of his works in 1688.

The Segovia Cathedral archive preserves manuscript performing parts for both settings along with Irízar's draft score for his version.
Irízar corresponded with a peninsular network of musicians, often about exchanging villancico poetry; and he drafted his scores on the backsides and in the margins of these letters in makeshift notebooks.
\begin{Footnote}
  The performing parts for Irízar's setting are in \signature{E-SE}{5/32}; those for Carrión's are in \signature{E-SE}{28/25}.
\end{Footnote}

The estribillo (\expoemref{expoem:Si_los_sentidos-Sanchez-estribillo}) invites hearers to imagine the senses \quoted{filing a complaint} against the bread of the Eucharist because \quoted{what they sense is not by faith consented}---playing on \foreign{sentido}, the word for sense.
Each of the coplas treats a different sense (\expoemref{expoem:Si_los_sentidos-Sanchez-coplas}), following nearly the same order as in Calderón's play: 
Sight comes first, followed by Touch; next are Taste and Smell, and Hearing comes last (\tableref{table:senses-order}).

% **************
\begin{expoem}
  \caption{\worktitle{Si los sentidos queja forman del Pan divino}, \shortcite[171--172]{Sanchez:LiraPoetica}, estribillo and coplas 1--2}
  \label{expoem:Si_los_sentidos-Sanchez-estribillo}
  \inputexpoem{Si_los_sentidos-Sanchez-estribillo}
\end{expoem}
% *************
% **************
\begin{expoem}
  \caption{\worktitle{Si los sentidos queja forman del Pan divino}, conclusion of coplas}
  \label{expoem:Si_los_sentidos-Sanchez-coplas}
  \inputexpoem{Si_los_sentidos-Sanchez-coplas}
\end{expoem}
% *************


Spanish theologians always presented Vision as the first and highest of the five exterior senses, as can be seen in treatises from seminary and convent libraries in Spain and Mexico.
This tradition draws on Aristotle as interpreted by Thomas Aquinas.
The widely read Dominican theologian Fray Luis de Granada, in his \worktitle{Introduction to the Creed} of 1583, summarizes the common physiological model of sensation and perception.
\tableref{table:senses-fray-luis} shows how Fray Luis explains the relationship of the five exterior senses to the interior senses, including the affective faculty, in which the sensory stimuli interacted with the balance of bodily humors.
The act of sensation involved the entire body and soul, in a pre-Cartesian wholistic model---but the external senses differed in how they connected the external world to the internal faculties and passions.

% ************
\begin{table}
  \caption{The exterior senses: Order of presentation in versions of \worktitle{Si los sentidos}, correlated with Calderón and Veracruce}
  \label{table:senses-order}
  \inputtable{senses-order}
\end{table}
% ***********


% ************
\begin{table}
  \caption{The senses and faculties of the sensible soul (\term{ánima sensitiva}), according to Fray Luis de Granada}
  \label{table:senses-fray-luis}
  \inputtable{senses-fray-luis}
\end{table}
% ***********

The hierarchy of the senses was based on the degree of mediation between the object of sensation and the person sensing.
The most base sense was taste, because the person actually had to physically consume the object of sensation.
The most powerful sense was sight, since it enabled a person to perceive objects a great distance away without any direct contact.
Hearing stood out from the other senses because for it alone, the object of perception was not identical with the thing sensed.
As Calderón's character Hearing says, \quoted{Sight sees, without doubting/ what she sees; Smell smells/ what he smells; Touch touches/ what he touches, and Taste tastes/ what he tastes, since the object/ is proximate \add{immediate} to the action}.%
\autocite[\textlinenums{577--582}]{Calderon:Retiro}
But Hearing hears a person's voice, not the person directly, as Calderón's text continues: \quoted{But what Hearing hears/ is only a fleeting echo,/ born of a distant voice/ without a known object}.%
\autocite[\textlinenums{583--586}]{Calderon:Retiro}
While this feature of hearing may have made it \quoted{easily deceived}, it also gave this sense a unique capability in spiritual matters, where the object of perception was not immediately sensible at all.

The poetic contests of the senses thus rearrange the traditional scholastic hierarchy by putting Hearing at the end for a dramatic climax.
Sight comes first, but Hearing, the underdog competitor, triumphs at the last.
In the Sánchez villancico, each of the coplas highlights the failure of one of the senses to rightly perceive the sacrament.
Sánchez presents hearing in the last copla, through the conceit of music.
The senses are \quoted{five instruments} like a musical consort, which must be \quoted{tempered} by faith.
Without Faith, sight is actually blind, and touch, taste, and smell are fooled; but when properly attuned by Faith, the sense can be harmonized into a pleasing concord.
Here Faith is not the object of sensation, but the subject, who delights in hearing the music of properly tuned senses.

% ***************************************************
\subsection{Musical Settings by Irízar and Carrión}

The two surviving settings of \worktitle{Si los sentidos} by Irízar and Carrión stage this contest of the senses in sound.
Their contrasting styles invite different types of involvement from listeners.
In the earlier setting, for Corpus Christi 1674 at Segovia Cathedral, Miguel de Irízar creates a musical competition in grand festival style by pitting his two four-voice choirs against each other in polychoral dialogue (\exmusicref{exmusic:Irizar-Si_los_sentidos}).%
\begin{Footnote}
  \signature{E-SE}{5/32} is the manuscript performing parts in a copyist's hand, while \signature{E-SE}{18/19} is the draft score in Irízar's hand, and includes the heading, \quoted{Fiesta del SSantissimo de este año del 1674}.
\end{Footnote}
Like a film editor creating a fight scene, Irízar builds intensity by cutting the text into shorter phrases to be tossed back and forth between the two choirs: \foreign{no se den por sentidos} becomes \foreign{no se den} and then \foreign{no, no}. 

% *************
\begin{exmusic}
  \inputexmusic{Irizar-Si_los_sentidos}
  \caption{\worktitle{Si los sentidos queja forman del pan divino}, Miguel de Irízar (\signature{E-SE}{18/19, 5/32}) \XXX examples}
  \label{exmusic:Irizar-Si_los_sentidos}
\end{exmusic}
% *************

Irízar creates a steadily increasing sense of excitement through shifts of rhythmic motion and style.
The setting of the opening phrase suggests a tone of hushed awe: the voices sing low in their registers, with a slow harmonic rhythm, and pause for prominent breaths.
In the next section Irízar has the ensemble switch to ternary meter and increases the rate of harmonic motion.
The sense of antagonism is heightened when one choir interrupts the other with exclamations of \foreign{no} on normally weak beats.
When Irízar returns to duple meter, the voices move in smaller note values and exchange shorter phrases, so that the tempo feels faster, as Irízar adds more offbeat accents and syncopations.
The estribillo builds to a climactic \term{peroratio} with the voices breaking into imitative texture in descending melodic lines.

The distinguishing stylistic characteristics of the setting suggest that Irízar is evoking a musical battle topic, a style one may find in \term{batallas} for organ as well as other villancicos on military themes.%
\begin{Footnote}
  [Cite keyboard 2ry source on batallas]\XXX
  Keyboard examples include the \term{batallas} in Martín y Coll's \worktitle{Huerto ameno de flores de música}\XXX, and in [Portuguese collection]\XXX and [Bruna works]\XXX.
  Another villancico in this style is Antonio de Salazar's \worktitle{Al campo, a la batalla} (\signature{MEX-Mc}{A28}).
\end{Footnote}
Battle pieces typically feature a slow, peaceful introduction followed by sections in contrasting meters and styles and a texture of dialogue between opposing groups (as in between high and low registers on the keyboard). 
Typical of the style is the reiteration of chord voicings in what we could call root position, with the bass moving by fourths and fifths, and the 3-3-2 syncopations on \foreign{no se den por sentidos los sentidos}.

Irízar's villancico seems to speak to a large crowd through grand, unsubtle gestures and sharp contrasts of bright colors.
By contrast, Jerónimo de Carrión's later setting of the same poem (\exmusicref{exmusic:Carrion-Si_los_sentidos}) invites a more personal reflection.
Carrión was capable of the festival style, but this setting fits more in the subgenre of \term{tono divino}, a continuo song used in more intimate settings like Eucharistic devotion.
The style similar to the \soCalled{high Baroque} music of contemporary Italy, with a tonal harmonic language, a running bass part in the accompaniment, and a single affective manner throughout.

% ************
\begin{exmusic}
  \inputexmusic{Carrion-Si_los_sentidos}
  \caption{\worktitle{Si los sentidos queja forman del pan divino}, Jerónimo de Carrión (\signature{E-SE}{28/25})}
  \label{exmusic:Carrion-Si_los_sentidos}
\end{exmusic}
% ************

The dialogue and rivalry of the poetic text happens now not through polychoral effects but through motivic exchanges between voice and accompaniment.
Instead of metrical contrasts from one section to the next, Carrión creates rhythmic contrasts between simultaneous voices.
Carrión dramatizes \foreign{queja} with a metrical disagreement between the two voices (normal ternary motion versus the voice's sesquialtera).
The descending pattern of leaps for \foreign{porque lo que ellos sienten} perhaps suggests the confusion and tumult of the senses, and it creates a certain amount of rhythmic confusion as it moves between voices.
Carrión creates a climax through a canon between soloist and accompaniment that leads the singer to the top of his register.
The upward leaps in the last line on \foreign{no se den} contrast with the downward leaping motive of the opening.

The similarities between these two settings of \worktitle{Si los sentidos} demonstrate the persistence of concerns about the hearing's role in faith.
Meanwhile the differences between versions reflect changing styles not only of composition but of devotional practice.
Irízar and Carrión take a verbal discourse on sensation and faith, in which music is the paradigm of something that pleases the ear, and bring it to life through actual music.
Thus the pieces seem designed to teach listeners how to hear music even as they are listening---they accomodate hearing while training it, as the catechism says.


% ******************
\section{Impaired Hearers, Incompetent Teachers: \quoted{Villancicos of the Deaf}}

The final section presents two \term{villancicos de sordos}---villancicos of the deaf---which dramatize the limitations of hearing, and poke fun at the difficulty some religious teachers faced in making faith appeal to this sense.
Similar pieces from both sides of the Atlantic use hearing disability as a symbol of spiritual deafness: the first is by Juan Gutiérrez de Padilla from Puebla Cathedral in New Spain, and the other is by Matías Ruíz for the Royal Chapel in Madrid.

The villancico by Padilla is labeled \foreign{sordo} in the partbooks: it creates a comic dialogue between a religious teacher and a hard-of-hearing man named Toribio.
The villancico, which begins \worktitle{Óyeme, Toribio} and is labeled as a \quoted{Dúo con bajón}, was performed in Matins for Christmas at Puebla Cathedral in 1651.%
\footnote{\signature{MEX-Pc}{Leg. 1/2}.}
It is part of Padilla's earliest surviving Christmas cycle for the new cathedral, which had been consecrated in 1649.
Though two key partbooks are missing, including the Tenor I part who played the deaf man, but the dialogue can be reconstructed because the lyrics of the deaf man's part were written in the surviving bass part.

% ***********
\begin{expoem}
  \caption{\worktitle{Óyeme, Toribio (El sordo)}, from setting by Juan Gutiérrez de Padilla, Puebla, 1651 (\signature{MEX-Pc}{Leg. 1/2}), excerpt}
  \label{expoem:Oyeme_Toribio-Padilla}
  \inputexpoem{Oyeme_Toribio-Padilla}
\end{expoem}
% ************

The friar's attempts to communicate with the \soCalled{deaf} man fail, and this prompts the chorus to warn the congregation against spiritual deafness.
Padilla dramatizes the two characters' unsuccessful attempts to communicate through disjunctions of rhythm and mode (\exmusicref{exmusic:Padilla-Sordo-intro}).
Rhythmically, Padilla gives each singer a distinct \quoted{accent} in music: the friar's musical speech is more fluent and sophisticated, while the deaf man's speech is halting and clumsy, such as his questions \quoted{Eh? eh?} on an offbeat figure.


Padilla illustrates the men's disagreement by having them fail to concur on where to cadence.
Given the one-flat \term{cantus mollis} signature, the cadence points articulated by the extant bass part, and the final on F, we may categorize this piece as mode 11 or 12.
The friar appropriately sings an opening phrase that surely would have cadenced on F; but the deaf man responds with a phrase that cadences on C.
At one point the friar moves to a cadence on A, but the deaf man, responding that he can't hear out of that ear, cadences on D.
The friar says he will try the other ear, but no sooner has the friar moved to D, than Toribio, saying, \quoted{Out of that ear I hear even less!} moves to a cadence on C.
This pushes the friar over the edge.
He bursts out, \quoted{You are a sheer idiot!}---mimicking the deaf man's halting short-long rhythms, and returning to his own final of F.

After the dialogue in the introducción, the Altus I, who has been representing the friar, seems to turn away from the dramatic scenario and address the congregation as a preacher: \quoted{The laughter of the dawn will turn to sobs}, he says---referring to the Virgin Mary---when, having given birth to the Word Incarnate, \quoted{her eyes see deaf men}.
Padilla sets the final phrase about deaf men, with ten notes in black mensural notation that make a rhetorical \term{catabasis} as they descend in a leaping melodic sequence (\figureref{figure:Padilla-Sordo-MS-estribillo}).

When the rest of Chorus I joins in for the responsión, their repeated dotted rhythm suggests vivacious laughter and comic offbeat sobs on \foreign{sollozos}.
The catabasis figure is passed through all the voices in imitation, leading to a harmonic catabasis when the Tiple I adds E\fl{}---shifting further away from the \soCalled{natural} into the \soCalled{weak} realm of flats.\citXXX[chafe?]
The heavy syncopation in each voice creates rhythmic confusion that is not sorted out until the final cadential flourish on F, validating the friar's initial choice of mode.

% *******************
\begin{exmusic}
  \inputexmusic{Padilla-Sordo-intro}
  \caption{Padilla, \worktitle{Óyeme, Toribio (El sordo)}, introducción, \measurenums{1--25}, extant parts (missing Tenor I, Bassus I)}
  \label{exmusic:Padilla-Sordo-intro}
\end{exmusic}
% *******************
% *******************
\begin{figure}
  \includeTallFigure{Padilla-Sordo-MS-estribillo}
  \caption{Padilla, \worktitle{Óyeme, Toribio (El sordo)}, Altus I manuscript partbook, introducción and estribillo}
  \label{figure:Padilla-Sordo-MS-estribillo}
\end{figure}
% *******************

Through a characteristic mixture of sophisticated musical technique, high-minded theology, and low caricature, Padilla belittles actual deaf people as deficient, undignified, and deserving of laughter.
Padilla's ensemble made their caricature at the same time that some Spanish churchmen, like Juan Pablo Bonet, were engaged in actually ministering to deaf people.%
\begin{Footnote}
  Juan Pablo Bonet, \worktitle{Reduction de las letras y Arte para enseñar a ablar a los mudos} (\XXX, 1620), cited and discussed in \autocite{Plann:DeafEducationSpain}.
\end{Footnote}
But the \term{villancicos de sordos} by Padilla and Ruíz are more interested in exploiting the deaf to amuse and edify the hearing.

Actual deafness is used here with the main goal of pointing to spiritual deafness: the \quoted{deaf men} that will make Mary weep are all people whose ears have been stopped by sin and cannot hear the divine Word of Christ with faith.
In his 1610 Spanish dictionary, Sebastián de Covarrubias defined the \term{sordo} as \quoted{he who does not hear}, not \quoted{who cannot hear}.
He adds, \quoted{There is no worse kind of deaf man than he who is unwilling to hear}.%
\begin{Footnote}
  \autocite[\sv{sordo}]{Covarrubias:Tesoro}:
  \quoted{SORDO, Lat. surdus, el que no oye. 
  No ay peor sordo que le que no quiere oyr.}
\end{Footnote}
In this view, if hearing is the paradigm of faith, then deafness is its opposite.
While Calderón's Judaism heard Faith without faith, the deaf men in villancicos cannot even hear Faith to begin with.

% **********
\subsection{Matías Ruíz (Madrid, 1671)}

The \worktitle{Villancico de los sordos} by Matías Ruiz extends its parody to the catechist as well.%
\footnote{\signature{E-E}{Mús. 83-12}.}
Ruiz was chapelmaster at the Real Convento de la Encarnación in Madrid, and the poetry imprint survives from what must have been the first performance there at Christmas, 1671.%
\autocite{1671-Madrid-Enc-Nav}
Here the \term{sordo} is a hard-of-hearing man, \quoted{very learned in humane letters}---a doddering old university professor, or perhaps a street sage.
The piece mocks his impairment while contrasting true faith with the book learning of this would-be humanistic scholar.
But the biggest laughs come at the expense of the friar, as the deaf man mishears his rote teaching formulas in increasingly absurd ways (\expoemref{expoem:Pues_la_fiesta-Ruiz-estribillo}).

% ****************
\begin{expoem}
  \caption{\worktitle{Pues la fiesta del Niño es (Villancico de los sordos)}, from setting by Matías Ruiz, Madrid, 1671 (\signature{E-E}{Mús. 83-12}, \signature{E-Mn}{R/34989/1}), estribillo}
  \label{expoem:Pues_la_fiesta-Ruiz-estribillo}
  \inputexpoem{Pues_la_fiesta-Ruiz-estribillo}
\end{expoem}
% ***************

The piece begins with soloist and chorus gleefully crying \quoted{On with the deaf man!}  rather like a bunch of high-school bullies, telling everyone to speak up so he can hear.
When the catechist and the \term{sordo} enter, Ruiz gives the two characters phrases that contrast melodically, harmonically, and rhythmically, to illustrate their inability to understand each other.
The deaf man's musical speech is abrupt, uncouth, and loud, fitting with the friar's mockery of the deaf man's unmodulated voice.
The deaf man bursts on the scene with a scale from the top of his register to the bottom (F\octave{4} to G\octave{3}).
The descent across vocal \term{passaggi} would encourages the singer to bawl the phrases in a coarse tone of voice.

% ***********
\begin{exmusic}
  \inputexmusic{Ruiz-Sordos-dialogue}
  \caption{Ruiz, \worktitle{Pues la fiesta del niño es (Villancico de los sordos)} (\signature{E-E}{Mús. 83-12}), estribillo, \measurenums{41--55}}
  \label{exmusic:Ruiz-Sordos-dialogue}
\end{exmusic}
% ***********


Here, though, the deaf man has a lesson of his own to teach.
He may not be able to hear well but has come with love to adore the Christ-child. 
Acting as a kind of holy fool, and echoing Covarrubias's definition of deafness, he reminds everyone within the sound of his voice that the truly deaf are \quoted{those who neither listen nor understand the sound}.

In the parodied catechism lesson presened in the coplas, (\expoemref{expoem:Pues_la_fiesta-Ruiz-coplas-1}) the friar quizzes his pupil on key doctrines of Christmas, classic topics in both scholastic and pastoral literature:
Tell, \term{sordo}, he asks, how did God fulfill his word to the prophet-king David?
What motivated Christ to become incarnate?
But the deaf man mishears every statement: he mistakes \foreign{sordo} for \gloss{gordo}{chubby}, and \gloss{profeta}{prophet} as \gloss{estafeta}{mailman}.

% ****************
\begin{expoem}
  \caption{\worktitle{Pues la fiesta del Niño es (Villancico de los sordos)}, from setting by Matías Ruiz, coplas 1--5}
  \label{expoem:Pues_la_fiesta-Ruiz-coplas-1}
  \inputexpoem{Pues_la_fiesta-Ruiz-coplas-1}
\end{expoem}
% ***************

% ****************
\begin{expoem}
  \caption{\worktitle{Pues la fiesta del Niño es (Villancico de los sordos)}, from setting by Matías Ruiz, conclusion of coplas}
  \label{expoem:Pues_la_fiesta-Ruiz-coplas-2}
  \inputexpoem{Pues_la_fiesta-Ruiz-coplas-2}
\end{expoem}
% ***************

His supposed learning in the humanities leads only to confusion.
When the friarn lauds the \gloss{bailes}{dances} of Christmas, and says \gloss{el portal es nuestro alivio}{the stable is our remedy}, the deaf man thinks he is citing \foreign{Tito Libio}.
The humanist is puzzled: he has read the Classical historian Livy, he says, but Livy doesn't say anything about \gloss{frailes}{friars}.

Hearing that the child Jesus is shivering with cold, the deaf man suggests he drink hot \foreign{chocolate}.
The friar reassures him that \quoted{the Queen}---the Virgin Mary---is keeping the child bundled, such that he glows with warmth (\foreign{arde}).
The deaf man now seems to feel that at last he has figured out what they are talking about, and sums up with satisfaction, \gloss{Esta es, por la mañana y tarde, la Reina de las bebidas}{Chocolate is, morning and evening, the Queen of beverages}.

We can imagine that Ruiz's deaf man would invite the sympathies of listeners.
He is an earthy, common character, focused on material comforts, keenly aware as many older men are of the chill, and doting tenderly on the infant.
The deaf man's bumbling but endearing staments contrast strikingly with the friar's abstract theology and clichéd poetic language.
The text even reminds listeners of the deaf man's response to mockery.
He cannot hear the nine choirs of Christmas angels, so he asks them to sing out loudly, as long as they don't say anything bad about him.

Ruiz's characters present a contrast of types of learning: the churchman who repeats the same teaching points in every catechism class, versus an ersatz humanist who has read Livy and perhaps Ovid but may not understand them at all.
Ruiz's hard-of-hearing humanities scholar demonstrates a central tenet of Reformation-era Catholicism: that Classical learning alone is not enough to understand Christianity.
Even so, the friar's inability to teach this man demonstrates yet again the pervasive uncertainty about how faith could be made to appeal to hearing.

Ruiz's \worktitle{Villancico de los sordos}, like the other pieces we have discussed, exalts hearing music as a means to faith---one which the deaf cannot understand.
This is because the word \foreign{son} in the epigrammatic conclusion of the estribillo could mean not only sound but was also a term for a type of dance or song, according to the 1610 dictionary of Covarrubias.
The villancico is built on a distinctive harmonic and rhythmic pattern of alternating ternary and sesquialtera groupings, which are especially clear on the phrase \foreign{los que no escuchan ni entienden el son}.
This pattern bears a close resemblance to dance forms known as \term{son} today (\exmusicref{exmusic:Ruiz-Sordos-son})---most obviously, to the Mexican \term{huarache} familiar from Leonard Bernstein's \quoted{America}.
Even if the reference to \term{son} is not this specific, the music certainly evokes the feeling of a social dance whose circle the deaf man is unable to join.


% **************
\begin{exmusic}
  \inputexmusic{Ruiz-Sordos-son}
  \caption{Ruiz, \worktitle{Villancico de los sordos}, conclusion of estribillo, \measurenums{76--83}: Possible evocation of \term{son} song/dance style}
  \label{exmusic:Ruiz-Sordos-son}
\end{exmusic}
% *************

The creators of villancicos of the deaf could certainly have imagined that this music was contributing to the goal of spiritual ear training.
There may be a higher theological point here about spiritual deafness, but both \term{sordo} villancicos make the point at the expense of real people with disabilities.
Whether the music really accomplished that goal, or just provided amusement and reinforced cultural stereotypes for commoners, is a question that haunts the whole repertoire.
In a pattern typical of this genre (and especially vexing in the \soCalled{ethnic} villancicos), someone on the margins of society is, through musical representation, welcomed to Christ's manger, but the way he is represented actually emphasizes his exclusion from the community.
At the same time they also demonstrate a surprising levity about the church and its incompetent teachers.


% *************
\section{Failures of Faithful Hearing}

Critiquing both the poor level of theological knowledge among the lay people, and the low quality of teaching among the clergy, were both characteristic Tridentine postures.
Antonio de Azevedo describes real-life scenes of failed catechesis:
\begin{quotation}
  Some will say that the doctrine of the gospel has already been taught everywhere or almost everywhere (I am speaking of our Spain), and we concede;
  but there are so many parts that so badly lack anyone who could teach matters of faith,
  that indeed it is a shame to see it happen in many parts of Spain, and particularly in the  mountains,
  where there are many so unlettered \add{\foreign{bozales}} in the matters of faith,
  that if you would ask them, how many are the persons of the Holy Trinity, some would say that they are seven, and others, fifteen; and others say about twenty---of this I am a good witness.

  And a principal friar of my order, I've heard that once he was asking a woman how many \add{persons in the Trinity} there were, and she said, \quoted{Fifteen}.
  And he said, \quoted{\foreign{Ay}, is that really your answer?}
  And then she wanted to correct herself, and she said, \quoted{\foreign{Ay Señor}, I think I was wrong---I'll say there are five hundred}.%
  \begin{Footnote}
    \Autocite[26]{Azevedo:Catecismo}:
    \quotedsp{Diran, o que ya ay dotrina del Euangelio en todas partes, o casi todas (hablo de nuestra España) concedamoslo: 
    Pero ay tanta falta en muchas, de quien enseñe las cosas de la fe; que cierto que es lastima, verlo que en muchas partes de España, y particularmente en montañas passa: a do estan muchos tan boçales en las cosas de la fe, que si les preguntays, quantas son las personas de la Santissima Trinidad, vnos dizen que son siete otros que quinze; y otros veynte desatinos, de los quales yo soy buen testigo.
    Y a un frayle principal de mi orden le oy, que preguntando el a vna muger, quantas eran, que dixo ella que quinze, y diziendole el ay, y esso aueys de dezir? y ella se quiso emendar, y dixo ay Señor, digo mi culpa, digo que son quinientas}.
  \end{Footnote}
\end{quotation}

Azevedo sees no humor in this lack of religious knowledge; and he faults not the illiterate laypeople but the religious orders and clerics who have failed to teach the basics of faith in a plain way, as Azevedo himself endeavors to do in his book:
\begin{quote}
  It is a shame to see the ignorance that there is in many, in things of such importance.
  \Dots{} because even though the religious orders and those who preach do declare the gospel, 
  they do not explain the ABCs \add{\foreign{b, a, ba}} of Christianity;
  they do not want to deal with giving milk because this is the task of mothers, those lordly Curates or Orators, who are responsible for this task, and what I have described is their fault.%
  \begin{Footnote}
    \Autocite[27]{Azevedo:Catecismo}
    \quotedsp{Es lastima ver la ignorancia que ay en muchos, en cosas de tanta importancia: Y preguntados algunos qual de las tres personas encarno, el vno dize, que el Padre otros que el Espiritu Santo: y en muy buenos pueblos lo he oydo yo, hartas vezes con mis oydos; porque dado los religiosos y los que predican declaren el Euangelio, no tratan del b,a ba de cristiandad, no tratan de dar leche porque esse es officio de madres, de los señores Curas, o Retores, a cuyo cargo esta esso; y cuya culpa es lo dicho.}
  \end{Footnote}
\end{quote}

The friars of Padilla's and Ruiz's villancicos seem to fit with Azevedo's description of \quoted{lordly orators} who delight in lofty language, rather than motherly teachers who spell out the fundamentals of Christian faith.
Azevedo's critique, though, is motivated by the late-sixteenth-century Tridentine agenda of reform and education.
By the mid-seventeenth century, the function of Catholic religious art under developing \soCalled{Baroque} aesthetics shifted away from the kind of \quoted{plain} instruction Azevedo models, toward more ornate, learned, and often arcane forms of expression.

Few villancicos of the seventeenth century would satisfy Azevedo's call to teach the \quoted{\foreign{b, a, ba} of Christianity}.
Even the comic villancicos depend on learned plays of language and music, like the Classical references in Ruiz's poetic text, or the play on modal cadences and black notation of Padilla's music.
This means that the depictions of imperfect hearing in the villancicos of the deaf themselves depended on the attention of listeners with well-trained ears.
The fundamental Tridentine problem, of making faith appeal to hearing by both accomodating the senses and training them, remained a challenge for Catholics, both those who would teach through speech or song and those who would listen.

%%% Local Variables:
%%% mode: latex
%%% TeX-master: "../main"
%%% End:
