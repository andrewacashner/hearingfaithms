% Andrew Cashner -- Faith, Hearing, and the Power of Music
% Chapter 2 -- Making Faith Appeal to Hearing

% 2016-08-31    Revision for book begun

%*******************************************
\label{ch:faith-hearing}

\quoted{How are they to believe if they have not heard?} St.\ Paul asked, since \quoted{faith comes through hearing, and hearing, by the Word of Christ} (\scripture{Rom 10:16-17}).
Villancicos in the Spanish Empire of the seventeenth century made faith audible in a way that potentially appealed to the ears of elite and common people alike.
We have seen a variety of ways that villancicos represent music-making, and thus should be considered as musical discourses on music.
As these pieces were performed in and around Spanish churches, they proclaimed and embodied religious beliefs about the relationship between music and faith.
At the same time church leaders used the pieces themselves to cultivate faith.

How, then, did early modern Catholics understand the relationship between hearing and faith?
And how could the auditory art form of villancicos affect that relationship?
This chapter situates villancicos within the context of early modern Catholic discussions of faith and sensation.
As the theological sources discussed in the first section reveal, Hispanic Catholics experienced a certain anxiety about the role of individual sensory perception in acquiring and developing faith; and they harbored uncertainty about whether all people really had the capacity for faith.
Though many people acknowledged music's power to inspire faith, their concepts of music left it unclear exactly how people could develop their hearing faculties in order to derive a spiritual benefit from listening to music.

The second section interprets villancicos that explicitly address themes of faith and sensation, demonstrating that these pieces, too, reflect uncertainty and doubt about hearing's role in acquiring faith.
These pieces stage allegorical contests of the senses, represent sensory confusion (such as synesthesia), and represent characters whose impairments of hearing render them unable to understand religious teaching.

Understanding the theological environment in which villancicos were performed, and considering how villancicos in turn contributed to that environment, makes it clear that villancicos functioned as much more than vehicles for simplistic religious teaching.
Rather, villancicos provided listeners with opportunities to contemplate the challenges of faithful hearing.

%***********
\section{The Challenge of Making Faith Appeal to Hearing}

% The theological controversies and political upheavals of the sixteenth century had propelled the Roman Catholic Church like Columbus's ships into ever stranger worlds as its laborers strained to bring Catholic faith into foreign lands and to restore it where the Reformation had advanced.



% The Roman Catholic Church of the seventeenth century continued to move forward in its global struggle to promulgate Catholic faith, 


% Truth for early modern Catholics was objective; but faith was subjective.
% That an eternal, all-powerful, and loving God had created the world; that humans had used their free will to turn away from God and created a legacy of sin whose end was death and damnation; that the triune God had become incarnate through Jesus Christ, whose life, death, and resurrection restored humanity's relationship to God; that Christ had founded the Church as the instrument through which the Holy Spirit would bring salvation to the world, in particular through the sacraments---these truths were the foundation of Catholic belief, and their truth was absolute.
% But to become part of the Church, and thus be incorporated in Christ's body, and thus join the communion of the holy in everlasting union with God---this required faith at the individual level.
% Somehow a connection had to be made between the transcendent objects of faith and the experience of the individual Christian subject.

% Faith meant more than assenting to intellectual propositions, believing that certain things were true.
% The technical term for that basic sort of faith, formulated by St.\ Thomas Aquinas, was \term{fides informata} or unformed faith.\citXXX
% Fully formed faith, \term{fides formata}, was a virtue (\term{virtus}) or capacity that \quoted{worked through} the higher virtues of hope and love.
% True faith for Catholics meant a commitment of the whole person to live faithfully in communion with Christ through his body, the Church.\citXXX

% The role of subjective experience in faith was a central point of dispute in the reformations of the sixteenth century.
% Martin Luther redefined faith as placing one's trust in Christ alone for salvation.\citXXX[Schreiner]


A central document for understanding the faith of post-Reformation Catholics is the catechism produced \quoted{for the parishes} by authority of the Council of Trent.%
  \autocite{Catholic:Catechismus1614}
The bishops at the Council of Trent were not only responding to Protestantism; many of them sought to address the underlying problems that had allowed Protestantism to take hold in the first place.\citXXX{}
Chief among those problems was a lack of education both of clergy and laity.
Through an elegantly composed Latin catechism, church leaders hoped to educate their clergy so that the clergy could better instruct the parishioners under their care.

Vernacular expositions of the catechism, like those of Antonio de Azevedo, Juan Eusebio de Nieremberg, and Juan de Palafox y Mendoza, brought this teaching down to a more accessible level, still addressing a clerical reading audience, but often in a colloquial tone, with earthy illustrations and lengthy paraphrases of Scripture in Spanish.
These texts come alive when read aloud, and indeed their goal is to prepare pastors to teach unlettered disciples through words and voice.
These disciples might be Indians---in Spanish the term was used for indigenous peoples both of America and Asia---or Europeans, such as rural folk in mountain passes where Christianity had still not fully penetrated.
These books prepared teachers for the challenge of making faith appeal to hearing.

The official Roman Catechism grounds the Church's authority to preserve and teach \soCalled{the faith} in the theology of Christ's Incarnation.
The catechism teaches that God communicated his own nature to humanity by taking on human flesh in Christ, and therefore the true Word of God was not the Scripture or any body of doctrines, but rather Jesus Christ himself as the \term{logos} or \term{verbum} (\scripture{Jn 1:1}).
But while Truth ultimately would consist in knowing God in Christ, the catechism teaches that Christ founded the Church to be the means through which people would come to know him after his resurrection.
Christ appointed apostles, chief among them St.\ Peter, to be the custodians of the true faith from his time up until the present.

The Church as Christ's body was the community through which people came to faith in Christ and learned to live faithful lives after Christ's example.
Drawing on St.\ Paul's dictum that faith came through hearing, and hearing, by the Word of Christ, the catechism challenges the Church's ministers to find a way to make Christ the Word audible:
\begin{quote}
Since, therefore, faith is conceived by means of hearing, it is apparent, how necessary for acheiving eternal life are the works of the legitimate teachers and ministers of the faith. \Dots{}
Those who are called to this ministry should understand that in passing along the mysteries of faith and the precepts of life, \emph{they must accommodate the teaching to the sense of hearing and intelligence}, so that by these \add{mysteries and precepts}, \emph{those who possess a well-trained sense} should be filled up by spiritual food.%
  \begin{Footnote}
  \Autocite[2, 8--9 (emphasis added)]{Catholic:Catechismus1614}: 
  \quotedla{Cvm autem fides ex auditu concipiatur, perspicuum est, 
  quàm necessaria semper fuerit ad {\ae}ternam vitam consequendam 
  doctoris legitmi fidelis opera, 
  ac ministerium \Dots\ vt videlicet intelligerent, 
  que ad hoc ministerium vocati sunt, 
  ita in tradendis fidei mysteriis, ac vit{\ae} pr{\ae}ceptis,
  doctrinam ad audientium sensum, atque intelligentiam accomodari oportere,
  vt cùm eorum animos, qui exercitatos sensus habent, 
  spirituali cibo expleuerint \Dots.}
  \end{Footnote}
\end{quote}

As the emphasized phrases show, the Church taught that for faith to come through hearing, both the teacher and the listener had to be involved.
Antonio de Azevedo begins his vernacular introduction to the catechism with the notion that faith requires both wise teachers and attentive listeners.%
  \autocite{Azevedo:Catecismo}
For Azevedo, faith is epitomized in an ancient image he read about in Pliny, depicting \quoted{an elderly man sitting inside a temple, who had a harp in his hand, and who was teaching a boy who lay at his feet}.%
  \autocite[f.~1a]{Azevedo:Catecismo}
The temple, Azevedo explains, represents that faith should be \quoted{firm and fixed, and also that there must be masters who teach it, and disciples who listen to it; and that the master needs to be old and mature in age and faithfulness; because the teaching is serious, ancient, and of weight and substance}.
Moreover, Azevedo explains, the teacher is shown \quoted{with a musical instrument which gives pleasure to the ear}:
\begin{quote}
So that we should understand that Faith enters through the ear \add{\foreign{oído}}, as St.\ Paul says, 
and that the disciple should be like a child, simple, without malice or duplicity, without knowing even how to respond or argue, but only how to listen and learn.
Thus this image depicts for us elegantly, what the hearer of the Faith \add{\foreign{el oyente de la Fe}} should be like.%
  \begin{Footnote}
  \Autocite[f.~1b]{Azevedo:Catecismo}:
  \quotedsp{Para pintar los Romanos la Fe, lo primero que hizieron templo y altar \Dots{} Numa pompilio \Dots{} puso vn idolo:
  de forma de vn viejo cano, que tenia vna harpa en la mano, i estava enseñando vn niño echado a sus pies.
  En esta figura o geroglifica esta encerrada mucha filosofia, i aun cristiana.
  En el templo i ara denota, que la fe a de ser firme i fixa, no movediça, ni flaca, que a cada ayre de novedad se mueva
  I tambien que a menester maestros que la enseñen, i dicipulos que la oygan.
  I, que el maestro a de ser anciano, i maduro en edad y bondad.
  Porque la dotrina es grave, antigua, i de tomo i sustancia. 
  No nueua, ni de pocos años sino antigue dende los Apostoles, i con instrumento musico que da gusto al oido.
  Paraque entendamos, que la Fe entra por el oido; como dize S.\ Pablo.
  I que el dicipulo sea como niño, sencillo, sin malicia ni doblez, sin saber ni replicar, ni arguir, mas de solo oir, y deprender.
  En lo qual nos dibuxa galanamente, qual a de ser el oiente de la Fe.}
  \end{Footnote}
\end{quote}
The teacher's task according to Azevedo, then, is not only to make the faith heard, but to make it \quoted{appeal to the ear}, just as he says music does; and the disciple's task is simply to listen and take heed.

But such teaching was limited by the sensory capacity of each listener, and therefore the Roman Catechism argues that the task of listening requires training.
The catechism exhorts its teachers to accommodate the limitations of their listeners' senses (\term{sensus}) even as they train their hearers to listen profitably.
This emphasis on accommodation was counterbalanced by the catechism's statement that the disciples who will receive the benefit of the teaching are those whose senses have been properly trained.
Pastors had to accommodate the ear even while training it.

The theology of the catechism might suggest that the value of music in propagating faith would come from the medium's ability to make the faith appeal to the ears of listeners.
Villancicos, as an auditory medium based on vernacular poetry, would seem like an ideal vehicle for this project.
If teaching should appeal to the ear \emph{like} music, as Azevedo says, then combining teaching with actual music would appeal all the more.
At the same time, the challenge of training the sense of hearing would seem to be multiplied with music, since a listener must learn to perceive musical structures in order to gain benefit from the music.

\subsection{Experiencing Spiritual Truth through Music}

% - traditional catholic notions of faith
% - new challenges in era of exploration, (re-)evangelization
% - new pressures in era of anxiety/doubt about subjective experience 
%   and cultural conditioning
%
% Kircher: experiencing spiritual truth through music
% but how to listen rightly?

% danger of subjective experience
% need for cultural conditioning
%  - missionary experiences, e.g., Japanese legates
% need to build society as part of evangelism; gospel = church, musica instrumentalis => musica humana

% obstacles to faith and mistrust of hearing
% Calderon 





