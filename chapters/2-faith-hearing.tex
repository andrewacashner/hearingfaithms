% Andrew Cashner -- Faith, Hearing, and the Power of Music
% Chapter 2 -- Making Faith Appeal to Hearing

% 2016-08-31    Revision for book begun


%*******************************************
\label{ch:faith-hearing}

\epigraph
{For I doubt it now, although I know it,\\
because I have listened to Faith without faith.}
{Calderón, \worktitle{El nuevo palacio del Retiro} (1634), \textlinenums{1319--1320}}

\quoted{How are they to believe if they have not heard?} St.\ Paul asked, since \quoted{faith comes through hearing, and hearing, by the Word of Christ} (\scripture{Rom 10:16-17}).
Villancicos in the Spanish Empire of the seventeenth century made faith audible in a way that potentially appealed to the ears of elite and common people alike.
We have seen a variety of ways that villancicos represent music-making, and thus should be considered as musical discourses on music.
As these pieces were performed in and around Spanish churches, they proclaimed and embodied religious beliefs about the relationship between music and faith.
At the same time church leaders used the pieces themselves to cultivate faith.

How, then, did early modern Catholics understand the relationship between hearing and faith?
And how could the auditory art form of villancicos affect that relationship?
This chapter situates villancicos within the context of early modern Catholic discussions of faith and sensation.
As the theological sources discussed in the first section reveal, Hispanic Catholics experienced a certain anxiety about the role of individual sensory perception in acquiring and developing faith; and they harbored uncertainty about whether all people really had the capacity for faith.
Though many people acknowledged music's power to inspire faith, their concepts of music left it unclear exactly how people could develop their hearing faculties in order to derive a spiritual benefit from listening to music.

The second section interprets villancicos that explicitly address themes of faith and sensation, demonstrating that these pieces, too, reflect uncertainty and doubt about hearing's role in acquiring faith.
These pieces stage allegorical contests of the senses, represent sensory confusion (such as synesthesia), and represent characters whose impairments of hearing render them unable to understand religious teaching.

Understanding the theological environment in which villancicos were performed, and considering how villancicos in turn contributed to that environment, makes it clear that villancicos functioned as much more than vehicles for simplistic religious teaching.
Rather, villancicos provided listeners with opportunities to contemplate the challenges of faithful hearing.

%***********
\section{The Challenge of Making Faith Appeal to Hearing}

The global discoveries and religious upheavals of the sixteenth century propelled the Roman Church and its inherited medieval theological system into a chaotic new world.
The church's missionaries struggled to bridge cultural divides in order to bring people to faith both outside and within Europe.
At the same time, the conflicts with Protestants, who emphasized the centrality of personal  experience in faith, along with changing philosophical and physiological understands of humanity and its place in the cosmos, created anxiety and doubt about subjective experience.
Though Catholic apologists distanced themselves from the Protestant emphasis on personal faith, Catholics could not avoid what was a fundamental issue in Christianity---namely, that somehow a connection had to be made between the objects of faith, which were understood as transcendent truth, and the experience of the individual Christian subject.

Traditional Catholic notions of faith, formulated by St.\ Thomas Aquinas, held that faith meant more than assenting to intellectual propositions, believing that certain things were true.\citXXX
The technical term for that basic sort of faith was \term{fides informata} or unformed faith.\citXXX
Fully formed faith, \term{fides formata}, was a virtue (\term{virtus}) or capacity that \quoted{worked through} the higher virtues of hope and love.
True faith for Catholics meant a commitment of the whole person to live faithfully in communion with Christ through his body, the Church.\citXXX

A central document for understanding the faith of post-Reformation Catholics is the catechism produced \quoted{for the parishes} by authority of the Council of Trent.%
  \autocite{Catholic:Catechismus1614}
The bishops at the Council of Trent were not only responding to Protestantism; many of them sought to address the underlying problems that had allowed Protestantism to take hold in the first place.\citXXX{}
Chief among those problems was a lack of education both of clergy and laity.
Through an elegantly composed Latin catechism, church leaders hoped to educate their clergy so that the clergy could better instruct the parishioners under their care.

Vernacular expositions of the catechism, like those of Antonio de Azevedo, Juan Eusebio de Nieremberg, and Juan de Palafox y Mendoza, brought this teaching down to a more accessible level, still addressing a clerical reading audience, but often in a colloquial tone, with earthy illustrations and lengthy paraphrases of Scripture in Spanish.
These texts come alive when read aloud, and indeed their goal is to prepare pastors to teach unlettered disciples through words and voice.
These disciples might be Indians---in Spanish the term was used for indigenous peoples both of America and Asia---or Europeans, such as rural folk in mountain passes where Christianity had still not fully penetrated.
These books prepared teachers for the challenge of making faith appeal to hearing.

The official Roman Catechism grounds the Church's authority to preserve and teach \soCalled{the faith} in the theology of Christ's Incarnation.
The catechism teaches that God communicated his own nature to humanity by taking on human flesh in Christ, and therefore the true Word of God was not the Scripture or any body of doctrines, but rather Jesus Christ himself as the \term{logos} or \term{verbum} (\scripture{Jn 1:1}).
But while Truth ultimately would consist in knowing God in Christ, the catechism teaches that Christ founded the Church to be the means through which people would come to know him after his resurrection.
Christ appointed apostles, chief among them St.\ Peter, to be the custodians of the true faith from his time up until the present.

The Church as Christ's body was the community through which people came to faith in Christ and learned to live faithful lives after Christ's example.
Drawing on St.\ Paul's dictum that faith came through hearing, and hearing, by the Word of Christ, the catechism challenges the Church's ministers to find a way to make Christ the Word audible:
\begin{quote}
Since, therefore, faith is conceived by means of hearing, it is apparent, how necessary for acheiving eternal life are the works of the legitimate teachers and ministers of the faith. \Dots{}
Those who are called to this ministry should understand that in passing along the mysteries of faith and the precepts of life, \emph{they must accommodate the teaching to the sense of hearing and intelligence}, so that by these \add{mysteries and precepts}, \emph{those who possess a well-trained sense} should be filled up by spiritual food.%
  \begin{Footnote}
  \Autocite[2, 8--9 (emphasis added)]{Catholic:Catechismus1614}: 
  \quotedla{Cvm autem fides ex auditu concipiatur, perspicuum est, 
  quàm necessaria semper fuerit ad {\ae}ternam vitam consequendam 
  doctoris legitmi fidelis opera, 
  ac ministerium \Dots\ vt videlicet intelligerent, 
  que ad hoc ministerium vocati sunt, 
  ita in tradendis fidei mysteriis, ac vit{\ae} pr{\ae}ceptis,
  doctrinam ad audientium sensum, atque intelligentiam accomodari oportere,
  vt cùm eorum animos, qui exercitatos sensus habent, 
  spirituali cibo expleuerint \Dots.}
  \end{Footnote}
\end{quote}

As the emphasized phrases show, the Church taught that for faith to come through hearing, both the teacher and the listener had to be involved.
Antonio de Azevedo begins his vernacular introduction to the catechism with the notion that faith requires both wise teachers and attentive listeners.%
  \autocite{Azevedo:Catecismo}
For Azevedo, faith is epitomized in an ancient image he read about in Pliny, depicting \quoted{an elderly man sitting inside a temple, who had a harp in his hand, and who was teaching a boy who lay at his feet}.%
  \autocite[f.~1a]{Azevedo:Catecismo}
The temple, Azevedo explains, represents that faith should be \quoted{firm and fixed, and also that there must be masters who teach it, and disciples who listen to it; and that the master needs to be old and mature in age and faithfulness; because the teaching is serious, ancient, and of weight and substance}.
Moreover, Azevedo explains, the teacher is shown \quoted{with a musical instrument which gives pleasure to the ear}:
\begin{quote}
So that we should understand that Faith enters through the ear \add{\foreign{oído}}, as St.\ Paul says, 
and that the disciple should be like a child, simple, without malice or duplicity, without knowing even how to respond or argue, but only how to listen and learn.
Thus this image depicts for us elegantly, what the hearer of the Faith \add{\foreign{el oyente de la Fe}} should be like.%
  \begin{Footnote}
  \Autocite[f.~1b]{Azevedo:Catecismo}:
  \quotedsp{Para pintar los Romanos la Fe, lo primero que hizieron templo y altar \Dots{} Numa pompilio \Dots{} puso vn idolo:
  de forma de vn viejo cano, que tenia vna harpa en la mano, i estava enseñando vn niño echado a sus pies.
  En esta figura o geroglifica esta encerrada mucha filosofia, i aun cristiana.
  En el templo i ara denota, que la fe a de ser firme i fixa, no movediça, ni flaca, que a cada ayre de novedad se mueva
  I tambien que a menester maestros que la enseñen, i dicipulos que la oygan.
  I, que el maestro a de ser anciano, i maduro en edad y bondad.
  Porque la dotrina es grave, antigua, i de tomo i sustancia. 
  No nueua, ni de pocos años sino antigue dende los Apostoles, i con instrumento musico que da gusto al oido.
  Paraque entendamos, que la Fe entra por el oido; como dize S.\ Pablo.
  I que el dicipulo sea como niño, sencillo, sin malicia ni doblez, sin saber ni replicar, ni arguir, mas de solo oir, y deprender.
  En lo qual nos dibuxa galanamente, qual a de ser el oiente de la Fe.}
  \end{Footnote}
\end{quote}
The teacher's task according to Azevedo, then, is not only to make the faith heard, but to make it \quoted{appeal to the ear}, just as he says music does; and the disciple's task is simply to listen and take heed.

But such teaching was limited by the sensory capacity of each listener, and therefore the Roman Catechism argues that the task of listening requires training.
The catechism exhorts its teachers to accommodate the limitations of their listeners' senses (\term{sensus}) even as they train their hearers to listen profitably.
This emphasis on accommodation was counterbalanced by the catechism's statement that the disciples who will receive the benefit of the teaching are those whose senses have been properly trained.
Pastors had to accommodate the ear even while training it.

The theology of the catechism might suggest that the value of music in propagating faith would come from the medium's ability to make the faith appeal to the ears of listeners.
Villancicos, as an auditory medium based on vernacular poetry, would seem like an ideal vehicle for this project.
If teaching should appeal to the ear \emph{like} music, as Azevedo says, then combining teaching with actual music would appeal all the more.
At the same time, the challenge of training the sense of hearing would seem to be multiplied with music, since a listener must learn to perceive musical structures in order to gain benefit from the music.

\subsection{Experiencing Spiritual Truth through Music}

This problem---that music has the ability to make faith appeal to the ear, but that the ear must be trained to hear it---may be seen in the explanations of music's power by Athanasius Kircher.
The Jesuit writer is one of the few who attempted to articulate exactly how music worked to propagate faith.\citXXX
After describing the miraculous powers of preaching reported to him by his missionary colleagues, Kircher discusses what music adds to the spoken word.
\begin{quote}
If a preacher should wish by the power of God to move a devout person to heavenly things, so that the listener is given over in meditation in otherworldly affects and raptured in his mind,
and if the preacher should take some notable theme expressed in words,
which would recall to the hearer's memory the sweetness of heavenly things and their mildness,
and then fittingly adapt that verbal theme through cadences and intervals in the Dorian mode,
the \emph{the listener could experience that what whas said is actually true}, 
since through harmonic sweetness he could be transported beyond himself by those heavenly things,
carried away by joy to where those things are true.%
  \begin{Footnote}
  \autocite[bk.\~7, 550 (emphasis added)]{Kircher:Musurgia}:
  \quotedla{Si quis Deo deuotum hominum rerumque c{\oe}lestium, meditationi deditum in exoticos affectus raptusque mentis commouere vellet is supra insigne aliquod verborum thema, quod rerum c{\ae}lestium dulcedium, \& suauitatem auditori in memoriam reuocaret, modulo dorio per clausulas interuallaque aptè adaptet, \& experietur quod dixi verum esse, statim extra se factos dulcedine harmonica eò, vbi vera sunt gaudi rapi.}
  \end{Footnote}
\end{quote}

Kircher's depiction of music's power goes well beyond the Jesuit formula of \quoted{teaching, pleasing, and persuading}.\citXXX[for Jesuit formula: Bailey?]
Music possesses its own powers, Kircher says, that go beyond what can be \quoted{expressed in words}.
In fact, the right kind of music could cause the listener \quoted{to experience the truth of what was said}.
Thus music not only makes the teaching of doctrinal truth appealing and persuasive; it actually transforms listeners, transporting them to a realm of heavenly truth, through affective experience.

For Kircher, music links the objective truth with subjective experience through the unique ways that music affects the human body.\citXXX[Kircher discussion of affects, passions]
Affective content in Kircher's theory is somehow written into the music through rhetorical tropes and intrinsic properties of music's natures, as well as being embodied by the performers.
Through principles of sympathetic vibration, the humoral-affective properties could be transferred from composer and performer to listeners.

But Kircher never fully resolves the problem of subjectivity in this process: namely, that each listener perceives music differently based on both cultural background and individual humoral temperament.
Kircher acknowledges that music does not communicate the same things, or produce the same effects, for listeners of different cultures.\citXXX[on cultural conditioning]
Moreover, he has no single answer to explain what (for example) music with a high melancholic component would do to a listener of a particular temperament.
Would the melancholic listener overflow with black bile and experience a cathartic balancing of his affections?
Or would he receive a fatal dose of melancholy that would be detrimental to his health?
When Kircher describes the \quoted{fitting} adaptation of preaching to \quoted{cadences and intervals in the Dorian mode}, he does not explain how one acquires the necessary knowledge and capacity to hear those musical structures and derive the intended benefit.

%****************
\subsection{The Danger of Subjective Experience in Faith}

The capacity to listen faithfully, and therefore music's power to make faith appeal to hearing, would then be limited by cultural conditioning as well as by personal subjectivity.
Regarding individual religious experience, there was a creeping anxiety within early modern Spanish culture regarding the relationship of the senses to faith.
This anxiety made the power of music potentially dangerous.

The question of what role individual subjective experience played in acquiring faith had vexed the Western church since before the start of the Reformation.
As we have already noted, pre-Reformation theologians taught that faith was a virtue, which had to be \quoted{formed} by working through hope and charity.
Unformed faith, or simple belief in certain ideas, was primarily a matter of the head or intellect; but fully formed faith was a matter of the hands.
\term{Fides formata} was really \quoted{faithfulness}, a conviction that manifested itself through ethical behavior in fidelity to God's will.

In the early sixteenth century, the Catholic Humanists such as Erasmus, who was influenced by the Devotio Moderna in which he was raised, stressed the need for an affective spirituality of the heart that could unite head belief with the faithfulness of the hands.\citXXX{}
For Erasmians, this heartfelt faith was to be the source for reformation of the individual and society, the Christian path to achieving the educational goals of the Renaissance---namely, by following the Classical models of Plato and Cicero to produce men of virtue and therefore a just society.\citXXX[Erasme en Espagne, etc]

At the same time, however, Martin Luther radically redefined \foreign{fides} as trust in the gospel of salvation through Christ alone, and separated \foreign{fides} from \foreign{virtus}---\soCalled{faith} from \soCalled{works}.
For Thomas More and other Catholics who polemicized against Luther, Luther's theology turned his followers away from the external, trustworthy, institutional Church and its objectively operating sacraments, and left them with nothing but a subjective internal experience as assurance of salvation.%
  \autocite[\XXX, also More]{Schreiner:Certainty}
Further, as Catholics understood the Protestant position, faith was no longer connected the cultivation of a just society, but was a purely individual matter.

Regardless of whether this critique was fair, the Catholic reaction to Luther produced widespread anxiety in Spain regarding the role of subjective sensory experience in faith.
The religion of the heart that had been promoted by the Humanists as a path to sanctity came to be seen by some after Trent as the gateway to heresy.\citXXX{}
As a result the Spanish Inquisition put heavy pressure on groups emphasizing individual spiritual experience, such as the mystically inclined \term{alumbrados} and the Carmelite reformers Teresa of Ávila and John of the Cross.
Teresa was one of many who claimed no authority for her teaching except her own visions of God, but unlike many others (such as the \term{beata} Francisca de los Apóstoles) she managed to deflect official suspicion and dodge Inquisitorial censorship.%
  \autocites[\XXX]{Ahlgren:TeresaPolitics}{Francisca:Inquisition}

John of the Cross's \worktitle{Ascent of Mount Carmel} uses \soCalled{negative theology}---moving toward God by contemplating what God is not, since God is beyond anything the human mind can imagine---to distance contemplatives from their visions and sensations.
By willingly depriving themselves of sensory experience they might come closer to union with the God who was beyond sensation.
John defines that union not in sensual terms but in ethical ones, as the total surrender and conformity of one's will to God.\citXXX[John plus secondary]
Ignatius of Loyola's \worktitle{Spiritual Exercises}, the foundation of Jesuit spirituality, are dedicated entirely to the task of discerning whether one's religious sensory experiences are truly from God.\citXXX{}

In such a climate, music's power over the sense and affects came under suspicion as well.
While some Catholics such as the Jesuit missionaries (despite their founder's suspicion of music) were eager to use this power to advance the cause of the Church, others saw the use of music as a potential distraction or distortion.\citXXX[who?]
In part this stemmed from the Spanish fascination with illusion (\foreign{engaño}) and the potential decpetion of the senses, as scholars of Spanish literature like Cervantes's \worktitle{Don Quixote} and Calderón's \worktitle{La vida es sueño} have long understood.\citXXX{}
Music's power over the senses and affects might be used for the purposes of cultivating faith, but this power had to be carefully controlled and submitted to reason to mitigate the dangers of individual subjectivity.

%*********************
\subsection{The Need for Cultural Conditioning in Hearing}

On the cultural side, the age of exploration required European Catholics to seek a distinction between the core of Christian religion---that which was to be preserved without chang in all cultural settings---and specific cultural expressions or incarnations of Christianity, which could be changed.\citXXX[on inculturation, acculturation, transculturation,etc]
The Chinese Rites Controversy with the Jesuits, in which the bishop of Puebla, Juan de Palafox y Mendoza, played an active role, was only one example where that distinction proved difficult to delineate.\citXXX[Chinese rites, Palafox]

To adapt the formulation of Ines Zúpanov, in her study of Jesuit missions in India, the \quoted{tropics} represented both a geographic zone and a process of cultural transformation (troping, turning).\citXXX[Zupanov]
Some missionaries like the Jesuits in Japan and Brazil actively sought to accomodate local customs and music; but everywhere that missionaries brought Christian faith, the process of cultural translation inevitably transformed it into something neither they nor their converts could necessarily predict.\citXXX[Japanese mission, Bailey, Castagna Brazil]

There is ample documentary evidence that native converts performed plainchant and classical-style polyphony like that of Palestrina and  Guerrero in Catholic missionary settlements from Luanda (Angola) to Goa (India) and Cuzco (Peru).\citXXX
It is also evident from ethnography today that the peoples of central Africa, south Asia, and the Andes sing with distinctive modes of vocal production, intonation, and rhythmic and melodic variation. 
In a history of Jesuit mission art, Gauvin Bailey points out the discrepancy between the Jesuits' accounts of art on their missions, which stress for supervisors in Europe how closely their converts had learned to imitate European models, and the surviving evidence of that art, which often shows strong indigenous influences.\citXXX[bailey]
There is every reason to think that mission music, even plainchant and polyphony imported directly from Europe, sounded quite different in actual performance from its European models.
As Protestant missionaries in New England observed, even when Indians sang the same songs, \quoted{they differ from us in sound}.\citXXX[goodman]
  
In Mexico, missionary friars allowed the Mexica people to continue singing and dancing in Nahuatl, and Bernardino de Sahagún even provided a lectionary of psalms for them to sing in their native language.\citXXX[Candelaria]
But the Mexican bishops\XXX{} complained that the natives' songs were written in such complex language that they lacked linguists sufficiently skilled to validate their orthodoxy.\citXXX{}
Lorenzo Candelaria has suggested that this anxiety about the content of songs in native languages, rather than any concern about styles of European church music, was actually the primary motivation for the Council of Trent's decree requiring the words of church music to be intelligible.\citXXX[Candelaria, Trent]

As the Church was adapting itself to native sensibilities, or being adapted by colonial subjects in ways the Church could not fully control, the question naturally arises, which parts of Christianity constituted \soCalled{the Faith} that was supposed to come through hearing?
And further, if hearing was culturally conditioned, then how might one reliably use music to appeal to that sense?

The problem of acquiring the capacity to hear properly---of training the sense, as the Roman Catechism puts it---is plainly stated in a 1590 dialogue that represents the impressions of four Japanese noble youths after the Jesuit missionaries took them on a grand tour of Spain and Italy between 1582 and 1590.%
  \autocite{Sande:DeMissioneLegatorum}
Their trip from Nagasaki to Rome took them to most of the major Iberian musical centers discussed in this study (not to mention the most important cities in Italy): on the outgoing trip, to Lisbon, Évora, Toledo, Madrid, and Alcalá; and on the return, to Barcelona, Montserrat, Zaragoza, and Daroca.
The leader of the Jesuit mission to Japan, Alessandro Valignano, hoped to persuade the authorities of his order and church that \quoted{European Jesuits must accommodate themselves to Japanese manners and customs}.%
  \autocite[4]{Massarella:JapaneseTravellers}

At the same time, Valignano recognized that the missionaries were asking the Japanese to accommodate a new culture as well. 
Therefore the boys' mission was both to represent Japan to Europe and on their return, to represent Europe to Japan.

Valignano and his Jesuit collaborators documented the trip in the form of a dialogue between the boys who traveled and their friends who stayed home.
Though ostensibly based on first-hand accounts of the Japanese \soCalled{legates}, the book reflects how the European missionaries hoped the Japanese would see Europe.

The Japanese boys had received training in music, and practiced and performed throughout their trip and upon their return.
In the dialogue, when their friends ask about European music, the boys tell them that it took them time to become accustomed to it, before they could recognize its superiority.%
  \footnote{\autocite[109--110]{Sande:DeMissioneLegatorum}, translation from \autocite[155-156]{Massarella:JapaneseTravellers}, emphasis added.}
In this excerpt the character Michael is one of the returned travellers---he corresponds to a real historical person---and Linus is one of his friends who stayed back in Japan.

%*******************
\begin{quotation}
\speaker{Michael}
You must remember, as we said earlier, how much we are swayed by longstanding custom, or on the other side, by unfamiliarity and inexperience, and the same is true of singing. 
\emph{You are not yet used to European singing and harmony, so you do not yet appreciate how sweet and pleasant it is, whereas we, since we are now accustomed to listening to it, feel that there is nothing more agreeable to the ear.}

But if we care to avert our minds from what is customary, and to consider the thing in itself, we find that European singing is in fact composed with remarkable skill; 
it does not always keep to the same note for all voices, as ours does, but some notes are higher, some lower, some intermediate, and when all of these are skillfully sung together, at the same time, they produce a certain remarkable harmony \Dots{} 
all of which, \Dots{} together with the sounds of the musical instruments, are wonderfully pleasing to the ear of the listener. \Dots{}

With our singing, since there is no diversity in the notes, but one and the same way of producing the voice, we don't yet have any art or discipline in which the rules of harmony are contained; 
whereas the Europeans, with their great variety of sounds, their skillful construction of instruments, and their remarkable quantity of books on music and note shapes, have hugely enriched this art.

\speaker{Linus}
I am sure all these things which you say are true; for the variety of the instruments and the books which you have brought back, as well as the singing and the modulation of harmony, testify to a remarkable artistic system.
Nor do I doubt that \emph{our normal expectations in listening to singing are an impediment when it comes to appreciating the beauties of European harmony.}%
  \begin{Footnote}
  {\XXX Complete original text}
  \end{Footnote}
\end{quotation}
%*******************

From this perspective, then, music could be a way to make faith pleasing to the ear, as Kircher describes, but only if the ear was trained in order for it to find such music pleasing.
This conclusion would apply on both individual and social levels; that is, both personal subjectivity and cultural conditioning were involved in the process of hearing music.

Using music to propagate faith meant appealing to the ear and training it at the same time. 
Since Catholics (as exemplified in the Tridentine Catechism) believed faith to have a social dimension---a faithful life as part of the community of the church---inculcating faith through music meant more than a one-on-one dialogue between \soCalled{the music} and \soCalled{the listener}.
The musical ritual of the post-Tridentine Church involved a large number of community participants, for whom performing music with the body and hearing it were inextricably linked.

Moreover, the musical efforts of the colonizing church concretely built social relationships through musical training.
Bringing a man to faith meant setting him on a path to becoming a whole, virtuous person in the image of Christ (\foreign{virum perfectum} in the words of the Catechism, the root of \foreign{virtus} being \foreign{vir}).%
  \autocite[8]{Catholic:Catechismus1614}
At the same time, the virtue of man as Neoplatonic microcosm was reflected in the broader society and in turn depended on it. 

Teaching faith, then, meant trying to establish not just individual Christians, but also building a Christian society as the body of Christ.
This is why the friars in Mexico not only started parishes, but they also trained choirs.
Catholic music was not \emph{about} society; it was a form of society.
Forming choirs of boys and training ensembles of village musicians in colonial Mexico were practical means of establishing the Church.
The economic aspect of paying for music and musicians, and the political aspect of creating social organizations and putting on public spectacles, though seemingly secular, were still an important part of the church's mission.

In other words, the evangelizing mission could not be easily separated from a civilizing mission, any more than the Church in Europe after Trent could separate its duty to preserve purity of doctrine from the need to unify liturgical practice. 
In the terms of Boethius, which was promulgated in the influential Spanish music treatises of Pedro Cerone, well-tuned \term{musica instrumentalis} could harmonize the \term{musica humana}---the harmony of the individual in body and soul, reason and passion, but also the concord of human society.\citXXX[Boethius, Cerone]
All this could better reflect the cosmic \term{musica mundana}, and ultimately the harmonies of the triune God.

%***********************
\subsection{Obstacles to Faith and Mistrust of Hearing}

But even if music could be adapted for individual temperaments and translated across cultural differences, what if a person simply lacked the proper disposition to hear the Word with faith?
Theologians debated, at some remove from the human beings at the center of their disputations, whether Indians and Africans truly had the capacity for faith.\citXXX{}
Most concluded that these groups could be converted from the Satanic darkness of their paganism despite severe handicaps in reason and moral sense.
Someone who was not fully a \gloss{vir}{man} could not be expected to arrive at the fullness of \term{virtus}.

Jews, on the other hand, were not accorded even this limited hope for salvation.
Even those Spanish subjects of Jewish descent who claimed to have converted---the \term{conversos}---were viewed with suspicion and subject to inquisitions at both the official and vigilante levels.\citXXX{}
In the genre of allegorical mystery play performed on  Corpus Christi across the Spanish empire (the \emph{auto sacramental}), playwrights made the character \term{Judaísmo} represent the perpetually unbelieving Jew.\citXXX{}
(\term{Hebraísmo} represented the Old Testament believers whom Catholics considered to be part of the Christian church.)\citXXX{}
Villancicos were typically performed before and after these plays, and the \term{auto sacramental} provides a crucial context for understanding the theological environment in which villancicos were heard.

In a politically important Corpus Christi play by Spanish court poet Pedro Calderón de la Barca, the figure of \term{Judaísmo} becomes a vivid representation of the incapacity to acquire faith, \term{despite} the sense of hearing.
Performed in 1634 to inaugurate Philip IV's new palace, the Buen Retiro, Calderón's \worktitle{El nuevo palacio del Retiro} centers on Judaism's incapacity for faith.%
  \autocite{Calderon:Retiro}
Judaism is forcefully excluded from the festivities celebrated within the play, which culminate with the consecration of the Eucharist.
Instead Judaism stands to the side and asks the character Faith to explain each event to him.
Faith answers with elaborate allegories.

When the Eucharistic host is consecrated and elevated, Judaism responds with a long monologue in which he attempts to understand the mysterious wafer by connecting it with stories from Hebrew Scripture such as the manna in the desert and the dew on Gideon's fleece.\citXXX[bible passages]
But Judaism cannot accept any of these explanations.
In fact he is unable to believe what Faith has explained to him, because, as he says in an increasingly embittered refrain (the source of this chapter's epigraph), \quotedgloss{a la Fe he escuchado sin la fe}{I have listened to Faith without Faith}.

%*****
\begin{verse}
Who are you, that I see you and I do not know it,\\
because I have listened to Faith without faith? 

\Dots\\
Are you that manna that quenched thirst\\
and satisfied hunger?\\
Are you the fruit that is dangling from the tree\\
that gave knowledge of good and evil?\\
Are you the serpent that gave us healing\\
staked to the staff of Moses?\\
For I doubt it now, even though I know it,\\
because I have listened to Faith without faith. 

But surely you must be the flower of Jericho,\\
surely you must be the lily of the valleys,\\
white manna that rained from Heaven for us,\\
pale dew that dampened the fleece,\\
dangling serpent, fire that showed the way,\\
forbidden fruit, rejected honey,\\
I cannot comprehend you nor know your enigma,\\
because I have listened to Faith without faith. 

And so, let him run to your singular goal\\
who can appraise your value,\\
for I will always doubt your being,\\
for I will never perceive your light,\\
because you are not the Host of my altar,\\
because you are not the sun of my setting,\\
because your dark cipher I did not comprehend,\\
because I have listened to Faith without faith. 

\emph{All the musical instruments play, shawms and snares, drums and trumpets,
and everyone enters crowned with wreaths, and with lances, as for battle, 
to the measure of the clarion.}%
  \begin{Footnote}
  Please see the appendix\XXX{} for the original text, from \autocite[\textlinenums{1303--1305, 1315--1336}]{Calderon:Retiro}.
  \end{Footnote}
\end{verse}
%******

If we recall how the Roman Catechism urged pastors to accommodate \quoted{the sense of hearing and intelligence}, the character Judaism's problem is neither with sensory ability nor even with his intellectual knowledge of doctrine: clearly he has listened carefully to Faith's explanations and graps their connections to Hebrew scriptures.
But something must be lacking in this character that permanently prevents him from moving from listening to \quoted{the Faith} into actual saving faith.\citXXX[discourse on anti-Semitic theology]

Judaism hears sound teaching but makes no connection between hearing and faith. 
Simply put, he does not believe what he hears.
Would contemporary listeners have understood the fault to be in Judaism's hearing or in his faith?
For Catholic theologians, faith required both a gift of God (that is, the result of grace), and an excercise of the human will.\citXXX{}
Calderón's Judaism is denied the grace needed to hear Faith with faith, and so stubbornly turns away from faith and willingingly persists in spiritual deafness.

What function could such a scene serve for an audience that, by Spanish law, contained no Jews?
Scholars have described \term{autos sacramentales} as \quoted{dramas of conversion}\citXXX[Wardropper?], but description fits poorly with a drama that centers on a character who is constitionally unable to convert.
Perhaps the scene is addressed to \term{conversos} who were not truly converted, who Spaniards believed to be hiding in every neighborhood.
But if the Jew is denied the grace needed to hear faithfully, then there would be little hope of conversion for crypto-Jewish listeners.
What Calderón does accomplish is to reassure pure-blooded listeners that they are better off for not being Jews, and perhaps issue a warning not to harden their hearts like those who had rejected Christ---a motive both to give thanks for the capacity of faith and to keep one's ears open for what the Faith would teach.

It seems no accident that Judaism's eloquent confession of unbelief is immediately drowned out by music.
As a procession enter bringing the King, Queen, and Man, the fanfares \quoted{to the  measure of the clarion} celebrate the King's triumph over Judaism through the sense of hearing.
For Calderón's listeners, who had been taught to regard Jews as the embodiment of willful unbelief and worse, the entry of the musicians would clear away the acrid sound of Judaisms's doubts.

The feeling of doubt about the senses, however, pervades the entire play.
Though there is no question that Calderón intended to exalt Catholic orthodoxy, he gives so much stage time to expressions of uncertainty that it almost seems to undermine that official goal.
Much of the rest of the play dramatizes a contest of the senses, in which Hearing prevails---but only after confessing to his own incertitude.

Just after Judaism's speech, Calderón has allegorical characters of the five senses---Sight, Touch, Smell, Taste, and Hearing---all approach the Eucharistic host, but each one fails to understand.
Sight sees only bread, Smell smells only bread, and so on.
But Hearing, the last to come near, simply believes Christ's statement in the mouth of the priest, \quoted{This is my body}, saying, \quoted{For I need no more than to hear it in order to believe it}.%
  \autocite[\textlinenums{1427--1428}: \quotedsp{que yo no he menester más/ de oírlo para creerlo}]{Calderon:Retiro}
Calderón thus presents Hearing as the only sense favored by Faith, and the only one capable of grasping the mysteries of faith.
But just as Judaism could hear Faith without faith, so Calderón represents Hearing as the most uncertain of all the senses.

Earlier in the play, each personified sense competes for a laurel prize awarded by Faith.
Each sense in turn boasts of his powers, but Faith rejects each one.
With a blunt couplet she spurns Taste, for example: \quoted{Do not speak, Taste, with Faith/ for Faith does not believe Taste}.%
  \autocite[\textlinenums{565--566}]{Calderon:Retiro}
Hearing is the last sense to present himself, and in contrast to the other senses, he speaks of his weakness, and how easily he can be fooled by echos or feigned voices.
Hearing perceives only the voice of a man, not the man himself.
Since he cannot trust his own powers, he must rely on faith.

%****
\begin{verse}
It is right to tremble before you,\\
thus, lame, humble, and blind,\\
I can hardly present myself,\\
for to that voice, that moves the lip,\\
I am a statue of snow,\\
although with a soul of fire.\\
I am Hearing, and I have\\
only been able to give notice\\
of a voice, being the Sense\\
that is easiest to deceive.\\
Sight sees, without doubting\\
what she sees; Smell smells\\
what he smells; Touch touches\\
what he touches, and Taste tastes\\
what he tastes, since the object\\
is proximate to the action;\\
but what Hearing hears\\
is only a fleeting echo,\\
born of a distant voice\\
without a known object.\\
Thus I am quite hemmed in,\\
for they do not have my errors,\\
not as Sight has its colors,\\
as Touch has its varied textures,\\
as Taste has its subtle delights,\\
nor as Smell has its aromas.%
  \autocite[\textlinenums{567--592}; original text in appendix\XXX]{Calderon:Retiro}
\end{verse}
%*****

In response, Faith crowns Hearing precisely because of his \foreign{desconfianza}---meaning here lack of confidence, mistrust, and humility all at the same time.
\begin{verse}
In this mistrust\\
Faith's love is found all the more;\\
this favor is earned by Hearing alone.

\emph{She gives him the wreath}

Do not give up hope,\\
nor suddenly take afright;\\
from today forward, human Sense,\\
you shall serve me, because\\
the favors of Faith\\
are only for Hearing.%
  \autocite[\textlinenums{593--602}]{Calderon:Retiro}
\end{verse}


What would it mean, then, for hearing to be the favored sense of faith not just because of its humility, but because of its actual weakness, its defects compared to the other senses?
Specifically, if indeed hearing is \quoted{the sense most easily deceived}, how could it rightly be a medium for faith?

In the Eucharistic context of this play for Corpus Christi, Hearing is presented as a portal to truths that go beyond normal sensation.
But at the same time, Calderón so vividly dramatizes Hearing's weakness and incertitude that he seems to encourage listeners to question what they hear as much as they are to trust in it.
This seemingly paradoxical message appears designed to urge Catholics to trust the Church but nothing else; to subject their personal experience to the Church's authority.
Given the triumphal tone of this blatantly political play, in which Philip IV and Christ are made nearly indistinguishable, it is remarkable how much room Calderón leaves on the stage for doubt.

What was music's function in such a theological climate?
How was the Church supposed to use the auditory medium of music to propagate faith, if hearing could not be trusted?
What if some people lacked the necessary capacity to hear \quoted{the Faith} \emph{with} Faith?
Kircher believed that music enabled a listener to experience the truth of faith affectively and bodily, rather than simply assenting to doctrines intellectually.
How, then, might might that process actually work if the sense to which music appeals the most is handicapped by so much uncertainty?  
How could music such as villancicos, then, provide a medium for propagating the faith---or instituting social control---if hearing was so easily deceived?

The uncertainty of hearing in Calderón should make scholars wary of two common approaches to Catholic arts of this period---on the one hand, a simplistic view of art as a means of religious indoctrination, and on the other, a more critical but still unsatisfactory anthropological analysis of art as reinforcing structures of political power.
Interpretations of \worktitle{El nuevo palacio} by Dominique Reyre and Margaret Greer, respectively, represent these two positions.%
  \autocites{Reyre:Retiro}{Greer:Retiro}

Reyre presents the play as \quoted{the promulgation of dogma through metaphor}.
The dogma in question for Reyre is Eucharistic transubstantiation, and she reads the play simply as an exposition of this doctrine.
But Reyre's only source for contextualizing the play theologically is a twentieth-century digest of Catholic theology; she draws on no early modern sources outside the drama to support her arguments.
And her concentration on the Eucharist fails to account for the large portion of the play that deals with other matters, such as the King's audience with the non-Catholic nations, the King's relationship with Man, the condemnation of Judaism, and the confusion of the senses.
Most importantly Reyre overlooks the overt political dimension of the whole play, which should be obvious given the circumstances of its performance and, among other elements, the fact that the character of the King is explicitly named Felipe.

Even if we grant that the play had a didactic purpose, we should recall that the Tridentine Catechism calls for teachers to accomodate the Word to their hearers.
The object of faith, we should recall, was not a set of propositions but a person---Christ the Word, embodied in his Church; it is too simplistic to view Catholic poets, artists, and musicians as simply reiterating doctrines.
Catholic artists of this period are constantly seeking new ways---such as the ever bolder and more intriguing conceits of villancicos---to help their audience connect the own experience with Christ in the Church.
When Calderón has Judaism confess that he will never understand the \quoted{enigma} of Faith, he is also inviting the Christian members of his audience to solve that enigma for themselves.
He challenges listeners to contemplate his learned riddles and paradoxes---such as Faith crowning Hearing because of its lack of certitude.

Contrasting with Reyre's doctrine-oriented approach, Margaret Greer interprets the play as a form of political ritual, drawing on twentieth-century anthropological and political theory.
Greer reads the play as a ritual enactment of Spain's hierarchical society, with the king, ruling by divine right and invested with divine power, at the top.
Her reading is similar to the studies of Corpus Christi celebrations in colonial Peru by Carolyn Dean (focusing on visual art) and Geoffrey Baker (focusing on music).%
  \autocites{Dean:Inka}{Baker:Harmony}{Baker:ResoundingCity}
Baker sees colonial music as a way of \quoted{imposing harmony} on society, particularly in urban settings.

Greer approach does not leave much more room than Reyre does for the play's ambiguities and doubts.
Greer does not sufficiently incorporate the role of listeners into a ritual model: if an open-air play is to function as a civic ritual, then everyone must be involved, and we must then consider not just the structures presented on stage but also the active engagement of listeners.
In other words, while the play certainly appears designed to reinforce the existing social hierarchy, that does not mean it actually fulfilled that function for every hearer.
It is important to distinguish between a contemporary anthropological analysis of a historical event (that is, an outsider's view) and historically grounded understandings.

Anthropologists did not invent the notion that the sacrament celebrated in the Corpus Christi festivities worked effectively through its performance to transform society.
Rather, that concept was a fundamental Catholic belief; and indeed, Catholic sacramental theology has been an influence on anthropological ritual studies from the beginning, as Catherine Bell has noted.%
  \autocite[\XXX]{Bell:RitualPerspectives}

Certainly, the music and other religious arts presented in the festivity had a teaching function, and certainly, these expressive practices also embodied the beliefs they proclaimed.
But perhaps it is not too simple to posit, as Jack Sage did, that people celebrated the king's divine right to rule because they actually believed in it and even depended on it.%
  \autocite{Sage:Instrumentum}
As Patrick Rietbergen has written about Roman religious theater of the same period, it may be easy for some scholars to underestimate how profoundly people's lives were shaped by religious belief and practice.%
  \autocite[\XXX]{Rietbergen:Power}
It is possible to affirm that historical subjects did believe in things that many now consider odious without making a judgment about the truth of those beliefs.

Seeking a historically grounded interpretation, at the same time, does not proscribe critical perspectives.
Indeed, it may be possible from our cultural and historical distance to identify aspects of historical beliefs that were not consistent, and to read religious texts in a way that highlights the questions they raise rather than the answers they propose.
Catholic theologians, particularly in the Spanish lands where the Scholastic tradition continued strongly throughout this period, certainly had any answers for questions about sensation and faith.
But the religious poetry and music of the seventeenth century speaks to a prevalent fascination, anxiety, and doubt regarding sensation and faith, especially where music was involved.


%****************************************
\section{Villancicos on Sensation and Faith}

Villancicos on themes of sensation and faith manifest many of these same theological preoccupations and anxieties.
These performative texts carried out a discourse on faith and hearing among both elite and common hearers.
Like all the other sources discussed in this chapter, villancicos were created by members of the literate elite, and thus they foremost represent a discourse between those elite poets, musicians, and churchmen.
At the same time, though, villancicos were generally performed in public festivities with varied audiences. 
Compared with the royal pomp and verbal virtuosity of a Calderón \term{auto} performances, villancicos were often performed in relatively more intimate settings where there could be more proximity between people of different social strata.

It is legitimate to ask, as José María Díez Borque has done for Calderonian drama, how much listeners could actually hear and understand of these complex poetic performances.%
  \autocite{DiezBorque:Publico}
The same question should be asked of Italian opera, German sacred concertos, and English anthems of the same period.
There are not, at present, sufficient sources to reconstruct the sociology of villancico audiences for all their varied performance contexts around the globe.

But while we may not know exactly who listened to villancicos, the musical sources do tell us what they heard.%
  \begin{Footnote}
  This assumes that we are discussing the perceptions and beliefs of people who actually paid attention to this music.
  Just as in every other genre of musical performance (such as the eighteenth-century symphony), there were always members of the audience for whom the music meant nothing at all---but these people are not the object of our study.
  \autocite{Lowe:PleasureSymphony} imaginately reconstructs how listeners of different social stations heard a symphony, including the times when they were not paying attention.
  Those who heard villancicos at more of a distance would certainly not have followed the intricate poetic and musical conceits, though perhaps they would have recognized the typical subgenres and tropes---shepherd pieces, black-caricature pieces, pieces emphasizing virtuoso musical display.
  \end{Footnote}
Certainly, performers made the music come alive in ways that we may not now be able to recover, just as stage actors did; but the musical performing parts of villancicos are much more highly determined than is the script of a play.
In other words, where we can only imagine the vocal inflections, accentuation, and speed with which Judaism spoke his bitter refrain, \quotedsp{A le Fe he escuchado sin la fe}, if this had been set to music we would know a great deal more about how the words were actually presented to the audiences.

Through vernacular poetry and likely also through musical style, the creators and performers of villancicos addressed a broad public audience.
These pieces represented complex theological concepts, but made these concepts accessible in quite a different way than a theological treatise, even one written in the vernacular.
As individual villancicos were repeated, and as conventional villancico types were performed at multiple festivals each year, these pieces must have shaped attitudes and beliefs in a larger portion of the population.
Moreover, the mere fact that certain types of pieces were so frequently repeated over more than a century must provide some evidence that the characteristic theological tropes of these villancico types resonated with their hearers.

%************
\subsection{Contests of the Senses and Early Modern Concepts of Sensation}

Calderón's contest of the senses is echoed in two villancicos performed at Segovia Cathedral in the later seventeenth century.
Successive chapelmasters in Segovia, Miguel de Irízar (1634--1684, at Segovia 1671--1684) and Jerónimo de Carrión (1660--1721), both set variants of the same poem, attributed to Zaragoza poet Vicente Sánchez.%
  \begin{Footnote}
  The performing parts for Irízar's setting are in \signature{E-SE}{5/32}; those for Carrión's are in \signature{E-SE}{28/25}.
  \end{Footnote}
The version of the poem published posthumously in Sánchez's collected poetic works in 1688 probably corresponds to a now-lost musical setting performed in Zaragoza, probably set by Diego de Cáseda.%
  \autocite[171--172]{Sanchez:LiraPoetica}
We will see throughout part~II that villancico texts moved along networks of musicians, and as chapter 8 will explore in detail, Segovia was linked to Zaragoza (along with Madrid, Toledo, and Seville) because of Miguel de Irízar's personal connections.

The estribillo (appendix\XXX) sets the scene as a competition or public debate between the senses, where each sense will receive a \quoted{hearing} before Faith.
The senses \quoted{file a complaint} amongst themselves regarding the bread of the Eucharist.
The senses cannot perceive the truth of \quoted{the divine bread} unless \quoted{what they sense} is \quoted{by faith consented}---\foreign{de fe consentido}, playing on \foreign{sentido}.
The last line of the estribillo punningly gives away the results of the contest, since the phrase \foreign{hoy todos con la fe sean oídos} can mean either \quoted{let them all today be heard with faith} or \quoted{let them all today with faith become ears}.


Each of the coplas treats a different sense.
The three textual sources (Sánchez, Irízar, Carrión) arrange the coplas differently, perhaps reflecting slightly different understandings of the hierarchy of the senses by each compositor of the text.

The order of coplas in the Sánchez edition nearly matches the presentation of the senses in Calderón's \term{auto}: Sight comes first, followed by Touch; next are Taste and Smell, and Hearing comes last (\tableref{table:senses-order}).
The only difference is that Calderón puts Smell just before Taste.
Spanish theologians and philosophers always discussed Vision as the first and highest of the five exterior senses, as can be seen in the treatises preserved from the old seminary and convent libraries.
A typical example is the 1557 natural-philosophy textbook \worktitle{Phisica, Speculatio} by a New Spanish Augustinian friar, Alphonsus à Veracruce.%
  \autocite{Veracruce:Phisica}
Veracruce summarizes the traditional Catholic teaching from Thomas Aquinas's Christian appropriation of Aristotle's \worktitle{De anima}.\citXXX[on Aristotle in early modern thought]

%************
\begin{table}
\caption{The exterior senses: Order of presentation in versions of \worktitle{Si los sentidos}, correlated with Calderón and Veracruce}
\label{table:senses-order}
\inputtable{senses-order}
\end{table}
%***********

His presentation accords with the synthesis of Galenic physiology in Fray Luis de Granada's contemporary \worktitle{Introduction to the Creed}.%
  \autocite[\XXX]{LuisdeGranada:Simbolo}
The common physiolgoical model of sensation and perception articulated by Fray Luis was based on the concept of exterior and interior senses, also called faculties.
As shown in \tableref{table:senses-fray-luis}, the five exterior senses mediated between the outside world and the interior senses by means of the \term{spiritus animales}.
The \term{spiritus} were an ethereal substance like invisible beams of light, engendered in the cerebral lobes and then circulated through the nerves from the organs of sensation back to the cerebrum.
Fray Luis calls them the source of all movement and all sensation.


%************
\begin{landscapetable}
\caption{The senses and faculties of the sensible soul (\term{ánima sensitiva}), according to Fray Luis de Granada}
\label{table:senses-fray-luis}
\inputtable{senses-fray-luis}
\end{landscapetable}
%***********

In the cerebrum or brain were housed the internal faculties, which \quoted{made sense} of what the external senses told them.
Starting from the front of the head, the first of these faculties was the \quoted{common} sense, a kind of reception area where the exterior senses met the deeper interior faculties.
Moving further back, these faculties included the imagination, the \quoted{estimative} or cogitative faculty, and finally, deepest in the brain was the memory.
Imagination retained the images brought to it by the senses; cogitation conceived of abstrct figures not drawn directly from external sensation and performs logical reasoning.
Memory preserves knowledge, conserves experiences as lessons for future action, and it provides a person with understanding by linking one thing with the next in chains of cause and effect.\XXX{}

All of these exterior and interior senses were part of the \term{ánima sensitiva}, the sensate, sensible, or reasonable soul.
In addition to these senses the \term{ánima sensitiva} possessed an affective faculty, in which the balance of humors in the body interacted with the interior and exterior senses to produce different \quoted{passions}, \quoted{affects}, or simply feelings.
  \begin{Footnote}
  Early modern writers aside from Descartes did not make sharp distinctions between these terms, and Descartes' works were not widely read in Spanish lands.\XXX{}
  \end{Footnote}
Based on a fundamental dichotomy (like magnetism) between attraction and repulsion, this \quoted{concupiscible} part of the soul experienced three primary pairs of passions: love and hate, desire and fear, joy and sadness.\citXXX[Fray Luis]

Expositors ranked the external senses in a hierarchy based on the dgree of mediation between the object of sensation and the person sensing.
The most base sense was taste, because the person actually had to physically consume the object of sensation in order to sense it.
Likewise, touch required direct physical contact with the object.
Smell depended on taking in fragrances emitted by the objects, so its contact was more remote.
Even more remote, then, was hearing, since hearing could only perceive the sounds that entered the ear, and not the things producing those sounds.
Sight was the most powerful sense because it enabled a person to perceive objects a great distance away without making any direct contact (though theories differed about the exact mechanism of light).\citXXX{}

Hearing stood out from these other senses, though, because for it alone, the object of perception was not identical with the thing sensed.
As Calderón's character Hearing says, \quoted{Sight sees, without doubting/ what she sees; Smell smells/ what he smells; Touch touches/ what he touches, and Taste tastes/ what he tastes, since the object/ is proximate \add{immediate} to the action}.%
  \autocite[\textlinenums{577--582}]{Calderon:Retiro}
But Hearing hears a person's voice, not the person directly, as Calderón's text continues: \quoted{But what Hearing hears/ is only a fleeting echo,/ born of a distant voice/ without a known object}.%
  \autocite[\textlinenums{583--586}]{Calderon:Retiro}

While this feature of hearing may have made it \quoted{easily deceived}, it also gave this sense a unique capability in spiritual matters, where the object of perception was not immediately sensible at all.
Thus in the theological contest of the senses in Calderón and in the Sánchez villancicos \worktitle{Si los sentidos queja forman del Pan divino}, Hearing wins out over the other senses because it alone can perceive matters of faith.
With regard to the Eucharist in particular, taste, touch, smell, and vision would all be deceptive, because they only perceive the accidents of bread and not the hidden spiritual substance of Christ's body.%
  \footnote{On tropes of deception and illusion in Eucharistic villancicos and devotional literature, see \autocite{Cashner:Cards}.}
The other mysteries of faith, such as Christ's death on the cross, would be completely inaccessible to the exterior senses, except through the medium of icons, statues, and relics.
Hearing's weakness, its dependence on the medium of sound such as a voice, becomes a strength when the primary access to truth is through \quoted{the Word of Christ}, or as the Roman Catechism explains it, the Word that is Christ.

The treatment of Hearing in these contests tells us not only about ideas of the auditory sense, but also about the ideal kind of faith.
The character of Hearing in Calderón is a model of humble faith, placing himself last in line (Luke\citXXX[last at the table]), refusing to tout his strengths but rather \quoted{boasting in his weakness} like St.\ Paul (\scripture{2Cor 11:30}).
As the Sánchez villancico proclaims, none of the senses---not even hearing---is worth anything without this kind of faith.

The order of the senses in the contests by Calderón and Sánchez, then, does not mirror directly the scholastic hierarchy of the senses; but rather these poets rearrange the order to put Hearing at then end for a dramatic climax, as the underdog competitor who triumphs at the last.
Sight is first and Hearing is presented last, but is accorded first place with regard to faith.
In the Sánchez contest, the eyes \quoted{do not look at what they see}, and the Eucharist reduces Sight to \quoted{blindness} (copla 1).
The \quoted{colors} and \quoted{rays of light} through which Sight normally operates are \quoted{hidden} \quoted{beneath transparent veils} and \quoted{transformed} so that \quoted{God Incarnate is not seen} (copla 2).

Touch may make direct, physical contact with the Eucharistic host, but it has no access to the \quoted{mystery} enclosed within the \quoted{accidents} of bread (copla 3).
Taste is given no \quoted{vote} in this meal, even though normally food is the domain of taste (copla 4).
In eating the transubstantiated host, Taste \quoted{knows it is not bread} but it still does not taste like flesh (copla 5).
Smell (copla 6) might perhaps smell the aroma of this \quoted{marvel}, if he were humble enough to perceive it.
This suggests that Smell might be considered more open to mystery, and this recalls both liturgical incense and the descriptions of John of the Cross about receiving not only visions from God, but even smells and tastes.\citXXX[and these are to be discounted in favor of the true, unknowable God]

Finally, Sánchez presents hearing in the last copla, through the conceit of music.
The senses are \quoted{five instruments} like a musical consort, which must be \quoted{tempered} by faith so that they can \quoted{be heard} by Faith.
Without Faith, sight is actually blind, and touch, taste, and smell are fooled if they believe what their direct sensation tells them about \quoted{the divine bread}.
But when properly attuned by Faith, the sense can be harmonized into a pleasing concord.
In a delightful twist, the subject of hearing is switched so that it is Faith herself who delights in hearing the music of the senses, properly tuned. 

Jerónimo de Carrión's musical setting presents the coplas in the same order as in the Sánchez print, though Carrión omits the second strophes for Sight and Taste.
Thus Carrión's setting has the same dramatic shape, saving Hearing for last, as Calderón's contest.
Miguel de Irízar, by contrast, presents the senses in their traditional Scholastic order, matching Veracruce's schema exactly.
Carrión most likely wrote his setting after Sánchez's works were published in 1688, though he would also have had access to Irízar's setting in the Segovia Cathedral archive.
Irízar's setting predates the publication of Sánchez's poem (Irízar died in 1684), so the variations in Irízar's version probably reflect an earlier version that came to the Segovia chapelmaster through his network of correspondents.\citXXX[Rodriguez and see later chapter]
Irízar may have been motivated by a training in scholastic theology to \soCalled{correct} the order of the coplas and thereby provide clearer instruction to the choirboys under his charge.

Regardless of the order of the coplas, the sense of hearing is not confined to a single strophe in this poem, since the whole piece in this auditory genre of sung poetry is about hearing and appeals to hearing.
Listeners could not touch, taste, or smell the villancico performance (though the latter might depend on standards of cleanliness for the musicians).
In Segovia Cathedral, with its architectural choir surrounded by high stone walls, few lay people could see it, either.
Only the sense of hearing gave access to this moral lesson about sensation and faith.

Like Calderón's discourse on faith and hearing, \worktitle{Si los sentidos} presents a paradox: Faith only listens to what Hearing tells her, but Hearing must trust only in Faith and not in his own \soCalled{sense}.
Once again, it requires faith to hear \quoted{the Faith} \emph{with} faith.
As in the catechism, the powers of sensation must be attuned and tempered by faith in order to attain the object of spiritual perception, Christ the Word.

%************
\subsection{Staging the Contest in Sound: Irízar and Carrión}

The two surviving settings of \worktitle{Si los sentidos} stage the contest of the senses in sound, in contrasting styles that invite different types of involvement from listeners.
In the earlier setting, Miguel de Irízar creates a musical competition in grand festival style by pitting his two four-voice choirs against each other in polychoral dialogue (\exmusicref{exmusic:Irizar-Si_los_sentidos}).
Like a film editor creating a fight scene, Irízar builds intensity by cutting the text into shorter phrases to be tossed back and forth between the two choirs: \foreign{no se den por sentidos} becomes \foreign{no se den} and then \foreign{no, no}. 
The sense of antagonism is heightened when one choir interrupts the other with an exclamation of \foreign{no} in \measurenums{29--30} and \measurenums{40--44}.

%*************
\begin{exmusic}
% \inputexmusic{Irizar-Si_los_sentidos}
\caption{\worktitle{Si los sentidos queja forman del pan divino}, Miguel de Irízar (\signature{E-SE}{5/32} \XXX examples}
\label{exmusic:Irizar-Si_los_sentidos}
\end{exmusic}
%*************

Irízar creates a steadily increasing sense of excitement through shifts of rhythmic motion and style.
The setting of the opening phrase suggests a tone of hushed awe: the voices sing low in their registers and pause for prominent breaths that emphasize \gloss{queja}{complaint} and \gloss{Pan divino}{divine bread} (\measurenums{1--9}).
The harmonies here change less frequently than in the following sections, creating a relatively static feeling for this introduction.
In \measurenum{10} Irízar has the ensemble switch to ternary meter and increases the rate of harmonic motion.
Irízar turns the word \foreign{no} into an interjection with strong entrances on normally weak beats (\measurenums{13--14}).
When Irízar returns to duple meter in \measurenum{18}, the voices move in \gloss{corcheas}{modern eighth notes} and exchange shorter phrases, so that the tempo feels faster (and the actual tempo could certainly be increased here in performance).
Each choir's entrances become more emphatic, repeating tones in simple triads, and Irízar adds more offbeat accents and syncopations, particularly for \foreign{no se den por sentidos los sentidos} in \measurenums{24--32}.
The estribillo builds to a climactic \term{peroratio} with th evoices breaking into imitative texture in descending melodic lines.

The distinguishing stylistic characteristics of the setting suggest that Irízar is evoking a musical battle topic, a style one may find in \term{batallas} for organ as well as other villancicos on military themes.%
  \begin{Footnote}
  [Cite keyboard 2ry source on batallas]\XXX
  Keyboard examples include the \term{batallas} in Martín y Coll's \worktitle{Huerto ameno de flores de música}\XXX, and in [Portuguese collection]\XXX and [Bruna works]\XXX.
  Another villancico in this style is Antonio de Salazar's \worktitle{Al campo, a la batalla} (\signature{MEX-Mc}{A28}).
  \end{Footnote}
Battle pieces typically feature a slow, peaceful introduction followed by sections in contrasting meters and styles and a texture of dialogue between opposing groups (as in between high and low registers on the keyboard). 
Typical of the style is the reiteration of \musfig{5}{3} (\soCalled{root-position}) chords with the bass moving by fourths and fifths, and the 3-3-2 syncopations (\measurenums{25--26, 31--32, 43--46}).\citXXX[show an example for comparison]

Irízar sets the coplas, by contrast, in a sober and deliberate style.
The melody moves more calmly in duple meter with melodic phrases that fit well with the rhetorical structure of the poetic strophes. 
Irízar has the treble soloist sing the third and fourth lines of each strophe in short phrases, where the latter phrase repeats the former down a fifth, creating a feeling of \quoted{on the one hand} and \quoted{on the other hand} that suits the general philosophical tone of the verses and matches the specific poetic phrasing of these lines.\citXXX[example]
To recall the Jesuit formula, Irízar's estribillo seem more designed to delight, while the coplas provide more of an opportunity to teach.

Irízar's villancico seems to speak to a large crowd through grand, unsubtle gestures and sharp contrasts of bright colors.
By contrast, Jerónimo de Carrión's later setting of the same poem (\exmusicref{exmusic:Carrion-Si_los_sentidos}) invites a more personal reflection on the nature of sensation.
Carrión was capable of the festival style as well (as in \worktitle{Qué destemplada armonía}, which almost takes on the dimensions of a cantata \XXX[signature]), but his \worktitle{Si los sentidos} actually represents a separate subgenre of villancico, the chamber villancico or \term{tono divino}.\citXXX[on public/festival vs private/chamber villancicos]
Tonos may have been performed during times of Eucharistic adoration, such as the Forty Hours' Devotion or daily \term{siesta} services.
They may have been offered as an accompaniment to relatively private deovtional practices like lighting candles, praying, and meditating before the Tabernacle containing the consecrated hosts.\citXXX[on siesta services etc]

Carrión's setting is a solo continuo song in a style similar to the \soCalled{high Baroque} music of contemporary Italy, with a tonal harmonic language and a running bass part in the accompaniment.
In contrast to the declamatory, phrase-by-phrase setting of Irízar, Carrión's setting moves continuously in one affective manner from beginning to end.
The dialogue and rivalry of the poetic text happens now not through textural contrasts between choirs but through motivic exchanges between voice and accompaniment.
Instead of metrical contrasts from one section to the next, Carrión creates rhythmic contrasts between simultaneous voices.
Carrión dramatized \foreign{queja} (\measurenum{2}) with a metrical disagreement between the two voices (normal CZ ternary motion versus the voice's sesquialtera).
The descending pattern of leaps for \foreign{porque lo que ellos sienten} perhaps suggests the confusion and tumult of the senses, and it creates a certain amount of rhythmic confusion as it moves between voices.
Carrión creates a climax through a canon between soloist and accompaniment in \measurenums{18--20} that leads the singer to the top of his register.
The upward leaps in the last line on \foreign{no se den} (\measurenum{16}) contrast with the downward leaping motive of the opening (on \foreign{sentidos}, \measurenum{1}).

These two settings of \worktitle{Si los sentidos} demonstrate, through their similarities, the persistence of concerns about the senses and a theology of hearing as the sense favored by faith.
Meanwhile the difference between the two versions show the contrasts between generations and between different performance contexts of the villancico---Irízar's would suit the processions and plays of the Corpus Christi festival while Carrión's seems designed for more private Eucharistic devotion.
Both settings are musical discourses on sensation and faith, but they invite a different kind of participation from listeners.
Irízar and Carrión take a verbal discourse on sensation and faith, in which music is the paradigm of something that pleases the ear, and bring it to life through actual music.
Thus the pieces seem designed to teach listeners how to hear music even as they are listening---they accomodate hearing while training it, as the catechism says.

%*****************
\subsection{Sensory Confusion}

While the \worktitle{Si los sentidos} villancicos may not project as much uncertainty about sensation and faith as does Calderón's \worktitle{El nuevo palacio del Retiro}, they still emphasize the need for all the senses to submit to faith, which means that listeners should not trust their senses alone.
Some villancico poets and composers go further than stating that senses can be deceitful; they use paradox to deliberately confuse the senses for pious purposes.
The simplest form of this, as discussed in chapter 1, is to create auditory \soCalled{special effects} like echoes, voices imitating instruments or one instrument imitating another (like \term{chirimías} for \term{clarines}), and voices imitating birdsong.
A parallel trend in visual art might be the rise of \term{trompe l'oeil} effects in the later seventeenth century, after the decline of Velásquez-style realism, like the illusion of the heavens opening in the \term{Transparente} of Toledo Cathedral.\citXXX[Velasquez, baroque art, transparente]

Villancicos with \soCalled{synesthetic} topics mismatch the senses in the spirit of paradox and enigma.%
  \begin{Footnote}
  \autocite{DoetschKraus:Sinestesia} explores connections between poetic \quoted{synthesis of the senses} in Spanish verse and the actual neurological phenomenon of synesthesia.
  \end{Footnote}
Sight and hearing are the principal objects in the anonymous fragment \worktitle{Porque cuando las voces puedan pintarla} (\quoted{If voices could only paint her}).%
  \footnote{\signature{E-Mn}{M3881/44}.}

Cristóbal Galán, master of the Royal Chapel (1680--1684, previously chapelmaster at Segovia Cathedral) juxtaposes hearing and vision in a villancico for the conception of Mary.%
  \footnote{\signature{D-Mbs}{Mus.\ ms. 2893}, edition in \autocite[567--568]{CaberoPueyo:PhD}.}
Galán's text exhorts listeners to \quoted{hear the bird}---probably a reference to the Holy Spirit as a dove---and \quoted{see the voice}.
The poem makes \quoted{confusion} of sight and hearing, which is projected partly through irregular poetic meter.%
  \footnote{The division into lines is speculative, but the syllable counts and line groupings in this arrangement could be scanned as 10.6 10.8 7 6.6.6 10.10.}
\begin{quotetranslation}
\begin{verseoriginal}
Oigan todos del ave los luces \\
y miren la voz \\
que ellos hablan con lenguas de fuego \\
y ellas con rayos del sol. \\
¡Qué equivocación! \\
pues voces y luces \\
se miran se oyan \\
con tanto primor \\
que la luz se oye brillar, \\
cuando pura se mira la voz.
\end{verseoriginal}
\begin{versetranslation}
Everyone hear the bird's lights\footnote{Its appearance, or perhaps its colors.} \\
and see the voice, \\
for the lights speak with tongues of fire \\
and the voices, with rays of the sun. \\
What confusion! \\
for voices and lights \\
are seen, are heard \\
with such virtuosity \\
that the light is heard to shine, \\
while in purity the voice is seen.
\end{versetranslation}
\end{quotetranslation}

In his musical setting for eleven voices in three choirs, Galán creates \term{equivocación} through rhythm, notation, and texture.
Galán juxtaposes the three voices of Chorus I against the other two choirs by having Chorus I sing primarily in a normal triple-meter motion (with dotted figures intensifying the ternary feeling), while the other two choirs interject \foreign{¡Oigan!} and \foreign{¡Miran} in sesquialtera rhythm.
To notate these rhythms, Galán must use white notes for the regular ternary motion in Chorus I, but blackened noteheads (mensural coloration) to indicate the hemiola pattern in the other choirs.
But when each voice in Chorus I sings the synesthetic phrase \quoted{the light is heard to shine}, Galán turns the lights out---on the page at least---by giving each voice a passage in all black-note sesquialtera (\figureref{fig:Galan-Oigan-coloratio}).
They return to white notation again for the following phrase, \quoted{while the voice is seen in purity}.
Any attentive listener could hear these juxtapositions and abrupt shifts in rhythmic patterns, though only the musicians themselves would likely have been in on the dark--light symbolism in the notation.

%***************
\begin{figure}
\includeWideFigure{figures/Galan-Oigan-coloratio}
\caption{Galán, \worktitle{Oigan todos del ave} (\signature{D-Mbs}{Mus.\ ms.\ 2893}), Tiple I-2, end of estribillo: Ironic play of coloration}
\label{fig:Galan-Oigan-coloratio}
\end{figure}
%***************

In texture Galán plays with \foreign{qué equivocación} literally by setting these words to a fugato on a long ascending scalar figure (starting with the Tiple II's stepwise ascent A\octave{4}--G\octave{5}).
In addition to the literal pun of \quoted{equal voices} for \foreign{equivocación}, the sudden outburst of polyphonic texture in the midst of primarily homophonic polychoral dialogue could create a more affective sense of confusion as well.
As the estribillo continues, Galán increasingly mixes up the music for \foreign{Oigan todos del ave}, the sesquialtera interjections, and the contrapuntal texture of \foreign{qué equivocación}, between the various choirs.

The visual language in this villancico evokes the common iconography of the Holy Spirit as a dove surrounded by golden rays.
One possible performance venue for this piece would have been the Monasterio de la Encarnación in Madrid, where the Royal Chapel frequently performed.\citXXX{}
Galán and his musicians might have seen\citXXX{access?} the ceiling of the convent's \gloss{Capilla del Cordero}{Chapel of the Lamb}, where there is an early seventeenth-century image of the Holy Spirit as a dove bursting forth from the Trinity in a blaze of golden beams.
  \autocite[69--70]{Sanz:GuiaDescalzasEncarnacion}
The same image was incorporated as the central element atop the high altar of the later church building.%
  \autocite[81]{Sanz:GuiaDescalzasEncarnacion}

The Encarnación's archive preserves another synesthetic villancico, dated 1702 in the parts, Matías Juan de Veana's \worktitle{Flor volante, ave fragrante} (Flying flower, fragrant bird).%
  \footnote{\signature{E-MO}{AMM4131}.}
It is notable for engaging not only sight and hearing but also smell as it mixes up flower and bird.

These pieces describe and seek to incite a condition of sensory overload, an ecstasy in which all the senses blend together in the effort to grasp something that is beyond them.%
  \begin{Footnote}
  This notion is part of the Catholic heritage that shaped the Olivier Messiaen's concept of \quoted{dazzlement}.
  Messiaen saw color when he heard sound (though he denied this was clinical synesthesia), and his theological aesthetic favored sensory overload as a path to the ineffable \quoted{world beyond}.\XXX
  \end{Footnote}
They seem intended to provoke listeners to a higher form of sensation, to a holy dismay and wonder that would lead to true, faithful perception.


%*******
\subsection{Sensory Deprivation}

From villancicos about too much sensation we may turn to villancicos that treat themes of sensory deprivation.
Since Catholics considered sight inadequate for the Eucharist, many Corpus Christi villancicos sing of blindness.
In a 1688 villancico \quoted{al Santísimo Sacramento} by Francesc Soler (\circa 1625--1688, chapelmaster of Girona Cathedral, 1682--88), the chorus delivers a moral warning and devotional challenge to themselves and their hearers: \quoted{to look without seeing}.%
  \footnote{\signature{E-Bbc}{M733/6}.}
\begin{quotetranslation}
\begin{verseoriginal}
Atender sin mirar, \\
mirar y no atender \\
es mucha fe, \\
es mucho adorar \\
y yo para creer \\ 
miraré para cegar.
\end{verseoriginal}
\begin{versetranslation}
To pay attention without looking, \\
to look and not pay attention \\
is great faith, \\
is great worship, \\
and I, in order to believe, \\
will look in order to be blind. 
\end{versetranslation}
\end{quotetranslation}

The wordplay in this poem strains the language almost beyond communication.
\foreign{Atender} can mean \quoted{listen}; so the first line might be stating the primacy of hearing over sight in Eucharistic faith.
But \foreign{atender} also means \quoted{pay attention}, suggesting that one must take heed of the sacrament, to \quoted{see} beyond the accidents of bread to discern its hidden substance.
To do this one must follow the paradoxical reversal of the second line, and pay no attention to what one sees externally.
The coplas (as is their usual function) make the riddle of the estribillo a bit clearer:
\begin{quotetranslation}
\begin{verseoriginal}
3. Ya es desengaño a los ojos \\
la luz que lisonja fue \\ 
pues hoy que esconde por clara \\
y por obscura también.

5. Si a ver imposible aspiras \\ 
porque imposible lo ves, \\
ceda la curiosidad \\
al respecto alguna vez.
\end{verseoriginal}
\begin{versetranslation}
3. Surely it is a deceit to the eyes, \\
the light that was a flattery, \\
since today it hides in the bright \\
and in the shadow as well.

If you aspire to see impossibly\footnote{Or, to see the impossible.} \\
because you see it as impossible, \\
at such a moment let curiosity \\
give way to respect.
\end{versetranslation}
\end{quotetranslation}

Spanish Catholics participated in the Eucharist primarily through seeing rather than eating,\citXXX{} and Spanish churches supplemented this visual devotion by offering the eyes an inexhaustible surplus of objects to contemplate.
The high-altar \term{retablo} of Seville Cathedral is one of the more overwhelming examples, crammed with twenty-eight separate scenes scuplted in three dimensions, gilded and painted in bright colors.\citXXX{}
Each scene itself, such as the Last Supper, contains multiple individual figures, all adorned with the maximum of gilded Gothic filigree.
There was indeed much to \quoted{attend to} in a Spanish church, and this meant not only seeing but also discerning the meaning behind the sacred symbols and stories represented all around, and the connection between these and the hidden mysteries of the Eucharistic sacrament consecrated in their midst.

Thus Soler's villancico challenges listeners \quoted{to see impossibly}, to strain against the limits of the senses to a deeper kind of discernment.
The resulting spiritual perception is compared to \quoted{blindness}---not seeing too little but seeing more than can be sensed, like St.\ Paul blinded on the raod to Damascus because of the brilliance of light with which Christ revealed himself.
Like Paul, Catholic worshipers could learn to be blind to simple appearances and instead rely in faith on hearing the Word of Christ.

%******************
\section{Impaired Hearers, Incompetent Teachers: \quoted{Villancicos of the Deaf}}


%%% Local Variables:
%%% mode: latex
%%% TeX-master: "../main"
%%% End:
