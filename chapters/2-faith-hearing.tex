% Andrew Cashner -- Faith, Hearing, and the Power of Music
% Chapter 2 -- Making Faith Appeal to Hearing

% 2016-08-31    Revision for book begun

%*******************************************
\label{ch:faith-hearing}

\epigraph
{For I doubt it now, although I know it,\\
because I have listened to Faith without faith.}
{Calderón, \worktitle{El nuevo palacio del Retiro} (1634), \linenums{1319--1320}}

\quoted{How are they to believe if they have not heard?} St.\ Paul asked, since \quoted{faith comes through hearing, and hearing, by the Word of Christ} (\scripture{Rom 10:16-17}).
Villancicos in the Spanish Empire of the seventeenth century made faith audible in a way that potentially appealed to the ears of elite and common people alike.
We have seen a variety of ways that villancicos represent music-making, and thus should be considered as musical discourses on music.
As these pieces were performed in and around Spanish churches, they proclaimed and embodied religious beliefs about the relationship between music and faith.
At the same time church leaders used the pieces themselves to cultivate faith.

How, then, did early modern Catholics understand the relationship between hearing and faith?
And how could the auditory art form of villancicos affect that relationship?
This chapter situates villancicos within the context of early modern Catholic discussions of faith and sensation.
As the theological sources discussed in the first section reveal, Hispanic Catholics experienced a certain anxiety about the role of individual sensory perception in acquiring and developing faith; and they harbored uncertainty about whether all people really had the capacity for faith.
Though many people acknowledged music's power to inspire faith, their concepts of music left it unclear exactly how people could develop their hearing faculties in order to derive a spiritual benefit from listening to music.

The second section interprets villancicos that explicitly address themes of faith and sensation, demonstrating that these pieces, too, reflect uncertainty and doubt about hearing's role in acquiring faith.
These pieces stage allegorical contests of the senses, represent sensory confusion (such as synesthesia), and represent characters whose impairments of hearing render them unable to understand religious teaching.

Understanding the theological environment in which villancicos were performed, and considering how villancicos in turn contributed to that environment, makes it clear that villancicos functioned as much more than vehicles for simplistic religious teaching.
Rather, villancicos provided listeners with opportunities to contemplate the challenges of faithful hearing.

%***********
\section{The Challenge of Making Faith Appeal to Hearing}

The global discoveries and religious upheavals of the sixteenth century propelled the Roman Church and its inherited medieval theological system into a chaotic new world.
The church's missionaries struggled to bridge cultural divides in order to bring people to faith both outside and within Europe.
At the same time, the conflicts with Protestants, who emphasized the centrality of personal  experience in faith, along with changing philosophical and physiological understands of humanity and its place in the cosmos, created anxiety and doubt about subjective experience.
Though Catholic apologists distanced themselves from the Protestant emphasis on personal faith, Catholics could not avoid what was a fundamental issue in Christianity---namely, that somehow a connection had to be made between the objects of faith, which were understood as transcendent truth, and the experience of the individual Christian subject.

Traditional Catholic notions of faith, formulated by St.\ Thomas Aquinas, held that faith meant more than assenting to intellectual propositions, believing that certain things were true.\citXXX
The technical term for that basic sort of faith was \term{fides informata} or unformed faith.\citXXX
Fully formed faith, \term{fides formata}, was a virtue (\term{virtus}) or capacity that \quoted{worked through} the higher virtues of hope and love.
True faith for Catholics meant a commitment of the whole person to live faithfully in communion with Christ through his body, the Church.\citXXX

A central document for understanding the faith of post-Reformation Catholics is the catechism produced \quoted{for the parishes} by authority of the Council of Trent.%
  \autocite{Catholic:Catechismus1614}
The bishops at the Council of Trent were not only responding to Protestantism; many of them sought to address the underlying problems that had allowed Protestantism to take hold in the first place.\citXXX{}
Chief among those problems was a lack of education both of clergy and laity.
Through an elegantly composed Latin catechism, church leaders hoped to educate their clergy so that the clergy could better instruct the parishioners under their care.

Vernacular expositions of the catechism, like those of Antonio de Azevedo, Juan Eusebio de Nieremberg, and Juan de Palafox y Mendoza, brought this teaching down to a more accessible level, still addressing a clerical reading audience, but often in a colloquial tone, with earthy illustrations and lengthy paraphrases of Scripture in Spanish.
These texts come alive when read aloud, and indeed their goal is to prepare pastors to teach unlettered disciples through words and voice.
These disciples might be Indians---in Spanish the term was used for indigenous peoples both of America and Asia---or Europeans, such as rural folk in mountain passes where Christianity had still not fully penetrated.
These books prepared teachers for the challenge of making faith appeal to hearing.

The official Roman Catechism grounds the Church's authority to preserve and teach \soCalled{the faith} in the theology of Christ's Incarnation.
The catechism teaches that God communicated his own nature to humanity by taking on human flesh in Christ, and therefore the true Word of God was not the Scripture or any body of doctrines, but rather Jesus Christ himself as the \term{logos} or \term{verbum} (\scripture{Jn 1:1}).
But while Truth ultimately would consist in knowing God in Christ, the catechism teaches that Christ founded the Church to be the means through which people would come to know him after his resurrection.
Christ appointed apostles, chief among them St.\ Peter, to be the custodians of the true faith from his time up until the present.

The Church as Christ's body was the community through which people came to faith in Christ and learned to live faithful lives after Christ's example.
Drawing on St.\ Paul's dictum that faith came through hearing, and hearing, by the Word of Christ, the catechism challenges the Church's ministers to find a way to make Christ the Word audible:
\begin{quote}
Since, therefore, faith is conceived by means of hearing, it is apparent, how necessary for acheiving eternal life are the works of the legitimate teachers and ministers of the faith. \Dots{}
Those who are called to this ministry should understand that in passing along the mysteries of faith and the precepts of life, \emph{they must accommodate the teaching to the sense of hearing and intelligence}, so that by these \add{mysteries and precepts}, \emph{those who possess a well-trained sense} should be filled up by spiritual food.%
  \begin{Footnote}
  \Autocite[2, 8--9 (emphasis added)]{Catholic:Catechismus1614}: 
  \quotedla{Cvm autem fides ex auditu concipiatur, perspicuum est, 
  quàm necessaria semper fuerit ad {\ae}ternam vitam consequendam 
  doctoris legitmi fidelis opera, 
  ac ministerium \Dots\ vt videlicet intelligerent, 
  que ad hoc ministerium vocati sunt, 
  ita in tradendis fidei mysteriis, ac vit{\ae} pr{\ae}ceptis,
  doctrinam ad audientium sensum, atque intelligentiam accomodari oportere,
  vt cùm eorum animos, qui exercitatos sensus habent, 
  spirituali cibo expleuerint \Dots.}
  \end{Footnote}
\end{quote}

As the emphasized phrases show, the Church taught that for faith to come through hearing, both the teacher and the listener had to be involved.
Antonio de Azevedo begins his vernacular introduction to the catechism with the notion that faith requires both wise teachers and attentive listeners.%
  \autocite{Azevedo:Catecismo}
For Azevedo, faith is epitomized in an ancient image he read about in Pliny, depicting \quoted{an elderly man sitting inside a temple, who had a harp in his hand, and who was teaching a boy who lay at his feet}.%
  \autocite[f.~1a]{Azevedo:Catecismo}
The temple, Azevedo explains, represents that faith should be \quoted{firm and fixed, and also that there must be masters who teach it, and disciples who listen to it; and that the master needs to be old and mature in age and faithfulness; because the teaching is serious, ancient, and of weight and substance}.
Moreover, Azevedo explains, the teacher is shown \quoted{with a musical instrument which gives pleasure to the ear}:
\begin{quote}
So that we should understand that Faith enters through the ear \add{\foreign{oído}}, as St.\ Paul says, 
and that the disciple should be like a child, simple, without malice or duplicity, without knowing even how to respond or argue, but only how to listen and learn.
Thus this image depicts for us elegantly, what the hearer of the Faith \add{\foreign{el oyente de la Fe}} should be like.%
  \begin{Footnote}
  \Autocite[f.~1b]{Azevedo:Catecismo}:
  \quotedsp{Para pintar los Romanos la Fe, lo primero que hizieron templo y altar \Dots{} Numa pompilio \Dots{} puso vn idolo:
  de forma de vn viejo cano, que tenia vna harpa en la mano, i estava enseñando vn niño echado a sus pies.
  En esta figura o geroglifica esta encerrada mucha filosofia, i aun cristiana.
  En el templo i ara denota, que la fe a de ser firme i fixa, no movediça, ni flaca, que a cada ayre de novedad se mueva
  I tambien que a menester maestros que la enseñen, i dicipulos que la oygan.
  I, que el maestro a de ser anciano, i maduro en edad y bondad.
  Porque la dotrina es grave, antigua, i de tomo i sustancia. 
  No nueua, ni de pocos años sino antigue dende los Apostoles, i con instrumento musico que da gusto al oido.
  Paraque entendamos, que la Fe entra por el oido; como dize S.\ Pablo.
  I que el dicipulo sea como niño, sencillo, sin malicia ni doblez, sin saber ni replicar, ni arguir, mas de solo oir, y deprender.
  En lo qual nos dibuxa galanamente, qual a de ser el oiente de la Fe.}
  \end{Footnote}
\end{quote}
The teacher's task according to Azevedo, then, is not only to make the faith heard, but to make it \quoted{appeal to the ear}, just as he says music does; and the disciple's task is simply to listen and take heed.

But such teaching was limited by the sensory capacity of each listener, and therefore the Roman Catechism argues that the task of listening requires training.
The catechism exhorts its teachers to accommodate the limitations of their listeners' senses (\term{sensus}) even as they train their hearers to listen profitably.
This emphasis on accommodation was counterbalanced by the catechism's statement that the disciples who will receive the benefit of the teaching are those whose senses have been properly trained.
Pastors had to accommodate the ear even while training it.

The theology of the catechism might suggest that the value of music in propagating faith would come from the medium's ability to make the faith appeal to the ears of listeners.
Villancicos, as an auditory medium based on vernacular poetry, would seem like an ideal vehicle for this project.
If teaching should appeal to the ear \emph{like} music, as Azevedo says, then combining teaching with actual music would appeal all the more.
At the same time, the challenge of training the sense of hearing would seem to be multiplied with music, since a listener must learn to perceive musical structures in order to gain benefit from the music.

\subsection{Experiencing Spiritual Truth through Music}

This problem---that music has the ability to make faith appeal to the ear, but that the ear must be trained to hear it---may be seen in the explanations of music's power by Athanasius Kircher.
The Jesuit writer is one of the few who attempted to articulate exactly how music worked to propagate faith.\citXXX
After describing the miraculous powers of preaching reported to him by his missionary colleagues, Kircher discusses what music adds to the spoken word.
\begin{quote}
If a preacher should wish by the power of God to move a devout person to heavenly things, so that the listener is given over in meditation in otherworldly affects and raptured in his mind,
and if the preacher should take some notable theme expressed in words,
which would recall to the hearer's memory the sweetness of heavenly things and their mildness,
and then fittingly adapt that verbal theme through cadences and intervals in the Dorian mode,
the \emph{the listener could experience that what whas said is actually true}, 
since through harmonic sweetness he could be transported beyond himself by those heavenly things,
carried away by joy to where those things are true.%
  \begin{Footnote}
  \autocite[bk.\~7, 550 (emphasis added)]{Kircher:Musurgia}:
  \quotedla{Si quis Deo deuotum hominum rerumque c{\oe}lestium, meditationi deditum in exoticos affectus raptusque mentis commouere vellet is supra insigne aliquod verborum thema, quod rerum c{\ae}lestium dulcedium, \& suauitatem auditori in memoriam reuocaret, modulo dorio per clausulas interuallaque aptè adaptet, \& experietur quod dixi verum esse, statim extra se factos dulcedine harmonica eò, vbi vera sunt gaudi rapi.}
  \end{Footnote}
\end{quote}

Kircher's depiction of music's power goes well beyond the Jesuit formula of \quoted{teaching, pleasing, and persuading}.\citXXX[for Jesuit formula: Bailey?]
Music possesses its own powers, Kircher says, that go beyond what can be \quoted{expressed in words}.
In fact, the right kind of music could cause the listener \quoted{to experience the truth of what was said}.
Thus music not only makes the teaching of doctrinal truth appealing and persuasive; it actually transforms listeners, transporting them to a realm of heavenly truth, through affective experience.

For Kircher, music links the objective truth with subjective experience through the unique ways that music affects the human body.\citXXX[Kircher discussion of affects, passions]
Affective content in Kircher's theory is somehow written into the music through rhetorical tropes and intrinsic properties of music's natures, as well as being embodied by the performers.
Through principles of sympathetic vibration, the humoral-affective properties could be transferred from composer and performer to listeners.

But Kircher never fully resolves the problem of subjectivity in this process: namely, that each listener perceives music differently based on both cultural background and individual humoral temperament.
Kircher acknowledges that music does not communicate the same things, or produce the same effects, for listeners of different cultures.\citXXX[on cultural conditioning]
Moreover, he has no single answer to explain what (for example) music with a high melancholic component would do to a listener of a particular temperament.
Would the melancholic listener overflow with black bile and experience a cathartic balancing of his affections?
Or would he receive a fatal dose of melancholy that would be detrimental to his health?
When Kircher describes the \quoted{fitting} adaptation of preaching to \quoted{cadences and intervals in the Dorian mode}, he does not explain how one acquires the necessary knowledge and capacity to hear those musical structures and derive the intended benefit.

%****************
\subsection{The Danger of Subjective Experience in Faith}

The capacity to listen faithfully, and therefore music's power to make faith appeal to hearing, would then be limited by cultural conditioning as well as by personal subjectivity.
Regarding individual religious experience, there was a creeping anxiety within early modern Spanish culture regarding the relationship of the senses to faith.
This anxiety made the power of music potentially dangerous.

The question of what role individual subjective experience played in acquiring faith had vexed the Western church since before the start of the Reformation.
As we have already noted, pre-Reformation theologians taught that faith was a virtue, which had to be \quoted{formed} by working through hope and charity.
Unformed faith, or simple belief in certain ideas, was primarily a matter of the head or intellect; but fully formed faith was a matter of the hands.
\term{Fides formata} was really \quoted{faithfulness}, a conviction that manifested itself through ethical behavior in fidelity to God's will.

In the early sixteenth century, the Catholic Humanists such as Erasmus, who was influenced by the Devotio Moderna in which he was raised, stressed the need for an affective spirituality of the heart that could unite head belief with the faithfulness of the hands.\citXXX{}
For Erasmians, this heartfelt faith was to be the source for reformation of the individual and society, the Christian path to achieving the educational goals of the Renaissance---namely, by following the Classical models of Plato and Cicero to produce men of virtue and therefore a just society.\citXXX[Erasme en Espagne, etc]

At the same time, however, Martin Luther radically redefined \foreign{fides} as trust in the gospel of salvation through Christ alone, and separated \foreign{fides} from \foreign{virtus}---\soCalled{faith} from \soCalled{works}.
For Thomas More and other Catholics who polemicized against Luther, Luther's theology turned his followers away from the external, trustworthy, institutional Church and its objectively operating sacraments, and left them with nothing but a subjective internal experience as assurance of salvation.%
  \autocite[\XXX, also More]{Schreiner:Certainty}
Further, as Catholics understood the Protestant position, faith was no longer connected the cultivation of a just society, but was a purely individual matter.

Regardless of whether this critique was fair, the Catholic reaction to Luther produced widespread anxiety in Spain regarding the role of subjective sensory experience in faith.
The religion of the heart that had been promoted by the Humanists as a path to sanctity came to be seen by some after Trent as the gateway to heresy.\citXXX{}
As a result the Spanish Inquisition put heavy pressure on groups emphasizing individual spiritual experience, such as the mystically inclined \term{alumbrados} and the Carmelite reformers Teresa of Ávila and John of the Cross.
Teresa was one of many who claimed no authority for her teaching except her own visions of God, but unlike many others (such as the \term{beata} Francisca de los Apóstoles) she managed to deflect official suspicion and dodge Inquisitorial censorship.%
  \autocites[\XXX]{Ahlgren:TeresaPolitics}{Francisca:Inquisition}

John of the Cross's \worktitle{Ascent of Mount Carmel} uses \soCalled{negative theology}---moving toward God by contemplating what God is not, since God is beyond anything the human mind can imagine---to distance contemplatives from their visions and sensations.
By willingly depriving themselves of sensory experience they might come closer to union with the God who was beyond sensation.
John defines that union not in sensual terms but in ethical ones, as the total surrender and conformity of one's will to God.\citXXX[John plus secondary]
Ignatius of Loyola's \worktitle{Spiritual Exercises}, the foundation of Jesuit spirituality, are dedicated entirely to the task of discerning whether one's religious sensory experiences are truly from God.\citXXX{}

In such a climate, music's power over the sense and affects came under suspicion as well.
While some Catholics such as the Jesuit missionaries (despite their founder's suspicion of music) were eager to use this power to advance the cause of the Church, others saw the use of music as a potential distraction or distortion.\citXXX[who?]
In part this stemmed from the Spanish fascination with illusion (\foreign{engaño}) and the potential decpetion of the senses, as scholars of Spanish literature like Cervantes's \worktitle{Don Quixote} and Calderón's \worktitle{La vida es sueño} have long understood.\citXXX{}
Music's power over the senses and affects might be used for the purposes of cultivating faith, but this power had to be carefully controlled and submitted to reason to mitigate the dangers of individual subjectivity.

%*********************
\subsection{The Need for Cultural Conditioning in Hearing}

On the cultural side, the age of exploration required European Catholics to seek a distinction between the core of Christian religion---that which was to be preserved without chang in all cultural settings---and specific cultural expressions or incarnations of Christianity, which could be changed.\citXXX[on inculturation, acculturation, transculturation,etc]
The Chinese Rites Controversy with the Jesuits, in which the bishop of Puebla, Juan de Palafox y Mendoza, played an active role, was only one example where that distinction proved difficult to delineate.\citXXX[Chinese rites, Palafox]

To adapt the formulation of Ines Zúpanov, in her study of Jesuit missions in India, the \quoted{tropics} represented both a geographic zone and a process of cultural transformation (troping, turning).\citXXX[Zupanov]
Some missionaries like the Jesuits in Japan and Brazil actively sought to accomodate local customs and music; but everywhere that missionaries brought Christian faith, the process of cultural translation inevitably transformed it into something neither they nor their converts could necessarily predict.\citXXX[Japanese mission, Bailey, Castagna Brazil]

There is ample documentary evidence that native converts performed plainchant and classical-style polyphony like that of Palestrina and  Guerrero in Catholic missionary settlements from Luanda (Angola) to Goa (India) and Cuzco (Peru).\citXXX
It is also evident from ethnography today that the peoples of central Africa, south Asia, and the Andes sing with distinctive modes of vocal production, intonation, and rhythmic and melodic variation. 
In a history of Jesuit mission art, Gauvin Bailey points out the discrepancy between the Jesuits' accounts of art on their missions, which stress for supervisors in Europe how closely their converts had learned to imitate European models, and the surviving evidence of that art, which often shows strong indigenous influences.\citXXX[bailey]
There is every reason to think that mission music, even plainchant and polyphony imported directly from Europe, sounded quite different in actual performance from its European models.
As Protestant missionaries in New England observed, even when Indians sang the same songs, \quoted{they differ from us in sound}.\citXXX[goodman]
  
In Mexico, missionary friars allowed the Mexica people to continue singing and dancing in Nahuatl, and Bernardino de Sahagún even provided a lectionary of psalms for them to sing in their native language.\citXXX[Candelaria]
But the Mexican bishops\XXX{} complained that the natives' songs were written in such complex language that they lacked linguists sufficiently skilled to validate their orthodoxy.\citXXX{}
Lorenzo Candelaria has suggested that this anxiety about the content of songs in native languages, rather than any concern about styles of European church music, was actually the primary motivation for the Council of Trent's decree requiring the words of church music to be intelligible.\citXXX[Candelaria, Trent]

As the Church was adapting itself to native sensibilities, or being adapted by colonial subjects in ways the Church could not fully control, the question naturally arises, which parts of Christianity constituted \soCalled{the Faith} that was supposed to come through hearing?
And further, if hearing was culturally conditioned, then how might one reliably use music to appeal to that sense?

The problem of acquiring the capacity to hear properly---of training the sense, as the Roman Catechism puts it---is plainly stated in a 1590 dialogue that represents the impressions of four Japanese noble youths after the Jesuit missionaries took them on a grand tour of Spain and Italy between 1582 and 1590.%
  \autocite{Sande:DeMissioneLegatorum}
Their trip from Nagasaki to Rome took them to most of the major Iberian musical centers discussed in this study (not to mention the most important cities in Italy): on the outgoing trip, to Lisbon, Évora, Toledo, Madrid, and Alcalá; and on the return, to Barcelona, Montserrat, Zaragoza, and Daroca.
The leader of the Jesuit mission to Japan, Alessandro Valignano, hoped to persuade the authorities of his order and church that \quoted{European Jesuits must accommodate themselves to Japanese manners and customs}.%
  \autocite[4]{Massarella:JapaneseTravellers}

At the same time, Valignano recognized that the missionaries were asking the Japanese to accommodate a new culture as well. 
Therefore the boys' mission was both to represent Japan to Europe and on their return, to represent Europe to Japan.

Valignano and his Jesuit collaborators documented the trip in the form of a dialogue between the boys who traveled and their friends who stayed home.
Though ostensibly based on first-hand accounts of the Japanese \soCalled{legates}, the book reflects how the European missionaries hoped the Japanese would see Europe.

The Japanese boys had received training in music, and practiced and performed throughout their trip and upon their return.
In the dialogue, when their friends ask about European music, the boys tell them that it took them time to become accustomed to it, before they could recognize its superiority.%
  \footnote{\autocite[109--110]{Sande:DeMissioneLegatorum}, translation from \autocite[155-156]{Massarella:JapaneseTravellers}, emphasis added.}
In this excerpt the character Michael is one of the returned travellers---he corresponds to a real historical person---and Linus is one of his friends who stayed back in Japan.

%*******************
\begin{quotation}
\speaker{MICHAEL}
You must remember, as we said earlier, how much we are swayed by longstanding custom, or on the other side, by unfamiliarity and inexperience, and the same is true of singing. 
\emph{You are not yet used to European singing and harmony, so you do not yet appreciate how sweet and pleasant it is, whereas we, since we are now accustomed to listening to it, feel that there is nothing more agreeable to the ear.}

But if we care to avert our minds from what is customary, and to consider the thing in itself, we find that European singing is in fact composed with remarkable skill; 
it does not always keep to the same note for all voices, as ours does, but some notes are higher, some lower, some intermediate, and when all of these are skillfully sung together, at the same time, they produce a certain remarkable harmony \Dots{} 
all of which, \Dots{} together with the sounds of the musical instruments, are wonderfully pleasing to the ear of the listener. \Dots{}

With our singing, since there is no diversity in the notes, but one and the same way of producing the voice, we don't yet have any art or discipline in which the rules of harmony are contained; 
whereas the Europeans, with their great variety of sounds, their skillful construction of instruments, and their remarkable quantity of books on music and note shapes, have hugely enriched this art.

\speaker{LINUS}
I am sure all these things which you say are true; for the variety of the instruments and the books which you have brought back, as well as the singing and the modulation of harmony, testify to a remarkable artistic system.
Nor do I doubt that \emph{our normal expectations in listening to singing are an impediment when it comes to appreciating the beauties of European harmony.}%
  \begin{Footnote}
  {\XXX Complete original text}
  \end{Footnote}
\end{quotation}
%*******************

From this perspective, then, music could be a way to make faith pleasing to the ear, as Kircher describes, but only if the ear was trained in order for it to find such music pleasing.
This conclusion would apply on both individual and social levels; that is, both personal subjectivity and cultural conditioning were involved in the process of hearing music.

Using music to propagate faith meant appealing to the ear and training it at the same time. 
Since Catholics (as exemplified in the Tridentine Catechism) believed faith to have a social dimension---a faithful life as part of the community of the church---inculcating faith through music meant more than a one-on-one dialogue between \soCalled{the music} and \soCalled{the listener}.
The musical ritual of the post-Tridentine Church involved a large number of community participants, for whom performing music with the body and hearing it were inextricably linked.

Moreover, the musical efforts of the colonizing church concretely built social relationships through musical training.
Bringing a man to faith meant setting him on a path to becoming a whole, virtuous person in the image of Christ (\foreign{virum perfectum} in the words of the Catechism, the root of \foreign{virtus} being \foreign{vir}).%
  \autocite[8]{Catholic:Catechismus1614}
At the same time, the virtue of man as Neoplatonic microcosm was reflected in the broader society and in turn depended on it. 

Teaching faith, then, meant trying to establish not just individual Christians, but also building a Christian society as the body of Christ.
This is why the friars in Mexico not only started parishes, but they also trained choirs.
Catholic music was not \emph{about} society; it was a form of society.
Forming choirs of boys and training ensembles of village musicians in colonial Mexico were practical means of establishing the Church.
The economic aspect of paying for music and musicians, and the political aspect of creating social organizations and putting on public spectacles, though seemingly secular, were still an important part of the church's mission.

In other words, the evangelizing mission could not be easily separated from a civilizing mission, any more than the Church in Europe after Trent could separate its duty to preserve purity of doctrine from the need to unify liturgical practice. 
In the terms of Boethius, which was promulgated in the influential Spanish music treatises of Pedro Cerone, well-tuned \term{musica instrumentalis} could harmonize the \term{musica humana}---the harmony of the individual in body and soul, reason and passion, but also the concord of human society.\citXXX[Boethius, Cerone]
All this could better reflect the cosmic \term{musica mundana}, and ultimately the harmonies of the triune God.

%***********************
\subsection{Obstacles to Faith and Mistrust of Hearing}

But even if music could be adapted for individual temperaments and translated across cultural differences, what if a person simply lacked the proper disposition to hear the Word with faith?
Theologians debated, at some remove from the human beings at the center of their disputations, whether Indians and Africans truly had the capacity for faith.\citXXX{}
Most concluded that these groups could be converted from the Satanic darkness of their paganism despite severe handicaps in reason and moral sense.
Someone who was not fully a \gloss{vir}{man} could not be expected to arrive at the fullness of \term{virtus}.

Jews, on the other hand, were not accorded even this limited hope for salvation.
Even those Spanish subjects of Jewish descent who claimed to have converted---the \term{conversos}---were viewed with suspicion and subject to inquisitions at both the official and vigilante levels.\citXXX{}
In the genre of allegorical mystery play performed on  Corpus Christi across the Spanish empire (the \emph{auto sacramental}), playwrights made the character \term{Judaísmo} represent the perpetually unbelieving Jew.\citXXX{}
(\term{Hebraísmo} represented the Old Testament believers whom Catholics considered to be part of the Christian church.)\citXXX{}
Villancicos were typically performed before and after these plays, and the \term{auto sacramental} provides a crucial context for understanding the theological environment in which villancicos were heard.

In a politically important Corpus Christi play by Spanish court poet Pedro Calderón de la Barca, the figure of \term{Judaísmo} becomes a vivid representation of the incapacity to acquire faith, \term{despite} the sense of hearing.
Performed in 1634 to inaugurate Philip IV's new palace, the Buen Retiro, Calderón's \worktitle{El nuevo palacio del Retiro} centers on Judaism's incapacity for faith.%
  \autocite{Calderon:Retiro}
Judaism is forcefully excluded from the festivities celebrated within the play, which culminate with the consecration of the Eucharist.
Instead Judaism stands to the side and asks the character Faith to explain each event to him.
Faith answers with elaborate allegories.

When the Eucharistic host is consecrated and elevated, Judaism responds with a long monologue in which he attempts to understand the mysterious wafer by connecting it with stories from Hebrew Scripture such as the manna in the desert and the dew on Gideon's fleece.\citXXX[bible passages]
But Judaism cannot accept any of these explanations.
In fact he is unable to believe what Faith has explained to him, because, as he says in an increasingly embittered refrain (the source of this chapter's epigraph), \quoted{I have listened to Faith without Faith}.

%*****
\begin{verse}
Who are you, that I see you and I do not know it,\\
because I have listened to Faith without faith? 

\Dots\\
Are you that manna that quenched thirst\\
and satisfied hunger?\\
Are you the fruit that is dangling from the tree\\
that gave knowledge of good and evil?\\
Are you the serpent that gave us healing\\
staked to the staff of Moses?\\
For I doubt it now, even though I know it,\\
because I have listened to Faith without faith. 

But surely you must be the flower of Jericho,\\
surely you must be the lily of the valleys,\\
white manna that rained from Heaven for us,\\
pale dew that dampened the fleece,\\
dangling serpent, fire that showed the way,\\
forbidden fruit, rejected honey,\\
I cannot comprehend you nor know your enigma,\\
because I have listened to Faith without faith. 

And so, let him run to your singular goal\\
who can appraise your value,\\
for I will always doubt your being,\\
for I will never perceive your light,\\
because you are not the Host of my altar,\\
because you are not the sun of my setting,\\
because your dark cipher I did not comprehend,\\
because I have listened to Faith without faith. 

\emph{All the musical instruments play, shawms and snares, drums and trumpets,
and everyone enters crowned with wreaths, and with lances, as for battle, 
to the measure of the clarion.}%
  \begin{Footnote}
  Please see the appendix\XXX{} for the original text, from \autocite[\linenums{1303--1305, 1315--1336}]{Calderon:Retiro}.
  \end{Footnote}
\end{verse}
%******

If we recall how the Roman Catechism urged pastors to accommodate \quoted{the sense of hearing and intelligence}, the character Judaism's problem is neither with sensory ability nor even with his intellectual knowledge of doctrine: clearly he has listened carefully to Faith's explanations and graps their connections to Hebrew scriptures.
But something must be lacking in this character that permanently prevents him from moving from listening to \quoted{the Faith} into actual saving faith.

It seems no accident that Judaism's eloquent confession of unbelief is immediately drowned out by music.
As a procession enter bringing the King, Queen, and Man, the fanfares \quoted{to the  measure of the clarion} celebrate the King's triumph over Judaism through the sense of hearing.
For Calderón's listeners, who had been taught to regard Jews as the embodiment of willful unbelief and worse, the entry of the musicians would clear away the acrid sound of Judaisms's doubts.

% START phd/chapters/theology.tex line 324

% the portrayal of judaism probably meant to reinforce orthodoxy by portraying its opposite; but even when calderon highlights orhtodox doctrine of senses there is considerable doubt and uncertainty



%****************************************
\section{Villancicos on Sensation and Faith}


