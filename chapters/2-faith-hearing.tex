% Andrew Cashner -- Faith, Hearing, and the Power of Music
% Chapter 2 -- Making Faith Appeal to Hearing

% 2016-08-31    Revision for book begun

%*******************************************
\label{ch:faith-hearing}

\quoted{How are they to believe if they have not heard?} St.\ Paul asked, since \quoted{faith comes through hearing, and hearing, by the Word of Christ} (\scripture{Rom 10:16-17}).
Villancicos in the Spanish Empire of the seventeenth century made faith audible in a way that potentially appealed to the ears of elite and common people alike.
We have seen a variety of ways that villancicos represent music-making, and thus should be considered as musical discourses on music.
As these pieces were performed in and around Spanish churches, they proclaimed and embodied religious beliefs about the relationship between music and faith.
At the same time church leaders used the pieces themselves to cultivate faith.

How, then, did early modern Catholics understand the relationship between hearing and faith?
And how could the auditory art form of villancicos affect that relationship?
This chapter situates villancicos within the context of early modern Catholic discussions of faith and sensation.
As the theological sources discussed in the first section reveal, Hispanic Catholics experienced a certain anxiety about the role of individual sensory perception in acquiring and developing faith; and they harbored uncertainty about whether all people really had the capacity for faith.
Though many people acknowledged music's power to inspire faith, their concepts of music left it unclear exactly how people could develop their hearing faculties in order to derive a spiritual benefit from listening to music.

The second section interprets villancicos that explicitly address themes of faith and sensation, demonstrating that these pieces, too, reflect uncertainty and doubt about hearing's role in acquiring faith.
These pieces stage allegorical contests of the senses, represent sensory confusion (such as synesthesia), and represent characters whose impairments of hearing render them unable to understand religious teaching.
Understanding the theological environment in which villancicos were performed, and considering how villancicos in turn contributed to that environment, makes it clear that villancicos functioned as much more than vehicles for simplistic religious teaching.
Rather, villancicos provided listeners with opportunities to contemplate the challenges of faithful hearing.

%***********
\section{The Challenge of Making Faith Appeal to Hearing}

% Catechism of Trent: accommodating but also training hearing for faith
% + Azevedo
%
% - traditional catholic notions of faith
% - new challenges in era of exploration, (re-)evangelization
% - new pressures in era of anxiety/doubt about subjective experience 
%   and cultural conditioning
%
% Kircher: experiencing spiritual truth through music
% but how to listen rightly?

% danger of subjective experience
% need for cultural conditioning
%  - missionary experiences, e.g., Japanese legates
% need to build society as part of evangelism; gospel = church, musica instrumentalis => musica humana

% obstacles to faith and mistrust of hearing
% Calderon 





