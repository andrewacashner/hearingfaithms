
The scholarly neglect of this huge body of sources stems in part from a
number of still prevalent prejudices and stereotypes about the genre.
My study addresses two of these errors directly:
\begin{enumerate}
    \item Villancicos are not worthy of study because they were trivial,
        occasional, popular, and secular in nature, and because like other
        Spanish music they are not part of any important narrative of Western
        music history.
        Perhaps they preserve traces of folkloric culture, or perhaps they are
        purely European; either way they did not contribute to the
        development of Western music.
    \item Villancicos were an instrument of power used by the Spanish church
        and state to indoctrinate people and oppress them, especially in
        colonial contexts.
        They merely repeat the official dogmas of the Tridentine Church,
        without any creativity; Catholic religious beliefs were always the same
        and their understanding holds little value for an Enlightened person.
\end{enumerate}
Both errors consist of undervaluing these sources as evidence of real religious
beliefs or actual tools of devotion; they share an underlying
dismissal of Catholic Christianity.
They also assume that there is little to learn from the actual music of
villancicos.
To the extent that these views scorn religion and wilfully ignore available
sources, they have no integrity as scholarly positions and should be dismissed
out of hand.
It should require no special pleading to insist that we need to understand this
music within the profoundly theological context in which it was patronized,
created, performed, and heard; or that we need to discuss the actual music.

Nevertheless, in response to the first prejudice, I would emphasize that most
seventeenth-century villancicos are as long as a motet, anthem, or sacred
concerto, and are scored for polychoral ensembles with a complex interplay of
styles and techniques that is of high interest and aesthetic values. 
They are not folkloric music---this is a literate tradition with more debt to
learned music like Palestrina than to popular traditions; but it does involve a
mixture of styles and registers and does seem designed to engage many kinds of
people in their own language.  
Somewhat like mass-mediated popular music today, this music was not typically
created by common people themselves, but it both reflected and shaped popular
tastes and attitudes.

Like other Spanish music and arts they are the product of the most powerful
realm in the world at the time; they are less connected to Italian and northern
European innovations because they saw themselves as the center, not the
periphery.  
Thoroughly integrating music of the Spanish Empire into our history would
radically transform our narratives in ways we will not appreciate until we
begin to do it.

Moreover, villancicos represent a dialogue between elite and common, sacred and
profane---not one side of a dichotomy or the other (and those dichotomies
should not be mapped onto each other simplistically).  
Though they do often feature common-culture elements, they include them in
pieces with quite explicitly religious themes and social functions.
Villancicos were performed in church liturgies by priests and monastics (among
others), with the full patronage and support of the church.

To the second prejudice, I would say that though the church did cultivate
villancicos for its purposes, the religious beliefs reflected in villancicos
are not simply equal to the dogma of the church.
The genre makes surprising and puzzling metaphorical connections between
eternal truths and everyday experience.  
While the results rarely challenge church teachings directly, they do invite
listeners to think about them, compare them with other things, and creatively
connect them in ways that add considerable meaning, or potential for meaning,
to the church's teachings.
Few villancicos actually present religious doctrine in a straightforward
teaching manner.
They were tools of worship and devotion more than teaching.  
Most villancicos depend on knowledge of doctrine more than they teach it.
As Padre Daniel Codina of the Abbey of Montserrat remarked in puzzlement at a
villancico by Montserrat monk Joan Cererols, they seem like \quoted{an
explanation that itself needs to be explained}.%
    \footnote{Personal communication, \XXX[date].}
Rather like contemporary emblem books (see \cref{ch:zaragoza}), which featured
a picture with a Latin motto, a Spanish poem, and a prose explanation, each
element of a villancico increased one's depth of understanding of all the other
parts, with the result that the whole could became a mnemonic device for a
whole complex of ideas.
Of course, the texts were subject to church censorship, and we should not
expect to find anything radically subversive in them, not at least at the
surface level that would catch a censor's eye.
But it was possible to stay close to church teachings on the more theological
side of a poetic conceit while reaching far afield into worldly experience for
the other side, and in fact this is what we find in villancicos.
Moreover, the musical settings, which were not subject to censorship, add
layers of meaning and shape the experience of the performance in ways that
cannot be reduced to a simple idea of teaching doctrine.
Because there are not many written texts from imperial Spain telling us how
people interpreted the music they heard, the body of villancicos about music
provide vital evidence for how and why Spaniards listened to music.
Through words and music, these pieces asked hearers to hold together unexpected
combinations of elements and search out a new meaning to be found by going back
and forth between the two.
The religous purpose of villancicos was not so much \term{doctrine} as
\term{doxology}, as theologians use the term with its Greek meaning of
glorification or worship.%
    \citXXX[theology, lex orandi etc.]
The Catholic liturgical texts and sermons for Christmas do aim to instruct
believers about the theological concept of Christ's Incarnation, but they also
provide a way for them to celebrate the Incarnation, and through the Eucharist,
to actually share in it.


