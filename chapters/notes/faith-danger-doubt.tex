Regarding the problem of individual hearing, Reformation controversies had
pushed Catholics into an increasingly negative and anxious attitude toward
subjective sensation and experience.
Catholic polemicists like Thomas More accused Martin Luther of turning his
followers away from the trustworthy institutional church with its objectively
operating sacraments and leaving them with only a subjective experience as
assurance of salvation.


As Susan Schreiner argues, Catholic theologians rejected this reliance on
individual experience, teaching instead that the certainty of faith and
salvation came through the work of the Holy Spirit in the institutional church.%
    \Autocite[131--208]{Schreiner:Certainty}

Whereas for Martin Luther, the central plank of his platform---perhaps the only
real plank---was the gospel that human beings were saved by God's grace alone,
through Christ alone, by faith alone, as revealed in Scripture alone.
For Luther the Reformation was a controversy about the theology of salvation,
but Catholics saw it as a debate about authority.%
    \Autocite{Schreiner:Certainty}
Luther believed that his doctrine of salvation by faith gave comfort to
believers because it relieved them of the burden of trying to earn salvation by
good works.%
    \citXXX[Luther on Christian freedom?]
Catholic defenders like Thomas More, on the other hand, saw Luther as replacing
the trustworthy institutional church with a foolhardy reliance on individual
subjective experience.%
    \citXXX[More, Responsio ad Lutherum, 1523]
It was the work of the Holy Spirit in the divinely sanctioned hierarchy of the
Roman church that gave its sacraments their objectively operating power and
allowed that faithful to have certainty of faith and salvation---not personal
interpretations of Scripture or some kind of inner conviction.%
    \Autocite[131--208]{Schreiner:Certainty}
From the Catholic perspective, Luther was leading his flock into danger by
asking common people to listen to his voice alone and ignore the chorus of
church fathers who condemned his heresy.
Not every individual perception or experience was valid, but only those that
accorded with the revelation already given through the Church.


