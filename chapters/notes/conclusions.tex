
Villancicos about music challenged composers to use the craft of music to
communicate theological ideas about music itself, and they presented listeners
with the opportunity to listen for higher forms of music through the lower form
of music they could hear---provided listeners were equipped with the requisite
musical, poetic, and religious knowledge and aural training to interpret them.
The practice of imitation was central to this tradition in several senses. 
In the sense of Neoplatonic theology, Spanish musicians were imitating heavenly
harmony in earthly music, though we have seen that their creations manifest a
complex and nuanced reflection on the links between different levels of music
both human, celestial, and divine.

Juan Gutiérrez de Padilla and Joan Cererols both used topical references to
different types of human music to help listeners rethink the relationship
between human and divine, especially in the Incarnation of Jesus celebrated in
the Christmas feast.
Gutiérrez de Padilla sets poetic words about music to actual musical figures at
such a literal level of text-setting, even to the level of puns, that he has
the chorus discussing its own music-making, as it is happening: such as when
one choir admonishes the other to count while that choir is counting, or most
notably when the ensemble sings the phrase \term{a la mi re} on a series of
pitches that put that note and notes with those solmization syllables into the
mouths of the choir even as they are pronouncing them.
This is all the more meaningful given the theological background of that piece,
which celebrates Christ as the \term{Verbum infans}, the unspeaking/infant Word
who embodies God's communication to humanity and whose wordless cries give
utterance not to words but to the divine-and-human nature of the Word himself.
Joan Cererols is focused less on imitating celestial music than on using
worldly music to point beyond itself to that higher music of the \term{Verbum
infans}.
By creating a musical-rhetorical contrast between dissonant, expressive music
and relatively sober music in classical counterpoint, Cererols imitates the
discord between worldly music generally, including the imperfect music of the
spheres, and the true harmony of the divine as heard in the new song of the
Incarnate Christ.
Dissonance serves Cererols as an ironic symbol that points to a truth by
enacting its opposite, but this symbolism is still a form of imitation in the
Neoplatonic sense---in fact it may be heard as a sophisticated discourse
\emph{about} imitation itself.

Both composers also employ imitation in a less overtly theological sense as
well, as they created their villancicos within traditions of metamusical
composition among networks of teachers, colleagues, and rivals. 
The poets and compositors of these families of villancico texts cultivated
conventional ways of speaking theologically about music (while also playing
against these conventions), and the composers developed tropes of metamusical
representation that, aside from the lofty goals just described, also functioned
to establish a pedigree for these musicians.

This chapter concludes the trajectory traced in \cref{part:unhearable-music} by
focusing on the changing nature of imitation within a network of composers in
the province of Zaragoza.
