The first of the two coplas provides a clear example of this technique:
\begin{quotepoem}
    Daba un niño peregrino       & A baby gave a wandering song \\
    tono al hombre y subió tanto & to the Man, and ascended so high \\
    que en sustenidos de llanto  & that in sustained weeping \\
    dió octava arriba en un trino. 
    & he went up the eighth \add{day} into the triune.
\end{quotepoem}
One can read this strophe solely on the one plane referring to Christ's
Incarnation and Passion, however elliptically.
In the first copla, the Christ-child gave a \quoted{wandering song} to the
Man---referring to the first man, Adam, being cast out of Paradise.
Christ \quoted{went up so high} in \quoted{sustained tones of
weeping}---suffering on the cross for human redemption.
The poem says Christ \quoted{arose on the eighth} day, a traditional way of
referring to the the Resurrection on Easter Sunday.
He ascended \quoted{into the triune}, the Godhead of three persons in one
being.%
\begin{Footnote} 
    See the entries for the numbers eight and three in
    \autocite{Bongo:NumerorumMysteria} and
    \autocite{Ricciardo:CommentariaSymbolica}.
\end{Footnote}
Reading this copla according to the other side of the conceit, the strophe
describes a musical performance: the child intoned the \emph{tonus peregrinus}
chant formula, and, as a virtuoso singer, \quoted{he went up so high} that
\quoted{in a cry of sharps}, he \quoted{went up the octave in a trill}.

The poet has selected musical terms with double meanings that allow listeners
with musical knowledge to think about theological concepts in a new way, and
vice versa.
For example, the words \emph{peregrino tono} could have called up for educated
listeners a tradition of using the chant \emph{tonus peregrinus} to symbolize
the expectant wandering of sinful humanity waiting for the coming of Christ, as
well as concepts of the Christian life as a pilgrimage.%
\begin{Footnote}
    This trope was developed through medieval sources like the allegorical
    commentary on the liturgy of Guillelmus Durandus, whose works were available
    in Puebla: \autocite[234]{Wright:Maze}.
\end{Footnote}
The seventeenth-century Biblical interpreter Cornelius à Lapide comments that
Christ was born like a \quoted{pilgrim} \addorig{peregrinus} on a journey in
a borrowed stable, \quoted{in order to teach us to be pilgrims on earth, though
actually citizens of heaven}.
    \Autocites
    [884, on \scripture{Jn}{4:0}:
    \quoted{Hoc est tentorium vel tabernaculum fixit in nobis, id est, inter
    nos, ad modicum tempus, quasi hospes et peregrinus in terra aliena: erat
    enim ipse civis, incola et dominus coeli ac paradisi}.]
    {Lapide:Gospels19C}
    [669, on \scripture{Lk}{2:0}: 
    \quoted{ut doceret nos in terra esse peregrinos, cives vero coeli, ut ab hoc
    exilio magnis virtutum passibus tendamis in coelum, ceu patriam et civitatem
    nostram}.] 
    {Lapide:Gospels19C}
The composer and theorist Andrés Lorente in his 1672 music treatise takes up the
\quoted{pilgrim song} trope as a moral exhortation to aspiring musicians.
The musician of virtue, he says, should match the music of his compositions with
\quoted{the spiritual Music of his person, cleansing his conscience, and
rejoicing his soul with Divine Music, so that he may say with David \add{Vulgate
\scripture{Ps}{118:54}}, \quoted{Your right precepts have served as songs for me
in the place of my wanderings \addorig{in loco peregrinationis meae}}}.%
    \Autocite[609]{Lorente:Porque}
Within this tradition, then, the villancico poem uses the name of the chant tone
to present Christ himself as the song given to sinful Man in his pilgrimage. 
In the musical setting, the composer quotes the chant formula literally, so that
the symbol is present to the ear in both word and tone.
% XXX specific term 'tono peregrino' in Spanish letters (vs. octavo irregular)


