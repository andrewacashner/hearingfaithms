% for preface?
It seems appropriate in a study of historical theology to be frank about my own
personal relationship to the subject matter.
I am a Christian.
This means that I believe that the God who created the universe has redeemed me
from sin and death by taking on human flesh in the historic person of Jesus
Christ, suffering death on a cross and then rising again to new life on the
third day, and that the Spirit of God now lives in me and all other believers
and is working to give us ever-new and eternal life together with God; and that
the purpose of all of for all to share in the love that is God's nature.
I hope that most of the theologically educated people whose writings and
creations I have studied in this book would agree with at least most of the
preceding statement.
I am not a Roman Catholic, though I respect the Roman church and have learned
much from its liturgical and intellectual heritage.
In any case I am certainly not a person of the seventeenth-century Spanish
Empire: I do not believe in the authority of the pope or in purgatory any more
than I believe (as the people I study here did) that the earth is the center of
the universe or that the planets are arranged according to musical ratios and
influence the body through a balance of four humors.
As a Christian I may be inclined to take a more positive view of theology
generally, and perhaps as a non-Catholic I am actually less critical of the
Roman Church than I might be otherwise, as I have no skin in that game.
But for the same reason my own faith actually makes it clearer to me where I
would depart from these historical sources.
I am ashamed of the church's hypocrisy and corruption, especially its history
of justifying the subjugation of indigenous peoples and the enslavement of any
people, two problems which continue today.

% % %
Though I myself am a Christian believer I am not a Roman Catholic (I would say
I am an evangelical Methodist with some Lutheran leanings and a love for both
liturgical and contemporary forms of worship).

% % %
For all its engagement with theology, this book is not a work of constructive
or normative theology. 
Though I myself am a Christian believer I am not a Roman Catholic, and I
certainly do not subscribe to the worldview of seventeenth-century Spanish
subjects.
But this book takes no position on the truth claims of any religious tradition.
Understanding the religious aspect of historical listening practices in a
particular time and place may lead to a more nuanced understanding of
contemporary theology, just as it might hopefully enable more sensitive and
meaningful performances of historic repertoire; but that is up to the reader.
I also would not claim that the theological approach is the only valid way to
study villancicos, and indeed there is much more to the genre than the aspect
that I present here, for which I refer readers to my articles and to a growing
body of studies by scholars with different perspectives and priorities.

Nevertheless, the religious element of early modern Spanish culture is so
overwhelmingly evident to anyone who has visited Mexico or Spain or read any of
its literature, that it can hardly be justified if we overlook it or insist on
interpreting it through an anti-religious or anti-Catholic lens.
Instead, this book is for anyone who has gazed upwards in a Spanish or Mexican
church and wondered why there are so many images of angel musicians with harps,
\term{vihuelas}, \term{bajones}, and organs; or who has read a play by
Calderón, a poem by Sor Juana, or a devotional book by Ignatius of Loyola or
Saint John of the Cross, and has observed how often these writers use musical
metaphors; or who wonders what people thought was happening when they listened
to music in church and how they believed this connected them to God, to each
other, and to the cosmos.

These examples all show that theology was a major intellectual pursuit of the
Spanish and New Spanish elite.

\begin{Footnote}
    Whether in the official Latin translation or the original Greek, the word for
    hearing (Latin \term{auditus}, \Greek \term{akoē}) can mean the faculty of
    hearing, the act of hearing, the hearing organ, or that which is heard.
    The New Revised Standard Version translates this \quoted{So faith comes from
    what is heard, and what is heard comes through the word of Christ}.
    \Autocites{Weber:Vulgate}{Aland:GNT4}[\sv{akoē}]{BDAG}.
\end{Footnote}


% opening example:
% Padilla, solfa villancico from Puebla MS
% possibly also/or "En la gloria de un portalillo" as in diss

\begin{Footnote}
    The major studies of the villancico as a musical and poetic genre are, in
    chronological order,
    \autocites{Rubio:Forma}{Laird:VC}{Torrente:PhD}{Tenorio:SorJuana}
    {CaberoPueyo:PhD}{Illari:Polychoral}{Knighton-Torrente:VCs}
    {Davies:Guadalupe}
    {Cashner:Cards}{Cashner:PhD}
    {LopezLorenzo:VC-Sevillano}{Swadley:VillancicoPhD}{Torrente:Historia17C}
    {ChavezBarcenas:PhD}.
    For musical editions, see \autocite{Cashner:WLSCM32} and the other sources
    cited there.
\end{Footnote}


