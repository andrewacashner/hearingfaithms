% vim: set foldmethod=marker :

% Villancico book chapter on Zaragoza

% 2013-02-08 v0 
% 2013-07-01 v1 
% 2014-12-15 v2 	Caseda section based on revision for ACLS, 2013-09
% 2014-12-29 v2.1 	Real revision
% 2015-03               Dissertation defended
% 2019-04-18            Imported for monograph
% 2019-05-28            Monograph revision begun

\chapter{Offering and Imitation (Zaragoza, 1650--1700)}
\label{ch:zaragoza}

%{{{1 intro
Villancicos about music challenged composers to use the craft of music to
communicate theological ideas about music itself, and they presented listeners
with the opportunity to listen for higher forms of music through the lower form
of music they could hear---provided listeners were equipped with the requisite
musical, poetic, and religious knowledge and aural training to interpret them.
The practice of imitation was central to this tradition in several senses. 
In the sense of Neoplatonic theology, Spanish musicians were imitating heavenly
harmony in earthly music, though we have seen that their creations manifest a
complex and nuanced reflection on the links between different levels of music
both human, celestial, and divine.

Juan Gutiérrez de Padilla and Joan Cererols both used topical references to
different types of human music to help listeners rethink the relationship
between human and divine, especially in the Incarnation of Jesus celebrated in
the Christmas feast.
Gutiérrez de Padilla sets poetic words about music to actual musical figures at
such a literal level of text-setting, even to the level of puns, that he has
the chorus discussing its own music-making, as it is happening: such as when
one choir admonishes the other to count while that choir is counting, or most
notably when the ensemble sings the phrase \term{a la mi re} on a series of
pitches that put that note and notes with those solmization syllables into the
mouths of the choir even as they are pronouncing them.
This is all the more meaningful given the theological background of that piece,
which celebrates Christ as the \term{Verbum infans}, the unspeaking/infant Word
who embodies God's communication to humanity and whose wordless cries give
utterance not to words but to the divine-and-human nature of the Word himself.
Joan Cererols is focused less on imitating celestial music than on using
worldly music to point beyond itself to that higher music of the \term{Verbum
infans}.
By creating a musical-rhetorical contrast between dissonant, expressive music
and relatively sober music in classical counterpoint, Cererols imitates the
discord between worldly music generally, including the imperfect music of the
spheres, and the true harmony of the divine as heard in the new song of the
Incarnate Christ.
Dissonance serves Cererols as an ironic symbol that points to a truth by
enacting its opposite, but this symbolism is still a form of imitation in the
Neoplatonic sense---in fact it may be heard as a sophisticated discourse
\emph{about} imitation itself.

Both composers also employ imitation in a less overtly theological sense as
well, as they created their villancicos within traditions of metamusical
composition among networks of teachers, colleagues, and rivals. 
The poets and compositors of these families of villancico texts cultivated
conventional ways of speaking theologically about music (while also playing
against these conventions), and the composers developed tropes of metamusical
representation that, aside from the lofty goals just described, also functioned
to establish a pedigree for these musicians.

This chapter concludes the trajectory traced in \cref{part:unhearable-music} by
focusing on the changing nature of imitation within a network of composers in
the province of Zaragoza.
We have already seen more than once that Zaragoza was a principal node in the
network of villancico composers and poets.
The first case study in this chapter examines two successive settings of the
same textual family by Pablo Bruna and Miguel Ambiela.
Unlike the cases of Gutiérrez de Padilla and Cererols, where lost
musical settings are unavailable for comparison, here we have a clear case of a
younger composer imitating the work of a respected musician of the preceding
generation---not only using the same words but adapting some of the same
musical ideas as well.
Detailed analysis reveals that Ambiela is both expressing similarity to Bruna
while also articulating differences that only take on their full significance
through comparison.
Just as with Cererols's heavenly dissonance, both the demonstration of
similarity and of difference constitute aspects of imitation.
To play on the saying about imitation and flattery, expressing similarity may
be the the sincerest form of imitation, but signaling difference may constitute
an even higher form.
The contrast between treatments of the same source demonstrates how musical
representations of heavenly music, and the theological worldview that motivated
those representations, were changing.  

The final section discusses the villancico \wtitle{Qué música divina} by José
de Cáseda of Zaragoza, preserved in a convent collection from Puebla, in which
the conventions of the metamusical villancico tradition are pushed to a
particular extreme.
The increasing weight of convention in this type of music required composers to
devise ever-new ways to represent the relationship between heavenly and earthly
music.
Here Christ is compared to a \term{vihuela}, the Spanish plucked-string
instrument related to the guitar.
The theological significance of the metaphor depends on details of the
instrument's construction and tuning, and the musical structure actually works
to express the nature of the instrument through the singers' voices.
The composer appears to be intentionally asking the vocalists to perform
\quoted{false} or \quoted{wrong} music in order to convey the imperfection of
human music as well as to evoke the suffering of Christ.

These acts of imitation are also acts of offering: in an individual aspect,
each composer in a tradition of homage and imitation is offering a new piece as
a tribute to teachers and paragons; in a communal aspect, these pieces served
as offerings to communities of worship, through which members of those
communities offered themselves in devotion. 
Unlike the Christmas pieces in the previous case studies, the pieces in this
chapter are for devotion to the Eucharist (Bruna and Cáseda) and to Mary
(Ambiela).  
In keeping with their ritual functions, their poetic themes---voices rising to
heaven like flames from a soul burning with love like a phoenix
(Bruna/Ambiela), Christ's suffering on the cross as a musical
performance---present music as an affective devotional practice of
self-offering.
%}}}1

%{{{1 
\section{\quoted{Let Voices Ascend to Heaven}: Linked Settings by Pablo Bruna
and Miguel Ambiela}

%{{{2 background
Zaragoza was an important religious center in the crown of Aragon, with its two
principle churches, the cathedral of La Seo and the basilica of El Pilar (today
co-cathedrals).  
A large proportion of the surviving villancico poetry imprints were published
in Zaragoza, many of them commemorating performances of music by Diego de
Cáseda or his son José.
In the greater province of Zaragoza, the village of Daroca was home to the
acclaimed blind organist Pablo Bruna.  
Miguel Ambiela, from the same province, received his early training in Daroca
before graduating to highly prestigious posts in Zaragoza, Madrid, and Toledo.

The first known version of \wtitle{Suban las voces al cielo} is a Eucharistic
villancico composed by Pablo Bruna for four voices (SSAT) with accompaniment,
preserved in the archive of Girona Cathedral.%
    \citXXX[signature]
In 1986 Pedro Calahorra identified Miguel Ambiela's piece of the same incipit
and edited both scores with a comparative analysis.
Calahorra described Ambiela's \wtitle{Suban las voces} as a \quoted{parody
villancico}, calling it \quoted{an homage from a pupil to his master}.%
    \Autocite[9]{Calahorra:Suban}
Calahorra briefly analyzes the pieces' formal structures and situates them
within a history of musical style, with focus on a distinction between
Renaissance and Baroque aesthetics.

The similarities in the poetry and music between the versions by Bruna and
Ambiela do suggest that Ambiela knew Bruna's villancico and composed his
villancico as an intentional response to it.
But there is more to learn here, both on musical and theological levels, not
least because of a newly discovered source of the Bruna villancico, previously
unattributed, in the Catalan state library (\sig{E-BBc}{M/759/44}).
The music and words of the estribillo in this second manuscript are identical
to the Girona version, but the coplas differ.  
Both manuscripts have the same music for the coplas, but distinct poetic texts.
%}}}2

%{{{2 Bruna
\subsection{Bruna: Voices as Flames of Self-Offering}

%{{{3 bruna background, function
\wtitle{Suban las voces} is the only surviving villancico attributed to Pablo
Bruna.
Bruna was born in 1611 in Daroca, a small town about 93 km southwest of
Zaragoza.
He served as organist at the collegiate church of Santa María de los Sagrados
Corporales from 1631 until his death in 1679.%
	\Autocite[104]{Calahorra:Aragon}
His musicianship, which was not hindered by being blind from an early age, was
renowned throughout the region; in 1639 he was offered the organist position at
El Pilar in Zaragoza but declined it.% 
    \Autocite[123--125]{Calahorra:Aragon}
Bruna's organ music was disseminated more widely through its inclusion in the
anthology \wtitle{Huerto ameno de varias flores de música} collected by Fray
Antonio Martín y Coll and published in Madrid in 1709.

The watermark and appearance of the Barcelona manuscript suggests a copying
date in the last third of the seventeenth century, but the style of the music
suggests a much earlier date of composition.%
    \citXXX[watermark sources or methodology]
Bruna's madrigalesque counterpoint for four voices is similar to other such
pieces from before 1660, in contrast to the solo and duo continuo-songs that
predominate later.%
\begin{Footnote}
    The Girona manuscript does include an accompaniment part labeled
    \foreign{entablatura}, but this is primarily a \term{basso seguente} rather
    than the independent continuo part found in Irízar and Carrión.
\end{Footnote}
The musical style is actually rather similar to that of Gutiérrez de Padilla. 
Though Bruna was a generation younger, most of Padilla's villancicos are from
the last decades of his life, so the two were composing at the same time.

Bruna's villancico would function well for Eucharistic devotion, in both its
poetic content and musical style.  
The piece fits a subtype of chamber villancico dedicated \foreign{al Santísimo
Sacramento} (to the Blessed Sacrament) and intended not for the extravert
public festivities of Corpus Christi, but for more reflective and intimate
occasions of Eucharistic adoration.
This more intimate type of Eucharistic villancico frequently features
mystically-infused texts with an emphasis on personal affective devotion to
Christ in the sacrament.  
%}}}3

%{{{3 bruna analysis
\subsubsection{Hearing the Burning Heart}

The central image in Bruna's villancico is the soul as a burning phoenix,
consumed by the love of God.
This use of fire symbolism may be found in other Eucharistic villancicos, such
as a 1643 piece by Jaume Pexa from Lleida (Lérida)
(\cref{poem:Que_me_quemo-Pexa}), with estribillo similar to Bruna's.%
\begin{Footnote}
    \sig{E-Bbc}{M/765/15}; see \shortcite[\sv{Pexa, Jaume}]{DMEH}.
    The manuscript is in the same box of villancicos as the Barcelona version
    of Cererols's \wtitle{Suspended, cielos}, as well as a piece by Miguel
    Ambiela and another by one of the Cásedas (probably Diego, the elder).  
    This suggests that music of all these composers was valued by the same
    collector, whether individual or institutional.
\end{Footnote}
The partbooks of Pexa's eight-voice piece bear the heading \foreign{De amores
del exposo sancto} (About love for the exposed Sacrament)---making clear that
the piece was to be performed for Eucharistic adoration.  

%{{{4 poem pexa
\insertPoem{Que_me_quemo-Pexa}
{\wtitle{Que me quemo}, villancico for Eucharistic adoration by Jaume Pexa
(Lleida, 1643), estribillo}
%}}}4


The anonymous poet of Bruna's villancico combines the conceit of the soul as a
burning phoenix with a musical conceit announced in the first line: \quoted{Let
voices to ascend to heaven} (\cref{poem:Suban_las_voces-Bruna-estribillo,
poem:Suban_las_voces-Bruna-coplas}).
As a poem, the estribillo of \wtitle{Suban las voces} may be divided into three
sections. 
First, the opening quatrain (\poemlines{1--4}) establishes the primary conceit;
second, \poemlines{5--11} present a scene of music-making using technical terms
from music; and third, the estribillo closes with a couplet
(\poemlines{12--13}) that epitomizes the conceit.
These sections are articulated through the poetic-metrical structure, with the
first section in octosyllables (suggesting \term{romance}), the second in
seven-syllable \term{romancillo}, and an irregular closing couplet.
Lines 4 and 13, both truncated to five syllables, mark ending points in the
text; and assonance throughout unifies the poem.

This pattern is generally similar to the form of \wtitle{Voces, las de la
capilla}, though shorter and without a \term{respuesta} section.  
The second section of \wtitle{Suban}, \quoted{Y mudando el aire en veloces
corcheas}, is quite similar to the estribillo of \wtitle{Voces}, which begins
\quoted{Y a trechos las distancias en uno y otro coro} (and in the musical
setting, features \term{corcheas} for the first time).
Further, the concluding summary couplet is similar in function to the one at
the end of the \wtitle{Voces} estribillo, \quoted{Todo en el hombre es subir}.
These connections situate Bruna's text as part of an older poetic tradition,
though not necessarily linking it directly with \wtitle{Voces}.

%{{{4 poem bruna
\insertPoem{Suban_las_voces-Bruna-estribillo}
{\wtitle{Suban las voces al cielo}, setting I, poetic text set by Bruna,
estribillo}

\insertPoem{Suban_las_voces-Bruna-coplas}
{\wtitle{Suban las voces}, setting I, poetic text set by Bruna, coplas}
%}}}4

As in the other villancico families in \cref{part:unhearable-music}, the poem
exists in variant forms; in this case the coplas differ between the Girona and
Barcelona manuscripts.
The first copla is identical in both sources, but the Girona version follows it
with two stanzas, while the Barcelona version follows it with five unique
stanzas.
The Barcelona coplas all follow the same metrical scheme as copla 1: three
octosyllablic lines and the refrain \foreign{arde}, with \emph{a/e} assonance
the even lines.
The Girona coplas, by contrast, change the metrical pattern after copla 1: they
add an additional line and maintain assonance only in the last syllables; the
\foreign{arde} refrain no longer forms part of the metrical structure.
These differences suggest that the Barcelona coplas are closer to the original
source of this textual family, while the different Girona coplas were likely
penned by a later poet, possibly to replace verses that had been lost or
forgotten.
The theology and tone of the texts also suggests that the Barcelona source is
older, as the Girona coplas are slightly more didactic.
As was the case with the Jalón's \wtitle{Cantores de la capilla}
(\cref{ch:padilla-voces}) and with the revised textual family B of
\wtitle{Suspended, cielos} (\cref{ch:cererols-suspended}), the more simplified,
explanatory texts in a villancico tradition are often the later ones,
suggesting the accommodation of an earlier text to suit later tastes.

Similar to Juan Gutiérrez de Padilla, Bruna sets each phrase of poetic text to
a distinct phrase of music, with its own rhythmic and harmonic profile,
following closely the prosody of the words.  
The three poetic sections are also clearly articulated in the musical form
through shifts of meter and character.
Bruna has his musicians enact the musical concepts in the poem in a
madrigalistic manner.  
The singers perform \quoted{let voices ascend} by leaping upward on
\quoted{voces} (\measures{1--4}) and then repeating the whole phrase again a
third higher---a musical-rhetorical \term{anaphora}, since the term means both
repeating a phrase and \quoted{lifting up}.
Bruna follows the first section with a change of meter (to \meterC), and his
setting of \quoted{Y mudando el aire en veloces corcheas} uses appropriately
flying \foreign{corcheas} (eighth notes), ascending and descending in pairs
(\foreign{juntas}) (\cref{mux:Bruna-Suban_las_voces-estribillo}).
Bruna sets \quoted{en síncopas que elevan} exactly as would be expected, with
syncopated phrases that leap upwards like flames.
Bruna uses chromatic alterations for the phrase \quoted{bemoles blandos},
writing E flats in the outer voices (\measures{21}) and then setting up a
striking Phrygian cadence in \measures{21--22} in which the Alto leaps upwards
into an unprepared seventh.  
Bruna writes dotted figures on \quoted{trinados que suspendan} that would
invite any singer to add a trill, and in \measures{26--28}, these trills flow
into textbook suspensions (as in \measures{27}, Alto).
For \foreign{digan en paso} (let them say in turn) Bruna has the voices follow
each other in fugal imitation, in an evenly paced rhythmic pattern.

% XXX check or remove measure nos

%{{{4 music Bruna estribillo
\insertMusic{Bruna-Suban_las_voces-estribillo}
{Bruna, \wtitle{Suban las voces}, estribillo: Madrigalistic text setting
(accompaniment omitted)}
%}}}4

In the coplas Bruna breaks up the declamation-focused setting with a syncopated
rhythm on \foreign{si en Dios hallas nueva vida} (if you find new life in God):
by dividing three measures (\term{compases}) of duple time into what sounds
like four measures of triple, he animates this phrase with a new kind of
rhythmic life (\cref{mux:Bruna-Suban_las_voces-coplas}).
Similarly, in the ending refrain line on \foreign{arde}, Bruna syncopates the
rhythm by playing off the voices in pairs, where in each measure one group has
downbeat accents and the other has a minim rest followed by an offbeat accent.
Cerone calls rests \foreign{sospiros}, literally \quoted{sighs}, and their use
here creates a breathless intensity that fittingly portrays the soul in ardor.%
    \citXXX[Cerone]

This passage points to a more dramatic quality of Bruna's setting that goes
beyond literal representations of the musical figures in the poetry to create a
mystical affect, partly through the use of chromaticism.
For example, in \measure{7}, the voice and accompaniment undulate between a
G-over-E\fl{} harmony and F\sh-over-D, on the words \quoted{a phoenix is
burning, a soul}.
These minor-second gestures and chromatic alterations may be
ways of characterizing spiritual passion.
Altered notes could symbolize the changes wrought by fire, which in turn is a
metaphor for the soul's conversion to loving God.  
Beyond the symbolic level, such gestures seem to form part of a stylistic topic
connected with a certain kind of worship experience.  
The piece both depicts affective devotion and actually serves as a form of
affective devotion for its historic performers and their worshipping community.

% TODO add music example of copla rhythm and/or affective topic?
%}}}3
%}}}2

%{{{2 Ambiela
\subsection{Ambiela: Voices Rising in Intercession}

%{{{3 background, sources
The later setting in this villancico was composed by Miguel Ambiela
(1666--1733) to a text closely based on that set by Bruna.
Where Bruna's piece concentrated on the symbolism of fire and the phoenix,
Ambiela's is dedicated to the Assumption of the Virgin Mary.
The compositor of the text---quite possibly Ambiela himself---has excerpted
from the earlier poem all the lines with musical vocabulary and built a new
poem around them focused on raising voices to celebrate Mary as heavenly
intercessor (\cref{poem:Suban-2-estribillo, tab:Bruna-Ambiela-cf}).

%{{{4 poem Suban Ambiela estribillo
\insertPoem{Suban-2-estribillo}
{\wtitle{Suban las voces al cielo}, setting II by Miguel Ambiela, estribillo}
%}}}4

%{{{4 table Bruna Ambiela cf
\insertTable{Bruna-Ambiela-cf}
{\wtitle{Suban las voces}, Comparison of estribillos set by Bruna and Ambiela}
%}}}4

Like his contemporary Jerónimo de Carrión in Segovia, Miguel Ambiela lived
through the end of the Habsburg dynasty in Spain, the ensuing War of the
Spanish Succession, and the installment of the Bourbons.
His highly successful career took him from the town of La Puebla de Albortón in
Zaragoza province to the most prestigious positions in Spain, as chapelmaster
of El Pilar in Zaragoza (1700--1707), Las Descalzas Reales in Madrid
(1707--1710), and finally Toledo Cathedral (1710--1733).%
    \Autocites
    [1]{Calahorra:Suban}
    [\sv{Ambiela}]{Grove}
    {Alvarez:Ambiela}
His training began at age 15, when in 1681 his parents sent him to study music
in Daroca.
Within four years Ambiela had been appointed chapelmaster at the collegiate
church there.
This was the same church at which Pablo Bruna had served as the renowned
organist.

Ambiela's reworking of \wtitle{Suban las voces} further confirms the pattern we
have demonstrated throughout the previous chapters in which Spanish church
musicians used metamusical villancicos as ways of paying homage to paragons and
establishing a pedigree, particularly in chains of succession in chapelmaster
positions.
Since Bruna died three years before Ambiela came to Daroca it is rather
unlikely that the boy musician met him personally, much less studied with him
(as Calahorra speculated).%
    \Autocite{Calahorra:Suban}
Nevertheless, it is probable that Bruna's music was still being performed
during Ambiela's apprenticeship, and that Ambiela himself participated in the
performances.  
During the single year Ambiela served as chapelmaster in Daroca before moving
on to a position in Lleida (1685--1686), Ambiela would have made use of the
church's archive, where he could have encountered Bruna's villancico if he had
not already heard or performed it.
Ambiela's version of \wtitle{Suban las voces} could have signalled to listeners
in the parish that their new chief musician was at once the heir to the
esteemed legacy of Bruna and a creative new voice of his own.

Ambiela's ambitious setting for six voices (SST, SAT, continuo) is preserved in
manuscript performing parts in Barcelona, copied sometime before 1689.%
\begin{Footnote}
    \sig{E-Bbc}{M/733/1}.
    The edition in Calahorra's article misplaces one of the fugal entries by
    two measures. 
    %XXX details, or save it for WLSCM
\end{Footnote}
The front-facing leaf used as a title page bears a dedication to the Assumption
of Mary in one hand, and a series of doodles on the name Torrente in a second,
sloppier hand---the work, we can imagine, of an idle-handed choirboy by that
name.
The same writer also copied his own \term{bajón} part and, at the top of the
accompaniment part, dates and ascribes the piece quite specifically:
\quoted{Acompañamiento Continuo a 6 Vozes del Maestro Miguel Ambiela año 1689 a
24 octobre}.
The 24th of October was the feast of St. Raphael, Archangel, but since the
villancico was clearly intended for the Assumption of the Virgin on August 15,
this date was either a later performance or even a copying date.

On the title leaf, the same hand (Torrente) ascribes the piece to
\quoted{Master Miguel Ambiela, who was from Lérida and before that from Daroca,
where the wall is big and the city is small}.%
\begin{Footnote}
    \foreign{Del Maestro Miguel Ambiela/ que fue de Lerida y despues de daroca
    en donde el muro/ es Grande y la Ciutat es Poca}.
\end{Footnote}
The copyist, writing in Castilian, spells \foreign{ciudad} as \foreign{ciutat},
manifesting the final obstruent devoicing characteristic of Catalan, just as
observed in the previous chapter where \foreign{suspended} was copied
\foreign{suspendet}.  
In 1689 the 23-year-old Ambiela was chapelmaster in Lleida (Lérida in
Castilian), halfway between Zaragoza and Barcelona.
The trace of a Catalan accent, along with the joke disparaging Daroca---a small
town surrounded by huge medieval walls---both suggest that the writer was a
member of Ambiela's chapel in Lleida, a larger and more prestigious city on the
other side of the Catalonian regional divide.
That young master Torrente highlights Ambiela's previous position in Daroca
suggests that, regional rivalry aside, there was some local significance in
Ambiela's connection to that city, likely because of the piece's connection to
Pablo Bruna.

Even aside from the direct connection to Daroca, the two composers appear to
have been part of the same network of composers we have been tracing, and their
works were collected by those who valued composers within that network.  
Copies of both Bruna and Ambiela villancicos are present in the Biblioteca de
Catalunya's collection.
Though the two settings of \wtitle{Suban las voces} are not now in the same
archival signature (that is, they are in different boxes of villancicos), each
piece is gathered with pieces by some of the same composers.  
These composers include most notably Joan Cererols and one of the Cásedas
(probably the elder, Diego), including a setting by Cáseda of a text also set
by Cererols, \wtitle{Pues que para la sepultura}. % TODO signatures
It seems possible that Ambiela was acquainted with Cererols personally, if not
only by reputation; and it is almost certain that he knew one or both Cásedas
in Zaragoza, since he succeeded their successive tenures as chapelmaster of El
Pilar.
Ambiela was also connected to Madrid composers from at least the time of
his position at Las Descalzas; indeed immediately adjacent to Ambiela's
\wtitle{Suban las voces} in \sig{E-Bbc}{M/733} is a villancico by the
influential Madrid composer Sebastián Durón along with works by Carlos Patiño.

Supporting evidence that young Iberian composers modeled new music on older
works encountered in local archives comes from the surviving manuscript lesson
books from apprentice musicians.
One such notebook, written in Catalan, concludes with a section on \quoted{Some
Rules about Counterpoint observed from the Method of Some Masters in Girona}.%
\begin{Footnote}
    \sig{E-Bbc}{M/732/15}: \wtitle{Algunas Reglas sobre els Contrapunts
observadas del metodo de alguns mestres en Gerona}.
\end{Footnote}
% TODO convert to bib entry
The notes appear to represent lessons learned not from a course of theory
training, but from studying manuscripts in the archive---the same one,
incidentally, which holds a copy of Bruna's \wtitle{Suban las voces}.
%}}}3

%{{{3 outdoing
\subsubsection{Outdoing and Overdoing}

While Ambiela's poetic text copies the core of Bruna's text verbatim, Ambiela's
musical setting is an homage rather than a parody---that is, a creative
response to a model rather than a direct reworking.
Ambiela does copy one motive verbatim from Bruna, and uses it at the same point
in the text and with the same texture, the highest voice with continuo
(\cref{mux:Ambiela-Suban_las_voces-mudando}): compare Bruna \measure{6}
(B\fl--A--G--F\sh--G--F\sh) with Ambiela \measure{10} (F--E--D--C\sh--D--C\sh).
%XXX fix bar nos.
Aside from this one direct quotation, similarities are found more on the level
of procedure than style: Ambiela follows Bruna's basic formula for putting the
piece together though the result sounds quite different.
Both pieces begin with four voices in an upward-leaping gesture high in their
tessituras.  
Both composers set the words \foreign{Suban las voces al cielo} as a musical
phrase three measures (nine minims) long.
Both amplify the opening phrase through the rhetorical technique of
\term{anaphora} by repeating this first phrase transposed up a third. 
%XXX check interval for Ambiela
In both pieces this phrase is followed by a reduced texture (solo for Bruna,
duo for Ambiela) and then a fugato passage for the full ensemble.

Ambiela follows Bruna in embodying the musical terms in the text in several
passages: both composers switch to \meterC{} for the \quoted{change} in the
passage \foreign{mudando el aire}; both use an imitative texture in
\quoted{flying corcheas} with pairs of voices moving \quoted{together}.
Where the text speaks of syncopated notes, both composers write just that.
Like Bruna, Ambiela illustrates \foreign{pasos} with imitative counterpoint.

%{{{4 Ambiela Suban mudando
\insertMusic{Ambiela-Suban_las_voces-mudando}
{Ambiela, \wtitle{Suban las voces}, estribillo: Compare Bruna,
\cref{mux:Bruna-Suban_las_voces-mudando}}
% XXX bar numbers (and in example)
%}}}4

On the other hand, aside from the opening and the motivic similarity, many of
the similarities between pieces could be explained simply as two composers
setting the same words, with highly specific references to musical practice,
according to similar musical-rhetorical conventions.
The two pieces actually differ strongly in character.
In contrast to Bruna's intimate, chamber-style piece in mystically tinged
\term{cantus mollis}, Ambiela's villancico is a large-scale polychoral piece in
a public, celebratory manner, set in mode 9 (the authentic mode with an A
final).
Some of this difference stems from the pieces' differing liturgical functions. 
One is a chamber piece for Eucharistic devotion while the other is a
pull-out-all-the-stops celebration for one of the highest feasts of the Spanish
church year.
% XXX are there examples of parodies where the mode is changed?

Paradoxically, though, it is actually in the ostensible differences between the
settings where it becomes clearest that Ambiela's piece is an homage.
Ambiela takes Bruna's model and increases its complexity in every way he can.
Where Bruna begins with \term{anaphora} in one choir, Ambiela, with two choirs
at his disposal, does his paragon one better by giving the transposed repeat to
the second chorus and bringing them in even before the first chorus has
finished their phrase.
In the same place that Bruna has a solo passage followed by full chorus,
Ambiela both imitates and expands on Bruna's approach: he begins his phrase not
with a single voice but with two, and then writes a fugato for the full chorus.
In the passage about \foreign{corcheas} and \foreign{síncopas}, where Bruna set
each phrase of text in sequence, Ambiela uses his double-chorus texture to
overlap the phrases, creating a rich texture of flying eighth-notes and
syncopated figures in tension with each other.
For the end of the estribillo, Ambiela extends Bruna's brief imitative passage
into a double fugato.

Ambiela treats the repeat of the estribillo after the coplas differently than
Bruna as well.
Each copla is sung by a soloist and then leads into a repeat, not of the whole
estribillo, but only of a portion.
% XXX does Calahorra identify this?
The first copla leads into a repeat of the
opening section (\measures{1--9}); the second copla, into the middle section
(\measures{10--25}), and the last copla, into the concluding section
(\measures{25--67}).
Though much more steady of musical sources is needed, there seems to have been
a trend later in the seventeenth century away from full repetitions of the
estribillo, as composers were writing longer settings (eventually developing
multisectional \term{cantadas} in the eighteenth century).
Álvaro Torrente argues that in some cases the estribillo may not have been
repeated at all.% etc
    \citXXX[Torrente:estribillo new]
Ambiela here provides a novel and possibly unique solution to the problem, as
no other villancicos with this technique have yet been found.


In a few passages, rather than expanding on Bruna's ideas, Ambiela contradicts
them.
These passages, though, actually furnish the strongest evidence that Ambiela
knew the earlier piece and composed his in direct response.
For the text \foreign{bemoles blandos} (mild flats), Bruna does as any
seventeenth-century Spanish composer would have done, and adds flats.
Ambiela, by contrast, does not write a single flat for this phrase of text; in
fact, he writes an extended passage loaded with sharps
(\cref{mux:Ambiela-Suban_las_voces-bemoles}).
For \quoted{trinados que suspendan}, Ambiela does not write the classical
suspensions that Bruna does (which by rule always resolved downward by step).
Instead he writes a chromatic line that does nothing but ascend
(\measures{27--29}, Ti. I-2). 
Certainly there are contemporary examples (like Cererols's dissonances on
\foreign{la más nueva consonancia}) of representing something by embodying its
opposite, but there are many more examples in metamusical villancicos of
literal, even punning, matchups of poetic and musical devices. 
Indeed, the rest of Ambiela's setting abounds in such direct word--music
relationships.
Ambiela's choice to go against the text in these two passages, then, is
probably a response to Bruna's model.

%{{{4 Ambiela Suban bemoles
\insertMusic{Ambiela-Suban_las_voces-bemoles}
{Ambiela, \wtitle{Suban las voces}, poetry about \quoted{flats} set in sharps}
%}}}4

Here we see a central theme of the case studies in
\cref{part:unhearable-music}: the tradition of homage and competition among
composers in metamusical composition develops in parallel with changing notions
of music's place in the cosmos, and its effect on people.
As Calahorra observes, Ambiela's free-wheeling or even reckless approach to
counterpoint is worlds apart from the more traditional style of Bruna.
Bruna's work is relatively intimate, subtle, contemplative; Ambiela's is
extroverted, exuberant, even ostentatious.
For example, in Ambiela's first section in imitative texture, \measures{3--6},
the voices fly past each other almost as though they are not all singing the
same piece, creating frequent F\na{}/C\sh{} sonorities, and for one brief
moment (\measure{4}, third minim) only the pitches A, G, and D are sounding.
Calahorra provides an analysis of Ambiela's abruptly shifting cadences and
pervasive use of seventh chords on the first beat of the compás.

It could be that these are the marks of an inexperienced, twenty-three-year-old
composer, especially one trying to do something impressive without having yet
achieved the technical means to do it well.
But it is also possible that the effect of wild, heedless rejoicing, or of
mysterious music that defies earthly rules, is exactly what Ambiela wanted, in
line with the Neoplatonic theology seen developing in the other examples of
\cref{part:unhearable-music}, but now part of an emerging high-Baroque
aesthetic.  
The difference may be compared to the distance between \wtitle{Las meninas} of
Velázquez (1656), in which the painter paints himself painting; and an
extravagant \term{trompe-l'oeil} like the \term{transparente} of Toledo
Cathedral---which was completed by Narciso Tomé and his sons only a year before
Ambiela's death in 1733.
A decreasing confidence in music's power of representation, of music's
faithfulness to the real world, results in a continuously escalating demand for
self-conscious artifice.  
The pressure to imitate both the text and the earlier musical setting appears
to have pushed Ambiela to a more extreme type of musical representation.
Bruna was content with the mirror-in-a-mirror effect of singing flats while
singing \emph{about} flats; for Ambiela, by contrast, one must sing sharps.
The emphasis shifts from imitating heavenly music to representing earthly
music, and the functional goal of the music, theologically speaking, shifts
from a primarily contemplative one toward a more affective, expressive one.
%}}}3
%}}}2

%{{{2 theology
\subsection{Offering Hearts and Voices through Music}

%{{{3 intro
The function of Ambiela's homage to Bruna seems clear enough on the human level
of a young composer building a reputation: the metamusical villancico served as
a Spanish chapelmaster's proof piece.
It fulfilled functions of offering on this level by giving the parish a chance
to hear their choir perform a virtuoso representation of music through music,
invoking awe and wonder, while allowing the composer to offer a tribute to his
predecessor.
In contrast to the tight \term{conceptismo} of the pieces studied in
\cref{ch:padilla-voices, ch:cererols-suspended}, some of the musical terms in
these poems (\foreign{corcheas}, \foreign{síncopas}, \foreign{bemoles}) have no
obvious theological meaning, and may seem like superfluous excuses for
compositional showmanship.
But these pieces are not just about a composer's ability to move notes around.
The texts refer directly to actual music-making because the pieces are about
music as a form of devotion.
These pieces are examples of music as a form of self-offering, and their
symbolic conceits---the heart burning like a phoenix, voices rising like flames
to heaven---reveal much about how early modern Catholic understandings of music
in worship.
%}}}3

%{{{3 flame phoenix
\subsubsection{Flame and Phoenix}

In both texts, music is represented as voices rising from heaven from souls
afire with the love of God.
In the Bruna version, the ardent soul is compared to a burning phoenix, and in
the Ambiela, voices ascend with the Virgin to the heavenly realm.
The villancico family builds on the link between fire and music in early modern
physics: both \quoted{transform the air}.
In the physiology of the time, there was also a connection between the physical
mechanism of the voice and the element of fire.

The phoenix functions in the villancico as an emblem of self-offering, an icon
both of Christ's sacrifice and of the Christian's devotion.
Worshippers in Daroca, singing or hearing Bruna's piece during rituals of
Eucharistic adoration, would have been familiar with the phoenix as Covarrubias
defines it: the phoenix is \quoted{said to be a singular bird who is born in
the Orient, celebrated through all the world, raised in happy Arabia, \Dots{}
who lives six hundred and sixty years}.% 
    \Autocite[400, \sv{fenix}]{Covarrubias:Tesoro}
Covarrubias cites Pliny along with Tacitus and other Classical authorities for
the famous legend of this noble bird that builds its own funeral pyre and is
then reborn from the ashes---just as the villancico describes it (coplas B1/G1,
B5--B6).
The phoenix's self-immolation was widely interpreted as a symbol of Christ's
crucifixion and resurrection; Covarrubias recounts that some even claimed the
phoenix was reborn in a regular cycle, and that one of its rebirths coincided
with the year of Christ's death, \quoted{of which it seemed prophetic}.
Be these tales true or false, Covarrubias writes, \quoted{the sentiment is
pious, and many have formed hieroglyphics of the phoenix, applying them to the
resurrection of our Lord, and more have done so than could be counted, and
likewise many emblems and imprints that are moral or deal with amorous
subjects. \Dots{} The alchemists have their particular symbols under the name
of the phoenix bird}.%
    \Autocite[400, \sv{fenix}]{Covarrubias:Tesoro}

Covarrubias himself had joined these Christological and amorous facets of the
phoenix myth in exactly this kind of moral emblem one year previous in his
\wtitle{Emblemas morales}.
His emblem 90 combines the heraldic arms of the royal monastery of San Lorenzo
de El Escorial, St. Lawrence's grill, with a burning phoenix underneath a sun
with rays (\cref{fig:Covarrubias-phoenix}), and the motto \foreign{FOELICITER
ARDET}.%
    \Autocite[\range{f}{290, r--v}]{Covarrubias:Emblemas}
Each of Covarrubias's emblems presents a single moral concept through multiple
media---on the front of each, an engraved image including a brief motto,
usually in Latin and a Castilian poem; on the back, a prose explanation.
% XXX emblem 2ry literature
The image and texts interpret the other elements and contribute to the function
of the whole as a mnemonic device. 
A motto which at first may have been cryptic, after multiple explanations of
different kinds, ultimately encapsulates the message of the whole.
Here the motto is taken from Ovid: \quoted{If someone loves something that
gives joy when loved, he burns happily, rejoices, and as by wind sails directly
to the beloved}.% 
\begin{Footnote}
    Ovid, \wtitle{De remedio amoris}, quoted in
    \autocite[290v]{Covarrubias:Emblemas}: \quoted{Si quis amat, quod amare
    iubat; foeliciter ardet, Gaudeat, \& vento nauiget ille suo}.
\end{Footnote}
The explanatory poem applies the phoenix's burning to the soul transformed by
mystical love:
\begin{quoting}
\begin{verse}
Always in the mortal breast there burns\\
the celestial, divine, and holy fire;\\ 
it causes no conflict with the elemental body\\
For it causes neither fear nor alarm:\\
as a new Phoenix, in love it burns\\
and though it is consumed in its old mantle\\
it changes it for another, more precious---\\
of royal purple, incorruptible, and glorious.
\end{verse}
\end{quoting}
The image and its explanation correspond closely with the Bruna villancico's
conceit of the soul burning with holy fire, offering itself to Christ.
The villancico and others like it could even be thought to function as an
auditory emblem, which first presents a striking conceit (like the image and
motto), then expands on it in the rest of the estribillo (like the verse
explanation), and then explains in more detail in the coplas (like the prose on
the reverse).
The repeat of the estribillo after the coplas could even be compared to the
typesetting arrangement of the image and poem on the \term{recto} and the prose
on the \term{verso}, after which inevitably the reader will turn back and look
at the emblem again with new eyes.
The difference is that what the emblem describes and prescribes, the villancico
embodies as a ritual experience.

%{{{4 fig phoenix
\insertFigure{Covarrubias-phoenix}
{\quoted{IT BURNS HAPPILY}: The phoenix emblem from Covarrubias,
\wtitle{Emblemas morales} (1610), \term{centuria} III, \range{no}{90}}
%}}}4 

The connection between the phoenix and the soul depended on specific
understandings of fire and of worship.
For deeper understanding Covarrubias himself refers readers to the
\wtitle{Commentaria symbolica} of Antonio Ricciardo, as this exhaustive
reference lists no less than sixteen distinct emblematic uses of the phoenix.
Three of these are pertinent to Bruna's villancico:
\begin{quoting}
    \begin{enumerate}
        \item[3.] The phoenix signifies our souls in this bodily pilgrimage.  
            Indeed, while we are living here we are far removed from our
            homeland.
        \item[13.] A phoenix in the midst of burning flames, with the words,
            \quoted{Ne pereat} (Let him not perish), signifies a man who in the
            present life gives himself to be burned up through \add{bodily}
            mortification, lest he perish eternally.
        \item[15.] A phoenix over flames, expanding its wings to the rays of
            the sun, with these three letters, V. E. V., and with the words,
            \quoted{Ut uiuat} (Let him live) \Dots{} signifies a man, who puts
            all his hope in Christ the Lord, the sun of justice, from whom he
            hopes for renewal of life.  
            The fire, then, signifies the Holy Spirit, who should be embraced,
            who chooses everything that is best \add{for the man}, in order
            that he might be taken up from earthly heaviness, and live
            eternally.%
                \Autocite[\sv{Phoenix}, 132--133]
                {Ricciardo:CommentariaSymbolica}
    \end{enumerate}
\end{quoting}
In \wtitle{Suban las voces}, the phoenix is the soul (\range{no}{3}), which is
on pilgrimage (\quoted{Soul, you are on the road}, \poemline{3}, copla G3).
The motto \quoted{Ne pereat} described by Ricciardo (\range{no}{13}) recalls
the phrase \quoted{no peligras} in copla B3, which concentrates on
mortification to avoid the \quoted{flatteries of the wicked}.
In definition 15 Ricciardo describes an image very similar to Covarrubias's
phoenix emblem, and explains even more clearly that the soul's fire is the
result of the purifying work of the Holy Spirit, causing those who surrender
their own being (copla B2) to find new life in Christ, who as the sun is the
source of all fire (copla G3).
Just as Covarrubias used the imagery of flying on the winds, so Ricciardo also
connects fire to the idea of ascending from earth to heaven.

This last concept is central to the concept of music in \wtitle{Suban las voces
al cielo} from the opening line, and it depends on the way early
modern Europeans understood the physics of fire.
% XXX 2ry lit
First, fire and love were linked because it was love that kept the four
elements in harmony despite their perpetual war against each other, as
Covarrubias summarizes in another emblem:
\begin{quoting}
    \begin{verse}
        Heaven, fire, air, water, and earth\\
        and all this, as much as has been created,\\
        Love rules it, Love opens and closes it,\\
        in a sweet chain, linked together,\\
        and when one or the other wages war,\\
        the conquered is always left bettered;\\
        for the one is converted into the other,\\
        taking life even from death.%
            \Autocite[\term{centuria} I, \range{no}{45}]
            {Covarrubias:Emblemas}
    \end{verse}
\end{quoting}
This love, Covarrubias clarifies, is \quoted{God himself}.
Divine love is the energy that allows the elements to transform themselves into
each other for the mutual benefit of all.

Similarly, Fray Luis de Granda wrote (drawing on musical language) that the
Creator built the terrestrial world with its four elements \quoted{by such
order and measure \addorig{compás} that, though they are opposed to one
another, they have peace and harmony \addorig{concordia}, and not only do they
not disturb the world, but in fact they preserve and sustain it}.%
    \Autocite[204]{LuisdeGranada:Simbolo}
Much like Covarrubias's concept of a chain, the four elements are linked with
their neighbors in a \quoted{lineage of affinity and genealogy}: moving from
earth as the lowest of the elements, up through air, water, and the highest,
fire, the elements \quoted{do something like a saber dance \addorig{danza de
espadas}, each one continuing on amicably to the others in this way}.
    \Autocite[204]{LuisdeGranada:Simbolo}
Like the participants of a mock-war dance with swords (for which instrumental
survives from across the Spanish Empire), the elements appear to be at war but
are actually moving together in a well-ordered round dance.%
\begin{Footnote}
    See the \term{danzas de espadas} in, for example, the Peruvian Codex
    Martínez Compañon and \autocite{MartinyColl:HuertoAmeno}.
    %XXX others, 2ry, recordings
\end{Footnote}

Fire ascended skywards because in this cosmology it was the highest and
lightest of the elements, the closest to the world beyond the terrestrial.%
\begin{Footnote}
    Early modern thinkers drew most heavily on the treatment of this subject in
    Aristotle's \wtitle{Physics}; see \autocite{Lang:AristotleMedieval}.
\end{Footnote}
Just below fire in the chain of being was air, and Fray Luis describes the
atmosphere as embodying this relationship: the air above earth was divided into
three regions, the highest of which was \quoted{adjacent to the element of
fire, and is therefore extremely hot}.%
    \Autocite[207]{LuisdeGranada:Simbolo}
Highest of all, the sun and stars were giant balls of fire.%
    \Autocite[\XXX]{LuisdeGranada:Simbolo}
The process of burning, then, changed elemental air into fire through the
medium of flame: \quoted{we see the air become inflamed with fire, which is
adjacent, and be converted into fire}.%
    \Autocite[205]{LuisdeGranada:Simbolo}
As the element at the farthest extreme of the terrestrial realm, and as an
agent of transformation in a world governed by love, fire was an apt symbol for
love's ability to transform the soul and transport it into the heavenly realm.%
\begin{Footnote}
    The transformative power of fire explains the use of the phoenix by the
    alchemists, as noted by both Covarrubias and Ricciardo.  Covarrubias
    emphasizes transformation in one of several emblems focused on fire.
    His second emblem, of the Eucharist---the \foreign{sine qua non} of
    transformation through love---features the chalice and host flanked by two
    flaming braziers, all within a sun with flaming rays.
\end{Footnote}
%}}}3

%{{{3
\subsubsection{Voice as Burnt Offering}
Turning from physics to theology, we need to understand how the soul was to
be reborn through acts of loving devotion, and why fire was an apt metaphor
for this process.
Though this trope may be found throughout mystical literature, the definitive
treatment of fire as spiritual metaphor was the \wtitle{Llama de amor viva}
(Living Flame of Love) by Juan de la Cruz, written in Granada around 1582 and
published with his other works in Madrid in 1630.
The \wtitle{Llama de amor viva} is a poem followed by a treatise on affective
devotion in the form of a commentary on the poem.
In the poem, the soul professes its rapturous love for \quoted{the living flame
of love} with which the soul is burning
(\cref{poem:Juan_de_la_Cruz-Llama-opening}).
Though the Carmelite reformer's writings did not have as wide a reception in
the seventeenth century as those of his mentor Teresa of Ávila, the published
edition can be found in collections on both sides of the Atlantic.
It seems plausible that a mid-seventeenth-century poet writing about the
burning soul would have had some familiarity with Juan's writings or his
general theological emphasis, since Juan treats this particular theme with such
intensity.

Juan uses fire to represent the soul's purification and transformation, while
he uses the flames that rise from the fire to represent the holy acts that
proceed from a soul that has been thus purged and renewed.
\quoted{This flame of love}, he writes, \quoted{is the spirit of \add{the
soul's} Bridegroom, who is the Holy Spirit; and the soul indeed feels this
flame within herself, not only as a fire which holds her consumed and
transformed in tender love, but even as a fire that, beyond this, burns
\addorig{arde} in her and gives forth a flame}.% 
    \Autocite[790]{JuandelaCruz:Llama} 
The Holy Spirit works within \quoted{the soul tranformed in love} (or
\emph{into} love), bringing about a union of wills between the lover and the
object of love.
As the soul is being thus transformed, it gives forth \quoted{flames}, which
are the soul's acts of love for God.
Juan explains that \quoted{the difference between the habit and the act is that
between the transformation in love and the flame of love; that is, the
difference between the burning wood and the flame that comes from it; for the
flame is the effect of the fire that is there}.%
    \Autocite[790]{JuandelaCruz:Llama}

%{{{4 Juan llama
\insertPoem{Juan_de_la_Cruz-Llama-opening}
{Juan de la Cruz, \wtitle{Llama de amor viva}, poem (\poemlines{1--6})}
%}}}4

Juan's metaphor depends on his premodern scientific understanding. 
Flames go up for the same reason that stones go down, he says: a kind of
gravitation draws each object toward the center of the element from which it is
made, so the stone, made of earth, seeks the center of the earth, and the flame
seeks the center of the realm of fire, which Juan says is beyond the sky.%
    \Autocite[792--795]{JuandelaCruz:Llama}
The center of the soul, then, is God, and thus the soul is drawn to God like a
flame ascends to heaven.

In the full context of Juan's theology of divine union (particularly the
\wtitle{Subida del monte Carmelo}), the flames rising from the fire are the
actions of a person whose soul is united with God in will (not in nature) to
the extent that the soul \quoted{remains transformed into God through love}.%
    \Autocite[227]{JuandelaCruz:Subida}
In the condition of divine union these actions are the work of the Holy Spirit
alone, so that the person, the offering, and the recipient of the offering
become one:
\begin{quoting}
    And thus, in this state the soul can perform no acts; for the Holy Spirit
    does them all, and moves her to them all; and for this reason, all of her
    acts are divine, for she is made and moved by God.  
    From thence it seems to
    the soul that, each time this flame flickers up, making her to love with
    savour and divine temper, the flame is giving her eternal life, for it
    lifts her up, through the working of God, into God.%
        \Autocite[791]{JuandelaCruz:Llama} 
\end{quoting}
The soul's union with God in Juan's writings is a commitment of the self to
God, a sacrifice of all senses, pleasures, and desires like a burnt
offering---one of the primary images in Juan's \wtitle{Ascent of Mount Carmel}.
Just as Moses ascended Mount Sinai to build God an altar there (Exodus 34), the
soul must go through a process of self-emptying and thereby \quoted{ascend this
mountain and make of herself an altar upon it, on which to offer to God a
sacrifice of pure love, and pure praise and reverence}.%
    \autocite[191]{JuandelaCruz:Subida}
Despite the reformer's suspicion of elaborate church music (see
\cref{ch:faith-hearing}), Juan understands the voice as a medium for this kind
of offering, and he often uses the language of singing to refer to prayer.
Juan call his verses \foreign{canciones} (songs) and writes them in the
musically symbolic metrical form of \term{liras}. 
He writes that vocables like \foreign{ay} in his poem \quoted{signify
affectionate praising, which, each time they are said, reveal more about the
interior than what is said by the tongue}.%
    \Autocite[790]{JuandelaCruz:Llama} 
Even for this most ascetic of Spanish theologians, vocal expression, both
articulate and inarticulate, was one of the \quoted{acts of love} that were the
flames leaping up from the heart burning with the Holy Spirit.
Juan's concept of flames as the acts of love radiating from a soul being
transformed by the Holy Spirit accords well with the explanations of the
phoenix symbol by Covarrubia and Ricciardo, in which the soul is transformed by
the love of God and given new life through a purgative death.

In fact, a similar physical process was at play in vocal expression as in fire,
according to Fray Luis de Granada.
In his summary of widely held beliefs, the body as the microcosm of creation
is composed of the four elements, and the voice is generated from the tension
and exchange between them.
The heart is hot because it is fiery, and in order to cool off this fire and
prevent harm to the body, God surrounded the heart with the lungs, which are
full of cool air. 
The heat of the heart is cooled by the air of the lungs, and then comes forth
through the throat.%
    \Autocite[435]{LuisdeGranada:Simbolo}
In the throat, Fray Luis explains, the air meets with the voicebox, producing
the voice.
Fray Luis praises the virtues of musical song produced by this voice, not yet
articulated into words: how different bodies produce different voices, as in
the different parts of a chorus, and how their voices sounding together produce
\quoted{tuneful music} (\foreign{música acordada}) for the service of the
church.%	
    \Autocite[434]{LuisdeGranada:Simbolo}
When the voice meets with the tongue, lips, and teeth, 
\begin{quoting}
    \add{these organs} articulate the voice, and thus various words are formed,
    with which the man as a political being explains and declares his thoughts
    and ideas with other men. \Dots{} 
    In this once again his providence shines forth, since the hot air that the
    heart gives forth, being dangerous to itself, serves to produce such a
    beneficial thing as the voice and speech of man.%
        \Autocite[435]{LuisdeGranada:Simbolo}
\end{quoting}
The voice, then, literally arises through a kind of chemical process from the
fire in the heart.
The useful fruit of the interplay of opposed elements (in this case,
specifically opposed qualities of the elements---heat and cold) gives rise to
the means through which a person may make tuneful music and express his inner
thoughts to the external world.
	
Understood in this theological and scientific context, the villancico
\wtitle{Suban las voces} describes and embodies the worshipper's soul singing
as it burns with love, offering itself to God as a burnt offering like the
phoenix.
Its songs are the flames rising from the altar and are both the means and
result of the soul's regeneration.
Like Juan de la Cruz's poem, this poem uses the vocable \foreign{oh} and the
repeated refrain \foreign{arde} to express devotion that goes beyond words.
Through its explicit reference to the Eucharist the villancico unites the
soul's offering with that of Christ.
In the first musical setting, Bruna represents these ideas like an auditory
emblem and creates an affective world conducive to phoenix-like devotion.
The singers offer their voices to ascend to heaven like a burnt offering and
they invite listeners to join them in this devotional \quoted{act of love}.
The piece's function is not so much to provoke intellectual reflection or
wonderment (as in previous examples) but to kindle the hearts of performers and
listeners to affective devotion.

The villancico is a manifestation of one of the core theological foundations of
Christian, and particularly Catholic, worship: that the \quoted{reasonable
worship} Saint Paul describes in Romans 12 is not a particular liturgical form,
let alone a musical style; rather, as Saint Augustine writes in \wtitle{The
City of God}, true worship happens when the whole Christian community offers
itself as \quoted{a living sacrifice} to God, in union with Christ's sacrifice
on the cross and in communion with Christ present in the Eucharist.
When Christians speak of sacrifice and burnt offering, Augustine writes, they
do not mean pagan idols like the Greek gods or the Roman emperor: Only to the
one true God, he preaches, \quoted{we owe the service which is called in Greek
\foreign{latreía}, \Dots{} for we are all His temple, each of us severally and
all of us together}.%
    \Autocite[10:3]{Augustine:CityofGod}
\begin{quoting}
    \emph{Our heart when it rises to Him is His altar}: the priest who
    intercedes for us is His Only-begotten; we sacrifice to Him bleeding
    victims when we contend for His truth even unto blood; to Him we offer the
    sweetest incense when we come before Him burning with holy and pious love;
    to Him we devote and surrender ourselves and His gifts in us; to Him, by
    solemn feasts and on appointed days, we consecrate the memory of His
    benefits, lest through the lapse of time ungrateful oblivion should steal
    upon us; \emph{to Him we offer on the altar of our heart the sacrifice of
    humility and praise, kindled by the fire of burning love.}
    This is the sacrifice of Christians: we being many, are one body in Christ.
    And this also is the sacrifice which the Church continually celebrates in
    the sacrament of the altar, known to the faithful, in which she teaches
    that \emph{she herself is offered in the offering she makes to God}.%
        \Autocite[10:6, emphasis added]{Augustine:CityofGod}
\end{quoting}
Under this theology of worship, music and liturgy would be included as part of
the church's \quoted{sacrifice of praise and thanksgiving}, and Bruna's
villancico both proclaims and performs this kind of communal offering to God.

The modified version by Ambiela preserves this core theology of offering but
applies it to Mary rather than Christ, shifting the emphasis from communion to
intercession.
The musical references in the earlier poem are now put to a new ritual function.
Instead of drawing a comparison between the transformative power of fire and
that of music, now music's power to \quoted{transform the air} is connected to
the Blessed Virgin's translation from the realm of earth to that of Heaven.
All the same, Mary must pass through the worldly domain of the sky in order to
be assumed into the sphere beyond the physical realm (the Empyrean).
It is in that wordly sphere of air that human \term{musica instrumentalis} can
resound in praise of Mary and in imitation of her miraculous transformation.
Mary at her death remains \quoted{intact} (\foreign{entera}) not only as a
perpetual virgin but as a human body and soul in perfect harmony.  
Indeed, not only is Mary transformed from a mortal woman to an eternally living
heavenly saint, in her new domain she is crowned as Queen of Heaven, as she was
praised in the ubiquitous \wtitle{Salve Regina} chant and many others.
In this way some of the same aspects of musical theology that we have seen in
Christmas villancicos and in Bruna's Eucharistic villancico can be applied to
Mary.  
Instead of music pointing to the harmony between Christ's dual natures, here it
points to the unity of Mary's body and soul, not separated in death, and her
position between human and divine worlds as the one who bore God into the world
as Christ and the first of the saintly intercessors in heaven.
The coplas depict the world left behind by Mary, which cries out to her in her
new role as mediator through voices of music.
Their voices that ascend to heaven follow the path Mary followed: \quoted{sigan
sus pasos todos hasta la esfera}.
Thus human music is presented as a form of intercessory prayer, of which Mary
is the chapelmaster because she is the chief intercessor among all the saints.%
\begin{Footnote}
    Compare Sor Juana's villancico \wtitle{Silencio, que canta María}
    (\cref{ch:intro}).
\end{Footnote}
%}}}3
%}}}2
%}}}1

\endinput
% START

%{{{1 Caseda Christ as a vihuela
%************************************************************
\section{Christ as a \term{Vihuela} in \wtitle{Qué música divina} by José de Cáseda}
\label{sec:Caseda}

In the final case study of \cref{part:unhearable-music}, José de Cáseda's \wtitle{Qué música divina}, one can hear echoes from this whole tradition.
And as we will see, this villancico demonstrates what happens when imitation---both the musical tradition of metamusical villancicos and the philosophical concept of music's power to represent theological truths---is stretched to its limit.

José de Cáseda's Eucharistic villancico \wtitle{Qué música divina} represents Christ as a \term{vihuela}, or a \term{vihuela} as Christ.
The piece evokes instrumental music vocally in order to use music to symbolize not heavenly perfection, but instead to evoke the human suffering of Christ on the cross as a result of humankind's sinful nature.
Cáseda forces his singers to make untuneful music that both highlights the imperfection of human music (as in previous metamusical examples) and also provides a vehicle for dramatic, affective human expression. 
Thus the piece forms a good endpoint to the trajectory we have traced from music functioning as a reflection of heaven to music expressing human affections.
Since this villancico survives in a manuscript from a convent in Puebla, it also demonstrates the global reach of this kind of music (yet another result of the interconnections among Hispanic musicians), and raises questions about the effect of performing venue and audience on the piece.

The musical oddities of Cáseda's villancico may reflect the strain of continuing the metamusical villancico tradition.
This piece thus represents the limits of imitation, in both the difficulty of outdoing previous compositions in the same musical tradition, and in the increasing struggle to accommodate musical aesthetics to the old cosmology even as belief in the old system was eroding.

The composer José de Cáseda (fl. 1691--1716) was raised and trained within a closely integrated network of composers, centered in the Zaragoza region of Aragon but with ties all across Spain and the New World.%
	%
	\footnote{
For what little is known of his biography, see \autocite[120--121]{Calahorra:Zaragoza2}; and \autocite{Stevenson:CasedaD}. 
	}
	%
José's father Diego has already appeared several times in these case studies; in Zaragoza he was chapelmaster first at the Basilica de El Pilar and then later at the Cathedral of La Seo.
José first served as chapelmaster of cathedrals in Calahorra and Pamplona. 
After Diego's death in 1695, José was appointed to succeed him as chapelmaster at La Seo in Zaragoza, where he remained until at least 1705.
Possibly through a connection with Miguel Mateo de Dallo y Lana (who emigrated from Seville to Puebla), music by both Cásedas made it across the Atlantic to New Spain, where it became part of the repertoire at the Conceptionist Convento de la Santísima Trinidad in Puebla.
Today that convent's collection survives as the Colección Jesús Sánchez Garza at the Mexican national music research center (CENIDIM) in Mexico City.
The collection includes eight villancicos attributed to a Cáseda, with three clearly credited to José (MEX-Mcen: CSG.151, 154, 155).%
	%
	\footnote{
	The numbers are from the still-unpublished catalog edited by Aurelio Tello.
	}
	%

The cheap paper and unprofessional handwriting of the manuscript (CSG.154) suggest that \wtitle{Qué música divina}, like most of the other villancicos in the collection, was copied by the sisters of La Santísima Trinidad for their own use in the convent.
Indeed, their names are written into the parts---the same names that may be seen in many other Sánchez Garza manuscripts from the same period.
The Tiple I part belonged to \quoted{Tomasita}, the Tiple II to \quoted{María de Jesús}, the Alto part to \quoted{Me. [Maestra?] Besona}, and the Tenor part to \quoted{Rosa María de Jesús}.
The instrumental bass part is not attributed.
The paper's watermarks and the handwriting are consistent with a copying date around 1700.
The handwriting may not be of professional quality, but it is no worse than the Barcelona manuscript of Bruna's \wtitle{Suban las voces} or the Canet manuscript of Cererols's \wtitle{Suspended, cielos}. 
The sisters of La Santísima Trinidad must have had quite a high level of musical ability, as this is a difficult piece, both in its performance demands and its aesthetic character. 

The musical style and poetic themes suggest a function for Eucharistic devotional services, like Bruna's \wtitle{Suban las voces}.
The condition of the partbooks suggests that Cáseda's piece was performed frequently in the Puebla convent.
In fact, a sewn-in line of music in the Tenor part, which may have served as a way to abridge the coplas or adapt the villancico for solo performance, shows that later convent sisters valued this work highly enough to adapt it for new performative or liturgical demands.

%****************************************
\subsection{Metamusical Tropes in the Poetry: Experiencing a Higher Music}

The poetry of the estribillo reads like a catalog of the metamusical tropes we have been tracing in these case studies of \quoted{singing about singing} (\cref{poem:Que_musica_divina-estribillo}).
The terms used in Cáseda's villancico to describe music are all familiar by now from the vocabulary of this tradition: \quoted{acorde}, \quoted{soberana}, \quoted{tiernas}, \quoted{armoniosas}. 
The key terms \quoted{consonancia}, \quoted{quiebros}, and \quoted{accentos}, are common to all the pieces studied previously.
The term \quoted{cláusulas} specifically recalls \wtitle{Suspended, cielos}, as set by Cererols. 
Like Irízar's \wtitle{Qué música celestial} and the variant settings of \wtitle{Suban las voces al cielo} by Pablo Bruna and Miguel Ambiela, the poem also connects music with the element of air through reference to the winds.

% %********************
% \begin{expoem}
% 	\caption{\wtitle{Qué música divina}, poem as set by José de Cáseda, estribillo}
% 	\label{poem:Que_musica_divina-estribillo}
% 	\input{poems/Que_musica_divina-Caseda-estribillo}
% \end{expoem}
% %********************

% %*******************
% \begin{expoem}
% 	\caption{\wtitle{Qué música divina}, coplas}
% 	\label{poem:Que_musica_divina-coplas}
% 	\input{poems/Que_musica_divina-Caseda-coplas}
% \end{expoem}
% %*******************

Cáseda's poem begins with a rhetorical exclamation of wonder about a mysterious higher form of music, as did Irízar's \wtitle{¿Qué música celestial es la que hoy el aire altera?} and Jerónimo de Carrión's \wtitle{Qué destemplada armonía de confusas voces varias}.
Irízar's opening uses the same term (\quoted{el oído eleva}), which like Carrión, rhetorically presents dismay at the power of a strange, unexpected kind of music.
This emphasis on the human sensory and affective response to music fits with the general trend we have seen (in Irízar and Carrión) away from music as a reflection of heaven, such as the music of the spheres, and toward the evocation and incitement of human affects.

The final lines of Cáseda's estribillo describe the \quoted{divine music} as elevating the senses and dismaying the body's powers: here the poet distinguishes between two technical terms for human faculties, the \quoted{sentidos} and \quoted{potencias}.
As explained in chapter~\ref{ch:theology}, early modern physiology distinguished between the exterior senses, which Fray Luis calls \foreign{sentidoes exteriores}, and the interior senses (the imaginative, cogitative, and other faculties), which Fray Luis describes as \foreign{potencias}.
In Cáseda's villancico, the \quoted{divine music} \quoted{elevates the senses} but \quoted{confounds the powers}.
Cáseda's \quoted{sentidos} would most likely refer to the exterior senses---hearing in particular---and \quoted{potencias}, to these mental faculties, what might be called \quoted{powers of reason}.
Coplas 5 and 6 (\cref{poem:Que_musica_divina-coplas}) expand on this idea: the divine music \quoted{is not to the senses} (\poemlines{28}), but its \quoted{excellence} or virtuosity (\quoted{primor}) \quoted{elevates to the heavens the one who reaches it} (\poemlines{32}).
Here copla 6 connects the divine music to the Eucharist: the mystery of either is more than the exterior senses perceive it to be, and therefore \quoted{sensation does not eat it, for your music is fodder for the soul} (\poemlines{31--34}).

The basic concept of \quoted{sentidos} and \quoted{potencias} in this villancico, then, seems to be that the divine music defies the external senses but elevates the internal powers of the mind/soul.
This idea seems to fuse the Neoplatonic contemplative function of music with affective experience: perhaps rational/spiritual reflection on the divine mysteries is achieved not through ignoring the senses, but through an experience of the senses being confounded.
Such a theology of sensory experience would align closely with that of St. John of the Cross, in which affective experience functions like \quoted{mother's milk}, leading the contemplative toward a higher plane of spirituality in which that sensory experience is no longer necessary.

To describe this dismaying higher form of music, the coplas of Cáseda's villancico, contrasting with all the generic tropes in the estribillo, center on the theological-musical conceit of Christ as a \quoted{divine and human \term{cítara}}.
The villancico uses this musical instrument---whose ambiguous definition led to a rich range of symbolic meanings---as a metaphor for the highest form of Music.

In this poem, Christ unites the high and low like two strings of the \term{cítara} tuned to each other at the octave (copla 1). 
Christ unites these extremes as the strings are stretched between the two ends of a \term{cítara} (\quoted{lazo} also calls to mind the lash used on Christ in his crucifixion).
This is fitting, since Christ was stretched out on the cross just as the strings are stretched over the bridge, and the three nails put in his hands (\quoted{clavos}) are the pegs (\quoted{clavijas}) that hold the strings in place (copla 3).
The lance pierced his innards as a plectrum plucks strings (\quoted{herir} being the same word for both \quoted{pluck} and \quoted{wound}).
And as, according to John's Gospel (19:34) blood and water flowed from Christ's wound, which in Tridentine theology were understood as the fountainhead of the seven sacraments, these are compared to the \quoted{seven orders} or strings of a \term{vihuela}.

%********************
\subsection{Theological Tropes of the Cithara and Vihuela}

But if Christ is a musical instrument, is he a \term{cítara} or a \term{vihuela}?
The specific reference in copla 4 and other details make clear that the metaphor in this villancico is specific to the Spanish \quoted{seven-order} vihuela.
But the symbolic use of the vihuela is piggybacking on a millenium of allegorical treatments of the other instrument.
The meanings of the \term{cithara} in early modern Latin and its Spanish cognate \term{cítara} developed from a long practice of interpreting an ancient musical term of unclear meaning through the lens of familiar contemporary practice.

The Jewish translators of the Septuagint a century before Christ used
\term{kithara}, the term for some kind of lyre or harp used in ancient Greek writings, to translate several ancient Hebrew terms whose precise meanings in some cases remain unclear today.%
	%
	\footnote{%
	For a full discussion of the different terms in ancient Near Eastern languages and the archeological and iconographic evidence for their meaning, see \autocite{Lawergren:Lyres}.
	}
	%
Tracing the use of \term{cithara} in the Vulgate back through the corresponding passages in the Greek (the original text of the New Testament and the Septuagint translation of Old), through the Hebrew Scriptures, the most common Hebrew word standing in the same place as the Vulgate \term{cithara} is \term{kinnôr}.
The \term{kinnôr} was some kind of lyre with plucked or strummed strings; Josephus said it had ten strings.%
	%
	\autocites[\sv{Music: Strings}]{Bromily:BibleEncyclopedia}
	%
In Gen 4:21 this is one of the first instruments ever invented by \quoted{Jubal, the ancestor of all those who play the lyre and pipe} (NRSV).

In the Septuagint, the names of these two instruments are translated as the
\foreign{psaltērion} and \foreign{kithara}.
St. Jerome rendered this passage in the Latin Vulgate with the words
\foreign{cithara et organo}, transliterating the Greek \foreign{kithara} and perhaps trying to recuperate the sense of a wind instrument in the other Hebrew word, which the Septuagint translators had turned into a stringed instrument, the \term{psalterion}.

This example is one of several that demonstrate how the precise meaning of a musical term could be lost in the transfer between Hebrew, Greek, and Latin.
In the Vulgate, the instrument David plays for Saul in ISam 16:16 is a \term{cithara}.
In Hebrew this was again \term{kinnôr}, rendered in the Septuagint with another
frequently used translation, \foreign{kinura}.%
	%
	\footnote{%
	This Greek word may be a loanword from the Hebrew: \autocite{Brown:HebrewOTLexicon}{}.
	}
	%
The same transfer of terms happens in the descriptions of worship in the Davidic
tabernacle (1 Chron. 15:28, 16:4--6): in Hebrew David's temple musicians play
the \foreign{kinnôr}; in Greek, the \foreign{kinura}, and in the Vulgate they
play \foreign{citharae}.%

For Christian theology, the most prominent Biblical locus for the Greek
\foreign{kinura} and Latin Vulgate \term{cithara} is in the New Testament Revelation to John.
In Rev 14:2-4, John hears a chorus of 144,000 virgins singing \quoted{a new song
before the throne}, and both the Greek and the Latin onomatopoetically echo
their sound, \foreign{sicut citharoedorum citharizantium in citharis suis} (\quoted{harpers harping on their harps} in the King James), a passage whose meaning Craig Monson has already elucidated.%
	%
	\autocite[88--95]{Monson:DivasConvent}
	%

Many medieval exegetes commenting on the Latin Vulgate likely had no idea (or desire to know) what actual instrument the term \term{cithara} referred to, and concerned themselves instead with analogical interpretations.
By the seventeenth century, the term had become a rich node of allegorical connections.
In a 1603 commentary on the Apocalypse, the Jesuit Francisco Ribera draws on the venerable Bede to interpret the cithara played by the saints in Rev 14: as symbolic of the saints' bodily mortification: 
\quoted{Counted among the cithara-players of God are all the saints, who, having crucified their flesh with its vices and sinful desires praise God with the psalter and cithara}.%
	%
	\footnote{
	\autocite[429]{Ribera:Apocalypse}.
This is probably the symbolism behind the use of this reading for the Mass of the Feast of Holy Innocents (Jan. 28).
	} 
	%

Cornelius a Lapide takes the connection between cithara and crucifixion farther.
In his commentary on ISam 16:, when David (in the Latin version) plays the cithara to drive the demons away from King Saul, Lapide gives an epitome of the whole patristic and medieval tradition regarding the power of music over the affects.
Surveying the exegetical tradition to its earliest and most obscure sources---Lapide cites Angelomus of Luxeuil, Prosper of Aquitaine, Eucherius of Lyon, and Ambrosius Ansbertus---Lapide summarizes a traditional interpretation of the cithara:
\quoted{Allegorically, the cithara represents the cross of Christ; for just as the strings of a cithara are stretched out, thus Christ was stretched out on the cross}.%
	\Autocite[370]{Lapide:1Samuel}
Lapide cites Augustine (\wtitle{Sermo 3 de tempore}) to say that \quoted{the cithara represents the flesh of Christ}: speaking apparently of the Greek three-stringed lyre, Augustine makes this a metaphor for the unity of the three Persons of the Holy Trinity, incarnate in the body of Christ.%
	%
	\Autocite[370]{Lapide:1Samuel}
	%
Once again it is apparent how the ambiguous meaning of the instrument name gave theologians great flexibility in interpreting the instrument symbolically.

Lapide joins Ribera in citing the Venerable Bede for the allegorical reading of the cithara.%
	%
	\footnote{%
Bede's complete works had been published in Basel in 1563 (preserved in Madrid, E-Mn:~M/1069); they were newly published in Cologne in 1688 (preverved in Puebla's Biblioteca Palafoxiana,~MEX-Ppx: BS535-B4).
	}
	%
Indeed, in a commentary on ISam 19:10 (in which Saul attacks David with a lance while David is playing music), Bede uses language strikingly similar to that of Cáseda's villancico, as shown in the added emphases:

\begin{quoting}
The cithara [citara] of David especially may figuratively demonstrate the cross of the Lord, the lance [lancea] of Saul may be compared to the nails [clavos] of the cross, as well as the soldier's lance, by which the Lord's side was opened.%
	\autocite[123]{Bede:Commentaries2}
\end{quoting}
%
Note Bede's use of the terms \quoted{clavos} (the exact cognate is used in the Spanish villancico) and \quoted{lancea} (\quoted{lanza} in the Spanish).

All of these sources manifest a strong theological tradition behind the use of the cithara in Cáseda's villancico as an allegory for Christ's incarnate body and his crucifixion.
Moreover, Craig Monson has traced how this symbolic tradition was applied specifically to female monastics, citing St. Bonaventure for \quoted{this relationship between Christ's suffering body, the kithara, and the female monastic}---which would give the piece a special resonance in the context of the Puebla convent community.%
	%
	\autocite[93--94]{Monson:DivasConvent}
	%

By the time Lapide's allegorical reading of the cithara was published, humanist research was already beginning to uncover the original meanings of the cithara and the Hebrew instruments given that Latin name.
Athanasius Kircher, dedicated separate chapters of the \wtitle{Musurgia} to the musical instruments of the ancient Hebrews and Greeks, and attempted to clarify the differences between the various Hebrew and Greek terms used for stringed instruments.
Kircher cites the same passage from Gen 4:21 discussed above, printing both the Hebrew and Greek versions, and noting the variance in translation of the instrument names.
Kircher says that \foreign{cythara} is the translation for the Hebrew words (in his transliteration) \quoted{Assur, Neuel, Kinnor, Maghul, Minnim}.
The \quoted{Kinnor}, Kircher says, \quoted{is in fact of a similar character to the \emph{Cytharæ} of today}, and provides an illustration of what he thinks the Hebrew instrument looked like (\cref{fig:Kircher-kinnor}).%
	\autocite[I:~44--49]{Kircher:Musurgia}
Note how the Latin term constrains Kircher: though he is attempting to compare the Hebrew instrument to a modern one, his use of the ambiguous Latin \foreign{cythara} works against him---what was \quoted{the cithara of today}{}?
In his discussion of Greek music, Kircher speaks of \quoted{cytharæ} without giving a clear definition.
Comparing of ancient and modern music, Kircher boasts that the \quoted{cytharœdi} of today (cithara players, using the same term from Rev 14:2) are as superior to their ancient Greek counterparts as their modern instruments are superior to the ancient ones.
	%
	\autocite[I:~548]{Kircher:Musurgia}
	%

Kircher does attempt to define the \quoted{cythara} as a term for modern plucked string instruments, though again the Latin word leaves his definitions ambiguous.
In his illustrations of modern \quoted{cytharæ} (\cref{fig:Kircher-citharae}), the \quoted{common Cythara} has a round body and seventeen strings; the \quoted{German and Italian Cythara} has a pear-shaped body and four double courses (possibly a mandolin).
Most significantly, Kircher's \quoted{Spanish Cythara} appears to be a vihuela with five double courses.

% %********************
% \begin{figure}
% 	\includeShortFigure{figures/Kircher-1_49-kinnor}
% 	\caption{The Hebrew \term{kinnôr}, according to Athanasius Kircher}
% 	\label{fig:Kircher-kinnor}
% \end{figure}
% %********************
% %********************
% \begin{figure}
% 	\includeMediumFigure{figures/Kircher-1_477-citharae}
% 	\caption{Modern \quoted{cytharæ}, according to Kircher}
% 	\label{fig:Kircher-citharae}
% \end{figure}
% %********************


In describing modern instruments, using vernacular names would have increased the accuracy of his descriptions but diminished its universality, and all the rich associations of the cithara would have been lost.
There is a pronounced tension throughout the \wtitle{Musurgia}---and throughout early modern Catholic culture---between the desire to preserve the Latin Catholic traditions, with their analogical ways of thinking, and the desire to investigate the modern world scientifically and empirically.
Kircher presents \quoted{scientific} knowledge about the ancient world and encyclopedic descriptions of modern practice, but he also wants to preserve the allegorical traditions of Catholic theology and speculative music theory.
For example, the entire second book of the \wtitle{Musurgia} is based on a sustained metaphor comparing all of creation to a Greek four-stringed lyre, another instrument often grouped with the cithara as a source for allegory.

John Hollander appraises the early modern situation similarly, seeing the substitution of modern instruments for the cithara in the early modern period (\quoted{apparently based on the notion that any obsolete instrument is the equivalent of any other}) as emblematic of how poets struggled to accommodate musical concepts from the ancient world to modern reality.
Hollander describes a symbolic \quoted{{`lute-harp-lyre'} constellation, uniting the contemporary instrument with those of David and Orpheus}.%
	%
	\autocite[44--51]{Hollander:Untuning}
	%
Hollander argues that by this point in the seventeenth century, the interest of poets and musicians writing on musical subjects turned away from musical philosophy (music of the spheres and so on) and toward actual musical practice. 
In a similar way, the poet of Cáseda's villancico takes the cithara, with all the analogical possibilities connected to it, and maps it onto a specific modern instrument---the distinctly Spanish seven-course vihuela.

Spanish church music was distinctive among European traditions in its widespread inclusion of several plucked stringed instruments not used elsewhere, primarily, the harp, guitar, and vihuela. 
The terminological and symbolic ambiguity of the cithara allowed Spanish artists considerable license in connecting their contemporary musical practice to the ancient sources.
Many Spanish cathedral Choirs include depictions of King David with his cithara (or harp, or lyre) as a way of demonstrating continuity with the music of the ancient Hebrew Temple (a concept also expressed through the \quoted{Solomonic columns} built into the high altar of Puebla Cathedral in 1649, and thereafter widely imitated). 

On the walls and ceilings of the cathedrals of Puebla and Mexico City, these instruments may be seen in the hands of angelic musicians.
The Chapel of the Rosary (Capilla del Rosario), built in Puebla around 1680, also includes a \wtitle{Glorification of the Virgin} amid an angelic ensemble that includes either a vihuela or a guitar.
Juan Correa's painting of the same subject in the sacristy of Mexico City Cathedral (\circa1685) specifically includes a vihuela in a heavenly consort of harp, lyre, lute, and viols (to mention only the stringed instruments).
In this painting (\cref{fig:Correa-Sacristy}), it is as though Correa takes the Biblical passage about \quoted{citharoedorum citharizantium in citharis suis} (Rev 14:2) and, drawing on the diverse interpretations of the cithara, includes in the heavenly ensemble every instrument used in contemporary cathedral practice that could possibly be considered a cithara.


% %********************
% \begin{figure}
% 	\includeMediumFigure{figures/Correa-Sacristy}
% 	\caption{A heavenly consort of \quoted{citharoedorum} in Correa's \wtitle{Glorification of the Virgin} (\circa1685), sacristy of Mexico City Cathedral}
% 	\label{fig:Correa-Sacristy}
% \end{figure}
% %********************

Most of what is known about the vihuela comes from Fray Juan Bermudo's 1555 \wtitle{Declaración de instrumentos musicales}%
	%
	\autocite{Bermudo:Declaracion}.
	%
In his opening explanation of the Boethian three-fold division of music, the vihuela is the first instrument Bermudo lists as an example of \quoted{artificial} \wtitle{musica instrumentalis}.
Though the more common type of vihuela had six orders (that is, six pairs of strings), Bermudo also desribes a \quoted{vihuela de siete órdenes}, as mentioned in Cáseda's villancico.%
	%
	\footnote{%
	See \autocite[90v--110r]{Bermudo:Declaracion}.
	}
	%
This instrument (\cref{fig:Bermudo-vihuela7}), Bermudo says, was particularly suited to playing polyphony, such as five-voice works by Gombert (this is notable because the Cáseda villancico is scored for five voices).

% %********************
% \begin{figure}
% 	\includeNarrowFigure{figures/Bermudo-110r-vihuela7}
% 	\caption{The seven-course vihuela, in Bermudo, \wtitle{Declaración de instrumentos musicales}, 110r}
% 	\label{fig:Bermudo-vihuela7}
% \end{figure}
% %********************

Indeed, vihuelas were an important part of the Spanish royal musical ensemble from the days of Charles I.
Vihuela intabulations survive of masses by Cristóbal de Morales and Francisco Guerrero.%
	%
	\autocite{Grove:Vihuela}
	%
Vihuelas were almost certainly part of José de Cáseda's ensemble in Zaragoza, since the instrument is mentioned frequently in the cathedral chapter acts.%
	%
	\footnote{%
	Multiple citations in \autocite{Calahorra:Zaragoza2}.
	}
	%
In Puebla, the cathedral chapter paid 100 pesos to one Diego de León in 1676 for playing the \term{vihuela de arco},%
	%
	\autocite[44]{PerezRuiz:Aportes}
	%
demonstrating the use of vihuelas in the cathedral and suggesting their use as well in the closely related musical ensemble of the Convento de la Santísima Trinidad.

We know that women religious played the vihuela because one of the two surviving vihuelas from this period, in the church of La Compañia de Jesús in Quito, Ecuador, is believed to have been the possession of the nun Santa Mariana de Jesús (1618--1645).
According to contemporary accounts, Mariana was especially skilled on the instrument. 
The theological worth then attributed to the vihuela is shown in an account of one Christmas night (probably during the Matins service, when villancicos were performed), when Mariana \quoted{sat down to make music playing a vihuela, and she said that she wanted to offer this music among the angels who were attending there}.
	%
	\footnote{%
	\autocite[275]{EspinosaPolit:SantaMariana}, quoted in \autocite[73]{Bermudez:Vihuela}.
	}
	%
This form of devotional performance fits well with the original meaning of the
\foreign{kitharōdos} in Rev 14:, \quoted{lyre-player, harpist who plays an accompaniment to his own singing}.%
	%
	\autocite{BDAG}
	%
It is quite likely that the nuns of La Santísima Trinidad in Puebla also included at least one vihuela in their musical ensemble---and it would certainly seem strange for them not to use that instrument in performing a villancico that used it as a metaphor for Christ.

In the case of Cáseda's villancico, the specific details of the vihuela---its construction, tuning, playing technique, and typical stylistic idioms---provide new analogical possibilities for the poet to extend the cithara tradition.
More importantly, this poetic conceit of Christ as vihuela also provides Cáseda as a composer with possibilites for actually representing the cithara symbol through sound.

%********************

\subsection{Representing the Vihuela, Representing Christ}

In the poetry of Cáseda's villancico, the specifications of the vihuela become symbols for Christ, particularly for his body, which suffered on the cross, was raised and made present to believers through the Eucharist.
The vihuela's seven strings here symbolize the seven sacraments.
According to Bermudo, the most common tuning for the seven strings was in intervals of alternating fifths and fourths starting from a lowest string on G (\term{gamma, ut}) (that is, G\textsubscript{2}).%
	%
	\autocite[109r--109v]{Bermudo:Declaracion}
	%
That would make the strings G\octave{2}--D\octave{3}--G\octave{3}--D\octave{4}--G\octave{4}--D\octave{5}--G\octave{5}.
Thus the highest and lowest strings are tuned in octaves, as copla 1 says: \quoted{forma unida la alta con la baja}.
If this phrase referred to \quoted{that which is high} or low in general, it would have been gendered masculine, \quoted{lo alto con lo bajo}; but since it is feminine, it makes more sense as modifying \quoted{cuerdas} from the first line---the high string with the low string.
Moreover, all the strings are tuned in perfect intervals, as copla 2 says: \quoted{en cada punto entera consonancia} (in each point or note a whole consonance).
The \quoted{lazo} could refer either to the bow of a \term{vihuela de arco}, or to a plectrum that was sometimes used in place of the fingertips.

The poet has even incorporated the instrumental symbolism into the structure of the verse (here playing on the common conflation of cithara, lyre, and Greek \term{lira}).
The poem itself is composed primarily in lines of 7 and 11 syllables, beginning with the pattern 7--7--7--11--11; this would seem to be a type of \term{lira} meter, like the one used in the \wtitle{Llama de amor viva} of St. John of the Cross.
Seven is also the number of courses on the vihuela described in the poem (\poemlines{26}).
So the poem itself is a \term{lira}, even as it describes a cithara/lira/vihuela as symbolic of Christ.

Whether or not an actual vihuela was used for this villancico, Cáseda represents the vihuela musically through the compositional structure.
First, he features the continuo section prominently, spotlighting all the cithara-like instruments that might have been played.
The piece begins with the Tiple I (sister Tomasita in Puebla) singing a solo against the continuo with widely-spaced open fifths and octaves between them: the vihuela or other continuo instruments filling in the intervening space would have stood out clearly.
Again in \measures{2}, when the Tiple I makes the striking leap up to the B\fl{}, she is joined only by the continuo.
In \measures{13}, there is an abrupt harmonic shift initiated by the continuo alone, which here leads the singers rather than just accompanying them.

Cáseda also gives the continuo several solos throughout the piece.
The first solo comes at the conclusion of the first three lines of poetry (\measures{9}).
Surely the composer intended for more music to sound here than simply the falling fifth of the melodic bass line.
Indeed, with the vocalists having just sung that the \quoted{divine music \Dots\ rivals that of the birds}, it would seem natural for an instrumentalist to fill in a little trilling bird music here.
Cáseda allows more such possibilities in the coplas (\measures{52, 54, 57, 72, 76--77, 80}): in the first example, the ensemble sings \quoted{let the divine strings resound}, and then there are two semiminims of vocal rest while, we may assume, the continuo ensemble does just that.

Cáseda also has the singers themselves imitate the vihuela.
In the opening gesture, the Tiple I sings her solo with continuo accompaniment, followed by the rest of the vocal ensemble in a homorhythmic echo (\cref{ex:CasedaJ-Que_musica_divina-opening}).
The \quoted{voicing} here (as in chord voicing) resembles the tuning of a vihuela, with the open fifths and fourths in the three lower vocal parts of \measures{1--2}.
The dotted rhythm, sung all together, and the contrary motion between voices, mimic the effect of strummed open strings on a vihuela.
The general texture of soloist against a regular rhythmic, chordal accompaniment also evokes someone singing while playing (like Santa Mariana de Jesús of Quito, and the \quoted{citharoedi} of Rev. 14).
This image would be even clearer in the coplas sung by soloists with only continuo accompaniment.

% %*******************
% \begin{example}
% 	\includeWideFigure{scores-examples/CasedaJ-Que_musica_divina-opening}
% 	\caption{Cáseda, \wtitle{Qúe música divina}, \measures{1--6}}
% 	\label{ex:CasedaJ-Que_musica_divina-opening}
% \end{example}
% %*******************

The vocal textures from \measures{6} on are more typical of vocal music, particularly in the paired, ornament-like figures in the sections from \measures{11--15, 19--29}.
After \measures{18}, Cáseda realizes the common villancico poetic trope of birdsong by giving the singers birdlike trill patterns. 
At the end of the estribillo, Cáseda returns to musically representing the vihuela/cithara trope.
In the last eight compases (\measures{47--50}), the rhythmic pattern in the voices---a minim rest and two minims---again suggests strumming (see \cref{ex:CasedaJ-Que_musica_divina-clausulas} below).
If this passage were played by a vihuelist in an intabulation, the player would likely use a down-up-up strumming pattern, similar to the patterns recommended for guitarists and harpists in Lucas Ruíz de Ribayaz's manual \wtitle{Luz y norte musical} of 1677 (\cref{fig:Ruiz-9-strumming}).%
	%
	\autocite[9]{Ruiz:Luz}
	%
In this rhythmic pattern, Cáseda inserts rests between syllables of the words in this passage; this rhetorical technique of \wtitle{tmesis} creates the gasping effect of \quoted{dismayed}, arrested senses.

% %********************
% \begin{figure}
% 	\includeNarrowFigure{figures/Ruiz-9-strumming}
% 	\caption{Strumming patterns in Ruíz de Ribayaz, \wtitle{Luz y norte musical}}
% 	\label{fig:Ruiz-9-strumming}
% \end{figure}
% %********************


In the coplas, the homorhythmic, dotted opening phrase again seems to mimic strumming.
For the phrase \quoted{forma unida la alta con la baja} (\measures{62--66}), Cáseda sets the text on multiple levels at once.
In \measures{62--63}, the Tenor sings a pedal D\textsubscript{4}, like a droning open string, against the Alto's moving line.
Thus the \quoted{alta} Alto (with a woman singer the feminine adjective takes on an additional layer of meaning) is paired with the lowest voice.
The Tenor, meanwhile, forms perfect consonances (octave and fifth) with the Bass, which (if a vihuela) would likely be playing open D and G strings here. 
The Tenor is thus acting like one of the strings on the vihuela.
Finally, the Tenor line concludes with a minor-sixth leap down to F\sh{} on the word \quoted{baja}, painting the text with a traditional madrigalism.
All of these ideas are then repeated in the next phrase, \measures{64--66}, now with G pedals matching the highest string on the standard vihuela (G\textsubscript{5}).

These evocations of the vihuela would seem to fit with Hollander's thesis that early modern poets shifted their interest from speculative views of music based on ancient sources toward the details of practical contemporary music.
But Cáseda's villancico does not abandon traditional analogical thinking; if anything it makes the metaphor more specific (and therefore more powerful) by using not just the abstract term \term{cithara}, but a particular type of vihuela. 
In classic Neoplatonic fashion, the piece shows listeners how to hear \wtitle{musica instrumentalis} while listening for the higher Music of Christ. 
The real, sounding vihuela (if indeed there was one) is only a symbol of Christ.
It is Christ's musical \quoted{excellence} that is praised, not that of any human virtuosi.

%********************
\subsection{The Problem of Hearing and Faith}

Eucharistic villancicos like this one tended to emphasize the mysterious, sense-defying nature of the holy sacrament. 
In the words of St. Thomas Aquinas, the fundamental doctrinal authority on sacramental theology for early modern Catholics, \quoted{That the true body of Christ and his blood are in this sacrament, cannot be grasped by sensation [neque sensu] nor by intellect, but by faith alone, which rests on divine authority}.%
	\footnote{\wtitle{Summa theologica} III, question 75, article 1; 1859 ed. p. 274}
Cáseda's copla 5, \quoted{No son a los sentidos} (\poemlines{27}) seems to echo Aquinas's phrase \quoted{neque sensu}, and certainly reflects the same Eucharistic theology.
In Cáseda's poem (\poemlines{33}), \quoted{sensation does not eat it}; for as Aquinas explains in the same article, Christ's body is eaten in a sacramental, not literally physical way.

From the theology of transubstantiation it would follow that music should be especially apt for the Eucharist (both as metaphor and liturgical practice) because of the analogy between the mysterious way both appealed to the senses.
Like Christ's presence in the Eucharist, a musical voice was a kind of presence that was in the world but in a way not of it. 
This presence was independent of the singer's or player's body: it transmitted something of their essence without any physical contact.
Further, the ear could be deceived by echoes, feigned voices, or false speech, as the character Hearing says of himself in Calderón \wtitle{El nuevo palacio del Retiro} (discussed in chapter~\ref{ch:theology}).
One could even hear something said or sung with perfect clarity, but not understand it for lack of the proper disposition or intellectual preparation.

This problematic theology of faith and hearing is presented in Cáseda's villancico through the specific metaphor of Christ as a stringed instrument.
\quoted{Of faith he is the intrument}, the poem says, \quoted{and his music regales the ear [or, hearing]} (copla 2).
Coplas 3 and 4, as we have seen, describe the \quoted{music} played upon that instrument: namely, Christ's sacrificial death on the cross.
But, as copla 5 says, \quoted{those things that his sovereign voices [or words] sound are not for the senses}---literally, they do not sound to the senses. 
In other words, this music is not something that can be heard with the ears.
Indeed, the copla continues, \quoted{as many voices [or words] as the senses perceived from this instrument will be false} or out of tune.
So if the ears \emph{could} hear this music, they would perceive it as being out of tune, or it would sound otherwise false, untrue, strange. 
All this seems to be a way of saying---in line with Aquinas---that those who rely on their senses, understood through reason alone, to grasp the mystery of Christ in the Eucharist will fail.
Like Calderón's \quoted{Judaísmo}, they would be hearing Faith without faith.

Obviously the main reason for this is that (as Aquinas explains) the eyes see only bread, the hands feel only bread, where really in Catholic belief, after the consecration the substance of the bread is transformed into that of Christ.
Furthermore, in the seventeenth century, the words of consecration were either whispered by the priest or not spoken at all (as contemporary ritual books specify), and most participants in the liturgy received the bread only rarely and the cup never.
Thus the Eucharist for most Catholics was something literally beyond all sensation, an utter mystery kept out of reach and mostly out of sight.
But if participants in the liturgy did not have direct physical contact with Christ in the sacrament, they did have direct access---through their sense of hearing---to the liturgical music.
When this music was itself about communion with Christ, it was able to serve as a kind of \quoted{sacramental}, that is, something outside the seven official sacraments that nevertheless was a means of encountering God through the material world (such as holy oil or water outside the sacraments of Confirmation or Baptism).

As the Council of Trent and the Tridentine Catechism firmly emphasized, Christ's sacrifice on the cross was ritually enacted and made present again on the altar as an propitiatory offering to God and a means of communion for humanity. 
This connection between Christ's crucifixion and the Eucharist explains the emphasis on both in Cáseda's villancico, and it adds another dimension to the musical metaphor.
According to Cáseda's coplas 3 and 4, if Christ is a vihuela, the music played on the instrument of his body is his crucifixion.
On the cross, the sinless Christ exchanged places with sinful humanity.
As St. Paul puts it, \quoted{For our sake he [God] made him [Christ] to be sin who knew no sin, so that in him we might become the righteousness of God} (IICor 5:21, NRSV).

We have previously seen that metamusical villancicos frequently use dissonance or being out of tune as a symbol for sin. 
In Jerónimo de Carrión's \wtitle{Qué destemplada armonía} (chapter~\ref{ch:Segovia}), the fallen nature of humanity (that is, the \term{musica humana}) is presented as an out-of-tune or \quoted{untempered harmony}. 
To read Cáseda's villancico together with Carrión's, then, the reason why the music played on Christ the vihuela sounds \quoted{false} is that in his crucifixion Christ is taking on humanity's sinful nature.
This villancico represents Christ taking upon himself that out-of-tune music in order to create harmony between humanity and God.
Cáseda's villancico allows listeners to contemplate that Music, the \quoted{mysterious excellence} or \quoted{virtuosity} of Christ the divine musician, that \quoted{elevates to the heavens the one who reaches it} in mystical union with Christ.%
	%
	\footnote{%
	The term \quoted{primor} is also used in musical treatises such as that of Bermudo, as in the \quoted{primores} or basic principles (or perhaps fundamental skills) of counterpoint.
	}
	%
In musical terms, then, the believer must tune himself to Christ, since Christ already harmonized himself with humanity.

%********************
\subsection{False Music}

This theology of music provides Cáseda with a license to represent musical \quoted{falsehood} to the extreme.
Cáseda's exercises in depicting discord go beyond the mild dissonance used by Cererols.
In his opening (\cref{ex:CasedaJ-Que_musica_divina-opening}), Cáseda writes direct octaves between the voice and accompaniment (in the leap up to B\fl, \measures{2}), and emphasizes them by cutting out all the other voices.
In the next two compases, Cáseda sets the word \foreign{acorde} (tuneful) to bald parallel fifths between outer voices.
Cáseda suspends the Alto's B\fl, so that these fifths move into a dissonant seventh sonority.

These contrapuntal solecisms are what Bermudo, in his treatise on instruments cited earlier about the vihuela, called musical \quoted{falsehood}.
He gives specific examples of parallel fifths and octaves, comparing them to \quoted{barbarism in grammar}.
In condemning this error, which he says is common for beginners and instrumentalists, Bermudo uses some of the same key terms as Cáseda's villancico:

\begin{quoting}
I want to say that there are those taken for musicians who have learned without a teacher and with much labor, and they are faults, and they know few principles [primores]. 
This pestilence is especially great for keyboard players. 
This is what that outstanding musician of blessed memory, Cristóbal de Morales, told me once, that if what many organists played would be written out we would find great faults. 
And he had good reason to say it: because they can play two octaves and two fifths and not perceive it [because of the organ's timbre]: while singing it they would recognize the falsehood [falsedad].%
        \autocite[f.~128v]{Bermudo:Declaracion}
\end{quoting}

Thus the kind of counterpoint Cáseda has written in his opening passage is musical \quoted{falsedad}, the same term used in Cáseda's copla 5 to describe the Music of Christ's Passion.
To depict this idea, Cáseda has his musicians create \quoted{false} music through \quoted{dangerous} and even blatantly erroneous counterpoint.
Cáseda uses rhythmic \quoted{falsehood} as well: in \measures{18} Cáseda switches from C meter to CZ, but never gives the listeners a chance to hear a \quoted{normal} CZ  rhythmic pattern (such as the flowing three-minim groups after \measures{30}). 
Instead he uses sesquialtera to alter the rhythm, and since no normal CZ  pattern has been established, the effect is not of hemiola but simply of a slow triple meter.

But it is in Cáseda's counterpoint that the idea of \quoted{falsehood} is performed most strikingly. 
In the section beginning in \measures{30}, Cáseda tries out \foreign{cláusula varias} (various cadences), creating an effect as though all the voices were continually trying to cadence (\cref{ex:CasedaJ-Que_musica_divina-clausulas}).
Each of the voices sings a typical cadential pattern, but at different times and not quite aligning relative to the others.
The chromaticism becomes more acute as the passage continues, culminating in a bizarre collision in \measures{36--37}.
The bottom four voices here on their own in \measures{36} would appear to be cadencing on F, with the C in the bass to move up by fourth and the cadential E in the Tiple II to resolve up to F.
But the top voice seems determined to cadence on G, so that in the fifth minim of \measures{36} the top voice is an augmented fourth above the bass (F\sh{} against C), while the Alto's E is made to seem dissonant even when by rights it should not be.
In \measures{37} the voices manages to cadence on D, with the top voice settling back down to F\sh, though the Tenor has re-entered just at this point to sing an unprepared dissonant B\fl{} over the bass (making an augmented fifth against the Tiple I's F\sh).

% %*******************
% \begin{example}
% 	\includeWideFigure{scores-examples/CasedaJ-Que_musica_divina-clausulas-1}
% 	\caption{Cáseda, \wtitle{Qué música divina}, \measures{30--50}: Conflicting \soCalled{cadences} and false \term{ficta}}
% 	\label{ex:CasedaJ-Que_musica_divina-clausulas}
% \end{example}
% 
% \begin{example}
% 	\includeWideFigure{scores-examples/CasedaJ-Que_musica_divina-clausulas-2}
% 	\contcaption{Continued}
% \end{example}
% 
% \begin{example}
% 	\includeWideFigure{scores-examples/CasedaJ-Que_musica_divina-clausulas-3}
% 	\contcaption{Continued}
% \end{example}
% %*******************

The final section of the estribillo (\cref{ex:CasedaJ-Que_musica_divina-clausulas}) continues in this direction, as Cáseda's music evokes the poetic idea of \quoted{elevating the senses} and \quoted{dismaying the [bodily] powers}.
He begins with a wedge pattern between the Tenor and the accompaniment (vihuela?), again juxtaposing B\fl{} (in the bass) with F\sh{} (in the tenor).
Cáseda brings in the next two voices, each one singing one of the two patterns already introduced: the Alto has the ascending line, and the Tiple 2 has the descending one.
But their entrances are reversed from that of the bass and Tenor, so a reverse wedge is created.
As this is happening, in \measures{41}, the Tiple I enters from out of nowhere with a high A, skipping down to what is apparently an F\na{} (based on the F\sh{} specified at the beginning of the next compás).
The A creates a \musfig{\na6}{\fl3} sonority: on its own it is an imperfect consonance with the bass, but against the E\fl{} in the Tiple II it certainly has the effect of elevating and dismaying, amplified by the direct fifths it then forms (again) with the bass as it skips down to F.
As though to ensure that listeners did not think this a mistake, Cáseda repeats the whole passage again in \measures{46} (though with the voices rearranged).
The estribillo ends with an alternating pattern of seventh chords over D (dominant sevenths) and minor triads on G, like the strumming of the two most common chords on a vihuela or guitar, known to us as I and V\textsuperscript{7}.

These readings of the counterpoint have been dependent on guesses about how the singers would have inflected the written pitches using \term{musica ficta}.
The practice of improvised accidentals appears to have remained alive in the Spanish realms longer than in other lands, and in New Spain in particular. 
This manuscript of Cáseda's manuscript requires many added accidentals, as those in small print above the music in the transcription demonstrate.
But in a passage like the strange collision of cadences in \measures{36--37}, the rules become confusing.
The Tiple II might begin the last phrase (starting in \measures{34} on F), singing the E as part of a cadential formula on F and therefore natural; but when the cadence comes on D, an E\fl{} might be preferable. 
Either option clashes with the notated F\sh{} in the Tiple I.

In \measures{45}, on \quoted{potencias desmaya}, the ficta situation becomes actually impossible.
To maintain the fugal motive, the Tiple II would have to sing three minim B flats and then a semibreve B\na, leading up to C.
But in the same place as the Tiple II B flats, the Tenor has a notated sharp sign on the B (the only way of indicating B\na).
The accidental in the Tenor part is clearly a sharp, written in the same hand  and with the same ink as the rest of the music.
This creates a cross relation---in Spansih, a \quoted{falsa relación}---between the B flat and B natural.

Any educated musicians confronted with this score would attempt to \quoted{fix} these problems (probably by singing all B naturals in the Tiple II).
But the passage is impossible to fix.
One feels that any solution chosen is the wrong one.
And there is now the added difficulty for the musicians of how to perform music that is supposed to sound out of tune.
How to apply ficta, how to tune intervals, when the composer is forcing the musicians to break the traditional rules?

The music is false: it cannot be emended with the further falsehood of musica ficta or anything else.
That this should happen most blatantly at the opening of the piece on the very word \quoted{acorde} is certainly meaningful.
The fifths do recall the tuning of the vihuela's strings (\quoted{cuerdas} being etymologically related to \quoted{acorde}), and of course fifths are perfect consonances.
But sung in sequence they remained in Cáseda's day the very paradigm of bad musicianship and untuneful composition.

Pedro Cerone specifically warned composers not to write passages that would tempt singers to add incorrect accidentals and \quoted{falsify} the music.
Cerone uses the terms \quoted{falsa} and \quoted{falsificar} at different times to mean either \term{musica ficta} or \quoted{wrong} notes (as in \quoted{false relations}).
In certain situations (of which he gives a notated example, shown in \cref{fig:Cerone-false-cadences}), \quoted{the singer can easily add a sharp to the fifth, thinking it to be a cadence [cláusula]: for he will see that the notes are moving in the manner of a cadence, saying \emph{Sol fa sol}, \emph{Re ut re}, and so on, and raising the note he will make it become false [falsa], and very dissonant to the ear}.%
	\autocite[629]{Cerone:Melopeo}

% %********************
% \begin{figure}
% 	\includeNarrowFigure{figures/Cerone-629-false-cadences}
% 	\caption{Melodic patterns \quoted{in danger of being falsified} (sung with improper ficta) in Cerone, \wtitle{El melopeo y maestro}, 629}
% 	\label{fig:Cerone-false-cadences}
% \end{figure}
% %********************

This is precisely the kind of passage Cáseda has written in his setting of \quoted{cláusulas varias}: the parts all have typical cadential patterns like the ones Cerone describes, and the voices would be tempted to raise certain pitches at the wrong times. 
As with the parallel fifths that Bermudo warned against, Cáseda appears to be breaking the rules deliberately and with symbolic intent.

Through his musical falsehoods, Cáseda has pushed the Neoplatonic theology of music to the point where earthly music, rather than attempting to reflect heavenly perfection, even if only partially, now overtly highlights its own falsehood.
The emphasis is shifting toward using music not to reflect heaven at all, but to aim primarily at \quoted{elevating the senses} and \quoted{dismaying the powers}---that is, the goal for the hearers is shifting from contemplation to affective experience.

Cáseda seeks affective impact through a new use of dissonance, in which dissonant intervals are highlighted rather than passed over, such that the emphasis of the music moves more toward creating tension than release.
To give one example, in \measures{6--7}, Cáseda uses the same kind of \quoted{Phrygian} cadence we have seen already in Cererols, Irízar, Bruna, and Ambiela.  % \X example
All these composers use the cadence (usually with the same seventh suspension as Cáseda has in the Tiple 2) at moments depicting heavenly music or music exceeding earthly understanding, and suggesting an affect of mystical transport.
Cáseda spices this up further by having the Alto anticipate the E\fl{} triad (\measures{6}, third note).
The senses are elevated here not by the perfection of sounding number, but by the depiction and enactment of human affects, using a developing set of affective topics.

This change need not be seen as part of a \quoted{disenchantment} process (as Hollander portrays it).
Rather, in order for Cáseda's flaunting of contrapuntal rules to have meaning, the rules must still be preserved.
Breaking them for expressive purposes (whether affective expression or symbolic expression, as of the Neoplatonic imperfection of \wtitle{musica instrumentalis}) actually reflects a continued faith in the validity of those rules.
Cáseda is not disregarding the old musical-theological system, but rather insisting upon it so strongly that he passes over a reasonable limit and seems to contradict himself.

%}}}1

%{{{1 conclusions
\section{Conclusions}

As the tradition of metamusical villancicos developed, there was an increased demand for composers both to imitate the conventions established by predecessors and to differentiate their own works in some way. 
At the same time the concept of imitation itself, as a musical-rhetorical practice within a Neoplatonic framework, was changing.
These chapters have argued that villancicos that featured \soCalled{singing about singing} began to change in function from emphasizing human music's similarity to heavenly music, toward emphasizing its difference; and toward a greater interest in expresssing and inciting human affects rather than reflecting divine order.
The three villancicos studied in this chapter demonstrate the first kind of imitation---that of influence and homage---because they manifest a certain degree of strain as each successive composer pushes the tradition of musical self-representation further towards a limit of intelligibility.

They also demonstrate the second type of imitation because their devotional topics suggest a shift in how villancicos in this subgenre functioned in religious life.
The examples by Padilla, Cererols, and Irízar provided listeners an opportunity to listen for heavenly music in the music of earth. 
They focused on music itself as a symbol of theological truths.
Carrión's \wtitle{Qué destemplada armonía}, by contrast, was less focused on abstract levels of music like the music of the spheres or the angels, and instead used human music-making as an analogy for the dynamics of repentance and faith.
The pieces in this chapter go further in this direction, partly because of their differing ritual functions.
Whereas the previous case studies were primarily Christmas pieces, the Bruna and Cáseda villancicos are for Eucharistic devotion, and the Ambiela is for Marian sanctoral devotion.
Contemplating the heavenly spheres and divine harmonies is less important in these pieces than the actual human act of worship through music.
%}}}1

