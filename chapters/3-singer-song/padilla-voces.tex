% 3-singer-song/padilla-voces.tex

\section{\quoted{Voices of the Chapel Choir} and the \quoted{Unspeaking Word}
(Gutiérrez de Padilla)}

In the setting of \VCtitle{Voces, las de la capilla} by Juan Gutiérrez de
Padilla, the first of two choruses in the polychoral ensemble of Puebla
Cathedral begins with this admonition: 
\quoted{Voices of the chapel choir, keep count with what is sung, for the King
is a musician, and takes note of even the most venial dissonances, after the
manner of David the monarch, just as in the manner of Jesus the infant prince}.
As the voices declaim these words together like a solemn choral recitative, 
breaking the chordal texture only for a mild dissonant suspension on the words
\mentioned{light dissonances}, the other chorus is literally counting
twenty-seven measures of rests until their entrance.

Thus begins a tour-de-force of music about music, in which the composer and his
ensemble take a verbal discourse about music and turn it into a musical
discourse about music.
Moreover, even as this discourse is referring through music to the music that is
currently being performed and heard, this discourse is not only about sounding
music or \term{musica instrumentalis}.

When the first chorus refers to \quoted{what is sung}, they are not merely
referring to this literal level of human performance.
Instead, they point to \foreign{Jesús infante}---the child who is both
the \term{infante} or heir of the musician-king David and who is also God made
flesh as an infant, a child too young to be able to speak.

\subsection{Christ as Music in the Poem}

Any consideration of this complex piece should begin with a careful reading of
the poetic text and listening to the musical setting (even better, singing or playing it).%
    \autocite[\XXX]{Cashner:WLSCM-VCs}
It will probably be readily apparent that this is a different type of
villancico entirely from the comic villancicos more commonly heard in recent
performances and recordings.
The poem, whose author is unknown, is so elaborately contrived that it may seem
completely unintelligible at the first encounter. 
There is, indeed, no simple way to approach the text: anywhere one starts one
immediately becomes tangled in a thicket of ambiguous references and cryptic
phrases, and it is not possible to understand the text fully without reading it
as the distillation of tropes and doctrines developed through centuries of
theological literature and liturgy.
The piece demands a high level of intellectual engagement to tease out the
intricate conceit, and may thus be compared with what Bernardo Illari describes
as \quoted{enigma} villancicos.%
    \Autocite[vol.\ 2, 304--308]{Illari:Polychoral}

Part of its difficulty comes from the widespread influence of the poetry of Luis
de Góngora, who played a similar role for Spanish literature on both sides of
the Atlantic to that of Giambattista Marino in Italian letters, as the
originator of a new aesthetic (referred to as \term{barroco} by Spanish literary
scholars) that emphasized learned artifice and highly wraught dramatic effects.
% as opposed to ... XXX
Gongoresque poets like the author of \worktitle{Voces} reveled in arcane plays
on words, contorted Latinate syntax, and multiple levels of meanings.%
    \Autocites
    [222--237]{Gaylord:Poetry}
    {Tenorio:Gongorismo}
    [vol.\ 1, 1014--1061]{Valbuena:Literatura}

To begin simply, though, this poem represents the tradition of Spanish
\term{conceptismo}, in which the poet creates a sustained analogy between at
least two different things such that the understanding of each one informs the
other.
Here the two elements are music---specifically choral singing---and the
Incarnate Christ at his birth.

This may be easiest to see in the coplas.
One can read these two strophes solely on the one plane referring to Christ's
Incarnation and Passion, however elliptically.
My published translation attempts to convey some of the ambiguity of the
original; emphasizing one side of the conceit or other would lead to different
translations, of course.
In the first copla, the Christ-child gave a \quoted{wandering song} to the
Man---recalling to listener's minds that the first man Adam was rejected from
paradise for his sin and sent to wander across the earth; and his descendents
were made pilgrims in search for the promised land.
Christ \quoted{went up so high} in \quoted{sustained tones of
weeping}---suffering on the cross for human redemption.
Christ \quoted{arose on the eighth} day---a traditional way of referring to the
Sunday on the Resurrection---\quoted{in the triune}.%
    \begin{Footnote}
        See the entries for the numbers eight and three in
        \autocite{Bongo:NumerorumMysteria} and
        \autocite{Ricciardo:CommentariaSymbolica}.
    \end{Footnote}

Reading this copla according to the other side of the conceit, the Christ-child
gave the \term{tonus peregrinus} chant formula to the Man, and, as a virtuoso
singer, \quoted{he went up so high} that \quoted{in a cry of sharps}, he
\quoted{went up the octave in a trill}.
The poet has selected musical terms with double meanings that allow listeners
with musical knowledge to think about theological concepts in a new way, and
vice versa.
For example, the sharp sign (\foreign{sustenido}) is used (as J.\ S.\ Bach would
later use it) as an icon of the cross; and musically also seems to suggest the
\quoted{strange} (another meaning of \foreign{peregrino}) quality of this song
of suffering.


% copla 2 XXX

In many villancico poems, the relationship of estribillo and coplas is similar
to that between the image and accompanying texts in contemporary emblem books.%
    \Autocite{Covarrubias:Emblemas}
In the emblem book of Covarrubias, the emblem is often striking but cryptic,
juxtaposing (for example) two well-known symbols in an unusual way, often with a
Latin motto.
On the facing page is a poem that expands on the ideas in the image, and on the
reverse of that page is a prose explanation of both the poem and the image.
In this villancico text, then, the estribillo's function is similar to the
purpose of the emblem and poem as presenting attention-getting but enigmatic
symbols, while the coplas share the more explanatory function of the emblem
book's prose. 
Just as one would read the explanation and then turn back to consider the emblem
again, now with a heightened sense of meaning, so too in the villancico, when
the estribillo is repeated after the coplas, listeners have the opportunity to
reconsider what they have heard in light of the coplas. 

% form, meter

% musical-theological conceits at lexical level and theological context (e.g.,
% patristic, liturgical allusions)





\subsection{Music about Music in the Voices of Puebla's Chapel Choir}

\subsection{Theology of Christ as \term{Verbum infans}}

\subsection{Establishing a Pedigree in a Lineage of Metamusical Composition}

