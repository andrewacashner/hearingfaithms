% 3-singer-song/padilla-voces.tex

\section{\quoted{Voices of the Chapel Choir} and the \quoted{Unspeaking Word}
(Gutiérrez de Padilla)}

In the setting of \VCtitle{Voces, las de la capilla} by Juan Gutiérrez de
Padilla, the first of two choruses in the polychoral ensemble of Puebla
Cathedral begins with this admonition: 
\quoted{Voices of the chapel choir, keep count with what is sung, for the King
is a musician, and takes note of even the most venial dissonances, after the
manner of David the monarch, just as in the manner of Jesus the infant prince}.
As the voices declaim these words together like a solemn choral recitative, 
breaking the chordal texture only for a mild dissonant suspension on the words
\mentioned{light dissonances}, the other chorus is literally counting
twenty-seven measures of rests until their entrance.

Thus begins a tour-de-force of music about music, in which the composer and his
ensemble take a verbal discourse about music and turn it into a musical
discourse about music.
Moreover, even as this discourse is referring through music to the music that is
currently being performed and heard, this discourse is not only about sounding
music or \term{musica instrumentalis}.

When the first chorus refers to \quoted{what is sung}, they are not merely
referring to this literal level of human performance.
Instead, they point to \foreign{Jesús infante}---the child who is both
the \term{infante} or heir of the musician-king David and who is also God made
flesh as an infant, a child too young to be able to speak.

\subsection{Christ as Music in the Poem}

Any consideration of this complex piece should begin with a careful reading of
the poetic text and listening to the musical setting (even better, singing or playing it).%
    \autocite[\XXX]{Cashner:WLSCM-VCs}
It will probably be readily apparent that this is a different type of
villancico entirely from the comic villancicos more commonly heard in recent
performances and recordings.
The poem, whose author is unknown, is so elaborately contrived that it may seem
completely unintelligible at the first encounter. 
There is, indeed, no simple way to approach the text: anywhere one starts one
immediately becomes tangled in a thicket of ambiguous references and cryptic
phrases, and it is not possible to understand the text fully without reading it
as the distillation of tropes and doctrines developed through centuries of
theological literature and liturgy.
The piece demands a high level of intellectual engagement to tease out the
intricate conceit, and may thus be compared with what Bernardo Illari describes
as \quoted{enigma} villancicos.%
    \Autocite[vol.\ 2, 304--308]{Illari:Polychoral}

Part of its difficulty comes from the widespread influence of the poetry of Luis
de Góngora, who played a similar role for Spanish literature on both sides of
the Atlantic to that of Giambattista Marino in Italian letters, as the
originator of a new aesthetic (referred to as \term{barroco} by Spanish literary
scholars) that emphasized learned artifice and highly wraught dramatic effects.
% as opposed to ... XXX
Gongoresque poets like the author of \worktitle{Voces} reveled in arcane plays
on words, contorted Latinate syntax, and multiple levels of meanings.%
    \Autocites
    [222--237]{Gaylord:Poetry}
    {Tenorio:Gongorismo}
    [vol.\ 1, 1014--1061]{Valbuena:Literatura}

To begin simply, though, this poem represents the tradition of Spanish
\term{conceptismo}, in which the poet creates a sustained analogy between at
least two different things such that the understanding of each one informs the
other.
Here the two elements are music---specifically choral singing---and the
Incarnate Christ at his birth.

This may be easiest to see in the coplas.
One can read these two strophes solely on the one plane referring to Christ's
Incarnation and Passion, however elliptically.
My published translation attempts to convey some of the ambiguity of the
original; emphasizing one side of the conceit or other would lead to different
translations, of course.
In the first copla, the Christ-child gave a \quoted{wandering song} to the
Man---referring to the first man, Adam, being cast out of Paradise.
Christ \quoted{went up so high} in \quoted{sustained tones of
weeping}---suffering on the cross for human redemption.
Christ \quoted{arose on the eighth} day---a traditional way of referring to the
Sunday on the Resurrection---\quoted{in the triune}.%
    \begin{Footnote}
        See the entries for the numbers eight and three in
        \autocite{Bongo:NumerorumMysteria} and
        \autocite{Ricciardo:CommentariaSymbolica}.
    \end{Footnote}

Reading this copla according to the other side of the conceit, the Christ-child
gave the \term{tonus peregrinus} chant formula to the Man, and, as a virtuoso
singer, \quoted{he went up so high} that \quoted{in a cry of sharps}, he
\quoted{went up the octave in a trill}.
The poet has selected musical terms with double meanings that allow listeners
with musical knowledge to think about theological concepts in a new way, and
vice versa.
For example, the sharp sign (\foreign{sustenido}) is used (as J.\ S.\ Bach would
later use it) as an icon of the cross; and musically also seems to suggest the
\quoted{strange} (another meaning of \foreign{peregrino}) quality of this song
of suffering.

The two words \term{peregrino tono} would have called up for educated
listeners a long tradition of using the chant \term{tonus peregrinus} to
symbolize the expectant wandering of sinful humanity waiting for the coming of
Christ, as Craig Wright has traced in medieval sources.
In an allegorical commentary on the liturgy, which was disseminated in
manuscript and print versions across the Hispanic world---including copies in
Puebla---Guillelmus Durandus called the whole time from the fall of Adam to the
birth of Christ, including the Israelites' wandering in the wilderness and
Babylonian captivity, the \foreign{tempus peregrinationis}.%
    \Autocite[234]{Wright:Maze}

An interpreter contemporary to Padilla, Cornelius à Lapide, extends the
\quoted{pilgrim} concept to the New Adam, Christ.
In a commentary also available in Puebla and throughout the Spanish Empire that
Christ, as the New Adam, was born like a \quoted{pilgrim} on a journey in a
borrowed stable.
Explaining the Greek word \foreign{esk{\-e}nosen} in John 1:14,
Lapide says \quoted{he set up a tent or tabernacle in us, that is, among us, for
a short time, like a guest or pilgrim \add{peregrinus} in a foreign land}, and
this should \quoted{teach us to be pilgrims \add{peregrinos} on earth, though
actually citizens of heaven}.%
    \Autocite
    [884: \quoted{Hoc est tentorium vel tabernaculum fixit in nobis, id est,
    inter nos, ad modicum tempus, quasi hospes et peregrinus in terra aliena:
    erat enim ipse civis, incola et dominus cœli ac paradisi \Dots}; 
    669, on \scripture{Luke 2:5}: \quoted{ut doceret nos in terra esse
    peregrinos, cives vero cœli, ut ab hoc exilio magnis virtutum passibus
    tendamis in cœlum, ceu patriam et civitatem nostram}]
    {Lapide:Gospels19C}
A contemporary musician, the composer and theorist Andrés Lorente, took up  
the \quoted{pilgrim song} trope, drawing on Vulgate Psalm 118:54, when he
exhorted the aspiring musician to match the music of his compositions with
\quoted{the spiritual Music of his person, cleansing his conscience, and
rejoicing his soul with Divine Music, so that he may say with David,
\gloss{Cantabiles mihi erant iustificationes tuæ in loco peregrinationis
meæ}{Your right precepts have served as songs for me in the place of my
wanderings}}.%
    \Autocite
    [609: \quoted{la Musica espiritual de su persona, limpiando su conciencia, y
    alegrando su Anima con Musica diuina, para que pueda dezir con Dauid,
    \foreign{Cantabiles mihi} \Dots}]
    {Lorente:Porque}
Within this tradition, then, the villancico uses a chant tone to present Christ
himself as the song given to sinful Man in his pilgrimage.
It is no surprise that in the musical setting, Padilla quotes the chant formula
literally, so that the symbol is present to the ear both in word and tone.

This rich interplay between the musical and theological sides of the
conceit extends throughout the whole villancico, opening up so many possible
points of reference and interpretations that it will not be possible here to go
through them all in detail, as I have done elsewhere.%
    \Autocite[133--203]{Cashner:PhD}
Instead it may be most helpful to give a brief summary of the poem's
representation of Christ as both singer and song.

The central conceit of the poem is to link the \quoted{voices of the chapel
choir}---the actual sounding music being heard in the performance of the
villancico---with a higher, theological Music with a capital M, namely, the
Christ-child himself.
The poem evokes the musical voices of human singers, angelic choirs, ancient
prophets, and even the crying voice of the newborn Christ. 
The second copla encapsulates the whole conceit: Christ is a divine musical
\quoted{composition}, in which the divine chapelmaster \quoted{proves} 
his mastery at creating \quoted{consonances of a Man and God}.%
\begin{Footnote}
    This is a common practice in villancicos: a puzzling idea presented at the
    beginning of the piece is not fully explained until the end of the coplas,
    so that when the estribillo is then repeated, the listener can hear it with
    new understanding.
    % formal relationships here, diss p. 145 XXX
    In many villancico poems, the relationship of estribillo and coplas is
    similar to that between the image and accompanying texts in contemporary
    emblem books.%
        \Autocite{Covarrubias:Emblemas}
    In the emblem book of Covarrubias, the emblem is often striking but cryptic,
    juxtaposing (for example) two well-known symbols in an unusual way, often
    with a Latin motto.
    On the facing page is a poem that expands on the ideas in the image, and on
    the reverse of that page is a prose explanation of both the poem and the
    image.
    In this villancico text, then, the estribillo's function is similar to the
    purpose of the emblem and poem as presenting attention-getting but enigmatic
    symbols, while the coplas share the more explanatory function of the emblem
    book's prose. 
    Just as one would read the explanation and then turn back to consider the
    emblem again, now with a heightened sense of meaning, so too in the
    villancico, when the estribillo is repeated after the coplas, listeners have
    the opportunity to reconsider what they have heard in light of the coplas. 
    \end{Footnote}
The \quoted{chapel}, it would seem, is performing before the king, like the
Spanish \term{Capilla Real}; and this King, that is, God, \quoted{is a
musician}.
His only-begotten son---his royal \foreign{infante}---is this divine
chapelmaster's \term{oposición} or proof piece, in which he demonstrates his
mastery by creating concord between opposed elements.
Christ brings together infinite and finite (\quoted{maxima and breve}), and
creates a consonance to restore the discordant relationship between sinful Man
and the holy God by reconciling both in his own body.

% form, meter

% musical-theological conceits at lexical level and theological context (e.g.,
% patristic, liturgical allusions)





\subsection{Music about Music in the Voices of Puebla's Chapel Choir}

\subsection{Theology of Christ as \term{Verbum infans}}

\subsection{Establishing a Pedigree in a Lineage of Metamusical Composition}

