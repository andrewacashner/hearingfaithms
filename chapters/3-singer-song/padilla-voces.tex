% 3-singer-song/padilla-voces.tex

\section{\quoted{Voices of the Chapel Choir} and the \quoted{Unspeaking Word}
(Gutiérrez de Padilla)}

In the setting of \VCtitle{Voces, las de la capilla} by Juan Gutiérrez de
Padilla, the first of two choruses in the polychoral ensemble of Puebla
Cathedral begins with this admonition: 
\quoted{Voices of the chapel choir, keep count with what is sung, for the King
is a musician, and takes note of even the most venial dissonances, after the
manner of David the monarch, just as in the manner of Jesus the infant prince}.
As the voices declaim these words together like a solemn choral recitative, 
breaking the chordal texture only for a mild dissonant suspension on the words
\mentioned{light dissonances}, the other chorus is literally counting
twenty-seven measures of rests until their entrance.

Thus begins a tour-de-force of music about music, in which the composer and his
ensemble take a verbal discourse about music and turn it into a musical
discourse about music.
Moreover, even as this discourse is referring through music to the music that is
currently being performed and heard, this discourse is not only about sounding
music or \term{musica instrumentalis}.

When the first chorus refers to \quoted{what is sung}, they are not merely
referring to this literal level of human performance.
Instead, they point to \foreign{Jesús infante}---the child who is both
the \term{infante} or heir of the musician-king David and who is also God made
flesh as an infant, a child too young to be able to speak.

\subsection{Christ as Music in the Poem}

Any consideration of this complex piece should begin with a careful reading of
the poetic text and listening to the musical setting (even better, singing or
playing it).% 
    \autocite[\XXX]{Cashner:WLSCM-VCs}
It will probably be readily apparent that this is a different type of
villancico entirely from the comic villancicos more commonly heard in recent
performances and recordings.
The poem, whose author is unknown, is so elaborately contrived that it may seem
completely unintelligible at the first encounter. 
There is, indeed, no simple way to approach the text: anywhere one starts one
immediately becomes tangled in a thicket of ambiguous references and cryptic
phrases, and it is not possible to understand the text fully without reading it
as the distillation of tropes and doctrines developed through centuries of
theological literature and liturgy.
The piece demands a high level of intellectual engagement to tease out the
intricate conceit, and may thus be compared with what Bernardo Illari describes
as \quoted{enigma} villancicos.%
    \Autocite[vol.\ 2, 304--308]{Illari:Polychoral}

Part of its difficulty comes from the widespread influence of the poetry of Luis
de Góngora, who played a similar role for Spanish literature on both sides of
the Atlantic to that of Giambattista Marino in Italian letters, as the
originator of a new aesthetic (referred to as \term{barroco} by Spanish literary
scholars) that emphasized learned artifice and highly wraught dramatic effects.
% as opposed to ... XXX
Gongoresque poets like the author of \worktitle{Voces} reveled in arcane plays
on words, contorted Latinate syntax, and multiple levels of meanings.%
    \Autocites
    [222--237]{Gaylord:Poetry}
    {Tenorio:Gongorismo}
    [vol.\ 1, 1014--1061]{Valbuena:Literatura}

To begin simply, though, this poem represents the tradition of Spanish
\term{conceptismo}, in which the poet creates a sustained analogy between at
least two different things such that the understanding of each one informs the
other.
Here the two elements are music---specifically choral singing---and the
Incarnate Christ at his birth.

This may be easiest to see in the coplas.
One can read these two strophes solely on the one plane referring to Christ's
Incarnation and Passion, however elliptically.
My published translation attempts to convey some of the ambiguity of the
original; emphasizing one side of the conceit or other would lead to different
translations, of course.
In the first copla, the Christ-child gave a \quoted{wandering song} to the
Man---referring to the first man, Adam, being cast out of Paradise.
Christ \quoted{went up so high} in \quoted{sustained tones of
weeping}---suffering on the cross for human redemption.
Christ \quoted{arose on the eighth} day---a traditional way of referring to the
Sunday on the Resurrection---\quoted{in the triune}.%
    \begin{Footnote}
        See the entries for the numbers eight and three in
        \autocite{Bongo:NumerorumMysteria} and
        \autocite{Ricciardo:CommentariaSymbolica}.
    \end{Footnote}

Reading this copla according to the other side of the conceit, the Christ-child
gave the \term{tonus peregrinus} chant formula to the Man, and, as a virtuoso
singer, \quoted{he went up so high} that \quoted{in a cry of sharps}, he
\quoted{went up the octave in a trill}.
The poet has selected musical terms with double meanings that allow listeners
with musical knowledge to think about theological concepts in a new way, and
vice versa.
For example, the sharp sign (\foreign{sustenido}) is used (as J.\ S.\ Bach would
later use it) as an icon of the cross; and musically also seems to suggest the
\quoted{strange} (another meaning of \foreign{peregrino}) quality of this song
of suffering.

The two words \term{peregrino tono} would have called up for educated
listeners a long tradition of using the chant \term{tonus peregrinus} to
symbolize the expectant wandering of sinful humanity waiting for the coming of
Christ, as Craig Wright has traced in medieval sources.
In an allegorical commentary on the liturgy, which was disseminated in
manuscript and print versions across the Hispanic world---including copies in
Puebla---Guillelmus Durandus called the whole time from the fall of Adam to the
birth of Christ, including the Israelites' wandering in the wilderness and
Babylonian captivity, the \foreign{tempus peregrinationis}.%
    \Autocite[234]{Wright:Maze}

An interpreter contemporary to Padilla, Cornelius à Lapide, extends the
\quoted{pilgrim} concept to the New Adam, Christ.
In a commentary also available in Puebla and throughout the Spanish Empire that
Christ, as the New Adam, was born like a \quoted{pilgrim} on a journey in a
borrowed stable.
Explaining the Greek word \foreign{esk{\-e}nosen} in John 1:14,
Lapide says \quoted{he set up a tent or tabernacle in us, that is, among us, for
a short time, like a guest or pilgrim \add{peregrinus} in a foreign land}, and
this should \quoted{teach us to be pilgrims \add{peregrinos} on earth, though
actually citizens of heaven}.%
    \Autocite
    [884: \quoted{Hoc est tentorium vel tabernaculum fixit in nobis, id est,
    inter nos, ad modicum tempus, quasi hospes et peregrinus in terra aliena:
    erat enim ipse civis, incola et dominus cœli ac paradisi \Dots}; 
    669, on \scripture{Luke 2:5}: \quoted{ut doceret nos in terra esse
    peregrinos, cives vero cœli, ut ab hoc exilio magnis virtutum passibus
    tendamis in cœlum, ceu patriam et civitatem nostram}]
    {Lapide:Gospels19C}
A contemporary musician, the composer and theorist Andrés Lorente, took up  
the \quoted{pilgrim song} trope, drawing on Vulgate Psalm 118:54, when he
exhorted the aspiring musician to match the music of his compositions with
\quoted{the spiritual Music of his person, cleansing his conscience, and
rejoicing his soul with Divine Music, so that he may say with David,
\gloss{Cantabiles mihi erant iustificationes tuæ in loco peregrinationis
meæ}{Your right precepts have served as songs for me in the place of my
wanderings}}.%
    \Autocite
    [609: \quoted{la Musica espiritual de su persona, limpiando su conciencia, y
    alegrando su Anima con Musica diuina, para que pueda dezir con Dauid,
    \foreign{Cantabiles mihi} \Dots}]
    {Lorente:Porque}
Within this tradition, then, the villancico uses a chant tone to present Christ
himself as the song given to sinful Man in his pilgrimage.
It is no surprise that in the musical setting, Padilla quotes the chant formula
literally, so that the symbol is present to the ear both in word and tone.

This rich interplay between the musical and theological sides of the
conceit extends throughout the whole villancico, opening up so many possible
points of reference and interpretations that it will not be possible here to go
through them all in detail, as I have done elsewhere.%
    \Autocite[133--203]{Cashner:PhD}
Instead it may be most helpful to give a brief summary of the poem's
representation of Christ as both singer and song.

The central conceit of the poem is to link the \quoted{voices of the chapel
choir}---the actual sounding music being heard in the performance of the
villancico---with a higher, theological Music with a capital M, namely, the
Christ-child himself.
The poem evokes the musical voices of human singers, angelic choirs, ancient
prophets, and even the crying voice of the newborn Christ. 
The second copla encapsulates the whole conceit: Christ is a divine musical
\quoted{composition}, in which the divine chapelmaster \quoted{proves} 
his mastery at creating \quoted{consonances of a Man and God}.%
\begin{Footnote}
    This is a common practice in villancicos: a puzzling idea presented at the
    beginning of the piece is not fully explained until the end of the coplas,
    so that when the estribillo is then repeated, the listener can hear it with
    new understanding.
    % formal relationships here, diss p. 145 XXX
    In many villancico poems, the relationship of estribillo and coplas is
    similar to that between the image and accompanying texts in contemporary
    emblem books.%
        \Autocite{Covarrubias:Emblemas}
    In the emblem book of Covarrubias, the emblem is often striking but cryptic,
    juxtaposing (for example) two well-known symbols in an unusual way, often
    with a Latin motto.
    On the facing page is a poem that expands on the ideas in the image, and on
    the reverse of that page is a prose explanation of both the poem and the
    image.
    In this villancico text, then, the estribillo's function is similar to the
    purpose of the emblem and poem as presenting attention-getting but enigmatic
    symbols, while the coplas share the more explanatory function of the emblem
    book's prose. 
    Just as one would read the explanation and then turn back to consider the
    emblem again, now with a heightened sense of meaning, so too in the
    villancico, when the estribillo is repeated after the coplas, listeners have
    the opportunity to reconsider what they have heard in light of the coplas. 
    \end{Footnote}
The \quoted{chapel}, it would seem, is performing before the king, like the
Spanish \term{Capilla Real}; and this King, that is, God, \quoted{is a
musician}.
His only-begotten son---his royal \foreign{infante}---is this divine
chapelmaster's \term{oposición} or proof piece, in which he demonstrates his
mastery by creating concord between opposed elements.
Christ brings together infinite and finite (\quoted{maxima and breve}), and
creates a consonance to restore the discordant relationship between sinful Man
and the holy God by reconciling both in his own body.

On the theological side of the conceit, who exactly is this king who is
listening so carefully to the chapel choir's voices?
The poet explicitly connects the king to \quoted{David the monarch}, the paragon
of Biblical musicians as both the traditional author of the psalms and as the
founder of the first musical ensemble for worship in the ancient Hebrew
temple.\citXXX
Calling Jesus \foreign{infante} identifies this child as David's
heir---that is, as the Messiah; and the phrase \gloss{a lo de}{in the manner or
style of} suggests that this child will be no less exacting a musical taskmaster
than his ancestor.
There is also a deeper meaning to the term \foreign{infante} here, to which we will
return.

As the poem continues, it presents Christ as the theological fulfillment of the
prophecies made to and through David, especially the psalms; in musical terms,
Christ's life \quoted{is putting notes to his lyrics} (\textlinenums{8--9}), and thus
his life is recounted cryptically with the technical vocabulary for describing a
musical composition or performance.
Theologically, God had promised to David an heir to sit on his throne forever
and deliver his people,
and through Isaiah the prophet he renewed this promise by saying that a child
would be born upon whose shoulder would rest divine, eternal
authority.\citXXX[Is. 22:22 but note context]
As Biblical interpreters of the time all agreed, the complete fulfillment of
these prophecies, the culmination of all God's \quoted{centuries of heroic
exploits} (\textlinenum{8}) came not at Christ's birth, but at his death and
resurrection, traditionally thirty-three years (and three months and three days)
later.\citXXX
In musical terms, the words of David and the prophets are just the lyrics;
Christ's life is the song.
The key of authority---\foreign{clave}---is the same word for clef; and it
awaits \quoted{the thirty-three}, suggesting some kind of musical measure.

The second section of the introduction (\textlinenums{11--16}) depicts the moment of
Christ's birth as a musical performance. 
This happens \quoted{years before the sign, \quoted{dexterity in hope}}. 
Since the poet has just referred to \quoted{the thirty-three} as a metonym for
Christ's passion, the \foreign{divisa}---a sign of any kind but particularly a
heraldic emblem or motto---may refer to the cross.\citXXX
% others who say the same
The motto, \foreign{la destreza en la esperanza}, sounds like a gloss a phrase
from the Roman historian Tacitus, \foreign{spes in virtute}, which was cried in
the midst of a famous battle.\citXXX % details
The heraldic and chivalric imagery and language (\foreign{hazañas},
\foreign{destreza}) here continues the notion of Christ as a
David-like hero; but in the musical context it also established Christ as a
musical virtuoso.
The term \foreign{destreza} was commonly used in Spanish musical writing to
denote virtousity and consummate craftsmanship; it could be used both for
players and for composers.\citXXX

In the next lines we begin to hear this musician's song, and the theological and
musical come too close to cleanly separate.
The solmization syllables \term{sol} and \term{mi} here indicate musical
pitches, as well as the symbolic puns on \quoted{sun} and \quoted{me}.
The \quoted{Gloria} of the angelic choirs begins \quoted{on \term{sol}}, as many
Gregorian \term{Gloria in excelsis} chants do; it also begins \quoted{with the
sun}, a reference both to Christ's birth traditionally at midnight (\quoted{at
the half-measure of night}, \textlinenum{15}) and a symbol of his royalty, the same
symbol used by Spain's own king, Philip IV.\citXXX[also Is. 60:1, connect Jn
1:14, Tit 2:11]
The reference to \quoted{glory} and \quoted{grace} here is part of a long
theological tradition which we will explore later.

% or here

Now the poet turns even more strongly from using music as a metaphor to
describing actual music-making, as in the estribillo the text calls to mind the
whole creation of humans, animals (\foreign{hombres y brutos}, \textlinenum{20}; also
the reference to bidsong in \textlinenum{16}),
heavenly spheres (\foreign{las distancias}, a technical term in both astronomy
and music), all joining with the angels in the chorus to praise God made
Incarnate in their midst.
Like all poetry about music, this representation of music draws on the poet's
experience of actual, contemporary music---thus the spheres sing \quoted{in one
choir and the other}, like Spain's polychoral ensembles.
The scene recalls many contemporary images of the Adoration of the Sheperds, in
particular one by Francisco de Zurbarán painted in Cádiz only a few years after
Padilla left it for the New World, and an engraving in a 1617\XXX{} breviary
that both poet and composer are likely to have seen.\citXXX

The phrase \foreign{grave, suave y sonoro} seems to have been a stock
description of sacred music appropriate for liturgical worship.
The same words are used to describe the music of Christ as a musician (in fact,
as a musical instrument) in José de Cáseda's \worktitle{Qué música divina} from
a half-century later, as discussed in a later chapter. %\ref
A bequest\XXX{} to Mexico City Cathedral in \XXX[year] used nearly identical
terms to describe the Latin Responsories that should be sung.
Interestingly, in that case the term was used to specify Latin-texted music
\emph{as opposed to} villancicos.
Here, within a villancico, the term is used to refer to a higher form of
music-making, within the Neoplatonic tradition of using that which is lowly to
point toward higher truths. 
We will see that Padilla uses stylistic topics to match the reference to
different types of music within the same piece.

The numbers here (\quoted{three by three, two by two, one by one}) have a wide
range of possible symbolic meanings, while at the most literal level (the level
Padilla seizes on in his setting) it describes the number of voices in the
musical texture.
\quoted{Two by two} would seem to be a reference to the animals on Noah's Ark,
here referring not only to the animals in the stable, but also to the whole
scene as a picture of the Christian church, a symbolic connection going back to
the first century.\citXXX[1 Peter\XXX, Augustine]
\quoted{Three by three} could be the nine ranks of angelic choirs\citXXX{}, and
\quoted{one by one} could refer to humans or to Christ himself (particularly his
union of divine and human in a single body).\citXXX

Thus far the poet has taken us from listening to the chapel choir singing in the
present, to the ancient temple choir of David and the voices of prophets through
the centuries, and up to the moment when the angels led the song of Gloria at
Christ's birth.
But all these voices, the poem now says, have been \quoted{awaiting the
opportune time, the one who was before all time} (\textlinenums{24--25}).
Once again, the true music of Christmas is Christ himself, and thus the next
lines represent the voice of the baby Jesus.
The musical imagery here continues the conceit of the King as musician, a
\gloss{padre}{father, priest} like most Spanish chapelmasters including Padilla,
who either sounds the pitch \term{A (la, mi, re)} in Guidonian solmization with
his voice as a tuning note (the same one used today), or sings an intonation on
that note as a cantor would do.
Another translation of the contorted syntax here might suggest that the poetic
speaker actually \quoted{heard the voice of the Father singing}, that is,
\emph{through} the voice of the child.
At this cue, the speaker says, he heard singing, not in \foreign{puntos de canto
llano} (plainchant) but \foreign{puntos de llanto} (weeping); this connects
Christ's cries at his birth with those at his death.\citXXX
The song is \quoted{as much to be seen (or admired) as to be
heard}---because the song and the singer are one and the same.

This interpretation depends on reading \quoted{the sign of \term{A (la, mi,
re)}} as a symbol of Christ himself.
The poem says the voices were all awaiting \quoted{the one who was
before all time}, an appositive phrase referring to Christ, whom the Nicene
Creeds declares was \quoted{begotten of the father before all ages
\add{secula}}.%
    \Autocite
    [\quoted{ex patre natum ante omnia secula}]
    [42]
    {Catholic:Catechismus1614}
The reference to Christ's eternal existence together with the letter A recalls
Christ's description of himself in the Revelation to John as
\quoted{\term{alpha} and \term{omega}}---the Vulgate simply puts the Greek
letter A here.%
\begin{Footnote}
    This is an especially interesting choice since the Greek text of
    \scripture{Rev 1:8}, actually spells out \term{alpha} (but not
    \term{omega}).
\end{Footnote}
This title indicates that Christ is \quoted{the beginning and the end},
\quoted{the one who is and who was and who is to come} (\scripture{Rev 1:8,
21:6, 22:13}).%
    \begin{Footnote}
        In both Greek and Latin, the phrase in Revelation 1:8 repeats exactly
        the words used for the name of God (the Hebrew tetragrammaton, YHWH) in
        the translations of \scripture{Ex 3:14} into those languages (Greek
        \textgreek{ὁ ὤν}, Latin \foreign{qui est}).
    \end{Footnote}
The time symbolism works on a musical level as well, but the \quoted{sign of A}
also suggests the musical pitch A.

The final couplet epitomizes the theology of Incarnation through a play on
rising and falling musical lines: \quoted{Everything in Man is to ascend/ and
everything in God is to descend} (\textlinenums{32--33}).
Most discussions of the Incarnation in contemporary theological literature cited
some version of the maxim attributed to St. Ambrose\XXX, \quoted{God became Man
so that Man might become Divine}. 
In Christ, God humbled himself so that he might raise up sinful humans to share
in his glory (Philippians \XXX). 
This central theology of Christianity is represented in the poem through the
musical concept of a contrapuntal voice exchange.

The villancico began by drawing listeners' attention to the voices of Christmas,
and exhorting the singing voices of the chapel choir to take note of their own
singing while also listening for \quoted{what is sung} on a higher level.
The piece connects Christ and David as musician-kings, with Christ as the song
that puts the prophetic \quoted{lyrics} of David and other Scriptural authors to
music.
After long waiting, at the \quoted{opportune time}, Christ was born into the
world to begin a battle \quoted{in hope}, a virtuoso performance fulfilled in
his death and resurrection at \quoted{the thirty-three}, upon the \quoted{sign}
of the cross.
Christ himself is the incranate Word, and his infant cries are the true
\quoted{sign of A}, the \quoted{song} that sets the tone for all the other
voices, \quoted{in one choir and another} of the Christmas manger, and at the
Christmas liturgy in the present-day of the villancico's performance.

%******************************************************************************
\subsection{Music about Music in the Voices of Puebla's Chapel Choir}

How, then, does the composer turn the musical-theological conceits of this poem
into actual \term{musica instrumentalis}?

In any example of sung poetry, we may distinguish three aspects of text setting:
text projection, depiction, and expression.%
    \citXXX[Burkholder for the latter two]
In text projection, a composer projects the words clearly according to their
prosody and grammatical structure, and sets them to memorable melodic and
rhythmic patterns, so that listeners can hear and understand the words of the
poem.
In text depiction, a composer uses musical techniques (in this period, often
musical-rhetorical figures) to paint or represent the meaning of the text in a
symbolic way, such as in madrigalisms.
Text expression encompasses a variety of ways that a composer can communicate
the feeling, mood, and meaning of the text to listeners, including the use of
affective or other stylistic topics or anything that makes the listener feel and
experience the meaning of the text.

Padilla's setting of this poem is a tour-de-force of text depiction, but it
also demonstrates careful attention to text projection.
This approach fits with an aesthetic of Tridentine clarity in presenting the
text, as well as with traditions from Spanish popular and theatrical traditions
of singing poetry, especially traditions of adapting stock melodic formulas for
singing \term{romance} poetry.\citXXX{}

\subsubsection{Text Projection}

The formal structure of the musical setting is closely linked to that of the
poem.\citXXX[my edition]
The poem begins with an introductory section, which will be useful to label
\term{introducción} as many poetry imprints of other villancicos do.
The \term{introducción} consists of two six-line strophes, each followed by the
same \term{respuesta} or response of a four-line strophe.
The placement of rests and repeat markings in the performing parts makes clear
that the \term{respuesta} is sung after each of the six-line
strophes.\citXXX[distinguish from other edition, performance]
The whole \term{introducción} is in the poetic meter of \term{romance} in
\term{-a -a}: this means that every line is eight syllables long, and that there
is assonance in the last two vowels of the even-numbered lines.\citXXX[metrica]
Thus the poem uses \term{-a -a} assonance in \foreign{canta},
\foreign{disonancias}, \foreign{monarca}, and so on.

Another way to think of \term{romance} poetry is more similar to traditional
poetic forms in other languages, such as Old English and Germanic poetry---as
sixteen-syllable lines, with a \term{caesura} in the middle and assonance at the
end.\citXXX[crestomatia, romance publications, Old English, German exx]
In other words, the beginning of this villancico would probably have been
arranged like this (partly because of narrow columns in the typographical
layout):

\begin{blockquotation}
\begin{verse}
    Voces, las de la capilla,\\
    cuenta con lo que se canta,\\
    que es músico el rey, y nota\\
    las más breves disonancias\\
    a lo de Jesús infante\\
    y a lo de David monarca.
\end{verse}
\end{blockquotation}

But it could also be arranged like this:
\begin{blockquotation}
\begin{tabular}{l@{\quad}l}
    Voces, las de la capilla, & cuenta con lo que se canta,\\
    que es músico el rey, y nota & las más breves disonancias\\
    a lo de Jesús infante  & y a lo de David monarca.
\end{tabular}
\end{blockquotation}

Padilla sets the text as though it were arranged in the latter way: he pauses
briefly where the \term{caesurae} would be, but punctuates the assonant lines
with clear points of harmonic arrival.
He articulates the end of the strophes in the \term{introducción} with full
cadences.

% choral recitative, declamation style in intro
% similar in coplas, but they are redondillas

% estribillo, more madrigalistic, following sentences and images in text more
% than poetic meter; but meter change distinguishes the section & last couplet
% gets special treatment
% careful attention to accentuation patterns in estribillo

\subsubsection{Text Depiction}




%******************************************************************************
\subsection{Theology of Christ as \term{Verbum infans}}
% theological context (e.g., patristic, liturgical allusions)
% grace/glory
% admirabile sacramentum

% VERBUM INFANS
% infante
% A & O
% (blessing hand vs Guidonian hand)



\subsection{Establishing a Pedigree in a Lineage of Metamusical Composition}

