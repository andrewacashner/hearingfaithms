% vim: set foldmethod=marker : %

% Hearing Faith: Music as Theology in the Spanish Empire
% 
% Andrew A. Cashner
% 
% Chapter 4, Heavenly Dissonance
% (Cererols, *Suspended, cielos*)
%
% 2015-03     Dissertation defended
% 2018-04-13  New version begun for book
% 2018-05-22  Converted to LaTeX
% 2018-06-12  New version resumed
% 2018-10-01  Continued by hand
% 2019-03-04  Type MS

\chapter{Heavenly Dissonance (Montserrat, 1660s)}
\label{ch:cererols-suspended}

%{{{1 intro
The Christmas villancico \wtitle{Suspended, cielos, vuestro dulce canto}, set by
Joan Cererols for the school choir of the Abbey of Montserrat around 1660, is
part of one of the most widely-attested families of villancicos, with evidence
surviving for seven other versions, from the Royal Chapel in Madrid to a convent
in Ecuador.
In the setting by Cererols, the chorus exhorts the heavens to cease the music of
the spheres and listen for \quoted{the newest consonance} that has come into the
world through the newborn Christ-child.
The piece thus extends the \term{Verbum infans} trope discussed in chapter
\ref{ch:padilla-voces} with a greater focus on cosmology, creating a reflection
on the relationship between earthly and heavenly music.
Rather than the word-level text depiction of the madrigal-like \wtitle{Voces} by
Juan Gutiérrez de Padilla, Cererols evokes the contrast between levels of music
by a corresponding contrast of musical motives and styles.
Together with this polystylistic technique, Cererols also uses dissonance as an
ironic symbol to point to a higher kind of heavenly music, not governed by the
same rules as earthly music.
By the end of the piece, attentive listeners can actually hear the motive
representing Christ's voice become the \term{cantus firmus} of a new heavenly
music; but within the Neoplatonic listening tradition, they are challenged to
listen beyond this depiction of heavenly music for the higher, unhearable music
of Christ himself.
The many versions of this text were passed from hand to hand among a network of
affiliated composers and modified to suit both local needs and changing
conceptions of heavenly music; tracing these connections provides further
evidence that composers used the metamusical villancico to prove their craft and
situate themselves in a lineage of composition.

The basic trope of this villancico is the same as that of Padilla's
\wtitle{Voces, las de la capilla}---it invites people to adore the Christ-child
as \term{Verbum infans}, the unspeaking/infant Word.
This poem is more explicit in presenting Christ's voice---as an expression of
Christ himself---as a form of music.
The heavenly spheres with their sidereal harmonies are out of tune, imperfect
creations compared to their Creator and further \quoted{subjected to futility}
(\scripture{Rm}{\XXX}) because of human sin.
Christ's voice will form the \term{cantus firmus} on which the music of a
renewed creation will be based.

The musical setting by Cererols is remarkable because the composers enables
listeners to actually hear the contrast between untuneful, dissonance-laden
music evocative of Plato's \quoted{world of change and decay} and the perfect
consonances and logical structures of a higher, heavenly music.
Moreover, the discerning listener can hear a musical process analogous to the
theological conceit of music in the poem---namely, that Christ's voice becomes
the \quoted{plain chant} that forms the basis of the \quoted{new song} of the
new heavens and earth.
The piece thus serves as a kind of school for Neoplatonic listening practice,
one that disciplined hearers to distinguish between the ways human music was
like heavenly truth and the ways it was not.
It encouraged them to listen for an unhearable higher music, ultimately that of
Christ himself in the triune Godhead.

In addition to its significance as the sole complete surviving setting of a
widely attested villancico text that was performed in different versions around
the world, Cererols's setting also establishes conventions for representing
heavenly and earthly music that numerous other composers in the same tradition
continued to develop throughout the seventeenth century. 
It is one link in a chain of related poems and compositions about heavenly music
in the Hispanic world that reflect common attitudes about the relationship
between human, cosmic, heavenly, and divine music---attitudes that were under
pressure to change amid the Scientific Revolution.

The villancico family, and especially Cererols's setting, open a window into how
early modern Catholics understood heavenly music in a time when conceptions of
the cosmos were changing all around them.
John Hollander's study of English poetry on music, \wtitle{The Untuning of the
Sky}, takes its title from a poem by John Dryden which envisions the apocalypse,
when, at the sound of the last trumpet, \quoted{MUSICK shall untune the sky}.
Hollander uses the title to refer to a process he sees of increasing
secularization and disenchantment across the seventeenth and eighteenth
centuries, so that by the end of the period poets use the language of heavenly
harmony in a purely conventionalized way devoid of real connection to any faith
in the old conceptions of the cosmos.
But Dryden's poem, very much like Cererols's villancico, actually affirms
traditional views: it emphasizes the untunefulness of worldly music (including
the heavenly spheres) compared to the higher music of God.
In a Neoplatonic tradition, it was normal to emphasize the deficiencies of the
\quoted{world of change and decay} and point beyond them to a higher truth.
In this study of Spanish poetry on music, we can note a change of
understandings, but not one that moves in the direction of secularization.
Instead, poets and composers---and devotees as well, we may imagine---find new
ways to reconcile the old cosmos with changing understandings and sensibilities.
%}}}1

%{{{1 sec: montserrat
\section{Cererols and the Boys' School Choir of Montserrat}

The only known complete musical setting is preserved today in a manuscript of
the parrochial archive of the church of Saints Peter and Paul in Canet de Mar, a
village on the seaside about thirty miles north of Barcelona.%
\begin{Footnote}
    Grateful acknowledgments are due to the rector and archive director for
    making a digital image of the manuscript freely available for this study.
    The Canet archive, under supervision of the archdiocese of Gerona, is being
    fully digitized.
\end{Footnote}
The undated manuscript attributes the music to Joan Cererols (1618--1680), who
was a monk and director of the school choir (Escolania) at the Abbey of Our Lady
of Montserrat.
The mountaintop monastery of Montserrat continues to be an influential
religious and musical center in the region. 
Much has changed since on the mountain since the time of Cererols, especially
because Napoleon's troops burned the monastery, including its library, when they
invaded Spain in 1811.
But pilgrims still make the long journey to the peak to revere the statue of the
Black Virgin of Montserrat, patroness and symbol of Catalonia.
They petition the Virgin's intercession in her shrine, ensconced in a special
chapel above the high altar of the abbey church, which is filled daily for
performances of the renowned boys' choir of the Escolania, the oldest
continuously operated choral school in Europe.

The nineteenth-century destruction is the reason why there are no sources of
Cererols's music original to Montserrat.
They all survive in copies made by students, such as the one taken to Canet de
Mar, and the sizable collection today in Barcelona's Biblioteca de Catalunya.
Even without the original collection at Montserrat, for the first modern edition
of music by Cererols in the 1930s, Dom David Pujols of Montserrat assembled no
less than seventy-eight manuscripts.
The thirty-five extant villancicos outnumber the surviving output of many of
Cererols's contemporaries who may be better known today.\XXX[who?]

Joan Pau Cererols Fornell was baptized in 1618 in the village of Martorell, in
the shadow of Montserrat.%
    \Autocite{Balanza:CererolsFamily}
He was the youngest child of Jaume Cererols, a well-to-do tailor.
His mother died when he was ten, and it appears that only a few months later
Joan was sent off to boarding school as a chorister at the Escolania of
Montserrat.%
    \Autocite{Balanza:CererolsFamily}
In the Escolania he would have received a thorough training in performance,
practical music theory, Latin, and the other typical subjects taught in church
schools.
After graduating Cererols entered the novitiate of the Montserrat Benedictines
at age eighteen, in 1636.
He remained at the monastery until his death in 1680, having become chapelmaster
of the Escola and teacher in the Escolania, as well as serving as sacristan of
the abbey church.

A monastery chronicle (probably written several decades after Cererols's death)
says that Cererols was \quoted{Chapelmaster and master of the choir-school boys
for more than thirty years}, having \quoted{left behind many writted \add{i.e.,
manuscript} books of music}.
Moreover, Cererols was \quoted{an excellent poet}, learned in letters and
theology, and able \quoted{to speak Latin as fluently as if it were his mother
tongue}.%
    \Autocite
    [{\XXX[original]}]
    [7, note 2]
    {Estrada:CererolsBio}
This description fits well with the evidence of \wtitle{Suspended, cielos},
which pairs sophisitcated poetry and elegantly masterful musical technique.
Cererols's version includes several variant poetic readings that do not appear
in any of the other seven sources, and most of these appear to be deliberate
changes directed by a keend theological and literary intellect, as will be shown
below.

The chroncle stresses the influence of Cererols as a teacher in a line of great
teachers:
\begin{quoting}
    He had the gift and talent of teaching and thus had so many students that
    there was hardly a church in this principality \add{Catalonia} whose
    Chapelmasters and Organists were not his students, aside from the many
    others that he had in other provinces of Spain, all of whom manifested the
    excellent qualities of their Teacher, as the reverend Father himself also
    demonstrated those of Father Master Márquez, of whom he \add{Cererols} was a
    student.
\end{quoting}
Indeed, the textual history of \wtitle{Suspended, cielos} shows that Cererols
was part of a network of musican influence and exchange that spread across Spain
and the New World. 
Cererols may have drawn some of his own influences from a stay of several years
in Madrid, after the religious orders of Catalonia fled there from the 1640
Catalan revolt.
Cererols's time in Madrid coincided with the flourishing of new musical styles
and forms at the royla court, led by the composers of the Royal Chapel,
chapelmaster Carlos Patiño and court harpist and composer Juan Hidalgo.
This means that Cererols was probably in Madrid when the first known version of
\wtitle{Suspended, cielos} was performed by the Royal Chapel on Christmas Eve
1651, and likely had direct access to the commemorative poetry imprint, if not
the original performance.
Cererols's poetic text is quite close to this earliest imprint.
Additionally, the influence of new styles from the Madrid court may be heard in
Cererols's music.
As noted, the large number of surviving copies of his music from outside
Montserrrat attests to his influence.

It may have been one of his students who brought the setting of
\wtitle{Suspended, cielos}, along with several other Cererols compositions, to
the church of Saints Peter and Paul in Canet de Mar.%
    \footnote{\sig{E-CAN}{AU-0116}.}
This sources preserves a complete setting of estribillo and six coplas scores
for an eight-voice double-choir ensemble with continuo accompaniment.
The estribillo features the whole chorus in a variety of textures, while the
coplas alternate between a Tiple duet and the four voices of Chorus I (likely
soloists), both with accompaniment.
The performing parts are covered in layers of fingerprints on the fold of each
sheet of paper, indicating many years of performance.
This extended composition demands a virtuoso ensemble, and is thus more likely
to have been composed for the school choir at Monsterrat and brought to Canet as
part of a personal collection, unless the ensemble at Canet was extraordinarily
capable for a small parish church.

In addition to this Canet manuscript, there is another source for this music, a
previously unattributed set of manuscript performing parts in Barcelona's
Biblioteca de Catalunya.%
    \footnote{\sig{E-Bbc}{M/765/25}.}
The Barcelona source includes only the estribillo, with music almost identical
to that in the Canet source, even in most details of coloration and accidentals,
with one significant variant---a  high ending for the first treble.
This alternate version is missing its original Tenor I and accompaniment parts
and lacks any setting of the coplas.
It adds the dynamic markings \term{eco} and \term{falsete} in repeated phrases
of polychoral dialogue. 
These terms became common in Hispanic villancicos after about 1660 and probably
reflect an attempt to make the piece suit changing aesthetics later in the
century, but they could possibly record they copyist's memory of a performance
tradition at Monsterrat.

\wtitle{Suspended, cielos} was originally a villancico for Christmas, as
evidenced by all the surviving poetry imprints and the contents of the text
itself.
In the Barcelona version, however, one verse of the poem has been modified to
suit a Eucharistic dedication instead of Christmas: \foreign{y con sollozos
tiernos/ un niño soberano} (and with tender sobs,/ a sovereign child) becomes
\foreign{y desde un pan divino/ un hombre soberano} (and through divine bread,/
a sovereign man).
Not one of the seven poetry imprints from this tradition includes these altered
lines, but instead agree with the Canet version for Christmas.
Confusingly, the Canet manuscript actually includes the label, in a different
hand on the coverleaf, designating the piece for \quoted{the blessed
Sacrament}---but it is clearly a Christmas piece.
The meaning of the difficult poem might have escaped the grasp of a later
archivist who was seeking to quickly categorize the piece.

In the archive the piece is grouped with several other pieces by Cererols,
including one (\wtitle{Pues que para la sepultura}) that remarkably incluides
both the composer's score and the performing parts.%
    \footnote{\sig{E-Bbc}{M/765/14}.}
This can only have originated with the composer himself, and must have been
passed on through a chain of musicians connecting back to Cererols.
This archival signature may document Cererols's musical network and influence.
It includes a different setting of \wtitle{Pues que para la sepultura} by Diego
de Cáseda, chapelmaster in Zaragoza and composer of his own lost setting of
\wtitle{Suspended, cielos}, known from the poetry imprint.
It also includes a work by Miguel Ambiela, a later occupant of the same post in
Zaragoza (see \cref{ch:zaragoza}).

One musician who would seem a likely candidate for carrying the Cererols legacy
to Barcelona and perhaps this specific manuscript, is the organist Gabriel
Manalt (1657--1687).
Manalt was baptized in Cererols's own home town of Martorell.
In such a small village, Manalt must have known the locally prominent Cererols
family, and it seems likely he was a student at the Escolania while Cererols was
chapelmaster.
Manalt was organist at the church of Santa María del Mar in Barcelona from 1679,
according to records of his audition (\term{oposición}), until his death.
He also served as interim chapelmaster from August 2 to September 26, 1685.%
    \Autocite[70--71]{Balanza:CererolsFamily}
In the notice of his burial at his home church in Martorell, Manalt was praised
as \quoted{a man highly accomplished in the art of playing the organ, and unique
in Catalonia}.%
\begin{Footnote}
    Martorell parroquial archive, \wtitle{Llibre d'Òbits 1669--1689},
    \range{f}{146}, quoted in 
    \autocite
    [{\XXX[original]}]
    [70]
    {Balanza:CererolsFamily}.
\end{Footnote}
The Montserrat chronicle includes organists among the students of Cererols, and
there is a strong probability that Manalt knew or studied with Cererols, and if
so he could certainly have brought this manuscript to Barcelona, perhaps for use
at Santa María del Mar.
There it could have been heard by many sea travellers who made a point of
stopping at this church before and after voyages.\XXX[really?]
%}}}1 

%{{{1 sec: poetry
\section{Words about Music}
%}}}1

%{{{1 sec: music
\section{Music about Music}

Cererols sets this intricately metamusical text in a way that goes beyond the
madrigalistic sort of word painting practiced by Gutiérrez de Padilla
(\cref{ch:padilla-voces}).
Cererols uses the large-scale formal structure to mirror the musical discourse
of the poem in musical terms.
Cererols builds a musical structure that presents listeners with a contrast of
two melodic motives and two stylistic topics.
The first, motive A, is sounded by the Alto I in the opening gesture: it is a
rising, then falling stepwise pattern, A--B--C--B--A.
The pattern is symmetrical, palindromic, and inscribes an arc on paper and in an
imaginative ear (\cref{fig:Cererols-motiveA}).

%{{{4 figure motive A
\begin{figure}
    \caption{Cererols, \wtitle{Suspended, cielos}, motive A}
    \label{fig:Cererols-motiveA}
    %\includeFigure{Cererols-motiveA}
\end{figure}
%}}}4

The figure has rich symbolic potential; for an initial reading let us begin with
the obvious implication that the motive represents the heavenly spheres
(\term{cielos}).
The same motive recurs throughout the opening polychoral dialogue on the same
words.
When the choir exhorts the heavens to \quoted{hold, stop, listen}, motive A is
sounded in Tenor and Alto of both choirs in turn (\measures{21--22, 23--24},
\cref{mux:Cererols-opening}).

%{{{4 mux Cererols-opening
\begin{musicexample}
    \caption{Cererols, \wtitle{Suspended, cielos}, opening}
    \label{mux:Cererols-opening}
    \includeMusic{Cererols-opening}
\end{musicexample}
%}}}4

In \measures{29--33} Cererols sets \foreign{la más nueva consonancia} to the
first four notes of motive A in Tiple I-2, then has Tiple I-2 imitate.
The motive returns with the music of \foreign{la más nueva consonancia} in
\measures{57--65}, and then for \foreign{y con sollozos tiernos} (and with
tender sobs/sighs), especially the Alto II in \measures{77--78}
(\cref{mux:Cererols-sollozos}). 
Motive A recurs in the estribillo's closing gesture, most notably in the Alto II
of the final cadence, and in the alternate Tiple I-1 ending of the Bbc source.
Versions of the motive saturate the setting of paired copla strophes.

%{{{4 mux Cererols-sollozos
\begin{musicexample}
    \caption{Cererols, \wtitle{Suspended, cielos}, motive A in 
    \measures{77--78}}
    \label{mux:Cererols-sollozos}
    \includeMusic{Cererols-sollozos}
\end{musicexample}
%}}}4

Everywhere this motive appears it is connected with a musical style that has
relatively worldly or lowly connotations.
It is a more homophonic, melody-oriented style featuring more dissonances used
in untraditional ways---in short, a more modern style like the new sounds
emerging from Madrid in the 1650s and 60s such as the \term{tonos} and
theatrical works of Juan Hidalgo and Cristóbal Galán, or perhaps even a
conservate reference to recent Italian innovations.

The opening gesture is a polychoral declamation, an exordium addressed to the
spheres.
The concept of \quoted{suspending} is enacted both in the drawn-out rhythms an
din the sevenths generated by motive A.
The rests that follow the gesture are crucial for the effect, especially the
grand pause after \foreign{escuchad} in \measure{28}.
The most vivid evocation of worldly, modern style follows this exhortation, in
Cererols's depiction of \quoted{the newest consonance} (\measures{29--38},
\cref{mux:Cererols-consonancia}).

%{{{4 mux Cererols-consonancias
\begin{musicexample}
    \caption{Cererols, \wtitle{Suspended, cielos}, \quoted{worldly} style for
    \quoted{the newest consonance}}
    \label{mux:Cererols-consonancias}
    \includeMusic{Cererols-consonancias}
\end{musicexample}
%}}}4

After the reverberation of the full ensemble's emphatic cadence dies away, the
voice of the Tiple I-2 (possibly a solo) would draw listeners in to the
mysterious passage that follows, in which a voice-and-continuo texture with the
Tiple I-2 alternates with the chorus in a kind of call-and-response.
% F# vs Fna
Cererols introduces a paradox here that will serve as an interpretive key for
the whole work: as the Tiple I-2 sings motive A, he sings the word
\term{consonancia} on a strong dissonance of G against C\sh{} and A---not
prepared according to traditional counterpoint rules.
The same figure is repeated, and the offending dissonant pitch reiterated.
Other modern elements here are the mixture of modes (suggesting mode I in
\term{cantus mollis}) and the juxtapositions of F\sh{} vs. B\fl{}.
Cererols makes another notable dissonance, again with motive A, on the word
\term{distancias} (\measures{40--41}), an exquisite
\musfig{7--\hphantom{X--}6}{6--5--4} progression.
In the passage about \quoted{tender sobs} or sighs (\measures{75--86}) Cererols
moves motive A against a background of dissonant suspensions resolving at
different times, culminating in another voice-leading \quoted{crunch} in
\measures{85--86}.
Using this kind of affectively laden music for human \quoted{sobs} seems
obvious, but why use it for representations of the heavens, and why in
particular use a prominent dissonance for the crucial phrase \foreign{la más
nueva consonancia}?
To answer that we must look at the other primary motive and its associated
style, because the meaning emerges from the contrast between the two.
% notebook 28, p. 94

Motive B is a scalar stepwise descent of a perfect fifth, sometimes with an
extra note on either end: D--A--G--F--E--D(--C\sh). 
It is first heard in the Tiple I-2 (\measures{35--38}) on \foreign{consonancia},
emerging out of the paradoxical passage just discussed.
In \measures{42--50} Cererols uses the motive as a point of imitation for
\quoted{the eternal and the temporal}, and mirrors the motive with its
inversion.
% TODO figure

In \measure{66}, after a repeat of the soloistic dissonance-on-a-consonance
passage, Cererols uses motive B as the subject of an eight-voice fugue in the
the already polyphonic classic style of Palestrina and his peers, using the
duplel metere traditional in Iberia for Latin-texted sacred music, and evoking
\foreign{contrapunto celestial} in inversions, transpositions, and strettos. 
The motive vanishes again for the passage about \foreign{sollozos}, and then
returns boldly in \measure{89} on \foreign{canto llano} like a
\quoted{plainchant} \term{cantus firmus} in long notes for a section in the
style of a traditional cantus-firmus motet.
% TODO example
Cererols actually extends the motive into a full-octave descent in the Tiple II
of \measures{92--97}.

Cererols thus creates a contrast between triple and duple meter,
homophonic/soloistic and contrapuntal texture, modern and traditional style,
unorthodox dissonance and strict control of consonance and dissonance.
The first of these binaries tends to be associated with references to the
spheres; the second, to the music of the angels and \quoted{the eternal}. 
References to Christ's own voice as the \term{Verbum infans} seem to cross both
territories.

Like Gutiérrez de Padilla did a few years earlier in \wtitle{Voces, las de la
capilla}, Cererols takes a contrast between hierarchical levels of human
music-making and maps it onto a higher contrast between divine and angelic music
on the one hand and worldly music on the other.
He uses strict classical contrapuntal technique and a more serene style to point
to the more elevated kind of divine music, and more subjective, affective,
imperfect music for the lower level.

Like the use of modern dissonance for \foreign{sollozos}, using strict,
old-style counterpoint for angelic music is part of a widespread, pan-European
tradition of musical representation. 
It is used in nearly every villancico in this study that evokes angelic music,
and similar techniques have been noted in contemporary Italian as well as German
Lutheran music---a tradition that persisted through Haydn, Mozart, and Beethoven
to today.%
\citXXX[Yearley, Johnston, Kendrick, etc]
In addition to its associations with solemn liturgical music, this kind of
counterpoint was suited to symbolize divine harmony because of its intricate
patterning, its theoretic basis in Pythagorean ratios thus producing
\quoted{sounding number}, and, in a seventeenth-century context, its relatively
inexpressive, objective affective content.

The other kind of music---the dissonant music for \quoted{the newest
consonance}---is more puzzling.
First, it is important to recognize the difference between types of
\quoted{heavenly} music.
In seventeenth-century Spanish letters, \term{cielos} could mean either the
planetary spheres or the spiritual \quoted{world beyond} them in which God dwelt
with his angels and saints.
That outer realm was the \term{cielo Empyreo}---the Empyrean.
The contrast in English between \quoted{the heavens} and \quoted{Heaven}.
% TODO figure of spheres and Empyrean

The villancico begins, then, with an exhortation to the spheres to cease their
music and listen for the new consonance. 
Motive A and its associated styles, then, evoke not the music of the Empyrean,
but the worldly music of the celestial spheres as part of the lower, created
world, and necessarily imperfect in a Neoplatonic system in which only God is
perfect.

This is no empty gesture toward a vague notion of celestial music: the details
of poetry and music reveal specific understandings of music's relationship to
the cosmos in a time when those understandings were beginning to change across
Europe. 
Spaniards in the mid seventeenth century believed in an Earth-centered,
Ptolemaic cosmos, in which the celestial spheres moved at rates and were spaced
at intervals such that they produced harmony, though writers had disagreed since
antiquity on whether this music could actually be heard by human ears (or if
not, why).
In a Neoplatonic conception, the material world reflection the perfection of a
higher plane of reality.
What we may miss, however, is the key distinction that this reflection is by
definition imperfect.
Only the Supreme Good for Platonists---the triune God for Christian
Neoplatonists---was perfect; all else fell short in greater degrees as one moved
down the chain of being. 
This is why Boethius's \quoted{three musics} did not describe \emph{all} music,
but rather all music in the material world, the \quoted{world of change and
decay} in Plato's words.
The music of the spheres was still the worldly, imperfect music of change and
decay.
In fact, music served quite well as a symbol of the imperfect created world
because all music known to human ears is ruled by both change---rhythmic,
melodic, harmonic; changes of style from place to place and across time---and
decay, from the dying sound of the lute or vihuela (see \cref{ch:zaragoza}) to
the reverberation of voices and instruments in stone spaces after the breath has
given out.
Worldly music symbolized and embodied the Neoplatonic paradox---it pointed
beyond itself to the highest divine harmony but also was marked by its
difference from that heavenly perfection.

Christian theology added to the Platonic distinction between perfect and
imperfect realms a difference between Creator and creation, and between creation
as envisioned by God and the fallen realm after the original sin of Eve and
Adam, which, as St. Paul said, was \quoted{subjected to futility}.%
\citXXX[Romans etc]. 
If death was the curse for eating the forbidden fruit, what was the order of
things in Eden beforehand?
And, to turn to the other end of the Christian timeline, in the Last Day, when
God creates a new heaven and a new earth and fills it with the \quoted{glorified
bodies} of the redeemed, what will be the natural laws of a place in which
people do not die, there is no night or darkness, no seasons or turning---in
short, no change or decay?

Early modern writers on music acknowledged that harmony in worldly music
depended on a controlled balance of consonance and dissonance---just as writers
on the natural world recognized harmony in the cycles of life and death, light
and dark in the world in which \quoted{there is a season for every purpose under
heaven}.%
\citXXX[Eccl.]
Fray Luis described the relationship of the four elements as a \quoted{saber
dance}, and the motion of the planets as a \quoted{concerted music} under the
direction of a divine chapelmaster.
Luis glorifies God for providing every creature with the means both to provide
for itself and to defend itself. 
The grim prospect this implies of \quoted{nature, red in tooth and claw} does
not seem to have concerned him.
In the world \quoted{under the sun}, death was as natural as life---\quoted{the
Lord giveth and the Lord takeht away, blessed be the name of the Lord}.%
\citXXX[Job]
In the world beyond, different rules must apply, and therefore the music of the
Empyrean must be profoundly unlike any music we know.
David Yearsley has traced the attempts of German Lutheran theologians to make
sense of this paradox, and the efforts of Lutheran musicians to embody an image
of heavenly perfection in sounding music.
Strict counterpoint, especially fugue and canon with inversions, was the most
widely employed type of music or this purpose.
As we have seen, Cererols uses the same trope, particularly in his fugue on
\foreign{contrapunto celestial} and his cantus-firmus motet on the passage about
\foreign{canto llano}.
The relative perfection of the classical contrapuntal style, contrasted againt
the unorthodox dissonances and affective gestures of the other dominant styles,
stands symbolically for the higher realm of divine music relative to human
music.

But Cererols's music for \quoted{the heavens}---that is, the worldly
spheres---is also participating in a Neoplatonic tradition of evoking the
imperfection of sidereal music in the fallen world.
Where the \quoted{higher} motive B is a plainchant-like linear descent, motive A
is a circle, an arc, a palindrome.
Its shape represents the constant turning of the spheres, as its movement
through consonance and dissonance evokes the natural cycles of the world
\quoted{here below}.
The passage on \foreign{la más nueva consonancia}, in particular, evokes
planetary motions through the lilting, dance-like, triple-meter rhythmic feel,
the oscillation between the minor triad on D and the major triad on A (with a
tonic/dominant feel here), and the call-and-response of the voices seeming to
echo back in a reverberant space.

All of this strengthens the cetnral conceit of the villancico: the music of the
spheres is out of tune.
It must halt its ceaseless rotations of consonance and dissonance and listen to
a new kind of music brought into the world by the union of divine and human (the
ultimate \term{musica humana}) in Christ.
The created world can never be perfectly tuned: lurking behind the apparent
perfections of Pythagorean ratios in the overtone series---something that is in
fact physically inscribed in every created thing as a natural law---is the
\quoted{wolf tone} of the Pythagorean comma.
The intervals do not add up to perfection.
Worldly music can only be tempered, not tuned perfectly.
The divine music of Christ who will make a new creation at the Last Day will
\quoted{untune the sky}, in Dryden's words, by revealing its decadent
imperfection.

Cererols's sonic picture of discordant heavens coordinates closely with the
contemporary musical cosmology of Athanasius Kircher.
Kircher, like most early modern Europeans, believed that the planetary bodies
exerted both positive and negative influences on humanity, as witness also in
the contemporary plays of Calderón.
The material world was made of the four elements held in tension, and the body
was moved by a balance and flow among the four humors. 
LIkewise, the spheres were arranged not as a chord but at discrete intervals
like a scale, with some intervals consonant and ohters dissonant in relation to
the earth.

In Kircher's cosmology of music, published just a year before the first poem in
this villancico family was printed, the planets are arranged in specific
patterns of consonance and dissonance that are best understood through the fine
points of species counterpoint.
The harmony of the spheres arises from these interactions such that even the
apparently bad (dissonance) is, in an Augustinian line of thinking, actually a
manifestation of a greater good:
\begin{quoting}
	Therefore there is nothign bad in the nature of things, that does not
	also yield to the good for the preservation of the whole universe.
	What else, therefore, are Mars and Saturn, than certian kinds of
	dissonances?
	---which dissonances, in relation to the perfect consonance of Jupiter,
	syncopated and tied in ligatures \addorig{ligata}, resolve not only in
	sweet music but also in the best kind of ornamentation.
	What esle is Mercury if not a certain kind of dissonance syncopated and
	tied between the Moon and Venus, which are like two consonances, so that
	the earth (which is born in freedom and not obligated to anything),
	thanks to the benign influence of the Sun, Venus, and the Moon, should
	not be corrupted.
	Truly, anyone who can consider this on a little higher level would find
	the seven planets to sing continuously in perfect, perpetual four-part
	polyphony \addorig{tetraphoniam}, in which the dissonances and
	consonances thereof are brought together, so that they should resolve in
	the most comely harmony of the world.%
	\citXXX[KircherII:383--384]
\end{quoting}
Kircher acknowledges, then, not only that the planets influence earth, but also
that they influence each other: their motions must be understood relative to
each other, and they are part of a dynamic system held in perpetual balance by
the interaction of these attractions and repulsions.

Contrapuntal rules for Kircher provide a way to understand the hidden forces
that animated the universe.
Thirty years later Isaac Newton would provide a coherent explanation of these
hidden forces of attraction through his laws of motion and the concept of
gravity.
Newton, like Kepler before him, was seriously concerned about the implications
of his observations for concepts of heavenly harmony.
Kepler, a Lutheran minister, had rejected his own theory after all because he
could not believe that God had made the world so unharmonious.
Kircher actually cites Kepler's table of planetary distances in order to
reject it for the same reason. 
Kircher goes so far as to provide a \soCalled{corrected} table with more
harmonious figures.
This shows that for Kircher and Catholic thinkers like him, dissonance in the
heavens was no cause for abandoning faith in the old cosmos or its Creator.
Rather, it had to be understood in its proper place as part of the created world
with its cycles of birth adn death, light and dark.
For a Neoplatonic thinker, imperfection in the creation was to be expected.
If the heavens were untuneful, that was because they were meant to point to a
higher music of the spiritual realm.

Kircher symbolizes his conception of the spheres in an example of actual music,
which is not meant to \emph{sound} like the harmony of the spheres but rather to
encode their relationships through musical technique:
\quoted{so that the curious reader should have a certain example of the
celestial polyphony, this can be seen demonstrated in musical notes according to
our speculative idea}.%
\citXXX[Ibid 384]
% TODO example
Kircher provides a detailed analysis of the example that unites contrapuntal and
astrological theory, continuing an allegory he has been developing throughout
the last book of his treatise based on the four strings of the Greek lyre, whose
names he uses for each voice part:
\begin{quoting}
	In the example Saturn, Jupiter, and Mars form the \term{netodum}, that
	is, they sing the highest voice, in which notes the consonant Jupiter
	always unites in harmony \addorig{ligat} and undoes the influence of
	\addorig{confringit} the dissonant Mars and Saturn.
	The Sun proceeds truly as the \term{mesodum} \add{Alto}, singing in
	perfect consonances, looking at the earth, the \term{proslambanomenon}
	\add{bass} from the octave above, or an octave and a fifth.
	Venus, Mercury, adn the Moon truly sing the \term{hypatodon}
	(\add{Tenor}), and Venus and the Moon, which are consonant, carrying
	Mercury in the friendship beteen them as a dissonant passing tone
	\addorig{intermedium dissonum}, thereby tie him up in harmonic
	intervals \addorig{modulis}, so that they absolutely restore
	consonances, as can be seen there in the notes of the Tenor part.
	The Earth truly receives frrom the substance of all these, therefore,
	the perfect mixture of consonances and dissonances, so that it
	constitutes the most pefect music with the planets, which we can imagine
	by using this musical example.%
	\citXXX[Kircher II:383-384]
\end{quoting}
In this way Kircher attempts to unite the ancient Greek concep tof the planets
as notes in a scale (as also in Kepler) with an emerging early modern concept of
polyphonic harmony.
In the Soprano and Tenor voices, each pitch stands for one planet, and their
consonance or dissonance relative to the bass and the melodic motions linking
them, symbolize the planets' influence on Earth and each other.%
	\footnote{The clef of the Tenor system should be a Tenor clef.}
Venus prepares the dissonance of Mercury, for example, and the Moon resolves it.
The Alto and Bass, though, each represent a single celestial body, and the
symbolism is not as exact.
(The Earth is the Bass because it was unmovable, but the bass voice here also
moves.)
In terms of species counterpoint, the Alto (Sun) and Bass (Earth) move in
perfect first-species (note-for-note, all consonant) counterpoint.
The Tenor (inner planets) is fourth species (ligatures and suspensions), while
the Soprano is second species ($2:1$).
% TODO table from diss. p. 264
Kircher shows here both that the heavens could be understood in musical terms
and that music could be understood in heavenly terms.
Kircher's \term{clausula} or cadence of the planets demonstrates that
Neoplatonic thought about music did not always begin with theory and descend to
practice; it also used contemporary \term{musica instrumentalis} as a specific
model or metaphor for the higher conceptions of music on the cosmic level. 
People comparing the heavens to music had real music in their ears.
Kircher's \term{clausula} sounds like a perfectly ordinary seventeenth-century
cadence (in mode I, transposed to \term{cantus mollis}), but for Kircher even
the \quoted{mundane} details of the counterpoint such as passing tones and
suspensions had high symbolic potential.
This example suggests that composers and educated listeners thought symbolically
in Neoplatonic terms even about the basic fabric of their compositions.

Cererols's villancico exemplifies this close link between musical practice and
conceptions fo the heavens.
Cererols's poetic text may even recall this specific passage when it says CHrist
will be a \foreign{divina cláusula}; efen if not it is a similar attempt to
express the heavens through polyphonic harmony.

Kircher's detailed symbolism might prompt us to look more closely at the precise
contrapuntal relationships in Cererols's depiction of the heavenly.
As already noted, Cererols juxtaposes two contrasting styles to represent a
contrast between the untempered music ofthe material world and the higher music
of the divine.
He uses learned counterpoint with strictly controlled dissonance and a more
linear rather than harmonic conception to represent the higher music. 
Cererols represents worldly music (which includes the spheres of the
\quoted{heavens} invoked at the outset) with a more harmonically conceived,
dissonance-laden style.

In his use of a dissonance for the word \foreign{consonancia}, though, Cererols
takes a different approach from Kircher.
Kircher's cadence is, as he says, \quoted{perfect \Dots{} polyphony}, following
contrapuntal rules exactly.
Cererols, by contrast, breaks the rules with these unprepared dissonances.
On one level, the dissonance in these passages seems to symbolize the
imperfection of the worldly order.
But the dissonance on \foreign{la más nueva consonancia} also functions as an
ironic symbol, pointing to a higher kind of music whose rules defy human
understanding.
Cererols borrows modern style as if to say that dissonance is the new
consonance.

Kircher make a similar gesture in a different passage, where he describes the
music of heaven as beyond human imagining, and resorts to paradox to evoke it.
In the last book of the \wtitle{Musurgia} Kircher presents the whole creation as
a \quoted{Praeludium} played by God on a divine organ of creation.
In the engraving, the organ is depicted in exacting detail as a real instrument:
there is a pipe shown for every key of every rank.
The odd arrangement of the keyboard, though, would catch a reader's attention
and require explanation.
% TODO figure
The keys are arranged in groups of seven rather than twelve chromatic pitches in
octaves, with repeating groups of three black keys instead of the three-and-two
pattern of earthly keyboards. 
The seven-key groups represent the days of creation, and the three black keys,
it would seem, the Holy Trinity. 
Perhaps the keys correspond to a diatonic series, so that each seven-note group
forms a diatonic octave (and would map directly onto the planets as well).
Whatever the arrangement, this keyboard is not designed for playing earthly
music.
The Latin motto beneath the keys reads, \quoted{Thus does the eternal wisdom of
God play upon the spheres of the worlds}.%
\citXXX[Apocrypha?]
To imagine playing an actual Praeludium (say, by Kircher's countryman Buxtehude
or his neighbor Frescobaldi) on this organ is to contemplate what Olivier
Messiaen would later call \quoted{the charm of impossibilities}.
The image is a riddle that points to a divine music, governed by different rules
that that of music in the lower world.

Cererols, then, could be presenting his hearers with an auditory symbol of this
impossible music.
By pointing out the imperfect artifice of the music itself, Cererols prompts
listeners to reflect on how the imperfect reflects God's perfection.
In the theological context of this Christmas villancico, the \quoted{newest
consonance} is, of course, Christ himself.
As in Gutiérrez de Padilla's villancico, this piece makes Christ the
\term{Verbum infans}, the Word made flesh as an unspeaking infant.
The \quoted{sighs} of the baby, referred to several times in the text, are the
\quoted{new song} that becomes the \term{cantus firmus} of a renewed creation.
Through Cererols's interplay of motives and styles, listeners can actually hear
the music of human and divine emergin over the course of the piece, culminating
in the evocation of a motet based on motive B and ending with a \term{peroratio}
(final flourish) based on motive A.

It is even possible to read the whole estribillo as following a similar
rhetorical structure to that of contemporary organ praeludia, and this makes
sense since the poem is an exhortation of the heavens.
The piece follows the Quintilian pattern of \term{exordium}, \term{propositio},
\term{narratio} (\term{confutatio/confirmatio} pairs), and \term{peroratio}.
% TODO details
If the villancico is an oration, then its main subject is Christ as \term{Verbum
infans}.
Neither motive or style solely represents Christ's voice; rather the theme is
the paradoxical mixture of divine and human in the Incarnation.
In Christ God entered the world of change and decay \quoted{to bring consonance
to the clay}.
Christ, particularly in his Passion and then in the Eucharist, was both
consonance and dissonance, old and new, material and spiritual.

%}}}1

%{{{1 genealogy
\section{Genealogy}

If this villancico by Cererols were the only one of its kind it might register
as an interesting footnote, but in fact this setting is part of the
best-attested (to current knowledge) family of villancicos in the seventeenth
century.
Evidence survives for eight other settings of this poem in several variant
textual families, from its first known appearance in Madrid at Christmas 1651
through performances in Toledo, Zaragoza, Seville, and a fragmentary setting
from a convent in Ecuador. 
The multiple settings point to the lasting and widespread value of this text to
Spanish musicians and their congregations, and suggest a widespread fascination
with its theme of celestial and divine music.
Moreover, the composers of most of these settings were closely linked in a web
of personal affiliations, so that their choice to set a text (or their own local
variant of a text) previously set by a teacher, colleague, or rival enabled them
to situate themselves within a tradition of both composition---as they
demonstrated their prowess at musical \term{conceptismo}---and theology, as they
continued to develop tropes of heavenly music in the midst of changing
understandings of the cosmos.


%}}}1

\endinput

