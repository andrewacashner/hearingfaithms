% vim: set foldmethod=marker : %

% Hearing Faith: Music as Theology in the Spanish Empire
% 
% Andrew A. Cashner
% 
% Chapter 4, Heavenly Dissonance
% (Cererols, *Suspended, cielos*)
%
% 2015-03     Dissertation defended
% 2018-04-13  New version begun for book
% 2018-05-22  Converted to LaTeX
% 2018-06-12  New version resumed
% 2018-10-01  Continued by hand
% 2019-03-04  Type MS
% 2019-04-18  Complete typing MS and fill in genealogy, poetry

\chapter{Heavenly Dissonance (Montserrat, 1660s)}
\label{ch:cererols-suspended}

%{{{1 intro
Pilgrims who reached the mountaintop Abbey of Our Lady of Montserrat in time for
Christmas services were primed for experiences of exaltation.
Perched on a jagged peak overlooking the Barcelona region, the abbey church
enshrined the statue of the Black Virgin of Montserrat, patron of Catalonia.
Many came in search of miracles or in payment of spiritual debts, and though
they came to see the Virgin, they also came to hear the abbey's famous school
choir, the Escolonia---today the oldest continuously operating singing school in
the world.
This was the environment in which the villancico for eight-voice double chorus
by Joan Cererols, \wtitle{Suspended, cielos, vuestro dulce canto}, was first
heard.%
    \citXXX[signature etc.]
The chorus addressed the celestial spheres themselves, commanding them to cease
their perpetual music and listen.
A new song, \quoted{the newest consonance}, was forming in the person of the
Christ-child. 
As \term{Verbum infans} his cries would form the basis for the music of a
renewed creation.
By the end of the piece, attentive listeners can actually hear the theme
representing Christ's voice become the \term{cantus firmus} of a new heavenly
music. 
The piece was a school for spiritual listening within a Neoplatonic tradition:
it challenged hearers to listen beyond this depiction of heavenly music for the
higher, unhearable music of Christ himself.

The setting by Cererols is one of seven known versions of this villancico,
performed between 1650 and 1700 across Iberia, with a later version appearing as
far afield as Ecuador.
The themes of the poem---the relationship between worldly and divine music, the
symbolism of consonance and dissonance, the theme of Christ as both singer and
song---must have resonated with many musicians and worshippers.
This is all the more notable given that these poetic and musical celebrations of
the old earth-centered cosmological system were being developed precisely at the
same time as Newton was formulating the theories that would destroy that system.
The different versions of this villancico, then, allow us to see how Spanish
Catholics understood music's place in the cosmos even as their understanding of
the cosmos was being challenged.

John Hollander's study of seventeenth-century English poetry on music,
\wtitle{The Untuning of the Sky}, takes its title from a poem by John Dryden
which envisions the apocalypse, when, at the sound of the last trumpet,
\quoted{MUSICK shall untune the sky}.%
    \citXXX[Hollander, Dryden]
Hollander uses the title to refer to a process he sees of increasing
secularization and disenchantment across the seventeenth and eighteenth
centuries, so that by the end of the period poets use the language of heavenly
harmony in a purely conventionalized way devoid of real connection to any faith
in the old conceptions of the cosmos.
But Dryden's poem, very much like Cererols's villancico, actually affirms
traditional views: it emphasizes the untunefulness of worldly music (including
the heavenly spheres) compared to the higher music of God.
In a Neoplatonic tradition, it was normal to highlight the deficiencies of the
\quoted{world of change and decay} in order to point beyond them to a higher
truth.
In Spanish poetry on music, we can note a change of understandings, but not one
that moves in the direction of secularization.%
    \citXXX[Uribe Bracho, Texas guy]
Instead, poets and composers---and devotees as well, we may imagine---found new
ways to reconcile the old cosmos with changing sensibilities.
%}}}1

%{{{1 sec: montserrat
\section{Cererols and the Boys' School Choir of Montserrat}

The only known complete musical setting is preserved today in a manuscript of
the parochial archive of the church of Saints Peter and Paul in Canet de Mar, a
village on the seaside about thirty miles north of Barcelona.%
\begin{Footnote}
    Grateful acknowledgments are due to the rector and archive director for
    making a digital image of the manuscript freely available for this study.
    The Canet archive, under supervision of the archdiocese of Gerona, is being
    fully digitized.
\end{Footnote}
The undated manuscript attributes the music to Joan Cererols (1618--1680), who
was a monk and director of the school choir (Escolania) at the Abbey of Our Lady
of Montserrat.%
    \citXXX[catalog]
The Catalan-inflected phonetic spellings in the source (\term{suspendet} instead
of \term{suspended}, reflecting Catalan's final-obstruent devoicing) support an
origin for the source in this region.

The mountaintop monastery of Montserrat continues to be an influential
religious and musical center in the region. 
A seventeenth-century painting depicts the Escolania praising the Black Virgin:
their arrangement in double-choir formation, their instruments (\term{bajón},
\term{sacabuche}), and their looseleaf partbooks, all match with the performance
practice of villancicos like those of Cererols (\cref{fig:escolania}).

%\insertFigure{escolania}
%{\wtitle{Mare de Déu de Montserrat}, 17th cent., Abadia de Montserrat, showing
%Escolania}

Joan Pau Cererols Fornell was baptized in 1618 in the village of Martorell, in
the shadow of Montserrat.%
    \Autocite{Balanza:CererolsFamily}
He was the youngest child of Jaume Cererols, a well-to-do tailor.
His mother died when he was ten, and it appears that only a few months later
Joan was sent off to boarding school as a chorister at the Escolania of
Montserrat.%
    \Autocite{Balanza:CererolsFamily}
In the Escolania he would have received a thorough training in performance,
practical music theory, Latin, and the other typical subjects taught in church
schools.
After graduating Cererols entered the novitiate of the Montserrat Benedictines
at age eighteen, in 1636.
He remained at the monastery until his death in 1680, having become chapelmaster
of the Escolania and teacher in the Escola, as well as serving as sacristan of
the abbey church.

A monastery chronicle (probably written several decades after Cererols's death)
says that Cererols was \quoted{Chapelmaster and master of the choir-school boys
for more than thirty years}, having \quoted{left behind many written \add{i.e.,
manuscript} books of music}.
Moreover, Cererols was \quoted{an excellent poet}, learned in letters and
theology, and able \quoted{to speak Latin as fluently as if it were his mother
tongue}.%
    \Autocite
    [{\XXX[original]}]
    [7, note 2]
    {Estrada:CererolsBio}
This description fits well with the evidence of \wtitle{Suspended, cielos},
which pairs sophisticated poetry and elegant musical technique.
Cererols's version includes several variant poetic readings that do not appear
in any of the other seven sources, and most of these appear to be deliberate
changes directed by a keen theological and literary intellect, as will be shown
below.

The chronicle stresses the influence of Cererols as a teacher in a line of great
teachers:
\begin{quoting}
    He had the gift and talent of teaching and thus had so many students that
    there was hardly a church in this principality \add{Catalonia} whose
    Chapelmasters and Organists were not his students, aside from the many
    others that he had in other provinces of Spain, all of whom manifested the
    excellent qualities of their Teacher, as the reverend Father himself also
    demonstrated those of Father Master Márquez, of whom he \add{Cererols} was a
    student.
\end{quoting}
Indeed, the textual history of \wtitle{Suspended, cielos} shows that Cererols
was part of a network of musican influence and exchange that spread across Spain
and the New World. 
Cererols may have drawn some of his own influences from a stay of several years
in Madrid, after the religious orders of Catalonia fled there from the 1640
Catalan revolt.%
    \citXXX[Cererols bio]
Cererols's time in Madrid coincided with the flourishing of new musical styles
and forms at the royla court, led by the composers of the Royal Chapel,
chapelmaster Carlos Patiño and court harpist and composer Juan Hidalgo.%
    \citXXX[on court chapel, Hidalgo; Stein]
This means that Cererols was probably in Madrid when the first known version of
\wtitle{Suspended, cielos} was performed by the Royal Chapel on Christmas Eve
1651, and likely had direct access to the commemorative poetry imprint, if not
the original performance.
Cererols's poetic text is quite close to this earliest imprint.
Additionally, the influence of new styles from the Madrid court may be heard in
Cererols's music (see below).

The large number of surviving copies of his music from outside Montserrrat
attests to his influence.
There are no sources of Cererols's music original to Montserrat, because the
monastery library was burned by Napoleon's troops in 1811.%
    \citXXX[Montserrat history]
All his music survives in copies made by students, such as the one taken to
Canet de Mar, and the sizable collection today in Barcelona's Biblioteca de
Catalunya.
Even without the original collection at Montserrat, for the first modern edition
of music by Cererols in the 1930s, Dom David Pujols of Montserrat assembled no
less than seventy-eight manuscripts.%
    \citXXX[MEM]
The thirty-five extant villancicos outnumber the surviving output of many of
Cererols's contemporaries who may be better known today.\XXX[who?]

The Canet source preserves a complete setting of estribillo and six coplas
scored for an eight-voice double-choir ensemble with continuo accompaniment.
The estribillo features the whole chorus in a variety of textures, while the
coplas alternate between a Tiple duet and the four voices of Chorus I (likely
soloists), both with accompaniment.
The performing parts are covered in layers of fingerprints on the fold of each
sheet of paper, indicating many years of performance.
This extended composition demands a virtuoso ensemble, and is thus more likely
to have been composed for the school choir at Monsterrat and brought to Canet as
part of a personal collection, unless the ensemble at Canet was extraordinarily
capable for a small parish church.

In addition to this Canet manuscript, there is another source for this music, a
previously unattributed set of manuscript performing parts in Barcelona's
Biblioteca de Catalunya.%
    \footnote{\sig{E-Bbc}{M/765/25}.}
The Barcelona source includes only the estribillo, with music almost identical
to that in the Canet source, even in most details of coloration and accidentals,
with one significant variant---a  high ending for the first treble.
This alternate version is missing its original Tenor I and accompaniment parts
and lacks any setting of the coplas.
It adds the dynamic markings \term{eco} and \term{falsete} in repeated phrases
of polychoral dialogue. 
These terms became common in Hispanic villancicos after about 1660 and probably
reflect an attempt to make the piece suit changing aesthetics later in the
century, but they could possibly record they copyist's memory of a performance
tradition at Monsterrat.%
    \citXXX[on dynamic markings, or example, e.g., Carrión]

\wtitle{Suspended, cielos} was originally a villancico for Christmas, as
evidenced by all the surviving poetry imprints and the contents of the text
itself.
In the Barcelona version, however, one verse of the poem has been modified to
suit a Eucharistic dedication instead of Christmas: \foreign{y con sollozos
tiernos/ un niño soberano} (and with tender sobs,/ a sovereign child) becomes
\foreign{y desde un pan divino/ un hombre soberano} (and through divine bread,/
a sovereign man).
Not one of the seven poetry imprints from this tradition includes these altered
lines, but instead agree with the Canet version for Christmas.
Confusingly, the Canet manuscript actually includes the label, in a different
hand on the coverleaf, designating the piece for \quoted{the blessed
Sacrament}---but it is clearly a Christmas piece.
The meaning of the difficult poem might have escaped the grasp of a later
archivist who was seeking to quickly categorize the piece.

In the archive the piece is grouped with several other pieces by Cererols,
including one (\wtitle{Pues que para la sepultura}) that remarkably includes
both the composer's score and the performing parts.%
    \footnote{\sig{E-Bbc}{M/765/14}.}
This can only have originated with the composer himself, and must have been
passed on through a chain of musicians connecting back to Cererols.
This archival signature may document Cererols's musical network and influence.
It includes a different setting of \wtitle{Pues que para la sepultura} by Diego
de Cáseda, chapelmaster in Zaragoza and composer of his own lost setting of
\wtitle{Suspended, cielos}, known from the poetry imprint.
It also includes a work by Miguel Ambiela, a later occupant of the same post in
Zaragoza (see \cref{ch:zaragoza}).

One musician who would seem a likely candidate for carrying the Cererols legacy
to Barcelona and perhaps this specific manuscript, is the organist Gabriel
Manalt (1657--1687).
Manalt was baptized in Cererols's own home town of Martorell.
In such a small village, Manalt must have known the locally prominent Cererols
family, and it seems likely he was a student at the Escolania while Cererols was
chapelmaster.
Manalt was organist at the church of Santa María del Mar in Barcelona from 1679,
according to records of his audition (\term{oposición}), until his death.%
    \citXXX[Manalt]
He also served as interim chapelmaster from August 2 to September 26, 1685.%
    \Autocite[70--71]{Balanza:CererolsFamily}
In the notice of his burial at his home church in Martorell, Manalt was praised
as \quoted{a man highly accomplished in the art of playing the organ, and unique
in Catalonia}.%
\begin{Footnote}
    Martorell parroquial archive, \wtitle{Llibre d'Òbits 1669--1689},
    \range{f}{146}, quoted in 
    \autocite
    [{\XXX[original]}]
    [70]
    {Balanza:CererolsFamily}.
\end{Footnote}
The Montserrat chronicle includes organists among the students of Cererols, and
there is a strong probability that Manalt knew or studied with Cererols.
If so he could certainly have brought this manuscript to Barcelona, perhaps for
use at Santa María del Mar.
%}}}1 

%{{{1 sec: poetry
\section{Words about Music}

In the \term{conceptista} tradition, the anonymous poem turns a glossary of
musical terms into theological conceits, centered on the contrast between the
out-of-tune heavens and the \quoted{new consonance} of Christ.
The voice of Christ here manifests as the incarnate Word, creating
perfect harmony between God and Man. 
This voice is the new song before which the spheres must cease their chanting
and simply listen.
The implication is that what passes for consonance in the material world is
dissonant or out of tune by comparison to this new song.
Indeed, Christ's weeping cries---those at his birth and those upon the cross,
which the others presage---are to be the \quoted{plainchant} upon which the
music of a new creation will be based, forming the polyphonic \quoted{cadence}
of a new heavens and a new earth.%
    \citXXX[lit on new song]

Like many villancicos this one begins with an exhortation to listen, but here it
is addressed not to the worshipper but to the cosmic spheres, recalling numerous
Biblical passages including two of the psalms for Christmas Matins.%
    \citXXX[Ps. 18, 95; Dt. 32, Is. \XXX{}]
Above the planetary spheres of the Ptolemaic universe, \quoted{the hierarchies
are intoning \Dots{} celestial counterpoint}. 
\foreign{Jerarquías} was a technical term from theology for the levels of
heavenly beings like the cherubim and seraphim, as in books of angelology like
that of Blasco Lanuza (1652) and the widely read \wtitle{Celestial Hierarchy}
attributed to Dionysius the Areopagite.%
    \citXXX[diss p 242]
Athanasius Kircher likewise interprets Colossians 1:16 as a taxonomy of heavenly
beings.
When Kircher tabulates the relationship between notes of the scale and the
created world, he makes a clear distinction between angelic and worldly levels.
In this poem the spheres are to fall silent before the higher music of the
angels, based on the \term{cantus firmus} of Christ himself.%
\begin{Footnote}
    This recalls Rev. 8:1, in which the heavens fall silent when the Lord's word
    is spoken, after the seventh seal is opened and before the angels blow the
    seven trumpets of doom.
    That the heavens should fall silent when the last trumpet sounds is also the
    central conceit of Dryden's poem for St. Cecilia's day. 
    % XXX Hollander, or later
\end{Footnote}

The coplas situate the miracle of Christ's birth in salvation history through
subtle musical conceits.
In copla 1, Adam's fall from grace and expulsion from Eden are presented as a
\quoted{fugue} (the same word as \quoted{flight}, like the musical term
\term{catch} in seventeenth-century English), formed in \quoted{heedless paces}
or \quoted{careless steps}. 
\foreign{Pasos} could refer either to the steps of melodic intervals or to the
paces of rhythmic values; carelessnes in either regard would destroy the
counterpoint.
The fugal texture in which all the voices imitate the first one could be
interpreted as a picture of the transmission of original sin from Adam to every
person.
Christ's voice, the poem says, specifically fixes the \term{compás} or measure,
\quoted{through the pearls of his crying} (recall that Christ also gave the
\term{compás} in Jalón's \wtitle{Cantores}).
The second copla is the most obscure, even on a grammatical level, but the basic
idea still seems to be connecting terms with musical significance like
\term{despeños} (falls, as in melodic descents or ornaments?), \term{corriente}
(running melismas or fast notes?) and \term{blandos} (flats).
It envisages Christ's tears (a metonym for his crying voice as well) as either
restraining humanity from the full consequences of the Fall, or as a portrayal
of his passion.
The other coplas conceive of Christ as a \quoted{sovereign concord} which
\quoted{brings order to the dissonance of the clay}---that is, redeeming
humanity by taking on frail human flesh and in his own body reconciling humans
to God.%
    \citXXX[Eph 2 etc.]
The incarnate Christ will form (or perform) a \term{concierto}---a unified
composition made from disparate elements, or a concerto.
%}}}1

%{{{1 sec: music
\section{Music about Music}

%{{{2 analysis
Cererols sets this intricately musical text in a way that goes beyond the
madrigalistic sort of word painting practiced by Gutiérrez de Padilla
(\cref{ch:padilla-voces}).
Cererols uses the large-scale formal structure to mirror the musical discourse
of the poem in musical terms.
Cererols builds a musical structure that presents listeners with a contrast of
two melodic motives and two stylistic topics.
The first, motive A, is sounded by the Alto I in the opening gesture: it is a
rising, then falling stepwise pattern, A--B--C--B--A.
The pattern is symmetrical, palindromic, and inscribes an arc on paper and in an
imaginative ear (\cref{fig:Cererols-motiveA}).

%\insertFigure{Cererols-motiveA}
%{Cererols, \wtitle{Suspended, cielos}, motive A}

The figure has rich symbolic potential; for an initial reading let us begin with
the obvious implication that the motive represents the heavenly spheres
(\term{cielos}).
The same motive recurs throughout the opening polychoral dialogue on the same
words.
When the choir exhorts the heavens to \quoted{hold, stop, listen}, motive A is
sounded in Tenor and Alto of both choirs in turn (\measures{21--22, 23--24},
\cref{mux:Cererols-opening}).

%\insertMusic{Cererols-opening} 
%{Cererols, \wtitle{Suspended, cielos}, opening}

In \measures{29--33} Cererols sets \foreign{la más nueva consonancia} to the
first four notes of motive A in Tiple I-2, then has Tiple I-2 imitate.
In \measures{57--65} the motive returns with same music that was used for
\foreign{la más nueva consonancia}, which is also reworked for \foreign{y con
sollozos tiernos} (and with tender sobs/sighs).
The motive is especially prominent in the Alto II in \measures{77--78}
(\cref{mux:Cererols-sollozos}). 
Motive A recurs in the estribillo's closing gesture, most notably in the Alto II
of the final cadence, and in the alternate Tiple I-1 ending of the Bbc source.
Versions of the motive saturate the setting of paired copla strophes.

%\insertMusic{Cererols-sollozos} 
%{Cererols, \wtitle{Suspended, cielos}, motive A in \measures{77--78}}

Everywhere this motive appears it is connected with a musical style that has
relatively worldly or lowly connotations.
It is a more homophonic, melody-oriented style featuring more dissonances used
in untraditional ways---in short, a more modern style like the new sounds
emerging from Madrid in the 1650s and 60s such as the \term{tonos} and
theatrical works of Juan Hidalgo and Cristóbal Galán, or perhaps even a
conservate reference to recent Italian innovations.%
    \citXXX[examples]

The opening gesture is a polychoral declamation, an \term{exordium} addressed to
the spheres.
The concept of \quoted{suspending} is enacted both in the drawn-out rhythms an
din the sevenths generated by motive A.
The rests that follow the gesture are crucial for the effect, especially the
grand pause after \foreign{escuchad} in \measure{28}.
The most vivid evocation of worldly, modern style follows this exhortation, in
Cererols's depiction of \quoted{the newest consonance} (\measures{29--38},
\cref{mux:Cererols-consonancia}).

%\insertMusic{Cererols-consonancias}
%{Cererols, \wtitle{Suspended, cielos}, \quoted{worldly} style for \quoted{the
%newest consonance}}

After the reverberation of the full ensemble's emphatic cadence dies away, the
voice of the Tiple I-2 (possibly a solo) would draw listeners in to the
mysterious passage that follows, in which a voice-and-continuo texture with the
Tiple I-2 alternates with the chorus in a kind of call-and-response.
% F# vs Fna
Cererols introduces a paradox here that will serve as an interpretive key for
the whole work: as the Tiple I-2 sings motive A, he sings the word
\term{consonancia} on a strong dissonance of G against C\sh{} and A---not
prepared according to traditional counterpoint rules.
The same figure is repeated, and the offending dissonant pitch reiterated.
Other modern elements here are the mixture of modes (suggesting mode I in
\term{cantus mollis}) and the juxtapositions of F\sh{} vs. B\fl{}.
Cererols makes another notable dissonance, again with motive A, on the word
\term{distancias} (\measures{40--41}), an exquisite
\musfig{7}{6-5}\musfig{-}{-}\musfig{6}{4} progression. %XXX fix or cut
In the passage about \quoted{tender sobs} or sighs (\measures{75--86}) Cererols
moves motive A against a background of dissonant suspensions resolving at
different times, culminating in another voice-leading \quoted{crunch} in
\measures{85--86}.
Using this kind of affectively laden music for human \quoted{sobs} seems
obvious, but why use it for representations of the heavens, and why in
particular use a prominent dissonance for the crucial phrase \foreign{la más
nueva consonancia}?
To answer that we must look at the other primary motive and its associated
style, because the meaning emerges from the contrast between the two.
% notebook 28, p. 94

Motive B is a scalar stepwise descent of a perfect fifth, sometimes with an
extra note on either end: D--A--G--F--E--D(--C\sh). 
It is first heard in the Tiple I-2 (\measures{35--38}) on \foreign{consonancia},
emerging out of the paradoxical passage just discussed.
In \measures{42--50} Cererols uses the motive as a point of imitation for
\quoted{the eternal and the temporal}, and mirrors the motive with its
inversion.
% TODO figure

In \measure{66}, after a repeat of the soloistic dissonance-on-a-consonance
passage, Cererols uses motive B as the subject of an eight-voice fugue in the
the already classic polyphonic style of Palestrina and his peers, using the
duple meter traditional in Iberia for Latin-texted sacred music, and evoking
\foreign{contrapunto celestial} in inversions, transpositions, and strettos. 
The motive vanishes again for the passage about \foreign{sollozos}, and then
returns boldly in \measure{89} on \foreign{canto llano} like a
\quoted{plainchant} \term{cantus firmus} in long notes for a section in the
style of a traditional cantus-firmus motet.
Cererols actually extends the motive into a full-octave descent in the Tiple II
of \measures{92--97}.
% TODO example

Cererols thus creates a contrast between triple and duple meter,
homophonic/soloistic and contrapuntal texture, modern and traditional style,
unorthodox dissonance and strict control of consonance and dissonance.
The first of these binaries tends to be associated with references to the
spheres; the second, to the music of the angels and \quoted{the eternal}. 
References to Christ's own voice as the \term{Verbum infans} seem to cross both
territories.
%}}}2

%{{{2 symbolism
\subsection{Symbolism}

Like Gutiérrez de Padilla did a few years earlier in \wtitle{Voces, las de la
capilla}, Cererols takes a contrast between hierarchical levels of human
music-making and maps it onto a higher contrast between divine and angelic music
on the one hand and worldly music on the other.
He uses strict contrapuntal technique and a more serene style to point to the
more elevated kind of divine music, and more subjective, affective,
imperfect music for the lower level.

Like the use of modern dissonance for \foreign{sollozos}, using old-style
counterpoint for angelic music is part of a widespread, pan-European tradition
of musical representation. 
It is used in nearly every villancico in this study that evokes angelic music,
and similar techniques have been noted in contemporary Italian as well as German
Lutheran music---a tradition that persisted through Haydn, Mozart, and Beethoven
to today.%
\citXXX[Yearley, Johnston, Kendrick, etc]
In addition to its associations with solemn liturgical music, this kind of
counterpoint was suited to symbolize divine harmony because of its intricate
patterning, its theoretic basis in Pythagorean ratios thus producing
\quoted{sounding number}, and, in a seventeenth-century context, its relatively
inexpressive, objective affective content.

The other kind of music---the dissonant music for \quoted{the newest
consonance}---is more puzzling.
First, it is important to recognize the difference between types of
\quoted{heavenly} music.
In seventeenth-century Spanish letters, \term{cielos} could mean either the
planetary spheres or the spiritual \quoted{world beyond} them in which God dwelt
with his angels and saints.
That outer realm was the \term{cielo Empyreo}---the Empyrean.
The two concepts match the English terms \quoted{the heavens} versus
\quoted{Heaven}.
% TODO figure of spheres and Empyrean

The villancico begins, then, with an exhortation to the spheres to cease their
music and listen for the new consonance. 
Motive A and its associated styles, then, evoke not the music of the Empyrean,
but the worldly music of the celestial spheres as part of the lower, created
world, and necessarily imperfect in a Neoplatonic system in which only God is
perfect.

This is no empty gesture toward a vague notion of celestial music: the details
of poetry and music reveal specific understandings of music's relationship to
the cosmos in a time when those understandings were beginning to change across
Europe. 
Spaniards in the mid seventeenth century believed in an Earth-centered,
Ptolemaic cosmos, in which the celestial spheres moved at rates and were spaced
at intervals such that they produced harmony, though writers had disagreed since
antiquity on whether this music could actually be heard by human ears (or if
not, why).%
    \citXXX[music of spheres]
In a Neoplatonic conception, the material world reflected the perfection of a
higher plane of reality.%
    \citXXX[Neoplatonism]
What we may miss, however, is the key distinction that this reflection is by
definition imperfect.
Only the Supreme Good for Platonists---the triune God for Christian
Neoplatonists---was perfect; all else fell short in greater degrees as one moved
down the chain of being. 
This is why Boethius's \quoted{three musics} did not describe \emph{all} music,
but rather all music in the material world, the \quoted{world of change and
decay} in Plato's words.
The music of the spheres was still the worldly, imperfect music of change and
decay.
In fact, music served quite well as a symbol of the imperfect created world
because all music known to human ears is ruled by both change---rhythmic,
melodic, harmonic; changes of style from place to place and across time---and
decay, from the dying sound of the lute or vihuela (see \cref{ch:zaragoza}) to
the reverberation of voices and instruments in stone spaces after the breath has
given out.
Worldly music symbolized and embodied the Neoplatonic paradox---it pointed
beyond itself to the highest divine harmony but also was marked by its
difference from that heavenly perfection.

Christian theology added to the Platonic distinction between perfect and
imperfect realms a difference between Creator and creation, and between creation
as envisioned by God and the fallen realm after the original sin of Eve and
Adam, which, as St. Paul said, was \quoted{subjected to futility}.%
\citXXX[Romans etc; Augustine]. 
If death was the curse for eating the forbidden fruit, what was the order of
things in Eden beforehand?
And, to turn to the other end of the Christian timeline, in the Last Day, when
God creates a new heaven and a new earth and fills it with the \quoted{glorified
bodies} of the redeemed, what will be the natural laws of a place in which
people do not die, there is no night or darkness, no seasons or turning---in
short, no change or decay?

Early modern writers on music acknowledged that harmony in worldly music
depended on a controlled balance of consonance and dissonance---just as writers
on the natural world recognized harmony in the cycles of life and death, light
and dark in the world in which \quoted{there is a season for every purpose under
heaven}.%
    \citXXX[Eccl.; Cerone etc.]
Fray Luis described the relationship of the four elements as a \quoted{saber
dance}, and the motion of the planets as a \quoted{concerted music} under the
direction of a divine chapelmaster.%
    \citXXX[Luis]
Luis glorifies God for providing every creature with the means both to provide
for itself and to defend itself. 
The grim prospect this implies of \quoted{nature, red in tooth and claw} does
not seem to have concerned him.%
    \citXXX[Kipling]
In the world \quoted{under the sun}, death was as natural as life---\quoted{the
Lord giveth and the Lord taketh away, blessed be the name of the Lord}.%
    \citXXX[Job]
In the world beyond, different rules must apply, and therefore the music of the
Empyrean must be profoundly unlike any music we know.
David Yearsley has traced the attempts of German Lutheran theologians to make
sense of this paradox, and the efforts of Lutheran musicians to embody an image
of heavenly perfection in sounding music.
Strict counterpoint, especially fugue and canon with inversions, was the most
widely employed type of music or this purpose.
As we have seen, Cererols uses the same trope, particularly in his fugue on
\foreign{contrapunto celestial} and his cantus-firmus motet on the passage about
\foreign{canto llano}.
The relative perfection of the classical contrapuntal style, contrasted againt
the unorthodox dissonances and affective gestures of the other dominant styles,
stands symbolically for the higher realm of divine music relative to human
music.

But Cererols's music for \quoted{the heavens}---that is, the worldly
spheres---is also participating in a Neoplatonic tradition of evoking the
imperfection of sidereal music in the fallen world.
Where the \quoted{higher} motive B is a plainchant-like linear descent, motive A
is a circle, an arc, a palindrome.
Its shape represents the constant turning of the spheres, as its movement
through consonance and dissonance evokes the natural cycles of the world
\quoted{here below}.
The passage on \foreign{la más nueva consonancia}, in particular, evokes
planetary motions through the lilting, dance-like, triple-meter rhythmic feel,
the oscillation between the minor triad on D and the major triad on A (with a
tonic/dominant feel here), and the call-and-response of the voices seeming to
echo back in a reverberant space.

All of this strengthens the central conceit of the villancico: the music of the
spheres is out of tune.
It must halt its ceaseless rotations of consonance and dissonance and listen to
a new kind of music brought into the world by the union of divine and human (the
ultimate \term{musica humana}) in Christ.
The created world can never be perfectly tuned: lurking behind the apparent
perfections of Pythagorean ratios in the overtone series---something that is in
fact physically inscribed in every created thing as a natural law---is the
\quoted{wolf tone} of the Pythagorean comma.%
    \citXXX[temperament symbolism]
The intervals do not add up to perfection.
Worldly music can only be tempered, not tuned perfectly.
The divine music of Christ who will make a new creation at the Last Day will
\quoted{untune the sky}, in Dryden's words, by revealing its decadent
imperfection.
%}}}2

%{{{2 cosmology
\subsection{Cosmology}

Cererols's sonic picture of discordant heavens coordinates closely with the
contemporary musical cosmology of Athanasius Kircher.
Kircher, like most early modern Europeans, believed that the planetary bodies
exerted both positive and negative influences on humanity, as witnessed also by
the contemporary plays of Calderón.%
    \citXXX[exx, Calderón humors lit]
The material world was made of the four elements held in tension, and the body
was moved by a balance and flow among the four humors. 
Likewise, the spheres were arranged not as a chord but at discrete intervals
like a scale, with some intervals consonant and ohters dissonant in relation to
the earth.

In Kircher's cosmology of music, published just a year before the first poem in
this villancico family was printed, the planets are arranged in specific
patterns of consonance and dissonance that are best understood through the
technical details of species counterpoint.%
    \citXXX[on teaching of species cpt in Kircher and his time]
The harmony of the spheres arises from these interactions such that even the
apparently bad (dissonance) is, in an Augustinian line of thinking, actually a
manifestation of a greater good:
\begin{quoting}
	Therefore there is nothing bad in the nature of things, that does not
	also yield to the good for the preservation of the whole universe.
	What else, therefore, are Mars and Saturn, than certian kinds of
	dissonances?
	---which dissonances, in relation to the perfect consonance of Jupiter,
	syncopated and tied in ligatures \addorig{ligata}, resolve not only in
	sweet music but also in the best kind of ornamentation.
	What esle is Mercury if not a certain kind of dissonance syncopated and
	tied between the Moon and Venus, which are like two consonances, so that
	the earth (which is born in freedom and not obligated to anything),
	thanks to the benign influence of the Sun, Venus, and the Moon, should
	not be corrupted.
	Truly, anyone who can consider this on a little higher level would find
	the seven planets to sing continuously in perfect, perpetual four-part
	polyphony \addorig{tetraphoniam}, in which the dissonances and
	consonances thereof are brought together, so that they should resolve in
	the most comely harmony of the world.%
	\citXXX[KircherII:383--384]
\end{quoting}
Kircher acknowledges, then, not only that the planets influence earth, but also
that they influence each other: their motions must be understood relative to
each other, and they are part of a dynamic system held in perpetual balance by
the interaction of these attractions and repulsions.

Contrapuntal rules for Kircher provide a way to understand the hidden forces
that animated the universe.%
    \citXXX[Gouk et al]
Thirty years later Isaac Newton would provide a coherent explanation of these
hidden forces of attraction through his laws of motion and the concept of
gravity.
Newton, like Kepler before him, was seriously concerned about the implications
of his observations for concepts of heavenly harmony.
Kepler, a Lutheran minister, had rejected his own theory after all because he
could not believe that God had made the world so unharmonious.%
    \citXXX[Hawking?]
Kircher actually cites Kepler's table of planetary distances in order to
reject it for the same reason.%
    \citXXX[Kircher]
Kircher goes so far as to provide a \soCalled{corrected} table with more
harmonious figures.
But even though both Kepler and Kircher were troubled by the untuneful empirical
observations of the planets, Neoplatonism already provided a solution for
Kircher: if the world was perfectly tuned, then this certainly reflected the
order of God; but if it turned out to be untempered, then this imperfection,
too, could be understood to point toward a higher perfection.
Dissonance in the heavens, then, was no cause for abandoning faith in the old
cosmos or its Creator.
Rather, it had to be understood in its proper place as part of the created world
with its cycles of birth and death, light and dark.

Kircher symbolizes his conception of the spheres in an example of actual music,
which is not meant to \emph{sound} like the harmony of the spheres but rather to
encode their relationships through musical technique:
\quoted{so that the curious reader should have a certain example of the
celestial polyphony, this can be seen demonstrated in musical notes according to
our speculative idea} (\cref{mux:Kircher-Tetraphonium_coeleste}).%
\citXXX[Ibid 384]
% TODO example
Kircher provides a detailed analysis of the example that unites contrapuntal and
astrological theory (\cref{tab:kircher-tetraphonium_coeleste}).
(He has been developing an allegory throughout the last book of
his treatise based on the four strings of the Greek lyre, whose
names he uses for each voice part.)
\begin{quoting}
	In the example Saturn, Jupiter, and Mars form the \term{netodum}, that
	is, they sing the highest voice, in which notes the consonant Jupiter
	always unites in harmony \addorig{ligat} and undoes the influence of
	\addorig{confringit} the dissonant Mars and Saturn.
	The Sun proceeds truly as the \term{mesodum} \add{Alto}, singing in
	perfect consonances, looking at the earth, the \term{proslambanomenon}
	\add{bass} from the octave above, or an octave and a fifth.
	Venus, Mercury, adn the Moon truly sing the \term{hypatodon}
	(\add{Tenor}), and Venus and the Moon, which are consonant, carrying
	Mercury in the friendship beteen them as a dissonant passing tone
	\addorig{intermedium dissonum}, thereby tie him up in harmonic
	intervals \addorig{modulis}, so that they absolutely restore
	consonances, as can be seen there in the notes of the Tenor part.
	The Earth truly receives frrom the substance of all these, therefore,
	the perfect mixture of consonances and dissonances, so that it
        constitutes the most perfect music with the planets, which we can
        imagine by using this musical example.%
	\citXXX[Kircher II:383-384]
\end{quoting}
In this way Kircher attempts to unite the ancient Greek concept of the planets
as notes in a scale (as also in Kepler) with an emerging early modern concept of
polyphonic harmony.
In the Soprano and Tenor voices, each pitch stands for one planet, and their
consonance or dissonance relative to the bass and the melodic motions linking
them, symbolize the planets' influence on Earth and each other.%
	\footnote{The clef of the Tenor system should be a Tenor clef.}
Venus prepares the dissonance of Mercury, for example, and the Moon resolves it.
The Alto and Bass, though, each represent a single celestial body, and the
symbolism is not as exact.
(The Earth is the Bass because it was unmovable, but the bass voice here also
moves.)
In terms of species counterpoint, the Alto (Sun) and Bass (Earth) move in
perfect first-species (note-for-note, all consonant) counterpoint.%
    \citXXX[Zarlino? for species cpt]
The Tenor (inner planets) is composed in the fourth species (ligatures and
suspensions), while the Soprano uses the second species ($2:1$).
% TODO table from diss. p. 264

%{{{4 music, table Kircher tetraphonium
\insertMusic{Kircher-Tetraphonium_coeleste}
{Kircher, \ptitle{Tetraphonium coeleste ex planetarum corporibus constitutum},
four-part cadence of the planets, from \wtitle{Musurgia universalis}, bk. 2,
383}

\insertTable{kircher-tetraphonium_coeleste}
{Contrapuntal and planetary relationships in Kircher, \ptitle{Tetraphonium
coeleste}}
%}}}4

Kircher shows here both that the heavens could be understood in musical terms
and that music could be understood in heavenly terms.
Kircher's \term{clausula} or cadence of the planets demonstrates that
Neoplatonic thought about music did not always begin with theory and descend to
practice; it also used contemporary \term{musica instrumentalis} as a specific
model or metaphor for the higher conceptions of music on the cosmic level. 
People comparing the heavens to music had real music in their ears.
Kircher's \term{clausula} sounds like a perfectly ordinary seventeenth-century
cadence (in mode I, transposed to \term{cantus mollis}), but for Kircher even
the \quoted{mundane} details of the counterpoint such as passing tones and
suspensions had high symbolic potential.
This example suggests that composers and educated listeners thought symbolically
in Neoplatonic terms even about the basic fabric of their compositions.

Cererols's villancico exemplifies this close link between musical practice and
conceptions of the heavens.
Cererols's poetic text may even recall this specific passage when it says Christ
will be a \foreign{divina cláusula}; even if not it is a similar attempt to
express the heavens through polyphonic harmony.
Kircher's detailed symbolism might prompt us to look more closely at the precise
contrapuntal relationships in Cererols's depiction of the heavenly.
As already noted, Cererols juxtaposes two contrasting styles to represent a
contrast between the untempered music ofthe material world and the higher music
of the divine.
He uses learned counterpoint with strictly controlled dissonance and a more
linear rather than harmonic conception to represent the higher music. 
Cererols represents worldly music (which includes the spheres of the
\quoted{heavens} invoked at the outset) with a more harmonically conceived,
dissonance-laden style.

In his use of a dissonance for the word \foreign{consonancia}, though, Cererols
takes a different approach from Kircher.
Kircher's cadence is, as he says, \quoted{perfect \Dots{} polyphony}, following
contrapuntal rules exactly.
Cererols, by contrast, breaks the rules with these unprepared dissonances.
On one level, the dissonance in these passages seems to symbolize the
imperfection of the worldly order.
But the dissonance on \foreign{la más nueva consonancia} also functions as an
ironic symbol, pointing to a higher kind of music whose rules defy human
understanding.
Cererols borrows modern style as if to say that dissonance is the new
consonance.

Kircher make a similar gesture in a different passage, where he describes the
music of heaven as beyond human imagining, and resorts to paradox to evoke it.
In the last book of the \wtitle{Musurgia} Kircher presents the whole creation as
a \quoted{Praeludium} played by God on a divine organ of creation.%
    \citXXX[Kircher, Godwin]
%TODO figure
In the engraving, the organ is depicted in exacting detail as a real instrument:
there is a pipe shown for every key of every rank.
The odd arrangement of the keyboard, though, would catch a reader's attention
and require explanation.
% TODO figure
The keys are arranged in groups of seven rather than twelve chromatic pitches in
octaves, with repeating groups of three black keys instead of the three-and-two
pattern of earthly keyboards. 
The seven-key groups represent the days of creation, and the three black keys,
it would seem, the Holy Trinity. 
Perhaps the keys correspond to a diatonic series, so that each seven-note group
forms a diatonic octave (and would map directly onto the planets as well).
Whatever the arrangement, this keyboard is not designed for playing earthly
music.
The Latin motto beneath the keys reads, \quoted{Thus does the eternal wisdom of
God play upon the spheres of the worlds}.%
\citXXX[Apocrypha?]
To imagine playing an actual Praeludium (say, by Kircher's countryman Buxtehude
or his neighbor Frescobaldi) on this organ is to contemplate what Olivier
Messiaen would later call \quoted{the charm of impossibilities}.%
    \citXXX[music exx, Messiaen quote]
The image is a riddle that points to a divine music, governed by different rules
that that of music in the lower world.

Cererols, then, could be presenting his hearers with an auditory symbol of this
impossible music.
By pointing out the imperfect artifice of the music itself, Cererols prompts
listeners to reflect on how the imperfect reflects God's perfection.
In the theological context of this Christmas villancico, the \quoted{newest
consonance} is, of course, Christ himself.
As in Gutiérrez de Padilla's villancico, this piece makes Christ the
\term{Verbum infans}, the Word made flesh as an unspeaking infant.
The \quoted{sighs} of the baby, referred to several times in the text, are the
\quoted{new song} that becomes the \term{cantus firmus} of a renewed creation.
Through Cererols's interplay of motives and styles, listeners can actually hear
the music of human and divine emerging over the course of the piece, culminating
in the evocation of a motet based on motive B and ending with a \term{peroratio}
(final flourish) based on motive A.

It is even possible to read the whole estribillo as following a similar
rhetorical structure to that of contemporary organ praeludia, and this makes
sense since the poem is an exhortation of the heavens.%
    \citXXX[rhetorical analysis]
The piece follows the Quintilian pattern of \term{exordium}, \term{propositio},
\term{narratio} (\term{confutatio/confirmatio} pairs), and \term{peroratio}.
% TODO details
If the villancico is an oration, then its main subject is Christ as \term{Verbum
infans}.
Neither motive or style solely represents Christ's voice; rather the theme is
the paradoxical mixture of divine and human in the Incarnation.
In Christ God entered the world of change and decay \quoted{to bring consonance
to the clay}.
Christ, particularly in his Passion and then in the Eucharist, was both
consonance and dissonance, old and new, material and spiritual.
%}}}2

%}}}1

%{{{1 sec: genealogy
\section{Genealogy}

If this villancico by Cererols were the only one of its kind it might register
as an interesting footnote, but in fact this setting is part of the
best-attested family of villancicos yet discovered from the seventeenth century.
Evidence survives for eight other settings of this poem in several variant
textual families, from its first known appearance in Madrid at Christmas 1651
through performances in Toledo, Zaragoza, Seville, and a fragmentary setting
from a convent in Ecuador. 
The composers of most of these settings were closely linked in a web of personal
affiliations, so that their choice to set a text (or their own local variant of
a text) previously set by a teacher, colleague, or rival enabled them to situate
themselves within a tradition of both composition---as they demonstrated their
prowess at musical \term{conceptismo}---and theology, as they continued to
develop tropes of heavenly music in the midst of changing understandings of the
cosmos.

%{{{4 table Suspended sources
\insertTable{suspended-sources}
{Sources of \wtitle{Ha de los coros/Suspended cielos}: Poetry Imprints and
Musical Settings}
%}}}4

As was the case for Gutiérrez de Padilla's \wtitle{Voces}, the earliest
surviving versions of the \wtitle{Suspended, cielos} family suggest an earlier
common source, as yet undiscovered.
The \wtitle{Suspended, cielos} estribillo is the most consistent element across
the versions.
The earliest imprint from the Royal Chapel in Madrid 1651 lacks a line that,
despite other variations, is present in six of the other sources, and which
makes metrical and poetic sense.
An earlier source, with this missing line, would have the form shown in
\cref{tab:suspended-original}.
All of the versions except the highly abridged S10 are framed by eleven-syllable
lines (\poemlines{1, 16}).
After the \term{exordium}-like opening line, there follow three quatrains, each
in an \term{abbc} rhyme scheme where the final lines of all three stanzas rhyme
with each other.
The estribillo's conclusion is sealed with the \term{lira}-like three lines with
a 7--7--11 pattern of syllable counts.
The main variants constitute omitting one or two of these quatrains or portions
of them.
The Cererols version omits the second.
The compositors of these texts, whether chapelmasters or others, may have had
several motives for revision: for one, omitting the references to Matins 
\quoted{tonight} and the stable served to make the piece more adaptable for
general purposes; additionally, as polyphonic settings of villancicos became
more complex and included longer sections there were practical advantages to
shortening the text.

%{{{4 table Suspended original
\insertTable{suspended-original}
{Reconstructed original text of \wtitle{Suspended, cielos} estribillo}
%}}}4

The two earliest sources precede this section with a \term{romance} of notable
elegance, beginning \foreign{Ha de los coros del cielo}
(\cref{poem:Ha_de_los_coros}).%
    \footnote{This demonstrates the importance of cataloging more than just
    single incipits.}
These stanzas set the scene for the rest of the villancico and clarify its
central musical conceit.
They clearly address not Heaven (\term{el cielo Empyreo}) but the heavens, the
planetary spheres.
The fourth stanza goes beyond puns on musical terms to paint a dramatic moment
of performance.
The \quoted{sapphires}, scattered across the sky like notes on a \quoted{starry
notebook}, must \quoted{make a rest} at the sound of \quoted{the newest
consonance}.
As though looking up in wonder at this new music, higher even than them, the
astral musicians let their sheets of music fall closed.%
\begin{Footnote}
    This suggests another allusion to the Last Day, when \quoted{the skies will
    be rolled up like a scroll} at Christ's return; this again matches Dryden's
    conceit.
\end{Footnote}
Like the painting of the Monsterrat Escolania beneath the Black Virgin, the
musical terms here are specific to the performance practice of villancicos as
opposed to Latin-texted polyphony.
The spheres are singing from \foreign{cuadernos}---partbooks, not a
choirbook---made up of one or two folded sheets of paper (\foreign{hojas
dobladas}) gathered together in a folder (\foreign{cartapacio}).

%{{{4 poem Ha de los coros
\insertPoem{Ha_de_los_coros}
{\wtitle{Ha de los coros del cielo}, opening \term{romance} in \term{Suspended,
cielos} tradition, earliest version (S1) (roman numerals are indications of
different speakers in original source)}
%}}}4

Based on the variants of this romance and the coplas we may group the
texts in two families (\cref{tab:suspended-families}).
Family A starts with the romance, \wtitle{Ha de los coros de cielo},
followed by the estribillo, \wtitle{Suspended, cielos, vuestro dulce canto},
then several coplas.
S2 has a unique set of coplas; the others all start with \foreign{Las fugas del
primer hombre}.

%{{{4 table Suspended families
%\insertTable{suspended-families}
%{Sources and families of texts in the \wtitle{Suspended, cielos} tradition}
%}}}4

Family B descends from a recension by Manuel de León Marchante, who somewhat
simplified the earlier version.%
    \citXXX[on Marchante]
Family B moves all or part of the romance to the coplas in various ways.
These texts share other variant readings that mark their descent from a common
source (\cref{tab:suspended-variants-estribillo, 
tab:suspended-variants-coplas}).
This change demonstrates a transition in the later seventeenth century away from
using introductory romance sections and toward beginning directly with
the estribillo.%
    \citXXX[Torrente:estribillo]
In many of the same ways that \wtitle{Cantores de la capilla} simplified
\wtitle{Voces, las de la capilla}, León Marchante reduces the complexity and
ambiguity of the earlier versions at the expense of some nuances of the
theological and musical conceits.
He omits the most grammatically and poetically difficult copla of the 1651
source (\foreign{Qué mucho si a los despeños}) along with three others.
He changes the first word of \foreign{siglos de los astros} (centuries/ages of
the stars) to \foreign{signos} (signs); where \foreign{siglos} (also used in
\wtitle{Voces}) referred obliquely both to time and planetary motion,
\foreign{signos} is more clearly a musical term that still has a planetary
double meaning.
León Marchante also changes, for example, the last word of \foreign{En las pajas
subtenido} to \foreign{sustenido}, substituting a clear musical play on words
(Christ \quoted{sustained} on the bed of straw/Christ singing a \quoted{sharp})
for the more arcane geometrical concept of a line that joins together two
extremes of an arc, as Christ joins together eternal and temporal.

%{{{4 tables Suspended variants
\insertTable{suspended-variants-estribillo}
{Variants in the estribillo of \worktitle{Suspended, cielos} villancicos,
compared with reconstructed original source (\cref{tab:suspended-original})}

\insertTable{suspended-variants-coplas}
{Variants in the romance and coplas of \worktitle{Suspended, cielos}
villancicos}
%}}}4

Sources 4--6 were all produced in a three-year period by composers (cited by
name in the imprints) who were all personally acquianted with each other and
regularly exchanged villancico texts.
León Marchante wrote or edited a large number of texts for Toledo Cathedral,
where they were set to music by chapelmaster Pedro de Ardanaz.%
    \citXXX[exx, posthumous source; on Ardanaz and Toledo]
In Ardanaz's correspondence with Miguel de Irízar, chapelmaster in Segovia and
fellow student of Tomás Miciezes the elder, Ardanaz mentions also sending texts
to Diego de Cáseda, another fellow pupil who was chapelmaster in Zaragoza.%
    \citXXX[Irizar, Miciezes, Caseda]
Ardanaz probably also shared the texts with another Miciezes alumnus, Alonso
Xuárez (Suárez) in Seville (S4).%
    \citXXX[Xuarez]
After Xuárez's version at Seville Cathedral for Christmas 1680, S5 was performed
a year later at Seville's second most important church-music institution, the
church of San Salvador, in a closely related but abridged text with only five
coplas.
The composer was Miguel Mateo de Dallo y Lana.%
    \citXXX[Dallo] % TODO dates for all these composers
S6 was sung twelve days later for Epiphany 1683 in Zaragoza in a setting by
Cáseda.

Dallo y Lana later emigrated to Puebla and served as chapelmaster of Puebla
Cathedral in \XXX[years].%
    \citXXX[Puebla]
The collection of Puebla's Conceptionist Convento de la Santísima Trinidad
includes numerous villancicos by Dallo y Lana and both Diego de Cáseda and his
son and successor José; Dallo y Lana may be responsible for bringing these
scores to the New World personally or through correspondence, but in any case
the collection demonstrates that this was a global network of affiliated
composers.%
    \citXXX[SanchezGarza catalog]

The 1689 Royal Chapel version returns to the original source in a similar manner
to Gutíerrez de Padilla turning away from \wtitle{Cantores} back to
\wtitle{Voces}.
Some of the chapel musicians probably remembered the 1651, performance and their
leader at that time, interim directory Juan de Navas, likely had access to the
original parts and imprint in the archive.%
    \citXXX[Royal chapel, Navas]
The text is closer to family A but includes some of León Marchante's
revisions from family B.

The two musical manuscripts by Cererols belong in family A, even though they
lack the romance, because they do not have any of León Marchante's
alterations.
This suggests that this version of the text was arranged around 1660, before
León Marchante adapted it.
The transmission of these two families demonstrates a widespread fascination
with the themes of celestial music and the possibilities of metamusical
composition, as well as revealing a web of affiliated musicians across the
peninsula.

%{{{2 ecuador
\subsection{Ecuador}

This network extended to South America as well: Source 10 resides in the remote
parish archive of Ibarra, Ecuador, among a collection originally from the
Conceptionist convent there.
It is a fragment of an anonymous eleven-voice setting, in three choirs, based on
a variant of this textual tradition, with only the third-chorus voices of the
estribillo.
The music may have been composed by one of the chapelmasters of Quito Cathedral
and adapted by the convent sisters for their own use.%
    \citXXX[source, info, names, thanks to Cesar]
The manuscripts demonstrate that the piece was frequently performed and
readapted for different occasions.
The Tiple and Tenor parts bear the names of performers, \quoted{Sra. S. Martin}
and \quoted{Sra. S. Seçilia} respectively.
There is a second copy of the Alto part, copied in a different hand, with the
indication for the instrument \term{dulzaína}. %XXX what is?

%TODO text and music floats

The text, as far as can be determined, has been made much more general so that
it could be used for almost any occasion.
Additional lyrics have been added in a different hand above the first line of
lyric text. %XXX details
The text still mentions \foreign{planetas} but retains none of the themes of the
other sources.
Going farther than the variant Cererols manuscript (which changed \foreign{niño}
to \foreign{hombre} for a Eucharistic purpose), the compositor has changed
\foreign{niño} to \foreign{santo} (saint), allowing use for general sanctoral
devotion.
The music depicts \quoted{stopping} and \quoted{listening} with halting
entrances set apart with rests (probably echoes of the other choruses).
Like Cererols, this composer uses plainchant-like stepwise lines on \foreign{el
canto llano}.
Even without the rest of the piece it is clear that these tropes are still being
developed from the earlier tradition.
The piece now emphasizes \quoted{faith and devotion} over cosmology, symbolism,
and Neoplatonic contemplation, in a shift that may be representatitve of a
broader change in orientation way from speculative and symbolic thinking and
more toward a feeling-based personal piety around the turn of the eighteenth
century.

%}}}2
%}}}1

%{{{1 sec: conclusion
\section{Conclusion}

The poetic tradition of \wtitle{Suspended, cielos} villancicos manifests a
widespread cultural fascination with heavenly music, and with the use of music
to represent itself.
The genealogy of these pieces, passing from hand to hand among interrelated
musicians, strengthens the argument that the metamusical villancico subgenre did
function as a favored way for these composers to prove their craftsmanship and
establish a place in the musical community.
The prevalence of a text so heavily dependent on Ptolemaic and
Neoplatonic-Boethian musical cosmology challenges Hollander's hypothesis of
secularization during the seventeenth century, at least for Catholic Spain.
At the same time, the Cererols setting, with its ambiguous symbolic use of
dissonance and contrasting musical styles, does suggest a certain anxiety about
the relation of earthly and heavenly music that seems distinct to this period.

Cererols represents a Neoplatonic hierarchy of musical styles, and points to
higher levels on the chain of being (\quoted{in the distance}) in which there
would a new kind of music altogether.
His metamusical representations of earthly music's imperfections (all within, of
course, a single earthly performance) should not, however, be considered as a
condemnation of human music.
Quite on the contrary, to contemporary ears some of the most striking and
affectively expressive passages in the piece are those in the more modern
style which we have suggested represents human music.  
Within one piece Cererols includes music that lends itself to a contemplative
function (fugue, cantus-firmus motet), and affective music that expresses,
perhaps, the anguished \quoted{sobs} of the human world into which Christ was
born.

This piece manifests a latent problem within musical thought and practice (and
theological thought) in the early modern period: To what degree does the order
of nature reflect the perfection of God?
Is dissonance a necessary element of beautiful music, or is it a sign that
something is wrong with the system from its root?  
Does earthly music reflect the divine or express the human?
The development of the \term{Suspended, cielos} tradition, and the related
pieces studied in the other chapters, suggests that there was a gradual shift in
the function of Spanish villancicos about music from reflecting divine
perfection toward expressing human affects, away from emphasizing how worldly
music is like heavenly music and toward stressing its imperfections and
difference from higher forms.
This piece by Cererols stands at a midpoint where dissonance may function as a
symbol through which listeners may contemplate the imperfection and sin of the
fallen world; and as an ironic or paradoxical symbol that urges them to imagine
a higher form of music. 
At the same time dissonance is beginning to serve a positive aesthetic purpose
to elicit an affective response from listeners.
Later composers, as we will see, pushed the artifice even farther, going to more
extravagant ends to demonstrate the out-of-tuneness and imperfection of earthly
music.

The adaptations of this text for other purposes (the Eucharistic Cererols
variant; the two alternate texts of the Ecuador manuscript) demonstrates the
appeal that this poetic conceit offered, but also suggests that as the century
progressed people began to see it as simply an attractive poetic conceit,
missing the intricate theological complexity of the earlier sources.
The eloquent exposition of Neoplatonic musical theology in the poetry imprints
and the Cererols setting is watered down in the Ecuador manuscript to a series
of vague conventions.
This trend corroborates part of Hollander's thesis, showing a process of
conventionalization---but not necessarily one of secularization.  
Though the Ecuador text does not depend on the details of any particular
cosmology, that does not mean the authors or performers had ceased to believe in
the old system.
Besides, from only the third-chorus parts it is impossible to guess at the full
content of the villancico text.  And even if the musical theology of the Ecuador
setting is relatively vague compared to the rest of the tradition, the original
piece still calls for no less than eleven voices to sing out an exhortation to
the planets.
The complicated conceits of this poetic tradition as seen in the earliest
sources may have fallen out of fashion, but the fundamental theological
framework underlying this poetic tradition still continued to provide meaning.
While Isaac Newton's new physics would soon illuminate the real structure of the
solar system, in the years just after the publication of the \wtitle{Principia
mathematica} in 1680, there were four separate performances of a
\wtitle{Suspended, cielos} villancico in the Spanish Empire, where the old
earth-centered worldview blazed its brightest just before sunset.
%}}}1

\endinput

