% Faith, Hearing, and the Power of Music
% in Devotional Music of the Spanish Empire
% 
% Andrew A. Cashner
% 
% Chapter 4, Heavenly Dissonance
% (Cererols, *Suspended, cielos*)
%
% 2015-03     Dissertation defended
% 2018-04-13  New version begun for book
% 2018-05-22  Converted to LaTeX
% 2018-06-12  New version resumed

\chapter{Heavenly Dissonance (Montserrat, 1660s)}
\label{ch:cererols-suspended}

The Christmas villancico \wtitle{Suspended, cielos, vuestro dulce canto}, set by
Joan Cererols for the school choir of the Abbey of Montserrat around 1660, is
part of one of the most widely-attested families of villancicos, with evidence
surviving for seven other versions, from the Royal Chapel in Madrid to a convent
in Ecuador.
In the setting by Cererols, the chorus exhorts the heavens to cease the music of
the spheres and listen for \quoted{the newest consonance} that has come into the
world through the newborn Christ-child.
The piece thus extends the \emph{Verbum infans} trope discussed in chapter
\ref{ch:padilla-voces} with a greater focus on cosmology, creating a reflection
on the relationship between earthly and heavenly music.
Rather than the word-level text depiction of the madrigal-like \wtitle{Voces} by
Juan Gutiérrez de Padilla, Cererols evokes the contrast between levels of music
by a corresponding contrast of musical motives and styles.
Together with this polystylistic technique, Cererols also uses dissonance as an
ironic symbol to point to a higher kind of heavenly music, not governed by the
same rules as earthly music.
By the end of the piece, attentive listeners can actually hear the motive
representing Christ's voice become the \emph{cantus firmus} of a new heavenly
music; but within the Neoplatonic listening tradition, they are challenged to
listen beyond this depiction of heavenly music for the higher, unhearable music
of Christ himself.
The many versions of this text were passed from hand to hand among a network of
affiliated composers and modified to suit both local needs and changing
conceptions of heavenly music; tracing these connections provides further
evidence that composers used the metamusical villancico to prove their craft and
situate themselves in a lineage of composition.

The basic trope of this villancico is the same as that of Padilla's
\wtitle{Voces, las de la capilla}---it invites people to adore the Christ-child
as \emph{Verbum infans}, the unspeaking/infant Word.
This poem is more explicit in presenting Christ's voice---as an expression of
Christ himself---as a form of music.
The heavenly spheres with their sidereal harmonies are out of tune, imperfect
creations compared to their Creator and further \quoted{subjected to futility}
(\scripture{Rm}{\XXX}) because of human sin.
Christ's voice will form the \term{cantus firmus} on which the music of a
renewed creation will be based.

The musical setting by Cererols is remarkable because the composers enables
listeners to actually hear the contrast between untuneful, dissonance-laden
music evocative of Plato's \quoted{world of change and decay} and the perfect
consonances and logical structures of a higher, heavenly music.
Moreover, the discerning listener can hear a musical process analogous to the
theological conceit of music in the poem---namely, that Christ's voice becomes
the \quoted{plain chant} that forms the basis of the \quoted{new song} of the
new heavens and earth.
The piece thus serves as a kind of school for Neoplatonic listening practice,
one that disciplined hearers to distinguish between the ways human music was
like heavenly truth and the ways it was not.
It encouraged them to listen for an unhearable higher music, ultimately that of
Christ himself in the triune Godhead.

In addition to its significance as the sole complete surviving setting of a
widely attested villancico text that was performed in different versions around
the world, Cererols's setting also establishes conventions for representing
heavenly and earthly music that numerous other composers in the same tradition
continued to develop throughout the seventeenth century. 
It is one link in a chain of related poems and compositions about heavenly music
in the Hispanic world that reflect common attitudes about the relationship
between human, cosmic, heavenly, and divine music---attitudes that were under
pressure to change amid the Scientific Revolution.

The villancico family, and especially Cererols's setting, open a window into how
early modern Catholics understood heavenly music in a time when conceptions of
the cosmos were changing all around them.
John Hollander's study of English poetry on music, \wtitle{The Untuning of the
Sky}, takes its title from a poem by John Dryden which envisions the apocalypse,
when, at the sound of the last trumpet, \quoted{MUSICK shall untune the sky}.
Hollander uses the title to refer to a process he sees of increasing
secularization and disenchantment across the seventeenth and eighteenth
centuries, so that by the end of the period poets use the language of heavenly
harmony in a purely conventionalized way devoid of real connection to any faith
in the old conceptions of the cosmos.
But Dryden's poem, very much like Cererols's villancico, actually affirms
traditional views: it emphasizes the untunefulness of worldly music (including
the heavenly spheres) compared to the higher music of God.
In a Neoplatonic tradition, it was normal to emphasize the deficiencies of the
\quoted{world of change and decay} and point beyond them to a higher truth.
In this study of Spanish poetry on music, we can note a change of
understandings, but not one that moves in the direction of secularization.
Instead, poets and composers---and devotees as well, we may imagine---find new
ways to reconcile the old cosmos with changing understandings and sensibilities.

\section{Cererols and the Boys' School Choir of Montserrat}

The only known complete musical setting is preserved today in a manuscript of
the parrochial archive of the church of Saints Peter and Paul in Canet de Mar, a
village on the seaside about thirty miles north of Barcelona.%
\begin{Footnote}
    Grateful acknowledgments are due to the rector and archive director for
    making a digital image of the manuscript freely available for this study.
    The Canet archive, under supervision of the archdiocese of Gerona, is being
    fully digitized.
\end{Footnote}
The undated manuscript attributes the music to Joan Cererols (1618--1680), who
was a monk and director of the school choir (Escolania) at the Abbey of Our Lady
of Montserrat.
The mountaintop monastery of Montserrat continues to be an influential
religious and musical center in the region. 
Much has changed since on the mountain since the time of Cererols, especially
because Napoleon's troops burned the monastery, including its library, when they
invaded Spain in 1811.
But pilgrims still make the long journey to the peak to revere the statue of the
Black Virgin of Montserrat, patroness and symbol of Catalonia.
They petition the Virgin's intercession in her shrine, ensconced in a special
chapel above the high altar of the abbey church, which is filled daily for
performances of the renowned boys' choir of the Escolania, the oldest
continuously operated choral school in Europe.

The nineteenth-century destruction is the reason why there are no sources of
Cererols's music original to Montserrat.
They all survive in copies made by students, such as the one taken to Canet de
Mar, and the sizable collection today in Barcelona's Biblioteca de Catalunya.
Even without the original collection at Montserrat, for the first modern edition
of music by Cererols in the 1930s, Dom David Pujols of Montserrat assembled no
less than seventy-eight manuscripts.
The thirty-five extant villancicos outnumber the surviving output of many of
Cererols's contemporaries who may be better known today.\XXX[who?]

Joan Pau Cererols Fornell was baptized in 1618 in the village of Martorell, in
the shadow of Montserrat.%
    \Autocite{Balanza:CererolsFamily}
He was the youngest child of Jaume Cererols, a well-to-do tailor.
His mother died when he was ten, and it appears that only a few months later
Joan was sent off to boarding school as a chorister at the Escolania of
Montserrat.%
    \Autocite{Balanza:CererolsFamily}
In the Escolania he would have received a thorough training in performance,
practical music theory, Latin, and the other typical subjects taught in church
schools.
After graduating Cererols entered the novitiate of the Montserrat Benedictines
at age eighteen, in 1636.
He remained at the monastery until his death in 1680, having become chapelmaster
of the Escola and teacher in the Escolania, as well as serving as sacristan of
the abbey church.

A monastery chronicle (probably written several decades after Cererols's death)
says that Cererols was \quoted{Chapelmaster and master of the choir-school boys
for more than thirty years}, having \quoted{left behind many writted \add{i.e.,
manuscript} books of music}.
Moreover, Cererols was \quoted{an excellent poet}, learned in letters and
theology, and able \quoted{to speak Latin as fluently as if it were his mother
tongue}.%
    \Autocite
    [{\XXX[original]}]
    [7, note 2]
    {Estrada:CererolsBio}
This description fits well with the evidence of \wtitle{Suspended, cielos},
which pairs sophisitcated poetry and elegantly masterful musical technique.
Cererols's version includes several variant poetic readings that do not appear
in any of the other seven sources, and most of these appear to be deliberate
changes directed by a keend theological and literary intellect, as will be shown
below.

The chroncle stresses the influence of Cererols as a teacher in a line of great
teachers:
\begin{quoting}
    He had the gift and talent of teaching and thus had so many students that
    there was hardly a church in this principality \add{Catalonia} whose
    Chapelmasters and Organists were not his students, aside from the many
    others that he had in other provinces of Spain, all of whom manifested the
    excellent qualities of their Teacher, as the reverend Father himself also
    demonstrated those of Father Master Márquez, of whom he \add{Cererols} was a
    student.
\end{quoting}
Indeed, the textual history of \wtitle{Suspended, cielos} shows that Cererols
was part of a network of musican influence and exchange that spread across Spain
and the New World. 
Cererols may have drawn some of his own influences from a stay of several years
in Madrid, after the religious orders of Catalonia fled there from the 1640
Catalan revolt.
Cererols's time in Madrid coincided with the flourishing of new musical styles
and forms at the royla court, led by the composers of the Royal Chapel,
chapelmaster Carlos Patiño and court harpist and composer Juan Hidalgo.
This means that Cererols was probably in Madrid when the first known version of
\wtitle{Suspended, cielos} was performed by the Royal Chapel on Christmas Eve
1651, and likely had direct access to the commemorative poetry imprint, if not
the original performance.
Cererols's poetic text is quite close to this earliest imprint.
Additionally, the influence of new styles from the Madrid court may be heard in
Cererols's music.
As noted, the large number of surviving copies of his music from outside
Montserrrat attests to his influence.

It may have been one of his students who brought the setting of
\wtitle{Suspended, cielos}, along with several other Cererols compositions, to
the church of Saints Peter and Paul in Canet de Mar.%
    \footnote{\sig{E-CAN}{AU-0116}.}
This sources preserves a complete setting of estribillo and six coplas scores
for an eight-voice double-choir ensemble with continuo accompaniment.
The estribillo features the whole chorus in a variety of textures, while the
coplas alternate between a Tiple duet and the four voices of Chorus I (likely
soloists), both with accompaniment.
The performing parts are covered in layers of fingerprints on the fold of each
sheet of paper, indicating many years of performance.
This extended composition demands a virtuoso ensemble, and is thus more likely
to have been composed for the school choir at Monsterrat and brought to Canet as
part of a personal collection, unless the ensemble at Canet was extraordinarily
capable for a small parish church.

In addition to this Canet manuscript, there is another source for this music, a
previously unattributed set of manuscript performing parts in Barcelona's
Biblioteca de Catalunya.%
    \footnote{\sig{E-Bbc}{M/765/25}.}
The Barcelona source includes only the estribillo, with music almost identical
to that in the Canet source, even in most details of coloration and accidentals,
with one significant variant---a  high ending for the first treble.
This alternate version is missing its original Tenor I and accompaniment parts
and lacks any setting of the coplas.
It adds the dynamic markings \term{eco} and \term{falsete} in repeated phrases
of polychoral dialogue. 
These terms became common in Hispanic villancicos after about 1660 and probably
reflect an attempt to make the piece suit changing aesthetics later in the
century, but they could possibly record they copyist's memory of a performance
tradition at Monsterrat.

\wtitle{Suspended, cielos} was originally a villancico for Christmas, as
evidenced by all the surviving poetry imprints and the contents of the text
itself.
In the Barcelona version, however, one verse of the poem has been modified to
suit a Eucharistic dedication instead of Christmas: \emph{y con sollozos
tiernos/ un niño soberano} (and with tender sobs,/ a sovereign child) becomes
\emph{y desde un pan divino/ un hombre soberano} (and through divine bread,/ a
sovereign man).
Not one of the seven poetry imprints from this tradition includes these altered
lines, but instead agree with the Canet version for Christmas.
Confusingly, the Canet manuscript actually includes the label, in a different
hand on the coverleaf, designating the piece for \quoted{the blessed
Sacrament}---but it is clearly a Christmas piece.
The meaning of the difficult poem might have escaped the grasp of a later
archivist who was seeking to quickly categorize the piece.

In the archive the piece is grouped with several other pieces by Cererols,
including one (\wtitle{Pues que para la sepultura}) that remarkably incluides
both the composer's score and the performing parts.%
    \footnote{\sig{E-Bbc}{M/765/14}.}
This can only have originated with the composer himself, and must have been
passed on through a chain of musicians connecting back to Cererols.
This archival signature may document Cererols's musical network and influence.
It includes a different setting of \wtitle{Pues que para la sepultura} by Diego
de Cáseda, chapelmaster in Zaragoza and composer of his own lost setting of
\wtitle{Suspended, cielos}, known from the poetry imprint.
It also includes a work by Miguel Ambiela, a later occupant of the same post in
Zaragoza (see \cref{ch:zaragoza}).

One musician who would seem a likely candidate for carrying the Cererols legacy
to Barcelona and perhaps this specific manuscript, is the organist Gabriel
Manalt (1657--1687).
Manalt was baptized in Cererols's own home town of Martorell.
In such a small village, Manalt must have known the locally prominent Cererols
family, and it seems likely he was a student at the Escolania while Cererols was
chapelmaster.
Manalt was organist at the church of Santa María del Mar in Barcelona from 1679,
according to records of his audition (\term{oposición}), until his death.
He also served as interim chapelmaster from August 2 to September 26, 1685.%
    \Autocite[70--71]{Balanza:CererolsFamily}
In the notice of his burial at his home church in Martorell, Manalt was praised
as \quoted{a man highly accomplished in the art of playing the organ, and unique
in Catalonia}.%
\begin{Footnote}
    Martorell parroquial archive, \wtitle{Llibre d'Òbits 1669--1689},
    \range{f}{146}, quoted in 
    \autocite
    [{\XXX[original]}]
    [70]
    {Balanza:CererolsFamily}.
\end{Footnote}
The Montserrat chronicle includes organists among the students of Cererols, and
there is a strong probability that Manalt knew or studied with Cererols, and if
so he could certainly have brought this manuscript to Barcelona, perhaps for use
at Santa María del Mar.
There it could have been heard by many sea travellers who made a point of
stopping at this church before and after voyages.\XXX[really?]







%***********************************************************************

\section{\quoted{The Newest Consonance}}



\endinput
