% Faith, Hearing, and the Power of Music
% in Devotional Music of the Spanish Empire
% 
% Andrew A. Cashner
% 
% Chapter 4, Heavenly Dissonance
% (Cererols, *Suspended, cielos*)
%
% 2018-05-22  Converted to LaTeX
% 2018-04-13  New version begun for book
% 2015-03     Dissertation defended

\chapter{Heavenly Dissonance (Montserrat, 1660s)}
\label{ch:cererols-suspended}

The Christmas villancico \wtitle{Suspended, cielos, vuestro dulce canto}, set by
Joan Cererols for the school choir of the Abbey of Montserrat around 1660, is
part of one of the most widely-attested families of villancicos, with evidence
surviving for seven other versions, from the Royal Chapel in Madrid to a convent
in Ecuador.
In the setting by Cererols, the chorus exhorts the heavens to cease the music of
the spheres and listen for \quoted{the newest consonance} that has come into the
world through the newborn Christ-child.
The piece thus extends the \emph{Verbum infans} trope discussed in chapter
\ref{ch:padilla-voces} with a greater focus on cosmology, creating a reflection
on the relationship between earthly and heavenly music.
Rather than the word-level text depiction of the madrigal-like \wtitle{Voces} by
Juan Gutiérrez de Padilla, Cererols evokes the contrast between levels of music
by a corresponding contrast of musical motives and styles.
Together with this polystylistic technique, Cererols also uses dissonance as an
ironic symbol to point to a higher kind of heavenly music, not governed by the
same rules as earthly music.
By the end of the piece, attentive listeners can actually hear the motive
representing Christ's voice become the \emph{cantus firmus} of a new heavenly
music; but within the Neoplatonic listening tradition, they are challenged to
listen beyond this depiction of heavenly music for the higher, unhearable music
of Christ himself.
The many versions of this text were passed from hand to hand among a network of
affiliated composers and modified to suit both local needs and changing
conceptions of heavenly music; tracing these connections provides further
evidence that composers used the metamusical villancico to prove their craft and
situate themselves in a lineage of composition.

The villancico family, and especially Cererols's setting, open a window into how
early modern Catholics understood heavenly music in a time when conceptions of
the cosmos were changing all around them.
John Hollander's study of English poetry on music, \wtitle{The Untuning of the
Sky}, takes its title from a poem by John Dryden which envisions the apocalypse,
when, at the sound of the last trumpet, \quoted{MUSICK shall untune the sky}.
Hollander uses the title to refer to a process he sees of increasing
secularization and disenchantment across the seventeenth and eighteenth
centuries, so that by the end of the period poets use the language of heavenly
harmony in a purely conventionalized way devoid of real connection to any faith
in the old conceptions of the cosmos.
But Dryden's poem, very much like Cererols's villancico, actually affirms
traditional views: it emphasizes the untunefulness of worldly music (including
the heavenly spheres) compared to the higher music of God.
In a Neoplatonic tradition, it was normal to emphasize the deficiencies of the
\quoted{world of change and decay} and point beyond them to a higher truth.
In this study of Hispanic poetry on music, we can note a change of
understandings, but not one that moves in the direction of secularization.
Instead, poets and composers---and devotees as well, we may imagine---find new
ways to reconcile the old cosmos with changing understandings and sensibilities.


\section{\quoted{The Newest Consonance}}

%     + Poem
%     + Music
% - Theology, Cosmology; Symbolism of dissonance
% - Genealogy

\endinput
