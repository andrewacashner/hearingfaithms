% Andrew Cashner -- Faith, Hearing, and the Power of Music
% Chapter 3 -- The Sacred Power of Music

% 2016-09-21     Revision for book begun

%*********************************************************
\label{ch:sacred-power}

Spaniards and their colonial subjects in the seventeenth century were fascinated with the hidden powers of music, just like their European contemporaries.
The creators of Italian opera were not the only ones who sought to harness music's power to move the affections, and Lutheran cantors and Reformed precentors were not the only ones who believed music to connect the human community with the divine, while fretting over the danger of abusing its power.\citXXX{}

The supernatural power of music was so evident to Spaniards in the early era of exploration that music played a key role in both military and religious strategy---most vividly demonstrated when Cortez initiated an attack on the Aztecs by having a soldier cut off the arms of the indigenous man playing the ceremonial \term{teponaztli} drum.\citXX[Tomlinson etc]
In the Spaniards' view, the Aztecs were already attacking them with music which had the power to summon up demonic forces. 
For them, silencing the drummer was an act of defense that disrupted the sonic order of the Aztecs and made them vulnerable to defeat.
The missionaries wasted no time after the fall of the Aztec Empire in establishing a new musical soundscape of bells, plainchant, and sacred polyphony.\citXXX{}

Villancicos came into the colonial soundscape after most of the bloody work of conquest was done and the horrific waves of plague had subsided.
This genre appears designed to play a key role in the next stage of Spanish colonization---building a Christian civilization.
Central to that role was music's power to unite the community in shared affections, as well as to connect people with the divine source of power.
Villancicos, together with Latin-texted sacred music, dramas, non-musical poetry, and civic ritual, were employed at key moments in community life, both of celebration and of supplication.\citXXX{}
In mainland Spain, villancicos were employed, if not to build Christian civilization, then to reinforce it against threats of all kinds.
Spanish church music projected the values of traditional religion and society, and projected the image of a Church and state endowed with divine power, even as the worldly foundations of Spanish power were being eroded by debt, war, and famine.\citXXX{}



















%%% Local Variables:
%%% mode: latex
%%% TeX-master: "../main"
%%% End:
