% Andrew Cashner -- Faith, Hearing, and the Power of Music
% Chapter 3 -- The Sacred Power of Music

% 2016-09-21     Revision for book begun

%*********************************************************
\label{ch:sacred-power}

Spaniards and their colonial subjects in the seventeenth century were fascinated with the hidden powers of music, just like their European contemporaries.
The creators of Italian opera were not the only ones who sought to harness music's power to move the affections, and Lutheran cantors and Reformed precentors were not the only ones who believed music to connect the human community with the divine, while fretting over the danger of abusing its power.\citXXX{}

The supernatural power of music was so evident to Spaniards in the early era of exploration that music played a key role in both military and religious strategy---most vividly demonstrated when Cortez initiated an attack on the Aztecs by having a soldier cut off the arms of the indigenous man playing the ceremonial \term{teponaztli} drum.\citXXX[Tomlinson etc]
In the Spaniards' view, the Aztecs were already attacking them with music which had the power to summon up demonic forces. 
For them, silencing the drummer was an act of defense that disrupted the sonic order of the Aztecs and made them vulnerable to defeat.
The missionaries wasted no time after the fall of the Aztec Empire in establishing a new musical soundscape of bells, plainchant, and sacred polyphony.\citXXX{}

Villancicos came into the colonial soundscape after most of the bloody work of conquest was done and the horrific waves of plague had subsided.
This genre appears designed to play a key role in the next stage of Spanish colonization---building a Christian civilization.
Central to that role was music's power to unite the community in shared affections, as well as to connect people with the divine source of power.
Villancicos, together with Latin-texted sacred music, dramas, non-musical poetry, and civic ritual, were employed at key moments in community life, both of celebration and of supplication.\citXXX{}
In mainland Spain, villancicos were employed, if not to build a new Christian civilization, then to reinforce the old one against threats of all kinds.
Spanish church music projected the values of traditional religion and society, and projected the image of a Church and state endowed with divine power, even as the worldly foundations of Spanish power were being eroded by debt, war, and famine.\citXXX{}

This chapter seeks a deeper understanding of how Spanish Catholics understood the sacred power of music, and considers the implications of this theology of music for how people listened to villancicos.
This chapter develops a historically grounded conception of music listening as relevant to villancicos.

%***************
\section{Listening in the Christian Neoplatonic Tradition}

Villancicos on the subject of music, particularly those that will be discussed in part II, consistently articulate a conception of music within the tradition of Christian Neoplatonism.
We have already seen this in villancicos on sensation that cast doubt on the ability of any sense to perceive spiritual matters unless tempered by faith.
These pieces model a practice of listening to music in which the immediate object of hearing is not the primary goal of perception.
Instead, these pieces point toward a higher, more perfect form of music; often by drawing attention to the artificial and imperfect nature of earthly music.

This Neoplatonic theological tradition was developed by Augustine in the fifth century and Boethius in the sixth, and was reinvigorated by Renaissance humanists like Ramón Lull and Marsilio Ficino among many others.\citXXX[neoplatonism, renaissance revival thereof]
Fray Luis de Granada was one of the key exponents of Augustinian Neoplatonism in Spanish theology.\citXXX[Fray Luis, Spanish religious literature]
Music theorists whose works were read in Spain emphasized Boethian and Neoplatonic concepts of music, from the speculative theory of Francisco Salinas and Athanasius Kircher to the more practical reference works of Pedro Cerone (\worktitle{El melopeo y maestro}, Naples, 1613) and Andrés Lorente (\worktitle{El porqué de la música} Zaragoza, 1672).\citXXX[Salinas, Kircher, Spanish theory, speculative traditions ---Nasarre too]

Christian Neoplatonists followed Augustine in viewing the material world as a reflection of a higher spiritual reality which ultimately had its source in the Supreme Good which was the Godhead.
The universe was thus ordered in a chain of being---termed the \term{Spiritual Hierarchy} in the influential medieval treatise attributed to Dionysius the Areopagite.\citXXX{}
The material world reflected higher truths only imperfectly, but---since Augustine rejected Gnostic and Manichee dualism---this world was also the only means through which those truths could be reached.
This philosophy became foundational for Catholic sacramental theology, then, in which material objects (bread, wine, water) and physical actions (eating, washing)
became means through which humans could encounter divine grace.\citXXX{}
Neoplatonic contemplation could be understood as a dialectical process of discerning the degree both of likeness and of unlikeness between earthly objects and heavenly truth.

% Augustine's confessions
% source of music's power comes from the fact that it reflects order of creation and its creator

%***************
\subsection{Music in the Chain of Being}

The definition of music in Boethius's \worktitle{De musica} provided the classic formulation, throughout the early modern period, of how music fit into the Neoplatonic chain of being (\tableref{table:Neoplatonic-hierarchy-music}).\citXXX[Boethius,Cerone etc on Boethius, Bower, Gouk]
The three types of music are arranged hierarchically so that each one points beyond itself to a higher level.
At the lowest level is \term{musica instrumentalis}---music played and sounded, music that humans can hear.
Higher up is \term{musica humana}---the harmony of body and soul, and of one human being with another in society.
Still higher is \term{musica mundana}---the harmonies created by the perpetual movement of the planetary spheres.
All three forms of music were governed by the mathematical ratios that Pythagoras discovered, so that the structure of sounding music echoed the structures of the human body and of human society, and in turn these echoed the very structure of the universe.

%************
\begin{table}
\caption{Neoplatonic hierarchy of music}
\label{table:Neoplatonic-hierarchy-music}
\inputtable{Neoplatonic-hierarchy-music}
\end{table}
%************

Villancicos on the subject of music embody the notion that even these three levels of music are subordinate to the supernatural forms of music in Heaven---the chorus of saints and angels, and above them, the mysterious harmonies of three in one in the Trinity, and two in one in the divine-and-human nature of Christ.\citXXX[Yearsley on heavenly music]
The three Boethian musics in this system would all be \soCalled{worldly} music, and in Spanish musical poetry it is important to clarify the distinction between the music of the heavens or \foreign{cielos}---that is, the planetary spheres---and the \soCalled{heavenly} music of the \foreign{cielo Empyreo} or Heaven, the supernatural realm beyond the material world.\citXXX[definitions]

\term{Musica instrumentalis}, then, though the lowest form of music in the chain of being, was the only form of music to which humans had direct access through the sense of hearing.
Metamusical villancicos explicitly emphasize the challenge that was central to all music-making in the Christian Neoplatonic tradition: to use the imperfect medium of sounding music to evoke all the higher forms of music, to lead listeners in contemplation up the chain of being beyond simply what was heard.

Vocal music played a special role in this system because for Neoplatonists, the human body was the microcosm of the whole created world.
The voice, then, was the physical expression of the microcosm, and vocal music thus doubly reflects the order of nature: in its musical ratios and proportions, which reflect those of the spheres, and as an expression of the human body as microcosm.
\term{Musica instrumentalis} is the finite expression of \term{musica humana} and reflects and leads to the contemplation of \term{musica mundana}, and to the higher Music of the Triune God who created all these lower forms of music.

\subsection{Hearing the Book of Nature Read Aloud}

With these basic foundations in place, we may gain a more nuanced understanding of music's power in the Neoplatonic tradition through a close reading of the theology of Fray Luis de Granada and the speculative music theory of Athanasius Kircher.
Both writers openly aimed to synthesize all previous knowledge on their subjects, and both were widely read in the Hispanic world.\citXXX[holdings]
These writings may be taken, with the necessary caution, as representative of widely held beliefs of their own time and after, as well as a guide to how earlier sources were read and understood by early modern Catholics.

Fray Luis's introduction to the first article of the Apostle's Creed, \quoted{I believe in God, the Father almighty, Creator of heaven and earth}, is really a fulsome exposition of a theology of the created world after the model of Basel's \worktitle{Hexameron}.\citXXX{}
In the Neoplatonic tradition, Fray Luis teaches that the natural world is a reflection of a higher truth---God's own nature---and that the creation was given so that by reflecting on it people would come to know its Creator.
Fray Luis frequently uses musical metaphors to describe the harmonious workings of the created world, and he includes a discussion of the physiology and theology of he human voice that applies directly to a historical understanding of music.

Kircher describes in detail the latest scientific knowledge about the anatomy of hearing and vocal production and the physiology of bodily humors and affects.
He lays out specific examples of how particular musical structures work through these bodily systems.\citXXX{}
In his own musical \term{Hexameron}, Kircher presents a cosmic view of music according to Neoplatonic traditions of theology and music theory, in which the whole universe is encompassed in the \quoted{working of music}---a rough translation of his inventive Greek-and-Latin title.

\quoted{The ultimate and highest good of man}, Fray Luis states at the outset of his \worktitle{Introduction to the Creed}, \quoted{consists in the exercise and use of the most excellent work of man, which the knowledge and contemplation of God}.%
  \begin{Footnote}
  \Autocite[182]{LuisdeGranada:Simbolo}: \quoted{El último y summo bien del hombre consistia en el ejercicio y uso de la mas excelente obra del hombre, que es el conoscimiento y contemplación de Dios}.
  \end{Footnote}
Fray Luis teaches that the created world is a \quoted{book of nature}, in which is written the grandeur, love, wisdom, and faithfulness of its Creator.
The first goal of humankind, then, is to learn to read this book of nature in order to come through it to the knowledge of God.
The goal of contemplating creation is \quoted{ascending by the staircase of the creatures to the contemplation of the wisdom and beauty of the Maker}.%
  \begin{Footnote}
  \Autocite[184]{LuisdeGranada:Simbolo}: \quoted{subiendo por la escalera de las criaturas á la contemplación de la sabiduría y hermosura del Hacedor}.
  \end{Footnote}
The reason one can \quoted{read} God through nature, Fray Luis teaches, is that the created world is a reflection of God's perfect order---a concept the firar repeatedly expresses using musical metaphors.
Fray Luis compares the perfect order of nature to a harmonious musical composition in which everything fits together \gloss{con sumo concierto}{with the most perfect concord}.
All the created things in this world, \quoted{like concerted music for diverse voices, harmonize together \add{\foreign{concuerdan}} in the service of man, for whom they were created}.%
  \begin{Footnote}
  \Autocite[191]{LuisdeGranada:Simbolo}: \quoted{Mas entre todas ellas es mucho para considerar, de la manera que todas (como una música concertada de diversas voces) concuerdan en el servicio del hombre, para quien fuéro criadas\Dots}.
  \end{Footnote}
The movement of the heavenly spheres, and their effects on the earth, are like a great \quoted{chain, or, it can be said, this dance, so well ordered, of the creatures, and like music for diverse voices\Dots.
Because things so diverse could not be reduced to sa single end with a single order, if there were not one who was like a chapelmaster \add{\foreign{maestro de capilla}}, who reduces them to this unity and consonance}.%
  \begin{Footnote}
  \Autocite[191]{LuisdeGranada:Simbolo}: \quoted{Asimismo los otros planetas y estrellas, segun los diversos aspectos que tienen entre sí y con el sol, son causa de diversos efectos acá en la tierra, como son lluvias, serenidad, vientos, frio, y calor y cosas semejantes. 
  Esta cadena, ó, si se puede decir, esta danza tan ordenada de las criaturas, y como música de diversas voces, convenció á Averrois para creer que no habia mas que un solo Dios.
  Porque no se pueden reducir á un fin con una órden cosas tan diversas, si no hubiere uno que sea como maestro de capilla, que las reduzga á esta unidad y consonancia}.
  \end{Footnote}

In the Neoplatonic tradition, these references to music are more than just metaphors.
The universe is not only like music, it actually is in some sense musical.
While some might think of Neoplatonists as ignorin actual sounding music for the sake of abstracted higher music, it is not possible to compare something to music without having some kind of earthly music in mind.
When Fray Luis compares the world to music \quoted{in diverse voices} he obviously has in his \quoted{mind's ear} polyphonic music of his own time, such as he would have heard at the Portuguese Royal Chapel as confessor to the queen.\citXXX[whose music would he have heard; bio source]
Likewise, when he compares God to a \foreign{maestro de capilla}, that has all the implications of that office in the Hispanic context, which included composition, teaching, and leading the choir in some form of conducting.%
  \footnote{The trope of Christ as a chapelmaster is discussed in the next two chapters.}
Thus God for Fray Luis is creator, prime mover, and sovereign ruler over creation, actively and intimately involved in its ongoing progress.

For Fray Luis, not only does creation reflect God's order; it actively proclaims its Creator.
Created things speak or sing with their own voices to communicate God's glory to the human who knows how to listen.
Fray Luis glosses Augustine's commentary on Psalm 26 to say, \quoted{Look around at all these many things from the heaven to the earth, adn you will see that they all sing and preach their Creator; because all types of creatures are voices \add{or, perhaps, utterances} that sing his praises}.%
  \begin{Footnote}
  \Autocite[185]{LuisdeGranada:Simbolo}: \quoted{Rodea cuantas cosas hay dende el cielo hasta la tierra, y verás que todas cantan y predican á su criador; porque todas las especies de las criaturas voces son que cantan sus alabanzas}.
  \end{Footnote}

\subsection{Voice as Expression of Man, the Microcosm}


%*********************
\section{Theological Functions and Modes of Listening}

\subsection{The Mnemonic Function}

\subsection{The Contemplative Function}

\subsection{The Affective Function}

\subsection{Social and Economic Functions}























%%% Local Variables:
%%% mode: latex
%%% TeX-master: "../main"
%%% End:
