% 2017-05-01    New version for book begun

\part{Listening for Unhearable Music}
\label{part:ListeningForUnhearableMusic}

\chapter{Christ as Singer and Song}

The majority of villancicos that take music as their central conceit are based
on themes of heavenly music.
Within the Spanish literary tradition of \term{conceptismo}, in which, under a
single governing conceit, a poem may be read in two or more ways simultaneously,
these metamusical pieces create a parallel discourse on music and theology by
playing on the rich symbolic heritage of musical words and concepts.
In the terms of Boethius, these pieces use \term{musica instrumentalis} to point
beyond themselves to higher forms of music---to the \term{musica humana}, the
harmony of body and soul, and of human society; \term{musica mundana}, the
harmony of the planetary and astral spheres; and even beyond the material world,
to the music of the \term{cielo Empyreo}.
In the Empyrean (English \term{Heaven} instead of \term{heavens} or sky), these
pieces evoke the song of saints and angels, and even the mysterious harmonies of
the Godhead: three persons in one Trinity, and the two natures of Christ as
man and God.

By tracing families and lineages of related villancicos on themes of heavenly
music, part~\ref{part:ListeningForUnhearableMusic} allows us to see how the
musical theology of villancicos developed over the course of the seventeenth
century, and the shifting ways that poets, composers, and listeners understood
the spiritual power of music to link hearing and faith.
The pieces interpreted in these case studies all challenge hearers to listen for
something more than human sounds; indeed, as these pieces reflect and reinforce
a Neoplatonic theology of music, they invite devotees to listen for unhearable
music, a music that defies human imagining, which creates harmony between the
human community and the divine.

This chapter analyzes and interprets two families of villancicos that represent
Christ as both singer and song.
The first is \term{Voces, las de la capilla}, set by Juan Gutiérrez de Padilla
for Puebla Cathedral in 1657, which is connected to two other known settings of
similar texts.
The second is \term{Suspended, cielos, vuestro dulce canto}, one of the
best-attested families of villancicos, with eight documented settings of variant
poetic texts in the later seventeenth century; the one extant complete musical
setting is by Joan Cererols, probably for the Escolania of Montserrat, around
1660.
These villancicos for Christmas connect incarnation, voice, and creation, as
they invite hearers to contemplate human music-making as a reflection of
Christ's nature as the divine \quoted{Word made flesh} (\scripture{Jn 1}).
Both textual traditions build on a patristic trope of Christ as
\term{Verbum infans}---the infant, or unspeaking Word, who does not need to
speak because God is already communicating himself to humankind through the
Christ-child's incarnate body.
The villancicos set by Padilla and Cererols turn this into musical theology by
imagining the baby Jesus not speaking, but singing; and by considering Christ
himself as the song being sung.
Thus the pieces, I argue, connected faith and hearing by making Christ the Word
audible through poetic and musical structures.

\section{%
    \quoted{Voices of the Chapel Choir} and the \quoted{Unspeaking Word}:
    \worktitle{Voces, las de la capilla}, Juan Gutiérrez de Padilla (Puebla,
    1657)
}

\section{%
    Heavenly Dissonance: 
    \worktitle{Suspended, cielos, vuestro dulce canto}, Joan Cererols
    (Montserrat, \circa{1660})
}


