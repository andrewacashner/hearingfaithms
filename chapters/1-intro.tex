% vim: set foldmethod=marker :

% Cashner, *Hearing Faith*
% chapter 1: Villancicos as Musical Theology
% 
% 2015-03-18	Dissertation defended
% 2017-11-15    New start for book proposal
% 2018-05-21    Converted back to LaTeX
% 2018-07-25    Expanded for press readers
% 2019-07-08	New start for Brill, under contract
% 2019-07-21    Complete revised draft text
% 2019-08-09    Begin revision after ch1-5 draft complete

\part{Listening for Faith}
\label{part:faith}

\chapter{Villancicos as Musical Theology}
\label{ch:intro}

\epigraph
{ergo fides ex auditu\\
auditus autem per verbum Christi}
{Romans 10:17}

%{{{1 intro
%{{{2 intro^2
\quoted{Faith comes through hearing}, wrote Paul the apostle to the Christian
community in first-century Rome, \quoted{and what is heard, by the word of
Christ} (\scripture{Rom 10:17}).%
\begin{Footnote}
    In both the original Greek and the Latin Vulgate quoted in the epigraph,
    the word for hearing (\foreign{akoē}, \foreign{auditus}) can mean the
    faculty of hearing, the act of hearing, the hearing organ, or that which is
    heard; 
    \Autocite[\sv{akoē}]{BDAG}.
    All translations are my own unless another source is given.
\end{Footnote}
Sixteen centuries later, amid the ongoing reformations of the Western Church,
Christians were finding ever new ways to make faith audible.
Voices raised in acrid contention or pious devotion boomed from pulpits,
clamored in public squares, and were echoed in homes and schools.  
In new forms of vernacular music, the voices of the newly distinct communities
united to articulate their own vision of Christian faith.
Roman Catholic reformers and missionaries, charged by the Council of Trent
(1545--63) to educate and evangelize, enlisted music in their campaigns to
build Christian civilization, both in an increasingly divided Europe and in the
expanding global domains of the Spanish crown.
In these efforts to make \quoted{the word of Christ} to be heard and
believed---to make faith appeal to hearing---what did they understand to be the
role of music?
What kind of power did Catholics believe music held to affect the relationship
between hearing and faith?

\addtoindex{
    Faith;
    Hearing;
    Listening|see{Hearing};
    Trent, Council of
}

This book is a study of how Christians in early modern Spain and Spanish
America enacted religious beliefs about music, through the medium of music
itself.
It focuses on a genre of devotional music known as \term{villancicos}, which
were musical settings of poetry in vernacular language, most often Castilian
Spanish.
A genre that had previously been an elite form of courtly entertainment and
sometimes devotion was transformed around the turn of the seventeenth century
into variety of complex, large-scale forms of vocal and instrumental music
performed as an integral part of public church rituals.%
    \Autocites
    {Torrente:VC-chapter}
    {Laird:VC}
    {Knighton-Torrente:VCs}
    {Borrego-Marin:Villancico}
    {Illari:Polychoral}
    {CaberoPueyo:PhD}
    {Swadley:VillancicoPhD}
    {ChavezBarcenas:PhD}

Of all the musical forms of Catholic Spain, sacred villancicos address the
theological nature and function of music most frequently and directly.
Hundreds of surviving poems and music begin with calls to
listen---\foreign{Escuchad}, \foreign{Oíd}, \foreign{Atended}---and a large
proportion of villancicos explicitly refer to music in their texts, in some
cases using musical techniques and terms as metaphors for theological concepts.
One of the most common occasions for villancico performances was in the liturgy
of Matins on Christmas Eve, when large ensembles of singers and musicians
celebrated the paradox of the incarnation---the most high God being born
as an infant in a humble stable---with musical representations of
angel choruses and dancing shepherds.
In these depictions of heavenly and earthly music, the ensembles were making
music about music.
Composers used a variety of techniques to refer to music beyond that being
performed in the moment, including evocations of birdsong, musical instruments,
quotations of specific songs and dances, and more conceptual plays on terms
from music theory, such as writing a strict fugue on the words
\mentioned{celestial counterpoint}.
If a play within a play in seventeenth-century Spanish or English theater is
metatheatrical, then these pieces are \term{metamusical}.

\addtoindex{
    Metamusical|see{Music about music};
    Music about music;
    Villancico!Poetic genre!Definition;
    Villancico!Musical genre!Definition
}

Through this genre of musical performance people embodied their theological
conceptions of music through the structures of music itself.
I call this \term{musical theology}, by which I mean a practice that goes
beyond the theological discourse about music found in written treatises or the
use of music for religious purposes, and becomes a way of \emph{performing}
theology.
This is a form of music that embodies the beliefs it proclaims.
Charles Seeger theorized that music and language were two distinct forms of
discourse and ways of knowing, so that just as we can speak about music by
using verbal discourse to refer to musical discourse, we could also
\emph{music} about music.%
    \Autocites
    {Seeger:Unitary}
    {Small:Musicking}
I argue that metamusical villancicos were precisely this kind of musical
discourse about music.

\addtoindex{
    Musical theology;
    Seeger, Charles
}
%}}}2

%{{{2 singing about singing : examples
\section{Singing about Singing}

We can begin engaging with this musical form of knowledge immediately, by
listening to two villancicos that embody \quoted{singing about
singing}.%
    \Autocites
    {Murata:Singing}
The first piece was composed by Juan Gutiérrez de Padilla (\circa{1590}--1664)
for the Cathedral of Puebla de los Ángeles, the most important religious center
in Spanish America.%
\begin{Footnote}
    \sig{MEX-PC}{Leg. 1/3}; 
    recording, \autocite{Padilla:1652ChristmasCD}.
%    \XXX[Audio recordings of every numbered example of music and poetry in this
%    book are available on the companion website.]
\end{Footnote}
\wtitle{En la gloria de un portalillo} was first performed in the Matins liturgy
on Christmas Eve 1652.  
In just the first seven lines of the anonymous poem
(\poemref{poem:En_la_gloria_de_un_portalillo-Padilla-1652}), the villancico refers
to multiple kinds of music, referring to the sounds of voices, choirs singing,
birdsong, dancing, and using solmization syllables.

\addtoindex{
    \GdP;
    Padilla, Juan Gutiérrez de|see{\GdP};
    Puebla!Cathedral
}

%{{5 poem GdP En la gloria
\insertPoem{En_la_gloria_de_un_portalillo-Padilla-1652}
{\wtitle{En la gloria de un portalillo}, estribillo as set by Juan Gutiérrez
de Padilla, Puebla Cathedral, Christmas 1652 
(\sig{MEX-Pc}{Leg. 1/3})}
%}}}5

Gutiérrez de Padilla's setting is metamusical in that it enacts these
references through music (\musref{mus:Padilla-Portalillo}).
It also has the virtue of demonstrating several typical features of the genre.
In texture, the introduction builds up from a solo line to dialogue between two
four-voice choirs, concluding the first phrase with an emphatic cadence for the
full chorus.
The music moves rhythmically in a lively three-beat meter, notated as groupings
of three minims in mensural notation.%
\begin{Footnote}
    The meter is notated \meterCZ{} or CZ.
    Contemporary music writer Andrés Lorente explains that \meterCZ{} is a
    cursive shorthand for \meterCThreeTwo{} or \meterCThree, all representing
    \term{tiempo imperfecto menor de proporción menor}.
    The three minims in every measure (\term{compás} in Spanish, same as
    \term{tactus}) were understood to be in a three-to-two relationship to the
    two minims in a measure of \term{C} time (\term{tiempo imperfecto menor});
    \autocites
    [156, 165, 210]{Lorente:Porque}
    [537]{Cerone:Melopeo}.
\end{Footnote}
The composer frequently breaks with this pattern by momentarily creating a
feeling of accents grouped in twos rather than threes (this is
\term{sesquialtera} or hemiola).
The shifts of duple and triple stress combine with stresses on the second beat
of the measure to create an energetic atmosphere with a rejoicing affect.  
The choruses urge each other on, declaiming in the same highly rhythmic manner
as the soloist.
The boy trebles (\term{tiples}) of both choirs singing at the top of their
range would have helped this passage seize the attention of listeners.

The soloist and choir are inviting everyone to come to the stable in Bethlehem,
where, the Tiple I soloist says, the shepherds \quoted{are turned to
boys}.
Playing on this expression, Gutiérrez de Padilla has the musicians
\quoted{turn} modally by adding C sharps, accented in a sesquialtera group that
contrasts with the even ternary patterns that follow.
The two-measure groups in the next passage emphasize the rhymes in
\foreign{tonos sonoros, repiten a coros} and the clear triple meter creates a
feeling of dance for \foreign{en bailes lucidos}.
When the soloist refers to the newborn Sun, he sings the note identified in
Guidonian terminology as D \term{(la, sol, re)}---\term{sol} in the hard (G)
hexachord.  
On the same word, the bass accompanist plays a different \term{sol}, G
\term{(sol, re, ut)}.

\addtoindex{
    Villancico!Musical genre!Musical characteristics;
    Cerone, Pedro;
    Lorente, Andrés;
    Music theory;
    Rhythm;
    CZ meter|see{Rhythm};
    Solmization
}

%{{{5 music GdP portalillo start
\insertMusic{Padilla-Portalillo}
{Gutiérrez de Padilla, \wtitle{En la gloria de un portalillo}
(\sig{MEX-Pc}{Leg. 1/2}, Christmas 1652), estribillo}
%}}}5

This villancico exemplifies music about music on several levels.
The text, which is being performed through music, itself refers to musical
performance.
The performance by the Puebla Cathedral chapel dramatizes the historical
celebration of the first Christmas while also celebrating the festival in
Padilla's present day.  
The music is self-referential on a symbolic level (as in the plays on
\term{sol}), but also functions on a more simple affective level to model and
incite exuberant joy and wonder.
According to contemporary theological writers, those were the appropriate
affects for worshippers at the feast of Christmas (see
\chapref{ch:padilla-voces}).

\index{Christmas}

%{{{3 cererols fuera que va
Celebrating Christmas with the right kind of spirit is also the theme of
another metamusical villancico, \wtitle{Fuera, que va de invención} by Joan
Cererols (1618--1680), monk and chapelmaster at the Benedictine Abbey of
Montserrat near Barcelona.%	
\begin{Footnote}
    \sig{E-Bbc}{M/760}; \autocite[81--94]{Cererols:MEM-VC}.
\end{Footnote}
Rather like today's catalog-like Christmas songs (\wtitle{Deck the Halls},
\wtitle{Chestnuts Roasting on an Open Fire}), the piece summons up all the
elements of a Christmas festival---masques, \foreign{zarabandas} and other
dancing, lavish decorations and clothing, pipes, drums, and so on.
As in many villancicos, the chorus acts dramatically in the role of the
festival crowd, shouting affirmations (\foreign{¡vaya!}) for each element of
the celebration as the soloists name them.  
Whereas Gutiérrez de Padilla's \wtitle{En la gloria de un portalillo} focused
primarily on the music of the historical Christmas day, the villancico of
Cererols is unambiguously about celebrating \soCalled{Christmas present}.
The piece seeks a theological meaning behind the Christmas customs: the masques
of Christmas, the poem says, are appropriate because in the Incarnation of
Christ, \foreign{Dios se disfraza} (God masks himself).
Cererols's original audience of pilgrims to the mountaintop shrine of
Montserrat would not have sung along with this piece, but the piece still
invites their wholehearted participation in the rituals of Christmas, both
through enjoying the choral singing and in the many other common-culture
customs that the piece celebrates.

The villancico allowed performers and listeners to celebrate the festival in two
senses: to sing the praises of the Christmas feast, while also singing the
praises of Christ that were appropriate to that feast. 
It did not so much teach them anything about Christ's birth as it modeled for
them appropriate modes of devotion to the Christ-child.
The primary purpose, I argue, is not to say that Christ \term{is} something but
to worship Christ \term{as} something.

\addtoindex{
    Cererols, Joan;
    Montserrat!Abbey;
    Popular devotion
}

These pieces presented hearers with a discourse about music, through music.
Sometimes the music they refer to is literal, human music-making; other times
it is more abstract, like the music of the spheres or the harmony of human and
divine in the incarnate Christ.
In every case, analyzing the musical choices made to represent texts about
music helps us understand how the creators and their audiences heard different
kinds of music.
And interpreting their theological aspect enables us to see how these pieces
served to communicate with hearers at a spiritual level.
%}}}3
%}}}2

%{{{2 paying attention to villancicos
\section{Paying Attention to Villancicos}

Villancicos continue to demand that we listen. 
Because so many villancicos explicitly address concepts of music, sensation,
and faith, these remarkable but understudied pieces offer us unique insights
into Spanish beliefs about music, which can deepen our understand of music's
role in early modern religious culture.
Villancicos constituted a major element of the soundscape of the
early modern Ibero-American world.
They shaped the everyday experiences of thousands of people across social
strata.
They provide evidence for a sustained effort by Spanish church leaders to use
music to make faith appeal to the sense of hearing, and they reflect widely
held beliefs and attitudes about music's spiritual power.

This book is the first major effort to understand the meanings and functions of
this music as a form of religious practice, integrating musical and theological
interpretation.
Despite its rich potential to inform several disciplines including literary
studies, religious studies, and musicology, few scholars have sought to
understand this music in the theological context for which it was created,
after initial movements in this direction by Paul Laird and Bernardo Illari.%
    \Autocites
    {Laird:VC}
    {Illari:Polychoral}
Most of the known sources of villancico music and poetry have now been
catalogued and many are becoming available online, but only a handful have been
revived in performance, and the genre has received relatively little serious
scholarly attention.%
    \Autocites
    [Catalogs of villancico poetry imprints:][]{BNE:VCs17C}
    {BNE:VCs18C}
    {UK:VCs}
    {US:VCs}
    {Codina:MontserratVCs}
    [selected catalogs of collections including villancico 
    music manuscripts:][]{Pedrell:BNC}
    {Stanford:Catalog}
    {LopezCalo:Segovia}
    {Bonastre:CanetCatalog}
    {Ezquerro:CatalogoZaragoza}
    {Stevenson:Sources}
Even scholars of music in the Spanish Empire have overlooked these musical
sources, part of a recent trend toward a social-historical approach that
emphasizes critiquing colonial power and draws primarily on documentary
evidence rather than on musical texts, whether because of lack of sources or
by deliberate methodological choice.%
    \Autocites
    {Tomlinson:SingingNewWorld}
    {Baker:Harmony}
    {Irving:Colonial}
    {RamosKittrell:PlayingCathedral}
The growing discourse on sound and sensation in the early modern world will
only benefit from a greater engagement with Spanish musical sources.%
    \Autocites
    {Rath:EarlyAmerica}
    {Ochoa:Aurality}
    {Dean:ListeningPolyphony}
    {Gouk:MusicScienceMagic}
    {Gouk:Harmonics}
    {Gouk:Sciences}
    {Gouk:RepresentingEmotions}
    {Tomlinson:Magic}
    {Austern:Nature}
    {Daston:Wonders}
    {Feldman:Passions}
    {Wagstaff:Processions}
Another study of Catholic concepts of music and hearing, focused on Italy,
provides a valuable analysis of verbal discourse \emph{about} music but does
not connect it to actual practices of music-making.%
    \Autocite{DellAntonio:Listening}
This book provides a necessary complement to these studies by analyzing how
people expressed and shaped beliefs about music through the medium of music
itself.
At the same time, the book offers a fresh approach by considering this music as
a source for historical theology, something few scholars have done.

If inquirers today wish to know what early modern Christians believed, we
must listen carefully to how they made their faith heard.
And if we wish to understand not only what music meant to early
modern people but even the details of how music worked, we must contemplate
what the makers and hearers of that music believed about its sacred power. 
In that endeavor our goal must be to understand their beliefs, not to impose
ours.
My own Christian faith (I am a Methodist) does give me sympathy for some
historic Catholic beliefs but it also makes me more critical of others, in
particularly the Church's promotion of slavery and social inequity.
Interpreting villancicos requires us to distance ourselves from our own
religious ideas---or anti-religious ideas---in order to hear the world through
historic ears, to the extent we can venture to do so.%
    \Autocites
    {Burstyn:PeriodEar}
    [for an example of the anti-religious approach, see][]{Menache:Vox}

By theology I do not mean the tired repetition of settled church doctrines such
as articles of the Creed or dogmas of the Council of Trent.
Instead I understand theology as a creative activity of imaginatively, even
playfully, seeking out ever-new ways of connecting revealed truth to observed
experience.
Thinking theologically in an early modern Catholic sense meant building 
endless chains of association and allusion among Biblical texts, the liturgy,
and theological writings, ancient and contemporary.
It meant interpreting new texts in light of these old ones, and reinterpeting
the old ones in light of the new.
And as we will see below, it grew out of and reinforced a view of the world as
a book waiting to be read.

\index{Theology!As creative endeavor}

The creators of villancicos drew on common experiences of everyday life and
linked them to the sacred in inventive ways that met the spiritual needs of
specific communities.
Each piece provides a new answer to Christ's question, \quoted{With what can we
compare the kingdom of God, or what parable will we use for it?}
(\scripture{Mk 4:21}).
The more surprising and puzzling the connection, the better---such as
representing the Virgin Mary as the chapelmaster of the heavenly chorus (see
below), or imagining Christ as a gambling card player.%
    \Autocite{Cashner:Cards}

\index{Metaphor}

In this way villancicos embody much of Paul Ricoeur's theory of metaphor as the
bringing together of two contradictory interpretations of an utterance.
Through a \quoted{metaphorical twist}, a novel comparison changes the reader's
view of both elements being compared.%
    \Autocite{Ricoeur:InterpretationTheory}
Spanish literary critics have used the term \term{conceptismo} to label the
metaphorical technique used in this kind of poetry, pioneered by Alonso de
Ledesma in his \wtitle{Conceptos espirituales} of 1600 and described by
Baltasar Gracián.%
   \Autocites
   {Gracian:Ingenio}
   [227--228]{Gaylord:Poetry}
   {Valbuena:Literatura}
   [447--448]{Torrente:VC-chapter}
Ledesma's earthy approach, aimed at relatively uncultivated readers, differed
starkly from the later poetry of Luis de Góngora in which the language was
pushed to the brink of comprehensibility; but both approaches were well
represented in villancico poetry.%
    \Autocites{Tenorio:Gongorismo}
The label \term{conceptismo} alone, though, does not do much to explain how the
process of signification worked in these pieces or, more importantly, why
Spaniards enjoyed thinking this way.
It also does not help us understand the additional layer of signification added
through music, in which a poetic concept is realized through properly musical
kinds of concepts.
In religious poetry, I propose, \term{conceptismo} was a specific literary form
for theological thinking.
It provided a set of conventions through which writers used metaphor to provoke
readers to connect human and divine, temporal and eternal.
This study shows how Spanish musicians developed their own approaches in form
and style to amplify the theological thinking in the poetic texts. 
In other words, they cultivated music as a form of creative theology.

\addtoindex{
    Conceptismo;
    \Gongora;
    Ledesma, Alonso de;
    Villancico!Poetic genre;
    Ricoeur, Paul
}

The book is organized in two parts, in which the two chapters of the first part
explore the central questions about music's role in the relationship between
faith and hearing, based on a global sampling of the repertoire.
The musical and poetic sources have been edited, mostly for the first time,
from sources in nine archives in Mexico, Spain, the United Kingdom, and the
United States.
The rest of this chapter examines the different ways Spanish musicians used
the music of villancicos to refer to other kinds of music, and the theological
framework that shaped their beliefs about what this music could do.
\Chapref{ch:faith-hearing} draws on villancicos on the subject of sensation and
faith, including representations of deafness, to explore the problems Catholics
encountered as they attempted to make faith appeal to hearing.
The three chapters of \partref{part:unhearable-music}, then, present case studies
of individual villancicos or sets of closely related pieces focused in Puebla,
Montserrat, and Zaragoza, respectively, all of which use \quoted{music about
music} to foster a Neoplatonic listening practice.
They challenge hearers to move past the simple level of audible music and rise
to contemplate a higher, unhearable kind of music.
These chapters also demonstrate that Spanish composers used metamusical
villancicos to establish their place in a lineage of composition, as they
developed a set of conventions for metamusical representation.
%}}}2
%}}}1

%{{{1 music about music
\section{Music about Music in the Villancico Genre}

%{{{2 survey, types
Sacred villancicos flourished especially in the second half of the seventeenth
century but continued to be a prominent element of Spanish Christian worship
through the nineteenth century in some places.
(The simpler folkloric Christmas carols denotated by \term{villancico} today
probably developed in parallel to this tradition of complex notated music and
absorbed some influences from it.)
Communities from Madrid to Manila heard and performed villancicos on the
highest feast days of the year---not just Christmas, but also Epiphany, Corpus
Christi, the Conception of Mary, and saints' days of local importance.
(They were less common in Holy Week, but there examples of \quoted{passion}
villancicos.)
Villancicos were typically presented in sets of eight or more, and were
interpolated after or in place of the Responsory chants of the Matins liturgy.
They were also sung in Mass and during Eucharistic devotional services.
Festival crowds heard villancicos in the public square in processions and
mystery plays, especially on Corpus Christi.

\addtoindex{
    Villancico!History;
    Villancico!Liturgical function
}

Villancicos were at once a genre of poetry and of music.
As religious lyric poetry, villancicos were printed in unbound leaflets or
broadsheet (\term{pliegos sueltos}) that advertised or commemorated the
performance of the musical settings in a particular place.%
    \Autocite{LopezLorenzo:VC-Sevillano}
These imprints were disseminated all across the empire, partly through networks
of musicians.
A single \term{villancico poem} is in many cases just one individual variant of
a broader \term{villancico family}, a group of closely related texts and their
variants.

\addtoindex{
    Villancico!Poetic genre!Definition;
    Villancico!Sources!Poetry imprints;
    Villancico!Sources!Poetry imprints!Dissemination;
    Villancico!Families
}

As a musical genre, the settings of villancico poems prioritize the clarity and
meaning of the words, and like other seventeenth-century vocal music typically
present one phrase of text at a time.
They were performed by ensembles of voices, from as small as one or two solo
voices to as large as three separate choruses of a dozen vocal parts with
multiple singers per part.
The very small-scale pieces were similar in texture and style to Italian or
German sacred concertos and English verse anthems; later in the seventeenth
century they become more like continuo songs.%
    \Autocite{Kendrick:SacredSongs}
The pieces with larger scoring feature lively contrasts of texture between each
choir, the full ensemble, and soloists, rather like an English full anthem.
Voices were typically accompanied by a continuo group of bass-line instruments
like dulcian (\term{bajón}) and polyphonic instruments like harp and organ.
Vocal parts could be doubled with reed instruments like dulcians and shawms
(\term{chirimías}) and brass instruments like sackbuts (\term{sacabuches}).
%    \citXXX[instrumentation, performance practice] - Torrente diss?
The words for these pieces were in Spanish and sometimes other vernacular
languages including Portuguese, Catalan, and Náhuatl, along with pieces whose
texts imitated the dialects of African slaves.
The music varied in style and technique from elements of common dances
and popular tunes up to the most sophisticated polyphonic tone-painting.
Sets or cycles of villancicos for a particular feast like Christmas included
many different subtypes of villancicos within them, offering something for
everyone.

\addtoindex{
    Villancico!Musical genre!Scoring;
    Bajón;
    Dulcian|see{Bajón};
    Musical instruments!Use with voices;
    Villancico!Poetic genre!Languages
}

The structure of villancicos reflects the effort to communicate on multiple
levels.
The \term{estribillo} or refrain section of a typical villancico was presented
by the full ensemble at the beginning and then repeated at the end; composers
usually set this in relatively complex polyphony similar to what they would use
for a motet.
In the center of the piece, the \term{coplas} or verses were usually set
strophically for solo singers or a reduced ensemble with accompaniment.%
\begin{Footnote}
    Sometimes preceding the estribillo was an \term{introducción}, usually for
    a smaller group; and in earlier examples the estribillo could lead to a
    \term{responsión}, an amplified version of the estribillo for full
    ensemble.
\end{Footnote}
Bernardo Illari argues that the coplas spoke more directly to common people.%
    \Autocite{Illari:Popular}
It would have been easier for them to make sense of the words that were sung to
the simple, repeating melodies.
The tunes may also have been familiar if they were based on oral traditions for
singing poetry to stock melodic formulas, especially in \term{romance} meter.
%    \citXXX[romance sources]
The \term{estribillo}, by contrast, would appeal more to the cultivated elite,
as it was often much more complex and drew on traditions of learned
counterpoint.

\addtoindex{
    Villancico!Structure;
    Estribillo|see{Villancico!Structure};
    Coplas|see{Villancico!Structure}
}

Our initial examples of metamusical villancicos by Gutiérrez de Padilla and
Cererols combine several of the common tropes of \quoted{music about music} in
the villancico genre, as evidenced by a global survey of extant villancico
poems and music.%
\begin{Footnote}
    The survey was based on the listings in catalogs and published studies and
    from archival sources, some previously unknown, from all over the former
    Spanish Empire (see the bibliography).
\end{Footnote}
More than eight hundred villancicos were found in which themes of musical
hearing were central, a number that only hints at the original size of this
repertoire.
These metamusical villancicos may be grouped in eight main categories
(\tabref{tab:survey}): in descending order of frequency these are hearing and
sound, music and singing, birdsong, dance, musical instruments, angelic
musicians, music of the heavenly spheres, and pieces that treat the
relationship of sensation and faith.
An additional category of pieces about affects is also included, since these
pieces, though not explicitly about music, do address the question of how
listeners should respond in worship. 

\index{Villancico!Topics}

%{{{5 table survey topics
\insertTable{survey}
{Topics of metamusical villancicos in global survey}
%}}}5

In each of the categories in \tabref{tab:survey}, we may distinguish between two
main ways of referring to music.  
Some pieces are primarily imitative, referring to real human music-making
(Boethius's \term{musica instrumentalis}).
These pieces are highly \term{intermusical}, in the way a verbal text full of
references to other texts is intertextual.
In contrast to this first category of imitative pieces, villancicos in a second
category refer to music as more of an abstract concept, such as the higher
Boethian levels of music, music as a Neoplatonic ideal, and the music of
Heaven---notions that overlap in inconsistent ways in early modern thought.
Of course, the pieces in the latter group still refer to music in the abstract
through the medium of real sounding music.  
Some of these pieces depend more on \term{intramusical} relationships---that
is, musical references internal to the individual piece itself, such as melodic
or rhythmic motives or internal contrasts of musical style without overt
references to pre-existing styles \quoted{outside the piece}.
In this section we will look at key examples in several of these categories,
moving up a Neoplatonic chain of being from the simplest imitations of birdsong
to the more conceptually challenging evocations of heavenly music.

\addtoindex{
    Boethius;
    Neoplatonism;
    Heavenly music;
    Imitation;
    Reference;
    Musica instrumentalis|see{Boethius}
}
%}}}2

%{{{2 imitative references
\subsection{Imitative References to Music: Birdsong, Instruments, Songs and
Dances}

%{{{3 birdsong
A frequent example of imitative musical reference in villancicos is when the
ornamented vocal lines are used to represent birdsong.%
    \Autocite[295--301]{Illari:Polychoral}
In a piece called \wtitle{Sagrado pajarillo} (Little sacred bird), Zaragoza
composer José de Cáseda sets the lyrics \foreign{con gorgeos} (with trills) to
twittering melismas (\musref{mus:CasedaJ-Sagrado_pajarillo}).%
\begin{Footnote} 
    This piece comes from the archive of the Conceptionist Convento de la
    Santísima Trinidad in Puebla de los Ángeles and is now preserved at CENIDIM
    in Mexico City (\sig{MEX-Mcen}{CSG.155}): 
    \autocite{Tello:SanchezGarzaCatalogo}.
\end{Footnote}
Birdsong had theological importance as the paradigm of music-making in the
natural world, a reflection of God's own artifice in creation (see below).
Using human voices to imitate birdsong, then, prompted listeners to consider
how the artifice of human music reflected the order of creation.

\addtoindex{
    Musical topics!Birdsong;
    \CasedaD;
    \PueblaConvento
}

%{{{5 music CasedaJ Sagrado pajarillo
\insertMusic{CasedaJ-Sagrado_pajarillo}
{Bird-like trills in Cáseda, \wtitle{Sagrado pajarillo}, excerpt from the
estribillo, Tiple I-1}
%}}}5
%}}}3

%{{{3 instruments: percussion
Next to the musical sounds of animals, the sounds of musical instruments
provided rich possibilities for musical imitation in a theological context.
Wooden sounding boards, brass pipes, and gut strings allowed human players to
take the potential of music built into the created world---such as the perfect
Pythagorean ratios of the overtone series---and actualize them in sound.
To imitate percussion instruments, for example, villancico composers paired
onomatopoetic nonsense words with distinctive rhythmic patterns.
Juan Gutiérrez de Padilla had the chorus of Puebla Cathedral represent the
sound of the castanets and tabor with contrasting onomatopoetic rhythmic
patterns on the words \foreign{al chaz, chaz de la castañuela, y el tapalatán
de el tamboríl} (\musref{mus:Padilla-Alto_zagales-chaz}).
Such pieces about instrumental music imitate the instrument itself while also
playing with a stylistic topic associated with that instrument.

\addtoindex{
    Musical instruments!Symbolism;
    \GdP;
    Puebla!Cathedral
}

%{{{5 music GdP Alto zagales chaz
\insertMusic{Padilla-Alto_zagales-chaz}
{Gutiérrez de Padilla, \wtitle{Alto zagales de todo el ejido}
(\sig{MEX-Pc}{Leg. 2/1}, Christmas 1653), estribillo: Imitation of castanets
and tabor}
%}}}5

The same instrumental trope appears in a villancico poem performed at Toledo
Cathedral in 1645.%
    \footnote{\sig{E-Mn}{VE/88/12, no. 6}.}
Though the music, credited in the poetry imprint to Vicente García, has not
been found, the words alone conjure up a racket of percussion sound:
\begin{quotepoem}
    Porque los instrumentos sonaban así, 
        & Because the instruments sounded like this: \\
    El Atabal, tan, tan ,tan,	    & the drum, tan, tan, tan, \\
    El Almirez, tin, tin, tin, 	    & the mortar, tin, tin, tin \\
    la Esquila, dilín, dilín,	    & the chime, dilín, dilín, \\ 
    y la Campana, dalán, dalán,	    & the bell, dalán, dalán, \\
    Las Sonajas, chas, chas, chas,  & the rattle, chas, chas, chas, \\
    y el Pandero, tapalapatán.	    & and the tambourine, tapalapatán.
\end{quotepoem}
The instruments in this list are simple, rustic noisemakers from everyday
peasant life.%
\begin{Footnote}
    Note that this source from Toledo spells the rattling sound \foreign{chas}
    while Gutiérrez de Padilla's manuscripts from New Spain spell it
    \foreign{chaz}, where the latter sources reflect the Andalusian accent of
    the Spanish settlers of central Mexico.
\end{Footnote}
In this villancico these instruments, which are described further in the
coplas, join together with the sounds of the mule and other animals, and the
dances of the shepherds.  
This piece, like many villancicos, depicts a scene of common folk rejoicing
after their own fashion in the humble setting of the Bethlehem stable.
The focus here is not on instrumental performance in the present day but on
helping listeners imagine the sounds of the first Christmas.
Imitating an instrument did not mean that the instrument was actually used in
church; indeed in many cases the situation seems to have been the opposite.%
\begin{Footnote}
    Despite the fanciful reconstructions that can be heard in modern
    recordings, more evidence is needed to establish that percussion
    instruments were used in church.
\end{Footnote}

\addtoindex{
    García, Vicente;
    Toledo!Cathedral
}
%}}}3

%{{{3 clarines
\subsubsection{Becoming Clarions}

Another common example of the imitative, intermusical type would be the many
pieces that mention the \term{clarín} (clarion or bugle), in which the
singers perform patterns that are meant to sound like brass fanfares.
The typical style of clarion evocations may be seen in two examples from the
archive of the Escorial, which holds much of the surviving repertoire of the
Spanish Royal Chapel.
Most clarín pieces do not actually feature written-out clarín parts; in most
cases the instrument is imitated vocally or by other instruments, like
\term{chirimías} (shawms).
Matías Durango's \wtitle{Cajas y clarines} (Drums and bugles) evokes these
instruments with voices and shawms in martial style, as part of a broader
battle topic.%
    \footnote{\sig{E-E}{Mús. 29/15}.}
Durango's clarín topic is strikingly similar to one of the rare surviving
clarín parts from a villancico, in a fragment by the prominent Madrid composer
Sebastián Durón
(\musref{mus:Durango-Cajas_clarines},\ref{mus:Duron-Dulce_armonia_clarin}).
Both are in the same collection of music from the Royal Chapel preserved at the
Escorial.%
\begin{Footnote}
    \sig{E-E}{Mús. 29/15} (Durango), \sig{E-E}{Mús. 32/16} (Durón).
\end{Footnote}
A villancico by José Romero from about 1690, \wtitle{Suene el clarín} (Let the
clarion resound) includes an actual notated part for \foreign{los clarines de
los autos}, that is, for the clarions played in the \term{autos sacramentales}
or public Corpus Christi dramas.% 
\begin{Footnote} 
    \sig{D-Mbs}{Mus. ms. 2914}, edited in \autocite[655--661]{CaberoPueyo:PhD}.
\end{Footnote}
The sung voices layer bugle-like gestures above them, creating a more complex
fanfare than the valveless instruments could play on their own.

\addtoindex{
    Durango, Matías;
    Romero, José de;
    Escorial;
    Royal Chapel;
    Clarín;
    Clarion|see{Clarín};
    Auto sacramental;
    Musical topics!Military
}

%{{{5 music: clarin in voice vs actual, durango/durón
\insertMusic{Durango-Cajas_clarines}
{Matías Durango, \wtitle{Cajas y clarines} (\sig{E-E}{Mús. 29/15}, Tiple I-1,
estribillo): Imitation of \term{clarín} by voice and shawm}

\insertMusic{Duron-Dulce_armonia_clarin}
{Sebastián Durón, \wtitle{Dulce armonía} (\sig{E-E}{Mús. 32/16}, estribillo):
Extant \term{clarín} part}
%}}}5

Why were there so many references to the clarín but so few written parts?
The lack of parts apparently means that clarion music was more commonly
improvised.  
The references in villancicos suggest that clarions were typically used at
moments of civic importance, signalling a call to arms, announcing victory,
heralding the arrival of the king or his representative.
In other words, the clarion was a means of civic communication, not musical
performance, and its calls could not be imitated out of context any more than
someone could yell \quoted{fire} in a crowded cathedral.
Having the actual bugle call to would literally alarm people and the
opportunity to use the instrument symbolically would be lost.

The \term{clarín} was used in military, royal, and apocalyptic symbolism as far
back as the allegorical \foreign{clairon} fanfares in the 1454 Feast of the
Pheasant hosted by the ancestor of the Hapsburg monarchs, Philip the Fair of
Burgundy.%
\begin{Footnote}
    \Autocites
    [340--380]{LaMarche:Memoires}
    {Bloxam:JNV}
    {Perkins:Patronage15C}.
    On the symbolism of this instrument in contemporary Spanish drama, in which
    \term{Clarín} was the name of a comic stock character, see
    \autocite{Damjanovic:Clarin}.
\end{Footnote}
In \wtitle{No temas, no recelas} by another famed Madrid composer, Cristóbal
Galán (from \circa{1691}), the voices represent \term{clarín} music in a scene
of \quoted{heavenly armies} going to battle.% 
\begin{Footnote} 
    \sig{D-Mbs}{Mus.  ms. 2892}, 
    edited in \autocite[555--565]{CaberoPueyo:PhD}.
\end{Footnote}

\addtoindex{
    Philip the Fair;
    Musical instruments!Symbolism;
    \GalanC
}

Imitating the clarion within a battle topic was not always just a spiritual
symbol: it was often used like real bugle fanfares were, to celebrate military
victories, or boost morale in the midst of conflicts.%
    \Autocite[288--294]{Illari:Polychoral}
The anonymous villancico \wtitle{Noble clarín de la fama} states on the cover
page that it was performed \quoted{for the profession of the sisters
\foreign{Señoras} Sor Sagismunda and Sor Jacinta Perpinyà into the Convent of
Santa Clara of Gerona, 1693}.% 
    \footnote{\sig{E-Bbc}{M/772/35}.}
The surname of these siblings (sisters by blood and now by vow) is the name of
Perpignan, capital of the Catalan region of Rosselló, which had become the
French Roussillon after the Peace of the Pyrennees in 1659.
A long struggle over this border territory in the War of the Great Alliance
climaxed in the year this villancico was performed, as the French general
Catinat triumphed over the allied powers at Marsaglia.
The villancico appears to align Catalan identity with the French cause, as it
praises the \quoted{Catalan Amazons, who have the name of Perpignan}, who
\quoted{seek today good protection for their defense in Francisco}---that is,
they look for protection both to Saint Francis, the probable patron of their
order, and to France.
In enlisting for spiritual battle with Francis, the estribillo suggests, the
sisters themselves are becoming clarions of war.%
\begin{Footnote}
    Excerpt from the estribillo: 
    \quoted{Noble clarín de la fama/ 
    que de vozes te alimentas,/
    toca, toca, alarma, alarma,/
    que dos niñas hoy son aliento
    de tu voz excelsa,/
    Catalanas amazonas,/
    de Perpiñan nombre tienen,/
    pues bella guardia en Francisco,/
    buscan hoy por su defensa,/
    cuidado serafines,/
    resuenen los clarines}.
\end{Footnote}

\addtoindex{
    Girona;
    Sanctoral devotion!Francis;
    Convent music;
    Catalonia
}

At this moment of commitment in these young women's lives, coinciding with a
political crisis, the concept of \emph{becoming} a clarion held more
significance theologically than the mere sound of the actual instrument would
have held.
This concept is realized even more completely through musical representation in
the villancico \wtitle{Venid, querubines alados} by Juan Hidalgo (1614--1685,
composer of the first Spanish operas for the royal court).%
\begin{Footnote}
    \sig{D-Mbs}{Mus. ms. 2895}. 
    On Hidalgo's theater music, see \autocite{Stein:Songs}.
\end{Footnote}
In this chamber villancico or \term{tono divino}, the two voices sing that just
as the birds of the dawn are \term{clarines} celebrating the Blessed Virgin, so
too will their own voices become \term{clarines}
(\poemref{poem:Venid_querubines_alados-Hidalgo}).
Hidalgo interweaves the two voice parts in rising fanfare gestures that
actually allowed listeners to hear the singers transforming their voices into
\term{clarines} (\musref{mus:Hidalgo-Venid_querubines}).
Here individual piety, Marian devotion, and all the military and political
associations of the clarion are merged together in a way distinctive of
imperial Spain.

\addtoindex{
    Hidalgo, Juan;
    Tono divino;
    Sanctoral devotion!Mary;
    Theater!Opera
}

%{{{5 poem and music Hidalgo Venid querubines
\insertPoem{Venid_querubines_alados-Hidalgo}
{\wtitle{Venid querubines alados}, poem set by Hidalgo (\sig{D-Mbs}{Mus. ms.
2895}), copla 5}

\insertMusic{Hidalgo-Venid_querubines}
{Hidalgo, \wtitle{Venid querubines alados}, duo response at end of each copla}
%}}}5

Clarion villancicos depended on the instrument's signification of power to
reiterate the sovereignty of the Spanish church and state.
They proclaimed that God was working through a divinely ordained power
structure to govern his creation.
That vision of the world was certainly oppressive to many but its stability
surely gave comfort to others.
We can affirm that the \quoted{music of state} in the Spanish Empire served as
a \quoted{instrument of dominion} while also acknowledging that many Spanish
subjects actually believed in the theological foundations of their political
order and even actively contributed to reinforcing it.%
    \Autocites
    {Rodriguez:Villancico}
    {Sage:Instrumentum}

\index{Power!Musical projection}
%}}}4
%}}}3

%{{{3 dance, ethnic vcs
\subsubsection{Dance and Difference}

Dance topics in villancicos provided another way for the genre to point beyond
itself to other kinds of music in society, and like clarion topics these
references both reflected and reinforced Spanish attitudes toward social
structure.
Many dances are explicitly named and often the text proclaims the villancico
itself to \emph{be} a specific kind of dance, including \term{zarabanda},
\term{jácara}, \term{guarache}, \term{danza de espadas}, and
\term{papalotillo}.
In many cases no other music survives for dances by these names, or there are
only sketchy outlines for an improvised tradition, while the villancico
arrangements provide a complete musical texture.%
    \Autocites[For example,][]{Ruiz:Luz}
    [inventively reconstructed,][]{Lawrence-King:DancesCD}
Like clarion pieces, though, these are abstractions of the music they
reference.
They are not apparently meant for real dancing, but instead provide a discourse
about dancing.%
\begin{Footnote}
    There was ritual dance on Corpus Christi in Seville and Valencia
    cathedrals, performed by the boy choristers known as \term{seises}, but
    more research is needed on the question of whether villancico performances
    included actual dance or other staging elements.% 
    \Autocite{Comes:Danzas}
\end{Footnote}

\addtoindex{
    Musical topics!Dance;
    Zarabanda;
    Jácara;
    Seville!Cathedral;
    Valencia!Cathedral;
    Corpus Christi;
    Comes, Juan Bautista;
    Villancico!Folkloric sources
}

In the sacred \term{jácaras} by Gutiérrez de Padilla for each successive year
1651--53, the composer continues to develop the same tune, harmonic pattern,
and rhythmic groove.
The basic pattern is closely similar to the music Álvaro Torrente has
reconstructed for the secular \term{jácara}, which was like a ballad
celebrating the exploits of renegade heros with colorful and often bawdy
underworld slang.%
    \Autocites
    {Torrente:Jacara}
    [512--514]{Torrente:VC-chapter}
Mixing outlaw language with chivalric imagery and offbeat theological
references, the sacred \term{jácaras} from Puebla depict the baby Jesus with
typical braggodocio as a gunslinger arriving on the scene to finish off a feud
with the devil---in one case, he comes \quoted{from way up in Texas}.
As in hip hop today, verbal virtuosity and inside references were prized in this
genre. 
In Gutiérrez de Padilla's best-known \term{jácara}, \wtitle{A la jácara,
jacarilla} (1655), every line of the coplas is built from \foreign{principios
de romances} (the first lines of traditional \term{romance} ballads), an
ingenious secret only hinted at in the final verse.
The core of the music is clearly drawn from the secular dance, but each year
Gutiérrez de Padilla made the setting more complex, contrapuntally and
rhythmically.
The clever plays of words and music in this subgenre may stem from the
trickster quality of the \term{jaque} or \term{pícaro} (rogue).

\addtoindex{
    \GdP;
    Puebla!Cathedral;
    Romance poetry
}

The piece is explicit about its effort to bring together seemingly opposed
worlds:
\begin{quotepoem}
    A la jácara, jacarilla 
    & Let's have a \term{jácara}, a little \term{jácara}, \\

    de buen garbo y lindo porte,
    & one that is well mannered and gentile, \\

    traigo por plato de corte 
    & I bring as a dish of the court \\

    siendo pasto de la villa.
    & what is really feed from the village.
\end{quotepoem}
The contrast between \term{corte} and \term{villa} is between noble and common,
gentility and laborers, urban and rural, refined and crude---notably not sacred
and secular.
It is also a play on the term \term{villancico}, which comes from \term{villa},
and suggests an attempt to say something here about the function and meaning of
the whole genre as a way of bringing high and low social registers together.
Later in the century this subgenre, like most others, became increasingly
conventionalized and removed from its worldly origins.
Mateo de Villavieja's \wtitle{Jácara en anagramas} (from Madrid, Convento de
la Encarnación) does preserve the earlier emphasis on ingenuity, as the piece
is composed algorithmically from permutations of a set poetic and musical
phrases.
But it has lost all stylistic references to the secular \term{jácara}, and with
them any engagement with the world outside church.%
    \footnote{\sig{E-MO}{AMM.4261}.}

\addtoindex{
    Villavieja, Mateo de;
    \MadridEncarnacion
}

Metamusical references to traditional music-making of lower-class people
extended also to the depiction of ethnic difference.
There are villancicos that depict non-Castilian groups like Native Americans,
African people, Catalans, Frenchmen, even Irishmen, through caricatured
deformations of language and music.%
    \Autocites
    {Baker:EthnicVC}
    {Baker:PerformancePostColonial}
    {Davies:LocalContent}
    {AlvesSimao:VillancicosDeNegros}
    {Molinero:Negros}
    {Santamaria:Negrillas}
    {Goldberg:SonidosNegros}
What have come to be called \quoted{ethnic villancicos} were labeled in their
time as \term{villancicos de naciones} or by the name of the particular ethnic
type for that piece, like \term{gallego} (Galician), \term{gitano}
(\quoted{Gypsy}), \term{indio} (\quoted{Indian}), or \term{negro},
\term{guineo}, and similar terms for Africans.
Most of these pieces, and especially the \term{villancicos de negro}, refer
specifically to the characteristic music and dancing of these groups, often
naming their instruments and describing their motions.
The texts use some smatterings of foreign words but mostly ask the performers
to put on an accent in Spanish: in these caricatures the Gypsy ends all her
words with a Z (\foreign{Puez los trez zon Magoz,/ hombrez de ezfera}).
\begin{Footnote}
    \wtitle{Vamos al portal gitanilla}, Imprint from Epiphany 1666, Zaragoza,
    Iglesia de El Pilar (\sig{E-Mn}{VE/1303/1}), later attributed to Vicente
    Sánchez, \headlesscite[203--204]{Sanchez:LiraPoetica}.
\end{Footnote}
The Black character says L and S when he should say R and J, drops ending S
sounds, and mismatches genders and cases (\foreign{Mi siñol Manuele, \Dots{}
Sesu, \Dots{} pluque son linda cosa}).%
\begin{Footnote}
    Gutiérrez de Padilla, \wtitle{Al establo más dichoso (Ensaladilla)},
    \sig{MEX-Pc}{Leg. 1/3}; edited in \autocite{Cashner:WLSCM32}.
\end{Footnote}
Villancicos about African characters also frequently feature nonsense
syllables, whether lullaby phonemes like \foreign{ro ro ro ro} and \foreign{le
le le le}, or apparent gibberish like \foreign{tumbucutú, cutú, cutú} and
\foreign{gulumbé, gulumbá} that tells us what African languages like Kikongo
sounded like to a Spanish ear.
This type of piece represents Africans as always happily engaged in drumming
and dancing, even as it caricatures their music and movement through
artificially distorted language and rhythm
(\musref{mus:Padilla-Al_establo-Negrilla}).

\addtoindex{
    Social class;
    Race;
    Villancico!Ethnic;
    Africans;
    Slavery;
    Zaragoza!El Pilar;
    \SanchezV;
    Negrilla|see{Villancico!Ethnic};
    Villancico de negro|see{Villancico!Ethnic};
    Caricature;
    Humor
}

%{{{5 GdP Al establo negrilla
\insertMusic{Padilla-Al_establo-Negrilla}
{Gutiérrez de Padilla, \wtitle{Al establo más dichoso (Ensaladilla)} 
(\sig{MEX:Pc}{Leg. 2/1}, Puebla Cathedral, Christmas 1652), \term{Negrilla}
section, Dialogue of the Angolans}
%}}}5

In their metamusical references, these pieces employ literal
imitation (as of percussion, and of the \soCalled{musical} sound of foreign
languages) and also point to the musical practices of their characters.
Though no one has yet demonstrated any concrete links with African music
traditions, such as may be found in other Latin American musics like
\term{capoeira Angola} in Brazil, the music of \term{negrillas} must have
sounded African to its Spanish audience, at least according to the conventions
of exoticism.%
    \Autocite{Kubik:AngolanTraits} % XXX Locke
These pieces also make more abstract references to music through the use of
nonsense words that, somewhat like solmization syllables (see below), symbolize
and enact music-making.
Like \term{jácaras}, ethnic villancicos grow increasingly conventionalized and
distant from their origins so that the \term{negro} character in one year's
villancicos was much more similar to the \term{negro} of the previous year's
set that he probably was to any real African person.
And like \term{clarín} pieces, ethnic villancicos both reflected and reinforced
imperial Spain's power structure by projecting a theological vision of that
structure as divinely ordained and immobile.

Their discourse on racial difference must be understood within a Neoplatonic
theological concept of music and society, in which the lowliest elements in the
created world could lead a person to the knowledge of the highest.
Juan Gutiérrez de Padilla, who included a \quoted{black villancico} in most of
his Christmas cycles for Puebla, depicts the paradox of Neoplatonic thought
when in 1652 he has a group of Black characters say \quoted{Listen, for we are
singing like the angels}.
As the Angolans go on to sing a vernacular \term{Gloria in excelsis} in their
dancing triple meter, full of syncopations notated by coloring in the mensural
noteheads, above them suddenly enter the two boy soprano parts of the second
chorus, which have been silent until now, singing the same \term{Gloria} with
them---but in the white notes of duple meter, and quoting a plainchant
intonation (\musref{mus:Padilla-Al_establo-Gloria}).
The Angolans and the angels are brought together for a miraculous moment
through contrasting types of rhythmic movement in which the hidden harmony
between earthly and heavenly music is revealed.
The Angolans are in some ways depicted sympathetically, as instead
of the gold, frankincense, myrrh of the magi (one of whom was portrayed on
Puebla's high altar as a black African), bring the Christ-child the more homely
but practical gifts of a potato, a toy, and diapers.
The paradox is not only between the lowest kind of earthly music and the
highest music of heaven, but between the idealized world represented in the
music and the reality of 1650s Puebla.
Gutiérrez de Padilla himself owned an enslaved Angolan man named Juan, and
his depiction of slaves singing with angels only reinforced the structures that
kept Juan in his place.%
    \Autocite{Mauleon:PadillaCivil}

\addtoindex{
    \GdP;
    Puebla!Cathedral;
    Rhythm;
    Africans!Angola/Congo;
    Magi;
    Slavery;
    Neoplatonism
}

%{{{5 music GdP Al establo Gloria
\insertMusic{Padilla-Al_establo-Gloria}
{Gutiérrez de Padilla, \wtitle{Al establo más dichoso} (1652), \term{Negrilla}:
Polymetrical \term{Gloria} of Angolans and angels}
%}}}5

It is possible, though more evidence is needed, that the vogue for Black
villancicos was linked to the practice across the Spanish and Portuguese
Empires of \quoted{Black Kings} festivals.
At Epiphany, confraternities of enslaved and free people of African descent
elected a mock royal court and paraded them around their city with music and
dancing.%
    \Autocite{Fromont:DancingKingCongo}
Their dances included military elements that originated from intercultural
exchange with the Portuguese in the Christian Kingdom of Kongo before the start
of slavery.%
    \Autocites
    {Fromont:Kongo}

\addtoindex{
    Africans!Black Kings Festivals;
    Epiphany
}

These pieces were created by Spaniards primarily for other Spaniards.
\quoted{Black villancicos} are not really about depicting African identity but
are rather ways of constructing a Spanish one by reference to the Other.
These pieces do not allow us direct access to subaltern voices, but as
caricatures they do tell us about the insecurities, fears, and prejudices of
Spaniards and may help us understand how they use theology to justify their
position in an unjust society.
%}}}3

%{{{3 VCs about VCs
\subsubsection{Villancicos about Villancicos}

The conventions of the villancico genre itself become the subject of a special
type of self-referential villancico \emph{about} villancicos.
In one sense, the many pieces beginning \quoted{Listen} or \quoted{Pay
attention}, might be considered self-referential, since in these pieces the
singers usually announce something about the piece, as in the setting of
\wtitle{Oigan, oigan la jacarilla} by José de Cáseda, or the poem performed in
Montilla in 1689, \wtitle{Oíganme cantar una tonadilla}---\quoted{hear me sing
a tonadilla}.% 
\begin{Footnote}
    Respectively, \sig{MEX-Mcen}{CSG.151}, \autocite[116 (no signature
    listed)]{BNE:VCs17C}.
    See \autocite{LeGuin:Tonadilla}.
\end{Footnote}
This rhetorical posture owes something to the genre's close
relationship with the psalms in Matins, which are filled with such
self-referential statements like \quoted{Sing to the Lord a new song},
\scripture{Ps 97:1}.
But it also draws on the comic and satirical elements of the Spanish minor
theater, the low-register plays (\term{entremeses}) performed between acts of
the more highbrow \term{comedias} by, for example, Lope de Vega and Calderón.%
    \Autocite{Cotarelo:Entremeses}
Similar to the Italian \term{intermezzi} skits of the eighteenth
century that were the cradle of comic opera, Spanish \term{entremeses} were
built out of formulaic scenarios and stock characters---many of the same ones
like Gil, Pascual, Bras, and Bartolo who appear in villancicos---and parodied
the conventions of the \term{comedia}.

\addtoindex{
    Music about music;
    \CasedaD;
    Villancico!Stock characters;
    Villancico!Conventions
}

The anonymous villancico \wtitle{Antón Llorente y Bartolo}
(\musref{mus:Anton_Llorente}) presents two characters from a well-known
\term{entremés} with close links to Cervantes' \wtitle{Don Quijote}, who want
listeners to hear out their complaint about villancicos.
The villancico poem is found in a 1639 Christmas imprint from Toledo Cathedral
and an anonymous musical setting survives from the Convento de la Santísima
Trinidad in Puebla.%
\begin{Footnote}
    \sig{E-Mn}{VE/88/6}, \sig{MEX-Mcen}{CSG.014}.
\end{Footnote}
The more well-known stock characters Gil and Bras, they say, have held the
stage for too long:
\begin{quotepoem}
    Antón Llorente y Bartolo	& Antón Llorente and Bartolo \\
    trazaron un memorial	& drew up a complaint \\
    de que con los villancicos	& that with all the villancicos \\
    se han alzado Gil y Bras.	& Gil and Bras have gotten the spotlight.
\end{quotepoem}
Anton Llorente and Bartolo insist that they could make a good enough villancico
of their own if given the chance:
\begin{quotepoem}
    Si ha de sonar el pandero,	& If the tambourine is going to be played, \\
    solo Gil le ha de tocar,	& it is only Gil who ever plays it, \\
    y si ha de haber castañetas,& it if there have to be castanets, \\
    ha de repicarlas Bras.	& Bras is the one to rattle them. \\
    También acá somos gentes	& But here we are, we too are good fellows, \\
    y alcanzar podemos ya	& and we can even manage \\
    de un villancico un bocado	& a nibble of a villancico \\
    y un pellizco de un cantar.	& and a pinch of a song.
\end{quotepoem}
In the succeeding \term{responsión} section, the full eight-voice chorus joins
in endorsing the new characters and denouncing the old:
\begin{quotepoem}
    No quiero que me Brasen y que me Gilen 
    & I don't want them to Bras me or Gil me \\

    sino que me Llorenten y me Toribien. 
    & but only to Llorente me and Toribio me.
\end{quotepoem}

\addtoindex{
    Cervantes Saavedra, Miguel de;
    Toledo!Cathedral;
    \PueblaConvento;
    Theater, minor;
    Entremes|see{Theater, minor}
}

The anonymous musical setting for this embodies all the stereotypes of the
villancico genre.%
\begin{Footnote}
    One possible composer is the Seville Cathedral chapelmaster Fray Francisco
    de Santiago, whose setting of this text was cataloged as part of the
    now-lost library of King John (João) IV of Portugal. 
    \autocite[caixão 26, \range{no}{675}]{JohnIV:Catalog}.
\end{Footnote}
Set against the anticonventional words, the music seems like an attempt to
\emph{represent} typical villancico style, a fitting way to portray Anton
Llorente and Bartolo performing a villancico.
This is a villancico, then, in the style of villancicos.

\addtoindex{
    Santiago, Francisco de;
    Seville!Cathedral
}

%{{{5 music Anton Llorente
\insertMusic{Anton_Llorente}
{Anonymous, \wtitle{Anton Llorente y Bartolo} (\sig{MEX-Mcen}{CSG.014}), first
stanza of introducción and beginning of responsión (Accompaniment omitted)}
%}}}5

As though the 1639 Toledo text were not self-referential enough, the creative
team at the cathedral followed up the next year with another villancico that
specifically referred back to \wtitle{Anton Llorente y Bartolo}.%
\begin{Footnote}
    \ptitle{Quejosos de la sentencia que dio el alcalde Pasqual}, in imprint
    from Christmas 1640 at Toledo Cathedral, \sig{E-Mn}{VE/88/7,
    \range{no}{2}}.
\end{Footnote}
The narrator says that the \foreign{Brases} and \foreign{Giles} were so
\quoted{frustrated by the sentence that Mayor Pasqual decreed against them last
Christmas}, that \quoted{they appealed to another one who was more learned}
(the \quoted{Mayor of Bethlehem} was another stock character in comic
villancicos).
Each one states his case for why he is needed at the Nativity, and Bras's
conclusion wryly sends up the conventionality of villancico poetry:
\begin{quotepoem}
Cuanto ha qué Belén lo es,	& As long as Bethlehem has been what it is, \\
y ha sido el portal portal,	& and the stable has been a stable, \\
a peligros de poetas		& where poets have been in danger, \\
ha sido socorro Bras.		& Bras is always there for aid.
\end{quotepoem}
The new mayor, in the name of keeping traditions, undoes the sentence of the
previous year, and the chorus rejoices, because without Bras and Gil it would
not be Christmas:
\begin{quotepoem}
Que me Brasen, y Gilen	& I wish them \\
quiero zagales,		& to Bras me and Gil me, lads, \\
porque no soy amigo	& because I am no friend \\
de novedades.		& of novelties.
\end{quotepoem}
The chorus, speaking here for the mayor's subjects in the community, affirms
the decision to keep their familiar Christmas characters:
\begin{quotepoem}
Porque en saltando a esta fiesta & For if you take from this feast \\
el pesebre, y el portal,  	 & the manger, the stable, \\
las pajas, Brases, y Giles, 	 & the straw, Brases, and Giles, \\
no es fiesta de Navidad.	 & it is no festival of Christmas.
\end{quotepoem}
Here we have a scene of people clamoring for villancicos with all their corny
conventions as a central part of making Christmas feel like Christmas.
As the mayor says, one reason villancicos were so conventional may be because
the feast they were most closely associated with was one where customs are
carefully preserved.
Part of cultivating those traditions meant naming them explicitly in song, as we
have already seen in Cererols's \wtitle{Fuera que va de invención}, like a
North American Christmas tree ornament in the shape of a Christmas tree.
The villancico's emphasis on having fun with Christmas customs still
contributed to a theological function, even though the piece presents no
learned doctrines.%
    \Autocite[161--185]{Illari:Polychoral}
Music that could prompt hearers to laughter and enjoyment could attract
parishioners and make them feel at home within the church community, and for
Catholics incorporating people into the church was nearly equivalent to the
gospel.

\addtoindex{
    Villancico!Social function!Entertainment;
    Humor
}
%}}}4
%}}}3
%}}}2

%{{{2 abstract references
\subsection{Abstract References to Music as Concept or Symbol}

%{{{3 overview
In the second category of metamusical villancicos are pieces that refer to
music more as an abstract concept, rather than to a specific, identifiable
reference to another kind of music.
Two early examples are \wtitle{Gil, pues a cantar} by court composer Pedro
Ruimonte (one of the few villancico settings to be printed) and \wtitle{Sobre
vuestro canto llano} by Gaspar Fernández, the Guatemala-born chapelmaster of
Puebla Cathedral before Gutiérrez de Padilla.%
    \Autocites
    {Ruimonte:Parnaso}
    {Fernandez:Cancionero}
    [for crucial emendations to the latter composer's biography, see][]
    {Morales:Fernandez}
When Ruimonte sets the word \foreign{cantar} (sing) to a long melisma, or when
Fernández illustrates the term \foreign{canto llano} (plainchant) with
imitative counterpoint around a Tenor part that sounds like a cantus-firmus,
these composers are using the characteristic emblems of vocal music to refer to
the concept of singing in general.

\addtoindex{
    Ruimonte, Pedro;
    \FernandezG;
    Fernandes, Gaspar|see{\FernandezG}
}
%}}}3

%{{{3 solfa
\subsubsection{Solmization Puns and Theology of Voice Alone}

One of the most common ways of explicitly singing about singing was to use
solmization syllables---\term{ut, re, mi, fa, sol, la}---in the poetry.
References to Christ as \term{sol} (sun) are ubiquitous, and as shown in the
opening example by Gutiérrez de Padilla, composers missed no opportunity to put
this word on a pitch that could be solmized with that syllable (G, C, or D in
the three Guidonian hexachords).
Solmization tropes brought the rudiments of musical artifice into the
foreground, forcing educated listeners to take note of the constructed nature
of what they were hearing.
No pun was too obvious.
In composer Miguel de Aguilar's \term{oposición}---his audition piece---for a
position at Zaragoza, \wtitle{Mi sol nace y tiembla}, any choirboy could have
guessed what he would choose for the opening pitches: E for \term{mi} and G for
\term{sol}.
\begin{Footnote}
    \sig{E-Zac}{B-11/233}, \ptitle{Villancico de Oposición en Zaragoza}, edited
    in \autocite[34--64]{Ezquerro:MME55}. 
\end{Footnote}
Solmization syllables were sometimes used for their own sake, without a
symbolic meaning, somewhat like the \quoted{fa la la} refrains in contemporary
English madrigals.
Here sign and signified become one: the voice bears no message except the
musical voice itself.

\addtoindex{
    Solmization;
    Theology!Voice;
    Aguilar, Miguel de; % XXX Águilar?
    Zaragoza
}

Passages of self-conscious solmization are not alluding to a particular kind of
song but to the abstract category of singing.
In Aguilar's \wtitle{Mi sol nace}, the words have dual function: on one side
they communicate linguistic meaning (\quoted{my sun}, which is itself
metaphorical), but on the other side these musical syllables go beyond
language, to both symbolize and embody music-making. 
Aguilar made this obvious gesture at the beginning of a piece intended to
demonstrate his own skill at composition, in keeping with the tradition we will
trace in \partref{part:unhearable-music} of Spanish composers using metamusical
villancicos to establish their compositional pedigree.

\index{Influence, Homage}

At the same time, the syllables themselves could also take on deep symbolic
meanings, as we will see in \chapref{ch:padilla-voces}.
In his 1672 music treatise Andrés Lorente uses the six solmization syllables as
an acrostic to help musicians tune \quoted{the spiritual music of the
person}---that is, to live with moral virtue.%
    \Autocite[689]{Lorente:Porque}
One hexachord leads up to communion with God; the other goes down to perdition.

\index{Lorente, Andrés}

The system of three hexachords offered additional symbolic potential because
they were seen as transpositions of each other.%
\begin{Footnote}
    These are overlapping six-note scales on the syllables \term{ut, re, me,
    fa, sol, la}.
    The natural hexachord began on C; the soft or \term{mollis} hexachord, on
    F; and the hard or \term{durus} hexachord on G.
    \Autocites{Judd:RenaissanceModalTheory}
    {Barnett:TonalOrganization17C}
    {Berger:Ficta}
\end{Footnote}
In 1678, Segovia Cathedral chapelmaster Miguel de Irízar began the festivities
of Christmas with the \term{calenda} piece, \wtitle{Qué música celestial}, in
which he used the hexachordal system to depict heavenly music coming down to
earth (\musref{mus:Irizar-Que_musica_celestial}).%
    \footnote{\sig{E-SE}{18/36}.}
The piece dramatizes the moment when the shepherds of Bethlehem first heard the
music of the angelic choir (\scripture{Lk 2}) by having three speakers ask,
in turn,
\quoted{What celestial music is this which alters the air?}
\quoted{What sovereign harmony is this which elevates hearing?}
\quoted{What light is this that transforms the dense night into day?}
Of course, as the first villancico of Christmas heard in Segovia Cathedral that
year, \quoted{this music} also refers to the music being performed in
the present.
Irízar gives the first voice to the Alto I, who sings down all the steps of the
soft hexachord from \pitch{D}{5} (\term{la}) to \pitch{F}{4} (\term{ut}).
The figure is an epitome of music itself, a textbook example of solmization
that begins at the very top of the Guidonian gamut (the second highest note on
the hand).
The second voice (Tiple I-1), then, imitates the first phrase exactly, but
transposed down a fourth into the natural hexachord (from \pitch{A}{4} to
\pitch{C}{4}).

\addtoindex{
    Music theory!Guidonian hexachords;
    Music theory!Symbolism;
    \IrizarM;
    Segovia!Cathedral;
    Heavenly music
}

%{{{5 music Irizar Que musica celestial
\insertMusic{Irizar-Que_musica_celestial}
{Irízar, \wtitle{Qué música celestial} (\sig{E-SE}{18/36}, Segovia Cathedral,
Christmas 1678), Opening}
%}}}5

The shift from the \quoted{altered} soft hexachord, with B flat, to the plain
natural hexachord, symbolizes the movement of music from heaven to earth.
Between the two singers, the two phrases outline the plagal ambitus of the
second mode (from A to A, with a final on D), thus presenting hearers with a
paradigm of perfect music, according to the most ancient of rules known to a
late-seventeenth-century Spanish chapelmaster.
Meanwhile, the bass line for the continuo accompaniment adds a further
heaven--earth contrast, as it moves in canon with the singers but with a
rhythmic displacement so that the bass and melody voices form a chain of 7--6
suspensions.
The way the bass voice moves at a delay from the solo voice suggests the way
earthly music imitates or echoes heavenly music.
On the other hand, the contrapuntal pattern is a textbook example of
fourth-species counterpoint, so it could also be a way of representing heavenly
music itself.  
This kind of heavenly music defies human expectations but is at the same time
governed by its own laws.
A listener untrained in counterpoint might only have perceived a mysterious,
haunting affect, and in any case the passage does evoke a \foreign{soberana
armonía} that \quoted{elevates the sense of hearing} or \quoted{lifts up the
ear}.

A little-known source from the Puebla Cathedral archive shows how much
musicians treasures this kind of ingenuity.
In a separate part of the archive from Gutiérrez de Padilla's other villancicos
or his Latin-texted polyphony is a small handwritten notebook containing an
anthology of exemplary works gathered by an unknown musician, apparently a
student to judge from the immature handwriting.
After selections from Palestrina is a set of \quoted{Villancicos of various
authors}, in which there is copied just the tenor part of a setting by
Gutiérrez de Padilla, \wtitle{Miraba el sol el águila bella}.
%    \citXXX[signature, check info, esp. attribution]
The part is a virtuoso demonstration of solmization puns.
The text of the estribillo and responsión is made almost entirely of
solmization syllables, as in \foreign{y ella al sol mire y la mire el sol}.
Gutiérrez de Padilla sets every syllable to the corresponding pitches so that
for much of the piece, singing the lyrics is almost identical to singing the
solmization (\musref{mus:Padilla-Miraba_el_sol-estribillo}).

\addtoindex{
    \GdP;
    Puebla!Cathedral;
    Solmization
}

%{{{5 music GdP Miraba el sol estribillo
\insertMusic{Padilla-Miraba_el_sol-estribillo}
{Gutiérrez de Padilla, \wtitle{Miraba el sol}, extant Tenor part from
manuscript anthology (\sig{MEX-Pc}{Leg. 34}), estribillo: Melody matching
solmization syllables in text (Abbreviations for hexachords: \term{NAT},
\term{naturalis} on C; \term{MOL}, \term{mollis} on F; \term{DUR}, \term{durus}
on G; \term{FIC}, \term{ficta} alteration}
%}}}5 

But the piece is not nonsense---in fact, the poet (perhaps Gutiérrez de Padilla
himself) has managed to craft a semantically and theologically coherent text
based on an entirely separate conceit relevant to the feast of the Conception
of Mary, that of the Virgin Mary as an eagle.
The eagle, Spaniards believed, had the power to look directly at the sun
without harming its eyes, and thus the eagle was a fitting symbol for Mary as
Immaculate.%
    \Autocite[\sv{águila}]{Covarrubias:Tesoro} % XXX check
That Mary was conceived without original sin was enforced as official dogma in
Spain long before the rest of the church approved it, and Puebla Cathedral was
dedicated to Mary as Immaculate.
Gutiérrez de Padilla takes advantage of the hexachordal system, which means
that there are usually three possible notes that could be solmized with a
particular syllable, to add an additional symbolic layer embodying the eagle
conceit through musical technique.
To represent the eagle turning to the sun (and therefore Mary seeing the face
of God), Gutiérrez de Padilla has the Tenor shift from the soft hexachord,
through a ficta alteration, into the natural hexachord
(\diaref{dia:Padilla-Miraba_el_sol-hexachords}).
He thus quite literally moves \foreign{al sol}---both because he moves to a
note on \term{sol} and because he shifts to the hexachord that starts on the
\term{sol} of the previous hexachord.
Where Irízar shifted from the soft hexachord down to the natural for moving from
heaven to earth, Gutiérrez de Padilla makes the same shift upwards to represent
the eagle/Mary looking up to the heavens.

\index{Sanctoral devotion!Mary}

%{{{5 dia GdPadilla Miraba el sol hexachords
\insertDiagram{Padilla-Miraba_el_sol-hexachords}
{Gutiérrez de Padilla, \wtitle{Miraba el sol}, estribillo: Hexachordal shift
symbolizing Mary/Eagle in \term{mollis} turning to
Christ/Sun (\term{sol}) in \term{naturalis}}
%}}}5

Even when solmization passages seem to lack lexical or symbolic meaning, they
bear theological meaning as an embodiment of the voice itself, within the
Neoplatonic system (explained more fully below).
The voice expressed the essence of Man as the microcosm and a reflection of the
Creator, a meaning it communicated independent of linguistic expression or
musical-rhetorical patterns.
Far from \quoted{signifying nothing}, as in the \quoted{aesthetics of pure
voice} that Mauro Calcagno identifies in the contemporary Venetian theater
productions of the Accademia degli Incogniti, and much less any modern
philosophical notions of the \quoted{voice itself} as separate from meaning,
wordless passages in villancicos come nearer to signifying everything.%
    \Autocites
    {Calcagno:SignifyingNothing}
    {Feldman:Voice}
    {Barthes:GrainOfVoice}
    {Dolar:Voice}
    {Cavarero:Voice}
%}}}3

%{{{3 music itself as a conceit
\subsubsection{Music Itself as a Conceit}

Solmization villancicos should be understood as a subtype of a category of
villancicos in which music itself is the central conceit---not a specific type
of music (human, animal, or angelic), but music in the abstract.
Such pieces often play on technical musical terms using the technique of
\term{conceptismo} to create a double discourse about both music and theology.
The most renowned of villancico poets today, Sor Juana Inés de la Cruz
(1651--1695), used the conceit of Mary as a heavenly chapelmaster to create
such a piece for the feast of the Assumption in Mexico City, 1676, though no
musical setting survives.% 
    \Autocite[\range{no}{220}, \range{p}{7}]{SorJuana:VC} 
The estribillo exhorts congregants to listen for Mary's voice
(\poemref{poem:Silencio_Maria-Sor_Juana}).
In the second copla Sor Juana uses the hexachord to trace the arc of Mary's
exaltation from Annunciation to Assumption. 
Mary begins at \term{ut} with her response to the angel that she will bear
Christ (\term{Ecce ancilla}, \bibleverse{Lk}{1:46-55}), and ends on the
\term{la} of the verse that recurs throughout the liturgies for the Assumption,
\wtitle{Exaltata est sancta Dei Genitrix/ Super choros Angelorum ad caelestia
regna} (The holy bearer of God has been exalted above the choirs of angels to
the heavenly realm).
Copla 4 uses the dual identity of B as either natural (\term{mi} in the hard
hexachord) or flat (\term{fa} in the soft hexachord or in ficta) as a symbol of
Christ's dual nature as both human and divine, united in Mary's womb.
Sor Juana demonstrates her own considerable learning in both music theory and
theology---not to mention the elegance of her poetic craft---and also reveals
how much she is expecting her readers to know.%
    \Autocites
    {Stevenson:SorJuanaMusicalRapports}
    {Tenorio:SorJuana}

\addtoindex{
    Juana Inés de la Cruz, Sor;
    Sanctoral devotion!Mary;
    Music theory!Symbolism
}

%{{{5 poem sor juana silencio
\insertPoem{Silencio_Maria-Sor_Juana}
{Sor Juana Inés de la Cruz, \wtitle{Silencio, atención, que canta Mariá},
excerpts}
%}}}5

When poetry like this was set to music, composers had the opportunity to match
this intricate musical-theological discourse with another layer of symbols in
the sounding music.
This conjunction of verbal and musical play on musical concepts was no
accident: the texts of villancicos were written specifically as lyrics for
musical compositions, as Juan Díaz Rengifo stated in one of the earliest
literary descriptions of the genre, and composers had every reason to favor
poems that gave them opportunities for clever musical crafstmanship.%
    \Autocite{Rengifo:ArteMetrica}
Juan Gutiérrez de Padilla (\chapref{ch:padilla-voces}) and Joan Cererols
(\chapref{ch:cererols-suspended}) both took finely wrought, Gongoresque texts with
musical conceits and added a rich musical commentary on those conceits in their
intricate settings.
Each chapter in \chapref{part:unhearable-music} traces a family of related
villancicos with the same or similar texts and demonstrates that this type of
high-concept metamusical villancico served a special purpose for Spanish
musicians, enabling them to situate themselves within a tradition of
composition.

\index{Díaz Rengifo, Juan}

%}}}3

%{{{3 heavenly, angelic music
\subsubsection{Pointing to a Higher Music: Heavenly and Angelic Music}

Thus far we have seen how Spanish composers represented other kinds of music
like birdsongs, instrumental music, and dances of different social groups
within the villancico genre; and how they created songs that pointed to
themselves, whether by parodying the genre's own conventions or by using
solmization to draw listeners' attention to the act of singing itself.
How, then, did composers use villancicos to refer not to any kind of earthly
music, but rather to point to celestial (planetary) and heavenly (angelic,
divine) music?
When a villancico referred to the music of the spheres or to angelic music, the
music signified was impossible to hear with earthly ears, so the human music
would only function as a sign to the extent that a listener believed it to
correspond to what those higher forms of music sounded like, or understood them
to be mere imitations of something higher.

\index{Heavenly music}

As we will trace in \partref{part:unhearable-music}, Spanish composers developed a
family of conventional tropes for evoking heavenly music, and one of the msot
common was to set up a contrast between stylistic allusions to distinct types
of human music with different values in a hierarchy of musical styles.
The most elevated form of earthly music, learned counterpoint in the by-then
classic style of Palestrina, was typically contrasted with more worldly types
of music, such as the rhythms of dance and the melody-and-accompaniment style
of popular songs.
This hierarchy of human musics was mapped on to the greater hierarchy of
earthly and heavenly music, so that old-style counterpoint stood in for
angelic and divine music, though really it was the contrast between musical
topics that gave it this meaning.%
\begin{Footnote}
    This approach was used across confessional lines in early modern Europe,
    with well-known Lutheran examples by Heinrich Schütz, Dieterich Buxtehude,
    and J. S. Bach (all in music envisioning heavenly bliss after death):
    \autocites
    {Johnston:Rhetorical}
    {Yearsley:Buxtehude}
    {Yearsley:BachCounterpoint}.
    Lutherans cultivated a literature and iconography of musical encomium that
    overlaps in many ways with metamusical villancicos, partly because it drew
    on the same sources:
    \autocite{Schmidt:Lob_der_Musik}.
\end{Footnote}
We have already seen this when Juan Gutiérrez de Padilla contrasts the music of
Angolans and angels in his 1652 ethnic villancico.
A typical example of the angelic trope is \wtitle{Angélicos coros con gozo
cantad}, a Christmas villancico by Antonio de Salazar from a Conceptionist
convent in Puebla (\musref{mus:Salazar-Angelicos_coros-1}).%
\begin{Footnote} 
    \sig{MEX-Mcen}{CSG.256}; edited in \autocite{Cashner:WLSCM32}.  
\end{Footnote}
Salazar (\circa{1650}--1715) was probably born in Puebla and may have sung in
the Puebla Cathedral chapel under Gutiérrez de Padilla; he served as
chapelmaster of Mexico City Cathedral from 1679.%
    \Autocite{Koegel:Salazar} 
The convent collection features numerous pieces by Salazar, possibly composed
or arranged specifically for this community.
The convent was closely linked to the cathedral, and in fact some of the
villancico parts by Gutiérrez de Padilla in the cathedral archive still bear
the sisters' names, from some occasion when the parts were taken to the convent
for the women to perform there.%
    \Autocites
    {Favila:Profession}
    {Tello:SanchezGarzaCatalogo}

\addtoindex{
    \PueblaConvento;
    Convent music;
    Salazar, Antonio de;
    Mexico City!Cathedral;
    Counterpoint!Symbolism
}

    %{{{5 music Salazar Angelicos coros 1
\insertMusic{Salazar-Angelicos_coros-1}
{Salazar, \wtitle{Angélicos coros con gozo cantad} (\sig{MEX-Mcen}{CSG.256}),
opening}
%}}}5

The anonymous poem echoes the first Responsory of Christmas Matins
(\wtitle{Gaudet exercitus Angelorum}) as it invites the choirs of Christmas
angels to sing their \quoted{Gloria} over the stable in Bethlehem on the night
of Christ's birth.
Since \mentioned{Bethlehem} in Hebrew means \quoted{House of Bread}, the
villancico also celebrates the sacramental presence of Christ in the
Eucharistic host on the Christmas of Salazar's \soCalled{present day}.
Though the words speak to the angels, the musicians who sing these words also
play the part of the angels, so that hearers are invited to listen for the
angelic voices \emph{through} the voices of the church ensemble. 
The call to the angels is sung first by the Tiple I, in a gesture beginning
with a rising fifth and then falling by step, as though looking up to the
heavens and then following the angels' descent.
In the Puebla convent choir, this part was performed by \quoted{Madre Andrea},
whose name is written into her part.
As though answering the call, the other two voice parts of Chorus I enter in
\measure{3}, Tiple II in canonic imitation, and Alto I harmonizing with it
homorhythmically. 
In \measure{7} the second chorus joins with a similar imitative
pattern, until all join together in a lilting, dancelike cadence on
\foreign{cantad}.
Salazar uses contrapuntal imitation again on \foreign{celestes esquadras},
inverting the opening motive (\measure{26}).
For the command \foreign{bajad} (come down), Salazar switches from CZ
triple meter to duple (C or \term{compasillo}), and creates a cascading
contrapuntal passage passed from voice to voice, moving from high F\octave{5}
down to C\octave{3} (\musref{mus:Salazar-Angelicos_coros-2}).
The general affect of the piece seems gentle and sweet, partly because of the
largely static diatonic harmony and the lilting or dotted rhythms.

\index{Musical topics!Angels}

%{{{5 music Salazar Angelicos coros 2
\insertMusic{Salazar-Angelicos_coros-2}
{Salazar, \wtitle{Angélicos coros con gozo cantad} (\sig{MEX-Mcen}{CSG.256}):
Angels descending in imitative counterpoint}
%}}}5

Puebla, the original American \quoted{city of angels}, was built on a site
believed to have been revealed by angels to the bishop of Tlaxcala, and
buildings and artworks dedicated to the angels are everywhere in the city.%
    \Autocites
    {AngelContreras:Puebla}
    {Garcia-Castellanos:Puebla-Utopia}
    {Davies:HarmonyConversion}
Salazar belonged to the Confraternity of Saint Michael the Archangel in Puebla,
according to a printed sermon by Fray Andrés de San Miguel of the same city.
Andrés preached the sermon on Saint Michael, \ptitle{The Chapelmaster of the
Music of the Angels}, to a gathering of Salazar and his fellows in the Church
of the Incarnation.%
    \Autocite[65--95]{SanMiguel:Sermones} % XXX check
The friar's self-deprecating introduction, saying he does not really know enough
about music to address such a group, makes it clear that he is addressing an
elite congregation of accomplished men with practical and theoretical knowledge
of music.
The sermon uses the six solmization syllables as an acrostic device to
teach about Saint Michael: \term{ut} reminds us of the angel's Hebrew name,
rendered in Latin as \foreign{Quis ut Deus}; \term{re} (king) reminds us that
Michael is the chief of all the angels; and so on.
Don Salazar, he says, could compose a better sermon in music than he himself
could preach in words, indicating both his high respect for the chapelmaster
and giving a hint to how a theologically educated musical amateur listened to
church music.

\addtoindex{
    Confraternities;
    Sanctoral devotion!Michael, Archangel;
    Villancico!Social functions
}

We do not know what music may have been performed at this service, but
Salazar's villancico \wtitle{Angélicos coros} would seem an appropriate choice.
The surviving version was composed or arranged for the Convento de la Santísima
Trinidad, another semi-private music venue with elite, high-level musical
performance.
In this and other angel pieces, the singers turn their attention heavenwards to
address the angels directly, while they also stand in for, or sing along
with, their heavenly counterparts.
Angel pieces exhort the audience to lift their ears upwards as well and listen
for a higher music.
%}}}3

These many ways of using one kind of music to represent another amount to a
musical form of \term{conceptismo}.
Just as the metaphor in a \term{conceptista} poem asks readers to reflect
on the hidden resemblance between two apparently distinct concepts, metamusical
villancicos bring together different types of musical references in a way that
asks hearers to reconsider the nature of music itself.
They invite an intentional, active process of theological listening.
What they have to say theologically, they communicate through their music to
anyone willing to engage with both aspects.
In short, they are musical theology.

\index{Hearing}
%}}}2
%}}}1

%{{{1 theology of music
\section{Theological Listening in the Neoplatonic Tradition}

%{{{2 intro sources luis, kircher, augustine, boethius
Villancicos on the subject of music consistently manifest a Neoplatonic
theological worldview, an understanding of which is necessary to grasp the
genre's religious functions.
Having drawn out aspects of this theology inductively from the examples of
metamusical compositions, in the final section of this chapter we may turn to
theological literature to establish a more systematic foundation.
The sixteenth and seventeenth centuries in Spain brought a revival of interest
in Neoplatonic theology in the tradition of Augustine, partly because of new
printed editions of his works.%
    \Autocite{Weber:ReligiousLitSpain}
Christian Neoplatonists followed Augustine in viewing the material
world as a reflection of a higher spiritual reality which ultimately had its
source in the Supreme Good which was the Godhead.%
\begin{Footnote}
    An important later source for this concept is the \wtitle{Spiritual
    Hierarchy} attributed to Dionysius the Areopagite.
%    \citXXX[neoplatonism lit]
\end{Footnote}
The material world reflected higher truths only imperfectly, but nevertheless
this world was also the only means through which those truths could be reached.
In connection with Catholic sacramental theology, material objects and physical
actions became means through which humans could encounter divine grace.
Neoplatonic contemplation could be understood as a dialectical process of
discerning the degree both of similarity and of dissimilarity between earthly
objects and heavenly truth.%
\begin{Footnote}
    I use terms like \mentioned{contemplation} and \mentioned{reflection} to
    refer to practices of listening, reading, and thinking in which a person
    seeks to understand and experience a higher meaning behind what is evident
    to their senses. 
    This broad definition overlaps with the much more precise technical use of
    the term \mentioned{contemplation} in mystical theology, where it refers to
    a state in which a soul communes with God and becomes conscious of God's
    nature in a way that surpasses all sensation and imagination.
    That second kind of contemplation is a more concentrated, disciplined
    version of the first kind.
\end{Footnote}

\addtoindex{
    Theology!Sacramental;
    Theology!Contemplative;
    Neoplatonism;
    Dionysius the Areopagite
}

Neoplatonic thought was promulgated in Spain through the copious writings of
the Dominican friar Luis de Granada (known as Fray Luis in Spanish), one of the
most widely read authors in the empire.%
    \Autocite{Weber:ReligiousLitSpain}
His work is a self-acknowledged synthesis of patristic and Classical
sources as well as a summary of common beliefs of his own time.
The 1589 \wtitle{Introducción del Símbolo de la Fe} (Introduction to the Creed)
presents a Neoplatonic theological interpretation of the created world.%
    \Autocites
    {LuisdeGranada-Balcells:SimboloPtI}
    {LuisdeGranada:Simbolo}
Fray Luis teaches that the natural world is a reflection of a higher
truth---God's own nature---and that the creation was given so that by
reflecting on it people would come to know its Creator.
In the friar's theology, music embodies the harmonious order of creation; he
suggests music could provide a way to contemplate creation and its
Creator.

\addtoindex{
    Granada, Fray Luis de;
    Luis de Granada, Fray|see{Granada, Fray Luis de};
    Theology!Music
}

Writers on music, too, developed a Neoplatonic approach, building on the
concept of the medieval philosopher Boethius that there were three kinds of
music (\tabref{tab:Neoplatonic-hierarchy-music}).%
    \Autocite{Boethius:Musica}
At the lowest level is \term{musica instrumentalis}---music played and sounded,
music that humans can hear.
Higher up is \term{musica humana}---the harmony of body and soul, and of one
human being with another in society.
Still higher is \term{musica mundana}---the harmonies created by the perpetual
movement of the planetary spheres.
The three Boethian types of music are arranged hierarchically and each one
points beyond itself to a higher level.
The treatises used to teach musical composition in seventeenth-century Spain,
most notably Pedro Cerone's \wtitle{El melopeo y maestro} (1613) and Andrés
Lorente's \wtitle{El porqué de la música} (1672), present music within this
cosmology of music.

\addtoindex{
    Boethius;
    Musica humana|see{Boethius};
    Musica mundana|see{Boethius};
    Lorente, Andrés;
    Cerone, Pedro
}

A key source for Neoplatonic thought on music is the \wtitle{Musurgia
universalis} of 1650 by the Jesuit polymath Athanasius Kircher.% 
\Autocite{Kircher:Musurgia}
This encyclopedic treatment of \quoted{the working of music} in all its aspects
was disseminated through Jesuit networks across the globe: a copy was sent as
far as Manila, and two copies are preserved today in Puebla.%
    \Autocites
    {Findlen:Kircher}
    {Godwin:KircherTheater}
    [48--50]{Irving:Colonial}
Kircher describes in detail the latest scientific knowledge about the anatomy
of hearing and vocal production and the physiology of bodily humors and
affects. 
He tries to explain precisely how particular musical structures
work through these bodily systems, and he presents all of this within a cosmic
view of music as part of the Ptolemaic universe.

\addtoindex{
    Kircher, Athanasius;
    Science
}

%{{{5 table Neoplatonic hierarchy
\insertTable{Neoplatonic-hierarchy-music}
{Hierarchy of types of music in Neoplatonic thought, after Boethius (read from
bottom up)}
%}}}5
%}}}2

%{{{2 hearing book of nature
\subsection{Hearing the Book of Nature Read Aloud}

Fray Luis de Granada begins his \wtitle{Introduction to the Creed} with an
epitome of Neoplatonic-Augustinian thought: \quoted{The ultimate and highest
good of man}, he writes, \quoted{consists in the exercise and use of the most
excellent work of man, which is the knowledge and contemplation of God}.%
    \Autocite[182]{LuisdeGranada:Simbolo}
The created world, he teaches, is a \quoted{book of nature} in which is written
the grandeur, love, wisdom, and faithfulness of its Creator.
The first goal of humankind, then, is to learn to read this \quoted{book of
nature} in order to come through it to the knowledge of God. 
The goal of contemplating creation is \quoted{ascending by the staircase of the
creatures to the contemplation of the wisdom and beauty of the Maker}.%
    \Autocite[184]{LuisdeGranada:Simbolo}

\addtoindex{
    Book of nature|see{Creation};
    Augustine, Saint;
    Granada, Fray Luis de;
    Reading
}

The reason one can \quoted{read} God through nature, Fray Luis teaches, is that
the created world is a reflection of God's perfect order---a concept the friar
repeatedly expresses using musical metaphors.
Fray Luis compares the perfect order of nature to a harmonious musical
composition in which everything fits together \foreign{con sumo concierto}
(with the most perfect concord).
All the created things in this world, Fray Luis writes, \quoted{like concerted
music for diverse voices, harmonize together \addorig{concuerdan} in the
service of man, for whom they were created}.%
    \Autocite[191]{LuisdeGranada:Simbolo}
The movement of the heavenly spheres, and their effects on the earth, are like
a great \quoted{chain, or, it can be said, this dance, so well ordered, of the
creatures, and like music for diverse voices \Dots.
Because things so diverse could not be reduced to a single end with a single
order, if there were not one who was like a chapelmaster \addorig{maestro de
capilla}, who reduces them to this unity and consonance}.%
    \Autocite[191]{LuisdeGranada:Simbolo}

\index{Theology!Music}

When Fray Luis compares the world to music \quoted{in diverse voices} he
obviously has in his \quoted{mind's ear} polyphonic music of his own time, such
as he would have heard at the Portuguese Royal Chapel as confessor to the
queen.
%    \citXXX[names of composers]
Likewise, when he compares God to a \foreign{maestro de capilla}, that has all
the implications of that office in the Iberian context, which included
composition, teaching, and leading the choir in some form of conducting.
According to Fray Luis's metaphor, then, God is creator, prime mover, and
sovereign ruler over creation, actively and intimately involved in its ongoing
progress.

\index{Royal Chapel, Portuguese}

At the same time, these references to music are more than metaphors, since the
universe for Neoplatonists is not only like music, it actually is musical in
its structure.
For Fray Luis, not only does creation reflect God's order; it actively
proclaims that fact.
It speaks or sings with its own voice to communicate God's glory to the human
who knows how to listen.
Paraphrasing Augustine's preaching, the friar writes: \quoted{Look around at
all these many things from the heaven to the earth, and you will see that they
all sing and preach their Creator; because all types of creatures are voices
that sing his praises}.% 
    \Autocite[185, glossing Augustine's commentary on \scripture{Ps 26}]
    {LuisdeGranada:Simbolo} 
While the full knowledge of God can only come with the aid of divine revelation
through the Scriptures and the Church, Fray Luis praises God that humans can
study his nature in \quoted{the university of created things, which declare to
us \add{literally, \quoted{give us voices}} that you love us, and teach us why
we should love you}.% 
    \Autocite[186]{LuisdeGranada:Simbolo}
Fray Luis acknowledges, however, that apart from angels and birds, most of
creation is mute and does not literally have its own voice with which to
communicate its message of divine glory.
This \quoted{message} is not a linguistic one, but rather, their message is
simply themselves: in the created world, the medium is the message.
\quoted{Now these admirable works do not speak or testify this with human
voices \Dots}, Fray Luis writes, \quoted{rather their speech and testimony is
their invariable order and their beauty, and the artifice with which they are
so perfectly made, as though they were made with a ruler and plumb line}.%
    \Autocite[192]{LuisdeGranada:Simbolo}

In this theological system, music has unique value because it actually provides
a voice through which creation can make audible its message-of-being.
If nature was a book, for a thinker in 1580s Spain, then it was meant to be
read out loud.
As Margit Frenk has documented, books in this period were not read silently,
but required someone to give them voice, and tomes like Fray Luis's devotional
books were written with that intention.%
    \Autocite{Frenk:Voz}
% One edition of Fray Luis's own book \wtitle{Doctrina Cristiana} was published
% with a notice at the beginning from the Archibishop of Toledo granting a
% certain number of days of indulgence for each paragraph that anyone
% \quoted{read or heard read} (that is, had read to them).%
%     \citXXX[signature]
To read the \quoted{book of nature}, therefore, someone must perform it
vocally---and this is what music could do.
In the Christian Neoplatonic tradition, human music unlocks the musical voice
contained within the substance of created things.
Through metal pipes, horns, and bells; through wood viol cases, gut strings,
and skin drums; even through reverberant stone church walls, the very matter of
creation is made to resound with the perfectly ordered mathematical-harmonic
proportions placed within it by the Creator---proportions which themselves
reflect God's own perfect order.

\addtoindex{
    Theology!Voice;
    Musical instruments!Symbolism;
    Reading
}
%}}}2

%{{{2 voice microcosm
\subsection{Voice as Expression of Man, the Microcosm}

If pipes and strings testify to the order of creation, then the human body as
the microcosm of creation is the ultimate instrument through which nature is
given voice.
Fray Luis concludes his exposition of the six days of creation (based largely
on the \emph{Hexameron} of Saint Basil) by saying that God's creation of man on
the sixth day was like the conclusion of an oration, when the speaker draws
together all his themes into a final epitome.
Thus man is the summation of all that God had created in the previous five days
and encompasses them all within himself.% 
    \Autocite[243]{LuisdeGranada:Simbolo}

\index{Theology!Body}

When Athanasius Kircher (in the tenth book of the \emph{Musurgia}) continues
this hexameral tradition with his own treatment of the six days of creation, he
replaces the rhetorical metaphor with a musical one.
Instead of creation being God's oration, Kircher presents it as a musical
improvisation (a \quoted{Praeludium}) on God's cosmic organ.%
    \Autocite[\range{vol}{2}, 366--367]{Kircher:Musurgia}
On the sixth day, Kircher says, God recapitulates all his themes and pulls out
all the stops by creating man.
As with Fray Luis, Kircher's comparison to music must be based on some actual
music he knew. 
His description closely resembles the structure of a Praeludium by the
likes of Dieterich Buxtehude, which develops a motivic kernel through various
sections and culminates in a fugue for the full organ.
In Kircher's worldview, all the systems and elements of creation (stars,
planets, humors, rocks, animals, and so on) intersect in the individual human
body.% 
    \Autocite[\range{vol}{2}, 402]{Kircher:Musurgia}

\addtoindex{
    Creation;
    Buxtehude, Dieterich
}

For Kircher, the human voice is the unique expression of the individual,
reflecting each person's unique temperament and blend of the four humors.%
    \Autocite[\range{vol}{1}, 23--24]{Kircher:Musurgia}
Kircher defines the voice thus: \quoted{The voice is a living sound \add{or,
sound of the soul}, produced by the glottis through the percussion of respired
breaths that serve to express the affects of the soul}.% 
    \Autocite[\range{vol}{1}, 20]{Kircher:Musurgia}
Since each voice is unique, only in concert do voices fully reflect nature and
nature's God.
Cantus, Altus, Tenor, and Bass parts provide a place for all types of human
voices, Kircher explains, and correspond respectively to fire, air, water, and
earth.
Thus they form a choral microcosm both of humanity and of all creation.%
    \Autocite[\range{vol}{1}, 217--219]{Kircher:Musurgia}

\addtoindex{
    Kircher, Athanasius;
    Microcosm;
    Theology!Voice
}

Fray Luis also ventures an explanation of the human voice in both musical and
theological terms.
He exalts the voice as the audible expression of the human body and
vocal music as the most perfect kind of music.%
    \Autocite[243]{LuisdeGranada:Simbolo} 
Fray Luis praises the human voice as the highest of all musical instruments
(indeed, as the paradigm for them), as a means of forming social relationships
between people, and as a form of communication between human and divine.
The voice is produced, he explains, when air from the lungs moves through the
narrow opening of the voicebox, a design imitated in the construction of flutes
and dulcians.
The friar would have heard these instruments regularly in church.
But unlike carved woodwinds, the body can change its shape to produce
different kinds of voices, and this fact \quoted{is something that declares the
power and the wisdom of that sovereign artisan, who in such a manner forged the
flesh of this windpipe so that in it could be formed a voice sweeter and milder
than that of all the flutes and instruments that human industry has invented}.%
    \Autocite[252]{LuisdeGranada:Simbolo}
The voice therefore expresses human individuality, and voices of different
types in concert enact harmony between people: 
\begin{quoting} 
    And there is no end of admiration for the variety that there is in this for
    the service of harmonious music \addorig{música acordada}.  
    For some throats are narrow, in which are formed the trebles
    \addorig{tiples}, and others in which are formed voices so full and
    resonant that they seem to thunder through an entire church, without which
    there could not be perfect music.
    All of which that divine presider traced and ordained, so that with this
    mildness and melody the divine offices and their praises should be
    celebrated, with which to awaken the devotion of the faithful.%
        \Autocite[252]{LuisdeGranada:Simbolo} 
\end{quoting}
Fray Luis wants his readers to hear God's glory reflected most fully in the
concerted harmony of diverse human voices, which he says were created for the
purpose of singing in divine worship.
The voice in church is the definitive example of vocal music for Fray Luis.
Sacred polyphony glorifies God, then, simply by realizing the potential for
which the voice (and the body) was made.

Fray Luis sees speech as something \quoted{added} to the voice, which makes it
possible for the voice to communicate and form social relationships:
\begin{quoting} 
    Now here it is to be noted that when to the voice which proceeds from this
    place is added the instrument of the tongue, we come to articulate and make
    distinctions with this voice, and thus is formed speech, serving us by this
    instrument and punctuating \addorig{hiriendo} with it sometimes in the
    teeth and other times in the interior of the mouth.
    And just as the flute produces different sounds by touching on different
    holes, likewise the tongue, touching in different parts of the mouth, forms
    different words.  
    By this manner the Creator gave us the faculty to speak and communicate our
    thoughts and concepts to other men.% 
        \Autocite[252]{LuisdeGranada:Simbolo}
\end{quoting}
Fray Luis might see music---with its own system of articulations and
distinctions---as another way to \quoted{communicate our thoughts and concepts}
just as well as spoken language, but he also presents music as a product of the
voice before any articulation is added.
This definition of voice would mean that in vocal music there are always two
layers---the articulated \quoted{speech} aspects, and behind these the wordless
sustained voice.
Citing Augustine's \wtitle{De doctrina christiana} (the classic exposition of
Christian preaching and teaching), Fray Luis---who was himself the author of
six volumes about \wtitle{Rhetorica ecclesiastica}---says that the main task of
the student of rhetoric is to hear and identify the rhetorical tropes and
techniques used by another orator.
In the same way, he says, the first task of humankind is to be a student of the
natural world, and to learn to recognize in creation the signs of God's
artifice as the Creator, which manifest his glory.

\addtoindex{
    Speech;
    Granada, Fray Luis de;
    Rhetoric;
    Augustine, Saint
}

This would mean that the hearer of music could and should seek out this level
of musical structure while listening.
In a polyphonic vocal piece like a villancico, the bulk of musical structure is
borne by the sustained tones of the voice, singing vowels.
Apart from the words being sung, musical elements like mode, meter, motivic
development, and stylistic or topical allusions are all communicated by these
musical voices, and not simply by the voice as the bearer of words.
Music could thus reflect the divine through its sonic structure, apart from any
sacred linguistic meaning that may be attached as well.
If music's value and sacredness are not comprised solely in the words being
sung, then one must know how to hear the musical structure in order to receive
the full benefit.

\index{Hearing}

Listening to music within a Neoplatonic worldview, according to these
theological sources, may be summarized as follows:
\begin{enumerate}
\item Music is a reflection of the natural order.
\item The natural order is itself a reflection of God.
\item By paying attention to nature one can come to know and believe in its
    Maker.
\item Therefore listening to music may be a primary way of \quoted{reading the
    book of nature} and coming to faith in nature's Creator.
\item Music, especially vocal music, conveys sacred meaning to those who know
    how to listen, even apart from words.
\item The performance of music actively creates concord in society and between
    people and God.  
\end{enumerate}

Metamusical villancicos explicitly emphasize the challenge that was central to
all music-making in the Christian Neoplatonic tradition, to use the imperfect
medium of sounding music to evoke all the higher forms of music, to lead
listeners in contemplation up the chain of being beyond simply what was heard.
The recurrence of the \quoted{Listen!} exordium in villancicos may indicate
that the genre itself was fundamentally about getting people to listen.
Metamusical pieces drew listeners' attention to the artifice of the music they
were hearing and pointed them toward a higher form of music.
These villancicos invited listeners not simply to hear, but to \quoted{take
heed}, to both discern deeper meanings in what they heard and to
put what they hear into practice.
Understanding how Spanish Catholics made these links between hearing and faith,
and the challenges they faced in using music to connect one to the other, will
be the challenge of the next chapter.
%}}}2
%}}}1

\endinput
