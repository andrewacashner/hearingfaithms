% Cashner, *Faith, Hearing, and the Power of Music*,
% chapter 1: Villancicos as Musical Theology
% 
% 2018-05-21    Converted back to LaTeX
% 2017-11-15    New start for book proposal

\part{Listening for Faith}
\label{part:faith}

\chapter{Villancicos as Musical Theology}
\label{ch:intro}

\epigraphTranslation
{ergo fides ex auditu\\
auditus autem per verbum Christi}
{Faith, then, comes through hearing, \\
and hearing, by the word of Christ.}
{Romans 10:17}

Saint Paul wrote to the Christian community in Rome, \quoted{How are they to
believe if they have not heard?} since \quoted{faith comes through hearing, and
hearing, by the Word of Christ} (Rm 10:16--17).%
\begin{Footnote}
    This is my own translation from the Latin Vulgate used by Spanish Catholics, in
    the modern edition, \autocite{Weber:Vulgate}.
    The original Greek is \emph{ara ē pistis ex akoēs, ē de akoē dia rēmatos
    Xristou}: \autocite{Aland:GNT4}
    The word \emph{akoē} can mean \quoted{the faculty of hearing}, \quoted{the
    act of hearing}, \quoted{the organ with which one hears}, or \quoted{that
    which is heard}: \autocite{BDAG}.
    The New Revised Standard Version translates this \quoted{So faith comes from
    what is heard, and what is heard comes through the word of Christ}.
    Early modern Catholic discussions of faith and hearing depend on the range of
    meanings of \emph{auditus} in Latin (as in the underlying Greek), including both
    \quoted{hearing} and \quoted{what is heard}.
\end{Footnote}
Sixteen centuries later, amid the ongoing reformations of the Western Church,
Catholic Christians were seeking ever new ways to make faith audible. 
Poets and composers of the Spanish Empire expanded a genre of sung poetry in the
vernacular---the \emph{villancico}---into large-scale choral and instrumental
performances that could appeal to the ears of elite and common people alike.%
\begin{Footnote}
    The major studies of the villancico as a musical and poetic genre are, in
    chronological order,
    \autocites{Rubio:Forma}{Laird:VC}{Torrente:PhD}{Tenorio:SorJuana}
    {CaberoPueyo:PhD}{Illari:Polychoral}{Knighton-Torrente:VCs}
    {Cashner:Cards}{Cashner:PhD}.
    % and more; some of this intro is only needed when using this chapter for
    % book proposal 
\end{Footnote}

With the Church's active patronage, villancicos became a central activity in
religious festivals throughout the year, particularly at Christmas, Corpus
Christi, and the Immaculate Conception of Mary. Church ensembles performed these
pieces, with their motet-like refrains or \emph{estribillos} surrounding a set of
strophic verses or \emph{coplas}, as an integral part of Matins and other liturgies.
Festival crowds from Madrid to Manila also heard villancicos in public
processions and in conjunction with mystery plays. 

A large number of villancicos begin with calls to listen---\emph{escuchad},
\emph{atended}, \emph{silencio}, \emph{atención}. 
Because so many villancicos explicitly address concepts of music, sensation, and
faith, these remarkable but understudied pieces offer us unique insights into
Spanish beliefs about music.
Much of the valuable new scholarship on Spanish colonial music, and on sound and
sensation, focuses primarily on social and institutional history, and on verbal
discourse \emph{about} music.%
    \Autocites{Baker:Harmony}{BakerKnighton:MusicUrbanSociety}{Irving:Colonial} 
    {RamosKittrell:PlayingCathedral}{DellAntonio:Listening}
This book provides a necessary complement to these studies, by analyzing how
people expressed and shaped beliefs about music through the medium of music
itself.

%
% - Focus on musical performative texts; what that means
% - Pros and cons of doing so
% - Sources and methods
%


\endinput
