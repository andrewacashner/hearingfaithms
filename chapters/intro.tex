% Cashner, *Faith, Hearing, and the Power of Music*,
% chapter 1: Villancicos as Musical Theology
% 
% 2017-11-15    New start for book proposal
% 2018-05-21    Converted back to LaTeX
% 2018-07-25    Expanded for UC Press readers

\part{Listening for Faith}
\label{part:faith}

\chapter{Villancicos as Musical Theology}
\label{ch:intro}

\epigraphTranslation
{ergo fides ex auditu\\
auditus autem per verbum Christi}
{Faith, then, comes through hearing, \\
and hearing, by the word of Christ.}
{Romans 10:17}

Saint Paul wrote to the Christian community in Rome, \quoted{How are they to
believe if they have not heard?} since \quoted{faith comes through hearing, and
hearing, by the Word of Christ} (Rm 10:16--17).%
\begin{Footnote}
    This is my own translation from the Latin Vulgate used by Spanish Catholics, in
    the modern edition, \autocite{Weber:Vulgate}.
    The original Greek is \emph{ara ē pistis ex akoēs, ē de akoē dia hrēmatos
    Xristou}: \autocite{Aland:GNT4}
    The word \emph{akoē} can mean \quoted{the faculty of hearing}, \quoted{the
    act of hearing}, \quoted{the organ with which one hears}, or \quoted{that
    which is heard}: \autocite{BDAG}.
    The New Revised Standard Version translates this \quoted{So faith comes from
    what is heard, and what is heard comes through the word of Christ}.
    Early modern Catholic discussions of faith and hearing depend on the range of
    meanings of \emph{auditus} in Latin (as in the underlying Greek), including both
    \quoted{hearing} and \quoted{what is heard}.
\end{Footnote}
Sixteen centuries later, amid the ongoing reformations of the Western Church,
Catholic Christians were seeking ever new ways to make faith audible. 
Poets and composers of the Spanish Empire expanded a genre of sung poetry in the
vernacular---the \emph{villancico}---into large-scale choral and instrumental
performances that could appeal to the ears of elite and common people alike.%
\begin{Footnote}
    The major studies of the villancico as a musical and poetic genre are, in
    chronological order,
    \autocites{Rubio:Forma}{Laird:VC}{Torrente:PhD}{Tenorio:SorJuana}
    {CaberoPueyo:PhD}{Illari:Polychoral}{Knighton-Torrente:VCs}
    {Cashner:Cards}{Cashner:PhD}. 
    % XXX Swadley, Chavez-Barcenas, Torrente historia
\end{Footnote}

With the Church's active patronage, villancicos became a central activity in
religious festivals throughout the year, particularly at Christmas, Corpus
Christi, and the Immaculate Conception of Mary. 
Church ensembles performed these pieces, with their motet-like refrains or
\emph{estribillos} surrounding a set of strophic verses or \emph{coplas}, as an
integral part of Matins and other liturgies.
Festival crowds from Madrid to Manila also heard villancicos in public
processions and in conjunction with mystery plays. 

A large number of villancicos begin with calls to listen---\emph{escuchad},
\emph{atended}, \emph{silencio}, \emph{atención}. 
Because so many villancicos explicitly address concepts of music, sensation, and
faith, these remarkable but understudied pieces offer us unique insights into
Spanish beliefs about music.
When villancicos focused on the theme of music itself, most often by playing on
terms from music theory to build elaborate theological metaphors, they become a
sounding discourse on musical sound.
If a play within a play in seventeenth-century Spanish or English theater is
metatheatrical, then these pieces are \emph{metamusical}.
Through this genre of musical performance people embodied their theological
conceptions of music through the structures of music itself.
For this reason they may be considered as \quoted{musical theology}.

Theology was a major intellectual pursuit of the Spanish and New Spanish elite,
and as such it was a creative activity---not merely reciting dogmas
approved by the church, but playfully seeking out ever-new ways of connecting
revealed truth to observed experience.
Thinking theologically in an early modern Catholic sense meant building 
endless chains of association and allusion among Biblical texts, writings of
church fathers (patristics), medieval theologians, and the liturgy. 
It meant interpreting new texts in light of these old ones, and reinterpeting
the old ones in light of the new.
And it grew out of and reinforced a view of the world as a book waiting to be
read (see \cref{ch:intro}).
One had to apply oneself to the effort of discerning how the sacred was
imminent in the mundane and common.%
    \Autocite{Chavez:DistortingReality}
% XXX cite diss when available 

It is the central argument of this book that devotional music provided Spanish
Catholics with a way of performing theology: making and hearing music was a
creative pursuit in which people sought to forge connections to God and to each
other through musical structures.
These same Spanish intellectuals who studied Augustine and Aquinas also learned
the fundamentals of music on both theoretical and practical levels: they had
learned from Boethius how human music was linked to cosmic harmonies, and they
had learned from Guido of Arezzo how to sing through the gamut using the
mnemonic device of the Guidonian hand.
Metamusical villancicos brought these two domains of knowledge into a mutually
illuminating relationship.
Even in simply reading the poetic texts of these pieces, or hearing them read,
a person must know a fair amount of music theory in order to understand the
theological concepts, and vice versa.
When someone performed the musical setting of this kind of text, or heard it
performed, they faced an even greater challenge to understand the words as
projected through the music and perceive the ways the music depicted the sense
and affect of the text. 

Much of the valuable new scholarship on Spanish colonial music, and on sound and
sensation, focuses primarily on social and institutional history, and on verbal
discourse \emph{about} music.%
    \Autocites{Baker:Harmony}{BakerKnighton:MusicUrbanSociety}{Irving:Colonial} 
    {RamosKittrell:PlayingCathedral}{DellAntonio:Listening}
This book provides a necessary complement to these studies, by analyzing how
people expressed and shaped beliefs about music through the medium of music
itself.
At the same time, the book offers a fresh approach by considering this music as
a source for historical theology, something few scholars have done.
The book interprets these pieces of devotional music within the framework of
early modern Catholic beliefs and looks closely at the ways each piece
represents a creative process of theological thought.
The reward for taking this music seriously today as a source for historical
theology about music is a richer understanding of the intellectual culture of
imperial Spain and a holistic sense of how devotional music served theological
and social functions in Hispanic communities, even creating relationships across
the Atlantic between poets, musicians, and institutions in the mainland and in
the colonial viceroyalties.
Villancicos allow us to hear how Spanish musicians, under the authority of
clergy, cathedral chapters, and as part of local elites, labored to make their
faith heard.
Because the pieces are so self-referential, they reflect on the very nature of
hearing and faith.
Through the many ways that the pieces engaged their audiences' sense of hearing,
and through the ways the pieces model musical hearing itself, they also offer a
glimpse of what a broader audience of common people listened for in music and
what powers they believed it had to shape their community.

% outline of parts and chapters
% sources
% problems, methodology

\endinput
