% Cashner, *Faith, Hearing, and the Power of Music*,
% chapter 1: Villancicos as Musical Theology
% 
% 2017-11-15    New start for book proposal
% 2018-05-21    Converted back to LaTeX
% 2018-07-25    Expanded for UC Press readers

\part{Listening for Faith}
\label{part:faith}

\chapter{Villancicos as Musical Theology}
\label{ch:intro}

\epigraph
{ergo fides ex auditu\\
auditus autem per verbum Christi}
{Romans 10:17}

% opening example:
% Padilla, solfa villancico from Puebla MS
% possibly also/or "En la gloria de un portalillo" as in diss

Saint Paul wrote to the Christian community in Rome, \quoted{How are they to
believe if they have not heard?} since \quoted{faith comes through hearing, and
hearing, by the Word of Christ} (Rm 10:16--17).%
\begin{Footnote}
    This is my own translation from the Latin Vulgate used by Spanish Catholics, in
    the modern edition, \autocite{Weber:Vulgate}.
    The original Greek is \emph{ara ē pistis ex akoēs, ē de akoē dia hrēmatos
    Xristou}: \autocite{Aland:GNT4}
    The word \emph{akoē} can mean \quoted{the faculty of hearing}, \quoted{the
    act of hearing}, \quoted{the organ with which one hears}, or \quoted{that
    which is heard}: \autocite{BDAG}.
    The New Revised Standard Version translates this \quoted{So faith comes from
    what is heard, and what is heard comes through the word of Christ}.
    Early modern Catholic discussions of faith and hearing depend on the range of
    meanings of \emph{auditus} in Latin (as in the underlying Greek), including both
    \quoted{hearing} and \quoted{what is heard}.
\end{Footnote}
Sixteen centuries later, amid the ongoing reformations of the Western Church,
Catholic Christians were seeking ever new ways to make faith audible. 
Poets and composers of the Spanish Empire expanded a genre of sung poetry in the
vernacular---the \emph{villancico}---into large-scale choral and instrumental
performances that could appeal to the ears of elite and common people alike.%
\begin{Footnote}
    The major studies of the villancico as a musical and poetic genre are, in
    chronological order,
    \autocites{Rubio:Forma}{Laird:VC}{Torrente:PhD}{Tenorio:SorJuana}
    {CaberoPueyo:PhD}{Illari:Polychoral}{Knighton-Torrente:VCs}
    {Davies:Guadalupe}
    {Cashner:Cards}{Cashner:PhD}
    {LopezLorenzo:VC-Sevillano}{Swadley:VillancicoPhD}{Torrente:Historia17C}
    {ChavezBarcenas:PhD}.
    For musical editions, see \autocite{Cashner:WLSCM32} and the other sources
    cited there.
\end{Footnote}

With the Church's active patronage, villancicos became a central activity in
religious festivals throughout the year, particularly at Christmas, Corpus
Christi, and the Immaculate Conception of Mary. 
Church ensembles performed these pieces, with their motet-like refrains or
\emph{estribillos} surrounding a set of strophic verses or \emph{coplas}, as an
integral part of Matins and other liturgies.
Villancicos were composed in sets of eight of more pieces, most commonly
interpersed between the readings of the Matins liturgy and replacing or
supplementing the Responsory chants according to local practice.
Festival crowds from Madrid to Manila also heard villancicos in public
processions and in conjunction with mystery plays. 

A large number of villancicos begin with calls to listen---\emph{escuchad},
\emph{atended}, \emph{silencio}, \emph{atención}. 
Because so many villancicos explicitly address concepts of music, sensation, and
faith, these remarkable but understudied pieces offer us unique insights into
Spanish beliefs about music.
When villancicos focused on the theme of music itself, most often by playing on
terms from music theory to build elaborate theological metaphors, they become a
sounding discourse on musical sound.
If a play within a play in seventeenth-century Spanish or English theater is
metatheatrical, then these pieces are \emph{metamusical}.
Through this genre of musical performance people embodied their theological
conceptions of music through the structures of music itself.
For this reason they may be considered as \quoted{musical theology}.

Theology was a major intellectual pursuit of the Spanish and New Spanish elite,
and as such it was a creative activity---not merely reciting dogmas
approved by the church, but playfully seeking out ever-new ways of connecting
revealed truth to observed experience.
Thinking theologically in an early modern Catholic sense meant building 
endless chains of association and allusion among Biblical texts, writings of
church fathers (patristics), medieval theologians, and the liturgy. 
It meant interpreting new texts in light of these old ones, and reinterpeting
the old ones in light of the new.
And it grew out of and reinforced a view of the world as a book waiting to be
read (see \cref{ch:intro}).
One had to apply oneself to the effort of discerning how the sacred was
imminent in the mundane and common.%
    \Autocite{Chavez:DistortingReality}
% XXX cite diss when available 

It is the central argument of this book that devotional music provided Spanish
Catholics with a way of performing theology: making and hearing music was a
creative pursuit in which people sought to forge connections to God and to each
other through musical structures.
These same Spanish intellectuals who studied Augustine and Aquinas also learned
the fundamentals of music on both theoretical and practical levels: they had
learned from Boethius how human music was linked to cosmic harmonies, and they
had learned from Guido of Arezzo how to sing through the gamut using the
mnemonic device of the Guidonian hand.
Metamusical villancicos brought these two domains of knowledge into a mutually
illuminating relationship.
Even in simply reading the poetic texts of these pieces, or hearing them read,
a person must know a fair amount of music theory in order to understand the
theological concepts, and vice versa.
When someone performed the musical setting of this kind of text, or heard it
performed, they faced an even greater challenge to understand the words as
projected through the music and perceive the ways the music depicted the sense
and affect of the text. 

Much of the valuable new scholarship on Spanish colonial music, and on sound and
sensation, focuses primarily on social and institutional history, and on verbal
discourse \emph{about} music.%
    \Autocites{Baker:Harmony}{BakerKnighton:MusicUrbanSociety}{Irving:Colonial} 
    {RamosKittrell:PlayingCathedral}{DellAntonio:Listening}
This book provides a necessary complement to these studies, by analyzing how
people expressed and shaped beliefs about music through the medium of music
itself.
At the same time, the book offers a fresh approach by considering this music as
a source for historical theology, something few scholars have done.
The book interprets these pieces of devotional music within the framework of
early modern Catholic beliefs and looks closely at the ways each piece
represents a creative process of theological thought.
The reward for taking this music seriously today as a source for historical
theology about music is a richer understanding of the intellectual culture of
imperial Spain and a holistic sense of how devotional music served theological
and social functions in communities, even creating relationships across the
Atlantic between poets, musicians, and institutions in the mainland and in
the colonial viceroyalties.
Villancicos allow us to hear how Spanish musicians, under the authority of
clergy, cathedral chapters, and as part of local elites, labored to make their
faith heard.
Because the pieces are so self-referential, they reflect on the very nature of
hearing and faith.
Through the many ways that the pieces engaged their audiences' sense of hearing,
and through the ways the pieces model musical hearing itself, they also offer a
glimpse of what a broader audience of common people listened for in music and
what powers they believed it had to shape their community.

\section{Hearing and Communication}

Villancicos were the most widespread form of religious music with words in
vernacular languages in the Catholic world after the Council of Trent, and they
provide evidence for a sustained endeavor by church leaders to establish
conventions of communication with ordinary people.
% XXX could mention other kinds of Catholic vernacular song (German, Czech?)
The creators of villancicos drew on common experiences of everyday life and
linked them to the sacred in inventive ways that met the spiritual needs of
specific communities.
Each piece provides a new answer to Christ's question, \quoted{With what can we
compare the kingdom of God, or what parable will we use for it?}
(\scripture{Mk}{4:21}).
Villancicos thus represent a key component of the Spanish church's effort
to use music to make faith appeal to hearing.
They are evidence of the church working to accommodate hearing and train it at
the same time.

This type of devotional music spoke to a variety of people at different levels
of understanding.
They were a central part of community festivals across the Spanish Empire,
performed both inside and outside the church, at Matins and Mass, in a multitude
of public and private contexts.
Each villancico cycles includes an array of subgenres that would speak to
different portions of the congregation.
These range from silly dialogues of Christmas shepherds that would have
entertained children and their parents alike to sophisticated meditations on
metaphorical conceits, such as the pieces based on musical terminology that will
be studied in \cref{part:unhearable-music} of this book.

Even the structure of individual villancicos reflects the effort to communicate
on multiple levels.
The \emph{estribillo} section of a typical villancico was scored for full
ensemble and performed at the beginning and then repeated at the end of the
piece; composers usually set this in relatively complex polyphony similar to
what they would use for a motet.
In the center of the piece, the \emph{coplas} or verses were usually set
strophically for solo singers or a reduced ensemble with accompaniment.
As Bernardo Illari argues, the \emph{copla} settings are probably based closely
on oral traditions for singing poetry, especially in the \emph{romance} meter,
to stock melodic formulas; and it would have been easier for common listeners to
make sense of the words that were sung to the simple, repeating melodies.%
    \Autocite{Illari:Polychoral}
The \emph{estribillo}, by contrast, is often much more complex and draws on
traditions of learned counterpoint; composers often invoke a variety of
stylistic registers and styles to convey the meaning of the words and heighten
their rhetorical impact.

But though villancicos have these aspects that seem designed to engage a wide
popular audience, they differ from other dominant forms of vernacular religious
music in this period---Lutheran chorales and Reformed psalms---in that they were
not sung by ordinary parishioners.
Rather, more like Anglican anthems and German sacred concertos, they were
performed by professional church musicians for the benefit of the congregation.
The printed commemorative chapbooks of villancico poetry, and the manuscript
performing parts of the musical settings preserve only one side of the church's
dialogue.
Catholic did not, generally speaking, cultivate a society of literate,
self-advocating lay people who would have left behind traces of their personal
beliefs and devotional practices.
For the Spanish Empire, then, we know what people heard, but not what they
understood or how they responded.%
    \Autocite{Burstyn:PeriodEar} % + Did people listen? etc.
And when villancicos represent types of people---such as deaf men, African
slaves, or Indians---they leave us only with conventional caricatures, not
ethnohistorical descriptions.%
    \Autocites
    {Baker:EthnicVC}
    {Baker:PerformancePostColonial}
    {Davies:LocalContent}

All the same, the devotional music that survives from imperial Spain
can still open a fascinating window into the process of religious communication.
First, villancicos should not be understood as an exclusively top-down
communication, and certainly not as a simple mode of religious indoctrination.
The creators of villancicos were not always members of the most elite strata,
and their readers and hearers included commoners.
The cultivated poet Francisco de Quevedo was credited with mocking \quoted{the
whole caste of villancico poets} as hacks, saying that \quoted{the poor are
drowning in poets, continually hearing their braying}.%
    \Autocite[37]{Torres:SuenosMorales}
If there is any truth to the critique of villancico poets as stringing together
clichés to satisfy the tastes of a lower-class market (an attack also leveled at
opera librettists in Italy), then the same low-class elements that those poets
disdained can provide us with insight into culture at a more common level.
On the musical side as well, some villancico composers were not prestigious
cathedral chapelmasters and we know of at least one who was of indigenous
ancestry, Juan de Araújo in Boliiva.%
    \Autocite{Illari:Popular}
%    \citXXX[non-MC composers, Illari]
Besides, regardless of their personal background, villancico poets and composers
had to produce something that met the needs of their community; though they
answered first to their cathedral chapter and the local ruling caste, it was in
everyone's interest to attract commoners to church and provide them something
that they would find satisfying.
According to contemporary accounts people turned out in droves to hear the
annual villancico performances, in annual traditions that in most cities lasted
from the first few decades of the seventeenth century all the way through the
beginning of the nineteenth.
%    \citXXX[audience turnout]
Somewhat like mass-mediated popular music today, this music was not typically
created by common people themselves, but it both reflected and shaped popular
tastes and attitudes.

\section{Hearing Faith in Community}

Catholic devotional music provided a practical medium for both appealing to the
ear and training it, though music amplified the challenges of acquiring faith
through hearing.
Catholic listeners were encouraged to doubt their senses as much as to trust
them; and church leaders struggled with the frightening possibility that some
people might simply lack the capacity for hearing with faith.
Religious ear training required individual discipline to avoid the danger of
over-reliance on subjective sensory experience and to learn to discern the
spiritual truth communicated through musical patterns.
This training would also need to discipline the whole community to overcome
misunderstandings based on cultural conditioning.

Propagating faith, then, meant trying to establish not just individual
Christians, but also building a Christian society as the body of Christ.
Faithful Catholics had to learn to submit their sensory experience to the
authority of the Church as the source of certainty, as the living, communal
embodiment of Christ the Word in the world.
For Roman Catholics, the Church \emph{was} the gospel, and the task of building
the Church could not be separated from the work of building an empire.

As Catholics worked to create Christian communities, music was a potent tool for
creating harmony, for instituting social discipline as a reflection of the
heavenly hierarchy.%
    \Autocites{Baker:Harmony}{Irving:Colonial}{Illari:Polychoral}
The virtue of man as Neoplatonic microcosm was reflected in the broader society
and in turn depended on it.
Spanish political thinkers conceived of the colonial project in terms of
establishing harmony in society.%
    \Autocite[22--31]{Baker:Harmony}
Most educated Spaniards were familiar with the medieval philosopher Boethius
(either directly or through expositions of his ideas in contemporary music
treatises like that of Pedro Cerone) and his concept that there were three kinds
of music: \emph{musica instrumentalis}, sounding, playing music; \emph{musica
humana}, the harmony of the individual in body and soul, reason and passion, and
the concord of human society; and \emph{musica mundana}, the music of the
celestial spheres.%
    \Autocites
    [\range{bk}{2}, \pagenums{187--189}]{Cerone:Melopeo}
    [203--208]{Boethius:Musica}
The proper performance of \emph{musica instrumentalis}, they believed, could
actually attune the \emph{musica humana} on individual and social levels,
bringing human society in concord with the order of the cosmos, and beyond it,
with the mysterious harmonious of the triune God.

Catholic music, then, was not \emph{about} society; it was a form \emph{of}
society.
This is why the Franciscan friars in New Spain and the Jesuit priests in Brazil
not only started parishes, but also trained choirs.
Forming choirs of boys and training ensembles of village musicians in colonial
Mexico were practical means of establishing the Church and propagating faith on
individual and communal levels.
The musical ritual of the seventeenth-century Church involved a large number of
community participants, for whom performing music with the body and hearing it
were inextricably linked.
The musical efforts of the colonizing church concretely built social
relationships through musical training.%
    \Autocite{RamosKittrell:PlayingCathedral}
For this reason, we cannot fully understand the faith of early modern Catholics
on the basis of verbal formulations alone; we need to see and hear how
communities practiced their faith through coordinated action---such as in
devotional music.%
\begin{Footnote}
    The Lutheran hymn composer Johann Crüger advocated a similar concept of
    \quoted{the musical practice of piety} (\wtitle{Praxis pietatis melica},
    1647 and many later editions), coming out of the Lutheran \quoted{new piety}
    movement of the seventeenth century, whose proponents (Martin Moller, Johann
    Arndt) were inspired by much of the same medieval devotional literature as
    their Catholic counterparts.
\end{Footnote}


% # other scholarship
% # outline of parts and chapters

% # sources
% ## non-musical sources for each chapter
% - ch1
%   + VCs: By Juan Gutiérrez de Padilla, Juan Hidalgo, Cristóbal Galán, Mateo
%   de Villavieja, Joan Cererols, Gaspar Fernández; texts by Sor Juana, Manuel
%   de León Marchante, Vicente Sánchez, etc.

% - ch2: 
%   + villancicos: *Si los sentidos* settings by Miguel de Irízar, Jerónimo de
%   Carrión; *sordos* by Juan Gutiérrez de Padilla, Matíás Ruíz; *Oigan todos
%   del ave* by  Cristóbal Galán 
% + doctrinal, catechetical, missionary, physiological literature,
%   chronicles of festivals, religious drama + (doctrinal/systematic theology,
%   ecclesiology, theological anthropology)

% - ch3: 
%   + VC: Juan Gutiérrez de Padilla, *Voces las de la capilla*; and texts of
%   related VCs from Seville and Lisbon
%   + exegetical, homiletical literature, liturgy, painting/architecture
%   + (exegetical theology, homiletics, liturgy and liturgical theology,
%   Christology, sacramental theology)

% - ch4: 
%   + VC: Joan Cererols, *Suspended, cielos, vuestro dulce canto*; and texts of
%   related VCs from Barcelona, Madrid, Zaragoza, Toledo, Seville; fragment of
%   music from Ibarra, Ecuador
%   + speculative music theory, scientific (astronomical) literature
%   + (natural/empirical theology (?), eschatology)

% - ch5: 
%   + VCs: *Suban las voces al cielo* by Pablo Bruna and Miguel Ambiela; *Qué
%   música divina* by José de Cáseda
%   + emblem books, painting, scientific literature, mystical theology, music
%   theory and performance literature
%   + (mystical/devotional theology)

% ## sources not emphasized
% - cathedral chapter acts, personnel records, inventories, payment records
% - municipal financial or other administrative records
% - diaries, personal accounts (mostly not extant?)
% - primary focus is on villancicos c1650-1700 with surviving music

% # problems, methodology

\endinput
