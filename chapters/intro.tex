% Cashner, *Faith, Hearing, and the Power of Music*,
% chapter 1: Villancicos as Musical Theology
% 
% 2017-11-15    New start for book proposal
% 2018-05-21    Converted back to LaTeX
% 2018-07-25    Expanded for UC Press readers

\part{Listening for Faith}
\label{part:faith}

\chapter{Villancicos as Musical Theology}
\label{ch:intro}

\epigraph
{ergo fides ex auditu\\
auditus autem per verbum Christi}
{Romans 10:17}

\quoted{Faith comes through hearing}, wrote Paul the apostle to the Christian
community in Rome, \quoted{and what is heard, by the word of Christ}
(\scripture{Rom}{10:17}).
Sixteen centuries later, amid the ongoing reformations of the Western Church,
Catholic Christians were seeking ever new ways to make faith audible.
In Europe, church leaders sought to counter the Protestants' effective use of
popular song with their own kinds of devotional music that, like the new forms
of sacred drama and visual art, moved the hearts and instructed the minds of
lay people through a strong appeal to the senses.
Missionaries, meanwhile, were carrying their European chants and songs to
remote outposts and burgeoning colonial cities, where complex processes of
intercultural translation and exchange were already beginning to shape the new
sound of a globally mobilizing Catholic Church.

While we are beginning to learn more about \emph{how} Catholic leaders around
the world used music to connect hearing to faith, we still know very little
about why they did so.
How did they understand music's role in the relationship between faith and
hearing?
What theological beliefs motivated their practices of music-making, and in
turn, how did their musical practices express and shape their beliefs?
Moreover, if we acknowledge that there were many kinds of people who
participated in Catholic ceremonies with music, from different social stations
and cultural backgrounds, what can we learn about how these people actually
listened?
What was the role of musical hearing in their spiritual lives?  

This book aims to recover the answers to these questions from the perspective
of Catholic believers in the Spanish Empire of the seventeenth century.
The worldwide domains of the Spanish crown provided the geographical,
political, and cultural framework for a large proportion of the Church's
evangelizing and colonizing efforts, such that the seemingly spiritual work of
propagating faith and the comparatively worldly project of building colonial
civilization were in most cases inseparably fused (or confused). 
Though the earliest years of exploration (from the 1450s) and confessional
conflict (from 1517) deserve study along the same lines, it was in the
seventeenth century that Spanish churches began incorporating a genre of
devotional music in the vernacular as a regular part of worship services
worldwide---the \term{villancico}.
Villancicos originated in the late medieval period as a genre of courtly 
entertainment and sometimes devotion, with elements drawn from common culture.
In the late sixteenth century, Spaniards began performing villancicos as an 
increasingly integral part of the liturgy, interpolating them among the lessons 
of Matins or in the Mass, especially at Christmas and Epiphany, Corpus Christi, 
and other high festivals like the Conception of Mary and locally significant
saints' feasts.
Villancicos were composed in sets of eight of more pieces, most commonly
interpersed between the readings of the Matins liturgy and replacing or
supplementing the Responsory chants according to local practice.
Festival crowds also heard villancicos in public processions and in conjunction
with mystery plays. 
This type of sacred villancico flourished through the nineteenth century in
some places, and villancicos live on today in simpler forms through folkloric
traditions of Spanish Christmas carols.

Villancicos in the seventeenth century were a genre of complex musical
performances for multiple voices, often arranged in two or more choirs with as
many as twelve vocal parts, doubled and accompanied by large instrumental
ensembles.
Most feature an \term{estribillo} or refrain section for the full ensemble,
usually through-composed rather like a motet or sacred concerto, and strophic
\term{coplas} or verses for soloists or a reduced group.
The words for these pieces were in Spanish and sometimes other vernacular
languages including Portuguese, Catalan, and Náhuatl (language of the Aztecs),
along with pieces whose texts imitated the dialects of African slaves.
The music varied in style and technique from elements of common dances
and popular tunes up to the most sophisticated tone-painting and contrapuntal
craft.
Moreover, sets or cycles of villancicos for a particular feast like Christmas
included many different types of villancicos within them, offering something
for everyone.

Churches of the seventeenth-century Spanish Empire, then, became places to hear
faith proclaimed, celebrated, explained, and embodied through music that
appealed to the ears of many different kinds of worshippers---elite and common,
learned and untaught, masters and slaves---and where contrary elements were
juxtaposed in lively and sometimes unruly counterpoint.
Villancicos were ubiquitous in colonial Spanish culture: most major churches
from Madrid to Manila performed two dozen or more of these pieces every year
for over a century; and even after considerable loss of sources, there still
survive hundreds of musical settings (mostly in manuscript performing parts)
and vastly more printed leaflets of the texts of villancico poems.
This genre provides a window into Catholic devotional culture in this period
relevant to the lives of many thousands of people around the world.

Most significantly for this book, villancicos were first and foremost
expressions of theology, attempts to connect often abstract religious concepts
to images and experiences from everyday life.
The more surprising and puzzling the connection, the better---such as comparing
Christ to a gambling card player.%
    \Autocite{Cashner:PlayingCards}
Christmas villancicos in particular often brought rogues, buffoons, peasants,
and slaves to Christ's manger to offer songs and dances characteristic to
them--even Don Quijote and Sancho Panza make an appearance in one villancico.%
    \citXXX[Don quijote VC]
Until recently, scholars and performers have focused on a small repertoire of
villancicos that does not really represent the full range of pieces
that were commonplace in this period, resulting in a conventional view of
villancicos as primarily folkloric, secular (as in worldly, irreverent), comic,
small in scale, and primarily of interest for the traces they appear to
preserve of popular culture, particularly that of Native Americans and
Africans.
No doubt, plenty of villancicos do fit that description, but as I have argued
elsewhere, even these pieces need to be understood within a theological
frame---and we would benefit from thinking more deeply about what we can
actually learn from these Spanish representations of non-Spaniards.%
    \Autocite{Cashner:BuildingSociety}
If we look beyond the handful of tired examples, we find a rich and largely
unmined vein of poetic and musical sources that present complex and
sophisticated discourses on theological concepts.
More than that, we find pieces that not only describe but embody their
concepts, and that offered local congregations opportunities for theological
reflection, learning, and worship.

Villancicos of the seventeenth century were, as a rule, religious music.
They formed a central part of the devotional life of Catholic communities in
the Ibero-American world, and for this reason we need to understand what they
have to say about those communities' theological beliefs. 
We are talking about a genre of poetry and music on religious subject matter,
composed by professional church musicians typically under contractual
obligation to the church, performed in liturgical worship and paraliturgical
celebrations in churches and convents by ensembles funded by the church and
often consisting of clerics or monastics.
It should require no special pleading to argue that we need to understand this
music within the profoundly theological context in which it was patronized,
created, performed, and heard.

A large number of villancicos begin with calls to listen---\emph{escuchad},
\emph{atended}, \emph{silencio}, \emph{atención}. 
Because so many villancicos explicitly address concepts of music, sensation, and
faith, these remarkable but understudied pieces offer us unique insights into
Spanish beliefs about music.
When villancicos focused on the theme of music itself, most often by playing on
terms from music theory to build elaborate theological metaphors, they become a
sounding discourse on musical sound.
If a play within a play in seventeenth-century Spanish or English theater is
metatheatrical, then these pieces are \emph{metamusical}.
Through this genre of musical performance people embodied their theological
conceptions of music through the structures of music itself.
For this reason they may be considered as \quoted{musical theology}.

It is the central argument of this book that devotional music provided Spanish
Catholics with a way of performing theology. 
Making and hearing music was a creative pursuit in which people sought to forge
connections to God and to each other through musical structures.
The pieces studied in this book present a discourse about the theology of
listening to music in a way that also teaches people how to listen to music ,
and also teaches people how to listen to music theologically.
Educated Spaniards had studied both the theology of Augustine and Aquinas and  
the fundamentals of music on theoretical and practical levels. 
They had learned from Boethius how human music was linked to cosmic harmonies,
and they had learned from Guido of Arezzo how to sing through the gamut using
the mnemonic device of the Guidonian hand.
Metamusical villancicos brought these two domains of knowledge into a mutually
illuminating relationship.
Even in simply reading the poetic texts of these pieces, or hearing them read,
a person must know a fair amount of music theory in order to understand the
theological concepts, and vice versa.
When someone performed the musical setting of this kind of text, or heard it
performed, they faced an even greater challenge to understand the words as
projected through the music and perceive the ways the music depicted the sense
and affect of the text. 

I must be clear from the outset that the only question here is one of
understanding a historical worldview through its embodiment in a particular
kind of music.
Interpreting villancicos requires us to set aside our own religious ideas---or
anti-religious ideas---in order to hear the world through the ears of
seventeenth-century Spanish Catholics, to the extent we can venture to do so.
This will always require imaginative leaps of interpretation that will never
bring us all the way to the origin, and may in fact lead us to concepts the
authors and their audiences did not foresee.\citXXX[Ricouer etc]
But our goal is to build a sufficient interpretive framework that we can
recover a plausible range of meanings that this music might have had for its
creators and first hearers.
The primary methods for accomplishing this here are comparative study within a
corpus of pieces on related themes, in which it is possible to identify
recurring and developing tropes and conventions; and contextual study of
related literature and arts.
The related literature includes poetry and sacred drama, theoretical and
practical treatises on music, and several branches of theological writing:
doctrinal (explanations of Christian beliefs), exegetical (interpreting
Scripture), homiletical (preaching, applying Scripture and doctrine to daily
life), and devotional (teaching and modeling prayer and worship for personal
and community life).

For all its engagement with theology, this book is not a work of constructive
or normative theology. 
My only aim is to use the interpretation of poetic and musical texts to provide
a historical perspective on music's role in the religious climate of the early
modern Spanish Empire, one that acknowledges the many contradictions and
tensions in the texts and which seeks to connect the imagined world of these
texts to the real social worlds around them. 
Though I myself am a Christian believer I am not a Roman Catholic, and I
certainly do not subscribe to the worldview of seventeenth-century Spanish
subjects.
But this book takes no position on the truth claims of any religious tradition.
Understanding the religious aspect of historical listening practices in a
particular time and place may lead to a more nuanced understanding of
contemporary theology, just as it might hopefully enable more sensitive and
meaningful performances of historic repertoire; but that is up to the reader.
I also would not claim that the theological approach is the only valid way to
study villancicos, and indeed there is much more to the genre than the aspect
that I present here, for which I refer readers to my articles and to a growing
body of studies by scholars with different perspectives and priorities.

Nevertheless, the religious element of early modern Spanish culture is so
overwhelmingly evident to anyone who has visited Mexico or Spain or read any of
its literature, that it can hardly be justified if we overlook it or insist on
interpreting it through an anti-religious or anti-Catholic lens.
Instead, this book is for anyone who has gazed upwards in a Spanish or Mexican
church and wondered why there are so many images of angel musicians with harps,
\term{vihuelas}, \term{bajones}, and organs; or who has read a play by
Calderón, a poem by Sor Juana, or a devotional book by Ignatius of Loyola or
Saint John of the Cross, and has observed how often these writers use musical
metaphors; or who wonders what people thought was happening when they listened
to music in church and how they believed this connected them to God, to each
other, and to the cosmos.

These examples all show that theology was a major intellectual pursuit of the
Spanish and New Spanish elite.
As such it was a creative activity---not merely reciting dogmas
approved by the church, but playfully seeking out ever-new ways of connecting
revealed truth to observed experience.
Thinking theologically in an early modern Catholic sense meant building 
endless chains of association and allusion among Biblical texts, writings of
church fathers (patristics), medieval theologians, and the liturgy. 
It meant interpreting new texts in light of these old ones, and reinterpeting
the old ones in light of the new.
And it grew out of and reinforced a view of the world as a book waiting to be
read (see below).
One had to apply oneself to the effort of discerning how the sacred was
imminent in the mundane and common.%
    \citXXX[Chavez:PhD]

Much of the valuable new scholarship on Spanish colonial music, and on sound
and sensation, focuses primarily on social and institutional history, and on
verbal discourse \emph{about} music.%
    \Autocites{Baker:Harmony}{BakerKnighton:MusicUrbanSociety}{Irving:Colonial} 
    {RamosKittrell:PlayingCathedral}{DellAntonio:Listening}
This book provides a necessary complement to these studies, by analyzing how
people expressed and shaped beliefs about music through the medium of music
itself.
At the same time, the book offers a fresh approach by considering this music as
a source for historical theology, something few scholars have done.
The reward for taking this music seriously today as a source for historical
theology about music is a richer understanding of the intellectual culture of
imperial Spain and a holistic sense of how devotional music served theological
and social functions in communities, even creating relationships across the
Atlantic between poets, musicians, and institutions in the mainland and in
the colonial viceroyalties.
Villancicos allow us to hear how Spanish musicians, under the authority of
clergy, cathedral chapters, and as part of local elites, labored to make their
faith heard.
Because the pieces are so self-referential, they reflect on the very nature of
hearing and faith.
Through the many ways that the pieces engaged their audiences' sense of hearing,
and through the ways the pieces model musical hearing itself, they also offer a
glimpse of what a broader audience of common people listened for in music and
what powers they believed it had to shape their community.


\section{Hearing and Communication}

Villancicos were the most widespread form of religious music with words in
vernacular languages in the Catholic world after the Council of Trent, and they
provide evidence for a sustained endeavor by church leaders to establish
conventions of communication with ordinary people.
% XXX could mention other kinds of Catholic vernacular song (German, Czech?)
The creators of villancicos drew on common experiences of everyday life and
linked them to the sacred in inventive ways that met the spiritual needs of
specific communities.
Each piece provides a new answer to Christ's question, \quoted{With what can we
compare the kingdom of God, or what parable will we use for it?}
(\scripture{Mk}{4:21}).
Villancicos thus represent a key component of the Spanish church's effort
to use music to make faith appeal to hearing.
They are evidence of the church working to accommodate hearing and train it at
the same time.

This type of devotional music spoke to a variety of people at different levels
of understanding.
They were a central part of community festivals across the Spanish Empire,
performed both inside and outside the church, at Matins and Mass, in a multitude
of public and private contexts.
Each villancico cycles includes an array of subgenres that would speak to
different portions of the congregation.
These range from silly dialogues of Christmas shepherds that would have
entertained children and their parents alike to sophisticated meditations on
metaphorical conceits, such as the pieces based on musical terminology that will
be studied in \cref{part:unhearable-music} of this book.

Even the structure of individual villancicos reflects the effort to communicate
on multiple levels.
The \emph{estribillo} section of a typical villancico was scored for full
ensemble and performed at the beginning and then repeated at the end of the
piece; composers usually set this in relatively complex polyphony similar to
what they would use for a motet.
In the center of the piece, the \emph{coplas} or verses were usually set
strophically for solo singers or a reduced ensemble with accompaniment.
As Bernardo Illari argues, the \emph{copla} settings are probably based closely
on oral traditions for singing poetry, especially in the \emph{romance} meter,
to stock melodic formulas; and it would have been easier for common listeners to
make sense of the words that were sung to the simple, repeating melodies.%
    \Autocite{Illari:Polychoral}
The \emph{estribillo}, by contrast, is often much more complex and draws on
traditions of learned counterpoint; composers often invoke a variety of
stylistic registers and styles to convey the meaning of the words and heighten
their rhetorical impact.

But though villancicos have these aspects that seem designed to engage a wide
popular audience, they differ from other dominant forms of vernacular religious
music in this period---Lutheran chorales and Reformed psalms---in that they were
not sung by ordinary parishioners.
Rather, more like Anglican anthems and German sacred concertos, they were
performed by professional church musicians for the benefit of the congregation.
The printed commemorative chapbooks of villancico poetry, and the manuscript
performing parts of the musical settings preserve only one side of the church's
dialogue.
Catholic did not, generally speaking, cultivate a society of literate,
self-advocating lay people who would have left behind traces of their personal
beliefs and devotional practices.
For the Spanish Empire, then, we know what people heard, but not what they
understood or how they responded.%
    \Autocite{Burstyn:PeriodEar} % + Did people listen? etc.
And when villancicos represent types of people---such as deaf men, African
slaves, or Indians---they leave us only with conventional caricatures, not
ethnohistorical descriptions.%
    \Autocites
    {Baker:EthnicVC}
    {Baker:PerformancePostColonial}
    {Davies:LocalContent}

All the same, the devotional music that survives from imperial Spain
can still open a fascinating window into the process of religious communication.
First, villancicos should not be understood as an exclusively top-down
communication, and certainly not as a simple mode of religious indoctrination.
The creators of villancicos were not always members of the most elite strata,
and their readers and hearers included commoners.
The cultivated poet Francisco de Quevedo was credited with mocking \quoted{the
whole caste of villancico poets} as hacks, saying that \quoted{the poor are
drowning in poets, continually hearing their braying}.%
    \Autocite[37]{Torres:SuenosMorales}
If there is any truth to the critique of villancico poets as stringing together
clichés to satisfy the tastes of a lower-class market (an attack also leveled at
opera librettists in Italy), then the same low-class elements that those poets
disdained can provide us with insight into culture at a more common level.
On the musical side as well, some villancico composers were not prestigious
cathedral chapelmasters and we know of at least one who was of indigenous
ancestry, Juan de Araújo in Boliiva.%
    \Autocite{Illari:Popular}
%    \citXXX[non-MC composers, Illari]
Besides, regardless of their personal background, villancico poets and composers
had to produce something that met the needs of their community; though they
answered first to their cathedral chapter and the local ruling caste, it was in
everyone's interest to attract commoners to church and provide them something
that they would find satisfying.
According to contemporary accounts people turned out in droves to hear the
annual villancico performances, in annual traditions that in most cities lasted
from the first few decades of the seventeenth century all the way through the
beginning of the nineteenth.
    \citXXX[audience turnout]
Somewhat like mass-mediated popular music today, this music was not typically
created by common people themselves, but it both reflected and shaped popular
tastes and attitudes.

\section{Hearing Faith in Community}

Catholic devotional music provided a practical medium for both appealing to the
ear and training it, though music amplified the challenges of acquiring faith
through hearing.
Catholic listeners were encouraged to doubt their senses as much as to trust
them; and church leaders struggled with the frightening possibility that some
people might simply lack the capacity for hearing with faith.
Religious ear training required individual discipline to avoid the danger of
over-reliance on subjective sensory experience and to learn to discern the
spiritual truth communicated through musical patterns.
This training would also need to discipline the whole community to overcome
misunderstandings based on cultural conditioning.

Propagating faith, then, meant trying to establish not just individual
Christians, but also building a Christian society as the body of Christ.
Faithful Catholics had to learn to submit their sensory experience to the
authority of the Church as the source of certainty, as the living, communal
embodiment of Christ the Word in the world.
For Roman Catholics, the Church \emph{was} the gospel, and the task of building
the Church could not be separated from the work of building an empire.

As Catholics worked to create Christian communities, music was a potent tool for
creating harmony, for instituting social discipline as a reflection of the
heavenly hierarchy.%
    \Autocites{Baker:Harmony}{Irving:Colonial}{Illari:Polychoral}
The virtue of man as Neoplatonic microcosm was reflected in the broader society
and in turn depended on it.
Spanish political thinkers conceived of the colonial project in terms of
establishing harmony in society.%
    \Autocite[22--31]{Baker:Harmony}
Most educated Spaniards were familiar with the medieval philosopher Boethius
(either directly or through expositions of his ideas in contemporary music
treatises like that of Pedro Cerone) and his concept that there were three kinds
of music: \emph{musica instrumentalis}, sounding, playing music; \emph{musica
humana}, the harmony of the individual in body and soul, reason and passion, and
the concord of human society; and \emph{musica mundana}, the music of the
celestial spheres.%
    \Autocites
    [\range{bk}{2}, \pagenums{187--189}]{Cerone:Melopeo}
    [203--208]{Boethius:Musica}
The proper performance of \emph{musica instrumentalis}, they believed, could
actually attune the \emph{musica humana} on individual and social levels,
bringing human society in concord with the order of the cosmos, and beyond it,
with the mysterious harmonious of the triune God.

Catholic music, then, was not \emph{about} society; it \emph{was} society.
This is why the Franciscan friars in New Spain and the Jesuit priests in
California not only started parishes, but also trained choirs.
Forming choirs of boys and training ensembles of village musicians in colonial
cities were practical means of establishing the Church and propagating faith on
individual and communal levels.
The musical ritual of the seventeenth-century Church involved a large number of
community participants, for whom performing music with the body and hearing it
were inextricably linked.
The musical efforts of the colonizing church concretely built social
relationships through musical training.%
    \Autocite{RamosKittrell:PlayingCathedral}
For this reason, we cannot fully understand the faith of early modern Catholics
on the basis of verbal formulations alone; we need to see and hear how
communities practiced their faith through coordinated action---such as in
devotional music.%
\begin{Footnote}
    The Lutheran hymn composer Johann Crüger advocated a similar concept of
    \quoted{the musical practice of piety} (\wtitle{Praxis pietatis melica},
    1647 and many later editions), coming out of the Lutheran \quoted{new piety}
    movement of the seventeenth century, whose proponents (Martin Moller, Johann
    Arndt) were inspired by much of the same medieval devotional literature as
    their Catholic counterparts.
\end{Footnote}


% # other scholarship
% # outline of parts and chapters

% # sources
% ## non-musical sources for each chapter
% - ch1
%   + VCs: By Juan Gutiérrez de Padilla, Juan Hidalgo, Cristóbal Galán, Mateo
%   de Villavieja, Joan Cererols, Gaspar Fernández; texts by Sor Juana, Manuel
%   de León Marchante, Vicente Sánchez, etc.

% - ch2: 
%   + villancicos: *Si los sentidos* settings by Miguel de Irízar, Jerónimo de
%   Carrión; *sordos* by Juan Gutiérrez de Padilla, Matíás Ruíz; *Oigan todos
%   del ave* by  Cristóbal Galán 
% + doctrinal, catechetical, missionary, physiological literature,
%   chronicles of festivals, religious drama + (doctrinal/systematic theology,
%   ecclesiology, theological anthropology)

% - ch3: 
%   + VC: Juan Gutiérrez de Padilla, *Voces las de la capilla*; and texts of
%   related VCs from Seville and Lisbon
%   + exegetical, homiletical literature, liturgy, painting/architecture
%   + (exegetical theology, homiletics, liturgy and liturgical theology,
%   Christology, sacramental theology)

% - ch4: 
%   + VC: Joan Cererols, *Suspended, cielos, vuestro dulce canto*; and texts of
%   related VCs from Barcelona, Madrid, Zaragoza, Toledo, Seville; fragment of
%   music from Ibarra, Ecuador
%   + speculative music theory, scientific (astronomical) literature
%   + (natural/empirical theology (?), eschatology)

% - ch5: 
%   + VCs: *Suban las voces al cielo* by Pablo Bruna and Miguel Ambiela; *Qué
%   música divina* by José de Cáseda
%   + emblem books, painting, scientific literature, mystical theology, music
%   theory and performance literature
%   + (mystical/devotional theology)

% ## sources not emphasized
% - cathedral chapter acts, personnel records, inventories, payment records
% - municipal financial or other administrative records
% - diaries, personal accounts (mostly not extant?)
% - primary focus is on villancicos c1650-1700 with surviving music

% # problems, methodology


St. Paul taught that faith came by means of hearing, and one of the distinctive effects of the sixteenth-century reformations of Western Christianity was that Christians discovered new ways to make their faith audible.
Voices raised in acrid contention or pious devotion boomed from pulpits, clamored in public squares, and were echoed in homes and schools.
In new forms of vernacular music, the voices of the newly distinct communities united to articulate their own vision of Christian faith.
Catholic reformers and missionaries enlisted music in their campaigns to educate, evangelize, and build Christian civilization, both in an increasingly divided Europe and in the new domains of the Spanish crown across the globe.
In these efforts to make \quoted{the word of Christ} to be heard and believed, then, what was the role of music?
What kind of power did Catholics believe music had over the dynamics of hearing and faith?

This dissertation is a study of how Christians in early modern Spain and Spanish America enacted religious beliefs about music through the medium of music itself.
It focuses on villancicos, a widespread genre of devotional poetry and musical performance, for two primary reasons.
First, these pieces were actively employed by the Spanish church and state as tools for propagating faith.
By the seventeenth century villancicos had grown into a complex, large-scale form of vocal and instrumental music based on poetic texts in the vernacular, and they were performed in and around liturgical celebrations on all the major feast days, across the Spanish world.

In their poetic themes and in their musical content, villancicos combined elements of elite and common culture.
In subject matter as well as in the places and occasions of their performance, villancicos stood on the threshold between the world in and outside of church (which is not quite the same as a modern divide between sacred and secular).
Sets of villancicos featured dramatic, often comic texts reminiscent of Spanish minor theater (\term{teatro menor}) alongside cultivated and even arcanely sophisticated theological reflections.
The music for villancicos covered a wide stylistic range from old-style polyphonic techniques to highly rhythmic music drawing on dance traditions.

Villancicos are valuable, then, for assessing the interaction of these distinctive elements that meet within the genre.
Were villancicos a form of top-down \soCalled{propaganda} intended to indoctrinate and control, as some have claimed of post-Tridentine religious arts?
Or were they a grassroots expression of popular devotion?
Did they work on multiple levels, even contradictory ones?%
	%
	\footnote{%
	This question is a primary focus of chapters~\ref{ch:theology} and \ref{ch:Puebla}.
	}
	%

The second reason for focusing on villancicos is that a large portion of the repertoire explicitly addresses theological beliefs about music.
The Spanish poetry of seventeenth-century villancicos frequently treats musical topics, sometimes using ingenious conceits that create rich and nuanced links between musical and theological ideas.
The musical settings turn a poetic discourse about music into a musical discourse about music.

Of all the musical forms of Catholic Spain (and perhaps in Catholicism generally), then, sacred villancicos address the theological nature and function of music most frequently and directly.
Among the hundreds of surviving musical manuscripts and the vastly larger quantity of printed poetry leaflets, a great many pieces begin with direct invocations of the sense of hearing, exhorting hearers to \foreign{oíd} \gloss{hear}, \foreign{escuchad} \gloss{listen}, and \foreign{atended} \gloss{pay attention}.
Villancico poets and composers favored themes of singing and dancing, as in Christmas sets, for example, they represented the angelic choirs of Christmas, singing and dancing shepherds, Magi, and even animals in the Nativity stable.
There are also representations of instrumental performance and characteristic dances of different ethnic groups (such as African slaves and \soCalled{gypsies}).
Early examples of villancicos about singing include \wtitle{Gil pues a cantar} from Pedro Ruimonte's \wtitle{Parnaso español} (Antwerp, 1614) and \wtitle{Sobre bro canto llano} by Gaspar Fernández (Puebla, 1610).%
	%
	\footnote{%
	\label{fn:Ruimonte}
	\autocites[296--309]{Ruimonte:Parnaso}[240--244]{Fernandes:Cancionero}.
	}
	%
A few \soCalled{black} villancicos or \term{negrillas}, which feature dancing and playing instruments, have become well known: \wtitle{A siolo Flasiquiyo} by Juan Gutiérrez de Padilla (Puebla, 1653) and \wtitle{Los coflades de la estleya} by Juan de Araujo (Sucre, \circa1700).%
	%
	\footnote{%
	See chapter 6 for a discussion of \wtitle{A siolo Flasiquiyo} and other \soCalled{ethnic} villancicos.
	One of several recordings of the Araujo piece is \autocite{Skidmore:NewWorldCD}.
	}
	%
These pieces constitute \soCalled{music about music}.
If a play within a play in Spanish Golden Age drama may be termed metatheatrical, then these pieces are \soCalled{metamusical}.

Understanding the theology of music articulated in villancicos can illuminate why and how villancicos were used to propagate faith.
Doing so will deepen our knowledge of how music in general fit into the religious worldview of early modern Catholics. 
For Catholic believers in the Spanish Empire of the seventeenth-century, what kind of power did music have to effect the relationship between faith and the sense of hearing?
If music had supernatural power, how was that power linked to the worldly powers of the church and state?
By engaging interpretively with these villancicos we may gain a better understanding of how early modern Catholics used music for spiritual ends, and how the spiritual intersected with the worldly functions of this music within Spanish society. 

This dissertation is the first large-scale attempt to understand the theological aspect of seventeenth-century villancicos across the Hispanic world.
It is also one of only a few studies to analyze villancicos musically in detail.
Most importantly, the primary goal of the project is to combine these two modes of analysis, to understand how theological beliefs were expressed and shaped through the details of musical composition and performance.
The goal is to understand the \term{musical theology} of villancicos---this does not mean just a verbal formulation of theological ideas about music, and it certainly does not mean a personal spiritual interpretation of music.
Instead, we may conceive of this historical form of devotional performance as a communal act in which religious ideas and values were enacted through musical structures.
To understand the theological content, we must understand the musical practices; and to make sense of the music, we must seek to hear it as a form of theological expression.

\section{%
Sources
}

The primary sources for this study are villancico poems and their musical settings.
The poems are preserved in musical manuscripts and in printed pamphlets (\term{pliegos sueltos} or \soCalled{poetry imprints}), and the musical settings are preserved in manuscript partbooks and some scores.
These sources were found in archives across Spain and Mexico, plus a few from farther abroad such as Ecuador, and some from published editions.
The poetic and musical sources are interpreted in the context of representative examples of the most influential music-theoretical, theological, and quasi-scientific literature in the seventeenth-century Spanish Empire, with additional reference to contemporary visual art, such as emblem books, altar paintings, and architecture.
These are supported with a limited selection of archival documentation, such as chapter acts.
The primary focus, though, will be engaging closely with the musical and poetic texts of villancicos themselves.

\section{%
Outline of Chapters
}

Part~\ref{part:first}  argues that villancicos on the subject of music may be interpreted as sources for historical theologies of music. 
It considers how certain conceptual problems regarding music's role in the relationship between faith and hearing manifested in this genre, and proposes a historically grounded model for understanding these pieces theologically.

The first chapter introduces the category of metamusical villancico in its several subtypes, using examples by composers who will be discussed further in the rest of the dissertation.
The chapter traces the roots of the interpretive approach in this dissertation within musicology and several other disciplines, and clarifies the project's relationship to existing scholarship on villancicos and early modern sacred music.

The second chapter argues that the relationship between hearing and faith was a theological problem in seventeenth-century Spain. 
Catholics had to balance the desire to make faith accommodate the sense of hearing with the need to train the sense of hearing.
Hearing had to be shaped by faith in order to perceive the content of faith.
To understand music's role in connecting hearing and faith, the chapter examines how villancicos that represent the senses, sensory confusion, and sensory impairment manifest theological concepts of perception.
The chapter situates villancicos within a Neoplatonic understanding of hearing and music and outlines three primary theological functions of villancicos, each of which requires a different kind of listening practice.

Part~\ref{part:Singing} presents detailed case studies of individual pieces, or pieces in specific traditions and places, on themes of heavenly music. 
These pieces constitute \soCalled{singing about singing}. 
Chapters~\ref{ch:Padilla-Voces} and \ref{ch:Cererols} each interpret a single villancico tradition that represents earthly music as a Neoplatonic reflection of heavenly music.
Chapters~\ref{ch:Segovia} and \ref{ch:Zaragoza} present groups of pieces from specific locations that demonstrate a shift in theological understanding of music, where earthly music is seen as more an expression of human affects than as a reflection of heavenly order.
All four chapters in part~\ref{part:Singing} also demonstrate that these metamusical villancicos functioned as a special subgenre in which composers could demonstrate their own mastery within the context of a lineage of composition and a tradition of treatments of this theme of heavenly music.

Part~\ref{part:Puebla} focuses on how the musical theology of villancicos was developed in coordination with the Spanish projects of colonizing and civilizing.
It looks at the relationship by Juan Gutiérrez de Padilla's Christmas villancico cycles (extant from 1651--1659) and the building of Puebla Cathedral (consecrated 1649).
It argues that Padilla's villancico cycles construct a utopian microcosm of hierarchical colonial society.
The chapter focuses on Padilla's representations of people at the bottom of the social hierarchy, such as Indians and black slaves, in a piece from 1652.
Especially through the interplay of language and musical rhythm, this composer and his ensemble constructed a world in which every member of colonial society was put into its proper hierarchical place, in a combination of Neoplatonic music theory and ethics.
The final chapter draws general conclusions and points to directions for future work.

The appendix includes, as an integral part of this project of interpretation and communication, transcriptions of poetry and music for the villancicos most fully discussed in the text.
Most of these have never been edited before, and a few now receive their first critical, corrected editions.
The English translations are among the first translations of seventeenth-century villancico poems into any language.

To begin, then, we must consider what metamusical villancicos are and what they reveal about seventeenth-century theological concepts of music.

%***********************************************************
\section{%
Music about Music in the Villancico Genre
}

The villancicos studied in this dissertation refer in some way to music.
Some focus on making music, others on hearing it.
As such, these pieces constitute music that refers to itself.
If we say that a villancico is \soCalled{music about music}, with the first \mentioned{music} we refer to a specific villancico as a musical entity, which includes 
\begin{enumerate}
\item the performance instructions encoded in notation,
\item the music as it sounds when performed, generalizing from various possible interpretations and guessing about elements of performance not recorded in notation, and also
\item the piece as it existed in history, such as at its first known performance in a particular place.
\end{enumerate}
%
By the second term \mentioned{music} we may mean several things depending on the piece in question: 
\begin{enumerate}
\item other sounding music (the \term{musica instrumentalis} of Boethius) that the villancico imitates or to which the piece alludes, quotes, or pays homage (as in musical topics and tropes); or 
\item music as an abstract concept, which can have increasing levels of abstractions along a Neoplatonic chain ascending to the \soCalled{music} of the Triune Godhead itself.
\end{enumerate}

A global survey of villancico poems and music reveals nine main categories of metamusical villancicos.%
	%
	\footnote{%
	This non-exhaustive survey was drawn from archival musical and poetic sources and from listings in catalogs and published studies (see \quoted{Primary Sources} in the Bibliography), covering a global range of sources.
	}
	%
The survey found more than nine hundred extant, cataloged villancicos that reference metamusical themes, a number that only hints at the original size of this repertoire.
Table~\ref{table:VCtopics} lists the most common topics in order of frequency.

%%*******************
%\begin{table}
%\caption{Topics of metamusical villancicos in global survey}
%\label{table:VCtopics}
%	\input{tables/VCtopics}
%\end{table}
%%*******************
%
%*******************
\subsection{%
Pieces with Multiple Topics: Padilla and Cererols
}

It is common to find references to several of these topics in a single piece, and looking at two typical examples of this sort will begin to make clear what is meant by metamusical villancicos.
The first example is a villancico from the 1652 Christmas cycle (MEX-Pc: Leg. 1/2) written for the Cathedral of Puebla de los Ángeles by Juan Gutiérrez de Padilla (\circa1590--1664).%
	%
	\footnote{%
	Chapter~\ref{ch:Puebla} discusses one piece from this cycle in depth, and another villancico by this composer is the subject of chapter~\ref{ch:Padilla-Voces}.
	Many scholars use full surname Gutiérrez de Padilla, but it will be convenient throughout the dissertation to refer to this composer the way the Puebla manuscripts do, as simply \soCalled{Padilla}.
	}
	%
In just the first seven lines of this anonymous text (poem~\ref{poem:Padilla-1652-Gloria_portalillo}), the villancico refers to sound, voices, singing, choirs, dancing, birds, and solmization.

%%*******************
%\begin{expoem}
%\caption{\wtitle{En la gloria de un portalillo}, estribillo as set by Juan Gutiérrez de Padilla, Puebla Cathedral, Christmas 1652 (MEX-Pc: Leg. 1/2)}
%\label{poem:Padilla-1652-Gloria_portalillo}
%\input{poems/Padilla-En_la_gloria_de_un_portalillo.tex}
%\end{expoem}
%%*******************
%
Padilla's setting demonstrates several typical features of the genre (example~\ref{ex:Padilla-Portalillo}).
The piece begins with a soloist whose words present a striking poetic conceit, and whose music likewise lays out a central musical theme for the \term{estribillo} \gloss{refrain}.
The solo line is followed by a passage of polychoral dialogue between two four-voice choirs, concluding (typically for polychoral technique) with an emphatic cadence for the full chorus.
Padilla's setting is in a lively triple meter (\term{tiempo menor de proporción menor}, notated CZ in Spanish sources) that makes frequent use of \term{sesquialtera} or hemiola.%
	%
	\footnote{%
	The preface provides additional background about the terminology and common structures of seventeenth-century villancicos.
	Please note the discussion there on common voicing and instrumentation patterns, and on rhythmic theory.
	}
	%
The shifts of duple and triple stresses combine with stresses on the second beat of the \term{compás} (\term{tactus}, measure) to create an energetic atmosphere with a rejoicing affect.
The polychoral dialogue, with the voices of each choir declaiming homorhythmically in the same highly rhythmic, syncopated manner as the soloist, and with the \term{tiples} (boy sopranos) of both choirs singing at the top of their range, would have brilliantly seized the attention of listeners.

After this introductory \term{exordium}, the Tiple I soloist continues to describe the scene at the manger.
As the shepherds \foreign{are turned to boys} \gloss{se vuelven niños}, Padilla has the musicians \soCalled{turn} modally by adding C sharps, accented in a sesquialtera ($3:2$) group.
The passage that follows this moment is in evenly accented ternary patterns, in two-compás groups.
These groups emphasize the rhymes in \foreign{tonos sonoros, repiten a coros} and the clear triple meter evokes the dances of \foreign{en bailes lucidos}.

When the soloist refers to the newborn Sun, he sings the note identified in Guidonian terminology as D \term{(la, sol, re)}---\term{sol} in the hard (G) hexachord.
On the same word, the bass accompanist plays a different \term{sol}, G \term{(sol, re, ut)}. 
(Note that \quoted{sol re} in Spanish means \quoted{sun king}.)%
	%
	\footnote{%
The major Spanish music-theoretical treatises of the seventeenth century give full expositions of the techniques of Guidonian solmization: \autocites{Cerone:Melopeo}{Lorente:Porque}.
The frequent symbolic use of Guido's syllables in villancicos suggests that these treatises do reflect how music was actually taught in practice.
	}
	%

%*******************
%\begin{example}
%\includegraphics[width=\linewidth]{scores-examples/Padilla-En_la_gloria_de_un_portalillo-ex}
%\caption{Padilla, \wtitle{En la gloria de un portalillo} (MEX-Pc:~Leg. 1/2, Christmas 1652), estribillo (\range{\measures}{6--17})}
%\label{ex:Padilla-Portalillo}
%\end{example}
%%*******************
%
Padilla's villancico may be understood as \soCalled{singing about singing} on several levels.
The text, which is being performed through music, itself refers to musical performance.
The performance by the Puebla Cathedral chapel dramatizes the historical celebration of the first Christmas while also celebrating the festival in Padilla's present day.
The music is self-referential on a symbolic level (as in the plays on \term{sol}), but also functions on a more simple affective level to model and incite affections of exuberant joy and wonder, which contemporary theological writers emphasized were the appropriate affects for the feast of Christmas.%
	%
	\footnote{%
	See the detailed investigation of such sources in chapter~\ref{ch:Padilla-Voces}.
	}
	%

%*******************
\subsubsection{Cererols}

A similar example of a villancico that includes multiple metamusical topics is \wtitle{Fuera, que va de invención} (E-Bbc: M/760) by Joan Cererols (1618--1680), monk and chapelmaster at the Benedictine Abbey of Montserrat near Barcelona.%	
	%
	\footnote{%
	\autocite[81--94]{Cererols:MEM-VC}.
	Another villancico by Cererols is the subject of chapter~\ref{ch:Cererols}.
	}
	%
The piece summons up all the elements of a Christmas festival---masques, \foreign{zarabandas} \gloss{sarabandes} and other dancing, lavish decorations and clothing, pipes, drums, and so on.%
	%
	\footnote{%
	The piece may be compared with the numerous catalog-like Christmas songs in English, from \wtitle{Deck the Halls with Boughs of Holly} to \wtitle{Chestnuts Roasting on an Open Fire}.
	}
	%
As in many villancicos, the chorus acts dramatically in the role of the festival crowd, shouting affirmations (\foreign{¡vaya!}) for each element of the celebration as the soloists name them.
Whereas Padilla's \wtitle{En la gloria de un portalillo} focused primarily on the music of the historical Christmas day, the villancico of Cererols is unambiguously about celebrating \soCalled{Christmas present}.
The piece seeks a theological meaning behind the Christmas customs: the masques of Christmas, the poem says, are appropriate because in the Incarnation of Christ, \foreign{Dios se disfraza} \gloss{God masks himself}.
The villancico allows performers and listeners to celebrate the festival in two senses: to sing the praises of the Christmas feast, while also singing the praises of Christ that are appropriate to that feast. 
Cererols's original audience of pilgrims to the mountaintop shrine of Montserrat would not have sung along with this piece, but the piece still invites their wholehearted participation in the rituals of Christmas, both through enjoying the choral singing (and joining \soCalled{in spirit}, perhaps), and in the many other common-culture customs that the piece celebrates.

%*******************
\subsection{%
Imitative References to Music: Topics of Instruments, Birdsong, Dance
}
From these general examples, we may turn to more specific cases of metamusical references in villancicos.
In each of the categories in \cref{tab:VCtopics}, we may distinguish between two main ways of referring to music.
Some pieces are primarily imitative, referring to real human music-making (\term{musica instrumentalis}).
These pieces are highly intermusical (in the way a verbal text full of references to other texts is intertextual).

In contrast to this first category of imitative pieces, villancicos in a second category refer to music as more of an abstract concept (such as the higher Boethian levels of music, music as a Neoplatonic ideal, and the music of Heaven---notions that overlap in inconsistent ways in early modern thought).
Of course, the pieces in the latter group still refer to music in the abstract through the medium of real sounding music.
Some of these pieces depend more on \soCalled{intramusical} relationships---that is, musical references internal to the individual piece itself, such as melodic or rhythmic motives or internal contrasts of musical style without overt references to pre-existing styles \soCalled{outside the piece.}


%*******************
\subsubsection{Birdsong}

A frequent example of imitative musical reference in villancicos is when the ornamented vocal lines are used to represent birdsong.
In a piece called \wtitle{Sagrado pajarillo} \gloss{Little sacred bird}, Zaragoza composer José de Cáseda sets the lyrics \foreign{con gorgeos} \gloss{with trills} to twittering melismas (example~\ref{ex:CasedaJ-Sagrado_pajarillo}).%
	%
	\footnote{%
This piece comes from the archive of the Conceptionist Convento de la Santísima Trinidad in Puebla de los Ángeles and is now preserved at CENIDIM in Mexico City (MEX-Mcen: CSG.155).
Like many other pieces in that collection, the original lyrics (beginning \quoted{Sagrado pajarillo}) were replaced by another text (beginning \quoted{Fecunda planta viva}), which was pasted and sewn over the original words with thin strips of paper.
The original may still be seen by lifting the strips.

For another example of the bird trope by this composer, see chapter~\ref{ch:Zaragoza}.
	}
	%
% 
% %*******************
% \begin{example}
% \includegraphics[width=\linewidth]{scores-examples/CasedaJ-Sagrado_pajarillo-ex}
% \caption{Bird-like trills in Cáseda, \wtitle{Sagrado pajarillo}, excerpt from the estribillo, Tiple I-1}
% \label{ex:CasedaJ-Sagrado_pajarillo}
% \end{example}
% %*******************
% 
%*******************
\subsubsection{\term{Clarines}}

Another common example of the imitative, intermusical type would be a piece that mentions \term{clarines} \gloss{clarions or bugles}, in which the singers perform patterns that are meant to sound like brass fanfares.
The typical style of \wtitle{clarín} evocations may be seen in two examples from the archive of the Escorial, which holds much of the surviving repertoire of the Spanish Royal Chapel.
Most clarín pieces do not actually feature written-out clarín parts; in most cases the instrument is imitated vocally or by other instruments, like \term{chirimías}.
Matías Durango's \wtitle{Cajas y clarines} \gloss{Drums and Bugles} (E-E:~Mús.29/15, example~\ref{ex:Durango-Cajas_clarines}) evokes these instruments with voices and chirimías in martial style, as part of a broader \soCalled{battle} topic.
Durango's clarín topic is strikingly similar to one of the rare written-out villancico clarín parts, in a fragmentary piece by the prominent composer Sebastián Durón (example~\ref{ex:Duron-Dulce_armonia_clarin}).
A villancico by José Romero from about 1690, \wtitle{Suene el clarín} \gloss{Let the clarion resound} (D-Mbs:~Mus.~ms.~2914), includes an actual notated part for \quoted{los clarines de los autos}, that is, for the clarions played in the \term{autos sacramentales} or public Corpus Christi dramas. 
The sung voices layer bugle-like gestures above them, creating a more complex fanfare than the valveless instruments could play on their own.%
	%
	\footnote{%
	Edited in \autocite[655--661]{CaberoPueyo:PhD}.
	}
	%

% Keep Durango and Duron on same page
%%*******************
%\begin{example}
%\includegraphics[width=\linewidth]{scores-examples/Durango-Cajas_clarines-ex}
%\caption{%
%\term{Clarín} music imitated by voice and \term{chirimía} in Durango, \wtitle{Cajas y clarines} (E-E: Mús. 29/15), beginning of estribillo, Tiple I-1
%}
%\label{ex:Durango-Cajas_clarines}
%\end{example}
%%*******************
%
%%*******************
%\begin{example}
%\includegraphics[width=\linewidth]{scores-examples/Duron-Dulce_armonia_clarin-ex}
%\caption{%
%\term{Clarín} part in Durón, \wtitle{Dulce armonía} (E-E: Mús. 32/16), beginning of estribillo
%}
%\label{ex:Duron-Dulce_armonia_clarin}
%\end{example}
%%*******************
%

Perhaps there are few clarín parts because these instruments may not always have been allowed in church, or perhaps their music was generally improvised, like that of the Lutheran \term{Stadtpfeiffer}.
In any case, the instrument was more important as a symbol than as part of the chapel ensemble.
The \term{clarín} was used in military, royal, and apocalyptic symbolism as far back as the allegorical \foreign{clairon} fanfares in the 1454 Feast of the Pheasant hosted by the ancestor of the Habsburg monarchs, Philip the Fair of Burgundy.%
	%
	\footnote{%
	\autocites[340--380]{LaMarche:Memoires}{Bloxam:JNV}{Perkins:Patronage15C}.
The dissertation in progress by Anita Damjanovic, \ptitle{The Biblical Clarín} (University of Chicago), will explore the symbolic functions of the stock character named Clarín in Spanish seventeenth-century dramas.
	}
	%
In \wtitle{No temas, no recelas} by the famed Madrid composer Cristóbal Galán (D-Mbs: Mus. ms. 2892, from \circa1691), the voices represent \term{clarín} music in a scene of \quoted{heavenly armies} going to battle.%
	%
	\footnote{%
	Edited in \autocite[555--565]{CaberoPueyo:PhD}.
	}
	%

The battle topic was not always just a spiritual symbol: it was often used to celebrate real military victories, or to boost morale in the midst of conflicts.
The anonymous villancico \wtitle{Noble clarín de la fama} (E-Bbc:~M/772/35) states on the cover page that it was performed \quoted{for the profession of the sisters \foreign{Señoras} Sor Sagismunda and Sor Jacinta Perpinyà into the Convent of Santa Clara of Gerona, 1693}.
The surname of these siblings (sisters by blood and now by vow) is the name of Perpignan, capital of the Catalan region of Rosselló, which had become the French Roussillon after the Peace of the Pyrennes in 1659.
A long struggle over this border territory in the War of the Great Alliance climaxed in the year this villancico was performed, as the French general Catinat scored a major victory against the allied powers at Marsaglia. %\X Check
The villancico appears to align Catalan identity with the French cause, as it praises the \quoted{Catalan Amazons, who have the name of Perpignan}, who \quoted{seek today good protection for their defense in Francisco}---that is, they look for protection both to St.~Francis, the probable patron of their order, and to France.
In enlisting for spiritual battle with St.~Francis, the estribillo suggests, the sisters themselves are becoming clarions of war.%
	%
	\footnote{%
	Excerpts from the estribillo: \foreign{Noble clarín de la fama que de vozes te alimentas, toca, toca, alarma, alarma, que dos niñas, hoy son aliento de tu voz excelsa, Catalanas amazonas, de Perpiñan nombre tienen, pues bella guardia en Francisco, buscan hoy por su defensa, cuidado serafines, resuenen los clarines}.
	}
	%

The clarín served as a symbol of music itself, much as birds did.
Spanish painters could evoke the whole realm of heavenly music and call upon rich Biblical, especially apocalyptic, symbolism by representing angels holding valveless trumpets; likewise villancico poets used the clarín as a metonym for music-making generally.
This may be another reason why the actual instrument was superfluous to many clarín pieces: there was more theological meaning in the voice imitating the clarín then there would be in the clarín by itself.
In a chamber villancico by the influential Juan Hidalgo (1614--1685), the two voices sing that just as the birds of dawn are \term{clarines} celebrating the Blessed Virgin, so too will their own voices become \term{clarines}.
Here the musical performance of a poetic metaphor turns it into a dramatic representation (poem~\ref{poem:Hidalgo-Venid_querubines_alados-poem} and example~\ref{ex:Hidalgo-Venid_querubines}).
%
%%*******************
%\begin{expoem}
%\caption{\wtitle{Venid querubines alados}, poem set by Hidalgo (D-Mbs: Mus. ms. 2895), copla 5} 
%\label{poem:Hidalgo-Venid_querubines_alados-poem}
%\input{poems/Hidalgo-Venid_querubines_alados-poem}
%\end{expoem}
%%*******************
%
%%*******************
%\begin{example}
%\includegraphics[width=\linewidth]{scores-examples/Hidalgo-Venid_querubines-ex}
%\caption{%
%Hidalgo, \wtitle{Venid querubines alados}, duo response at end of each copla
%}
%\label{ex:Hidalgo-Venid_querubines}
%\end{example}
%%*******************

%*******************
\subsubsection{Other Instruments: Onomatopoeia}

To imitate percussion instruments, villancico composers paired onomatopoetic nonsense words with distinctive rhythmic patterns.
Juan Gutiérrez de Padilla had the chorus of Puebla Cathedral represent the sound of the castanets and tabor with contrasting onomatopoetic rhythmic patterns on the words \quoted{al \emph{chaz}, \emph{chaz} de la castañuela, y el \emph{tapalatán} de el tamboríl} (example~\ref{ex:Padilla-Alto_zagales-chaz}).
Such pieces about instrumental music both imitate the instrument itself while also playing with a stylistic topic associated with that instrument.

The same instrumental trope appears in a villancico poem performed at Toledo Cathedral in 1645 in a setting by Vicente García (according to the poetry imprint, E-Mn: VE/88/12, no. 6).%
	%
	\footnote{%
	Note that the difference in spelling, \quoted{chas} instead of \quoted{chaz} in the Puebla source, reflects the difference in pronunciation between Castile and Padilla's Andalusia and New Spain, since \quoted{chaz} would have ended with a TH sound in Toledo.
	}
	%						
\begin{quotepoem}
Porque los instrumentos sonaban así,	& Because the instruments sounded like this:\\
El Atabal, tan, tan ,tan,				& the drum, tan, tan, tan,\\
El Almirez, tin, tin, tin, 				& the mortar, tin, tin, tin\\
la Esquila, dilín, dilín,				& the chime, dilín, dilín,\\ 
y la Campana, dalán, dalán,				& the bell, dalán, dalán,\\
Las Sonajas, chas, chas, chas,			& the rattle, chas, chas, chas,\\
y el Pandero, tapalapatán.				& and the tambourine, tapalapatán.\\
\end{quotepoem}

The instruments on this list are simple, rustic noisemakers from everyday peasant life.
In García's villancico these instruments, which are described further in the coplas, join together with the sounds of the mule and other animals, and the dances of the shepherds.
This piece, like many villancicos, depicts a scene of common folk rejoicing after their own fashion in the humble setting of the Bethlehem stable.

%%*******************
%\begin{example}
%\includegraphics[width=\linewidth]{scores-examples/Padilla-Alto_zagales-ex}
%\caption{Padilla, \wtitle{Alto zagales de todo el egido} (MEX-Pc: Leg. 2/1, Christmas 1653), estribillo, \range{\measures}{28--34}: Imitation of castanets and tabor (or tambourine?)}
%\label{ex:Padilla-Alto_zagales-chaz}
%\end{example}
%%*******************
%
%*******************
\subsubsection{Dance}

Dance topics are another prevalent form of imitative musical reference in villancicos.
Some pieces refer to dances like the \term{zarabanda}, \term{papalotillo}, or \term{danza de espadas}, while others actually proclaim themselves to \emph{be} dances, with the \term{jácara} being perhaps the most common.
In some cases, the stylistic referents can be identified by comparison with contemporary collections of dance music for keyboard or guitar, though there is always uncertainty regarding the relationship of the notated exemplars to the way the dances were actually played, and further doubt about what kinds of physical dances they may have accompanied.%
	%
	\footnote{%
	For example, the labeled dances in \shortcite{Ruiz:Luz} are little more than schemata, mostly only a few measures long and lacking melodic contours and larger structures.
	The elaborate versions of these dances recorded by Andrew Lawrence-King and the Harp Consort, \headlesscite{Lawrence-King:DancesCD}, can only claim the most fanciful of connections to the original notation.
	}
	%
In the case of the \term{jácara}, the distinctions between the poetic, theatrical, musical, and dance versions of this puzzling genre still need to be clarified.%
	%
	\footnote{%
	This subgenre deserves a study in its own right, and will be the topic of a future investigation. 
	The first work in this direction is \autocite{Torrente:Jacara}.
	}
	%
Though the details may be unclear, these pieces do appear to refer to some form of music outside themselves, which was probably familiar to listeners.%
	%
	\footnote{%
	The question of whether performers or listeners actually danced in church is another problem here, related to the question of whether or to what degree performers staged the dramatic villancicos in the sacred space.
	There certainly was ritual dance on Corpus Christi in Seville and Valencia Cathedrals, performed by the boy \term{seises}.
	Music for these dances in Valencia survives by \autocite{Comes:Danzas}.
	Understanding the function of the Comes pieces, and any other such sets that may survive, would shed light on the function of Corpus Christi villancicos as well.
	For now we may simply note that there are strong similarities between this Comes set and the first set of villancicos composed for Puebla Cathedral by Juan Gutiérrez de Padilla, for Corpus Christi 1628.
	}
	%

%*******************
\subsection{%
Abstract References to Music: Singing about Singing, Solmization, Nonsense
}

In the second category of metamusical villancicos are pieces that refer to music more as an abstract concept, rather than to a specific, identifiable reference to another kind of music.
When Pedro Ruimonte in \wtitle{Gil, pues a cantar} sets the word \foreign{cantar} \gloss{sing} to a long melisma, or when Gaspar Fernández in \wtitle{Sobre bro canto llano} illustrates the phrase {canto llano} \gloss{plainchant} with imitative counterpoint around a cantus-firmus-like Tenor part, the composer is using these characteristic emblems of vocal music to refer to the concept of singing in general.%
	%
	\footnote{%
	See note~\ref{fn:Ruimonte} above.
	}
	%
In doing so they are asking their singers to sing about singing.

Many villancico poems use solmization syllables in the poetry as part of a reference to singing.
References to Christ as \term{sol} \gloss{sun} are ubiquitous, and as shown in the opening example by Padilla, composers missed no opportunity to put this word on a pitch that could be solmized with that syllable (G, C, or D in the three Guidonian hexachords).
The more obvious this technique was, the better: in composer Miguel de Aguilar's \term{oposición} \gloss{audition} piece for a position at Zaragoza, \wtitle{Mi sol nace y tiembla} \gloss{My sun is born and is trembling}, it is not hard to guess the opening pitches.%
	%
	\footnote{%
	E-Zac:~B-11/233, \quoted{Villancico de Oposición en Zaragoza}, edited in \autocite[34--64]{Ezquerro:MME55}.
	}
	%
Solmization syllables were sometimes used for their own sake, without a symbolic meaning, somewhat like the \quoted{fa la la} refrains in contemporary English madrigals; in these cases the vocalists are \soCalled{singing about singing} in the most simple sense.
The voice in such cases bears no message except the musical voice itself.

These gestures, though seemingly without meaning, must be understood within a Neoplatonic system, which will be explained fully in the next chapters.
In the prevailing Catholic understanding of the world, every created thing, simply by being itself, reflected the nature of God, its Creator.
The human body was the microcosm, reflecting in turn both the whole Creation and the Creator who took on a human body in Christ.
The voice emanated from the body and expressed the essence of the one speaking or singing to another who heard it.
So the voice itself had quasi-sacramental meaning as an expression of Man the microcosm and a reflection of the Creator; and this meaning was independent of linguistic communication or even of music's own non-linguistic structures, which were understood by analogy to rhetoric.%
	%
	\footnote{%
The Victorian Catholic priest Gerard Manley Hopkins would later encapsulate this idea in a striking verse, drawing on the Neoplatonic ontology of Duns Scotus (Eriugena), \quoted{Each mortal thing does one thing and the same:/ Deals out that being indoors each one dwells; Selves---goes itself; \emph{myself} it speaks and spells;/ Crying \emph{What I do is me: for that I came.}} \quoted{As kingfishers catch fire}, in \autocite[95]{Hopkins:Poems}.

This theological view of the nature of the pure voice might be fruitfully contrasted with the \quoted{aesthetics of pure voice} that Mauro Calcagno has identified in Venetian productions of the Accademia degli Incogniti, \headlesscite{Calcagno:SignifyingNothing}.
These early modern perspectives should inform more recent discussions of philosophy of voice, particularly the idea of the \soCalled{voice itself} as separate from meaning, from \autocite{Barthes:GrainOfVoice} through \autocite{Dolar:Voice} and \autocite{Cavarero:Voice}.
	}
	%

Passages of self-conscious solmization are not alluding to a particular kind of song; rather, their song is pointing to the abstract category of \soCalled{singing} in general.
In a piece like Aguilar's \wtitle{Mi sol nace}, the words have dual function: on one side they communicate linguistic meaning (\quoted{my sun}), but on the other side these musical syllables go beyond language, to both symbolize and embody music-making. 
Aguilar made this obvious gesture at the beginning of a piece intended to demonstrate his own skill at composition; this strengthens the thesis developed throughout part~\ref{part:Singing} that metamusical villancicos as a subgenre served composers as \soCalled{master pieces} to prove their craft.
At the same time, the syllables could also take on deep symbolic meanings, as in the Christological \quoted{sign of A \term{(la, mi, re)}} discussed in chapter~\ref{ch:Padilla-Voces}.

% START Padilla, symbolism of solfa

%*******************
\subsection{%
Mixed References in \quoted{Ethnic} Villancicos
}

Like solmization villancicos, the subgenres of \soCalled{ethnic} villancicos play with the relationship between language and music. 
But instead of having singers \quoted{speak the language of music} with solmization syllables, these pieces have the performers speak distorted caricatures of the Spanish language, often turning speech into nonsense.
Though these pieces are rife with metamusical references---depictions of dancing, playing instruments, and singing in other languages---their primary subject is not music, but cultural difference.
These pieces parody the speech and culture of non-Castilian peoples by twisting Castilian Spanish to imitate their distinctive accents. 
For example, \term{gitano} \gloss{Roma or \soCalled{gypsy}} characters put a Z on the end of everything, as in \wtitle{Vamos al Portal Gitanilla}, a 1666 villancico from Zaragoza with poetry by Vicente Sánchez and music (lost?) by Joseph Ruiz Samaniego.%
	%
	\footnote{%
	Imprint from Epiphany (\foreign{Reyes}) 1666 at the Iglesia del Pilar in Zaragoza, E-Mn: VE/1303/1.
	The printed title lists Ruíz Samaniego as the chapelmaster, which would usually mean he was the composer. 
	A handwritten note on the page cross-references the imprint to the posthumous poetic works of Vicente Sánchez, \headlesscite[203--204]{Sanchez:LiraPoetica}.
	Other villancicos by this composer have been edited in \autocite{RuizSamaniego:MME63}.
	}
	%
A \soCalled{gypsy} woman responds to catechism-like questions from a \soCalled{normal} speaker.
When he asks, \quoted{who are these who have arrived [at the manger] with such grandeur?}, she responds, \quoted{Oyeme,/ que zerán imagino;/ puez los trez zon Magoz,/ hombrez de ezfera}.
Performers in Zaragoza, speaking Castilian Spanish, would probably have pronounced all these Zs with a lisping TH sound.
The misspelling in the poetry imprint serves as a visual marker of otherness, and points toward a probably fuller set of caricatured features like physical posture and perhaps even costume employed in the original dramatic performance.%
	%
	\footnote{%
	This should be compared with the mock-catechism scenes with \soCalled{deaf} men discussed in chapter~\ref{ch:theology}.
	}
	%

Even more notoriously, in the subgenre of \term{negrillas} or \soCalled{black villancicos}, black characters portrayed by white musicians speak in a stylized false dialect, the lexical meaning of which is often difficult to recover.%
	%
	\footnote{%
Some have taken \term{negrillas} to provide transparent ethnolinguistic evidence for the historical dialect of black Spanish speakers, but the situation is more complex.
	% (see \X new dissertation on "habla de negros").
As will be argued in chapter~\ref{ch:Puebla} with reference to the ethnic villancicos of Juan Gutiérrez de Padilla, these pieces may offer glimpses of black language and music, but only through a glass heavily darkened by racial prejudice and deliberate caricature for the sake of humor, and further clouded by the cultural distance from which modern observers must approach these pieces. 
	}
	%
The ethnic villancicos do not stop with parodying the sound of minority groups' speech in Spanish (itself a notable feature for this study of hearing and faith), but they extend further by actually having the characters sing foreign-sounding nonsense syllables like \quoted{gulumbé, gulumbá}, \quoted{tumbucatú, catú, catú}, or \quoted{turuturuyegá}.%
	%
	\footnote{%
	These examples are from the most famous composer of \term{negrillas}, Juan Gutiérrez de Padilla, because these works will be discussed in chapter~\ref{ch:Puebla}, but it should be noted that the \term{negrilla} genre was widespread across the Hispanic world, with many surviving examples in mainland Spain, and that these Peninsular examples are highly similar to those from America. 
	}
	%
This linguistic nonsense is paired (like the onomatopoetic \quoted{chaz, chaz de la castañuela}) with distinctively rhythmic musical motives. 
This produces the effect that the words seem to embody a form of speech perceived by Spaniards as closer to music than to language.
At the same time, these phrases evoke the action of the black characters' musical performance, such as African drumming and dance---as imagined by Spaniards.

These cultural Others, then, are represented as essentially musical.
As Geoffrey Baker has noted, black characters in villancicos do nothing but sing and dance happy songs all day, reiterating an age-old Euro-American stereotype about people of African descent.%
	%
	\autocite{Baker:EthnicVC}
	%

With regard to the nature of musical references in \soCalled{ethnic} villancicos, then, these pieces encompass a mixture of literal imitation (as of percussion, and of the \soCalled{musical} sound of foreign languages) and broader stylistic references (as, perhaps, to African musical styles, though no one has yet proved this). 
They also include more abstract references to music through the use of nonsense words that, somewhat like solmization syllables, symbolize and enact music-making.


%*******************
\subsection{%
Villancicos about Villancicos
}

The conventions of the villancico genre itself become the subject of a special type of self-referential villancico \emph{about} villancicos---a \soCalled{metavillancico}, perhaps.
These pieces stage performances of a villancico within the villancico itself.

In the surviving sets of villancico poems, one sees distinct recurring subgenres like angel pieces or dance pieces, and the conventions for these strengthen throughout the seventeenth century.
By the second half of the century, the patrons and creators of villancicos seem to have developed well-reinforced expectations not only for different types of villancicos, but also possibly for the dramatic shape of the whole villancico cycle.%
	%
	\footnote{%
	At present we may note only anecdotal evidence suggesting that certain subgenres like the jácara tended to recur at the same positions in the Matins cycle, which may be linked to the texts of the underlying lessons and Responsories.
	}
	%
In the smaller number of surviving musical settings of complete cycles, such as those in Puebla by Juan Gutiérrez de Padilla studied in chapter~\ref{ch:Puebla} and those in Segovia by Miguel de Irízar studied in chapter~\ref{ch:Segovia}, we may observe corresponding conventions on the musical plane for each of these subgenres.
A conjunction of markers in the poetic subject matter and in the poetic and musical style would have signaled to listeners, \quoted{This is one about angels}, or \quoted{This is one with those silly Catalans (or Galicians, or blacks)}.
Miguel de Irízar's requests to his chapelmaster peers for more \quoted{villancicos de chanza}---comic villancicos---suggests the need to fill out each villancico cycle with certain types of pieces, mixing serious and comic subgenres.%
	%
	\autocite[78]{Olarte:Irizar}
	%

In one sense, the many pieces beginning \quoted{Listen,} \quoted{Pay attention,} might be considered self-referential, since in these pieces the singers usually announce something about the piece, as in the setting of \wtitle{Oigan, oigan la jacarilla} by José de Cáseda (MEX-Mcen:~CSG.151), or the poem performed in Montilla in 1689, \wtitle{Oíganme cantar una tonadilla}---\quoted{hear me sing a tonadilla}.%
	%
	\footnote{%
	Pliego no.~334 in \autocite[116]{BNE:VCs17C}.
	No signature is listed.
	}
	%
But some villancico poems go further and overtly play with the conventions of the villancico genre itself.

This rhetorical posture seems to stem from two quite distinct sources: the liturgical psalms and Spanish minor theater.
Similar self-referential or even recursive statements may be found throughout the psalms, as in \quoted{Sing to the Lord a new song} (\bibleverse{Ps}(97:)), or \quoted{Come, let us worship and bow down} (\bibleverse{Ps}(95:)), which was sung as the Invitatory hymn at every Matins liturgy (and thus before most villancico performances). 
The first Responsory of Christmas Matins describes and enacts the angels' \quoted{Gloria}, and the third Responsory is devoted to the shepherds' adoration.%	
	%
	\footnote{%
	\autocite[172--173]{Catholic:Breviarium1631}.
	See the discussion of the Matins text in chapter~\ref{ch:Padilla-Voces}.
	}
	%
On the other side, Spanish minor theater developed as low-register plays (\term{entremeses}) in between acts of the more literary \term{comedias} by, for example, Lope de Vega and Calderón, much as eighteenth-century Italian \term{opera buffa} emerged from \term{intermezzi} during \term{opera seria} performances.
Like the later Italian musical intermezzi, the Spanish entremeses were built out of stock characters and formulaic scenarios, and parodied the conventions of both the \term{comedia} and of the \term{entremés} itself.

%*******************
\subsubsection{Playing with Villancico Conventions: \wtitle{Anton Llorente y Bartolo}}

The anonymous villancico \wtitle{Anton Llorente y Bartolo} (example~\ref{ex:Anton_Llorente}) presents two characters from a well-known \term{entremés} with close links to Cervantes' \wtitle{Don Quijote}, who raise a complaint against the stock villancico shepherd characters, Gil and Bras.%
	%
	\footnote{%
The villancico poem is found in a 1639 Christmas imprint from Toledo Cathedral (E-Mn: VE/88/6), and an anonymous musical setting survives from the Convento de la Santísima Trinidad in Puebla (MEX-Mcen: CSG.014).
The title has been erroneously \quoted{corrected} in the forthcoming Sánchez Garza catalog to \wtitle{Anton, Lorente y Bartolo} despite the clear double L in the manuscript (which was never used when a hard L sound was intended).
	}
	%

%*******************
\begin{quotepoem}
%
Antón Llorente y Bartolo	& Antón Llorente and Bartolo\\
trazaron un memorial		& drew up a complaint\\
de que con los villancicos	& that with all the villancicos\\
se han alzado Gil y Bras.	& Gil and Bras have gotten the spotlight.\\
%
\end{quotepoem}
%*******************

These two characters, they assert, have held the stage for too long; but Anton Llorente and Bartolo, too, can make a good enough villancico of their own if given the chance.

%*******************
\begin{quotepoem}
%
Si ha de sonar el pandero, 		& If the tambourine is going to be played,\\
solo Gil le ha de tocar,		& it is only Gil who ever plays it,\\
y si ha de haber castañetas,	& it if there have to be castanets,\\
ha de repicarlas Bras.			& Bras is the one to rattle them.\\[1ex]
%
También acá somos gentes		& But here we are, we too are good fellows,\\
y alcanzar podemos ya			& and we can even manage\\
de un villancico un bocado		& a nibble of a villancico\\
y un pellizco de un cantar.		& and a pinch of a song.\\
%
\end{quotepoem}
%*******************

In the succeeding \term{responsión}, the full eight-voice chorus joins in endorsing the new characters and denouncing the old:

%*******************
\begin{quotepoem}
%
No quiero que me Brasen				& I don't want them Bras-ing me\\
\tabindent\ y que me Gilen,			& \tabindent\ or Gil-ing me,\\
sino que me Llorenten				& but only Llorente-ing me\\ % or NEITHER ? \X
\tabindent\ y me Toribien.			& \tabindent\ y me Toribien.\\[2ex]
%
\end{quotepoem}
%*******************

The surviving musical setting for this  embodies all the stereotypes of the villancico genre, first encountered here in Padilla's \wtitle{En la gloria de un portalillo}.%
	%
	\footnote{%
As will be discussed in chapter~\ref{ch:Padilla-Voces}, the Seville chapelmaster Fray Francisco de Santiago wrote a setting of this text, which was cataloged as part of the now-lost library of King John IV of Portugal, and it is possible that the version in Puebla is Santiago's music. \autocite[caixão 26, no. 675]{JohnIV:Catalog}.
	}
	%
The piece is in highly accented triple (CZ) meter with continual use of sesquialtera, and it opens with a declamatory section for full chorus, followed by a vocal solo that is then echoed in polychoral dialogue by the full ensemble (example~\ref{ex:Anton_Llorente}).
The text is both dramatized and symbolized by leaping gestures that leap from voice to voice in points of imitation on \quoted{salten y brinquen} (jumping and leaping).
These features may just constitute typical villancico style, or they may be taken to \emph{represent} typical villancico style.
The highly conventional music casts the anticonventional text into relief while also dramatizing the scene, since the piece is meant to portray Anton Llorente and Bartolo performing a villancico.
This is a villancico, then, in the style of villancicos.
%	See also in lafragua connection; uber-conventional villancico next to anticonventional one \X

% %*******************
% \begin{example}
% \includegraphics[width=\linewidth]{scores-examples/Anton_Llorente-ex}
% \caption{Anonymous, \wtitle{Anton Llorente y Bartolo} (MEX-Mcen: CSG.014), first stanza of introducción and beginning of responsión (Accompaniment omitted)}
% \label{ex:Anton_Llorente}
% \end{example}
% %*******************
% 
As though the 1639 Toledo text were not self-referential enough, the creative team at the cathedral followed up the next year with another villancico that specifically referred back to \wtitle{Anton Llorente y Bartolo}.%
	%
	\footnote{
	\ptitle{Quejosos de la sentencia que dio el alcalde Pasqual}, in imprint from Christmas 1640 at Toledo Cathedral, E-Mn: VE/88/7, no. 2.
	}
	%
The narrator says that the \quoted{Brases} and \quoted{Giles} were \quoted{frustrated by the sentence that Mayor Pasqual decreed against them last Christmas} (the \quoted{Mayor of Bethlehem} was another stock character in comic villancicos); and so \quoted{they appealed to another one who was more learned}.
Each one states his case for why he is needed at the Nativity, and Bras's conclusion wryly sends up the conventionality of villancico poetry:

%*******************
\begin{quotepoem}
Cuanto ha qué Belén lo es,	& As long as Bethlehem has been what it is,\\
y ha sido el portal portal,	& and the stable has been a stable,\\
a peligros de poetas		& where poets have been in danger,\\
ha sido socorro Bras.		& Bras is always there for aid.\\
\end{quotepoem}
%*******************

The new mayor, in the name of keeping traditions, undoes the sentence of the previous year, and the chorus rejoices, because without Bras and Gil it would not be Christmas:

%*******************
\begin{quotepoem}
\emph{[Alcalde.]} 									& \emph{Mayor.}\\
Que me Brasen, y Gilen								& I wish them\\
quiero zagales,										& to \quoted{Bras} me and \quoted{Gil} me, boys\\
porque no soy amigo									& because I am no friend\\
de novedades.										& of novelties.\\
\Dots												& \\[1ex]
%
\emph{[Chorus.]} 									& \emph{Chorus.}\\
Porque en saltando a esta fiesta					& For if you take from this feast\\
el pesebre, y el portal,							& the manger, the stable,\\
las pajas, Brases, y Giles,							& the straw, Brases, and Giles,\\
no es fiesta de Navidad.							& it is no festival of Christmas.\\
\end{quotepoem}
%*******************

As the mayor says, one reason villancicos were so conventional may be because the feast they were most closely associated with was (and is) one where customs are carefully preserved.
Novelty at Christmas is expected but only within certain traditional boundaries.
Part of cultivating those customs meant naming them explicitly in song, as we have already seen in Cererols's \wtitle{Fuera que va de invención} and several other pieces.%
	%
	\footnote{%
In a North American setting, these self-preserving, self-referential customs might be compared with the phenomenon of Christmas-tree ornaments in the shape of a Christmas tree.
	}
	%

Comic villancicos like these should not be written off as less theologically motivated than the more cultivated pieces.
Though the \wtitle{Anton Llorente} pieces present no learned theological doctrines, they still serve a religious function in prompting hearers to laughter and enjoyment, and that function contributed to the effect of a set of villancicos within the liturgy.
The comic pieces may even have been the most likely to provoke direct responses of laughter and surprise in many listeners, and therefore could be the most effective in actually moving those assembled toward sympathetic vibration and harmony together. 

%*******************
\section{%
Sign and Signified 
}

The previous discussion has distinguished between ways of referring to music (imitative or abstract), and between distinct kinds of music referred to (dance styles, birdsong).
These are semiotic distinctions, and they articulate different relationships between sign (the music performed) and signified (the other music to which the performed music refers).
The relationships between sign and signified in villancicos about music may be distinguished by a loose application of C. S. Peirce's semiotic terms \term{icon}, \term{index}, and \term{symbol}.%
	%
	\footnote{%
The terms are used here as Peirce defined them in 1903, according to the synthesis of Albert Atkin, as a trichotomy of ways that the object of a sign worked in the process of signification.
\autocite{Atkin:Peirce}.
Our primary goal is not to further Peircian semiotic theory, but to use Peirce's distinctions to clarify the function of musical signs in metamusical villancicos.
See also \autocite{Turino:Signs}.
	}
	%
To be an icon, the sign must \quoted{reflect qualitative features of the object}, as in \quoted{portraits and paintings}.
An index utilizes \quoted{some existential or physical connection between it and its object}, such as \quoted{natural and causal signs} (smoke as a sign of fire) and also \quoted{pointing fingers and proper names}.
As a symbol, the sign utilizes \quoted{some convention, habit, or social rule or law that connects it with its object}, as in most speech acts.%
	%
	\autocite[§3.2]{Atkin:Peirce}
	%

A bird-like trill gesture in vocal music functions as an icon, then, to the extent that the listener connects it with the actual sound of birdsong; likewise for imitations of the \term{clarín} or castanets. 
The same gesture always also functions as a symbol, especially the more stylized and conventional it is; in other words, the trill reminds the listener of birdsong because it sounds like similar gestures in other pieces of music that also imitate birdsong.

In the case of the \term{clarín} pieces, though, the piece refers not only to the sound of that instrument but to the type of music that instrument usually plays and thus would be an index.
A clearer example is a piece that names a specific type of dance music and seems to quote or allude to that style of music.
In the few cases when we can actually identify the stylistic referent in contemporary collections of dance music, we can verify that there is a \soCalled{factual} correspondence between the sound of the villancico and the sound of the dance style mentioned in the villancico.
This type of reference is closest to Peirce's notion of index.

In the abstract type of metamusical villancico, these relationships become much harder to track.
In a villancico that literally refers to itself, as in the many pieces that begin by inviting the audience to listen to the piece currently being performed, the piece itself is both sign and signified.
In a villancico that refers to human music as an abstract category, as in the solmization example mentioned above, the music heard symbolizes the notion of music in the abstract.

%*******************
\subsection{%
Pointing to a Higher Music: Topics of Heavenly and Angelic Music
}

When a villancico refers to the music of the spheres or to angelic music, the music signified is impossible to hear with earthly ears, so the human music as a sign is only an icon or index to the extent that a listener believes it to correspond to what those higher forms of music might actually sound like.
These pieces depend to a high degree on conventional---symbolic---ways of evoking heavenly music, which are developed over time within an interpretive tradition (as shown in part~\ref{part:Singing}).

One of these conventional ways of evoking heavenly music is to set up a contrast between indexical references (stylistic allusions or quotations) pointing to types of human music with different value in a hierarchy of musical styles.
Villancicos on topics of angelic and heavenly music provide an interesting case for a semiotic analysis, since they use references to elevated forms of human music in order to refer to an unheard higher music of heaven.
Similar to the way German Lutherans used learned counterpoint to symbolize heavenly music (as David Yearsley has shown), Spanish Catholics used old-style contrapuntal music, particularly canons and fugues, to point to higher Neoplatonic levels of music.%
	%
	\footnote{%
\autocites{Yearsley:Buxtehude}{Yearsley:BachThron}.

This topic of heavenly music is one of the most potentially fruitful areas for interconfessional research, since the ideas Yearsley discusses may be found in similar form in Lutheran, Catholic, and Anglican sources. 
For example, Lutherans mapped the relationship of the boys' choir to the congregation onto that between the angelic chorus and the church; see \autocite{Cashner:Gerhardt} and cf. chapter~\ref{ch:Puebla}.
	}
	%
Most typically, Hispanic composers use polyphonic techniques and styles reminiscent of Palestrina, Guerrero, and Morales to represent angelic music, as demonstrated in chapter~\ref{ch:Cererols}.

In this way a form of earthly music is placed relatively higher on a Neoplatonic chain and used to symbolize an even higher level of music. 
Each style by itself functions as an index pointing to these styles of earthly music, such as (on the higher level) a fugue in duple meter, indexing typical Latin liturgical music in Spain; and (on the lower level) the homophonic declamation in triple meter typical of vernacular villancicos.
The contrast between these two styles, however, functions symbolically, so that the relationship of higher and lower forms of earthly music is mapped onto the imagined relationship between heavenly music and all earthly music.%
	%
	\footnote{%
	See chapter~\ref{ch:Cererols} and \ref{ch:Zaragoza} for detailed examples of this practice.
	}
	%

This is only a simplified description of how heavenly music is represented, however; the stylistic references do not usually form such a tidy dichotomy.
The question of heavenly music is also complicated by the possibility of multiple kinds of \soCalled{superterrestrial} music: the perpetual song of the angelic choirs in heaven, the song of the redeemed at the Last Day, the harmonies of the cosmic spheres, and so on.
It is particularly important to distinguish between the \soCalled{the heavens} (\term{cielos} in Spanish), meaning the dome of the sky and the planetary spheres, and \soCalled{Heaven} (\term{el cielo Empyreo}), meaning the spiritual realm outside of the material world where the Godhead dwells with the angels and saints.

%*******************
\subsection{The Angelic Trope: Salazar, \wtitle{Angélicos coros}}

A typical example of the angelic trope is \emph{Angélicos coros con gozo cantad} (MEX-Mcen: CSG.256), a Christmas villancico by Antonio de Salazar (\circa1650--1715), preserved in a collection from a Conceptionist convent in Puebla de los Ángeles (example~\ref{ex:Salazar-Angelicos_coros-1}).%
	%
	\footnote{%
	See appendix~\ref{app:poems} for the complete poem, and appendix~\ref{app:scores} for the complete musical edition.
	
Salazar was probably born in Puebla and may have sung in the Puebla Cathedral chapel under Juan Gutiérrez de Padilla; he served as chapelmaster of Mexico City Cathedral from 1679: \autocite{Koegel:Salazar}. 
The Sánchez Garza collection features numerous pieces by Salazar, probably composed or arranged specifically for this convent.
	}
	%
The anonymous poem echoes the first Responsory of Christmas Matins (\quoted{Gaudet exercitus Angelorum}) as it invites the choirs of Christmas angels to sing their \quoted{Gloria} over the stable in Bethlehem on the night of Christ's birth.
Since \mentioned{Bethlehem} in Hebrew means \quoted{House of Bread}, the villancico also celebrates the sacramental presence of Christ in the Eucharistic host on the Christmas of Salazar's \soCalled{present day}.

Though the words speak to the angels, the musicians who sing these words also play the part of the angels, so that hearers are invited to listen for the angelic voices \emph{through} the voices of the church ensemble. 
The invocation to the angels is sung first by the Tiple I, in a gesture beginning with a rising fifth and then falling by step, as though looking up to the heavens and then following the angels' descent.
In the Puebla convent choir, this part was performed by \quoted{Madre Andrea}, whose name is written into her part.
As though answering the call, the other two voice parts of Chorus I enter in \range{\measures}{2}, Tiple II in canonic imitation, and Alto I harmonizing with it homorhythmically. 
In \range{\measures}{4-5} the second chorus joins with a similar imitative pattern, until all join together in a lilting, dancelike cadence on \foreign{cantad}.
Salazar uses contrapuntal imitation again on \foreign{celestes esquadras}, inverting the opening motive (\range{\measures}{14-22}).
For the command \foreign{bajad} \gloss{come down}, Salazar switches from CZ triple meter to duple (C or \term{compasillo}), and creates a cascading contrapuntal passage passed from voice to voice, moving from high F\octave{5} down to C\octave{3} (example~\ref{ex:Salazar-Angelicos_coros-2}).
The general affect of the piece seems gentle and sweet, partly because of the largely static diatonic harmony and the lilting or dotted rhythms.

% %*******************
% \begin{example}
% \includegraphics[width=\linewidth]{scores-examples/Salazar-Angelicos_coros-ex1}
% \caption{%
% Salazar, \wtitle{Angélicos coros con gozo cantad} (MEX-Mcen: CSG.256), \range{\measures}{1-9}
% }
% \label{ex:Salazar-Angelicos_coros-1}
% \end{example}
% %*******************
% 
% %*******************
% \begin{example}
% \includegraphics[width=\linewidth]{scores-examples/Salazar-Angelicos_coros-ex2}
% \caption{%
% Salazar, \wtitle{Angélicos coros con gozo cantad}, \range{\measures}{22--31} 
% }
% \label{ex:Salazar-Angelicos_coros-2}
% \end{example}
% %*******************
% 

All of these musical characteristics are typical ways that villancico composers represented angelic music: especially contrapuntal imitation, in a reference to the ordered music of heaven, and symbolic patterns of ascent and descent.
Salazar uses different styles of earthly music---particularly the contrast between contrapuntal and homophonic styles---to point to the contrast between different levels of music on a cosmic scale, between human music, music of the spheres, and angelic song.
Because the triple-meter style of the first section, which asks the angel choirs to sing, is more typical of villancico style, this part might be heard to represent the actual singing of the angels.
The duple-meter section on \quoted{bajad} might be understood as a more literal portrayal of the angels themselves.
There is not, though, any obvious one-to-one mapping of style to symbol.
It would be difficult to fit such a piece into a Peircian model.

Understood in a more historical model of theological symbolism, the piece connects Boethian \term{musica instrumentalis} to the higher forms of human and cosmic music. 
\soCalled{Heavenly} villancicos map a lower level of music onto a higher one within the Neoplatonic cosmos, in which the perceptible \soCalled{world of change and decay} is an imperfect reflection of a higher world of ideal forms.
%\citX{Augustine De doctrina christiana on signs,symbols, sacramental presence? Salazar piece connecting sacrament of Xmas with sacrament of EuX with multiple kinds of musical presence}
Thus earthly music of any kind, metamusical or not, would always point beyond itself to higher forms of music and ultimately to God.

Metamusical pieces intensify this aspect of music by calling the listener's direct attention to the artifice of the music itself.
Such pieces give listeners the opportunity to rise in Neoplatonic contemplation from what is heard by the ears to the higher music (ultimately of the divine nature) that can only be discerned by the soul through faith.


%*******************
\section{%
Music Itself as a Conceit
}

Villancicos, we have seen, may refer to other kinds of music or even to themselves; a special subgenre of villancico makes music itself the governing conceit for the whole piece.
Such pieces often play on technical musical terms to create a double discourse about both music and theology.
The most renowned of villancico poets today, Sor Juana Inés de la Cruz (1651--1695), used the conceit of Mary as a heavenly chapelmaster to create such a piece for the feast of the Assumption in Mexico City, 1676, though no musical setting survives.%
	%
	\autocite[no.~220, p.~7]{SorJuana:VC}
	%
The estribillo exhorts congregants to listen for Mary's voice (poem~\ref{poem:Silencio-Maria-Sor-Juana}).
The coplas demonstrate how much theology could be drawn from musical terms, and how much knowledge of both domains is necessary to understand both sides of the concept.

%%*******************
%\begin{expoem}
%\caption{Sor Juana Inés de la Cruz, \emph{Silencio, atención, que canta Mariá}, excerpts}
%\label{poem:Silencio-Maria-Sor-Juana}
%	\input{poems/Sor_Juana-Silencio_atencion_Maria}
%\end{expoem}
%%********************

As the succeeding chapters will show, when poetry like this was set to music, composers had the opportunity to match this intricate musical-theological discourse with another layer of symbols in the sounding music.
It should be kept in mind that villancico poems were written specifically as lyrics for musical compositions, as Juan Díaz Rengifo stated in one of the first literary descriptions of the genre.
	%
	\autocite{Rengifo:ArteMetrica}
	%
Poems like Sor Juana's circulated independently of musical settings through the medium of the printed commemorative poetry leaflets, which composers circulated widely across the empire.%
	%
	\footnote{%
	Chapter~\ref{ch:Segovia} demonstrates how Segovia chapelmaster Miguel de Irízar composited the poetic texts for his 1678 Christmas cycle from several poetry leaflets sent to him by colleagues.
	}
	%
Composers had every reason to favor villancico poems that gave them with opportunities for clever musical craftsmanship, and musically knowledgable poets like Sor Juana were motivated to provide them.%
	%
	\footnote{%
	On Sor Juana's musical knowledge, see \autocite{Stevenson:SorJuanaMusicalRapports}.
	}
	%
It is also possible that in many cases the composers themselves wrote the poetic texts, in which case they likely already had ideas for the musical setting.

Sor Juana is writing within a Spanish literary tradition of \term{conceptismo}, in which poets, especially those under the spell of Luis de Góngora, developed poems from ingenious extended metaphors.%	
	%
	\footnote{%
	On \term{conceptismo}, see \autocites{Tenorio:Gongorismo}[227--228]{Gaylord:Poetry}; for the most important period source, see \autocite{Gracian:Ingenio}. 
	}
	%
In the most finely wrought villancicos within this Gongoresque tradition, such as those studied in chapters~\ref{ch:Padilla-Voces} and \ref{ch:Cererols}, the whole poem can be read in two ways simultaneously, so that the poem says something meaningful about music while also using the musical terms to speak metaphorically about theology.%
	%
	\footnote{%
	Stevenson ventures a theological and musical interpretation of this villancico in \headlesscite[16--17]{Stevenson:SorJuanaMusicalRapports}.
	}
	%
Many metamusical poetic texts, however, work at a simpler level, providing the composer with an excuse to play with musical techniques, but not necessarily making any profound theological statement.

%*******************
\subsection{Sounding Number: Hidalgo}

One such piece is \wtitle{Cuando el Alba aplaude alguno} (D-Mbs: Mus. ms. 2895), a villancico for three voices (probably soloists) and continuo by Juan Hidalgo (1614--1685), who was court harpist to Philip IV and the cocreator with Calderón of the first fully sung Spanish music dramas.%
	%
	\footnote{%
See \autocite{Stein:Songs}.
This piece survives today in Munich, as part of a group of villancicos by composers associated with the royal musical institutions in Madrid and purchased by a German collector during the nineteenth-century Romantic vogue for Golden Age Spain.
See Bernat Cabero-Pueyo's study of this collection, \headlesscite{CaberoPueyo:PhD}.
	}
	%
The style of this piece, with single voices imitating themselves in sequential patterns, recalls earlier Italian sacred concertos.

Though the piece bears the devotional designation \foreign{Santissmo y Nuestra Señora} \gloss{for the Eucharist and Our Lady}, the text of the estribillo has little theological content (poem~\ref{poem:Hidalgo-Cuando_el_alba}).
The \quoted{dawn} here, as in many villancicos, is an epithet for the Virgin Mary, and the \quoted{sun}, for Christ (since the rising sun is \quoted{born} out of the dawn).
Hidalgo, as expected, has the Tenor sing the word \term{sol} on the proper pitch; in fact he does it twice in a row, first in the hard hexachord with G \term{(sol, re, ut)}, and then in the soft hexachord on C \term{(sol, fa, ut)} (example~\ref{ex:Hidalgo-Cuando_el_alba_aplaude-1}).
In the poem, Christ is \foreign{uno} as Mary's firstborn and God the Father's only-begotten; Christ is \foreign{dos} because he is both divine and human; and the triune God is \foreign{tres}.
But the main point of these theological symbols, it seems, is to justify a play on numbers in the musical setting.

%%*******************
%\begin{expoem}
%\caption{\wtitle{Cuando el Alba aplaude alguno}, estribillo set by Juan Hidalgo (D-Mbs: Mus. ms. 2895)}
%\label{poem:Hidalgo-Cuando_el_alba}
%	\input{poems/Hidalgo-Cuando_el_alba}
%\end{expoem}
%%*******************

Indeed, Hidalgo's music is a rather elaborate musical game of numbers.
On one level, the numbers determine how many voices are singing and what they sing. 
The Tenor delivers the theological prompts for each number, and the other voices answer with the lines about singing, each time with the number of voices specified in the poem.
The accompaniment part primarily functions as an independent continuo line, but on the phrase \foreign{cantar a tres} it shifts to function like a \term{basso seguente}, doubling the Tenor line, and reducing the effective texture to three real voices.%
	%
	\footnote{%
While it was common in Hispanic villancicos for the bass line to shift in this way---apparently the resulting parallel octaves and unisons were not considered a violation of contrapuntal practice---in this particular case, the shift in function causes the listener to hear only three distinct voices on \foreign{cantar a tres.}
	}
	%
For \foreign{a dos a modo de uno} (\range{\measures}{22--24}, example~\ref{ex:Hidalgo-Cuando_el_alba_aplaude-1}), only two singers sing these words, while the third continues singing \foreign{cantémosle}; and these two voices are in canon at the fourth---so that in a sense the two voices are singing the music of one voice.
In the next line, \foreign{o a uno a modo de tres}, Hidalgo first has one voice sing this, then passes it through an imitative polyphonic texture for all three voices where only one voice of three ever sings these words at a time.

%%*******************
%\begin{example}
%\includegraphics[width=\linewidth]{scores-examples/Hidalgo-Cuando_el_alba_aplaude-ex1}
%\caption{%
%Hidalgo, \wtitle{Cuando el Alba aplaude alguno}, \range{\measures}{20-29}
%}
%\label{ex:Hidalgo-Cuando_el_alba_aplaude-1}
%\end{example}
%%*******************
%
Hidalgo's numbers game is most interesting at the level of rhythm, because the composer plays with the numeric possibilities of triple meter in every way conceivable.
The manuscript partbooks are distinguished by a very high proportion of black (\term{coloratio}) notation.%
	%
	\footnote{%
	See the preface for the theory and transcription of \term{coloratio} notation.
	}
	%
Coloration already suggests a numbers game, where $2:3$ is exchanged for $3:2$, but Hidalgo goes beyond conventional sesquialtera by writing colored passages that extend through as many as five compases (Tenor, \range{\measures}{2-4}; Acomp., \range{\measures}{41-43}).
The phrase \foreign{cantar a dos} in \range{\measures}{12-15} is sung from colored notation so that the two vocal lines singing these words, taken by themselves, sound like they are singing in duple meter, in a steady succession of imperfect semibreve stresses.
For \foreign{cantar a tres}, by contrast (\range{\measures}{16-19}), Hidalgo uses a short--long pattern (colored minim--semibreve) that was typical of triple meter.%
	%
	\footnote{%
	Lorente says that in CZ meter, the hand falls on the first minim and rises on the second (not the third); so this short-long rhythm is paradigmatic of the meter. \headlesscite[165--166]{Lorente:Porque}.
	}
	%
Hidalgo uses a textbook example of sesquialtera on the Tenor's phrase \foreign{a uno a modo de tres} \gloss{one in the mode of three} (\range{\measures}{24-25}, example~\ref{ex:Hidalgo-Cuando_el_alba_aplaude-1}), and this is fitting, since the pattern here divides one breve into three imperfect semibreve stresses.
Building to a climax at the end of the estribillo (example~\ref{ex:Hidalgo-Cuando_el_alba_aplaude-2}), Hidalgo creates overlapping sesquialtera groups in the voices as they imitate each other (\range{\measures}{35-37}), like a \soCalled{sesquialtera stretto}.
He follows this with a long passage of coloration (\range{\measures}{38-39}) which does not divide evenly; instead the stresses fall in irregular groups of two and three minims.
All this settles down to a perfect breve on the word \foreign{uno} in the two Tiple voices and accompaniment.

%%*******************
%\begin{example}
%\includegraphics[width=\linewidth]{scores-examples/Hidalgo-Cuando_el_alba_aplaude-ex2}
%\caption{%
%\soCalled{Sesquialtera stretto} in Hidalgo, \wtitle{Cuando el Alba aplaude alguno}, \range{\measures}{35-41}
%}
%\label{ex:Hidalgo-Cuando_el_alba_aplaude-2}
%\end{example}
%%*******************
%
There is much more musical subtlety here than this brief analysis conveys.
The work might even be seen as a demonstration of rhythmic techniques in triple meter, and a deeper study would reveal much more than could be gleaned from the theory treatises alone.
Hidalgo's piece, then, is a musical discourse on music itself.

% Padilla solfa here? \X

%**************************************
\section{%
Villancicos about Music as a Key to Theological Understanding of Music
}

Villancicos on the subject of music encapsulate theological understandings of music within musical performance.
These pieces offer a modern interpreter more than just verbal explanations of music, such as may be found in music-theory treatises or doctrinal statements; rather, they provide the opportunity to hear how early modern musicians created a true \term{musical theology}---a form of music that embodies the beliefs it proclaims.

These villancicos flourished during a time in which understandings of faith, hearing, and the power of music were rapidly changing.
The Renaissance and Reformation had brought new attention to human perception and feeling, a new concern with rediscovering the power of music over the human body and over society about which the Greeks had written so much.
Music was being employed not only as a beautifully ordered adornment, but as a means to moving the affects of listeners.
And new methods of musical rhetoric were transforming musical meaning from being understood as primarily symbolic and objective (where the meaning was inherent in the musical structures) toward a more dynamic, experiential model based on associations of figures, gestures, and stylistic topics, and dependent on communication between musicians and listeners through conventions.

The new discoveries in physiology and astronomy that were contributing to the nascent Scientific Revolution changed understandings of music as well.%
	%
	\autocite{Gouk:Sciences}
	%
The new scientific empiricism did not support the traditional theory of music based on harmonies between celestial spheres, the four elements, and the four humors of the body.
At the same time, though, Spanish Catholic poets and musicians continued to represent music according to the old system, even emphasizing the traditional cosmology more strongly.

In the Hispanic world, the contexts for hearing sacred music were changing, especially in the case of the villancico, as this courtly genre became part of church festivities, and developed from a form of intimate, aristocratic chamber music into a dramatic and expansive public genre.%
	%
	\footnote{%
	Corresponding to this development was a parallel growth of sacred chamber songs (\term{tonos divinos}).
	}
	%
In Spain, musicians toiled to keep up appearances of Habsburg splendor in the midst of continual wars, economic depression, famine, and plague.
In colonial America, the Spaniards' task shifted from evangelization and conquest toward civilization-building, in which the economic and social structures of music pedagogy, in cathedral schools and seminaries, played an important role.

Recent scholarship is increasingly demonstrating how central music was to Spain's imperial and colonial project.
Bernardo Illari, Geoffrey Baker, and David Irving have all interpreted Spanish colonial music as both reflecting and enacting hierarchical, Catholic, colonial society.%
	%
	\autocites{Illari:Polychoral}{Baker:Harmony}{Irving:Colonial}
	%
As these scholars have shown, Hispanic Catholics ritually performed their changing identities as members of the Church and subjects of the Spanish crown. 
Through sacred villancicos, these subjects also gave sounding expression to their faith, using music as a way not only to form earthly identities but also to establish connections with the divine.

Charles Seeger theorized that music and language were two distinct forms of discourse and ways of knowing.%
	%
	\autocite{Seeger:Unitary}
	%
Just as one could speak about making music (that is, create verbal discourse that referred to a musical form of discourse), one might also make music about speaking.
More intriguingly, if one could speak about speaking (which would constitute much of academic discourse), would it be possible to \emph{music} about music? 
The villancicos studied in this dissertation do just that. 
They reflect on the nature of music through the medium of music.
Musical experience can be shared across time more readily than religious experience; or to put it another way, it is easier to know through hearing how Spanish Catholics made music (from the testimony of the notated music) than it is to share in their religious experiences.
So this form of religious music offers insights into the world of historical subjects that no other form of historical document or art form can provide.

If inquirers today wish to know what early modern Christians believed, then, we must listen carefully to how they made their faith heard.
The sound of early modern sacred music---the way voices move in relationship to each other, the characteristic stylistic features of common chordal progressions, rhythmic gestures, the dramatic experience of musical forms unfolding in time---all of this provides a window for us to glimpse something of the religious experience of historical believers.
Put the other way, if we wish to understand not only what music meant to early modern people but even the details of how music worked, we must contemplate what the makers and hearers of that music believed about its sacred power. 

%************************************************************************
%************************************************************************
\section{%
Musicological Context of the Study
}

This dissertation contributes to a growing musicological conversation about music's relation to power and faith, though for the most part this has dealt with other musics and other places.
Regarding music's power, Margaret Murata has shown how \quoted{singing about singing} in certain Italian secular chamber cantatas of the mid-seventeenth century was a way for composers to question or even mock the power and truth of musical representation.%
	%
	\autocite{Murata:Singing}
	%
These composers deliberately drew attention to musical representation in order to comment ironically on the \term{seconda prattica} and conventions of operatic music.
The case studies in part~\ref{part:Singing} build on Murata's notion that \soCalled{singing about singing} highlights the artifice of music and allows for commentary on music within musical performance.

This dissertation's concern with changing understandings of senses, affects, and cosmology intersects with a large literature on these topics in other fields. 
Penelope Gouk, Gary Tomlinson, Linda Austern, Lorraine Daston and Katharine Park, Martha Feldman, and others have begun to bring these discourses---primarily from the history of science and philosophy but also from studies of magic and medicine---into historical musicology.%
	%
	\autocites{Gouk:MusicScienceMagic}{Gouk:Harmonics}{Gouk:Sciences}{Gouk:RepresentingEmotions}{Tomlinson:Magic}{Austern:Nature}{Daston:Wonders}{Feldman:Passions}
	%
As pathbreaking as this work is, much more still needs to be done to connect the theoretical discourses around music to specific, historical musical practices. 
Grayson Wagstaff has modeled how this might be done, exploring the role of senses and ritual in music for a Mexican funeral procession.%
	%
	\autocite{Wagstaff:Processions}	
	%
Any attempt to hear with \quoted{period ears,} limited as that enterprise must be, will have to be based on both how people made music in a particular historical moment and cultural orbit, and on how people, as far as can be determined, heard that music.%
	%
	\autocite{Burstyn:PeriodEar}
	%
For example, in an important colloquy on historical listening practice in \wtitle{Early Music}, Jeffrey Dean argued that the audience for much music before the eighteenth century should be considered to include the performers, and in many cases, the performers were the sole \quoted{listeners} to church music in particular.%
	%
	\autocite{Dean:ListeningPolyphony}
	%
Elisabeth Le Guin demonstrates how notated music may be read as a record of bodily experience, and Melanie Lowe and Richard Cullen Rath model different ways that we might begin to reconstruct historical hearing of music.
	%
	\autocites{LeGuin:BoccheriniBody}{Lowe:PleasureSymphony}{Rath:EarlyAmerica}
	%

The relationship between music and faith has been central in recent research into Lutheran music in the early modern period.
Researchers in that field have an advantage over scholars of Catholic music in that the Lutherans produced more writing about music in the form of vernacular hymns, hymnal prefaces, and polemical texts.
The work of Christian Bunners and Joyce Irwin on Lutheran theology of music is notable for drawing on some of these sources, but does not make clear enough connections between the polemics and particular musical repertoires.%
	%
	\autocites{Bunners:Kirchenmusik}{Bunners:Singende}{Irwin:VoiceHeart}
	%
Gregory Johnston, David Yearsley, and Eric Chafe have interpreted the music of Schütz, Buxtehude, and J.~S.~Bach, respectively, in the context of Lutheran theology and piety.%
	%
	\autocites{Johnston:Rhetorical}{Yearsley:Buxtehude}{Yearsley:BachThron}{Chafe:Tonal}
	%
Mary Frandsen's current work is furthering this effort of theological interpretation in historical context.%
	%
	\footnote{%
	\autocite{Frandsen:Crossing}, and a monograph in progress on Christocentric devotion through music in seventeenth-century Lutheran piety.
	}
	%

Some scholars of Roman Catholic music, in contrast, have not taken early modern sacred music seriously as a source for theological understanding, or have simply not been interested in theology. 
Lorenzo Bianconi, for example, accepts the now questioned narratives of confessionalization and secularization, and portrays Catholic theology and liturgy in the period as rigid, conformist, and unchanging.%
	%
	\autocite{Bianconi:17C}
	%
Bianconi presents Monteverdi as covertly bringing the \quoted{secular} styles of opera into the church with little concern for theology or piety, when we might just as well view the sacred and secular production of Monteverdi and his contemporaries as integrated parts of a whole.

These two narratives of confessionalization and secularization have especially plagued scholarship on villancicos: scholars who have investigated the music at all tend to either consider villancicos as an incursion of \quoted{popular,} \quoted{secular} music into the liturgy, or if they do consider the pieces' theology, they tend to see them as reiterations of preprogrammed Tridentine dogma.
Scholars with this latter perspective have tended to see Catholic music and arts in terms of twentieth-century propaganda or mass marketing.%
	%
	\footnote{%
	An especially embittered example is \autocite{Menache:Vox}.
	}
	%

Patrick Rietbergen and Jack Sage have criticized the \quoted{propaganda} approach as anachronistic, and have instead sought to avoid reductionism and take historical insider beliefs seriously.%
	%
	\autocites{Rietbergen:Power}{Sage:Instrumentum}
	%
Historical musicologists are beginning to catch up to the work in this vein by ethnomusicologists like Glenn Hinson, Jeff Todd Titon, and Monique Ingalls, that has sought to take seriously the beliefs of religious insiders regarding music's supernatural powers.%
	%
	\footnote{%
	\autocites{Hinson:Fire}{Titon:Powerhouse}{Ingalls:Awesome}.
	For perspectives from religious studies, see also the essays in \autocite{McCutcheon:InsiderOutsider}.
	}
	%
Gregory Barnett has argued that a form previously written off as \quoted{secular music in church}---the \term{sonata da chiesa}---was perceived by early modern Catholic worshippers as sacred on the basis of musical topics indexing other types of liturgical music such as choral Kyries and organ fugues.%
	%
	\autocite{Barnett:Bolognese}
	%

The old narratives of secularization and confessionalization are slowly being overturned, particularly because of increasing studies of Catholicism outside Europe.%
	%
	\footnote{%
	For example, \autocites{Bailey:Art}{Dean:Inka}{Ditchfield:Dancing}.
	}
	%
Seventeenth-century Catholicism is increasingly being presented as a much more colorful, diverse, and dynamic entity than previously thought.
Robert Kendrick's study of music for Holy Week in the early modern world demonstrates a comprehensive approach to music as a social activity with economic and political aspects as well as a religious expression with multivalent meanings to hearers of different stations.
	%
	\autocite{Kendrick:Jeremiah}
	%
Post-Tridentine Catholicism, while not the monolithic, quasi-totalitarian entity some once thought it was, nevertheless cannot be understood without considering the tensions between \quoted{top} and \quoted{bottom} strata of the Church, and positions in between.

%*******************
\subsection{%
Bringing Hispanic Music into the Conversation
}

Despite the value of all this scholarly work on questions of music, power, and faith, these discourses have not generally considered Spanish music. 
Music scholars continue to leave Spain entirely out of the grand narrative of early modern music, not to mention the Americas.
Unfortunately, on the other end, scholars Spanish music have not generally considered these larger questions.
Aside from what the many metamusical villancicos can teach us about early modern understandings of music, the villancico repertoire deserves further study in its own right. 
Numerically speaking it may well be the largest category of vernacular sacred music in the early modern world, and one of the first of any music to have a truly global spread---and yet it has received hardly any attention outside of Hispanic music studies.

Literary scholars and cultural historians have certainly contributed to our understanding of early modern worldviews. 
Francisco Rico documents the Neoplatonic notion of man as microcosm in Golden Age Spanish literature, and Frederick de Armas has shown how astrology was linked to notions of political power in the plays of Calderón, for example.%
	%
	\autocites{Rico:PequenoMundo}{DeArmas:Astraea}
	%
And anthropologists like William Christian and historians of religion like Gillian Ahlgren (among many others) have shed light on the devotional piety of Spanish Catholics in this era, in the lives of both common people and uncommon ones like Teresa of Ávila and other visionaries.%
	%
	\autocites{Christian:LocalReligion}{Christian:PersonAndGod}{Ahlgren:TeresaPolitics}
	%

But most scholars of Spanish literature and culture have not seriously considered the villancico, despite its being one of the most common forms of religious devotion and of sacred lyric poetry disseminated throughout the Hispanic world.
The one exception is the singular attention lavished on the prolific villancico poet Sor Juana Inés de la Cruz, but even music scholars such as Geoffrey Baker have written about Sor Juana's villancico texts without discussing their extant musical settings.%
	%
	\autocites{Tenorio:SorJuana}{Tenorio:Gongorismo}{Baker:EthnicVC}
	%

Miguel Querol Gavaldá and José María Díez Borque have written about the meaning and function of music in the plays of Calderón, with the latter scholar raising important questions about the public's understanding of and involvement with these plays. 
These literary scholars, though, do not connect Calderón's words about music with the actual music that survives for these plays.%
	%
	\autocites{Querol:Calderon}{DiezBorque:Publico}
	%
No one has yet done for music in the \term{auto sacramental} what Louise Stein has done for the Calderonian \term{comedia}.%
	%
	\autocite{Stein:Songs}
	%Insert Larissa brewer cite in this paragraph? \X
	%
These cultural forms demand interdisciplinary perspectives, and though individual scholars may attempt to integrate multiple approaches in a single project such as this one, the best way forward will be through real dialogue and collaboration across disciplines.

Part of the reason for this neglect of villancicos by musicological and literary scholars is that the relatively small amount of musicological research on villancicos has not generally broached themes of wider interest.
Important research by Samuel Rubio, Paul Laird, Bernat Cabero Pueyo, and others focused primarily on tracing the structural evolution of the villancico as an abstract form from the fifteenth century through the seventeenth.%
	%
	\autocites{Rubio:Forma}{Laird:VC}{CaberoPueyo:PhD}
	%
Other scholars have attempted taxonomies of villancico types, or studied particular subgenres of villancico such as the \quoted{ethnic villancico.}
Álvaro Torrente and others pushed research in a more contextual direction by investigating the liturgical function of villancicos in particular places.%
	%
	\footnote{%
	\autocite{Torrente:PhD} and the essays by Bégue, Bombi, Cabral, Hathaway, and Knighton in \autocite{Knighton-Torrente:VCs}.
	}
	%
Dianne Goldmann situates the villancico's sister genre, the Latin Responsory, in its ritual context in Mexico City Cathedral.
	%
	\autocite{Goldman:Responsory}
	%

But there are still few studies that interpret villancico poetry and music and also connect it with broader discourses.
Laird is to be commended as the first to suggest how such interpretation might proceed.
Bernardo Illari's thorough study of villancicos in eighteenth-century La Plata (Sucre, Bolivia) admirably combines both detailed local context and interpretive analysis of pieces of music.%
	%
	\autocites{Illari:Polychoral}{Illari:Popular}
	%
Illari was the first to identify the category of metamusical villancicos, which he labels \quoted{singing about singing,} but because this is not his primary question, he does not pursue the implications of these pieces far beyond that.
Illari was also one of the first to consider the theological dimension of villancicos, in distinction to other work that has focused more on political, social, or racial elements.
Like this other scholarship, though, Illari's ultimate focus is less on theological interpretation than on how the La Plata villancicos reflect and ritually enact structures of secular power.
Illari's thesis of \quoted{polychoral culture}---that music was a way of creating a harmoniously balanced, hierarchical society, a medium for ritually enforcing the \term{ancien régime}---is echoed in other recent work on Hispanic music by Baker (on Cuzco) and Irving (on Manila, who calls a similar idea \quoted{colonial counterpoint}).%
	%
	\autocites{Baker:Harmony}{Irving:Colonial}
	%
Ricardo Miranda has begun to move in this direction by contextualizing Juan Gutiérrez de Padilla's Latin-texted music with historical theology, though a deeper engagement with primary theological sources and musical manuscripts is needed.%
	%
	\autocite{Miranda:PadillaLuz}
	%
Miranda connects Padilla's Latin music for the cathedral of Puebla with seventeenth-century theological notions of light. 
Extending the study to Padilla's many villancicos based on tropes of light (such as pieces referring to the Virgin as the dawn and Christ as the sun) would yield even more fruitful results.

There is ample need, then, for an interpretive project that considers villancicos as sources of musical theology, focusing close analysis on music and poetry, and grounding interpretation in the intellectual context of specific times and places, but also with an eye to common global trends.
We will begin by focusing on the theological problem at the center of this inquiry: what kind of power did early modern Catholics believe music exerted in the relationship between faith and hearing?

\section{%
Listening for Unhearable Music:
The Power of Music in the Neoplatonic Tradition
}

The villancicos on sensation and faith elevate hearing above the other senses, and use listening to music as the paradigm of faithful listening.
But these pieces, like Calderón's \wtitle{Nuevo palacio}, also cast doubt on the ability of any sense to perceive spiritual matters unless tempered by faith.
These pieces, then, model a practice of listening to music in which the immediate object of hearing is not the primary goal of perception.
By explicitly drawing attention to the imperfections of music and its listeners, these villancicos challenge hearers to go beyond mere sound and listen for a higher, unhearable music of faith.

This gesture toward a higher, more perfect form of music is rooted in Christian Neoplatonism.
Villancicos on the subject of music, particularly those that will be discussed in part~\ref{part:Singing}, consistently manifest a Neoplatonic theological worldview.
The treatises used to teach musical composition in seventeenth-century Spain, most notably Pedro Cerone's \wtitle{El melopeo y maestro} (1613) and Andrés Lorente's \wtitle{El porqué de la música} (1672), present music within a Boethian cosmology of music, which has its roots in a Neoplatonic-Augustinian tradition.
Augustine was by far the most influential theologian for early modern Catholics, not only in Spain: his works were directly available in printed editions starting early in the sixteenth century and reissued and re-edited many times after, and through many compendia and digests of patristic theology; and his ideas infused every genre of theological writing.%
	%
	\footnote{%
	Chapter~\ref{ch:Padilla-Voces} includes a more detailed discussion of the kinds of theological literature that were most influential for poets and composers of villancicos.
	}
	%

One of the foremost proponents of Christian Neoplatonism in the Augustinian tradition was the Dominican Fray Luis de Granada, especially in his \wtitle{Introducción del Símbolo de la Fe} \gloss{Introduction to the Creed} of 1589.
Fray Luis's writings were widely read across the Hispanic world through the eighteenth century. 
Because his work is a self-acknowledged synthesis of patristic and Classical sources, his writings may be taken as both representative of widely held beliefs of his own time and after, as well as a guide to how earlier sources were read and understood by early modern Catholics.

	%
	%\footnote{%
	%William Christian recounts how a rural family in northern Spain in the 1970s showed him their copy of a book by Fray Luis de Granada which had been handed down for more than two centuries. cite \X
	%}
Fray Luis's introduction to the first article of the Apostle's Creed, \quoted{I believe in God, the Father almighty, Creator of heaven and earth}, is really a fulsome exposition of a theology of the created world.
In the Neoplatonic tradition, Fray Luis teaches that the natural world is a reflection of a higher truth---God's own nature---and that the creation was given so that by reflecting on it people would come to know its Creator.
Fray Luis frequently uses musical metaphors to describe the harmonious workings of the created world, and he includes a discussion of the physiology and theology of the human voice that applies directly to a historical understanding of music.

A second key source for Neoplatonic theology, in this case specific to music, is the encyclopedic \wtitle{Musurgia universalis} by another great synthesist of received wisdom, Athanasius Kircher.
% cite \X
The Jesuit polymath's 1650 work was disseminated through Jesuit networks across the globe: a copy was sent as far as Manila, and two copies are preserved today in Puebla.
Kircher describes in detail the latest scientific knowledge about the anatomy of hearing and vocal production and the physiology of bodily humors and affects; and lays out specific examples of how particular musical structures work through these bodily systems.
Kircher presents a cosmic view of music according to Neoplatonic traditions of theology and music theory, in which the whole universe is encompassed in the \quoted{working of music}---a rough translation of his inventive Greek-and-Latin title.%
%	\X secondary Kircher lit

The writings of Fray Luis de Granada and Athanasius Kircher provide the basis for a provisional historical theology of music within the Neoplatonic tradition. 
The fundamental concepts of this theology of music are the Neoplatonic chain of being and the Boethian three-fold division of music.
In brief, Christian Neoplatonists followed Augustine in viewing the material world as a reflection of a higher spiritual reality which ultimately had its source in the Supreme Good which was the Godhead.%
	%
	\footnote{%
	An important later source for this concept is the \wtitle{Spiritual Hierarchy} attributed to Dionysius the Areopagite. %\X
	}
	%
The material world reflected higher truths only imperfectly, but nevertheless this world was also the only means through which those truths could be reached.
In connection with Catholic sacramental theology, material objects and physical actions became means through which humans could encounter divine grace.
Neoplatonic contemplation could be understood as a dialectical process of discerning the degree both of similarity and of dissimilarity between earthly objects and heavenly truth.

Augustine's writings on music in the \emph{Confessions} (10:23) are sometimes interpreted today in an excessively negative way, as though Augustine only accepted music's value when it could serve as a neutral medium for sacred words.
When Augustine chastizes himself for being seduced by the song, rather than \quoted{that which is sung}, he is not condemning music or the voice as un-sacred.
Instead it is probably more accurate to interpret this statement as a confession---like those throughout the book---that he failed at the task of distinguished the created thing from its Creator.
He was captivated by the song as an object, that is, as an idol, rather than by the \quoted{holy thoughts} or \quoted{sentiments} that the song was meant to communicate to him and move him towards. 
He failed at the task of Neoplatonic-Christian listening by getting stuck at the lowest level of sensory experience, by letting carnal pleasure lead his mind rather than the reverse.

In short, he failed to rise from what he was hearing to contemplate the higher truths to which the singing was meant to move him.
He is not questioning music's Platonic value as a reflection of the divine.
Augustine values music's power over the affections, and acknowledges that it was his own weakness that made this a problem.  

While Augustine as a Neoplatonist does emphasize ideas over material forms, he certainly does not negate or reject the material world.
This would have been Manichaeism, which Augustine left behind for Christianity and spent much of his career refuting.
Christian Neoplatonists after Augustine, then, would see material forms as imperfect reflections of higher realities, but as necessary ones, for a person could only rise to contemplate higher things through the lower things.
Music reflects the structure of the universe and thereby points to God; this is why it is one of the higher liberal arts, leading on to philosophy and theology.
Music for Neoplatonists was not a neutral medium, but was sacred in and of itself because it embodied number and therefore truth.

The definition of music in Boethius's \wtitle{De musica} provided the classic formulation of how music fit into the Neoplatonic chain of being.%
The three Boethian types of music are arranged hierarchically and each one points beyond itself to a higher level.
At the lowest level is \term{musica instrumentalis}---music played and sounded, music that humans can hear.
Higher up is \term{musica humana}---the harmony of body and soul, and of one human being with another in society.
Still higher is \term{musica mundana}---the harmonies created by the perpetual movement of the planetary spheres.

Villancicos on the subject of music embody the notion that even these three levels of music are subordinate to the supernatural forms of music in Heaven---the chorus of angels and saints, and above them, the mysterious harmonies of three in one in the Trinity and two in one in the divine-and-human nature of Christ.
The three Boethian musics in this system would all be \soCalled{worldly} music, and it is important to clarify the distinction between the music of the \foreign{cielos} or heavens---that is, the planetary spheres---and the \soCalled{heavenly} music of the \foreign{cielo Empyreo} or Heaven, the supernatural realm beyond the material world.
Table~\ref{table:Neoplatonic-hierarchy-music} presents a synthesis of the Neoplatonic hierarchy of  music.

%%*******************
%\begin{table}
%	\caption{Neoplatonic hierarchy of music}
%	\label{table:Neoplatonic-hierarchy-music}
%	% Neoplatonic hierarchy of music
% table:neoplatonic-hierarchy-music

\begin{tabular}{lll}
\toprule
               & Harmony of Trinity           & \\
Heavenly music & Chorus of saints, angels     & \\
\midrule
               & \term{Musica mundana}        & Spheres\\
               & \term{Musica humana}         & Bodies\\
Worldly music  & \term{Musica instrumentalis} & Sounding music \\
\bottomrule
\end{tabular}
%\end{table}
%%*******************


\term{Musica instrumentalis}, then, though the lowest form of music in the chain of being, was the only form of music to which humans had direct access through the sense of hearing.
Metamusical villancicos explicitly emphasize the challenge that was central to all music-making in the Christian Neoplatonic tradition, to use the imperfect medium of sounding music to evoke all the higher forms of music, to lead listeners in contemplation up the chain of being beyond simply what was heard.

Vocal music played a special role in this system because for Neoplatonists, the human body was the microcosm of the whole created world. 
The voice, then, is the physical expression of the microcosm, and vocal music thus doubly reflects the order of nature: in its musical ratios and proportions, which reflect those of the spheres, and as an expression of the human body as the microcosm.
\emph{Musica instrumentalis} is the finite expression of \emph{musica humana} and reflects and leads to the contemplation of the \emph{musica mundana}, and to the higher Music of the Triune God who created all these lower forms of music.

%******************** FRAY LUIS ********************
\subsection{%
Hearing the Book of Nature Read Aloud
}

This Neoplatonic worldview was disseminated through the post-Tridentine Hispanic world through books like Fray Luis de Granada's \wtitle{Introduction to the Creed}.
Fray Luis begins his exposition of the Creed in the traditional manner of catechists (as modeled by the Roman Catechism), by using the first article to teach the theology of creation, through which, according to St.~Paul and Thomas Aquinas people can come to the natural knowledge of God.

\quoted{The ultimate and highest good of man}, Fray Luis states at the outset, \quoted{consists in the exercise and use of the most excellent work of man, which is the knowledge and contemplation of God}.%
	%
	\footnote{%
	\Autocite[182]{LuisdeGranada:Simbolo}: \quoted{El último y summo bien del hombre consistia en el ejercicio y uso de la mas excelent obra del hombre, qu es el conoscimiento y contemplación de Dios}.
	}
Fray Luis teaches that the created world is a \quoted{book of nature} in which is written the grandeur, love, wisdom, and faithfulness of its Creator.
The first goal of humankind, then, is to learn to read this \quoted{book of nature} in order to come through it to the knowledge of God. 
The goal of contemplating creation is \quoted{ascending by the staircase of the creatures to the contemplation of the wisdom and beauty of the Maker}.%
	%
	\footnote{%
	\Autocite[184]{LuisdeGranada:Simbolo}: \quoted{subiendo por la escalera de las criaturas á la contemplación de la sabiduría y hermosura del Hacedor}.
	}
	%

The reason one can \quoted{read} God through nature, Fray Luis teaches, is that the created world is a reflection of God's perfect order---a concept the friar repeatedly expresses using musical metaphors.
Fray Luis compares the perfect order of nature to a harmonious musical composition in which everything fits together \foreign{con sumo concierto} \gloss{with the most perfect concord}.%
	%
	\footnote{%
	Or harmony, agreement; the same word is used for a musical \quoted{concerto} or \quoted{concerted} music.
	This seems to be Fray Luis's Castilian equivalent for the Latin \emph{concordia}, the word most frequently used in this context by Augustine (as in the \emph{City of God}).
	}
	%
All the created things in this world, Fray Luis writes, \quoted{like concerted music for diverse voices, harmonize together [concuerdan] in the service of man, for whom they were created}.%	
	%
	\footnote{%
	\Autocite[191]{LuisdeGranada:Simbolo}: \quoted{Mas entre todas ellas es mucho para considerar, de la manera que todas (como una música concertada de diversas voces) concuerdan en el servicio del hombre, para quien fuéron criadas \Dots}.
	}
	%
The movement of the heavenly spheres, and their effects on the earth, are like a great \quoted{chain, or, it can be said, this dance, so well ordered, of the creatures, and like music for diverse voices \Dots.
Because things so diverse could not be reduced to a single end with a single order, if there were not one who was like a chapelmaster [maestro de capilla], who reduces them to this unity and consonance}.%
	%
	\footnote{%
	\Autocite[191]{LuisdeGranada:Simbolo}: \quoted{Asimismo los otros planetas y estrellas, segun los diversos aspectos que tienen entre sí y con el sol, son causa de diversos efectos acá en la tierra, como son lluvias, serenidad, vientos, frio, y calor y cosas semejantes. Esta cadena, ó, si se puede decir, esta danza tan ordenada de las criaturas, y como música de diversas voces, convenció á Averrois para creer que no habia mas que un solo Dios.
	Porque no se pueden reducir á un fin con una órden cosas tan diversas, si no hubiere uno que sea como maestro de capilla, que las reduzga á esta unidad y consonancia}.
	}
	%

In the Neoplatonic tradition, these references to music are more than just metaphors. 
The universe is not only like music, it acutally is in some sense musical.%
	%
	\footnote{%
	It should be said, though, that the tradition is not always clear on whether the music of the spheres is actual music that someone could hear or is only \quoted{music} in the sense of movement in perfect proportions.
	}
	%
While some might think of Neoplatonists as ignoring actual sounding music for the sake of abstracted \quoted{higher music}, it is not possible to compare something to music without having some kind of earhly music in mind.
When Fray Luis compares the world to music \quoted{in diverse voices} he obviously has in his \quoted{mind's ear} polyphonic music of his own time, such as he would have heard at the Portuguese Royal Chapel as confessor to the queen.
Likewise, when he compares God to a \foreign{maestro de capilla}, that has all the implications of that office in the Hispanic context, which included composition, teaching, and leading the choir in some form of conducting.%
	%
	\footnote{%
	The trope of Christ as a chapelmaster is discussed in chapter~\ref{ch:Padilla-Voces}.
	}
	%
Thus God for Fray Luis is creator, prime mover, and sovereign ruler over creation, actively and intimately involved in its ongoing progress.

For Fray Luis, not only does creation reflect God's order; it actively proclaims that fact.
It speaks or sings with its own voice to communicate God's glory to the human who knows how to listen.
Fray Luis glosses Augustine's commentary on Psalm 26 to say, \quoted{Look around at all these many things from the heaven to the earth, and you will see that they all sing and preach their Creator; because all types of creatures are voices [or perhaps, utterances] that sing his praises}.%
	%
	\footnote{%
	\Autocite[185]{LuisdeGranada:Simbolo}: \quoted{Rodea cuantas cosas hay dende el cielo hasta la tierra, y verás que todas canta y predican á su criador; proque todas las especies de las criaturas voces son que cantan sus alabanzas}.
	}
	%
While the full knowledge of God can only come with the aid of divine revelation through the Scriptures and the Church, Fray Luis praises God that humans can study his nature in \quoted{the university of created things, which declare to us [literally, \quoted{give us voices}] that you love us, and teach us why we should love you}.%
	%
	\footnote{%
	\Autocite[186]{LuisdeGranada:Simbolo}: \quoted{Ayúdanos tambien la universidad de las criaturas, las cuales nos dan voces que os amemos, y nos enseñan por que os habemos de amar}.
	}
	%

Fray Luis acknowledges, however, that apart from angels and birds, most of creation is mute and does not literally have its own voice with which to communicate its message of divine glory.
This \quoted{message} is not a linguistic one, but rather, their message is simply themselves: in the created world, the medium is the message.
\quoted{Now these admirable works do not speak or testify this with human voices \Dots}, Fray Luis writes, \quoted{rather their speech and testimony is their invariable order and their beauty, and the artifice with which they are so perfectly made, as though they were made with a ruler and plumb line}.%
	%
	\footnote{%
	\Autocite[192]{LuisdeGranada:Simbolo}: \quoted{Mas estas obras admirables no hablan ni testifican esto con voces humanas (las cuales no pudieran llegar al cabo del mundo); mas su habla y testimonio es la órden invariable, y la hermosura dellas, y el artificio con que están hechas tan perfectamente, como si se hicieran con regla y plomada}.
	}
	%

In this theological system, music has unique value because it actually provides a voice through which creation can make audible its message-of-being.
As Margit Frenk has documented, books in this period were not read silently, but required someone to give them voice, and were written with that intention.%
	%
	\footnote{%
	\autocite{Frenk:Voz}.
One edition of Fray Luis's own book \wtitle{Doctrina Cristiana} bore at the beginning a notice from the Archibishop of Toledo granting a certain number of days of indulgence for each paragraph that anyone \quoted{read or heard read} (that is, had read to them).
	}
	%

To read the \quoted{book of nature}, therefore, someone must perform it vocally---and this is what music could do.
In the Christian Neoplatonic tradition, human music unlocks the musical voice contained within the substance of created things.
Through metal pipes, horns, and bells; through wood viol cases, gut strings, and skin drums; even through reverberant stone church walls, the very matter of creation is made to resound with the perfectly ordered mathematical-harmonic proportions placed within it by the Creator---proportions which themselves reflect God's own perfect order.%
	%
	\footnote{%
	This idea recalls \bibleverse{Lk}(19:40): \quoted{If these [Christ's disciples] were silent, the stones would shout out} (dico vobis quia si hii tacuerint lapides clamabunt).
	}
	%

If pipes and strings testify to the order of creation, then the human body as the microcosm of creation is the ultimate instrument through which nature is given voice.
Fray Luis concludes his exposition of the six days of creation (based largely on the \emph{Hexameron} of St.~Basil) by saying that God's creation of man on the sixth day was like the conclusion of an oration, when the speaker draws together all his themes into a final epitome.
Thus man is the summation of all that God had created in the previous five days and encompasses them all within himself.%
	%
	\footnote{%
	\Autocite[243]{LuisdeGranada:Simbolo}
	}
	%

%*******************
\subsection{%
Voice as Expression of Man, the Microcosm
}

When Athanasius Kircher (in the tenth book of the \emph{Musurgia}) continues this hexameral tradition with his own treatment of the six days of creation, he replaces the rhetorical metaphor with a musical one.
Instead of creation being God's oration, Kircher presents it as a musical improvisation (a \quoted{Praeludium}) on God's cosmic organ.%
	%
	\footnote{%
	\Autocite[Vol. 2, 366--367]{Kircher:Musurgia}.
	Kircher's illustration of this is used as an interpretive key in chapter~\ref{ch:Cererols}.
	}
	%
On the sixth day, Kircher says, God recapitulates all his themes and pulls out all the stops by creating man.
Here again, the comparison to music must be based on some actual music known to Kircher; his description closely resembles the structure of a Praeludium by the likes of Dieterich Buxtehude, which develops a motivic kernel through various sections and culminates in a fugue for the full organ.
In Kircher's worldview, all the systems and elements of creation (stars, planets, humors, rocks, animals, and so on) intersect in the individual human body.%
	%
	\footnote{%
	See the illustration in \autocite[vol. 2, 402]{Kircher:Musurgia}.
	}
	%

For Kircher, the human voice is the unique expression of the individual, reflecting each person's unique temperament and blend of the four humors.%
	%
	\footnote{%
	In \headlesscite[vol. 1, 23--24]{Kircher:Musurgia}, among other places, Kircher discusses why different voices have unique qualities.
	}
	%
Kircher defines the voice thus: \quoted{The voice is a living sound [or, sound of the soul], produced by the glottis through the percussion of respired breaths that serve to express the affects of the soul}.%
	%
	\footnote{%
	\Autocite[vol. 1, 20]{Kircher:Musurgia}: \quoted{Vox est sonus animalis à glottide ex percussione respirati aeris ad affectus animi explicandos productus}.
	}
	%
Since each voice is unique, only in concert do voices fully reflect nature and nature's God.
Cantus, Altus, Tenor, and Bass parts provide a place for all types of human voices, Kircher explains, and correspond respectively to fire, air, water, and earth.
Thus they form a choral microcosm both of humanity and of all creation.%
	%
	\Autocite[vol. 1, 217--219]{Kircher:Musurgia}
	%

Fray Luis says that most of creation does not proclaim its message \quoted{with human voices}, but he also presents the human being as the \quoted{mundo menor} or microcosm;%
	%
	\footnote{%
	\Autocite[243]{LuisdeGranada:Simbolo}: El \quoted{mundo menor, que es el hombre}.
	}
	%
  he exalts the voice as the audible expression of the human body and vocal music as the most perfect kind of music. 
If man is the microcosm and human voices in concert are even more so, than polyphonic choral music could actually give voice to the spheres and all below them.
Fray Luis praises the human voice as the highest of all musical instruments (indeed, as the paradigm for them), as a means of forming social relationships between people, and as a form of communication between human and divine:

%********************
\begin{quote}
%
The lungs also serve to create the voice, because, when the air that they exhale leaves them with a great impetus, and touches the voicebox or \quoted{little bell} that we have at the entrance of the lungs, the voice is formed. \Dots\
But here it is to be noted that the mouth of the pipe coming out of the lungs is neither large nor round, but is drawn tight [hendida] just like the slot of an alms box. 
Which opening serves to form the voice; this is why the mouths of flutes and dulcians are constructed in this fashion, because in this manner, the compressed air entering through them, the voice is caused.%
	%
	\footnote{%
	\Autocite[252]{LuisdeGranada:Simbolo}: \quoted{Sirve también el pulmon para la voz, porque saliendo el aire que él despide de sí con algun ímpteu, y tocando en el calillo ó campanilla que tenemos á la entrada dél, se forma la voz. \Dots\
	Mas aquí es de notar que la boca de la caña deste pulmon, ni es grande ni redonda, ántes es hendida, así como la abertura de una alcancía. 
	Lo cual sirve para formar la voz; porque deste modo están fabricadas las bocas de las flautas y dulzainas; porque desta manera entrando por ellas el aire colado se causa la voz.}
	}
	%
%
\end{quote}
%********************

We may note again that the friar's reference to music corresponds to contemporary practice: flutes and dulcians were played in Iberian church music, both as independent instruments and in the form of organ pipes of those names (and we may also observe that as in Kircher, the organ seems to represent the highest of musical instruments and the closest analogue to the voice).
Fray Luis praises the flexibility of the voice, which unlike a wooden flute can take on any shape needed, and which is unique because each person's body is unique.
The voice therefore expresses human individuality, and voices of different types in concert enact harmony between people:

%********************
\begin{quotation}
%
Moreover, here is a thing worthy of much consideration, to see the omnipotence and wisdom of the Creator, who was able to form something like a flute from flesh, which serves for singing.
For to make a flute or trumpet from a solid material such as wood or some metal, is not much; for the hardness of the material serves for the resonance of the voice.
But to  make this out of flesh (such as is the windpipe of the lungs), and such that through it some voices are formed of women and of men, so sweet that they seem more like those of angels than of humans, and these with such variety of notes [punctos], without having the finger holes of flutes that provide this variety, this is something that declares the power and the wisdom of that sovereign artisan, who in such a manner forged the flesh of this windpipe so that in it could be formed a voice sweeter and milder than that of all the flutes and instruments that human industry has invented.

And there is no end of admiration for the variety that there is in this for the service of harmonious music [música acordada]. 
For some throats are narrow, in which are formed the trebles [tiples], and others in which are formed voices so full and resonant that they seem to thunder through an entire church, without which there could not be perfect music.

All of which that divine presider [presidente] traced and ordained, so that with this mildness and melody the divine offices and their praises should be celebrated, with which to awaken the devotion of the faithful.%
	%
	\footnote{%
	\Autocite[252]{LuisdeGranada:Simbolo}: \quoted{Mas aquí es cosa digna de mucha consideración, ver la omnipotencia y sabiduría del Criador, que pudo formar una como flauta de carne, la cual sirve para cantar.
	Porque hacer una flauta ó trompeta de materia sólida, como es de madera ó de algun metal, no es mucho; porque la dureza de la materia sirve para la resonancia de la voz.
	Mas hacer esto de carne (cual es la caña del pulmon), y que en ella se formen algunas voces de mujeres y de hombres, tan suaves, que mas parecen de ángeles que de hombres, y estas con tanta variedad de punctos, sin tener los agujeros de las flautas que sirven para esta variedad, esto es cosa que declara el poder y la sabiduría de aquel artifice soberano, que de tal manera fraguó la carne desta caña que se pudiese en ella formar una voz mas dulce y mas suave que la de todas las flautas y instrumentos que la industria humana ha inventado.
	Y aun no carece de admiracion la variedad que en esto hay para servicio de la música acordada. 
	Porque unas cañales hay delgadas, en las cuales se forman los tiples, y otras en que se forman voces tan llenas y tan resonantes, que parecen atronar toda una iglesia, sin las cuales no podia haber música perfecta.
	Lo cual todo trazó y ordenó así aquel divino presidente, para que con esta suavidad y melodía se celebrasen los divinos oficios y sus alabanzas, con que se despertare la devoción de los fieles}.
	}
	%
%
\end{quotation}
%********************

Fray Luis wants his readers to hear God's glory reflected most fully in the concerted harmony of diverse human voices, which he says were created for the purpose of singing of singing in divine worship.
The voice in church is the definitive example of vocal music for Fray Luis.
Sacred polyphony glorifies God, then, simply by realizing the potential for which the voice (and the body) was made.
In the above passage, the sound of the voice alone proclaims God's power and wisdom just in itself, apart from whatever words or musical figures it might articulate.

Fray Luis sees speech as something \quoted{added} to the voice, which makes it possible for the voice to communicate and form social relationships:

%********************
\begin{quote}
%
Now here it is to be noted that when to the voice which proceeds from this place is added the instrument of the tongue, we come to articulate and make distinctions with this voice, and thus is formed speech, serving us by this instrument and punctuating [hiriendo] with it sometimes in the teeth and other times in the interior of the mouth.
And just as the flute produces different sounds by touching on different holes, likewise the tongue, touching in different parts of the mouth, forms different words.
By this manner the Creator gave us the faculty to speak and communicate our thoughts and concepts to other men.
	%
	\footnote{%
	\Autocite[252]{LuisdeGranada:Simbolo}: \quoted{Mas aquí es de notar que cuando á la voz, que por aquí sale, se añade el instrumento de la lengua, venimas á articular y distinguir esa voz, y así se forma la habla, sirviéndonos deste instrumento, y hiriendo con él unas veces en los dientes y otras en lo interior de nuestra boca. \Dots\
	Y así como la flauta hace diversos sonidos tocando en diversos agujeros, así la lengua, tocando en diversas partes de nuestra boca, forma diversas palabras.
	Desta manera nos dió el Criador facultad para hablar y comunicar nuestros pensamientos y conceptos á otros hombres.}
	}
	%
%
\end{quote}
%********************

Fray Luis might see music---with its own system of articulations and distinctions---as another way to \quoted{communicate our thoughts and concepts} just as well as spoken language, but he also presents music as a product of the voice before any articulation is added.
This definition of voice would mean that in vocal music there are always two layers---the articulated \quoted{speech} aspects, and behind these the wordless sustained voice.
In a polyphonic vocal piece like a villancico, the bulk of musical structure is borne by the sustained tones of the voice, singing vowels.
Apart from the words being sung, musical elements like mode, meter, motivic development, and stylistic or topical allusions are all communicated by these musical voices, and not simply by the voice as the bearer of words.
Music could thus reflect the divine through its sonic structure, apart from any sacred linguistic meaning that may be attached as well.

This would mean that the hearer of music could and should seek out this level of musical structure while listening.
If music's value and sacredness are not comprised solely in the words being sung, then one must know how to hear the musical structure in order to receive the full benefit.
Citing Augustine's \wtitle{De doctrina christiana} (the classic exposition of Christian preaching and teaching), Fray Luis---who was himself the author of six volumes about \wtitle{Rhetorica ecclesiastica}---says that the main task of the student of rhetoric is to hear and identify the rhetorical tropes and techniques used by another orator.
In the same way, he says, the first task of humankind is to be a student of the natural world, and to learn to recognize in creation the signs of God's artifice as the Creator, which manifest his glory.


%*******************
\section{%
Three Theological Functions of Villancicos
}

The basic Neoplatonic logic traced above may be summarized as follows: 

\begin{enumerate}
\item Music is a reflection of the natural order.
\item The natural order is itself a reflection of God.
\item By paying attention to nature one can come to know and believe in its Maker.
\item Therefore listening to music may be a primary way of \quoted{reading the book of nature} and coming to faith in nature's Creator.
\end{enumerate}

In light of these theological sources we may point towards a tentive answer for the central question, How did Catholics understand music's role in the relationship between faith and hearing?
First, the Faith of the Church, according to the Roman Catechism and traditional teachings, was not only a collection of precepts but a dynamic encounter with the Word of God, which was the Incarnate Christ himself revealing himself through his body, the Church.
So if faith came through hearing, and hearing, by the Word of Christ, then it was the church that made Christ the Word audible through preaching, liturgy, community life---and music.
And though the Church by necessity attempted to accommodate the Faith to peoples' sensory and intellectual capacities, the Church also worked to train those capacities for faithful listening, which included understanding, belief, and obedience.
To put it simply if obscurely, one needed faith to hear the Faith with faith, and thereby to grow in faithfulness.
The ultimate origin of faith in Catholic theology was the grace of God, bestowed in the sacrament of Baptism and renewed in the other sacraments of the Church---and therefore denied to those like Calderón's \soCalled{Judaísmo} who rejected the Church and were in turn rejected.
For those given the grace of faith and welcomed into the Church, however, music contributed to the ongoing proclamation and celebration of the Faith, and learning to participate in music through listening or performance (which should include listening) could be a way to grow in faith and come closer to the object of faith, the Word that was Christ.

We may distinguish at least three theological functions of villancicos as part of this theological agenda of linking faith and hearing: mnemonic, contemplative, and affective. 
The three functions in this model are outlined in \cref{tab:three-functions}.
Each of the three categories overlaps with and includes the others; all are present to some degree in any villancico, though we may distinguish certain pieces and subgenres in which one function predominates, and we may observe shifts in the emphasis on the different functions across time. 
These functions include both a definition of the role of music and a task for listeners (the latter shown in italics in the table).

%%*******************
%\begin{table}
%\caption{Three theological funtions of villancicos: The role of music and the task of listeners}
%\label{table:three-functions}
%	\input{tables/three-functions}
%\end{table}
%%*******************
%
%*******************
\subsection{The Mnemonic Function}

At a simple level, vocal music could make faith appeal to hearing simply by pairing words about Christ with music that pleased the ears of hearers.
Whether the music pleased, however, would depend on the individual temperaments and cultural conditioning of the listeners.
This may be considered a \term{mnemonic function} of music, in which the music serves primarily as an aid (in Spanish, \term{mnemotecnia}, a tool for memory) for remembering a verbal text.
Villancicos could serve a mnemonic function when they were based on set memory texts, such as \wtitle{Señor mío Jesucristo}, an official Act of Contrition text set by Joan Cererols and likely used to teach the choirboys at the Abbey of Montserrat this important prayer.%
	%
	\footnote{%
	Edition in \autocite{Cererols:MEM-VC}. % \X
	}
	%
The strophic coplas of villancicos, which may be related to oral traditions of reciting \term{romances}, also contribute to the genre's mnemonic power by associating the words with a simple, memorable, and frequently repeated melody.

%*******************
\subsection{The Contemplative Function}

In the contemplative function, the structure of music itself worked within a Neoplatonic framework to reflect and proclaim the nature of God, as discussed in the previous section.
Since music was conceived of in rhetorical terms in this period (one classic source for that being Kircher), the listener was challenged to learn to recognize the rhetorical structure of the music.
The properly disposed listener could learn to perceive the artifice of the composer within a piece of music, and thereby hear how the piece reflects the artifice of God who created the numerical ratios and physical objects through which the music can be produced.
In this function the music works somewhat independently of accompanying verbal texts as an object for Neoplatonic contemplation, to point beyond itself to higher truths.
It is contingent on the listeners' intellectual capacity to perceive musical artifice, as well as their disposition toward \quoted{spiritual listening}.

The contemplative function is primary in most of the metamusical villancicos in chapter~\ref{ch:intro} and the pieces about heavenly music studied in part~\ref{part:Singing}.
Enigma pieces and game pieces could also function as objects of intellectual contemplation.
For instance, in Mateo de Villavieja's \wtitle{Jácara en anagramas}, not only are the poetic lines combinatorial, but so are the musical phrases and even the individual voice parts, in an example of algorithmic composition.%
	%
	\footnote{%
	E-MO: AMM.4261, from the Convento de la Encarnación in Madrid, preserved at the Abbey of Montserrat.
	}
	%

If the mnemonic function was emphasized in educating choirboys, then the contemplative function was favored among the lettered elite.
One venue for this would be in meetings or sponsored services of religious confraternities.
Don Antonio de Salazar (late-seventeenth-century chapelmaster of Mexico City Cathedral, composer of \wtitle{Angélicos coros}, discussed in chapter~\ref{ch:intro}) belonged to the Confraternity of St.~Michael the Archangel.
A collection of printed sermons by the Carmelite preacher Fray Andrés de San Miguel of Puebla includes a sermon Fray Andrés was invited to preach at a gathering of Salazar's confraternity.
The title of the sermon explicitly mentions Salazar as having commissioned the sermon, and this suggests that Salazar may have been head of the confraternity as well. 
In the opening of the sermon, the preacher mentions that they are gathered in the \quoted{church of the Encarnación}.
Both Salazar and Fray Andrés were from Puebla, and the devotion to the angels and St.~Michael particularly were characteristic of that city. 
As such, this church may be that of the Augustinian monastery in Puebla, which had that dedication. 
The monk's self-deprecating introduction makes it clear that he is addressing an elite congregation of highly learned and accomplished men.
The cleric expresses his concern that he does not really know enough music to be addressing such a group, which suggests these were men specifically educated in music.
%  \X cite source
The preacher's praise for Salazar (though expected if Salazar was responsible for the paid commission) indicates a high level of respect and appreciation for this chapelmaster.
He goes on to imply that there will be a musical performance at the same liturgy, in which the audience will be able to hear this for themselves.
In fact, Fray Andrés says that Don Salazar could compose a better sermon in music than he himself could preach in words. 

The sermon that follows is an exposition of the identity and deeds of St.~Michael the Archangel.
The friar structures this discussion not according to the verses or portions of a Scriptural citation (as was more common), but according to each syllable of the Guidonian hexachord.
Since Michael's name in Hebrew means \quoted{he who is like God}, or \quoted{Quis ut Deus} in Latin, the friar uses UT to discuss Michael's name.
Since Michael is the chief of all the angels, he is their \quoted{king} or RE; and so on.
As the preacher had warned, his musical knowledge does seem to have been rather thin, since he does not use any other musical terminology or musical metaphors in the sermon. 
The sermon is about angels, not about music; but it uses the terminology of music as a framework to discuss the angels.

We do not know what music may have been performed at this service (or even what kind of liturgy it was), but Salazar's \wtitle{Angélicos coros} would seem to be the type of music that might have been chosen.
The version surviving in the Colección Jesús Sánchez Garza is arranged for the sisters of the Convento de la Santísima Trinidad in Puebla---another semi-private music venue with elite, high-level musical performance.
This is not Salazar's most learned composition, or even his most metamusical, but it does allow us to imagine how themes of music were treated in villancicos performed in closed and private spaces for circles of musical and theological connoisseurs.
Villancicos about angels have a strong Neoplatonic charge to them: the singers turn their attention heavenwards to address the angels directly (since angels were believed to be present at every liturgy, \bibleverse{ICor}(11:10)), while they also stand in for, or sing along with, their heavenly counterparts.
Angel pieces exhort the audience to lift their ears upwards as well, and ascend in the chain of contemplation beyond even \emph{musica mundana} to the music of the \emph{cielo Empyreo} in the highest Heaven.

Salazar's confraternity is similar to the private societies that sponsored the culturally important poetry competitions of imperial Spain.
Martha Tenorio sees these competitions as the defining venue for poetry in New Spain, and demonstrates that for most of the seventeenth century, these events (memorialized and disseminated in special imprints) were centers for highly cultivated poetic virtuosity after the model of Góngora.% 
	%
	\autocite{Tenorio:PoesiaNovohispana}
	%
A musical counterpart to these competitions could be the \emph{oposiciones} that religious institutions held to select new chapelmasters and organists.
Composers had every incentive to use these pieces to demonstrate their virtuosity.
Though the metamusical pieces in part~\ref{part:Singing} were probably not used as official \emph{oposición} pieces, they serve a similar function, allowing a composer to demonstrate a particular kind of skill in music and theology, and to establish a place in a tradition of such pieces.

%*******************
\subsection{The Affective Function}

A third way for music to work in the relationship between hearing and faith would be through the affects.
Music with a primarily affective function would go beyond simply projecting the verbal text (the mnemonic function), and do more than serve as a passive object of contemplation; music in its affective function would exercise direct physical power over hearers by means of sympathetic vibration and the humoral system.
The affective function may have been the closest to the body (and therefore in a sense universal), but it was also the most conditioned by culture and even by individual personality.
Music moved the affects through a developing set of associative, inter-musical relationships---that is, through musical topics and tropes.

The affect of joy is predominant in Christmas villancicos, affects of wonder and awe are emphasized in Corpus Christi villancicos, and affects of love are cultivated in more intimate villancicos for Eucharistic devotion. 
Less common are the affects of grief and pain in villancicos intended for Passion contemplation (such as \wtitle{Ay, que dolor} by Joan Cererols), or pieces dedicated to the Virgin of Solitude.%
	%
	\autocite{Cererols:MEM-VC}
	%

The affective function of villancicos had a pedagogical and formative component.
The setting of the Act of Contrition \wtitle{Señor mío Jesucristo} by Cererols mentioned above under the mnemonic function would serve not only to teach the boys the words of the prayer, but also to model for the boys how to feel and express contrition, which was necessary according to canon law for a valid confession in the sacrament of Penance. 
Affective pedagogy could also mean training the body internally and externally to move in certain ways: this could include the movements of weeping lament, quiet reverence, or vigorous dancing.
In each case, these are actions that the whole assembly must do together, in harmony, with dance as the best example (even if the \quoted{dancing} happened mostly in the interplay between musicians or was only spiritual or conceptual).

The task for the listener in the affective function of villancicos would be to let oneself be moved in sympathy with the performers and with the subject of the song to holy affections--contrition, sorrow for Christ in his passion, joy for Christ in his grace and glory, and above all, love for God and neighbor.
The goal was to be changed by this experience and formed into a community with other listeners, all attuned in sympathetic vibration (which was understood as a real physical action uniting people bodily in their affections).

Affective villancicos suggested a way that a person might be moved through the sense of hearing to believe, not just through rational proofs, but through the affective faculty, by principles of sympathetic vibration.
The affective function united \quoted{sensation} (the external sense of hearing) and \quoted{feeling} (the internal affects in response to what was heard), and could produce a physical response---a bodily reaction to the feeling, such as provoked weeping, or more commonly for villancicos, provoked laughter, gladness, and even dancing---and not alone, but together with the whole listening church.


%************************************************************
%************************************************************
\section{%
Conclusions
}

Vernacular villancicos could evoke a response in their hearers in a way that the rest of the liturgy did not.
Villancicos offered listeners an opportunity to think about (and \quoted{feel about}) the content of the faith in their own language (or at least in one they understood better than Latin), an opening otherwise only offered in the sermon, when one was preached.
As Azevedo, following the Roman Catechism, taught, the faith had to be made \quoted{pleasing to the ear}, and villancicos did just that.
Further, Azevedo taught that the \quoted{hearer of the faith} must remain completely silent and be fully attentive to the master's teaching; so likewise the many villancicos beginning \quoted{Listen!} demand just that.
And being quiet was not only a prerequisite for receiving religious teaching; it was also the expected response to the miracles and mysteries of the faith. 
Thus a villancico by Joan Cererols celebrates the logic-defying doctrine of the Assumption of the Virgin Mary and concludes with the reiterated phrase, \quoted{Callar y creer}---hush, and believe.%
	%
	\footnote{%
	Cererols, \wtitle{Serrana, tú que en los valles}, edition in \autocite{Cererols:MEM-VC}. % \X check dedication 
	}
	%

Villancicos invite listeners not simply to hear, but to \quoted{take heed}, to both discern deeper meanings in what they hear and to put what they hear into practice.
Moreover, while many villancicos do begin with an exhortation to listen, most pieces also include imperatives to actively respond to what is heard.
The command is often affective and devotional---\quoted{llorad}, \quoted{sentid}, \quoted{arde} \gloss{weep, feel, burn}, or (more commonly with Christmas pieces) \quoted{Cantad}, \quoted{Alegren}, \quoted{Repican} \gloss{sing, be joyful, repeat the angels' song}.
Other pieces call on listeners to dance and play instruments.
In other words, villancicos ask listeners to both contemplate and obey---in short, to hear the Faith with faith and to respond in faithfulness. 

From this perspective we may cautiously accept some aspects of the conventional wisdom about the function of villancicos.
The current exotic stereotype of villancicos is as a popular form of devotion that was \quoted{allowed} in the liturgy by the church authorities because they hoped it would attract the common people to Church and to faith---and that it retained some element of impious subversion from its popular roots.
But villancicos were not just passively \quoted{permitted}; they were actively cultivated by cathedral chapters and paid for with generous sums.
The poetry imprints celebrating the performances at a particular church must have been a point of pride for those who endowed the festival, and for this reason were eagerly disseminated far and wide (such as the many Seville imprints in Puebla).
And villancicos may have had a lower social register, and may have been influenced by oral traditions now lost, but the surviving written repertoire was largely composed and performed by highly educated professionals in elite settings.

Nevertheless---we can affirm that villancicos did \quoted{appeal to the people} in an active sense, imploring them to be quiet, to listen, to take heed; even as they appealed in the \quoted{entertainment} sense as well, allowing people to find humor, delight, and wonder in religious mysteries that might otherwise have remained inaccessible and uninteresting to them.
Not only because of the words, but because of the lively, diverse musical styles used, which probably were associated with a lower social register, villancicos \quoted{appealed to the ear} both in the sense of being \quoted{pleasing} and in the sense of \quoted{reaching out to} or \quoted{making a claim upon}.

There are so many \quoted{Listen!} openings that this gesture deserves deeper reflection than dismissing it as simply a practical way of getting attention, or as a generic convention. 
The recurrence of this kind of exordium in villancicos may indicate that the genre itself was fundamentally about getting people to listen.
The rest of the liturgy may have passed through the lay people's ears like the incense wafted by their noses, creating a general atmosphere of devotion but not evoking any specific sentiments (thoughts, ideas, images) in the mind and not provoking any direct response.
But the vernacular villancicos demanded attention.
Many of them presented hearers with bold, striking images at their openings, projecting an intriguing poetic conceit or scenario through musical structures and styles that made people take notice. 
In other words, villancicos made faith appeal to hearing.

But did this effort really work?
Did people understood the riddles or get the jokes, and if so, which people were they? 
Did the musical rhetoric and symbolism that is unpacked in such detail in this study really serve as objects of contemplation to anyone at the time, including the performers?

It is not possible to answer these questions fully without more documentation about how Spanish subjects at different levels of society heard and thought about music.
But from the poetic and musical texts themselves, it is possible to affirm that church leaders made a hearty effort to reach out to a wide public, both cultivated and common, and to get them to listen to the faith in a new way.

It is certainly possible that the whole culture of producing and consuming villancicos was the domain of a cultivated elite and did little actually to propagate faith among a broader range of hearers.
In the late sixteenth century Antonio de Azevedo lamented that the church continued celebrating its doctrines through festive ceremonies, even while lay people had hardly any idea what they were about:
\quoted{For this [teaching] is such an important business, especially given that it is a mortal sin that they should not know what day it is, what it means that our Lord was born, what it means that he died and rose to the heavens: and all this which the Church celebrates with such great festivities}.%
	%
	\footnote{%
	\autocite[27]{Azevedo:Catecismo}:
	\foreign{Pues es negocio tan importante, tanto que es pecado mortal, que no sepan el dia de oy, que cosa es nacer nuestro Señor, ni que cosa es morir, y subir a los cielos: y esto que la Iglesia celebra, con tan grandes festiuidades, como lo vemos.}
	}
	%
The added difficulties of teaching in the colonial context and the increasing aesthetic of complexity in learned Spanish poetry and music of the seventeenth century would suggest that many commoners remained unformed in such basic matters of faith.
As Azevedo suggests, the church's ceremonies, including the dozens of poetic verses and musical figures of villancicos in festival Matins, did actively celebrate the Church's faith---but just because cathedrals echoed with these words and music does not mean that everyone understood them on the same level.

And even for literate listeners, the complexity of music across a whole cycle of eight or more villancicos would present a challenge for any listener seeking \quoted{to hear Faith with faith} through music.
This would be especially true in metamusical pieces, which required a sophisticated knowledge of music to even make sense of the text.
As Padre Daniel Codina, monk at the present-day Abbey of Monsterrat, said of a villancico by Montserrat's seventeenth-century chapelmaster Joan Cererols (\wtitle{Suspended, cielos}, the subject of chapter~\ref{ch:Cererols}), such a piece is an explanation which itself needs to be explained.%
	%
	\footnote{%
	Personal conversation, November 2012.
	}
	%

\endinput
