% vim: set foldmethod=marker :

% Cashner, *Hearing Faith*
% chapter 1: Villancicos as Musical Theology
% 
% 2015-03-18	Dissertation defended
% 2017-11-15    New start for book proposal
% 2018-05-21    Converted back to LaTeX
% 2018-07-25    Expanded for press readers
% 2019-07-08	New start for Brill, under contract
% 2019-07-21    Complete revised draft text

\part{Listening for Faith}
\label{part:faith}

\chapter{Villancicos as Musical Theology}
\label{ch:intro}

\epigraph
{ergo fides ex auditu\\
auditus autem per verbum Christi}
{Romans 10:17}

%{{{1 intro
\section{Music in Catholic Spain, between Hearing and Faith}

%{{{2 intro^2
\quoted{Faith comes through hearing}, wrote Paul the apostle to the Christian
community in Rome, \quoted{and what is heard, by the word of Christ}
(\scripture{Rom}{10:17}).
Sixteen centuries later, amid the ongoing reformations of the Western Church,
Christians were seeking ever new ways to make faith audible.
Voices raised in acrid contention or pious devotion boomed from pulpits,
clamored in public squares, and were echoed in homes and schools.  
In new forms of vernacular music, the voices of the newly distinct communities
united to articulate their own vision of Christian faith.
Roman Catholic reformers and missionaries, charged by the Council of
Trent (\XXX[years]) to educate and evangelize, enlisted music in their
campaigns to build Christian civilization, both in an increasingly divided
Europe and in the expanding global domains of the Spanish crown.
In these efforts to make \quoted{the word of Christ} to be heard and
believed---to make faith appeal to hearing---what did they understand to be the
role of music?
What kind of power did Catholics believe music held to affect the relationship
between hearing and faith?

This book is a study of how Christians in early modern Spain and Spanish
America enacted religious beliefs about music through the medium of music
itself.
Within the political and cultural framework of the Spanish Empire, the
spiritual work of propagating faith and the comparatively worldly project of
building colonial society were in most cases inseparably fused (or confused),
and both state and church leaders cultivated music to serve both purposes.
Beginning around 1590, Spanish churches began incorporating music with
vernacular texts into their worship, in a genre of music they labeled with the
catch-all term \term{villancico}.
What had previously been an elite form of courtly entertainment and sometimes
devotion was transformed into a variety of complex, large-scale forms of vocal
and instrumental music performed as an integral part of public church rituals.

Villancicos engaged the creativity and piety of poets, composers, and
performers in every major religious institution across the empire.
Their performances constituted a major element of the soundscape of the early
modern Ibero-American world, and shaped the everyday experiences of thousands
of people across social strata.
They flourished especially in the second half of the seventeenth century but
continued to be a prominent element of Spanish Christian worship through the
nineteenth century in some places.
(The term today refers to much simpler folkloric Christmas carols, which may
descend from the earlier genre of complex notated music.)
Communities from Madrid to Manila heard and performed villancicos on the
highest feast days of the year---Christmas and Epiphany, Corpus Christi, the
Conception of Mary, and saints' days of local importance (though they
favored other genres during Holy Week).
Villancicos were typically presented in sets of eight or more, and were
interpolated after or in place of the Responsory chants of the Matins liturgy.
They were also sung in Mass and during Eucharistic devotional services.
Festival crowds also heard villancicos in the public square in processions and
mystery plays, especially on Corpus Christi.

In their seventeenth-century heyday, villancicos were a genre of religious
lyric poetry, often published in unbound leaflets or broadsheets; that
were set to music and performed by ensembles of voices, doubled or accompanied
by instruments like shawms, dulcians, and harps.
These ensembles could be as small as one or two solo voices and as large as
three separate choruses of a dozen vocal parts with multiple singers per part
and a full continuo group providing partly improvised accompaniment.
In structure most villancicos feature an \term{estribillo} or refrain section
for the full ensemble, usually through-composed rather like a motet or sacred
concerto, and strophic \term{coplas} or verses for soloists or a reduced group.
The words for these pieces were in Spanish and sometimes other vernacular
languages including Portuguese, Catalan, and Náhuatl (language of the Aztecs),
along with pieces whose texts imitated the dialects of African slaves.
The music varied in style and technique from elements of common dances
and popular tunes up to the most sophisticated polyphonic tone-painting.
Moreover, sets or cycles of villancicos for a particular feast like Christmas
included many different types of villancicos within them, offering something
for everyone.

Churches of the seventeenth-century Spanish Empire, then, became places to hear
faith proclaimed, celebrated, explained, and embodied through music that
appealed to the ears of many different kinds of worshippers---elite and common,
learned and untaught, masters and slaves---and where contrary elements were
juxtaposed in lively and sometimes unruly counterpoint.
Even after considerable loss of sources, hundreds if not thousands of musical
manuscripts of villancico settings and imprints of the poetic texts open a
window into Catholic devotional culture in this period relevant to the lives of
many thousands of people around the world.
Most of the known sources of villancico music and poetry have now been
catalogued, but only a handful have been revived in performance and the
genre has received relatively little serious scholarly attention in terms of
poetic and musical analysis or interpretation.

This book is the first major effort to understand the meanings and functions of
this music as a form of religious practice, integrating musical and theological
interpretation.
The goal is not to provide a comprehensive musicological treatment of the genre
in all its aspects, but to use select examples of this genre to reach a deeper
understanding of early modern religious culture.
Theological interpretation is only one of many valid and necessary approaches
to this multifaceted genre, but it is my conviction that we will never fully
understand villancicos without understanding their role first and foremost as
expressions of religious belief. 
Villancicos played a central role in the devotional life of Spanish Catholic
communities.
Most focus on Jesus (especially his Incarnation and his sacramental presence in
the Eucharist), Mary, saints, or angels.
They were composed by professional church musicians, who were typically under
contractual obligation to provide this music to be performed in liturgical
worship and paraliturgical celebrations.
They were heard in cathedrals, parish churches, and monastic houses, brought to
life by performers who were often under a vow of religious life.
It should require no special pleading, then, to insist that we need to
understand this music within the profoundly theological context in which it was
patronized, created, performed, and heard.

Of all the musical forms of Catholic Spain, in fact, sacred villancicos address
the theological nature and function of music most frequently and directly.
A large number of villancicos begin with calls to listen---\emph{escuchad},
\emph{atended}, \emph{silencio}, \emph{atención}. 
Because so many villancicos explicitly address concepts of music, sensation, and
faith, these remarkable but understudied pieces offer us unique insights into
Spanish beliefs about music.
When villancicos focused on the theme of music itself, most often by playing on
terms from music theory to build elaborate theological metaphors, they become a
sounding discourse on musical sound.
If a play within a play in seventeenth-century Spanish or English theater is
metatheatrical, then these pieces are \emph{metamusical}.

Through this genre of musical performance people embodied their theological
conceptions of music through the structures of music itself.
For this reason they may be considered as \quoted{musical theology}.
It is the central argument of this book that this devotional music provided
Spanish Catholics with a way of performing theology. 
Making and hearing music was a creative pursuit in which people sought to forge
connections to God and to each other through musical structures.

Educated Spaniards had studied both the theology of Augustine and Aquinas and  
the fundamentals of music on theoretical and practical levels. 
They had learned from Boethius how human music was linked to cosmic harmonies,
and they had learned from Guido of Arezzo how to sing through the gamut using
the mnemonic device of the Guidonian hand.
Metamusical villancicos brought these two domains of knowledge into a mutually
illuminating relationship.
Someone reading the poems or hearing them read had to know a fair amount of
music theory in order to understand the theological concepts, and vice versa.
Hearing the music settings of poems that were themselves about music required
an even higher degree of training to understand the words as projected through
the music and perceive the ways the music depicted the sense and affect of the
text.
Though these challenges surely must have kept some hearers from understanding
villancicos fully (just as we might say about audiences for Monteverdi and
Schütz, or Calderón and Shakespeare), I argue that the pieces studied in this
book actually had the potential to effect a spiritual kind of ear training. 
They were discourses about musical listening, through the medium of music, that
also taught people how to listen.
%}}}2

%{{{2 singing about singing : examples
\subsection{Singing about Singing}

It will be helpful at this point to listen to two villancicos that I would
classify as metamusical---pieces that embody \quoted{singing about singing}.%
    \autocites{Murata:Singing}
    [\XXX]{Illari:Polychoral}
The first example is a villancico from the 1652 Christmas cycle written for the
Cathedral of Puebla de los Ángeles in New Spain by Juan Gutiérrez de Padilla
(\circa{1590}--1664).%
\begin{Footnote}
    \sig{MEX-PC}{Leg. 1/2}. 
    \XXX[Audio recordings of every numbered example of music and poetry in this
    book are available on the companion website.]
    Another villancico by this composer is the subject of
    \cref{ch:padilla-voces}.
\end{Footnote}
In just the first seven lines of this anonymous text
(\cref{poem:En_la_gloria_de_un_portalillo-Padilla-1652}), the villancico refers
to multiple kinds of music, referring to the sounds of voices, choirs singing,
birdsong, dancing, and using solmization syllables.

%{{5 poem GdP En la gloria
\insertPoem{En_la_gloria_de_un_portalillo-Padilla-1652}
{\wtitle{En la gloria de un portalillo}, estribillo as set by Juan Gutiérrez
de Padilla, Puebla Cathedral, Christmas 1652 
(\sig{MEX-Pc}{Leg. 1/2})}
%}}}5

Gutiérrez de Padilla's setting is metamusical in that it enacts these
references through music (\cref{mux:Padilla-Portalillo}).
It also has the virtue of demonstrating several typical features of the genre
which should provide a useful foundation for our study.
The solo line is followed by a passage of polychoral dialogue between two
four-voice choirs, concluding (typically for polychoral technique) with an
emphatic cadence for the full chorus.  
The setting is in a lively triple meter: this was termed \term{tiempo menor de
proporción menor} and notated \meterCZ{} (CZ), a cursive shorthand for
\meterCThreeTwo{} or \meterCThree{} (according to the contemporary music
treatise of Andrés Lorente).%
    \citXXX[Lorente]
Within this meter the composer makes frequent use of \term{sesquialtera} or
hemiola, an alteration of the rhythmic pattern that feels like a momentary
shift from ternary to duple stresses.%
\begin{Footnote}
    \XXX[See the glossary and preface]
    The preface provides additional background about the terminology and common structures of seventeenth-century villancicos.
    Please note the discussion there on common voicing and instrumentation patterns, and on rhythmic theory.
\end{Footnote}
The shifts of duple and triple stresses combine with stresses on the second
beat of the \term{compás} (\term{tactus}, measure) to create an energetic
atmosphere with a rejoicing affect.  
The polychoral dialogue, with the voices of each choir declaiming
homorhythmically in the same highly rhythmic, syncopated manner as the soloist,
and with the \term{tiples} (boy sopranos) of both choirs singing at the top of
their range, would have brilliantly seized the attention of listeners.

After this introductory \term{exordium}, the Tiple I soloist continues to
describe the scene at the manger.  
As the shepherds \quoted{are turned to boys} (\foreign{se vuelven niños}),
Gutiérrez de Padilla has the musicians \quoted{turn} modally by adding C
sharps, accented in a sesquialtera ($3:2$) group.
The passage that follows this moment is in evenly accented ternary patterns, in
two-measure groups.  

These groups emphasize the rhymes in \foreign{tonos sonoros, repiten a coros}
and the clear triple meter evokes the dances of \foreign{en bailes lucidos}.
When the soloist refers to the newborn Sun, he sings the note identified in
Guidonian terminology as D \term{(la, sol, re)}---\term{sol} in the hard (G)
hexachord.  
On the same word, the bass accompanist plays a different \term{sol}, G
\term{(sol, re, ut)}. 
(Note that \quoted{sol re} in Spanish means \quoted{sun king}.)%
\begin{Footnote}
    The major Spanish music-theoretical treatises of the seventeenth century
    give full expositions of the techniques of Guidonian solmization:
    \autocites{Cerone:Melopeo}{Lorente:Porque}.
    The frequent symbolic use of Guido's syllables in villancicos suggests that
    these treatises do reflect how music was actually taught in practice.
\end{Footnote}

%{{{5 music gdp portalillo start
\insertMusic{Padilla-Portalillo}
{Gutiérrez de Padilla, \wtitle{En la gloria de un portalillo}
(\sig{MEX-Pc}{Leg. 1/2}, Christmas 1652), estribillo}
%}}}5

This villancico may be understood as \quoted{singing about singing} on several
levels.  
The text, which is being performed through music, itself refers to musical
performance.
The performance by the Puebla Cathedral chapel dramatizes the historical
celebration of the first Christmas while also celebrating the festival in
Padilla's present day.  
The music is self-referential on a symbolic level (as in the plays on
\term{sol}), but also functions on a more simple affective level to model and
incite affections of exuberant joy and wonder, which contemporary theological
writers emphasized were the appropriate affects for the feast of Christmas (see
\cref{ch:padilla-voces}).

%{{{3 cererols fuera que va
A similar example of a villancico that includes multiple metamusical topics is \wtitle{Fuera, que va de invención} by Joan Cererols (1618--1680), monk and chapelmaster at the Benedictine Abbey of Montserrat near Barcelona.%	
\begin{Footnote}
    \sig{E-Bbc}{M/760}, \autocite[81--94]{Cererols:MEM-VC}.
    A villancico by this composer is the subject of
    \cref{ch:cererols-suspended}.
\end{Footnote}
Rather like catalog-like Christmas songs today (\wtitle{Deck the Halls},
\wtitle{Chestnuts Roasting on an Open Fire}), the piece summons up all the
elements of a Christmas festival---masques, \foreign{zarabandas} and other
dancing, lavish decorations and clothing, pipes, drums, and so on.
As in many villancicos, the chorus acts dramatically in the role of the
festival crowd, shouting affirmations (\foreign{¡vaya!}) for each element of
the celebration as the soloists name them.  
Whereas Gutiérrez de Padilla's \wtitle{En la gloria de un portalillo} focused
primarily on the music of the historical Christmas day, the villancico of
Cererols is unambiguously about celebrating \soCalled{Christmas present}.
The piece seeks a theological meaning behind the Christmas customs: the masques
of Christmas, the poem says, are appropriate because in the Incarnation of
Christ, \foreign{Dios se disfraza} (God masks himself).
The villancico allows performers and listeners to celebrate the festival in two
senses: to sing the praises of the Christmas feast, while also singing the
praises of Christ that are appropriate to that feast. 
Cererols's original audience of pilgrims to the mountaintop shrine of
Montserrat would not have sung along with this piece, but the piece still
invites their wholehearted participation in the rituals of Christmas, both
through enjoying the choral singing (and joining \quoted{in spirit}, perhaps),
and in the many other common-culture customs that the piece celebrates.

These pieces presented hearers with a discourse about music, through music.
Sometimes the music they refer to is literal, human music-making; other times
it is more abstract, like the music of the spheres or the harmony of human and
divine in the incarnate Christ.
In every case, analyzing the musical choices made to represent texts about
music helps us understand how the creators and their audiences heard different
kinds of music.
And interpreting their theological aspect enables us to see how these pieces
served to communicate with hearers at a spiritual level.
%}}}3
%}}}2

%{{{2 hearing/communication
\subsection{Hearing and Communication}

As the most widespread form of Catholic religious music with vernacular words
after the Council of Trent, villancicos provide evidence for a sustained
endeavor by church leaders to establish conventions of communication with
ordinary people.
% XXX could mention other kinds of Catholic vernacular song (German, Czech?)
The creators of villancicos drew on common experiences of everyday life and
linked them to the sacred in inventive ways that met the spiritual needs of
specific communities.
Each piece provides a new answer to Christ's question, \quoted{With what can we
compare the kingdom of God, or what parable will we use for it?}
(\scripture{Mk}{4:21}).
Villancicos thus played a key role in the Spanish church's effort to to make
faith appeal to hearing through music.

This music spoke to a variety of people at different levels of understanding.
It was a central part of community festivals in many different local
environments and a variety of public and private contexts. 
Each villancico cycle includes an array of subgenres that would speak to
different portions of the congregation, from the silly and child-friendly
dialogues of Christmas shepherds to high-concept pieces with, for
example, abstruse numerological symbolism.

Even the structure of individual villancicos reflects the effort to communicate
on multiple levels.
The \emph{estribillo} section of a typical villancico was presented by the full
ensemble at the beginning and then repeated at the end; composers usually set
this in relatively complex polyphony similar to what they would use for a
motet.
In the center of the piece, the \emph{coplas} or verses were usually set
strophically for solo singers or a reduced ensemble with accompaniment.
As Bernardo Illari argues, the \emph{copla} settings are probably based closely
on oral traditions for singing poetry, especially in the \emph{romance} meter,
to stock melodic formulas; and it would have been easier for common listeners to
make sense of the words that were sung to the simple, repeating melodies.%
    \Autocite{Illari:Polychoral}
The \emph{estribillo}, by contrast, is often much more complex and draws on
traditions of learned counterpoint; composers often invoke a variety of
stylistic registers and styles to convey the meaning of the words and heighten
their rhetorical impact.

But though villancicos have these aspects that seem designed to engage a wide
popular audience, they differ from other dominant traditions of vernacular
religious music in this period---Lutheran chorales and Reformed psalms---in
that they were not sung by ordinary parishioners.
Rather, more like Anglican anthems and German sacred concertos, they were
performed by professional church musicians for the benefit of the congregation.
The printed commemorative chapbooks and manuscript performing parts preserve
only one side of the church's dialogue.
Catholics did not, generally speaking, cultivate a society of literate,
self-advocating lay people who would have left behind traces of their personal
beliefs and devotional practices with regard to music.
For the Spanish Empire, then, we know what people heard, but not what they
understood or how they responded.%
    \Autocite{Burstyn:PeriodEar} % + Did people listen? etc.
And when villancicos represent types of people---such as deaf men, African
slaves, or Indians---they leave us only with conventional caricatures, not
ethnohistorical descriptions.%
    \Autocites
    {Baker:EthnicVC}
    {Baker:PerformancePostColonial}
    {Davies:LocalContent}

All the same, the devotional music that survives from imperial Spain
can still open a fascinating window into the process of religious communication.
First, villancicos should not be understood as an exclusively top-down
communication, and certainly not as a simple mode of religious indoctrination.
The creators of villancicos were not always members of the most elite strata,
and their readers and hearers included commoners.
The cultivated poet Francisco de Quevedo was credited with mocking \quoted{the
whole caste of villancico poets} as hacks, saying that \quoted{the poor are
drowning in poets, continually hearing their braying}.%
    \Autocite[37]{Torres:SuenosMorales}
If there is any truth to the critique of villancico poets as stringing together
clichés to satisfy the tastes of a lower-class market (an attack also leveled at
opera librettists in Italy), then the same low-class elements that those poets
disdained can provide us with insight into culture at a more common level.
On the musical side as well, some villancico composers were not prestigious
cathedral chapelmasters and we know of at least one who was of indigenous
ancestry, Juan de Araújo in Boliiva.%
    \Autocite{Illari:Popular}
%    \citXXX[non-MC composers, Illari]
Besides, regardless of their personal background, villancico poets and composers
had to produce something that met the needs of their community. 
Though they answered first to their cathedral chapter and the local ruling
caste, it was in everyone's interest to attract commoners to church and provide
them something that they would find satisfying.
According to contemporary accounts people turned out in droves to hear the
annual villancico performances, in annual traditions that in most Spanish
cities lasted for two centuries or more.%
    \citXXX[audience turnout]
Somewhat like mass-mediated popular music today, this music was not typically
created by common people themselves, but it both reflected and shaped popular
tastes and attitudes.
%}}}2

%{{{2 conceptismo
\subsection{\term{Conceptismo} and Creative Theology}

Most of the villancicos performed in seventeenth-century churches may be
described as attempts to connect often abstract religious concepts to images
and experiences from everyday life.
The more surprising and puzzling the connection, the better---such as
representing the Virgin Mary as the chapelmaster of the heavenly chorus (see
below), or imagining Christ as a gambling card player.%
    \Autocite{Cashner:PlayingCards}
Christmas villancicos in particular often brought rogues, buffoons, peasants,
and slaves to Christ's manger to offer songs and dances characteristic to
them--even Don Quijote and Sancho Panza make an appearance in one villancico.%
    \citXXX[Don quijote VC]
While these folkloric and comic elements can be amusing, we should not lose
sight of the fact that these elements are almost always used to point to some
theological aspect of Christmas.

In Spanish literary studies, the term \term{conceptismo} has been used to
denote a technique in which poets used extended metaphors to create parallel
discourses that brought two subjects into relationship with each other.
As early as 196X\XXX[name], English poetry scholar \XXX[name] cited Alonso de
Ledesma's (\XXX[year]) book of devotional poetry, \wtitle{Conceptos
espirituales}, as a fountainhead for the poetic technique of villancicos that
compared worldly and ethereal, human and divine elements.%
    \citXXX[source of both and quote?]
Ledesma's earthy approach, aimed at relatively uncultivated readers, differed
starkly from the later \term{conceptista} poetry of Luis de Góngora and his
followers on both sides of the Atlantic, in which the language was pushed to
the brink of comprehensibility to make elaborate and highly learned double and
triple conceits.%
    \Autocites{Tenorio:Gongorismo}
    [227--228]{Gaylord:Poetry}
    {Gracian:Ingenio}

Both ends of this spectrum are well represented in the villancico poetic
repertoire, and similar extremes may be found in approaches to the musical
setting as well.
It has been suggested that villancicos be explained as simply a religious
variant of \term{conceptismo}.%
    \citXXX[reviewer? Torrent?]
But the label \term{conceptismo} alone does not do much to help us understand
how this poetry and music actually worked for readers and hearers as a form of
religious devotion, any more than the term \term{chiaroscuro} explains what
Caravaggio actually hoped would happen for viewers when they looked at his
paintings.%
    \citXXX[Torrente, Begue?]
It does not explain why churches cultivated villancicos or tell us what people
got out of hearing them.

I propose that on the contrary, we understand \term{conceptismo} in the this
context as a specific literary form for doing a certain kind of theological
thinking.
The art, literature, and music of early modern Spain are so saturated with
religious themes that we must conclude that theology was a major intellectual
pursuit of the Spanish and New Spanish elite.
What I mean by theology is a creative activity---not merely reciting dogmas
approved by the Church, but imaginatively, even playfully, seeking out ever-new
ways of connecting revealed truth to observed experience.
Thinking theologically in an early modern Catholic sense meant building 
endless chains of association and allusion among Biblical texts, writings of
church fathers (patristics), medieval theologians, and the liturgy. 
It meant interpreting new texts in light of these old ones, and reinterpeting
the old ones in light of the new.
And it grew out of and reinforced a view of the world as a book waiting to be
read (see below).

Villancicos challenged listeners to discern how the sacred was imminent in the
mundane and common.%
    \citXXX[Chavez:PhD]
Through words and music, these pieces asked hearers to hold together unexpected
combinations of elements and search out a new meaning to be found by going back
and forth between the two.
If I may use some terms creatively myself, the religous purpose of villancicos
was not so much \term{doctrine} as \term{doxology}.
Theologians use the latter term with its Greek meaning of glorification or
worship; the Eastern Orthodox use this root to identify not as \quoted{true
believers} but \quoted{true worshipers} (with the assumption that true worship
is rooted in true belief and in turn gives rise to it).%
    \citXXX[theology, lex orandi etc.]
The point of comparing Christ to, say, a plucked string instrument (the
\term{vihuela}, see \cref{ch:zaragoza}) was not to teach a particular
doctrine about Christ but to worship Christ in a particular way.
Catholics who revere Our Lady of Montserrat know that they are asking for the
intercession of the same Blessed Virgin elsewhere honored as Our Lady of
Sorrows or Our Lady of Loreto: there is one Lady, but many ways of reflecting
on her life and applying it to one's own.%
    \citXXX[Johnson?]
The Catholic liturgical texts and sermons for Christmas do aim to instruct
believers about the theological concept of Christ's Incarnation, but they also
provide a way for them to celebrate the Incarnation, and through the Eucharist,
to actually share in it.

Most villancicos depend on knowledge of doctrine more than they teach it.
As Padre Daniel Codina of the Abbey of Montserrat remarked in puzzlement at a
villancico by Montserrat monk Joan Cererols, they seem like \quoted{an
explanation that itself needs to be explained}.%
    \footnote{Personal communication, \XXX[date].}
Rather like contemporary emblem books (see \cref{ch:zaragoza}), which featured
a picture with a Latin motto, a Spanish poem, and a prose explanation, each
element of a villancico increased one's depth of understanding of all the other
parts, with the result that the whole could became a mnemonic device for a
whole complex of ideas.

It would be wrong to assume, then, that examining the theology of villancicos
today would not reveal any more than the official teachings of the Tridentine
catechism.
Of course, the texts were subject to church censorship, and we should not
expect to find anything radically subversive in them, not at least at the
surface level that would catch a censor's eye.
But it was possible to stay close to church teachings on the more theological
side of a poetic conceit while reaching far afield into worldly experience for
the other side, and in fact this is what we find in villancicos.
Comparing Christ to a vihuela, to continue with that example, taught people who
knew about the vihuela new ways of thinking about Christ, just as it taught
people who knew about Christ new ways of thinking about the vihuela.
In short, it made the vihuela into an object for theological contemplation or
reflection.%
\begin{Footnote}
    I am not using \term{contemplation} in the highly technical theological
    sense in which it was used by the writers of treatises on prayer like
    Teresa de Ávila or San Juan de la Cruz, but rather in the more general, and
    widely used early modern sense of mental reflection and meditation.
    The technical sense denotes a practice of seeking to interact with God in
    prayer with a sense of God's absolute ineffability and not through lesser
    images and sensory stimulations. 
    It is used, I would argue, as a shorthand for \quoted{contemplating God as
    God really is}, and in this way the more general meaning of the verb is
    actually preserved.
    One could contemplate many things, but \quoted{religious contemplation} or
    \quoted{contemplative life} meant a dedicated process of learning to
    contemplate this one thing.
    Moreover, there are actually times in which the kind of contemplation
    encouraged by villancicos seems to share some of the goals of this kind of
    contemplative prayer.
\end{Footnote}
Moreover, the musical settings, which were not subject to censorship, add
layers of meaning and shape the experience of the performance in ways that
cannot be reduced to a simple idea of teaching doctrine.
Because there are not many written texts from imperial Spain telling us how
people interpreted the music they heard, the body of villancicos about music
provide vital evidence for how and why Spaniards listened to music.
%}}}2

%{{{2 scholarship: integrate music and theology
\subsection{The Need for an Integrated Musical and Theological Approach}

% TODO topic sentence; other sources (?) from diss lit review, in notes file

We can only understand how they listened, though, if we pay close attention to
the sounds they listened to.
Villancicos on the subject of music encapsulate theological understandings of
music within musical performance.
These pieces offer a modern interpreter more than just verbal explanations of
music, such as may be found in music-theory treatises or doctrinal statements;
rather, they provide the opportunity to hear how early modern musicians created
a true \term{musical theology}---a form of music that embodies the beliefs it
proclaims.
Charles Seeger theorized that music and language were two distinct forms of
discourse and ways of knowing.% 
    \Autocite{Seeger:Unitary} 
Just as one could speak about making music (that is, create verbal discourse
that referred to a musical form of discourse), one might also make music about
speaking.
More intriguingly, if one could speak about speaking (which would constitute
much of academic discourse), would it be possible to \emph{music} about music? 
The villancicos studied in this book do just that: they reflect on the nature
of music through the medium of music.

If inquirers today wish to know what early modern Christians believed, then, we
must listen carefully to how they made their faith heard.
The sound of early modern sacred music---the way voices move in relationship to
each other, the characteristic stylistic features of common chordal
progressions, rhythmic gestures, the dramatic experience of musical forms
unfolding in time---all of this provides a window for us to glimpse something
of the religious experience of historical believers.
Put the other way, if we wish to understand not only what music meant to early
modern people but even the details of how music worked, we must contemplate
what the makers and hearers of that music believed about its sacred power. 

Much of the valuable new scholarship on Spanish colonial music, and on sound
and sensation, focuses not on musical texts but on social and institutional
history, and on verbal discourse \emph{about} music.%
    \Autocites{Baker:Harmony}{BakerKnighton:MusicUrbanSociety}
    {Irving:Colonial}{RamosKittrell:PlayingCathedral}
    {DellAntonio:Listening}
% TODO more lit
This book provides a necessary complement to these studies, by analyzing how
people expressed and shaped beliefs about music through the medium of music
itself.
At the same time, the book offers a fresh approach by considering this music as
a source for historical theology, something few scholars have done.
The primary goal of the project is to combine musical and theological analysis,
to understand how theological beliefs were expressed and shaped through the
details of musical composition and performance.

Until recently, scholars and performers have focused on a small repertoire of
villancicos that does not really represent the full range of pieces
that were commonplace in this period, resulting in a conventional view of
villancicos as primarily folkloric, secular---that is, worldly,
irreverent---comic, small in scale, and primarily of interest for the traces
they appear to preserve of popular culture, particularly that of Native
Americans and Africans.
No doubt, plenty of villancicos do fit that description, but as I have argued
elsewhere, even these pieces need to be understood within a theological
frame---and we would benefit from thinking more deeply about what we can
actually learn from these Spanish representations of non-Spaniards.%
    \Autocite{Cashner:BuildingSociety}

Villancicos, when understood within a religious framework, offer unique
insights into the worldview and life experience of people in early modern
Spain.
Understanding them requires a process of interpretation in which we must
attempt to enter imaginatively and sympathetically into the same kind of
theological reflection that the original hearers may have done.
Necessarily this process, like all acts of interpretation, is subjective to a
degree.
If my readers are like me, the details of belief expressed in these pieces will 
be foreign and perhaps surprising.
I am a Christian but I am not a Roman Catholic, and while I may sympathize with
certain elements of the historical belief system I elucidate here, there is as
much that I find offensive or just silly.
Certainly our interpretations need to take a critical view of the many
contradictions and tensions in the texts, and recognize the distance between
the imagined world of these texts and the real social worlds around them. 
All the same I do want to take villancicos seriously as expressions of real
religious belief and articulations of a widespread historical worldview.

This book is not, then, a work of constructive or normative theology.
In fact, I argue that interpreting villancicos requires us to set aside our own
religious ideas---or anti-religious ideas---in order to hear the world through
historic ears, to the extent we can venture to do so.%
    \citXXX[historic ear]
While I hope that Christian readers may find the book edifying and relevant to
the contemporary challenge of making faith appeal to hearing, just as I hope
that musicians may discover here new possibilities for historically informed
and culturally sensitive performances, those practical applications must
remain outside the scope of this book.

Our challenge is great enough already---to build a sufficient interpretive
framework that we can recover a plausible range of meanings that this music
might have had for its creators and first hearers.
The primary methods for accomplishing this here are comparative study within a
corpus of pieces on related themes, mostly edited for the first time from the
original sources in nine archives in Mexico, Spain, the United States, and the
United Kingdom.
This corpus makes it is possible to identify recurring and developing tropes
and conventions, which may be understood more deeply through and contextual
study of related literature and arts.
The related literature includes poetry and sacred drama, theoretical and
practical treatises on music, and several branches of theological writing.
These sources include doctrinal theology (explanations of Christian beliefs),
exegetical theology (interpreting Scripture), homiletics (preaching, applying
Scripture and doctrine to daily life), and devotional literature (teaching and
modeling prayer and worship for personal and community life).
Sources more commonly used by institutional and social historians like chapter
acts, payroll records, and notarial documents play a much smaller role here
than in other recent studies, though I hope that future scholarship will
combine these and other approaches, most feasibly through collaboration, to
produce a fuller understanding of Spanish musical culture.

The religious element of early modern Spanish culture is so overwhelmingly
evident to anyone who has visited Mexico or Spain or read any of its
literature, that it can hardly be justified if we overlook it or insist on
interpreting it through an anti-Catholic lens.
This is simply a study of how people in the past expressed beliefs through
poetry and music, intended for readers who like me are interested in
understanding historical religious beliefs and practices, who enjoy moving
through the hermeneutic circle in pursuit of new perspectives on past and
present.%
    \citXXX[ricoeur]
This book is for anyone who has gazed upwards in a Spanish or Mexican church
and wondered why there are so many images of angel musicians with harps,
\term{vihuelas}, \term{bajones}, and organs; or who has read a play by
Calderón, a poem by Sor Juana, or a devotional book by Ignatius of Loyola or
Saint John of the Cross, and has observed how often these writers use musical
metaphors; or who wonders what people thought was happening when they listened
to music in church and how they believed this connected them to God, to each
other, and to the cosmos.
Through the many ways that Spanish villancicos engaged their audiences' sense
of hearing, and through the ways the pieces model musical hearing itself, they
also offer a glimpse of what a broader audience of common people listened for
in music and what powers they believed it had to shape their community.

The book is organized in two parts, in which the two chapters of the first part
uses a global sampling of the villancico repertoire to deal with the broad
questions of what Spanish Catholics believed about music, as expressed in their
music; and the problems and anxieties they encountered in their efforts to use
music to connect faith and hearing.
The three chapters of \cref{part:unhearable-music}, then, present case studies
of individual villancicos or sets of closely related pieces focused in Puebla,
Montserrat, and Zaragoza, respectively.
These chapters demonstrate that Spanish composers used metamusical villancicos
to establish their place in a lineage of composition, while also revealing how
their conventions for depicting heavenly music in particular developed over the
century.
Each chapter brings villancicos into dialogue with a different kind of sources:
in this chapter, Neoplatonic theological literature; in
\cref{ch:faith-hearing}, catechetical literature, sacred drama, and
philosophical and physiological treatises.
\cref{ch:padilla-voces} interprets a villancico in light of exegetical and
homiletical literature as well as visual art and architecture;
\cref{ch:cererols-suspended} focuses on astronomical science and cosmology; and
\cref{ch:zaragoza} engages with emblem books, physics, mystical literature, and
music theory.
The glossary and the accompanying website\XXX{} are meant to help fill any gaps
in understanding for readers approaching this work from various disciplinary
backgrounds.
% XXX

We will begin by surveying the repertoire of villancicos on the subject of
music and analyzing the different ways that these pices engage listeners by
referring to different kinds of music.
The examples are chosen with the additional goal of introducing readers to a
representative range of the different types and subgenres of villancicos, drawn
from most of the major archival collections, and by composers and poets who
were widely known and influential throughout the empire in their time
(Cristóbal Galán, Juan Hidalgo, Joan Cererols).
The music for Puebla Cathedral by Juan Gutiérrez de Padilla, though it does not
appear to have circulated across the Atlantic, also features prominently
because first, this composer was chiefly responsible for establishing one of
the most important musical ensembles in the colonial New World; second, his
music stands out from others for its close attention to textual declamation,
word-painting and symbolism, and the dramatic effect of its intricate and
virtuosic use of counterpoint and rhythm, and it is preserved in complete
cycles of villancicos; and third, because this composer appears to have
been especially fond of writing music about music.
%}}}2
%}}}1

%{{{1 music about music
\section{Music about Music in the Villancico Genre}

%{{{2 survey, types
The examples of metamusical villancicos by Gutiérrez de Padilla and Cererols
combine several of the common tropes of \quoted{music about music} in the
villancico genre, as evidenced by a global survey of extant villancico poems
and music.%
\begin{Footnote}
    The survey was based on the listings in catalogs and published studies and
    from archival sources, some previously unknown, from all over the former
    Spanish Empire (see \cref{biblio}).
    While I was able to examine hundreds of complete music manuscripts and
    poetry imprints in the nine archives which I visited personally or which
    made sources available to me electronically, for the others I had to infer
    content based on catalog incipits and descriptions.
    Experience with the pieces studied in detail in this book suggests that
    this approach revealed fewer rather than too many relevant sources, as it
    is not always possible to guess the themes of a villancico poem from its
    first line, and variants of the same textual family are often hidden by
    different opening verses.
\end{Footnote}
More than eight hundred villancicos were found in which themes of musical
hearing were central, a number that only hints at the original size of this
repertoire.
These metamusical villancicos may be grouped in eight main categories
(\cref{tab:survey}): in descending order of frequency these are hearing and
sound, music and singing, birdsong, dance, musical instruments, angelic
musicians, music of the heavenly spheres, and pieces that treat the
relationship of sensation and faith.
An additional category of pieces about affects is also included, since these
pieces, though not explicitly about music, do address the question of how
listeners should respond in worship. 

%{{{5 table survey topics
\insertTable{survey}
{Topics of metamusical villancicos in global survey}
%}}}5

The central section of this chapter will look at examples in the most
widespread categories to analyze the different ways that these pieces represent
making and hearing music in relation to faith.
Villancicos addressing the sense of hearing specifically will be the topic of
\cref{ch:faith-hearing}; and \cref{part:unhearable-music} will present case
studies of families of villancicos about music and singing, most of which also
concern heavenly and angelic music.

In each of the categories in \cref{tab:survey}, we may distinguish between two
main ways of referring to music.  
Some pieces are primarily imitative, referring to real human music-making
(Boethius's \term{musica instrumentalis}).
These pieces are highly \term{intermusical}, in the way a verbal text full of
references to other texts is intertextual.
In contrast to this first category of imitative pieces, villancicos in a second
category refer to music as more of an abstract concept, such as the higher
Boethian levels of music, music as a Neoplatonic ideal, and the music of
Heaven---notions that overlap in inconsistent ways in early modern thought.
Of course, the pieces in the latter group still refer to music in the abstract
through the medium of real sounding music.  
Some of these pieces depend more on \term{intramusical} relationships---that
is, musical references internal to the individual piece itself, such as melodic
or rhythmic motives or internal contrasts of musical style without overt
references to pre-existing styles \quoted{outside the piece}.
We will consider first examples with imitative references, and then look at the
more abstract or symbolic references.
The final section of the chapter will then outline the historical theological
foundations that undergirded these practices of musical representation.
%}}}2

%{{{2 imitative references
\subsection{Imitative References to Music: Birdsong, Instruments, Songs and
Dances}

%{{{3 birdsong
A frequent example of imitative musical reference in villancicos is when the
ornamented vocal lines are used to represent birdsong.
In a piece called \wtitle{Sagrado pajarillo} (Little sacred bird), Zaragoza
composer José de Cáseda sets the lyrics \foreign{con gorgeos} (with trills) to
twittering melismas (\cref{mu:CasedaJ-Sagrado_pajarillo}).%
\begin{Footnote} 
    This piece comes from the archive of the Conceptionist Convento de la
    Santísima Trinidad in Puebla de los Ángeles and is now preserved at CENIDIM
    in Mexico City (\sig{MEX-Mcen}{CSG.155}).
    As with many other music manuscripts in that collection, the original
    lyrics (beginning \foreign{Sagrado pajarillo}) were replaced by another
    text (beginning \foreign{Fecunda planta viva}), which was pasted and sewn
    over the original words with thin strips of paper.  
    The original may still be seen by lifting the strips.
    For another example of the bird trope by this composer, see
    \cref{ch:zaragoza}.
\end{Footnote}
Birdsong had theological importance as the paradigm of music-making in the
natural world.
The widely read Dominican writer Fray Luis de Granada, in his theological
interpretation of the natural world, says that birds reflect the harmony built
into the created world by its divine Creator because they sing purely by nature
rather than by reason, as humans do.%
    \citXXX[luis]
The only artifice to be heard in birdsong was that of God himself.
Using human voices to imitate birdsong, then, prompted listeners to consider
how the artifice of human music reflected the order of creation (this theology
is discussed in more detail below). 

%{{{5 music CasedaJ Sagrado pajarillo
\insertMusic{CasedaJ-Sagrado_pajarillo}
{Bird-like trills in Cáseda, \wtitle{Sagrado pajarillo}, excerpt from the
estribillo, Tiple I-1}
%}}}5
%}}}3

%{{{3 instruments: percussion
Next to the musical sounds of animals, the sounds of musical instruments
provided rich possibilities for musical imitation in a theological context.
Wooden sounding boards, brass pipes, and gut strings allowed human players to
take the potential of music built into the created world---such as the perfect
Pythagorean ratios of the overtone series---and actualize them in sound.

To imitate percussion instruments, for example, villancico composers paired
onomatopoetic nonsense words with distinctive rhythmic patterns.
Juan Gutiérrez de Padilla had the chorus of Puebla Cathedral represent the
sound of the castanets and tabor with contrasting onomatopoetic rhythmic
patterns on the words \foreign{al chaz, chaz de la castañuela, y el tapalatán
de el tamboríl} (\cref{mux:Padilla-Alto_zagales-chaz}).
Such pieces about instrumental music imitate the instrument itself while also
playing with a stylistic topic associated with that instrument.

%{{{5 music GdP Alto zagales chaz
\insertMusic{Padilla-Alto_zagales-chaz}
{Gutiérrez de Padilla, \wtitle{Alto zagales de todo el ejido}
(\sig{MEX-Pc}{Leg. 2/1}, Christmas 1653), estribillo: Imitation of castanets
and tabor (or tambourine?)}  % XXX check and signature
%}}}5

The same instrumental trope appears in a villancico poem performed at Toledo
Cathedral in 1645.%
    \footnote{\sig{E-Mn}{VE/88/12, no. 6}.}
Though the music, credited in the poetry imprint to Vicente García, has not
been found, the words alone conjure up a racket of percussion sound:
\begin{quotepoem}
    Porque los instrumentos sonaban así, 
        & Because the instruments sounded like this: \\
    El Atabal, tan, tan ,tan,	    & the drum, tan, tan, tan, \\
    El Almirez, tin, tin, tin, 	    & the mortar, tin, tin, tin \\
    la Esquila, dilín, dilín,	    & the chime, dilín, dilín, \\ 
    y la Campana, dalán, dalán,	    & the bell, dalán, dalán, \\
    Las Sonajas, chas, chas, chas,  & the rattle, chas, chas, chas, \\
    y el Pandero, tapalapatán.	    & and the tambourine, tapalapatán.
\end{quotepoem}
The instruments in this list are simple, rustic noisemakers from everyday
peasant life.%
\begin{Footnote}
    Note that this source from Toledo spells the rattling sound \foreign{chas}
    while Gutiérrez de Padilla's manuscripts from New Spain spell it
    \foreign{chaz}.
    This reflects the different pronunciation of the Andalusian settlers of
    central Mexico, who according to phonetic spellings in the music
    manuscripts pronounced \term{ci}, \term{zi}, and \term{si} all with an S
    sound.
    In Toledo, the first two of those phonemes would start with an English TH
    sound.
\end{Footnote}
In this villancico these instruments, which are described further in the
coplas, join together with the sounds of the mule and other animals, and the
dances of the shepherds.  
This piece, like many villancicos, depicts a scene of common folk rejoicing
after their own fashion in the humble setting of the Bethlehem stable.
The focus here is not on instrumental performance in the present day but on
helping listeners imagine the sounds of the first Christmas---but one cannot
help speculating whether peasants in Toledo might have brought such instruments
with them into the church at Christmas.

Imitating an instrument, though, did not mean that the instrument was actually
used in church; indeed in many cases the situation seems to have been the
opposite.
Despite the fanciful reconstructions that can be heard in modern recordings, no
one has yet provided documentary evidence that percussion instruments were used
in church.
They appear only rarely in images of church ensembles or iconography of angelic
consorts (see \cref{fig:BMV-Montserrat,fig:Correa-Sacristy}), and known
archival records do not record payments to makers or players of these
instruments, the way they do for shawm, dulcian, organ, and harp.%
    \citXXX[evidence needed]
%}}}3

%{{{3 clarines
\subsubsection{Becoming Clarions}

Another common example of the imitative, intermusical type would be the many
pieces that mention the \term{clarín} (clarion or bugle), in which the
singers perform patterns that are meant to sound like brass fanfares.
The typical style of clarion evocations may be seen in two examples from the
archive of the Escorial, which holds much of the surviving repertoire of the
Spanish Royal Chapel.
Most clarín pieces do not actually feature written-out clarín parts; in most
cases the instrument is imitated vocally or by other instruments, like
\term{chirimías} (shawms).
Matías Durango's \wtitle{Cajas y clarines} (Drums and bugles) evokes these
instruments with voices and shawms in martial style, as part of a broader
battle topic.%
    \footnote{\sig{E-E}{Mús. 29/15}.}
Durango's clarín topic is strikingly similar to one of the rare surviving
clarín parts from a villancico, in a fragment by the prominent Madrid composer
Sebastián Durón (\cref{mux:durango-duron-clarines}). 
Both are in the same collection of music from the Royal Chapel preserved at the
Escorial.%
\begin{Footnote}
    \sig{E-E}{Mús. 29/15} (Durango), \sig{E-E}{Mús. 32/16} (Durón).
\end{Footnote}
A villancico by José Romero from about 1690, \wtitle{Suene el clarín} (Let the
clarion resound) includes an actual notated part for \foreign{los clarines de
los autos}, that is, for the clarions played in the \term{autos sacramentales}
or public Corpus Christi dramas.% 
\begin{Footnote} 
    \sig{D-Mbs}{Mus. ms. 2914}, edited in \autocite[655--661]{CaberoPueyo:PhD}.
\end{Footnote}
The sung voices layer bugle-like gestures above them, creating a more complex
fanfare than the valveless instruments could play on their own.

%{{{5 music: clarin in voice vs actual, durango/durón
\begin{musicexample}
    \label{mux:durango-duron-clarines}
    \caption{An imitation of \term{clarín} music by voice and shawm compared
    with an actual \term{clarín} part: 
    (1) Durango, \wtitle{Cajas y clarines} (\sig{E-E}{Mús. 29/15}, Tiple I-1,
    estribillo); 
    (2) Durón, \wtitle{Dulce armonía} (\sig{E-E}{Mús. 32/16}, estribillo)}

    \includeFloatPDF[\floatwidth][0.5\floatheight]{Durango-Cajas_clarines}
    \includeFloatPDF[\floatwidth][0.5\floatheight]{Duron-Dulce_armonia_clarin}
\end{musicexample}
%}}}5

Perhaps there are few clarín parts because these instruments may not always
have been allowed in church, or perhaps their music was generally improvised by
the class of Spanish community musicians called \term{ministriles}, similar to 
the Lutheran \term{Stadtpfeiffer}.%    
    \citXXX[stadpfeiffer, ministriles (Illari), Praetorius CD]
In any case, the instrument was more important as a symbol than as part of the
chapel ensemble.
The \term{clarín} was used in military, royal, and apocalyptic symbolism as far
back as the allegorical \foreign{clairon} fanfares in the 1454 Feast of the
Pheasant hosted by the ancestor of the Habsburg monarchs, Philip the Fair of
Burgundy.%
\begin{Footnote}
    \Autocites[340--380]{LaMarche:Memoires}{Bloxam:JNV}{Perkins:Patronage15C};
    on the symbolism of this instrument in contemporary Spanish drama, in which
    \term{Clarín} was the name of a comic stock character, see
    \autocite{Damjanovic:Clarin}.
\end{Footnote}
In \wtitle{No temas, no recelas} by another famed Madrid composer, Cristóbal
Galán (from \circa{1691}), the voices represent \term{clarín} music in a scene
of \quoted{heavenly armies} going to battle.% 
\begin{Footnote} 
    \sig{D-Mbs}{Mus.  ms. 2892}, 
    edited in \autocite[555--565]{CaberoPueyo:PhD}.
\end{Footnote}

Imitating the clarion within a battle topic was not always just a spiritual
symbol: it was often used like real bugle fanfares were, to celebrate military
victories, or boost morale in the midst of conflicts.
The anonymous villancico \wtitle{Noble clarín de la fama} states on the cover
page that it was performed \quoted{for the profession of the sisters
\foreign{Señoras} Sor Sagismunda and Sor Jacinta Perpinyà into the Convent of
Santa Clara of Gerona, 1693}.% 
    \footnote{\sig{E-Bbc}{M/772/35}.}
The surname of these siblings (sisters by blood and now by vow) is the name of
Perpignan, capital of the Catalan region of Rosselló, which had become the
French Roussillon after the Peace of the \XXX[Pyrennees] in 1659.
A long struggle over this border territory in the War of the Great Alliance
climaxed in the year this villancico was performed, as the French general
Catinat scored a major victory against the allied powers at Marsaglia.%
    \citXXX[history]
The villancico appears to align Catalan identity with the French cause, as it
praises the \quoted{Catalan Amazons, who have the name of Perpignan}, who
\quoted{seek today good protection for their defense in Francisco}---that is,
they look for protection both to Saint Francis, the probable patron of their
order, and to France.
In enlisting for spiritual battle with Francis, the estribillo suggests, the
sisters themselves are becoming clarions of war.%
\begin{Footnote}
    Excerpt from the estribillo: 
    \foreign{Noble clarín de la fama/ 
    que de vozes te alimentas,/
    toca, toca, alarma, alarma,/
    que dos niñas hoy son aliento
    de tu voz excelsa,/
    Catalanas amazonas,/
    de Perpiñan nombre tienen,/
    pues bella guardia en Francisco,/
    buscan hoy por su defensa,/
    cuidado serafines,/
    resuenen los clarines}.
\end{Footnote}

At this moment of commitment in these young women's lives, coinciding with a
political crisis, the concept of \emph{becoming} a clarion held more
significance theologically than the mere sound of the actual instrument would
have held.
This concept is realized even more completely through musical representation in
a villancico by Juan Hidalgo (1614--1685, composer of the first Spanish operas
for the royal court), \wtitle{Venid, querubines alados}.%
\begin{Footnote}
    \sig{D-Mbs}{Mus. ms. 2895}. 
    On Hidalgo's theater music, see \citXXX[Stein].
\end{Footnote}
In this chamber villancico or \term{tono divino}, the two voices sing that just
as the birds of the dawn are \term{clarines} celebrating the Blessed Virgin, so
too will their own voices become \term{clarines}
(\cref{poem:Hidalgo-Venid_querubines_alados}).
Hidalgo interweaves the two voice parts in rising fanfare gestures that
actually allowed listeners to hear the singers transforming their voices into
\term{clarines} (\cref{mux:Hidalgo-Venid_querubines}).

%{{{5 poem and music Hidalgo Venid querubines
\insertPoem{Venid_querubines_alados-Hidalgo}
{\wtitle{Venid querubines alados}, poem set by Hidalgo (\sig{D-Mbs}{Mus. ms.
2895}), copla 5}

\insertMusic{Hidalgo-Venid_querubines}
{Hidalgo, \wtitle{Venid querubines alados}, duo response at end of each copla}
%}}}5

These villancicos use the \term{clarín} as a metonym for music-making
generally, subsuming both theological and political aspects of the instrument
and its characteristic style as central to the notion of music itself.
Theologically, they summon a range of scriptural references to brass
instruments---the trumpets of King David's priestly musicians
(\scripture{IChr}{0:0}), the trumpets of the apocalypse
(\scripture{Rev}{0:0}; \scripture{IThess}{0:0})---where clarion-like
instruments were used in earthly and heavenly worship and served as signs of
God's divine authority and calls to attention at moments of God's judgment,
signals of divinely ordained seasons and times.%
    \citXXX[biblical trumpets]
At the same time, we must not overlook the obvious political significance of
clarions in the militaristic society of early modern Spain---meanings that were
also understood in Biblical terms.
Just as the biblical Jericho had fallen by the divine hand at the sound of the
trumpets of Gideon's army (\scripture{Josh}{0:0}), clarion fanfares heralded
the arrival of the Spanish monarch or viceroy, communicated to troops on the
battlefield, and generally proclaimed to the ears what flags and triumphal arches
conveyed to the eyes---the dominion and divinely sanctioned authority of the
Spanish king over his global realms.\citXXX[bible verses]

The clarion, put simply, was a sign of power.
Clarion-themed villancicos, then, depended on the instrument's signification of
power to proclaim and reinforce the sovereignty and authority of the Spanish
church and state.
Under the Spanish \term{padronado} (the Spanish monarch's self-appointed
guardianship by direct rule of the Catholic Church in his realm), church and
state power were aspects of the same governing authority, and together they
administered rewards and punishment in both temporal and eternal domains.
Spanish subjects were taught a theological concept of society in which the
world was stratified in a static hierarchy of types and stations of people, in
a way understood to be harmonious and divinely ordained.

The continued reiteration of power through music, and the conflation of
political and theological symbolism, could signify terror and oppression for
many people but also serve as a consoling reminder to others of the order and
stability of the world.
The right man was in charge, the clarion call announced: God was working
through the rulers of the world to govern his creation; common people were
protected and defended against the forces of evil.
That early modern Spaniards understood evil in the human forms of the Moor, the
Jew, the heretic Lutheran;
that they believed the divinely sanctioned order of society consigned
\soCalled{Indians} and \soCalled{Blacks} for roles of servitude; 
that they believed only males could hold authority and that
an inequitable distribution of wealth was part of the natural hierarchy---the
clarion call signified all this, too.
We can affirm that the \quoted{music of state} in the Spanish Empire served as
a \quoted{instrument of dominion}, even in some ways fulfilling functions that
may be seen as prototypes for later totalitarian propaganda, while also
acknowledging that many Spaniards and their colonial subjects actually believed
in the theological foundations of their political order and even actively
contributed to reinforcing the hierarchical power structure, including through
patronizing, performing, and listening to music.%
    \citXXX[Rodriguez, Sage, Rietbergen, Menache, etc]
%}}}4
%}}}3

%{{{3 dance, ethnic vcs
\subsubsection{Dance and Difference: \term{Jácaras} and Social Class}

Dance topics in villancicos provided another way for the genre to point beyond
itself to other kinds of music in society, and like clarion topics these
references both reflected and reinforced Spanish attitudes toward social
structure.
Many dances are explicitly named and often the text proclaims the villancico
itself to \emph{be} a specific kind of dance, including \term{zarabanda},
\term{jácara}, \term{guarache}, \term{danza de espadas}, and
\term{papalotillo}.%
    \citXXX[sources]
Only a few of these have been corroborated with other notated sources of dance
music, though some of those sources provide only schemata and can only be
reconstructed with a high degree of imagination.%
\begin{Footnote}
    Compare, for example, the elaborate instrumental dances recorded by Andrew
    Lawrence-King and the Harp Consort, \headlesscite{Lawrence-King:DancesCD}
    with the rudimentary notation of a few bars of chords and minimal strumming
    patterns in the source, \shortcite{Ruiz:Luz}.
\end{Footnote}
The question of whether performers or listeners actually danced in church is
another problem here, related to the question of whether or to what degree
performers staged the dramatic villancicos in the sacred space.%
    \citXXX[?]
There certainly was ritual dance on Corpus Christi in Seville and Valencia
Cathedrals, performed by the boy choristers known as \term{seises}.%
    \Autocite{Comes:Danzas}
At present I would speculate that like \term{clarín} pieces with no actual
clarion, dance references in villancicos are not evidence of actual dancing;
their purpose is to call to mind dancing that happened elsewhere and to make
use of the symbolic meanings of dance. % XXX
	
The \term{jácara} (also spelled \term{xácara} but always pronounced with a
guttural H sound) originated as a type of song and dance in Spanish theater and
street performances, typically recounting the deeds of ruffian outlaws in the
rough and sometimes bawdy language of the underworld (a comparison with rap
would not be inappropriate).%
    \Autocites{Torrente:Jacara}{XXX}
Juan Gutiérrez de Padilla included a sacred adaptation of the genre in every
one \XXX[check] of his Christmas villancico cycles for Puebla Cathedral from
1651--1659: these pieces herald the exploits of not a human \term{pícaro} but
the baby Jesus, adapting the outlaw language markers from the worldly genre to
sacred purposes.
Gutiérrez de Padilla's most well-known contribution to this subgenre is
\wtitle{A la jácara, jacarilla} from the 1655 cycle.
As he had done with his previous \term{jácaras}, the Puebla chapelmaster
borrowed the poetic text from the imprint of an earlier Royal Chapel
performance in Madrid (in this case, from the previous year).%
    \citXXX[pliego, Torrente book]
% TODO example
He puts this text to a variant of the same tune that he had used in his three 
preceding \term{jácaras} that survive and the same general style: the main tune
features a stepwise gesture ascending and descending motive
C\sh--D--E--F--E--D--C\sh{} harmonized with \musFig{5 3} chords on the first
and fifth degree of the first mode (to modern ears, this sounds like \term{i}
and \term{V} in D minor).
% TODO example
This matches the basic outlines of the improvised tune type reconstructed by
Álvaro Torrente for the secular \term{jácara} (secular as in worldly and
irreverent, in theme and performance venue).

Like the improvised model, Gutiérrez de Padilla's setting moves rhythmically in
triple meter (\meterCZ) with extraordinarily heavy use of syncopation and
sesquialtera.
As this composer developed this tune in each year's successive setting, he made
the rhythm and phrasing more complex each time.
% TODO examples
For a good portion of the estribillo in 1655, he creates what to current
knowledge is an unprecedented nine-minim pattern of three-measure groups: a
normal group of three minims is followed by a sesquialtera group with pulses in
three imperfect semiminims.
The melody in the coplas defies any attempt at regular rhythmic grouping. 
% XXX

Why would Gutiérrez de Padilla create such a complex rhythmic and polyphonic
setting to represent a dance with such common, even sordid origins?
First, the beginning of the text proclaims a specific intention to bring
contrasting worlds together.
% TODO quote
The contrast between \term{corte} and \term{villa} is between noble and common,
gentility and laborers, urban and rural, refined and crude---notably not sacred
and secular.
It is also a play on the term \term{villancico}, which comes from \term{villa},
and suggests an attempt to say something here about the function and meaning of
the genre as a whole.
Mixing the style and specific motives of a low-life ballad genre with the
techniques of learned counterpoint; in fact using compositional technique to
actually amplify the complexity of the oral source material perhaps
beyond what would more readily be improvised, certainly contributed to this
goal of mixing high and low elements of society.
Theologically the Christmas feast actually centered on the mixture of high and
low, as the infinite and all-powerful God had confined himself to the
vulnerable body of the tiny Christ-child (see \cref{ch:padilla-voces}).
Christ's birth in a feed-trough and his manifestation to lowly shepherds and
heathen magi were also understood in the Spanish context to elevate the dignity
of lower-class people, though typically in way that ultimately reinforced the
social hierarchy rather than challenging it.%
    \citXXX[al establo]
Compared to source material like the \soCalled{Frog \term{Jácara}}, which
catalogs sexual positions in explicit detail, Gutiérrez de Padilla's
representations of Christ as an outlaw---in one piece, a gunslinger from
\quoted{way up in Texas}---seem quite tame, but within the context of what New
Spanish worshippers could hear in church they must have brought some delight
and sense of play into the liturgy.%
    \citXXX[Torrente, playing cards; vc ex]
In the last copla of \term{A la jacara, jacarilla}, the singer tells the
Christ-child, \quoted{We will leave you here with these \term{principios de
romances}}, tipping off listeners who had not yet figured it out that the
preceding set of \XXX[no.] verses were all constructed from the first lines of
traditional \term{romance} ballads.
% TODO table
Here again is a popular practice (again comparable to hip hop) of rapid-fire
quotations riffing on existing texts and reshaping them into new meanings, but
written down and given a fully notated musical setting in a complex, highly
literate manner.

The trickster quality of the typical \term{jácara} hero may also explain the
cryptic texts and puzzling musical patterns: the \term{jaque} was often a
gambler, a swindler, and a quick draw, so the sacred \term{jácara} became a
site for poetic and musical trickery.
Later in the century, pieces called \term{jácara} did not always have the
distinct musical markers connected to the secular source traditions; but they
did retain this sense of playful ingenuity.
Mateo de Villavieja's \term{Jácara en anagramas} (\XXX[date, place], from the
Convento de la Encarnación in Madrid)
features a poetic text written in anagrams, such that the lines and phrases of
one stanza are shuffled to create the next.%
    \footnote{\sig{E-MO}{AMM.4261}.}
\XXX[details]
The music, too, is composed in anagrams: the voice parts are rotated for each
successive copla; as are the phrases. \XXX[details]

The reasons for Gutiérrez de Padilla's rhythmic play with triple meter may be
hinted in a \term{jácara} villancico poem by Manuel de León Marchante.
In \XXX[16XX], León Marchante wrote a set of villancicos for \XXX[Toledo]
Cathedral in which, after an introductory piece, each villancico represented
one of the seven liberal arts.
(The next year he would balance things out with a set on the \quoted{manual
arts}, including sailing, surgery, and blacksmithing.\XXX[check!])
It is fascinating how León Marchante pairs the conventional subgenres of
villancicos with each of the divisions of learning: geometry is a
\term{villancico de naciones} (an \quoted{ethnic} villancico, see below),
because one needs geometry to make maps and navigate\XXX[other examples].
Where does the \term{jácara} appear?
As \term{arithmetic}---because, León Marchante says in the villancico, it is
\quoted{governed by the rule of threes}.
Perhaps there is a connection here to Gutiérrez de Padilla's three-measure
groups of triple meter.%
    \footnote{Perhaps also to his use of the symbolic number 33 in his
    depiction of Christ as a card player, which I have proposed is a
    proto-jácara.}
The jácara, then, would be a game of numbers, celebrating the ultimate
trickster who hid divine identity inside a child's body\XXX[etc].

The cost of turning the jácara into a theologically signicant display of
wit and ingenuity, it would seem, is losing a connection to the lower-class
sources of the genre.
Sacred jácaras became yet another pious entertainment for the educated classes,
perhaps with a bit of added thrill by their association with ribald origins,
but increasingly losing any sense of crossing boundaries of \term{corte} and
\term{villa} in a way that would have had any meaning for residents of the
latter.
%}}}3

%{{{3 ethnic VCs
\subsubsection{Representing Ethnic Difference}

Metamusical references to traditional music-making of lower-class people
extended also to the depiction of ethnic difference.
There are villancicos that depict non-Castilian groups like Native Americans,
African people, Catalans, Frenchmen, even Irishmen, through caricatured
deformations of language and music.
What have come to be called \quoted{ethnic villancicos} were labeled in their
time as \term{villancicos de naciones} or by the name of the particular ethnic
type for that piece, like \term{gallego} (Galician), \term{gitano}
(\quoted{Gypsy}), \term{indio} (\quoted{Indian}), or \term{negro},
\term{guineo}, and similar terms for Africans.
Most of these pieces, and especially the \term{villancicos de negro}, refer
specifically to the characteristic music and dancing of these groups, often
naming their instruments and describing their motions.
The texts use some smatterings of foreign words but mostly ask the performers
to put on an accent in Spanish: in these caricatures the Gypsy ends all her
words with a Z (\foreign{Puez los trez zon Magoz,/ hombrez de ezfera}).
\begin{Footnote}
    \term{Vamos al portal gitanilla}, Imprint from Epiphany 1666, Zaragoza,
    Iglesia de El Pilar (\sig{E-Mn}{VE/1303/1}), later attributed to Vicente
    Sánchez, \headlesscite[203--204]{Sanchez:LiraPoetica}.
\end{Footnote}
The Black says when he should say R and J, drops ending S sounds, and
mismatches genders and cases (\foreign{Mi siñol Manuele, \Dots{} Sesu, \Dots{}
pluque son linda cosa}).%
\begin{Footnote}
    \term{Al establo más dichoso}, Music manuscripts by Juan Gutiérrez de
    Padilla of \term{ensaladilla} for Christmas 1652, Puebla Cathedral
    (\sig{MEX-Pc}{Leg. 1/3\XXX}), \XXX[WLSCM32].
\end{Footnote}
Villancicos about African characters also frequently feature nonsense
syllables, whether lullaby phonemes like \foreign{ro ro ro ro} and \foreign{le
le le le}, or apparent gibberish like \foreign{tumbucutú, cutú, cutú} and
\foreign{gulumbé, gulumbá} that tells us what African languages like Kikongo
sounded like to a Spanish ear.%
    \citXXX[al establo, other]
This type of piece represents Africans as always happily engaged in drumming
and dancing.%
    \citXXX[baker etc]

These pieces were created by Spaniards primarily for other Spaniards;
\quoted{black villancicos} are not really about depicting African identity but
are rather ways of constructing a Spanish one by reference to the Other.
Immediately after purging Iberia of both Moors and Jews, the Castilians had
been overwhelmed with encounters with new ethnic groups, languages, and
religions around the world; these pieces offered the potential to create an
imagined world in which all these groups were situated in their proper place
within a well-controlled social hierarchy.
These pieces may offer glimpses of the language and music of non-Spanish
groups, but only through a glass heavily darkened by racial prejudice and
deliberate caricature for the sake of humor and mockery; they further clouded
by the cultural distance from which modern observers must approach these
pieces.
With regard to the nature of musical references in \soCalled{ethnic}
villancicos, then, these pieces encompass a mixture of literal imitation (as of
percussion, and of the \soCalled{musical} sound of foreign languages) and
broader stylistic references (as, perhaps, to African musical styles, though no
one has yet demonstrated concrete evidence for a connection).
They also include more abstract references to music through the use of nonsense
words that, somewhat like solmization syllables (see below), symbolize and
enact music-making.
Like \term{jácaras}, ethnic villancicos grow increasingly conventionalized over
the years and more distant from the low-caste sources they grew from, so that
the \term{negro} character in one year's villancicos was much more similar to
the \term{negro} of the previous year's set that he probably was to any real
person of African descent.
And like \term{clarín} pieces, ethnic villancicos both reflected and reinforced
imperial Spain's power structure by projecting a theological vision of that
structure as divinely ordained and immobile.
That said, we must note that very few of these pieces have received any serious
scrutiny, especially their music; and that those that are known are not simply
racist caricatures like later minstrel shows in the United States.

Their discourse on racial difference must be understood within a Neoplatonic
theological concept of music and society, in which the lowliest elements in the
created world could lead a person to the knowledge of the highest.
Juan Gutiérrez de Padilla, who included a \quoted{black villancico} in most of
his Christmas cycles for Puebla, depicted the paradox of Neoplatonic thought
when in 1652 he had his black characters, described as Angolans in the piece
like Gutiérrez de Padilla's own slave Juan, say \quoted{Listen, for we are
singing like the angels}.
As the Angolans go on to sing a vernacular \term{Gloria in excelsis} in their
dancing triple meter, full of syncopations notated by coloring in the mensural
noteheads, above them suddenly enter the two boy soprano parts of the second
chorus, which have been silent until now, singing the same \term{Gloria} with
them---but in the white notes of duple meter, and quoting a plainchant
intonation.
The Angolans and the angels are brought together for a miraculous moment
through contrasting types of rhythmic movement in which the hidden harmony
between earthly and heavenly music is revealed.
The Angolans are in some ways depicted sympathetically, as instead
of the gold, frankincense, myrrh of the magi (one of whom was portrayed on
Puebla's high altar as a black African), bring the Christ-child the homelier
and more practical gifts: a potato, a toy, and diapers.
But nothing about the piece really exalts the African characters in any way
that would affect the lives of real Africans like Gutiérrez de Padilla's slave.

It is possible, though more evidence is needed, that the vogue for black
villancicos at Christmas and Epiphany was linked to the practice across the
Spanish and Portuguese Empires of \quoted{Black Kings} festivals, in which
confraternities of enslaved and free people of African descent elected a mock
royal court and paraded them around their city with music and dancing, usually
with military elements with origins in the Christian Kingdom of Kongo before
the start of slavery.%
    \citXXX[sources]
This possible connection does not mean that these villancicos express any real
African voice or viewpoint; rather, they tell us about the insecurities, fears,
and prejudices of Spaniards and may help us understand how they justified their
place in an unjust society by appeal to theology and aesthetic beauty.
Gutiérrez de Padilla's polymetrical Gloria fits perfectly with the image of
angels singing and dancing on the round painted high the new Puebla
Cathedral's high altar, hovering over the images of shepherds and magi greeting
the newborn Christ at the altar's base (see \cref{ch:padilla-voces}), and we
can imagine the theological aesthetics of this were some comfort to this
chapelmaster-priest and his peers; but they were no help to enslaved men and
women and, when such pieces are revived uncritically today, they continue to do
their historic work of reinforcing racial prejudice and appeasing the
consciences of elite (and typically white) listeners.

With regard to hearing and faith, ethnic villancicos and black villancicos in
particular enabled Spaniards to envision themselves as the rightful masters of
a society in which other groups were naturally subordinate; in other words what
they heard helped them believe in the rightness of the social order as governed
by the church. %XXX
Though their representations are purposefully distorted, they do suggest that
the Spanish elite accepted the coexistence of multiple languages and types of
music within society, and enjoyed sampling these exotic sounds through the safe
filter of their own caricatures.
%}}}3

%{{{3 VCs about VCs
\subsubsection{Villancicos about Villancicos}

%{{{4 overview
The conventions of the villancico genre itself become the subject of a special
type of self-referential villancico \emph{about} villancicos.
In one sense, the many pieces beginning \quoted{Listen} or \quoted{Pay
attention}, might be considered self-referential, since in these pieces the
singers usually announce something about the piece, as in the setting of
\wtitle{Oigan, oigan la jacarilla} by José de Cáseda, or the poem performed in
Montilla in 1689, \wtitle{Oíganme cantar una tonadilla}---\quoted{hear me sing
a tonadilla}.% 
    \begin{Footnote}
    Respectively, \sig{MEX-Mcen}{CSG.151}, \autocite[116 (no signature
    listed)]{BNE:VCs17C}.
    See \XXX[LeGuin:Tonadilla].
    \end{Footnote}
This posture rhetorical posture owes something to the genre's close
relationship with the psalms in Matins, which are filled with such
self-referential statements (\quoted{Sing to the Lord a new song},
\scripture{Ps}{97:0}; \quoted{Come, let us worship and bow down},
\scripture{Ps}{95:0}).
But it also draws on the comic and satirical elements of the Spanish minor
theater, the low-register plays (\term{entremeses}) performed between acts of
the more highbrow \term{comedias} by, for example, Lope de Vega and Calderón.%
    \citXXX[entremeses]
Similar to the Italian \term{intermezzi} skits of the eighteenth
century that were the cradle of comic opera, Spanish \term{entremeses} were
built out of formulaic scenarios and stock characters---many of the same ones
like Gil, Pascual, Bras, and Bartolo who appear in villancicos---and parodied
the conventions of the \term{comedia}.

The patrons and creators of villancicos developed well-reinforced expectations
in their audience not only for different types of villancicos, but also
possibly for the dramatic shape of the whole villancico cycle (such as in León
Marchante's cycles on the liberal and manual arts).
The surviving musical settings of complete cycles, such as those in Puebla by
Juan Gutiérrez de Padilla and those in Segovia by Miguel de Irízar
(\cref{ch:faith-hearing}), demonstrate a range of expected musical conventions
for each of these sbgenres.
A conjunction of markers in the poetic subject matter and in the poetic and
musical style would have signaled to listeners, \quoted{This is one about
angels}, or \quoted{Here come those silly shepherds}.
Miguel de Irízar's requests to his chapelmaster peers for more
\quoted{villancicos de chanza}---comic villancicos---suggests the need to fill
out each villancico cycle with certain types of pieces, mixing serious and
comic subgenres.%
    \Autocite[78]{Olarte:Irizar} 
When a villancico represents the performance of a villancico and commentary on
it, we are offered a glimpse of how people listened to villancicos.
%}}}4

%{{{4 ex: Anton Llorente
The anonymous villancico \wtitle{Antón Llorente y Bartolo}
(\cref{mux:Anton_Llorente}) presents two characters from a well-known
\term{entremés} with close links to Cervantes' \wtitle{Don Quijote}, who want
listeners to hear out their complaint about villancicos.
The villancico poem is found in a 1639 Christmas imprint from Toledo Cathedral
and an anonymous musical setting survives from the Convento de la Santísima
Trinidad in Puebla.%
\begin{Footnote}
    \sig{E-Mn}{VE/88/6}, \sig{MEX-Mcen}{CSG.014}.
    The title has been erroneously \quoted{corrected} in the Sánchez Garza
    catalog to \wtitle{Anton, Lorente y Bartolo} despite the clear double L in
    the manuscript (which was never used when a hard L sound was
    intended).\XXX[check]
\end{Footnote}
The more well-known stock characters Gil and Bras, they say, have held the
stage for too long:
\begin{quotepoem}
    Antón Llorente y Bartolo	& Antón Llorente and Bartolo \\
    trazaron un memorial	& drew up a complaint \\
    de que con los villancicos	& that with all the villancicos \\
    se han alzado Gil y Bras.	& Gil and Bras have gotten the spotlight.
\end{quotepoem}
Anton Llorente and Bartolo insist that they could make a good enough villancico
of their own if given the chance:
\begin{quotepoem}
    Si ha de sonar el pandero,	& If the tambourine is going to be played, \\
    solo Gil le ha de tocar,	& it is only Gil who ever plays it, \\
    y si ha de haber castañetas,& it if there have to be castanets, \\
    ha de repicarlas Bras.	& Bras is the one to rattle them. \\
    También acá somos gentes	& But here we are, we too are good fellows, \\
    y alcanzar podemos ya	& and we can even manage \\
    de un villancico un bocado	& a nibble of a villancico \\
    y un pellizco de un cantar.	& and a pinch of a song.
\end{quotepoem}
In the succeeding \term{responsión} section, the full eight-voice chorus joins
in endorsing the new characters and denouncing the old:
\begin{quotepoem}
    No quiero que me Brasen y que me Gilen 
    & I don't want them to Bras me or Gil me \\

    sino que me Llorenten y me Toribien. 
    & but only to Llorente me and Toribio me.
\end{quotepoem}

The anonymous musical setting for this embodies all the stereotypes of the
villancico genre, first encountered here in Gutiérrez de Padilla's \wtitle{En
la gloria de un portalillo}.% 
\begin{Footnote}
    One possible composer is the Seville Cathedral chapelmaster Fray Francisco
    de Santiago, whose setting of this text was cataloged as part of the
    now-lost library of King John (João) IV of Portugal (see
    \cref{ch:padilla-voces}), 
    \autocite[caixão 26, \range{no}{675}]{JohnIV:Catalog}.
\end{Footnote}
The piece is in highly accented triple meter (\meterCZ) with continual use of
sesquialtera, and it opens with a declamatory section for full chorus, followed
by a vocal solo that is then echoed in polychoral dialogue by the full
ensemble.
The text is both dramatized and symbolized by leaping gestures that leap from
voice to voice in points of imitation on \foreign{salten y brinquen} (jumping
and leaping).
These features may just constitute typical villancico style, or they may be
taken to \emph{represent} typical villancico style.
The highly conventional music casts the anticonventional text into relief while
also dramatizing the scene, since the piece is meant to portray Anton Llorente
and Bartolo performing a villancico.
This is a villancico, then, in the style of villancicos.

% TODO	See also in lafragua connection; uber-conventional villancico next to
% anticonventional one

% TODO is there a Lisbon Anton Llorente?

%{{{5 music Anton Llorente
\insertMusic{Anton_Llorente}
{Anonymous, \wtitle{Anton Llorente y Bartolo} (\sig{MEX-Mcen}{CSG.014}), first
stanza of introducción and beginning of responsión (Accompaniment omitted)}
%}}}5

As though the 1639 Toledo text were not self-referential enough, the creative
team at the cathedral followed up the next year with another villancico that
specifically referred back to \wtitle{Anton Llorente y Bartolo}.%
\begin{Footnote}
    \ptitle{Quejosos de la sentencia que dio el alcalde Pasqual}, in imprint
    from Christmas 1640 at Toledo Cathedral, \sig{E-Mn}{VE/88/7,
    \range{no}{2}}.
\end{Footnote}
The narrator says that the \foreign{Brases} and \foreign{Giles} were so
\quoted{frustrated by the sentence that Mayor Pasqual decreed against them last
Christmas}, that \quoted{they appealed to another one who was more learned}
(the \quoted{Mayor of Bethlehem} was another stock character in comic
villancicos).
Each one states his case for why he is needed at the Nativity, and Bras's
conclusion wryly sends up the conventionality of villancico poetry:
\begin{quotepoem}
Cuanto ha qué Belén lo es,	& As long as Bethlehem has been what it is, \\
y ha sido el portal portal,	& and the stable has been a stable, \\
a peligros de poetas		& where poets have been in danger, \\
ha sido socorro Bras.		& Bras is always there for aid.
\end{quotepoem}
The new mayor, in the name of keeping traditions, undoes the sentence of the
previous year, and the chorus rejoices, because without Bras and Gil it would
not be Christmas:
\begin{quotepoem}
Que me Brasen, y Gilen	& I wish them \\
quiero zagales,		& to Bras me and Gil me, lads, \\
porque no soy amigo	& because I am no friend \\
de novedades.		& of novelties.
\end{quotepoem}
The chorus, speaking here for the mayor's subjects in the community, affirms
the decision to keep their familiar Christmas characters:
\begin{quotepoem}
Porque en saltando a esta fiesta & For if you take from this feast \\
el pesebre, y el portal,  	 & the manger, the stable, \\
las pajas, Brases, y Giles, 	 & the straw, Brases, and Giles, \\
no es fiesta de Navidad.	 & it is no festival of Christmas.
\end{quotepoem}
Here we have a scene of people clamoring for villancicos with all their corny
conventions as a central part of making Christmas feel like Christmas.

As the mayor says, one reason villancicos were so conventional may be because
the feast they were most closely associated with was (and is) one where customs
are carefully preserved.
Novelty at Christmas is expected but only within certain traditional
boundaries.
Part of cultivating those customs meant naming them explicitly in song, as we
have already seen in Cererols's \wtitle{Fuera que va de invención} and several
other pieces, like a North American Christmas tree ornament in the shape of a
Christmas tree.
Comic villancicos like these should not be written off as less theologically
motivated than the more cultivated pieces.
Though the \wtitle{Anton Llorente} pieces present no learned theological
doctrines, they still serve a religious function in prompting hearers to
laughter and enjoyment, and that function contributed to the effect of a set of
villancicos within the liturgy.
The comic pieces may even have been the most likely to provoke direct responses
of laughter and surprise in many listeners, and therefore could be the most
effective in actually moving those assembled toward sympathetic vibration and
harmony together. 
They also served the practical goal of attracting and pleasing parishioners and
making them feel at home in the church---a purpose that might even be
considered more truly theological in the sense of fulfilling the church's
religious purpose as a faith community rather than projecting theological
concepts.%
\begin{Footnote}
    Contemporary Catholic theologian David Fagerberg argues that the first
    and foundational meaning of theology as an activity---what he calls
    \term{theologia prima}---is what the simple worshipper does during the
    liturgy.
\end{Footnote}
%}}}4
%}}}3
%}}}2

%{{{2 abstract references
\subsection{Abstract References to Music as Concept or Symbol}

%{{{3 overview
In the second category of metamusical villancicos are pieces that refer to
music more as an abstract concept, rather than to a specific, identifiable
reference to another kind of music.
When Pedro Ruimonte in \wtitle{Gil, pues a cantar} (one of the few villancico
settings to be published in print) sets the word \foreign{cantar} (sing) to a
long melisma, or when Gaspar Fernández in \wtitle{Sobre vuestro canto llano}
illustrates the phrase \term{canto llano} (plainchant) with imitative
counterpoint around a cantus-firmus-like Tenor part, these composers are using
the characteristic emblems of vocal music to refer to the concept of singing in
general.%
    \citXXX[ruimonte, fernandez]
%}}}3

%{{{3 solfa
\subsubsection{Solmization Puns and the Theology of Voice Alone}

One of the most common ways of explicitly singing about singing was to use
solmization syllables---\term{ut, re, mi, fa, sol, la}---in the poetry.
References to Christ as \term{sol} (sun) are ubiquitous, and as shown in the
opening example by Gutiérrez de Padilla, composers missed no opportunity to put
this word on a pitch that could be solmized with that syllable (G, C, or D in
the three Guidonian hexachords).
Solmization tropes brought the rudiments of musical artifice into the
foreground, forcing educated listeners to take note of the constructed nature
of what they were hearing.
No pun was too obvious: in composer Miguel de Aguilar's \term{oposición}---his
audition piece---for a position at Zaragoza, \wtitle{Mi sol nace y tiembla}, it
is not hard to guess the opening pitches he chose.%
\begin{Footnote}
    \sig{E-Zac}{B-11/233}, \ptitle{Villancico de Oposición en Zaragoza}, edited
    in \autocite[34--64]{Ezquerro:MME55}.
\end{Footnote}
Solmization syllables were sometimes used for their own sake, without a
symbolic meaning, somewhat like the \quoted{fa la la} refrains in contemporary
English madrigals.
Here sign and signified become one: the voice bears no message except the
musical voice itself.

Passages of self-conscious solmization are not alluding to a particular kind of
song; rather, their song is pointing to the abstract category of
singing in general.
In Aguilar's \wtitle{Mi sol nace}, the words have dual function: on one side
they communicate linguistic meaning (\quoted{my sun}, which is itself
metaphorical), but on the other side these musical syllables go beyond
language, to both symbolize and embody music-making. 
Aguilar made this obvious gesture at the beginning of a piece intended to
demonstrate his own skill at composition, in keeping with the tradition we will
trace in \cref{part:unhearable-music} of Spanish composers using metamusical
villancicos to establish their compositional pedigree.

At the same time, the syllables themselves could also take on deep symbolic
meanings, as in the Christological \quoted{sign of A \term{(la, mi, re)}} in
Gutiérrez de Padilla's 1657 \wtitle{Voces, las de la capilla} (the subject of
\cref{ch:padilla-voces}).
The syllables as an ordered series---that is, as a scale---served a symbolic
purpose with roots in medieval tropes like Mary as the \term{scala
peccatoris}.%
    \citXXX[Gilbert?]
Andrés Lorente began his 1672 music treatise with a dedication to the Virgin
Mary and a warning to the reader that moral virtue was the key to being a true
musician; he uses the six syllables of the hexachord as a ladder ascending to
heaven, where each syllable matches the start of a verse of Scripture
pertaining to a particular virtue, and as a second ladder descending to
hell\XXX[check], with each syllable denoting a particular vice.

The system of three hexachords offered additional symbolic potential.
Spanish music pedagogy preserved the widespread medieval system in which were
three groups of the six syllables from \term{ut} to \term{la}: the
\quoted{natural} hexachord (\term{cantus naturalis}\XXX[check]) starting on C;
the \quoted{hard} hexachord (\term{cantus durus}) starting on G and featuring
the \quoted{hard} or square B (ancestor of the natural sign \na{}); and the
\quoted{soft} hexachord (\term{cantus mollis}) starting on F and featuring the
\quoted{soft} or round B (ancestor of the flat sign \fl{}).
A singer moved or mutated between these three sets at certain prescribed
positions (for example, shifting from D as \term{re} in the natural hexachord
to \term{la} in the soft hexachord. % XXX cut all this?
The hexachords could thus be understood as transpositions of each other.
The practice of \term{musica ficta} allowed further transformations in order
to sing notes \quoted{outside the hand} (not included on the mnemonic
device of the Guidonian hand): in certain situations (e.g., going one note
above \term{la} and back again), one sang \term{fa} or \term{mi} instead of the
normal syllable to create a semitone where there would not otherwise be one.
This much music theory was part of the basic training of every cathedral
chorister and university-educated Spanish subject.%
    \citXXX[sources] % XXX distinguish cantus mollis as "key signature"

In 1678, Segovia Cathedral chapelmaster Miguel de Irízar began the festivities
of Christmas with the \term{calenda} piece, \wtitle{Qué música celestial}, in
which he used the hexachordal system to depict heavenly music coming down to
earth.%
    \footnote{\sig{E-SE}{18/36}.}
The piece dramatizes the moment when the shepherds of Bethlehem first heard the
music of the angelic choir (\scripture{Lk}{2:0}) by having three speakers ask,
in turn,
\quoted{What celestial music is this which alters the air?}
\quoted{What sovereign harmony is this which elevates hearing?}
\quoted{What light is this that transforms the dense night into day?}%
    \citXXX[wlscm32+]
Of course, as the first villancico of Christmas heard in Segovia Cathedral that
year, \quoted{this music} also refers to the music currently resounding under
the massive vaults of the last Gothic cathedral in Europe.
Irízar gives the first voice to a boy soprano, who sings down all the steps of
the soft hexachord from \pitch{D}{5} (\term{la}) to \pitch{F}{4} (\term{ut}).
% XXX check voice parts
The figure is an epitome of music itself, a textbook example of solmization
that begins at the very top of the Guidonian gamut (the second highest note on
the hand).
The second voice, then, imitates the first phrase exactly, but transposed down
a fourth into the natural hexachord (from \pitch{A}{4} to \pitch{C}{4}).

% START
% TODO music example

The shift from the \quoted{altered} soft hexachord, with B flat, to the plain
natural hexachord, symbolizes the movement of music from heaven to earth.
% XXX other examples of cantus mollis for heaven
Between the two singers, the two phrases outline the plagal ambitus of the
second mode (from A to A, with a final on D), thus presenting hearers with a
paradigm of perfect music, according to the most ancient of rules known to a
late-seventeenth-century Spanish chapelmaster.
Meanwhile, the bass line for the continuo accompaniment adds a further
heaven--earth contrast, as it moves in canon with the singers but with a
rhythmic displacement so that the bass and melody voices form a chain of 7--6
suspensions.
The way the bass voice moves at a delay from the solo voice suggests the way
earthly music imitates or echoes heavenly music.
On the other hand, the contrapuntal pattern is a textbook example of
fourth-species counterpoint, so it could also be a way of representing heavenly
music itself.  
This kind of heavenly music defies human expectations but is at the same time
governed by its own laws.
A listener untrained in counterpoint might only have perceived a mysterious,
haunting affect, and in any case the passage does evoke a \foreign{soberana
armonía} that \quoted{elevates the sense of hearing} or \quoted{lifts up the
ear}.

For a virtuosic demonstration of solfa composition we turn again to Juan
Gutiérrez de Padilla.
In a separate part of the archive from Gutiérrez de Padilla's other villancicos
or his Latin-texted polyphony is a small handwritten notebook containing an
anthology of exemplary works gathered by an unknown musician, apparently a
student to judge from the immature handwriting.
After selections from Palestrina among others is a set of \quoted{Villancicos
of various authors}, in which there is copied just the tenor part of a setting
by Gutiérrez de Padilla, \wtitle{Miraba el sol el águila bella}.%
    \citXXX[signature, check info, esp. attribution]
The copyist had good reason to preserve this part.
Here is a villancico with a text (for the estribillo and responsión) comprised
almost entirely of solmization syllables: \foreign{Cuando mire al sol, el sol
la mire} \XXX[etc, check].
Gutiérrez de Padilla sets every syllable to the corresponding pitches so that
for much of the piece, singing the lyrics is almost identical to singing the
solmization.%
\begin{Footnote}
    We will see a similar treatment of solmization in \cref{ch:padilla-voces}.
\end{Footnote}
% TODO example
% XXX avoid rep
But the piece is not nonsense---in fact, the poet (perhaps Gutiérrez de Padilla
himself) has managed to craft a semantically and theologically coherent text
based on an entirely separate conceit relevant to the feast of the Conception
of Mary, that of the Virgin Mary as an eagle.
The eagle, Spaniards believed, had the power to look directly at the sun
without harming its eyes, and thus the eagle was a fitting symbol for Mary as
Immaculate.%
    \citXXX[sources eagle]
That Mary was conceived without original sin was enforced as official dogma in
Spain long before it received papal approval in the first Vatican Council in
\XXX[18XX], and Puebla Cathedral was dedicated to Mary as Immaculate.
Gutiérrez de Padilla takes advantage of the hexachordal system, which means
that there are usually three possible notes that could be solmized with a
particular syllable, to add an additional symbolic layer embodying the eagle
conceit through musical technique.
To represent the eagle turning to the sun (and therefore Mary seeing the face
of God\XXX[?]), Gutiérrez de Padilla has the Tenor shift from the soft
hexachord, through a ficta alteration, into the natural hexachord. 
% TODO example
He thus quite literally moves \foreign{al sol}---both because he moves to a
note on \term{sol} and because he shifts to the hexachord that starts on the
\term{sol} of the previous hexachord.
Where Irízar shifted from the soft hexachord down to the natural for moving from
heaven to earth, Gutiérrez de Padilla makes the same shift upwards to represent
the eagle/Mary looking up to the heavens.
It is hard to imagine a more complete epitome of the concept of singing about
singing.

Even when solmization passages seem to lack lexical or symbolic meaning, they
bear a profound theological meaning as an embodiment of the voice itself,
within the Neoplatonic system (explained more fully below).
In the prevailing Catholic understanding of the world, every created thing,
simply by being itself, reflected the nature of God, its Creator.
The human body was the microcosm, reflecting in turn both the whole Creation
and the Creator who took on a human body in Christ.
The voice emanated from the body and expressed the essence of the one speaking
or singing to another who heard it.
So the voice itself had quasi-sacramental meaning as an expression of Man the
microcosm and a reflection of the Creator; and this meaning was independent of
linguistic communication or even of music's own non-linguistic structures,
which were understood by analogy to verbal rhetoric.
Far from \quoted{signifying nothing}, as in the \quoted{aesthetics of pure
voice} that Mauro Calcagno identifies in the contemporary Venetian theater
productions of the Accademia degli Incogniti (far less modern philosophical
notions of the \quoted{voice itself} as separate from meaning), wordless
passages in villancicos come nearer to signifying \emph{everything}.%
    \citXXX[Calcagno; {Barthes:GrainOfVoice}{Dolar:Voice}{Cavarero:Voice}] 
    %+ TODO Feldman ed volume
%}}}3

%{{{3 music itself as a conceit
\subsubsection{Music Itself as a Conceit}

Solmization villancicos should be understood as a subtype of a category of
villancicos in which music itself is the central conceit---not a specific type
of music (human, animal, or angelic), but music in the abstract.
Such pieces often play on technical musical terms using the technique of
\term{conceptismo} to create a double discourse about both music and theology.
The most renowned of villancico poets today, Sor Juana Inés de la Cruz
(1651--1695), used the conceit of Mary as a heavenly chapelmaster to create
such a piece for the feast of the Assumption in Mexico City, 1676, though no
musical setting survives.% 
    \Autocite[\range{no}{220}, \range{p}{7}]{SorJuana:VC} 
The estribillo exhorts congregants to listen for Mary's voice
(\cref{poem:Silencio-Maria-Sor-Juana}).
The coplas demonstrate how much theology could be drawn from musical terms, and
how much knowledge of both domains is necessary to understand both sides of the
concept.%
    \Autocite{Stevenson:SorJuanaMusicalRapports}.
% XXX relevant sor juan lit

% %{{{5 poem sor juana silencio
% \insertPoem{Silencio-Maria-Sor-Juana}
% {Sor Juana Inés de la Cruz, \wtitle{Silencio, atención, que canta Mariá},
% excerpts}
% %}}}5

As the succeeding chapters will show, when poetry like this was set to music,
composers had the opportunity to match this intricate musical-theological
discourse with another layer of symbols in the sounding music.
This conjunction of verbal and musical play on musical concepts was no
accident: the texts of villancicos were written specifically as lyrics for
musical compositions, as Juan Díaz Rengifo stated in one of the earliest
literary descriptions of the genre, and composers had every reason to favor
poems that gave them opportunities for clever musical crafstmanship.%
    \Autocite{Rengifo:ArteMetrica}
Juan Gutiérrez de Padilla (\cref{ch:padilla-voces}) and Joan Cererols
(\cref{ch:cererols-suspended}) both took finely wrought, Gongoresque texts with
musical conceits and added a rich musical commentary on those conceits in their
intricate settings.
Each chapter in \cref{part:unhearable-music} traces a family of related
villancicos with the same or similar texts and demonstrates that this type of
high-concept metamusical villancico served a special purpose for Spanish
musicians, enabling them to situate themselves within a tradition of
composition.
%}}}3

%{{{3 heavenly, angelic music
\subsubsection{Pointing to a Higher Music: Heavenly and Angelic Music}

Thus far we have seen how Spanish composers represented other kinds of music
like birdsongs, instrumental music, and dances of different social groups
within the villancico genre; and how they created songs that pointed to
themselves, whether by parodying the genre's own conventions or by using
solmization to draw listeners' attention to the act of singing itself.
How, then, did composers use villancicos to refer not to any kind of earthly
music, but rather to point to celestial (planetary) and heavenly (angelic,
divine) music?
When a villancico referred to the music of the spheres or to angelic music, the
music signified was impossible to hear with earthly ears, so the human music
would only function as a sign to the extent that a listener believed it to
correspond to what those higher forms of music sounded like, or understood them
to be mere imitations of something higher.
As we will trace in \cref{part:unhearable-music}, Spanish composers developed a
family of conventional tropes for evoking heavenly music.
One of these conventions was to set up a contrast between stylistic allusions
to distinct types of human music with different values in a hierarchy of
musical styles.
The most elevated form of earthly music, learned counterpoint in the by-then
classic style of Palestrina, was typically contrasted with more worldly types
of music, such as the rhythms of dance and the melody-and-accompaniment style
of popular songs.
This hierarchy of human musics was mapped on to the greater hierarchy of
earthly and heavenly music, so that old-style counterpoint stood in for
angelic and divine music, though really it was the contrast between musical
topics that gave it this meaning.
We have already seen how Juan Gutiérrez de Padilla contrasted the music of
Angolans and angels in his 1652 ethnic villancico, with the angels quoting
plainchant and singing in perfect species counterpoint.
This approach was used across confessional lines in early modern Europe,
with well-known Lutheran examples by Heinrich Schütz, Dieterich Buxtehude, and
J. S. Bach (all in music envisioning heavenly bliss after death).%
\begin{Footnote}
    \Autocites{Johnston:RhetoricalPersonification}
    {Yearsley:Buxtehude}{Yearsley:BachThron}.
    Lutherans saw the role of the boys' choir in relation to the congregation
    as analogous to that of the angelic chorus in relation to all human
    singers; see \autocite[\XXX]{Cashner:Gerhardt}.
\end{Footnote}

A typical example of the angelic trope is \wtitle{Angélicos coros con gozo
cantad}, a Christmas villancico by Antonio de Salazar that was part of the
repertoire of the Conceptionist sisters of the Convento de la Santísima
Trinidad in Puebla de los Ángeles.%
\begin{Footnote} 
    \sig{MEX-Mcen}{CSG.256}; edited in \autocite{Cashner:WLSCM32}.  
\end{Footnote}
Salazar (\circa{1650}--1715) was probably born in Puebla and may have sung in
the Puebla Cathedral chapel under Gutiérrez de Padilla; he served as
chapelmaster of Mexico City Cathedral from 1679.%
    \Autocite{Koegel:Salazar} % + DMEH
The convent collection features numerous pieces by Salazar, possibly composed
or arranged specifically for this community.% 
    \citXXX[catalog]
Puebla, the original American \quoted{city of angels}, was built on a site
believed to have been revealed by angels to the bishop of Tlaxcala, and
buildings and artworks dedicated to the angels, especially Saint Michael the
Archangel, are everywhere in the city.

The anonymous poem echoes the first Responsory of Christmas Matins
(\quoted{Gaudet exercitus Angelorum}) as it invites the choirs of Christmas
angels to sing their \quoted{Gloria} over the stable in Bethlehem on the night
of Christ's birth.
Since \mentioned{Bethlehem} in Hebrew means \quoted{House of Bread}, the
villancico also celebrates the sacramental presence of Christ in the
Eucharistic host on the Christmas of Salazar's \soCalled{present day}.
Though the words speak to the angels, the musicians who sing these words also
play the part of the angels, so that hearers are invited to listen for the
angelic voices \emph{through} the voices of the church ensemble. 
The invocation to the angels is sung first by the Tiple I, in a gesture
beginning with a rising fifth and then falling by step, as though looking up to
the heavens and then following the angels' descent.
In the Puebla convent choir, this part was performed by \quoted{Madre Andrea},
whose name is written into her part.
As though answering the call, the other two voice parts of Chorus I enter in
\measures{2}, Tiple II in canonic imitation, and Alto I harmonizing with it
homorhythmically. 
In \measures{4-5} the second chorus joins with a similar imitative
pattern, until all join together in a lilting, dancelike cadence on
\foreign{cantad}.
Salazar uses contrapuntal imitation again on \foreign{celestes esquadras},
inverting the opening motive (\measures{14-22}).
For the command \foreign{bajad} (come down), Salazar switches from CZ
triple meter to duple (C or \term{compasillo}), and creates a cascading
contrapuntal passage passed from voice to voice, moving from high F\octave{5}
down to C\octave{3} (example~\ref{ex:Salazar-Angelicos_coros-2}).
The general affect of the piece seems gentle and sweet, partly because of the
largely static diatonic harmony and the lilting or dotted rhythms.

%TODO fix bar numbers

%{{{5 example Salazar
% TODO
%}}}5

All of these musical characteristics are typical ways that villancico composers
represented angelic music: especially contrapuntal imitation, in a reference to
the ordered music of heaven, and symbolic patterns of ascent and descent.
Salazar uses different styles of earthly music---particularly the contrast
between contrapuntal and homophonic styles---to point to the contrast between
different levels of music on a cosmic scale, between human music, music of the
spheres, and angelic song.
Because the triple-meter style of the first section, which asks the angel
choirs to sing, is more typical of villancico style, this part might be heard
to represent the actual singing of the angels.
The duple-meter section on \quoted{bajad} might be understood as a more literal
portrayal of the angels themselves.

The piece connects Boethian \term{musica instrumentalis} to the higher forms of
human and cosmic music. 
\soCalled{Heavenly} villancicos map a lower level of music onto a higher one
within the Neoplatonic cosmos, in which the perceptible \soCalled{world of
change and decay} is an imperfect reflection of a higher world of ideal forms.
Thus earthly music of any kind, metamusical or not, would always point beyond
itself to higher forms of music and ultimately to God.
Metamusical pieces intensify this aspect of music by calling the listener's
direct attention to the artifice of the music itself.
Such pieces give listeners the opportunity to rise in Neoplatonic contemplation
from what is heard by the ears to the higher music (ultimately of the divine
nature) that can only be discerned by the soul through faith.

This contemplative or meditative approach to music was favored among the
Spanish elite.
One venue in which this kind of listening was favored was in meetings or
sponsored services of religious confraternities.
Salazar belonged to the Confraternity of Saint Michael the Archangel.
A collection of printed sermons by the Carmelite preacher Fray Andrés de San
Miguel of Puebla includes a sermon Fray Andrés was invited to preach at a
gathering of Salazar's confraternity.
The title of the sermon explicitly mentions Salazar as having commissioned the
sermon, and this suggests that Salazar may have been head of the confraternity
as well. 
In the opening of the sermon, the preacher mentions that they are gathered in
the \quoted{church of the Encarnación}---possibly the Augustinian monastery in
Puebla.
The monk's self-deprecating introduction makes it clear that he is addressing
an elite congregation of highly learned and accomplished men.
The cleric expresses his concern that he does not really know enough music to
be addressing such a group, which suggests these were men specifically educated
in music.
%  \X cite source
The preacher's praise for Salazar (though expected if Salazar was responsible
for the paid commission) indicates a high level of respect and appreciation for
this chapelmaster.
He goes on to imply that there will be a musical performance at the same
liturgy, in which the audience will be able to hear this for themselves.
In fact, Fray Andrés says that Don Salazar could compose a better sermon in
music than he himself could preach in words. 

The sermon that follows is an exposition of the identity and deeds of
Saint Michael.
The friar structures this discussion not according to the verses or portions of
a Scriptural citation (as was more common), but according to each syllable of
the Guidonian hexachord.
Since Michael's name in Hebrew means \quoted{he who is like God}, or
\quoted{Quis ut Deus} in Latin, the friar uses UT to discuss Michael's name.
Since Michael is the chief of all the angels, he is their \quoted{king} or RE;
and so on.
As the preacher had warned, his musical knowledge does seem to have been rather
thin, since he does not use any other musical terminology or musical metaphors
in the sermon. 
The sermon is about angels, not about music; but it uses the terminology of
music as a framework to discuss the angels.

We do not know what music may have been performed at this service (or even what
kind of liturgy it was), but Salazar's \wtitle{Angélicos coros} would seem to
be the type of music that might have been chosen.
The version surviving in the Colección Jesús Sánchez Garza is arranged for the
sisters of the Convento de la Santísima Trinidad in Puebla---another
semi-private music venue with elite, high-level musical performance.
This is not Salazar's most learned composition, or even his most metamusical,
but it does allow us to imagine how themes of music were treated in villancicos
performed in closed and private spaces for circles of musical and theological
connoisseurs.
Villancicos about angels have a strong Neoplatonic charge to them: the singers
turn their attention heavenwards to address the angels directly (since angels
were believed to be present at every liturgy, \bibleverse{ICor}(11:10)), while
they also stand in for, or sing along with, their heavenly counterparts.
Angel pieces exhort the audience to lift their ears upwards as well, and ascend
in the chain of contemplation beyond even \emph{musica mundana} to the music of
the \emph{cielo Empyreo} in the highest Heaven.
%}}}3
%}}}2
%}}}1

%{{{1 theology of music
\section{Theological Listening in the Neoplatonic Tradition}

%{{{2 intro sources luis, kircher, augustine, boethius

% XXX apologize for long quotations? or cut them?
Villancicos on the subject of music consistently manifest a Neoplatonic
theological worldview, an understanding of which is necessary to grasp the
genre's religious functions.
Having drawn out aspects of this theology inductively from the examples of
metamusical compositions, in the final section of this chapter we may turn to
theological literature to establish a more systematic foundation.
The sixteenth and seventeenth centuries in Spain there was a revival of
interest in Neoplatonic and Augustinian thought.%
    \citXXX[revivals]
Augustine was by far the most influential theologian for early modern
Catholics, not only in Spain: his works were directly available in printed
editions starting early in the sixteenth century and reissued and re-edited
many times after, and through many compendia and digests of patristic theology;
and his ideas infused every genre of theological writing (see
\cref{ch:padilla-voces}).
In theological writing, the Dominican Fray Luis de Granada was one of the
foremost proponents of Christian Neoplatonism in the Augustinian tradition, in
copious writings that circulated so broadly throughout the Spanish Empire that
he may have been the most widely read author in those realms.%
    \citXXX[readership]
Because his work is a self-acknowledged synthesis of patristic and Classical
sources, his writings may be taken as both representative of widely held
beliefs of his own time and after, as well as a guide to how earlier sources
were read and understood by early modern Catholics.

Fray Luis's \wtitle{Introducción del Símbolo de la Fe} (Introduction to the
Creed) of 1589 presents a theological interpretation of the created world
within this tradition.
His introduction to the first article of the Apostle's Creed, \quoted{I believe
in God, the Father almighty, Creator of heaven and earth}, is really a fulsome
exposition of a theology of the created world.
In the Neoplatonic tradition, Fray Luis teaches that the natural world is a
reflection of a higher truth---God's own nature---and that the creation was
given so that by reflecting on it people would come to know its Creator.
Fray Luis frequently uses musical metaphors to describe the harmonious workings
of the created world, and he includes a discussion of the physiology and
theology of the human voice that applies directly to a historical understanding
of music.

Writers on music drew on Neoplatonic traditions as well, especially as
articulated by Boethius.
The treatises used to teach musical composition in seventeenth-century Spain,
most notably Pedro Cerone's \wtitle{El melopeo y maestro} (1613) and Andrés
Lorente's \wtitle{El porqué de la música} (1672), present music within a
Boethian cosmology of music, which has its roots in a Neoplatonic-Augustinian
tradition.
A key source for Neoplatonic thought on music is the encyclopedic
\wtitle{Musurgia universalis} by another great synthesist of received wisdom,
Athanasius Kircher.%
    \Autocite{Kircher:Musurgia}
The Jesuit polymath's 1650 work was disseminated through Jesuit networks across
the globe: a copy was sent as far as Manila, and two copies are preserved today
in Puebla.
Kircher describes in detail the latest scientific knowledge about the anatomy
of hearing and vocal production and the physiology of bodily humors and
affects; and lays out specific examples of how particular musical structures
work through these bodily systems.
Kircher presents a cosmic view of music according to Neoplatonic traditions of
theology and music theory, in which the whole universe is encompassed in the
\quoted{working of music}---a rough translation of his inventive
Greek-and-Latin title.%
        \citXXX[secondary Kircher lit]

The writings of Fray Luis de Granada and Athanasius Kircher provide the basis
for a provisional historical theology of music within the Neoplatonic
tradition. 
The fundamental concepts of this theology of music are the Neoplatonic chain of
being and the Boethian three-fold division of music.
In brief, Christian Neoplatonists followed Augustine in viewing the material
world as a reflection of a higher spiritual reality which ultimately had its
source in the Supreme Good which was the Godhead.%
\begin{Footnote}
    An important later source for this concept is the \wtitle{Spiritual
    Hierarchy} attributed to Dionysius the Areopagite.
\end{Footnote}
The material world reflected higher truths only imperfectly, but nevertheless
this world was also the only means through which those truths could be reached.
In connection with Catholic sacramental theology, material objects and physical
actions became means through which humans could encounter divine grace.
Neoplatonic contemplation could be understood as a dialectical process of
discerning the degree both of similarity and of dissimilarity between earthly
objects and heavenly truth.

The famous passage in Augustine's \wtitle{Confessions} (10:23) in which he
expresses his conflicted feelings about music in the church, confessing that he
was often moved \quoted{more by the singer than by what was sung}, is not a
condemnation of music and certainly not a statement that music's spiritual
value comes only from its words.
Rather, Augustine as the master of a Neoplatonic theology of sign and signified
(see \wtitle{De doctrina Christiana}) is upbraiding himself for failing to
follow the sign (the song) to what it signified (the spiritual meaning of the
song), within a conception where everything is ultimately a sign pointing to
God.
He was captivated by the song as an object, that is, as an idol, rather than by
the \quoted{holy thoughts} or \quoted{sentiments} that the song was meant to
communicate to him and move him towards. 
He failed at the task of Neoplatonic-Christian listening by getting stuck at
the lowest level of sensory experience, by letting carnal pleasure lead his
mind rather than the reverse.

The definition of music in Boethius's \wtitle{De musica} provided the classic
formulation of how music fit into the Neoplatonic chain of being
(\cref{tab:Neoplatonic-hierarchy-music}).
The three Boethian types of music are arranged hierarchically and each one
points beyond itself to a higher level.
At the lowest level is \term{musica instrumentalis}---music played and sounded,
music that humans can hear.
Higher up is \term{musica humana}---the harmony of body and soul, and of one
human being with another in society.
Still higher is \term{musica mundana}---the harmonies created by the perpetual
movement of the planetary spheres.
Villancicos on the subject of music embody the notion that even these three
levels of music are subordinate to the supernatural forms of music in
Heaven---the chorus of angels and saints, and above them, the mysterious
harmonies of three in one in the Trinity and two in one in the divine-and-human
nature of Christ.
The three Boethian musics in this system would all be \soCalled{worldly} music,
and it is important to clarify the distinction between the music of the
\foreign{cielos} or heavens---that is, the planetary spheres---and the
\soCalled{heavenly} music of the \foreign{cielo Empyreo} or Heaven, the
supernatural realm beyond the material world.
\term{Musica instrumentalis}, then, though the lowest form of music in the
chain of being, was the only form of music to which humans had direct access
through the sense of hearing.
Metamusical villancicos explicitly emphasize the challenge that was central to
all music-making in the Christian Neoplatonic tradition, to use the imperfect
medium of sounding music to evoke all the higher forms of music, to lead
listeners in contemplation up the chain of being beyond simply what was heard.

% %{{{5 table Neoplatonic hierarchy
% \insertTable{Neoplatonic-hierarchy-music}
% {Hierarchy of types of music in Neoplatonic thought, after Boethius}
% %}}}5
%}}}2

%{{{2 hearing book of nature
\subsection{Hearing the Book of Nature Read Aloud}

Fray Luis de Granada begins his \wtitle{Introduction to the Creed} with an
epitome of Neoplatonic-Augustinian thought: \quoted{The ultimate and highest
good of man}, he writes, \quoted{consists in the exercise and use of the most
excellent work of man, which is the knowledge and contemplation of God}.%
    \Autocite[182]{LuisdeGranada:Simbolo}
The created world, he teaches, is a \quoted{book of nature} in which is written
the grandeur, love, wisdom, and faithfulness of its Creator.
The first goal of humankind, then, is to learn to read this \quoted{book of
nature} in order to come through it to the knowledge of God. 
The goal of contemplating creation is \quoted{ascending by the staircase of the
creatures to the contemplation of the wisdom and beauty of the Maker}.%
    \Autocite[184]{LuisdeGranada:Simbolo}

The reason one can \quoted{read} God through nature, Fray Luis teaches, is that
the created world is a reflection of God's perfect order---a concept the friar
repeatedly expresses using musical metaphors.
Fray Luis compares the perfect order of nature to a harmonious musical
composition in which everything fits together \foreign{con sumo concierto}
(with the most perfect concord).
All the created things in this world, Fray Luis writes, \quoted{like concerted
music for diverse voices, harmonize together [concuerdan] in the service of
man, for whom they were created}.%
    \Autocite[191]{LuisdeGranada:Simbolo}
The movement of the heavenly spheres, and their effects on the earth, are like
a great \quoted{chain, or, it can be said, this dance, so well ordered, of the
creatures, and like music for diverse voices \Dots.
Because things so diverse could not be reduced to a single end with a single
order, if there were not one who was like a chapelmaster \add{maestro de
capilla}, who reduces them to this unity and consonance}.%
    \Autocite[191]{LuisdeGranada:Simbolo}

Everyone who compares something to music has some kind of earthly music in
mind.
When Fray Luis compares the world to music \quoted{in diverse voices} he
obviously has in his \quoted{mind's ear} polyphonic music of his own time, such
as he would have heard at the Portuguese Royal Chapel as confessor to the
queen.%
    \citXXX[names of composers]
Likewise, when he compares God to a \foreign{maestro de capilla}, that has all
the implications of that office in the Iberian context, which included
composition, teaching, and leading the choir in some form of conducting.%
\begin{Footnote}
    Compare the conceit of Christ as chapelmaster in \cref{ch:padilla-voces}).
\end{Footnote}
According to Fray Luis's metaphor, then, God is creator, prime mover, and
sovereign ruler over creation, actively and intimately involved in its ongoing
progress.

At the same time, these references to music are more than metaphors: the
universe for Neoplatonists is not only like music, it actually is musical in
its structure.
For Fray Luis, not only does creation reflect God's order; it actively
proclaims that fact.
It speaks or sings with its own voice to communicate God's glory to the human
who knows how to listen.
Paraphrasing Augustine's preaching, the friar writes: \quoted{Look around at
all these many things from the heaven to the earth, and you will see that they
all sing and preach their Creator; because all types of creatures are voices
\add{or, utterances} that sing his praises}.% 
    \Autocite[185, glossing Augustine's commentary on Psalm 26]
    {LuisdeGranada:Simbolo} 
While the full knowledge of God can only come with the aid of divine revelation
through the Scriptures and the Church, Fray Luis praises God that humans can
study his nature in \quoted{the university of created things, which declare to
us \add{literally, \quoted{give us voices}} that you love us, and teach us why
we should love you}.% 
    \Autocite[186]{LuisdeGranada:Simbolo}
Fray Luis acknowledges, however, that apart from angels and birds, most of
creation is mute and does not literally have its own voice with which to
communicate its message of divine glory.
This \quoted{message} is not a linguistic one, but rather, their message is
simply themselves: in the created world, the medium is the message.
\quoted{Now these admirable works do not speak or testify this with human
voices \Dots}, Fray Luis writes, \quoted{rather their speech and testimony is
their invariable order and their beauty, and the artifice with which they are
so perfectly made, as though they were made with a ruler and plumb line}.%
    \Autocite[192]{LuisdeGranada:Simbolo}

In this theological system, music has unique value because it actually provides
a voice through which creation can make audible its message-of-being.
If nature was a book, for a thinker in 1580s Spain, then it was meant to be
read out loud.
As Margit Frenk has documented, books in this period were not read silently,
but required someone to give them voice, and tomes like Fray Luis's devotional
books were written with that intention.%
    \Autocite{Frenk:Voz}
One edition of Fray Luis's own book \wtitle{Doctrina Cristiana} was published
with a notice at the beginngin from the Archibishop of Toledo granting a
certain number of days of indulgence for each paragraph that anyone
\quoted{read or heard read} (that is, had read to them).%
    \citXXX[signature]
To read the \quoted{book of nature}, therefore, someone must perform it
vocally---and this is what music could do.
In the Christian Neoplatonic tradition, human music unlocks the musical voice
contained within the substance of created things.
Through metal pipes, horns, and bells; through wood viol cases, gut strings,
and skin drums; even through reverberant stone church walls, the very matter of
creation is made to resound with the perfectly ordered mathematical-harmonic
proportions placed within it by the Creator---proportions which themselves
reflect God's own perfect order.
%}}}2

%{{{2 voice microcosm
\subsection{Voice as Expression of Man, the Microcosm}

If pipes and strings testify to the order of creation, then the human body as
the microcosm of creation is the ultimate instrument through which nature is
given voice.
Fray Luis concludes his exposition of the six days of creation (based largely
on the \emph{Hexameron} of Saint Basil) by saying that God's creation of man on
the sixth day was like the conclusion of an oration, when the speaker draws
together all his themes into a final epitome.
Thus man is the summation of all that God had created in the previous five days
and encompasses them all within himself.% 
    \Autocite[243]{LuisdeGranada:Simbolo}

When Athanasius Kircher (in the tenth book of the \emph{Musurgia}) continues
this hexameral tradition with his own treatment of the six days of creation, he
replaces the rhetorical metaphor with a musical one.
Instead of creation being God's oration, Kircher presents it as a musical
improvisation (a \quoted{Praeludium}) on God's cosmic organ (see
\cref{ch:cererols-suspended}).%
        \Autocite[\range{vol}{2}, 366--367]{Kircher:Musurgia}
On the sixth day, Kircher says, God recapitulates all his themes and pulls out
all the stops by creating man.
Here again, the comparison to music must be based on some actual music known to
Kircher; his description closely resembles the structure of a Praeludium by the
likes of Dieterich Buxtehude, which develops a motivic kernel through various
sections and culminates in a fugue for the full organ.%
    \citXXX[check dates, cite rhetorical analysis of buxtehude]
In Kircher's worldview, all the systems and elements of creation (stars,
planets, humors, rocks, animals, and so on) intersect in the individual human
body.% 
    \Autocite[vol. 2, 402]{Kircher:Musurgia}

For Kircher, the human voice is the unique expression of the individual,
reflecting each person's unique temperament and blend of the four humors.%
    \Autocite[\range{vol}{1}, 23--24]{Kircher:Musurgia}
Kircher defines the voice thus: \quoted{The voice is a living sound \add{or,
sound of the soul}, produced by the glottis through the percussion of respired
breaths that serve to express the affects of the soul}.% 
    \Autocite[\range{vol}{1}, 20]{Kircher:Musurgia}
Since each voice is unique, only in concert do voices fully reflect nature and
nature's God.
Cantus, Altus, Tenor, and Bass parts provide a place for all types of human
voices, Kircher explains, and correspond respectively to fire, air, water, and
earth.
Thus they form a choral microcosm both of humanity and of all creation.%
        \Autocite[\range{vol}{1}, 217--219]{Kircher:Musurgia}

Fray Luis also ventures an explanation of the human voice in both musical and
theological terms.
He exalts the voice as the audible expression of the human body and
vocal music as the most perfect kind of music.%
    \Autocite[243]{LuisdeGranada:Simbolo} 
Fray Luis praises the human voice as the highest of all musical instruments
(indeed, as the paradigm for them), as a means of forming social relationships
between people, and as a form of communication between human and divine:
\begin{quoting} 
    The lungs also serve to create the voice, because, when the air that they
    exhale leaves them with a great impetus, and touches the voicebox or
    \quoted{little bell} that we have at the entrance of the lungs, the voice
    is formed. \Dots{}
    But here it is to be noted that the mouth of the pipe coming out of the
    lungs is neither large nor round, but is drawn tight \add{hendida} just like
    the slot of an alms box,
    which opening serves to form the voice; this is why the mouths of flutes
    and dulcians are constructed in this fashion, because in this manner, the
    compressed air entering through them, the voice is caused.%
    \Autocite[252]{LuisdeGranada:Simbolo} 
\end{quoting}
Flutes and dulcians would be the example ready at hand for the friar because
these instruments were commonly played in the Iberian church music he knew.
But unlike a wooden flute, he says, the voice can take on any shape needed and
is as unique as each person's body.
The voice therefore expresses human individuality, and voices of different
types in concert enact harmony between people: 
\begin{quoting} 
    Moreover, here is a thing worthy of much consideration, to see the
    omnipotence and wisdom of the Creator, who was able to form something like
    a flute from flesh, which serves for singing.
    For to make a flute or trumpet from a solid material such as wood or some
    metal, is not much; for the hardness of the material serves for the
    resonance of the voice.  
    But to  make this out of flesh (such as is the windpipe of the lungs), and
    such that through it some voices are formed of women and of men, so sweet
    that they seem more like those of angels than of humans, and these with
    such variety of notes \add{punctos}, without having the finger holes of
    flutes that provide this variety, this is something that declares the power
    and the wisdom of that sovereign artisan, who in such a manner forged the
    flesh of this windpipe so that in it could be formed a voice sweeter and
    milder than that of all the flutes and instruments that human industry has
    invented.

    And there is no end of admiration for the variety that there is in this for
    the service of harmonious music \add{música acordada}.  
    For some throats are narrow, in which are formed the trebles \add{tiples},
    and others in which are formed voices so full and resonant that they seem
    to thunder through an entire church, without which there could not be
    perfect music.

    All of which that divine presider traced and ordained, so that with this
    mildness and melody the divine offices and their praises should be
    celebrated, with which to awaken the devotion of the faithful.%
        \Autocite[252]{LuisdeGranada:Simbolo} 
\end{quoting}

Fray Luis wants his readers to hear God's glory reflected most fully in the
concerted harmony of diverse human voices, which he says were created for the
purpose of singing of singing in divine worship.
The voice in church is the definitive example of vocal music for Fray Luis.
Sacred polyphony glorifies God, then, simply by realizing the potential for
which the voice (and the body) was made.
In the above passage, the sound of the voice alone proclaims God's power and
wisdom just in itself, apart from whatever words or musical figures it might
articulate.

Fray Luis sees speech as something \quoted{added} to the voice, which makes it
possible for the voice to communicate and form social relationships:
\begin{quoting} 
    Now here it is to be noted that when to the voice which proceeds from this
    place is added the instrument of the tongue, we come to articulate and make
    distinctions with this voice, and thus is formed speech, serving us by this
    instrument and punctuating \add{hiriendo} with it sometimes in the teeth
    and other times in the interior of the mouth.
    And just as the flute produces different sounds by touching on different
    holes, likewise the tongue, touching in different parts of the mouth, forms
    different words.  
    By this manner the Creator gave us the faculty to speak and communicate our
    thoughts and concepts to other men.% 
        \Autocite[252]{LuisdeGranada:Simbolo}
\end{quoting}
Fray Luis might see music---with its own system of articulations and
distinctions---as another way to \quoted{communicate our thoughts and concepts}
just as well as spoken language, but he also presents music as a product of the
voice before any articulation is added.
This definition of voice would mean that in vocal music there are always two
layers---the articulated \quoted{speech} aspects, and behind these the wordless
sustained voice.
Citing Augustine's \wtitle{De doctrina christiana} (the classic exposition of
Christian preaching and teaching), Fray Luis---who was himself the author of
six volumes about \wtitle{Rhetorica ecclesiastica}---says that the main task of
the student of rhetoric is to hear and identify the rhetorical tropes and
techniques used by another orator.
In the same way, he says, the first task of humankind is to be a student of the
natural world, and to learn to recognize in creation the signs of God's
artifice as the Creator, which manifest his glory.

This would mean that the hearer of music could and should seek out this level
of musical structure while listening.
In a polyphonic vocal piece like a villancico, the bulk of musical structure is
borne by the sustained tones of the voice, singing vowels.
Apart from the words being sung, musical elements like mode, meter, motivic
development, and stylistic or topical allusions are all communicated by these
musical voices, and not simply by the voice as the bearer of words.
Music could thus reflect the divine through its sonic structure, apart from any
sacred linguistic meaning that may be attached as well.
If music's value and sacredness are not comprised solely in the words being
sung, then one must know how to hear the musical structure in order to receive
the full benefit.

Listening to music within a Neoplatonic worldview, according to these
theological sources, may be summarized as follows:
\begin{enumerate}
\item Music is a reflection of the natural order.
\item The natural order is itself a reflection of God.
\item By paying attention to nature one can come to know and believe in its
    Maker.
\item Therefore listening to music may be a primary way of \quoted{reading the
    book of nature} and coming to faith in nature's Creator.
\item Music, especially vocal music, conveys sacred meaning to those who know
    how to listen, even apart from words.
\item The performance of music actively creates concord in society and between
    people and God.  
\end{enumerate}

%}}}2
%}}}1

%{{{1 conclusions
\section{Conclusions: Musical Theology in Society}

Metamusical villancicos drew listeners' attention to the artifice of the music
they were hearing. 
These villancicos name the musical techniques and styles they are using, as
they use them.  
In the most self-referential examples like Gutiérrez de Padilla's solmization
villancico, listening to the piece and analyzing the its technique are
virtually the same thing. % XXX
When Joan Cererols sets \wtitle{contrapunto celestial} to an eight-voice fugue
(see \cref{ch:cererols-suspended}), listeners are hearing the word counterpoint
as they hear actual counterpoint; but the purpose is to give them the
opportunity to rise from the \term{musica instrumentalis} they are hearing to
celestial and divine music.

Villancicos invite listeners not simply to hear, but to \quoted{take heed}, to
both discern deeper meanings in what they hear and to put what they hear into
practice.
Vernacular villancicos could evoke a response in their hearers in a way that
the rest of the liturgy did not.
Villancicos offered listeners an opportunity to think about (and \quoted{feel
about}) the content of the faith in their own language (or at least in one they
understood better than Latin), an opening otherwise only offered in the sermon,
when one was preached.
Moreover, while many villancicos do begin with an exhortation to listen, most
pieces also include imperatives to actively respond to what is heard.
The command is often affective and devotional---\quoted{llorad},
\quoted{sentid}, \quoted{arde} (weep, feel, burn), or (more commonly with
Christmas pieces) \quoted{Cantad}, \quoted{Alegren}, \quoted{Repican} (sing, be
joyful, repeat the angels' song).
Other pieces call on listeners to dance and play instruments.
In other words, villancicos ask listeners to both contemplate and obey---in
short, to hear the Faith with faith and to respond in faithfulness. 

Even though seventeenth-century villancicos should not be thought of as folk or
popular music in the sense of music passed on through oral tradition, the elite
createors of villancicos certain did \quoted{appeal to the people} in an active
sense.
Not only because of the words, but because of the lively, diverse musical
styles used, which probably were associated with a lower social register,
villancicos \quoted{appealed to the ear} both in the sense of being
\quoted{pleasing} and in the sense of \quoted{reaching out to} or
\quoted{making a claim upon}.
Even if considered as a kind of sacred entertainment, we must acknowledge the
theological function of pieces that allowed people to find humor, delight, and
wonder in religious mysteries that might otherwise have remained inaccessible
and uninteresting to them.

There are so many \quoted{Listen!} openings that this gesture deserves deeper
reflection than dismissing it as simply a practical way of getting attention,
or as a generic convention. 
The recurrence of this kind of exordium in villancicos may indicate that the
genre itself was fundamentally about getting people to listen.
The rest of the liturgy may have passed through the lay people's ears like the
incense wafted by their noses, creating a general atmosphere of devotion but
not evoking any specific sentiments (thoughts, ideas, images) in the mind and
not provoking any direct response.
But the vernacular villancicos demanded attention.
Many of them presented hearers with bold, striking images at their openings,
projecting an intriguing poetic conceit or scenario through musical structures
and styles that made people take notice. 
In other words, villancicos made faith appeal to hearing.

But did this effort really work?
Did people understood the riddles or get the jokes, and if so, which people
were they? 
Did the musical rhetoric and symbolism that is unpacked in such detail in this
study really serve as objects of contemplation to anyone at the time, including
the performers?
It is certainly possible that the whole culture of producing and consuming
villancicos was the domain of a cultivated elite and did little actually to
propagate faith among a broader range of hearers.

One theological teacher, to whose writing we will return in the next chapter,
warns preachers of the danger of celebrating church festivals without
adequately teaching common people what they mean:
\quoted{For this \add{teaching} is such an important business, especially given
that it is a mortal sin that they should not know what day it is, what it means
that our Lord was born, what it means that he died and rose to the heavens: and
all this which the Church celebrates with such great festivities}.%
    \Autocite[27]{Azevedo:Catecismo} 
The added difficulties of teaching in the colonial context and the increasing
aesthetic of complexity in learned Spanish poetry and music of the seventeenth
century would suggest that many commoners remained unformed in such basic
matters of faith.
As Azevedo suggests, the church's ceremonies, including the dozens of poetic
verses and musical figures of villancicos in festival Matins, did actively
celebrate the Church's faith---but just because cathedrals echoed with these
words and music does not mean that everyone understood them on the same level.
Even for literate listeners, the complexity of music across a whole cycle of
eight or more villancicos would present a challenge for any listener seeking
\quoted{to hear Faith with faith} through music.
This would be especially true in metamusical pieces, which required a
sophisticated knowledge of music to even make sense of the text.
The next chapter explores this theological problem, the challenge of making
faith appeal to hearing.

% last word TODO
%}}}1

\endinput
