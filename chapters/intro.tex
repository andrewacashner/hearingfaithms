% vim: set foldmethod=marker :

% Cashner, *Hearing Faith*
% chapter 1: Villancicos as Musical Theology
% 
% 2015-03-18	Dissertation defended
% 2017-11-15    New start for book proposal
% 2018-05-21    Converted back to LaTeX
% 2018-07-25    Expanded for press readers
% 2019-07-08	New start for Brill, under contract

\part{Listening for Faith}
\label{part:faith}

\chapter{Villancicos as Musical Theology}
\label{ch:intro}

\epigraph
{ergo fides ex auditu\\
auditus autem per verbum Christi}
{Romans 10:17}

%{{{1 intro
\section{Music in Catholic Spain, between Hearing and Faith}

%{{{2 intro^2
\quoted{Faith comes through hearing}, wrote Paul the apostle to the Christian
community in Rome, \quoted{and what is heard, by the word of Christ}
(\scripture{Rom}{10:17}).
Sixteen centuries later, amid the ongoing reformations of the Western Church,
Christians were seeking ever new ways to make faith audible.
Voices raised in acrid contention or pious devotion boomed from pulpits,
clamored in public squares, and were echoed in homes and schools.  
In new forms of vernacular music, the voices of the newly distinct communities
united to articulate their own vision of Christian faith.
Roman Catholic reformers and missionaries, charged by the Council of
Trent (\XXX[years]) to educate and evangelize, enlisted music in their
campaigns to build Christian civilization, both in an increasingly divided
Europe and in the expanding global domains of the Spanish crown.
In these efforts to make \quoted{the word of Christ} to be heard and
believed---to make faith appeal to hearing---what did they understand to be the
role of music?
What kind of power did Catholics believe music held to affect the relationship
between hearing and faith?

This book is a study of how Christians in early modern Spain and Spanish
America enacted religious beliefs about music through the medium of music
itself.
Within the political and cultural framework of the Spanish Empire, the
spiritual work of propagating faith and the comparatively worldly project of
building colonial society were in most cases inseparably fused (or confused),
and both state and church leaders cultivated music to serve both purposes.
Beginning around 1590, Spanish churches began incorporating music with
vernacular texts into their worship, in a genre of music they labeled with the
catch-all term \term{villancico}.
What had previously been an elite form of courtly entertainment and sometimes
devotion was transformed into a variety of complex, large-scale forms of vocal
and instrumental music performed as an integral part of public church rituals.

Villancicos engaged the creativity and piety of poets, composers, and
performers in every major religious institution across the empire.
Their performances constituted a major element of the soundscape of the early
modern Ibero-American world, and shaped the everyday experiences of thousands
of people across social strata.
They flourished especially in the second half of the seventeenth century but
continued to be a prominent element of Spanish Christian worship through the
nineteenth century in some places.
(The term today refers to much simpler folkloric Christmas carols, which may
descend from the earlier genre of complex notated music.)
Communities from Madrid to Manila heard and performed villancicos on the
highest feast days of the year---Christmas and Epiphany, Corpus Christi, the
Conception of Mary, and saints' days of local importance (though they
favored other genres during Holy Week).
Villancicos were typically presented in sets of eight or more, and were
interpolated after or in place of the Responsory chants of the Matins liturgy.
They were also sung in Mass and during Eucharistic devotional services.
Festival crowds also heard villancicos in the public square in processions and
mystery plays, especially on Corpus Christi.

In their seventeenth-century heyday, villancicos were a genre of religious
lyric poetry, often published in unbound leaflets or broadsheets; that
were set to music and performed by ensembles of voices, doubled or accompanied
by instruments like shawms, dulcians, and harps.
These ensembles could be as small as one or two solo voices and as large as
three separate choruses of a dozen vocal parts with multiple singers per part
and a full continuo group providing partly improvised accompaniment.
In structure most villancicos feature an \term{estribillo} or refrain section
for the full ensemble, usually through-composed rather like a motet or sacred
concerto, and strophic \term{coplas} or verses for soloists or a reduced group.
The words for these pieces were in Spanish and sometimes other vernacular
languages including Portuguese, Catalan, and Náhuatl (language of the Aztecs),
along with pieces whose texts imitated the dialects of African slaves.
The music varied in style and technique from elements of common dances
and popular tunes up to the most sophisticated polyphonic tone-painting.
Moreover, sets or cycles of villancicos for a particular feast like Christmas
included many different types of villancicos within them, offering something
for everyone.

Churches of the seventeenth-century Spanish Empire, then, became places to hear
faith proclaimed, celebrated, explained, and embodied through music that
appealed to the ears of many different kinds of worshippers---elite and common,
learned and untaught, masters and slaves---and where contrary elements were
juxtaposed in lively and sometimes unruly counterpoint.
Even after considerable loss of sources, hundreds if not thousands of musical
manuscripts of villancico settings and imprints of the poetic texts open a
window into Catholic devotional culture in this period relevant to the lives of
many thousands of people around the world.
Most of the known sources of villancico music and poetry have now been
catalogued, but only a handful have been revived in performance and the
genre has received relatively little serious scholarly attention in terms of
poetic and musical analysis or interpretation.

This book is the first major effort to understand the meanings and functions of
this music as a form of religious practice, integrating musical and theological
interpretation.
The goal is not to provide a comprehensive musicological treatment of the genre
in all its aspects, but to use select examples of this genre to reach a deeper
understanding of early modern religious culture.
Theological interpretation is only one of many valid and necessary approaches
to this multifaceted genre, but it is my conviction that we will never fully
understand villancicos without understanding their role first and foremost as
expressions of religious belief. 
Villancicos played a central role in the devotional life of Spanish Catholic
communities.
Most focus on Jesus (especially his Incarnation and his sacramental presence in
the Eucharist), Mary, saints, or angels.
They were composed by professional church musicians, who were typically under
contractual obligation to provide this music to be performed in liturgical
worship and paraliturgical celebrations.
They were heard in cathedrals, parish churches, and monastic houses, brought to
life by performers who were often under a vow of religious life.
It should require no special pleading, then, to insist that we need to
understand this music within the profoundly theological context in which it was
patronized, created, performed, and heard.

Of all the musical forms of Catholic Spain, in fact, sacred villancicos address
the theological nature and function of music most frequently and directly.
A large number of villancicos begin with calls to listen---\emph{escuchad},
\emph{atended}, \emph{silencio}, \emph{atención}. 
Because so many villancicos explicitly address concepts of music, sensation, and
faith, these remarkable but understudied pieces offer us unique insights into
Spanish beliefs about music.
When villancicos focused on the theme of music itself, most often by playing on
terms from music theory to build elaborate theological metaphors, they become a
sounding discourse on musical sound.
If a play within a play in seventeenth-century Spanish or English theater is
metatheatrical, then these pieces are \emph{metamusical}.

Through this genre of musical performance people embodied their theological
conceptions of music through the structures of music itself.
For this reason they may be considered as \quoted{musical theology}.
It is the central argument of this book that this devotional music provided
Spanish Catholics with a way of performing theology. 
Making and hearing music was a creative pursuit in which people sought to forge
connections to God and to each other through musical structures.

Educated Spaniards had studied both the theology of Augustine and Aquinas and  
the fundamentals of music on theoretical and practical levels. 
They had learned from Boethius how human music was linked to cosmic harmonies,
and they had learned from Guido of Arezzo how to sing through the gamut using
the mnemonic device of the Guidonian hand.
Metamusical villancicos brought these two domains of knowledge into a mutually
illuminating relationship.
Someone reading the poems or hearing them read had to know a fair amount of
music theory in order to understand the theological concepts, and vice versa.
Hearing the music settings of poems that were themselves about music required
an even higher degree of training to understand the words as projected through
the music and perceive the ways the music depicted the sense and affect of the
text.
Though these challenges surely must have kept some hearers from understanding
villancicos fully (just as we might say about audiences for Monteverdi and
Schütz, or Calderón and Shakespeare), I argue that the pieces studied in this
book actually had the potential to effect a spiritual kind of ear training. 
They were discourses about musical listening, through the medium of music, that
also taught people how to listen.
%}}}2

%{{{2 singing about singing : examples
\subsection{Singing about Singing}

It will be helpful at this point to listen to two villancicos that I would
classify as metamusical---pieces that embody \quoted{singing about singing}.%
    \autocites{Murata:Singing}
    [\XXX]{Illari:Polychoral}
The first example is a villancico from the 1652 Christmas cycle written for the
Cathedral of Puebla de los Ángeles in New Spain by Juan Gutiérrez de Padilla
(\circa{1590}--1664).%
\begin{Footnote}
    \sig{MEX-PC}{Leg. 1/2}. 
    \XXX[Audio recordings of every numbered example of music and poetry in this
    book are available on the companion website.]
    Another villancico by this composer is the subject of
    \cref{ch:padilla-voces}.
\end{Footnote}
In just the first seven lines of this anonymous text
(\cref{poem:Padilla-1652-Gloria_portalillo}), the villancico refers to multiple
kinds of music, referring to the sounds of voices, choirs singing, birdsong,
dancing, and using solmization syllables.

% %{{{5 poem GdP En la gloria
% \insertPoem{Padilla-1652-Gloria_portalillo}
% {\caption{\wtitle{En la gloria de un portalillo}, estribillo as set by Juan
% Gutiérrez de Padilla, Puebla Cathedral, Christmas 1652 (\sig{MEX-Pc}{Leg.
% 1/2})}
% %}}}5

Gutiérrez de Padilla's setting is metamusical in that it enacts these
references through music (\cref{mux:Padilla-Portalillo}).
It also has the virtue of demonstrating several typical features of the genre
which should provide a useful foundation for our study.
The solo line is followed by a passage of polychoral dialogue between two
four-voice choirs, concluding (typically for polychoral technique) with an
emphatic cadence for the full chorus.  
The setting is in a lively triple meter: this was termed \term{tiempo menor de
proporción menor} and notated \meterCZ{} (CZ), a cursive shorthand for
\meterCThreeTwo{} or \meterCThree{} (according to the contemporary music
treatise of Andrés Lorente).%
    \citXXX[Lorente]
Within this meter the composer makes frequent use of \term{sesquialtera} or
hemiola, an alteration of the rhythmic pattern that feels like a momentary
shift from ternary to duple stresses.%
\begin{Footnote}
    \XXX[See the glossary and preface]
    The preface provides additional background about the terminology and common structures of seventeenth-century villancicos.
    Please note the discussion there on common voicing and instrumentation patterns, and on rhythmic theory.
\end{Footnote}
The shifts of duple and triple stresses combine with stresses on the second
beat of the \term{compás} (\term{tactus}, measure) to create an energetic
atmosphere with a rejoicing affect.  
The polychoral dialogue, with the voices of each choir declaiming
homorhythmically in the same highly rhythmic, syncopated manner as the soloist,
and with the \term{tiples} (boy sopranos) of both choirs singing at the top of
their range, would have brilliantly seized the attention of listeners.

After this introductory \term{exordium}, the Tiple I soloist continues to
describe the scene at the manger.  
As the shepherds \quoted{are turned to boys} (\foreign{se vuelven niños}),
Gutiérrez de Padilla has the musicians \quoted{turn} modally by adding C
sharps, accented in a sesquialtera ($3:2$) group.
The passage that follows this moment is in evenly accented ternary patterns, in
two-measure groups.  
These groups emphasize the rhymes in \foreign{tonos sonoros, repiten a coros}
and the clear triple meter evokes the dances of \foreign{en bailes lucidos}.
When the soloist refers to the newborn Sun, he sings the note identified in
Guidonian terminology as D \term{(la, sol, re)}---\term{sol} in the hard (G)
hexachord.  
On the same word, the bass accompanist plays a different \term{sol}, G
\term{(sol, re, ut)}. 
(Note that \quoted{sol re} in Spanish means \quoted{sun king}.)%
\begin{Footnote}
    The major Spanish music-theoretical treatises of the seventeenth century
    give full expositions of the techniques of Guidonian solmization:
    \autocites{Cerone:Melopeo}{Lorente:Porque}.
    The frequent symbolic use of Guido's syllables in villancicos suggests that
    these treatises do reflect how music was actually taught in practice.
\end{Footnote}

% %{{{5 music gdp portalillo start
% \insertMusic{Padilla-Portalillo}
% {Gutiérrez de Padilla, \wtitle{En la gloria de un portalillo}
% (\sig{MEX-Pc}{Leg. 1/2}, Christmas 1652), estribillo}
% %}}}5

This villancico may be understood as \quoted{singing about singing} on several
levels.  
The text, which is being performed through music, itself refers to musical
performance.
The performance by the Puebla Cathedral chapel dramatizes the historical
celebration of the first Christmas while also celebrating the festival in
Padilla's present day.  
The music is self-referential on a symbolic level (as in the plays on
\term{sol}), but also functions on a more simple affective level to model and
incite affections of exuberant joy and wonder, which contemporary theological
writers emphasized were the appropriate affects for the feast of Christmas (see
\cref{ch:padilla-voces}).

%{{{3 cererols fuera que va
A similar example of a villancico that includes multiple metamusical topics is \wtitle{Fuera, que va de invención} by Joan Cererols (1618--1680), monk and chapelmaster at the Benedictine Abbey of Montserrat near Barcelona.%	
\begin{Footnote}
    \sig{E-Bbc}{M/760}, \autocite[81--94]{Cererols:MEM-VC}.
    A villancico by this composer is the subject of
    \cref{ch:cererols-suspended}.
\end{Footnote}
Rather like catalog-like Christmas songs today (\wtitle{Deck the Halls},
\wtitle{Chestnuts Roasting on an Open Fire}), the piece summons up all the
elements of a Christmas festival---masques, \foreign{zarabandas} and other
dancing, lavish decorations and clothing, pipes, drums, and so on.
As in many villancicos, the chorus acts dramatically in the role of the
festival crowd, shouting affirmations (\foreign{¡vaya!}) for each element of
the celebration as the soloists name them.  
Whereas Gutiérrez de Padilla's \wtitle{En la gloria de un portalillo} focused
primarily on the music of the historical Christmas day, the villancico of
Cererols is unambiguously about celebrating \soCalled{Christmas present}.
The piece seeks a theological meaning behind the Christmas customs: the masques
of Christmas, the poem says, are appropriate because in the Incarnation of
Christ, \foreign{Dios se disfraza} (God masks himself).
The villancico allows performers and listeners to celebrate the festival in two
senses: to sing the praises of the Christmas feast, while also singing the
praises of Christ that are appropriate to that feast. 
Cererols's original audience of pilgrims to the mountaintop shrine of
Montserrat would not have sung along with this piece, but the piece still
invites their wholehearted participation in the rituals of Christmas, both
through enjoying the choral singing (and joining \quoted{in spirit}, perhaps),
and in the many other common-culture customs that the piece celebrates.

These pieces presented hearers with a discourse about music, through music.
Sometimes the music they refer to is literal, human music-making; other times
it is more abstract, like the music of the spheres or the harmony of human and
divine in the incarnate Christ.
In every case, analyzing the musical choices made to represent texts about
music helps us understand how the creators and their audiences heard different
kinds of music.
And interpreting their theological aspect enables us to see how these pieces
served to communicate with hearers at a spiritual level.
%}}}3
%}}}2

%{{{2 hearing/communication
\subsection{Hearing and Communication}

As the most widespread form of Catholic religious music with vernacular words
after the Council of Trent, villancicos provide evidence for a sustained
endeavor by church leaders to establish conventions of communication with
ordinary people.
% XXX could mention other kinds of Catholic vernacular song (German, Czech?)
The creators of villancicos drew on common experiences of everyday life and
linked them to the sacred in inventive ways that met the spiritual needs of
specific communities.
Each piece provides a new answer to Christ's question, \quoted{With what can we
compare the kingdom of God, or what parable will we use for it?}
(\scripture{Mk}{4:21}).
Villancicos thus played a key role in the Spanish church's effort to to make
faith appeal to hearing through music.

This music spoke to a variety of people at different levels of understanding.
It was a central part of community festivals in many different local
environments and a variety of public and private contexts. 
Each villancico cycle includes an array of subgenres that would speak to
different portions of the congregation, from the silly and child-friendly
dialogues of Christmas shepherds to high-concept pieces with, for
example, abstruse numerological symbolism.

Even the structure of individual villancicos reflects the effort to communicate
on multiple levels.
The \emph{estribillo} section of a typical villancico was presented by the full
ensemble at the beginning and then repeated at the end; composers usually set
this in relatively complex polyphony similar to what they would use for a
motet.
In the center of the piece, the \emph{coplas} or verses were usually set
strophically for solo singers or a reduced ensemble with accompaniment.
As Bernardo Illari argues, the \emph{copla} settings are probably based closely
on oral traditions for singing poetry, especially in the \emph{romance} meter,
to stock melodic formulas; and it would have been easier for common listeners to
make sense of the words that were sung to the simple, repeating melodies.%
    \Autocite{Illari:Polychoral}
The \emph{estribillo}, by contrast, is often much more complex and draws on
traditions of learned counterpoint; composers often invoke a variety of
stylistic registers and styles to convey the meaning of the words and heighten
their rhetorical impact.

But though villancicos have these aspects that seem designed to engage a wide
popular audience, they differ from other dominant traditions of vernacular
religious music in this period---Lutheran chorales and Reformed psalms---in
that they were not sung by ordinary parishioners.
Rather, more like Anglican anthems and German sacred concertos, they were
performed by professional church musicians for the benefit of the congregation.
The printed commemorative chapbooks and manuscript performing parts preserve
only one side of the church's dialogue.
Catholics did not, generally speaking, cultivate a society of literate,
self-advocating lay people who would have left behind traces of their personal
beliefs and devotional practices with regard to music.
For the Spanish Empire, then, we know what people heard, but not what they
understood or how they responded.%
    \Autocite{Burstyn:PeriodEar} % + Did people listen? etc.
And when villancicos represent types of people---such as deaf men, African
slaves, or Indians---they leave us only with conventional caricatures, not
ethnohistorical descriptions.%
    \Autocites
    {Baker:EthnicVC}
    {Baker:PerformancePostColonial}
    {Davies:LocalContent}

All the same, the devotional music that survives from imperial Spain
can still open a fascinating window into the process of religious communication.
First, villancicos should not be understood as an exclusively top-down
communication, and certainly not as a simple mode of religious indoctrination.
The creators of villancicos were not always members of the most elite strata,
and their readers and hearers included commoners.
The cultivated poet Francisco de Quevedo was credited with mocking \quoted{the
whole caste of villancico poets} as hacks, saying that \quoted{the poor are
drowning in poets, continually hearing their braying}.%
    \Autocite[37]{Torres:SuenosMorales}
If there is any truth to the critique of villancico poets as stringing together
clichés to satisfy the tastes of a lower-class market (an attack also leveled at
opera librettists in Italy), then the same low-class elements that those poets
disdained can provide us with insight into culture at a more common level.
On the musical side as well, some villancico composers were not prestigious
cathedral chapelmasters and we know of at least one who was of indigenous
ancestry, Juan de Araújo in Boliiva.%
    \Autocite{Illari:Popular}
%    \citXXX[non-MC composers, Illari]
Besides, regardless of their personal background, villancico poets and composers
had to produce something that met the needs of their community. 
Though they answered first to their cathedral chapter and the local ruling
caste, it was in everyone's interest to attract commoners to church and provide
them something that they would find satisfying.
According to contemporary accounts people turned out in droves to hear the
annual villancico performances, in annual traditions that in most Spanish
cities lasted for two centuries or more.%
    \citXXX[audience turnout]
Somewhat like mass-mediated popular music today, this music was not typically
created by common people themselves, but it both reflected and shaped popular
tastes and attitudes.
%}}}2

%{{{2 conceptismo
\subsection{\term{Conceptismo} and Creative Theology}

Most of the villancicos performed in seventeenth-century churches may be
described as attempts to connect often abstract religious concepts to images
and experiences from everyday life.
The more surprising and puzzling the connection, the better---such as
representing the Virgin Mary as the chapelmaster of the heavenly chorus (see
below), or imagining Christ as a gambling card player.%
    \Autocite{Cashner:PlayingCards}
Christmas villancicos in particular often brought rogues, buffoons, peasants,
and slaves to Christ's manger to offer songs and dances characteristic to
them--even Don Quijote and Sancho Panza make an appearance in one villancico.%
    \citXXX[Don quijote VC]
While these folkloric and comic elements can be amusing, we should not lose
sight of the fact that these elements are almost always used to point to some
theological aspect of Christmas.

In Spanish literary studies, the term \term{conceptismo} has been used to
denote a technique in which poets used extended metaphors to create parallel
discourses that brought two subjects into relationship with each other.
As early as 196X\XXX[name], English poetry scholar \XXX[name] cited Alonso de
Ledesma's (\XXX[year]) book of devotional poetry, \wtitle{Conceptos
espirituales}, as a fountainhead for the poetic technique of villancicos that
compared worldly and ethereal, human and divine elements.%
    \citXXX[source of both and quote?]
Ledesma's earthy approach, aimed at relatively uncultivated readers, differed
starkly from the later \term{conceptista} poetry of Luis de Góngora and his
followers on both sides of the Atlantic, in which the language was pushed to
the brink of comprehensibility to make elaborate and highly learned double and
triple conceits.%
    \citXXX[Gongora]
Both ends of this spectrum are well represented in the villancico poetic
repertoire, and similar extremes may be found in approaches to the musical
setting as well.
It has been suggested that villancicos be explained as simply a religious
variant of \term{conceptismo}.%
    \citXXX[reviewer? Torrent?]
But the label \term{conceptismo} alone does not do much to help us understand
how this poetry and music actually worked for readers and hearers as a form of
religious devotion, any more than the term \term{chiaroscuro} explains what
Caravaggio actually hoped would happen for viewers when they looked at his
paintings.%
    \citXXX[Torrente, Begue?]
It does not explain why churches cultivated villancicos or tell us what people
got out of hearing them.

I propose that on the contrary, we understand \term{conceptismo} in the this
context as a specific literary form for doing a certain kind of theological
thinking.
The art, literature, and music of early modern Spain are so saturated with
religious themes that we must conclude that theology was a major intellectual
pursuit of the Spanish and New Spanish elite.
What I mean by theology is a creative activity---not merely reciting dogmas
approved by the Church, but imaginatively, even playfully, seeking out ever-new
ways of connecting revealed truth to observed experience.
Thinking theologically in an early modern Catholic sense meant building 
endless chains of association and allusion among Biblical texts, writings of
church fathers (patristics), medieval theologians, and the liturgy. 
It meant interpreting new texts in light of these old ones, and reinterpeting
the old ones in light of the new.
And it grew out of and reinforced a view of the world as a book waiting to be
read (see below).

Villancicos challenged listeners to discern how the sacred was imminent in the
mundane and common.%
    \citXXX[Chavez:PhD]
Through words and music, these pieces asked hearers to hold together unexpected
combinations of elements and search out a new meaning to be found by going back
and forth between the two.
If I may use some terms creatively myself, the religous purpose of villancicos
was not so much \term{doctrine} as \term{doxology}.
Theologians use the latter term with its Greek meaning of glorification or
worship; the Eastern Orthodox use this root to identify not as \quoted{true
believers} but \quoted{true worshipers} (with the assumption that true worship
is rooted in true belief and in turn gives rise to it).%
    \citXXX[theology, lex orandi etc.]
The point of comparing Christ to, say, a plucked string instrument (the
\term{vihuela}, see \cref{ch:zaragoza}) was not to teach a particular
doctrine about Christ but to worship Christ in a particular way.
Catholics who revere Our Lady of Montserrat know that they are asking for the
intercession of the same Blessed Virgin elsewhere honored as Our Lady of
Sorrows or Our Lady of Loreto: there is one Lady, but many ways of reflecting
on her life and applying it to one's own.%
    \citXXX[Johnson?]
The Catholic liturgical texts and sermons for Christmas do aim to instruct
believers about the theological concept of Christ's Incarnation, but they also
provide a way for them to celebrate the Incarnation, and through the Eucharist,
to actually share in it.

Christmas villancicos depend on knowledge of doctrine more than they teach it.
As Padre Daniel Codina of the Abbey of Montserrat remarked in puzzlement at a
villancico by Montserrat monk Joan Cererols, they seem like \quoted{an
explanation that itself needs to be explained}.%
    \footnote{Personal communication, \XXX[date].}
Rather like contemporary emblem books (see \cref{ch:zaragoza}), which featured
a picture with a Latin motto, a Spanish poem, and a prose explanation, each
element of a villancico increased one's depth of understanding of all the other
parts, with the result that the whole could became a mnemonic device for a
whole complex of ideas.

It would be wrong to assume, then, that examining the theology of villancicos
today would not reveal any more than the official teachings of the Tridentine
catechism.
Of course, the texts were subject to church censorship, and we should not
expect to find anything radically subversive in them, not at least at the
surface level that would catch a censor's eye.
But it was possible to stay close to church teachings on the more theological
side of a poetic conceit while reaching far afield into worldly experience for
the other side, and in fact this is what we find in villancicos.
Comparing Christ to a vihuela, to continue with that example, taught people who
knew about the vihuela new ways of thinking about Christ, just as it taught
people who knew about Christ new ways of thinking about the vihuela.
In short, it made the vihuela into an object for theological contemplation or
reflection.%
\begin{Footnote}
    I am not using \term{contemplation} in the highly technical theological
    sense in which it was used by the writers of treatises on prayer like
    Teresa de Ávila or San Juan de la Cruz, but rather in the more general, and
    widely used early modern sense of mental reflection and meditation.
    The technical sense denotes a practice of seeking to interact with God in
    prayer with a sense of God's absolute ineffability and not through lesser
    images and sensory stimulations. 
    It is used, I would argue, as a shorthand for \quoted{contemplating God as
    God really is}, and in this way the more general meaning of the verb is
    actually preserved.
    One could contemplate many things, but \quoted{religious contemplation} or
    \quoted{contemplative life} meant a dedicated process of learning to
    contemplate this one thing.
    Moreover, there are actually times in which the kind of contemplation
    encouraged by villancicos seems to share some of the goals of this kind of
    contemplative prayer.
\end{Footnote}
Moreover, the musical settings, which were not subject to censorship, add
layers of meaning and shape the experience of the performance in ways that
cannot be reduced to a simple idea of teaching doctrine.
Because there are not many written texts from imperial Spain telling us how
people interpreted the music they heard, the body of villancicos about music
provide vital evidence for how and why Spaniards listened to music.
%}}}2

%{{{2 scholarship: integrate music and theology
\subsection{The Need for an Integrated Musical and Theological Approach}

% TODO topic sentence; other sources (?) from diss lit review, in notes file

We can only understand how they listened, though, if we pay close attention to
the sounds they listened to.
Much of the valuable new scholarship on Spanish colonial music, and on sound
and sensation, focuses not on musical texts but on social and institutional
history, and on verbal discourse \emph{about} music.%
    \Autocites{Baker:Harmony}{BakerKnighton:MusicUrbanSociety}
    {Irving:Colonial}{RamosKittrell:PlayingCathedral}
    {DellAntonio:Listening}
% TODO more lit
This book provides a necessary complement to these studies, by analyzing how
people expressed and shaped beliefs about music through the medium of music
itself.
At the same time, the book offers a fresh approach by considering this music as
a source for historical theology, something few scholars have done.
The primary goal of the project is to combine musical and theological analysis,
to understand how theological beliefs were expressed and shaped through the
details of musical composition and performance.

Until recently, scholars and performers have focused on a small repertoire of
villancicos that does not really represent the full range of pieces
that were commonplace in this period, resulting in a conventional view of
villancicos as primarily folkloric, secular---that is, worldly,
irreverent---comic, small in scale, and primarily of interest for the traces
they appear to preserve of popular culture, particularly that of Native
Americans and Africans.
No doubt, plenty of villancicos do fit that description, but as I have argued
elsewhere, even these pieces need to be understood within a theological
frame---and we would benefit from thinking more deeply about what we can
actually learn from these Spanish representations of non-Spaniards.%
    \Autocite{Cashner:BuildingSociety}

Villancicos, when understood within a religious framework, offer unique
insights into the worldview and life experience of people in early modern
Spain.
Understanding them requires a process of interpretation in which we must
attempt to enter imaginatively and sympathetically into the same kind of
theological reflection that the original hearers may have done.
Necessarily this process, like all acts of interpretation, is subjective to a
degree.
If my readers are like me, the details of belief expressed in these pieces will 
be foreign and perhaps surprising.
I am a Christian but I am not a Roman Catholic, and while I may sympathize with
certain elements of the historical belief system I elucidate here, there is as
much that I find offensive or just silly.
Certainly our interpretations need to take a critical view of the many
contradictions and tensions in the texts, and recognize the distance between
the imagined world of these texts and the real social worlds around them. 
All the same I do want to take villancicos seriously as expressions of real
religious belief and articulations of a widespread historical worldview.

This book is not, then, a work of constructive or normative theology.
In fact, I argue that interpreting villancicos requires us to set aside our own
religious ideas---or anti-religious ideas---in order to hear the world through
historic ears, to the extent we can venture to do so.%
    \citXXX[historic ear]
While I hope that Christian readers may find the book edifying and relevant to
the contemporary challenge of making faith appeal to hearing, just as I hope
that musicians may discover here new possibilities for historically informed
and culturally sensitive performances, those practical applications must
remain outside the scope of this book.

Our challenge is great enough already---to build a sufficient interpretive
framework that we can recover a plausible range of meanings that this music
might have had for its creators and first hearers.
The primary methods for accomplishing this here are comparative study within a
corpus of pieces on related themes, mostly edited for the first time from the
original sources in nine archives in Mexico, Spain, the United States, and the
United Kingdom.
This corpus makes it is possible to identify recurring and developing tropes
and conventions, which may be understood more deeply through and contextual
study of related literature and arts.
The related literature includes poetry and sacred drama, theoretical and
practical treatises on music, and several branches of theological writing.
These sources include doctrinal theology (explanations of Christian beliefs),
exegetical theology (interpreting Scripture), homiletics (preaching, applying
Scripture and doctrine to daily life), and devotional literature (teaching and
modeling prayer and worship for personal and community life).
Sources more commonly used by institutional and social historians like chapter
acts, payroll records, and notarial documents play a much smaller role here
than in other recent studies, though I hope that future scholarship will
combine these and other approaches, most feasibly through collaboration, to
produce a fuller understanding of Spanish musical culture.

The religious element of early modern Spanish culture is so overwhelmingly
evident to anyone who has visited Mexico or Spain or read any of its
literature, that it can hardly be justified if we overlook it or insist on
interpreting it through an anti-Catholic lens.
This is simply a study of how people in the past expressed beliefs through
poetry and music, intended for readers who like me are interested in
understanding historical religious beliefs and practices, who enjoy moving
through the hermeneutic circle in pursuit of new perspectives on past and
present.%
    \citXXX[ricoeur]
This book is for anyone who has gazed upwards in a Spanish or Mexican church
and wondered why there are so many images of angel musicians with harps,
\term{vihuelas}, \term{bajones}, and organs; or who has read a play by
Calderón, a poem by Sor Juana, or a devotional book by Ignatius of Loyola or
Saint John of the Cross, and has observed how often these writers use musical
metaphors; or who wonders what people thought was happening when they listened
to music in church and how they believed this connected them to God, to each
other, and to the cosmos.
Through the many ways that Spanish villancicos engaged their audiences' sense
of hearing, and through the ways the pieces model musical hearing itself, they
also offer a glimpse of what a broader audience of common people listened for
in music and what powers they believed it had to shape their community.
%}}}2
%}}}1

%{{{1 music about music
\section{Music about Music in the Villancico Genre}

%{{{2 survey, types
The examples of metamusical villancicos by Gutiérrez de Padilla and Cererols
combine several of the common tropes of \quoted{music about music} in the
villancico genre, as evidenced by a global survey of extant villancico poems
and music.%
\begin{Footnote}
    The survey was based on the listings in catalogs and published studies and
    from archival sources, some previously unknown, from all over the former
    Spanish Empire (see \cref{biblio}).
    While I was able to examine hundreds of complete music manuscripts and
    poetry imprints in the nine archives which I visited personally or which
    made sources available to me electronically, for the others I had to infer
    content based on catalog incipits and descriptions.
    Experience with the pieces studied in detail in this book suggests that
    this approach revealed fewer rather than too many relevant sources, as it
    is not always possible to guess the themes of a villancico poem from its
    first line, and variants of the same textual family are often hidden by
    different opening verses.
\end{Footnote}
More than eight hundred villancicos were found in which themes of musical
hearing were central, a number that only hints at the original size of this
repertoire.
These metamusical villancicos may be grouped in eight main categories
(\cref{tab:survey}): in descending order of frequency these are hearing and
sound, music and singing, birdsong, dance, musical instruments, angelic
musicians, music of the heavenly spheres, and pieces that treat the
relationship of sensation and faith.
An additional category of pieces about affects is also included, since these
pieces, though not explicitly about music, do address the question of how
listeners should respond in worship. 

%{{{5 table survey topics
\insertTable{survey}
{Topics of metamusical villancicos in global survey}
%}}}5

The central section of this chapter will look at examples in the most
widespread categories to analyze the different ways that these pieces represent
making and hearing music in relation to faith.
Villancicos addressing the sense of hearing specifically will be the topic of
\cref{ch:faith-hearing}; and \cref{part:unhearable-music} will present case
studies of families of villancicos about music and singing, most of which also
concern heavenly and angelic music.

In each of the categories in \cref{tab:survey}, we may distinguish between two
main ways of referring to music.  
Some pieces are primarily imitative, referring to real human music-making
(Boethius's \term{musica instrumentalis}).
These pieces are highly \term{intermusical}, in the way a verbal text full of
references to other texts is intertextual.
In contrast to this first category of imitative pieces, villancicos in a second
category refer to music as more of an abstract concept, such as the higher
Boethian levels of music, music as a Neoplatonic ideal, and the music of
Heaven---notions that overlap in inconsistent ways in early modern thought.
Of course, the pieces in the latter group still refer to music in the abstract
through the medium of real sounding music.  
Some of these pieces depend more on \term{intramusical} relationships---that
is, musical references internal to the individual piece itself, such as melodic
or rhythmic motives or internal contrasts of musical style without overt
references to pre-existing styles \quoted{outside the piece}.
We will consider first examples with imitative references, and then look at the
more abstract or symbolic references.
The final section of the chapter will then outline the historical theological
foundations that undergirded these practices of musical representation.
%}}}2

%{{{2 imitative references
\subsection{Imitative References to Music: Birdsong, Instruments, Songs and
Dances}

%{{{3 birdsong
A frequent example of imitative musical reference in villancicos is when the ornamented vocal lines are used to represent birdsong.
In a piece called \wtitle{Sagrado pajarillo} (Little sacred bird), Zaragoza
composer José de Cáseda sets the lyrics \foreign{con gorgeos} (with trills) to
twittering melismas (\cref{mu:CasedaJ-Sagrado_pajarillo}).%
\begin{Footnote} 
    This piece comes from the archive of the Conceptionist Convento de la
    Santísima Trinidad in Puebla de los Ángeles and is now preserved at CENIDIM
    in Mexico City (\sig{MEX-Mcen}{CSG.155}).
    As with many other music manuscripts in that collection, the original
    lyrics (beginning \foreign{Sagrado pajarillo}) were replaced by another
    text (beginning \foreign{Fecunda planta viva}), which was pasted and sewn
    over the original words with thin strips of paper.  
    The original may still be seen by lifting the strips.
    For another example of the bird trope by this composer, see
    \cref{ch:zaragoza}.
\end{Footnote}
Birdsong had theological importance as the paradigm of music-making in the
natural world.
The widely read Dominican writer Fray Luis de Granada, in his theological
interpretation of the natural world, says that birds reflect the harmony built
into the created world by its divine Creator because they sing purely by nature
rather than by reason, as humans do.%
    \citXXX[luis]
The only artifice to be heard in birdsong was that of God himself.
Using human voices to imitate birdsong, then, prompted listeners to consider
how the artifice of human music reflected the order of creation (this theology
is discussed in more detail below). 

% %{{{5 music CasedaJ Sagrado pajarillo
% \insertMusic{CasedaJ-Sagrado_pajarillo}
% {Bird-like trills in Cáseda, \wtitle{Sagrado pajarillo}, excerpt from the
% estribillo, Tiple I-1}
% %}}}5
%}}}3

%{{{3 instruments: percussion
Next to the musical sounds of animals, the sounds of musical instruments
provided rich possibilities for musical imitation in a theological context.
Wooden sounding boards, brass pipes, and gut strings allowed human players to
take the potential of music built into the created world---such as the perfect
Pythagorean ratios of the overtone series---and actualize them in sound.

To imitate percussion instruments, for example, villancico composers paired
onomatopoetic nonsense words with distinctive rhythmic patterns.
Juan Gutiérrez de Padilla had the chorus of Puebla Cathedral represent the
sound of the castanets and tabor with contrasting onomatopoetic rhythmic
patterns on the words \foreign{al chaz, chaz de la castañuela, y el tapalatán
de el tamboríl} (\cref{mux:Padilla-Alto_zagales-chaz}).
Such pieces about instrumental music imitate the instrument itself while also
playing with a stylistic topic associated with that instrument.

% %{{{5 music GdP Alto zagales chaz
% \insertMusic{Padilla-Alto_zagales-chaz}
% {Gutiérrez de Padilla, \wtitle{Alto zagales de todo el ejido}
% (\sig{MEX-Pc}{Leg. 2/1}, Christmas 1653), estribillo: Imitation of castanets
% and tabor (or tambourine?)} 
%}}}5

The same instrumental trope appears in a villancico poem performed at Toledo
Cathedral in 1645.%
    \footnote{\sig{E-Mn}{VE/88/12, no. 6}.}
Though the music, credited in the poetry imprint to Vicente García, has not
been found, the words alone conjure up a racket of percussion sound:
\begin{quotepoem}
    Porque los instrumentos sonaban así, 
        & Because the instruments sounded like this: \\
    El Atabal, tan, tan ,tan,	    & the drum, tan, tan, tan, \\
    El Almirez, tin, tin, tin, 	    & the mortar, tin, tin, tin \\
    la Esquila, dilín, dilín,	    & the chime, dilín, dilín, \\ 
    y la Campana, dalán, dalán,	    & the bell, dalán, dalán, \\
    Las Sonajas, chas, chas, chas,  & the rattle, chas, chas, chas, \\
    y el Pandero, tapalapatán.	    & and the tambourine, tapalapatán.
\end{quotepoem}
The instruments in this list are simple, rustic noisemakers from everyday
peasant life.%
\begin{Footnote}
    Note that this source from Toledo spells the rattling sound \foreign{chas}
    while Gutiérrez de Padilla's manuscripts from New Spain spell it
    \foreign{chaz}.
    This reflects the different pronunciation of the Andalusian settlers of
    central Mexico, who according to phonetic spellings in the music
    manuscripts pronounced \term{ci}, \term{zi}, and \term{si} all with an S
    sound.
    In Toledo, the first two of those phonemes would start with an English TH
    sound.
\end{Footnote}
In this villancico these instruments, which are described further in the
coplas, join together with the sounds of the mule and other animals, and the
dances of the shepherds.  
This piece, like many villancicos, depicts a scene of common folk rejoicing
after their own fashion in the humble setting of the Bethlehem stable.
The focus here is not on instrumental performance in the present day but on
helping listeners imagine the sounds of the first Christmas---but one cannot
help speculating whether peasants in Toledo might have brought such instruments
with them into the church at Christmas.

Imitating an instrument, though, did not mean that the instrument was actually
used in church; indeed in many cases the situation seems to have been the
opposite.
Despite the fanciful reconstructions that can be heard in modern recordings, no
one has yet provided documentary evidence that percussion instruments were used
in church.
They appear only rarely in images of church ensembles or iconography of angelic
consorts (see \cref{fig:BMV-Montserrat,fig:Correa-Sacristy}), and known
archival records do not record payments to makers or players of these
instruments, the way they do for shawm, dulcian, organ, and harp.%
    \citXXX[evidence needed]
%}}}3

%{{{3 clarines
\subsubsection{Becoming Clarions}

Another common example of the imitative, intermusical type would be the many
pieces that mention the \term{clarín} \gloss{clarion or bugle}, in which the
singers perform patterns that are meant to sound like brass fanfares.
The typical style of clarion evocations may be seen in two examples from the
archive of the Escorial, which holds much of the surviving repertoire of the
Spanish Royal Chapel.
Most clarín pieces do not actually feature written-out clarín parts; in most
cases the instrument is imitated vocally or by other instruments, like
\term{chirimías} (shawms).
Matías Durango's \wtitle{Cajas y clarines} (Drums and bugles) evokes these
instruments with voices and shawms in martial style, as part of a broader
battle topic.%
    \footnote{\sig{E-E}{Mús. 29/15}.}
Durango's clarín topic is strikingly similar to one of the rare surviving
clarín parts from a villancico, in a fragment by the prominent Madrid composer
Sebastián Durón (\cref{mux:durango-duron-clarines}). 
Both are in the same collection of music from the Royal Chapel preserved at the
Escorial.%
\begin{Footnote}
    \sig{E-E}{Mús. 29/15} (Durango), \sig{E-E}{Mús. 32/16} (Durón).
\end{Footnote}
A villancico by José Romero from about 1690, \wtitle{Suene el clarín} (Let the
clarion resound) includes an actual notated part for \foreign{los clarines de
los autos}, that is, for the clarions played in the \term{autos sacramentales}
or public Corpus Christi dramas.% 
\begin{Footnote} 
    \sig{D-Mbs}{Mus. ms. 2914}, edited in \autocite[655--661]{CaberoPueyo:PhD}.
\end{Footnote}
The sung voices layer bugle-like gestures above them, creating a more complex
fanfare than the valveless instruments could play on their own.

%{{{5 music: clarin in voice vs actual, durango/durón
\begin{musicexample}
    \label{mux:durango-duron-clarines}
    \caption{An imitation of \term{clarín} music by voice and shawm compared
    with an actual \term{clarín} part: 
    (1) Durango, \wtitle{Cajas y clarines} (\sig{E-E}{Mús. 29/15}, Tiple I-1,
    estribillo); 
    (2) Durón, \wtitle{Dulce armonía} (\sig{E-E}{Mús. 32/16}, estribillo)}

%    \includeFloatPDF[\floatwidth][0.5\floatheight]{Durango-Cajas_clarines}
%    \includeFloatPDF[\floatwidth][0.5\floatheight]{Duron-Dulce_armonia_clarin}
\end{musicexample}
%}}}5

Perhaps there are few clarín parts because these instruments may not always
have been allowed in church, or perhaps their music was generally improvised by
the class of Spanish community musicians called \term{ministriles}, similar to 
the Lutheran \term{Stadtpfeiffer}.%    
    \citXXX[stadpfeiffer, ministriles (Illari), Praetorius CD]
In any case, the instrument was more important as a symbol than as part of the
chapel ensemble.
The \term{clarín} was used in military, royal, and apocalyptic symbolism as far
back as the allegorical \foreign{clairon} fanfares in the 1454 Feast of the
Pheasant hosted by the ancestor of the Habsburg monarchs, Philip the Fair of
Burgundy.%
\begin{Footnote}
    \Autocites[340--380]{LaMarche:Memoires}{Bloxam:JNV}{Perkins:Patronage15C};
    on the symbolism of this instrument in contemporary Spanish drama, in which
    \term{Clarín} was the name of a comic stock character, see
    \autocite{Damjanovic:Clarin}.
\end{Footnote}
In \wtitle{No temas, no recelas} by another famed Madrid composer, Cristóbal
Galán (from \circa{1691}), the voices represent \term{clarín} music in a scene
of \quoted{heavenly armies} going to battle.% 
\begin{Footnote} 
    \sig{D-Mbs}{Mus.  ms. 2892}, 
    edited in \autocite[555--565]{CaberoPueyo:PhD}.
\end{Footnote}

Imitating the clarion within a battle topic was not always just a spiritual
symbol: it was often used like real bugle fanfares were, to celebrate military
victories, or boost morale in the midst of conflicts.
The anonymous villancico \wtitle{Noble clarín de la fama} states on the cover
page that it was performed \quoted{for the profession of the sisters
\foreign{Señoras} Sor Sagismunda and Sor Jacinta Perpinyà into the Convent of
Santa Clara of Gerona, 1693}.% 
    \footnote{\sig{E-Bbc}{M/772/35}.}
The surname of these siblings (sisters by blood and now by vow) is the name of
Perpignan, capital of the Catalan region of Rosselló, which had become the
French Roussillon after the Peace of the \XXX[Pyrennees] in 1659.
A long struggle over this border territory in the War of the Great Alliance
climaxed in the year this villancico was performed, as the French general
Catinat scored a major victory against the allied powers at Marsaglia.%
    \citXXX[history]
The villancico appears to align Catalan identity with the French cause, as it
praises the \quoted{Catalan Amazons, who have the name of Perpignan}, who
\quoted{seek today good protection for their defense in Francisco}---that is,
they look for protection both to Saint Francis, the probable patron of their
order, and to France.
In enlisting for spiritual battle with Francis, the estribillo suggests, the
sisters themselves are becoming clarions of war.%
\begin{Footnote}
    Excerpt from the estribillo: 
    \foreign{Noble clarín de la fama/ 
    que de vozes te alimentas,/
    toca, toca, alarma, alarma,/
    que dos niñas hoy son aliento
    de tu voz excelsa,/
    Catalanas amazonas,/
    de Perpiñan nombre tienen,/
    pues bella guardia en Francisco,/
    buscan hoy por su defensa,/
    cuidado serafines,/
    resuenen los clarines}.
\end{Footnote}

At this moment of commitment in these young women's lives, coinciding with a
political crisis, the concept of \emph{becoming} a clarion held more
significance theologically than the mere sound of the actual instrument would
have held.
This concept is realized even more completely through musical representation in
a villancico by Juan Hidalgo (1614--1685, composer of the first Spanish operas
for the royal court), \wtitle{Venid, querubines alados}.%
\begin{Footnote}
    \sig{D-Mbs}{Mus. ms. 2895}. 
    On Hidalgo's theater music, see \citXXX[Stein].
\end{Footnote}
In this chamber villancico or \term{tono divino}, the two voices sing that just
as the birds of the dawn are \term{clarines} celebrating the Blessed Virgin, so
too will their own voices become \term{clarines}
(\cref{poem:Hidalgo-Venid_querubines_alados}).
Hidalgo interweaves the two voice parts in rising fanfare gestures that
actually allowed listeners to hear the singers transforming their voices into
\term{clarines} (\cref{mux:Hidalgo-Venid_querubines}).

% %{{{5 poem and music Hidalgo Venid querubines
% \insertPoem{Hidalgo-Venid_querubines_alados}
% {\wtitle{Venid querubines alados}, poem set by Hidalgo (\sig{D-Mbs}{Mus. ms.
% 2895}), copla 5}
%
% \insertMusic{Hidalgo-Venid_querubines}
% {Hidalgo, \wtitle{Venid querubines alados}, duo response at end of each copla}
% %}}}5

These villancicos use the \term{clarín} as a metonym for music-making
generally, subsuming both theological and political aspects of the instrument
and its characteristic style as central to the notion of music itself.
Theologically, they summon a range of scriptural references to brass
instruments---the trumpets of King David's priestly musicians
(\scripture{IChr}{0:0}), the trumpets of the apocalypse
(\scripture{Rev}{0:0}; \scripture{IThess}{0:0})---where clarion-like
instruments were used in earthly and heavenly worship and served as signs of
God's divine authority and calls to attention at moments of God's judgment,
signals of divinely ordained seasons and times.%
    \citXXX[biblical trumpets]
At the same time, we must not overlook the obvious political significance of
clarions in the militaristic society of early modern Spain---meanings that were
also understood in Biblical terms.
Just as the biblical Jericho had fallen by the divine hand at the sound of the
trumpets of Gideon's army (\scripture{Josh}{0:0}), clarion fanfares heralded
the arrival of the Spanish monarch or viceroy, communicated to troops on the
battlefield, and generally proclaimed to the ears what flags and triumphal arches
conveyed to the eyes---the dominion and divinely sanctioned authority of the
Spanish king over his global realms.\citXXX[bible verses]

The clarion, put simply, was a sign of power.
Clarion-themed villancicos, then, depended on the instrument's signification of
power to proclaim and reinforce the sovereignty and authority of the Spanish
church and state.
Under the Spanish \term{padronado} (the Spanish monarch's self-appointed
guardianship by direct rule of the Catholic Church in his realm), church and
state power were aspects of the same governing authority, and together they
administered rewards and punishment in both temporal and eternal domains.
Spanish subjects were taught a theological concept of society in which the
world was stratified in a static hierarchy of types and stations of people, in
a way understood to be harmonious and divinely ordained.

The continued reiteration of power through music, and the conflation of
political and theological symbolism, could signify terror and oppression for
many people but also serve as a consoling reminder to others of the order and
stability of the world.
The right man was in charge, the clarion call announced: God was working
through the rulers of the world to govern his creation; common people were
protected and defended against the forces of evil.
That early modern Spaniards understood evil in the human forms of the Moor, the
Jew, the heretic Lutheran;
that they believed the divinely sanctioned order of society consigned
\soCalled{Indians} and \soCalled{Blacks} for roles of servitude; 
that they believed only males could hold authority and that
an inequitable distribution of wealth was part of the natural hierarchy---the
clarion call signified all this, too.
We can affirm that the \quoted{music of state} in the Spanish Empire served as
a \quoted{instrument of dominion}, even in some ways fulfilling functions that
may be seen as prototypes for later totalitarian propaganda, while also
acknowledging that many Spaniards and their colonial subjects actually believed
in the theological foundations of their political order and even actively
contributed to reinforcing the hierarchical power structure, including through
patronizing, performing, and listening to music.%
    \citXXX[Rodriguez, Sage, Rietbergen, Menache, etc]
%}}}4
%}}}3

%{{{3 dance, ethnic vcs
\subsubsection{Dance and Difference: \term{Jácaras} and Social Class}

Dance topics in villancicos provided another way for the genre to point beyond
itself to other kinds of music in society, and like clarion topics these
references both reflected and reinforced Spanish attitudes toward social
structure.
Many dances are explicitly named and often the text proclaims the villancico
itself to \emph{be} a specific kind of dance, including \term{zarabanda},
\term{jácara}, \term{guarache}, \term{danza de espadas}, and
\term{papalotillo}.%
    \citXXX[sources]
Only a few of these have been corroborated with other notated sources of dance
music, though some of those sources provide only schemata and can only be
reconstructed with a high degree of imagination.%
\begin{Footnote}
    Compare, for example, the elaborate instrumental dances recorded by Andrew
    Lawrence-King and the Harp Consort, \headlesscite{Lawrence-King:DancesCD}
    with the rudimentary notation of a few bars of chords and minimal strumming
    patterns in the source, \shortcite{Ruiz:Luz}.
\end{Footnote}
The question of whether performers or listeners actually danced in church is
another problem here, related to the question of whether or to what degree
performers staged the dramatic villancicos in the sacred space.%
    \citXXX[?]
There certainly was ritual dance on Corpus Christi in Seville and Valencia
Cathedrals, performed by the boy choristers known as \term{seises}.%
    \Autocite{Comes:Danzas}
At present I would speculate that like \term{clarín} pieces with no actual
clarion, dance references in villancicos are not evidence of actual dancing;
their purpose is to call to mind dancing that happened elsewhere and to make
use of the symbolic meanings of dance. % XXX
	
The \term{jácara} (also spelled \term{xácara} but always pronounced with a
guttural H sound) originated as a type of song and dance in Spanish theater and
street performances, typically recounting the deeds of ruffian outlaws in the
rough and sometimes bawdy language of the underworld (a comparison with rap
would not be inappropriate).%
    \Autocites{Torrente:Jacara}{XXX}
Juan Gutiérrez de Padilla included a sacred adaptation of the genre in every
one \XXX[check] of his Christmas villancico cycles for Puebla Cathedral from
1651--1659: these pieces herald the exploits of not a human \term{pícaro} but
the baby Jesus, adapting the outlaw language markers from the worldly genre to
sacred purposes.
Gutiérrez de Padilla's most well-known contribution to this subgenre is
\wtitle{A la jácara, jacarilla} from the 1655 cycle.
As he had done with his previous \term{jácaras}, the Puebla chapelmaster
borrowed the poetic text from the imprint of an earlier Royal Chapel
performance in Madrid (in this case, from the previous year).%
    \citXXX[pliego, Torrente book]
% TODO example
He puts this text to a variant of the same tune that he had used in his three 
preceding \term{jácaras} that survive and the same general style: the main tune
features a stepwise gesture ascending and descending motive
C\sh--D--E--F--E--D--C\sh{} harmonized with \musFig{5 3} chords on the first
and fifth degree of the first mode (to modern ears, this sounds like \term{i}
and \term{V} in D minor).
% TODO example
This matches the basic outlines of the improvised tune type reconstructed by
Álvaro Torrente for the secular \term{jácara} (secular as in worldly and
irreverent, in theme and performance venue).

Like the improvised model, Gutiérrez de Padilla's setting moves rhythmically in
triple meter (\meterCZ) with extraordinarily heavy use of syncopation and
sesquialtera.
As this composer developed this tune in each year's successive setting, he made
the rhythm and phrasing more complex each time.
% TODO examples
For a good portion of the estribillo in 1655, he creates what to current
knowledge is an unprecedented nine-minim pattern of three-measure groups: a
normal group of three minims is followed by a sesquialtera group with pulses in
three imperfect semiminims.
The melody in the coplas defies any attempt at regular rhythmic grouping. 
% XXX

Why would Gutiérrez de Padilla create such a complex rhythmic and polyphonic
setting to represent a dance with such common, even sordid origins?
First, the beginning of the text proclaims a specific intention to bring
contrasting worlds together.
% TODO quote
The contrast between \term{corte} and \term{villa} is between noble and common,
gentility and laborers, urban and rural, refined and crude---notably not sacred
and secular.
It is also a play on the term \term{villancico}, which comes from \term{villa},
and suggests an attempt to say something here about the function and meaning of
the genre as a whole.
Mixing the style and specific motives of a low-life ballad genre with the
techniques of learned counterpoint; in fact using compositional technique to
actually amplify the complexity of the oral source material perhaps
beyond what would more readily be improvised, certainly contributed to this
goal of mixing high and low elements of society.
Theologically the Christmas feast actually centered on the mixture of high and
low, as the infinite and all-powerful God had confined himself to the
vulnerable body of the tiny Christ-child (see \cref{ch:padilla-voces}).
Christ's birth in a feed-trough and his manifestation to lowly shepherds and
heathen magi were also understood in the Spanish context to elevate the dignity
of lower-class people, though typically in way that ultimately reinforced the
social hierarchy rather than challenging it.%
    \citXXX[al establo]
Compared to source material like the \soCalled{Frog \term{Jácara}}, which
catalogs sexual positions in explicit detail, Gutiérrez de Padilla's
representations of Christ as an outlaw---in one piece, a gunslinger from
\quoted{way up in Texas}---seem quite tame, but within the context of what New
Spanish worshippers could hear in church they must have brought some delight
and sense of play into the liturgy.%
    \citXXX[Torrente, playing cards; vc ex]
In the last copla of \term{A la jacara, jacarilla}, the singer tells the
Christ-child, \quoted{We will leave you here with these \term{principios de
romances}}, tipping off listeners who had not yet figured it out that the
preceding set of \XXX[no.] verses were all constructed from the first lines of
traditional \term{romance} ballads.
% TODO table
Here again is a popular practice (again comparable to hip hop) of rapid-fire
quotations riffing on existing texts and reshaping them into new meanings, but
written down and given a fully notated musical setting in a complex, highly
literate manner.

The trickster quality of the typical \term{jácara} hero may also explain the
cryptic texts and puzzling musical patterns: the \term{jaque} was often a
gambler, a swindler, and a quick draw, so the sacred \term{jácara} became a
site for poetic and musical trickery.
Later in the century, pieces called \term{jácara} did not always have the
distinct musical markers connected to the secular source traditions; but they
did retain this sense of playful ingenuity.
Mateo de Villavieja's \term{Jácara en anagramas} (\XXX[date, place], from the
Convento de la Encarnación in Madrid)
features a poetic text written in anagrams, such that the lines and phrases of
one stanza are shuffled to create the next.%
    \footnote{\sig{E-MO}{AMM.4261}.}
\XXX[details]
The music, too, is composed in anagrams: the voice parts are rotated for each
successive copla; as are the phrases. \XXX[details]

The reasons for Gutiérrez de Padilla's rhythmic play with triple meter may be
hinted in a \term{jácara} villancico poem by Manuel de León Marchante.
In \XXX[16XX], León Marchante wrote a set of villancicos for \XXX[Toledo]
Cathedral in which, after an introductory piece, each villancico represented
one of the seven liberal arts.
(The next year he would balance things out with a set on the \quoted{manual
arts}, including sailing, surgery, and blacksmithing.\XXX[check!])
It is fascinating how León Marchante pairs the conventional subgenres of
villancicos with each of the divisions of learning: geometry is a
\term{villancico de naciones} (an \quoted{ethnic} villancico, see below),
because one needs geometry to make maps and navigate\XXX[other examples].
Where does the \term{jácara} appear?
As \term{arithmetic}---because, León Marchante says in the villancico, it is
\quoted{governed by the rule of threes}.
Perhaps there is a connection here to Gutiérrez de Padilla's three-measure
groups of triple meter.%
    \footnote{Perhaps also to his use of the symbolic number 33 in his
    depiction of Christ as a card player, which I have proposed is a
    proto-jácara.}
The jácara, then, would be a game of numbers, celebrating the ultimate
trickster who hid divine identity inside a child's body\XXX[etc].

The cost of turning the jácara into a theologically signicant display of
wit and ingenuity, it would seem, is losing a connection to the lower-class
sources of the genre.
Sacred jácaras became yet another pious entertainment for the educated classes,
perhaps with a bit of added thrill by their association with ribald origins,
but increasingly losing any sense of crossing boundaries of \term{corte} and
\term{villa} in a way that would have had any meaning for residents of the
latter.
%}}}3

%{{{3 ethnic VCs
\subsubsection{Representing Ethnic Difference}

Metamusical references to traditional music-making of lower-class people
extended also to the depiction of ethnic difference.
There are villancicos that depict non-Castilian groups like Native Americans,
African people, Catalans, Frenchmen, even Irishmen, through caricatured
deformations of language and music.
What have come to be called \quoted{ethnic villancicos} were labeled in their
time as \term{villancicos de naciones} or by the name of the particular ethnic
type for that piece, like \term{gallego} (Galician), \term{gitano}
(\quoted{Gypsy}), \term{indio} (\quoted{Indian}), or \term{negro},
\term{guineo}, and similar terms for Africans.
Most of these pieces, and especially the \term{villancicos de negro}, refer
specifically to the characteristic music and dancing of these groups, often
naming their instruments and describing their motions.
The texts use some smatterings of foreign words but mostly ask the performers
to put on an accent in Spanish: in these caricatures the Gypsy ends all her
words with a Z (\foreign{Puez los trez zon Magoz,/ hombrez de ezfera}).
\begin{Footnote}
    \term{Vamos al portal gitanilla}, Imprint from Epiphany 1666, Zaragoza,
    Iglesia de El Pilar (\sig{E-Mn}{VE/1303/1}), later attributed to Vicente
    Sánchez, \headlesscite[203--204]{Sanchez:LiraPoetica}.
\end{Footnote}
The Black says when he should say R and J, drops ending S sounds, and
mismatches genders and cases (\foreign{Mi siñol Manuele, \Dots{} Sesu, \Dots{}
pluque son linda cosa}).%
\begin{Footnote}
    \term{Al establo más dichoso}, Music manuscripts by Juan Gutiérrez de
    Padilla of \term{ensaladilla} for Christmas 1652, Puebla Cathedral
    (\sig{MEX-Pc}{Leg. 1/3\XXX}), \XXX[WLSCM32].
\end{Footnote}
Villancicos about African characters also frequently feature nonsense
syllables, whether lullaby phonemes like \foreign{ro ro ro ro} and \foreign{le
le le le}, or apparent gibberish like \foreign{tumbucutú, cutú, cutú} and
\foreign{gulumbé, gulumbá} that tells us what African languages like Kikongo
sounded like to a Spanish ear.%
    \citXXX[al establo, other]
This type of piece represents Africans as always happily engaged in drumming
and dancing.%
    \citXXX[baker etc]

These pieces were created by Spaniards primarily for other Spaniards;
\quoted{black villancicos} are not really about depicting African identity but
are rather ways of constructing a Spanish one by reference to the Other.
Immediately after purging Iberia of both Moors and Jews, the Castilians had
been overwhelmed with encounters with new ethnic groups, languages, and
religions around the world; these pieces offered the potential to create an
imagined world in which all these groups were situated in their proper place
within a well-controlled social hierarchy.
These pieces may offer glimpses of the language and music of non-Spanish
groups, but only through a glass heavily darkened by racial prejudice and
deliberate caricature for the sake of humor and mockery; they further clouded
by the cultural distance from which modern observers must approach these
pieces.
With regard to the nature of musical references in \soCalled{ethnic}
villancicos, then, these pieces encompass a mixture of literal imitation (as of
percussion, and of the \soCalled{musical} sound of foreign languages) and
broader stylistic references (as, perhaps, to African musical styles, though no
one has yet demonstrated concrete evidence for a connection).
They also include more abstract references to music through the use of nonsense
words that, somewhat like solmization syllables (see below), symbolize and
enact music-making.
Like \term{jácaras}, ethnic villancicos grow increasingly conventionalized over
the years and more distant from the low-caste sources they grew from, so that
the \term{negro} character in one year's villancicos was much more similar to
the \term{negro} of the previous year's set that he probably was to any real
person of African descent.
And like \term{clarín} pieces, ethnic villancicos both reflected and reinforced
imperial Spain's power structure by projecting a theological vision of that
structure as divinely ordained and immobile.
That said, we must note that very few of these pieces have received any serious
scrutiny, especially their music; and that those that are known are not simply
racist caricatures like later minstrel shows in the United States.

Their discourse on racial difference must be understood within a Neoplatonic
theological concept of music and society, in which the lowliest elements in the
created world could lead a person to the knowledge of the highest.
Juan Gutiérrez de Padilla, who included a \quoted{black villancico} in most of
his Christmas cycles for Puebla, depicted the paradox of Neoplatonic thought
when in 1652 he had his black characters, described as Angolans in the piece
like Gutiérrez de Padilla's own slave Juan, say \quoted{Listen, for we are
singing like the angels}.
As the Angolans go on to sing a vernacular \term{Gloria in excelsis} in their
dancing triple meter, full of syncopations notated by coloring in the mensural
noteheads, above them suddenly enter the two boy soprano parts of the second
chorus, which have been silent until now, singing the same \term{Gloria} with
them---but in the white notes of duple meter, and quoting a plainchant
intonation.
The Angolans and the angels are brought together for a miraculous moment
through contrasting types of rhythmic movement in which the hidden harmony
between earthly and heavenly music is revealed.
The Angolans are in some ways depicted sympathetically, as instead
of the gold, frankincense, myrrh of the magi (one of whom was portrayed on
Puebla's high altar as a black African), bring the Christ-child the homelier
and more practical gifts: a potato, a toy, and diapers.
But nothing about the piece really exalts the African characters in any way
that would affect the lives of real Africans like Gutiérrez de Padilla's slave.

It is possible, though more evidence is needed, that the vogue for black
villancicos at Christmas and Epiphany was linked to the practice across the
Spanish and Portuguese Empires of \quoted{Black Kings} festivals, in which
confraternities of enslaved and free people of African descent elected a mock
royal court and paraded them around their city with music and dancing, usually
with military elements with origins in the Christian Kingdom of Kongo before
the start of slavery.%
    \citXXX[sources]
This possible connection does not mean that these villancicos express any real
African voice or viewpoint; rather, they tell us about the insecurities, fears,
and prejudices of Spaniards and may help us understand how they justified their
place in an unjust society by appeal to theology and aesthetic beauty.
Gutiérrez de Padilla's polymetrical Gloria fits perfectly with the image of
angels singing and dancing on the round painted high the new Puebla
Cathedral's high altar, hovering over the images of shepherds and magi greeting
the newborn Christ at the altar's base (see \cref{ch:padilla-voces}), and we
can imagine the theological aesthetics of this were some comfort to this
chapelmaster-priest and his peers; but they were no help to enslaved men and
women and, when such pieces are revived uncritically today, they continue to do
their historic work of reinforcing racial prejudice and appeasing the
consciences of elite (and typically white) listeners.

With regard to hearing and faith, ethnic villancicos and black villancicos in
particular enabled Spaniards to envision themselves as the rightful masters of
a society in which other groups were naturally subordinate; in other words what
they heard helped them believe in the rightness of the social order as governed
by the church. %XXX
Though their representations are purposefully distorted, they do suggest that
the Spanish elite accepted the coexistence of multiple languages and types of
music within society, and enjoyed sampling these exotic sounds through the safe
filter of their own caricatures.
%}}}3

%{{{3 VCs about VCs
\subsubsection{Villancicos about Villancicos}

%{{{4 overview
The conventions of the villancico genre itself become the subject of a special
type of self-referential villancico \emph{about} villancicos.
In one sense, the many pieces beginning \quoted{Listen} or \quoted{Pay
attention}, might be considered self-referential, since in these pieces the
singers usually announce something about the piece, as in the setting of
\wtitle{Oigan, oigan la jacarilla} by José de Cáseda, or the poem performed in
Montilla in 1689, \wtitle{Oíganme cantar una tonadilla}---\quoted{hear me sing
a tonadilla}.% 
    \begin{Footnote}
    Respectively, \sig{MEX-Mcen}{CSG.151}, \autocite[116 (no signature
    listed)]{BNE:VCs17C}.
    See \XXX[LeGuin:Tonadilla].
    \end{Footnote}
This posture rhetorical posture owes something to the genre's close
relationship with the psalms in Matins, which are filled with such
self-referential statements (\quoted{Sing to the Lord a new song},
\scripture{Ps}{97:0}; \quoted{Come, let us worship and bow down},
\scripture{Ps}{95:0}).
But it also draws on the comic and satirical elements of the Spanish minor
theater, the low-register plays (\term{entremeses}) performed between acts of
the more highbrow \term{comedias} by, for example, Lope de Vega and Calderón.%
    \citXXX[entremeses]
Similar to the Italian \term{intermezzi} skits of the eighteenth
century that were the cradle of comic opera, Spanish \term{entremeses} were
built out of formulaic scenarios and stock characters---many of the same ones
like Gil, Pascual, Bras, and Bartolo who appear in villancicos---and parodied
the conventions of the \term{comedia}.

The patrons and creators of villancicos developed well-reinforced expectations
in their audience not only for different types of villancicos, but also
possibly for the dramatic shape of the whole villancico cycle (such as in León
Marchante's cycles on the liberal and manual arts).
The surviving musical settings of complete cycles, such as those in Puebla by
Juan Gutiérrez de Padilla and those in Segovia by Miguel de Irízar
(\cref{ch:faith-hearing}), demonstrate a range of expected musical conventions
for each of these sbgenres.
A conjunction of markers in the poetic subject matter and in the poetic and
musical style would have signaled to listeners, \quoted{This is one about
angels}, or \quoted{Here come those silly shepherds}.
Miguel de Irízar's requests to his chapelmaster peers for more
\quoted{villancicos de chanza}---comic villancicos---suggests the need to fill
out each villancico cycle with certain types of pieces, mixing serious and
comic subgenres.%
    \Autocite[78]{Olarte:Irizar} 
When a villancico represents the performance of a villancico and commentary on
it, we are offered a glimpse of how people listened to villancicos.
%}}}4

%{{{4 ex: Anton Llorente
The anonymous villancico \wtitle{Antón Llorente y Bartolo}
(\cref{mux:Anton_Llorente}) presents two characters from a well-known
\term{entremés} with close links to Cervantes' \wtitle{Don Quijote}, who want
listeners to hear out their complaint about villancicos.
The villancico poem is found in a 1639 Christmas imprint from Toledo Cathedral
and an anonymous musical setting survives from the Convento de la Santísima
Trinidad in Puebla.%
\begin{Footnote}
    \sig{E-Mn}{VE/88/6}, \sig{MEX-Mcen}{CSG.014}.
    The title has been erroneously \quoted{corrected} in the Sánchez Garza
    catalog to \wtitle{Anton, Lorente y Bartolo} despite the clear double L in
    the manuscript (which was never used when a hard L sound was
    intended).\XXX[check]
\end{Footnote}
The more well-known stock characters Gil and Bras, they say, have held the
stage for too long:
\begin{quotepoem}
    Antón Llorente y Bartolo	& Antón Llorente and Bartolo \\
    trazaron un memorial	& drew up a complaint \\
    de que con los villancicos	& that with all the villancicos \\
    se han alzado Gil y Bras.	& Gil and Bras have gotten the spotlight.
\end{quotepoem}
Anton Llorente and Bartolo insist that they could make a good enough villancico
of their own if given the chance:
\begin{quotepoem}
    Si ha de sonar el pandero,	& If the tambourine is going to be played, \\
    solo Gil le ha de tocar,	& it is only Gil who ever plays it, \\
    y si ha de haber castañetas,& it if there have to be castanets, \\
    ha de repicarlas Bras.	& Bras is the one to rattle them. \\
    También acá somos gentes	& But here we are, we too are good fellows, \\
    y alcanzar podemos ya	& and we can even manage \\
    de un villancico un bocado	& a nibble of a villancico \\
    y un pellizco de un cantar.	& and a pinch of a song.
\end{quotepoem}
In the succeeding \term{responsión} section, the full eight-voice chorus joins
in endorsing the new characters and denouncing the old:
\begin{quotepoem}
    No quiero que me Brasen y que me Gilen 
    & I don't want them to Bras me or Gil me \\

    sino que me Llorenten y me Toribien. 
    & but only to Llorente me and Toribio me.
\end{quotepoem}

The anonymous musical setting for this embodies all the stereotypes of the
villancico genre, first encountered here in Gutiérrez de Padilla's \wtitle{En
la gloria de un portalillo}.% 
\begin{Footnote}
    One possible composer is the Seville Cathedral chapelmaster Fray Francisco
    de Santiago, whose setting of this text was cataloged as part of the
    now-lost library of King John (João) IV of Portugal (see
    \cref{ch:padilla-voces}), 
    \autocite[caixão 26, \range{no}{675}]{JohnIV:Catalog}.
\end{Footnote}
The piece is in highly accented triple meter (\meterCZ) with continual use of
sesquialtera, and it opens with a declamatory section for full chorus, followed
by a vocal solo that is then echoed in polychoral dialogue by the full
ensemble.
The text is both dramatized and symbolized by leaping gestures that leap from
voice to voice in points of imitation on \foreign{salten y brinquen} (jumping
and leaping).
These features may just constitute typical villancico style, or they may be
taken to \emph{represent} typical villancico style.
The highly conventional music casts the anticonventional text into relief while
also dramatizing the scene, since the piece is meant to portray Anton Llorente
and Bartolo performing a villancico.
This is a villancico, then, in the style of villancicos.

% TODO	See also in lafragua connection; uber-conventional villancico next to
% anticonventional one

% TODO is there a Lisbon Anton Llorente?

% %{{{5 music Anton Llorente
% \insertMusic{Anton_Llorente}
% {Anonymous, \wtitle{Anton Llorente y Bartolo} (\sig{MEX-Mcen}{CSG.014}), first
% stanza of introducción and beginning of responsión (Accompaniment omitted)}
% %}}}5

As though the 1639 Toledo text were not self-referential enough, the creative
team at the cathedral followed up the next year with another villancico that
specifically referred back to \wtitle{Anton Llorente y Bartolo}.%
\begin{Footnote}
    \ptitle{Quejosos de la sentencia que dio el alcalde Pasqual}, in imprint
    from Christmas 1640 at Toledo Cathedral, \sig{E-Mn}{VE/88/7,
    \range{no}{2}}.
\end{Footnote}
The narrator says that the \foreign{Brases} and \foreign{Giles} were so
\quoted{frustrated by the sentence that Mayor Pasqual decreed against them last
Christmas}, that \quoted{they appealed to another one who was more learned}
(the \quoted{Mayor of Bethlehem} was another stock character in comic
villancicos).
Each one states his case for why he is needed at the Nativity, and Bras's
conclusion wryly sends up the conventionality of villancico poetry:
\begin{quotepoem}
Cuanto ha qué Belén lo es,	& As long as Bethlehem has been what it is, \\
y ha sido el portal portal,	& and the stable has been a stable, \\
a peligros de poetas		& where poets have been in danger, \\
ha sido socorro Bras.		& Bras is always there for aid.
\end{quotepoem}
The new mayor, in the name of keeping traditions, undoes the sentence of the
previous year, and the chorus rejoices, because without Bras and Gil it would
not be Christmas:
\begin{quotepoem}
Que me Brasen, y Gilen	& I wish them \\
quiero zagales,		& to Bras me and Gil me, lads, \\
porque no soy amigo	& because I am no friend \\
de novedades.		& of novelties.
\end{quotepoem}
The chorus, speaking here for the mayor's subjects in the community, affirms
the decision to keep their familiar Christmas characters:
\begin{quotepoem}
Porque en saltando a esta fiesta & For if you take from this feast \\
el pesebre, y el portal,  	 & the manger, the stable, \\
las pajas, Brases, y Giles, 	 & the straw, Brases, and Giles, \\
no es fiesta de Navidad.	 & it is no festival of Christmas.
\end{quotepoem}
Here we have a scene of people clamoring for villancicos with all their corny
conventions as a central part of making Christmas feel like Christmas.

As the mayor says, one reason villancicos were so conventional may be because
the feast they were most closely associated with was (and is) one where customs
are carefully preserved.
Novelty at Christmas is expected but only within certain traditional
boundaries.
Part of cultivating those customs meant naming them explicitly in song, as we
have already seen in Cererols's \wtitle{Fuera que va de invención} and several
other pieces, like a North American Christmas tree ornament in the shape of a
Christmas tree.
Comic villancicos like these should not be written off as less theologically
motivated than the more cultivated pieces.
Though the \wtitle{Anton Llorente} pieces present no learned theological
doctrines, they still serve a religious function in prompting hearers to
laughter and enjoyment, and that function contributed to the effect of a set of
villancicos within the liturgy.
The comic pieces may even have been the most likely to provoke direct responses
of laughter and surprise in many listeners, and therefore could be the most
effective in actually moving those assembled toward sympathetic vibration and
harmony together. 
They also served the practical goal of attracting and pleasing parishioners and
making them feel at home in the church---a purpose that might even be
considered more truly theological in the sense of fulfilling the church's
religious purpose as a faith community rather than projecting theological
concepts.%
\begin{Footnote}
    Contemporary Catholic theologian David Fagerberg argues that the first
    and foundational meaning of theology as an activity---what he calls
    \term{theologia prima}---is what the simple worshipper does during the
    liturgy.
\end{Footnote}
%}}}4
%}}}3
%}}}2

\endinput

%{{{2 abstract references
\subsection{Abstract References to Music as Concept or Symbol}

%{{{3 overview
In the second category of metamusical villancicos are pieces that refer to music more as an abstract concept, rather than to a specific, identifiable reference to another kind of music.
When Pedro Ruimonte in \wtitle{Gil, pues a cantar} sets the word \foreign{cantar} \gloss{sing} to a long melisma, or when Gaspar Fernández in \wtitle{Sobre bro canto llano} illustrates the phrase {canto llano} \gloss{plainchant} with imitative counterpoint around a cantus-firmus-like Tenor part, the composer is using these characteristic emblems of vocal music to refer to the concept of singing in general.%
	%
	\footnote{%
	See note~\ref{fn:Ruimonte} above.
	}
	%
In doing so they are asking their singers to sing about singing.

Many villancico poems use solmization syllables in the poetry as part of a reference to singing.
References to Christ as \term{sol} \gloss{sun} are ubiquitous, and as shown in the opening example by Padilla, composers missed no opportunity to put this word on a pitch that could be solmized with that syllable (G, C, or D in the three Guidonian hexachords).
The more obvious this technique was, the better: in composer Miguel de Aguilar's \term{oposición} \gloss{audition} piece for a position at Zaragoza, \wtitle{Mi sol nace y tiembla} \gloss{My sun is born and is trembling}, it is not hard to guess the opening pitches.%
	%
	\footnote{%
	E-Zac:~B-11/233, \quoted{Villancico de Oposición en Zaragoza}, edited in \autocite[34--64]{Ezquerro:MME55}.
	}
	%
Solmization syllables were sometimes used for their own sake, without a symbolic meaning, somewhat like the \quoted{fa la la} refrains in contemporary English madrigals; in these cases the vocalists are \soCalled{singing about singing} in the most simple sense.
The voice in such cases bears no message except the musical voice itself.

These gestures, though seemingly without meaning, must be understood within a Neoplatonic system, which will be explained fully in the next chapters.
In the prevailing Catholic understanding of the world, every created thing, simply by being itself, reflected the nature of God, its Creator.
The human body was the microcosm, reflecting in turn both the whole Creation and the Creator who took on a human body in Christ.
The voice emanated from the body and expressed the essence of the one speaking or singing to another who heard it.
So the voice itself had quasi-sacramental meaning as an expression of Man the microcosm and a reflection of the Creator; and this meaning was independent of linguistic communication or even of music's own non-linguistic structures, which were understood by analogy to rhetoric.%
	%
	\footnote{%
The Victorian Catholic priest Gerard Manley Hopkins would later encapsulate this idea in a striking verse, drawing on the Neoplatonic ontology of Duns Scotus (Eriugena), \quoted{Each mortal thing does one thing and the same:/ Deals out that being indoors each one dwells; Selves---goes itself; \emph{myself} it speaks and spells;/ Crying \emph{What I do is me: for that I came.}} \quoted{As kingfishers catch fire}, in \autocite[95]{Hopkins:Poems}.

This theological view of the nature of the pure voice might be fruitfully contrasted with the \quoted{aesthetics of pure voice} that Mauro Calcagno has identified in Venetian productions of the Accademia degli Incogniti, \headlesscite{Calcagno:SignifyingNothing}.
These early modern perspectives should inform more recent discussions of philosophy of voice, particularly the idea of the \soCalled{voice itself} as separate from meaning, from \autocite{Barthes:GrainOfVoice} through \autocite{Dolar:Voice} and \autocite{Cavarero:Voice}.
	}
	%

Passages of self-conscious solmization are not alluding to a particular kind of song; rather, their song is pointing to the abstract category of \soCalled{singing} in general.
In a piece like Aguilar's \wtitle{Mi sol nace}, the words have dual function: on one side they communicate linguistic meaning (\quoted{my sun}), but on the other side these musical syllables go beyond language, to both symbolize and embody music-making. 
Aguilar made this obvious gesture at the beginning of a piece intended to demonstrate his own skill at composition; this strengthens the thesis developed throughout part~\ref{part:Singing} that metamusical villancicos as a subgenre served composers as \soCalled{master pieces} to prove their craft.
At the same time, the syllables could also take on deep symbolic meanings, as in the Christological \quoted{sign of A \term{(la, mi, re)}} discussed in chapter~\ref{ch:Padilla-Voces}.
%}}}3

%{{{3 heavenly, angelic music
\subsection{Pointing to a Higher Music: Topics of Heavenly and Angelic Music}

%{{{4 overview
When a villancico refers to the music of the spheres or to angelic music, the music signified is impossible to hear with earthly ears, so the human music as a sign is only an icon or index to the extent that a listener believes it to correspond to what those higher forms of music might actually sound like.
These pieces depend to a high degree on conventional---symbolic---ways of evoking heavenly music, which are developed over time within an interpretive tradition (as shown in part~\ref{part:Singing}).

One of these conventional ways of evoking heavenly music is to set up a contrast between indexical references (stylistic allusions or quotations) pointing to types of human music with different value in a hierarchy of musical styles.
Villancicos on topics of angelic and heavenly music provide an interesting case for a semiotic analysis, since they use references to elevated forms of human music in order to refer to an unheard higher music of heaven.
Similar to the way German Lutherans used learned counterpoint to symbolize heavenly music (as David Yearsley has shown), Spanish Catholics used old-style contrapuntal music, particularly canons and fugues, to point to higher Neoplatonic levels of music.%
	%
	\footnote{%
\autocites{Yearsley:Buxtehude}{Yearsley:BachThron}.

This topic of heavenly music is one of the most potentially fruitful areas for interconfessional research, since the ideas Yearsley discusses may be found in similar form in Lutheran, Catholic, and Anglican sources. 
For example, Lutherans mapped the relationship of the boys' choir to the congregation onto that between the angelic chorus and the church; see \autocite{Cashner:Gerhardt} and cf. chapter~\ref{ch:Puebla}.
	}
	%
Most typically, Hispanic composers use polyphonic techniques and styles reminiscent of Palestrina, Guerrero, and Morales to represent angelic music, as demonstrated in chapter~\ref{ch:Cererols}.

In this way a form of earthly music is placed relatively higher on a Neoplatonic chain and used to symbolize an even higher level of music. 
Each style by itself functions as an index pointing to these styles of earthly music, such as (on the higher level) a fugue in duple meter, indexing typical Latin liturgical music in Spain; and (on the lower level) the homophonic declamation in triple meter typical of vernacular villancicos.
The contrast between these two styles, however, functions symbolically, so that the relationship of higher and lower forms of earthly music is mapped onto the imagined relationship between heavenly music and all earthly music.%
	%
	\footnote{%
	See chapter~\ref{ch:Cererols} and \ref{ch:Zaragoza} for detailed examples of this practice.
	}
	%

This is only a simplified description of how heavenly music is represented, however; the stylistic references do not usually form such a tidy dichotomy.
The question of heavenly music is also complicated by the possibility of multiple kinds of \soCalled{superterrestrial} music: the perpetual song of the angelic choirs in heaven, the song of the redeemed at the Last Day, the harmonies of the cosmic spheres, and so on.
It is particularly important to distinguish between the \soCalled{the heavens} (\term{cielos} in Spanish), meaning the dome of the sky and the planetary spheres, and \soCalled{Heaven} (\term{el cielo Empyreo}), meaning the spiritual realm outside of the material world where the Godhead dwells with the angels and saints.
%}}}4

%{{{4 angelic trope, salazar
\subsubsection{The Angelic Trope: Salazar, \wtitle{Angélicos coros}}

A typical example of the angelic trope is \emph{Angélicos coros con gozo cantad} (MEX-Mcen: CSG.256), a Christmas villancico by Antonio de Salazar (\circa1650--1715), preserved in a collection from a Conceptionist convent in Puebla de los Ángeles (example~\ref{ex:Salazar-Angelicos_coros-1}).%
	%
	\footnote{%
	See appendix~\ref{app:poems} for the complete poem, and appendix~\ref{app:scores} for the complete musical edition.
	
Salazar was probably born in Puebla and may have sung in the Puebla Cathedral chapel under Juan Gutiérrez de Padilla; he served as chapelmaster of Mexico City Cathedral from 1679: \autocite{Koegel:Salazar}. 
The Sánchez Garza collection features numerous pieces by Salazar, probably composed or arranged specifically for this convent.
	}
	%
The anonymous poem echoes the first Responsory of Christmas Matins (\quoted{Gaudet exercitus Angelorum}) as it invites the choirs of Christmas angels to sing their \quoted{Gloria} over the stable in Bethlehem on the night of Christ's birth.
Since \mentioned{Bethlehem} in Hebrew means \quoted{House of Bread}, the villancico also celebrates the sacramental presence of Christ in the Eucharistic host on the Christmas of Salazar's \soCalled{present day}.

Though the words speak to the angels, the musicians who sing these words also play the part of the angels, so that hearers are invited to listen for the angelic voices \emph{through} the voices of the church ensemble. 
The invocation to the angels is sung first by the Tiple I, in a gesture beginning with a rising fifth and then falling by step, as though looking up to the heavens and then following the angels' descent.
In the Puebla convent choir, this part was performed by \quoted{Madre Andrea}, whose name is written into her part.
As though answering the call, the other two voice parts of Chorus I enter in \range{\measures}{2}, Tiple II in canonic imitation, and Alto I harmonizing with it homorhythmically. 
In \range{\measures}{4-5} the second chorus joins with a similar imitative pattern, until all join together in a lilting, dancelike cadence on \foreign{cantad}.
Salazar uses contrapuntal imitation again on \foreign{celestes esquadras}, inverting the opening motive (\range{\measures}{14-22}).
For the command \foreign{bajad} \gloss{come down}, Salazar switches from CZ triple meter to duple (C or \term{compasillo}), and creates a cascading contrapuntal passage passed from voice to voice, moving from high F\octave{5} down to C\octave{3} (example~\ref{ex:Salazar-Angelicos_coros-2}).
The general affect of the piece seems gentle and sweet, partly because of the largely static diatonic harmony and the lilting or dotted rhythms.

% %*******************
% \begin{example}
% \includegraphics[width=\linewidth]{scores-examples/Salazar-Angelicos_coros-ex1}
% \caption{%
% Salazar, \wtitle{Angélicos coros con gozo cantad} (MEX-Mcen: CSG.256), \range{\measures}{1-9}
% }
% \label{ex:Salazar-Angelicos_coros-1}
% \end{example}
% %*******************
% 
% %*******************
% \begin{example}
% \includegraphics[width=\linewidth]{scores-examples/Salazar-Angelicos_coros-ex2}
% \caption{%
% Salazar, \wtitle{Angélicos coros con gozo cantad}, \range{\measures}{22--31} 
% }
% \label{ex:Salazar-Angelicos_coros-2}
% \end{example}
% %*******************
% 

All of these musical characteristics are typical ways that villancico composers represented angelic music: especially contrapuntal imitation, in a reference to the ordered music of heaven, and symbolic patterns of ascent and descent.
Salazar uses different styles of earthly music---particularly the contrast between contrapuntal and homophonic styles---to point to the contrast between different levels of music on a cosmic scale, between human music, music of the spheres, and angelic song.
Because the triple-meter style of the first section, which asks the angel choirs to sing, is more typical of villancico style, this part might be heard to represent the actual singing of the angels.
The duple-meter section on \quoted{bajad} might be understood as a more literal portrayal of the angels themselves.
There is not, though, any obvious one-to-one mapping of style to symbol.
It would be difficult to fit such a piece into a Peircian model.

Understood in a more historical model of theological symbolism, the piece connects Boethian \term{musica instrumentalis} to the higher forms of human and cosmic music. 
\soCalled{Heavenly} villancicos map a lower level of music onto a higher one within the Neoplatonic cosmos, in which the perceptible \soCalled{world of change and decay} is an imperfect reflection of a higher world of ideal forms.
%\citX{Augustine De doctrina christiana on signs,symbols, sacramental presence? Salazar piece connecting sacrament of Xmas with sacrament of EuX with multiple kinds of musical presence}
Thus earthly music of any kind, metamusical or not, would always point beyond itself to higher forms of music and ultimately to God.

Metamusical pieces intensify this aspect of music by calling the listener's direct attention to the artifice of the music itself.
Such pieces give listeners the opportunity to rise in Neoplatonic contemplation from what is heard by the ears to the higher music (ultimately of the divine nature) that can only be discerned by the soul through faith.
%}}}4
%}}}3

%{{{3 music itself as a conceit
\section{Music Itself as a Conceit}

Villancicos, we have seen, may refer to other kinds of music or even to themselves; a special subgenre of villancico makes music itself the governing conceit for the whole piece.
Such pieces often play on technical musical terms to create a double discourse about both music and theology.
The most renowned of villancico poets today, Sor Juana Inés de la Cruz (1651--1695), used the conceit of Mary as a heavenly chapelmaster to create such a piece for the feast of the Assumption in Mexico City, 1676, though no musical setting survives.%
	%
	\autocite[no.~220, p.~7]{SorJuana:VC}
	%
The estribillo exhorts congregants to listen for Mary's voice (poem~\ref{poem:Silencio-Maria-Sor-Juana}).
The coplas demonstrate how much theology could be drawn from musical terms, and how much knowledge of both domains is necessary to understand both sides of the concept.

%%*******************
%\begin{expoem}
%\caption{Sor Juana Inés de la Cruz, \emph{Silencio, atención, que canta Mariá}, excerpts}
%\label{poem:Silencio-Maria-Sor-Juana}
%	\input{poems/Sor_Juana-Silencio_atencion_Maria}
%\end{expoem}
%%********************

As the succeeding chapters will show, when poetry like this was set to music, composers had the opportunity to match this intricate musical-theological discourse with another layer of symbols in the sounding music.
It should be kept in mind that villancico poems were written specifically as lyrics for musical compositions, as Juan Díaz Rengifo stated in one of the first literary descriptions of the genre.
	%
	\autocite{Rengifo:ArteMetrica}
	%
Poems like Sor Juana's circulated independently of musical settings through the medium of the printed commemorative poetry leaflets, which composers circulated widely across the empire.%
	%
	\footnote{%
	Chapter~\ref{ch:Segovia} demonstrates how Segovia chapelmaster Miguel de Irízar composited the poetic texts for his 1678 Christmas cycle from several poetry leaflets sent to him by colleagues.
	}
	%
Composers had every reason to favor villancico poems that gave them with opportunities for clever musical craftsmanship, and musically knowledgable poets like Sor Juana were motivated to provide them.%
	%
	\footnote{%
	On Sor Juana's musical knowledge, see \autocite{Stevenson:SorJuanaMusicalRapports}.
	}
	%
It is also possible that in many cases the composers themselves wrote the poetic texts, in which case they likely already had ideas for the musical setting.

Sor Juana is writing within a Spanish literary tradition of \term{conceptismo}, in which poets, especially those under the spell of Luis de Góngora, developed poems from ingenious extended metaphors.%	
	%
	\footnote{%
	On \term{conceptismo}, see \autocites{Tenorio:Gongorismo}[227--228]{Gaylord:Poetry}; for the most important period source, see \autocite{Gracian:Ingenio}. 
	}
	%
In the most finely wrought villancicos within this Gongoresque tradition, such as those studied in chapters~\ref{ch:Padilla-Voces} and \ref{ch:Cererols}, the whole poem can be read in two ways simultaneously, so that the poem says something meaningful about music while also using the musical terms to speak metaphorically about theology.%
	%
	\footnote{%
	Stevenson ventures a theological and musical interpretation of this villancico in \headlesscite[16--17]{Stevenson:SorJuanaMusicalRapports}.
	}
	%
Many metamusical poetic texts, however, work at a simpler level, providing the composer with an excuse to play with musical techniques, but not necessarily making any profound theological statement.
%}}}3

%{{{3 G d Padilla, symbolism of solfa
%}}}3

%{{{3 sounding number
\subsection{Sounding Number: Hidalgo}

One such piece is \wtitle{Cuando el Alba aplaude alguno} (D-Mbs: Mus. ms. 2895), a villancico for three voices (probably soloists) and continuo by Juan Hidalgo (1614--1685), who was court harpist to Philip IV and the cocreator with Calderón of the first fully sung Spanish music dramas.%
	%
	\footnote{%
See \autocite{Stein:Songs}.
This piece survives today in Munich, as part of a group of villancicos by composers associated with the royal musical institutions in Madrid and purchased by a German collector during the nineteenth-century Romantic vogue for Golden Age Spain.
See Bernat Cabero-Pueyo's study of this collection, \headlesscite{CaberoPueyo:PhD}.
	}
	%
The style of this piece, with single voices imitating themselves in sequential patterns, recalls earlier Italian sacred concertos.

Though the piece bears the devotional designation \foreign{Santissmo y Nuestra Señora} \gloss{for the Eucharist and Our Lady}, the text of the estribillo has little theological content (poem~\ref{poem:Hidalgo-Cuando_el_alba}).
The \quoted{dawn} here, as in many villancicos, is an epithet for the Virgin Mary, and the \quoted{sun}, for Christ (since the rising sun is \quoted{born} out of the dawn).
Hidalgo, as expected, has the Tenor sing the word \term{sol} on the proper pitch; in fact he does it twice in a row, first in the hard hexachord with G \term{(sol, re, ut)}, and then in the soft hexachord on C \term{(sol, fa, ut)} (example~\ref{ex:Hidalgo-Cuando_el_alba_aplaude-1}).
In the poem, Christ is \foreign{uno} as Mary's firstborn and God the Father's only-begotten; Christ is \foreign{dos} because he is both divine and human; and the triune God is \foreign{tres}.
But the main point of these theological symbols, it seems, is to justify a play on numbers in the musical setting.

%%*******************
%\begin{expoem}
%\caption{\wtitle{Cuando el Alba aplaude alguno}, estribillo set by Juan Hidalgo (D-Mbs: Mus. ms. 2895)}
%\label{poem:Hidalgo-Cuando_el_alba}
%	\input{poems/Hidalgo-Cuando_el_alba}
%\end{expoem}
%%*******************

Indeed, Hidalgo's music is a rather elaborate musical game of numbers.
On one level, the numbers determine how many voices are singing and what they sing. 
The Tenor delivers the theological prompts for each number, and the other voices answer with the lines about singing, each time with the number of voices specified in the poem.
The accompaniment part primarily functions as an independent continuo line, but on the phrase \foreign{cantar a tres} it shifts to function like a \term{basso seguente}, doubling the Tenor line, and reducing the effective texture to three real voices.%
	%
	\footnote{%
While it was common in Hispanic villancicos for the bass line to shift in this way---apparently the resulting parallel octaves and unisons were not considered a violation of contrapuntal practice---in this particular case, the shift in function causes the listener to hear only three distinct voices on \foreign{cantar a tres.}
	}
	%
For \foreign{a dos a modo de uno} (\range{\measures}{22--24}, example~\ref{ex:Hidalgo-Cuando_el_alba_aplaude-1}), only two singers sing these words, while the third continues singing \foreign{cantémosle}; and these two voices are in canon at the fourth---so that in a sense the two voices are singing the music of one voice.
In the next line, \foreign{o a uno a modo de tres}, Hidalgo first has one voice sing this, then passes it through an imitative polyphonic texture for all three voices where only one voice of three ever sings these words at a time.

%%*******************
%\begin{example}
%\includegraphics[width=\linewidth]{scores-examples/Hidalgo-Cuando_el_alba_aplaude-ex1}
%\caption{%
%Hidalgo, \wtitle{Cuando el Alba aplaude alguno}, \range{\measures}{20-29}
%}
%\label{ex:Hidalgo-Cuando_el_alba_aplaude-1}
%\end{example}
%%*******************
%
Hidalgo's numbers game is most interesting at the level of rhythm, because the composer plays with the numeric possibilities of triple meter in every way conceivable.
The manuscript partbooks are distinguished by a very high proportion of black (\term{coloratio}) notation.%
	%
	\footnote{%
	See the preface for the theory and transcription of \term{coloratio} notation.
	}
	%
Coloration already suggests a numbers game, where $2:3$ is exchanged for $3:2$, but Hidalgo goes beyond conventional sesquialtera by writing colored passages that extend through as many as five compases (Tenor, \range{\measures}{2-4}; Acomp., \range{\measures}{41-43}).
The phrase \foreign{cantar a dos} in \range{\measures}{12-15} is sung from colored notation so that the two vocal lines singing these words, taken by themselves, sound like they are singing in duple meter, in a steady succession of imperfect semibreve stresses.
For \foreign{cantar a tres}, by contrast (\range{\measures}{16-19}), Hidalgo uses a short--long pattern (colored minim--semibreve) that was typical of triple meter.%
	%
	\footnote{%
	Lorente says that in CZ meter, the hand falls on the first minim and rises on the second (not the third); so this short-long rhythm is paradigmatic of the meter. \headlesscite[165--166]{Lorente:Porque}.
	}
	%
Hidalgo uses a textbook example of sesquialtera on the Tenor's phrase \foreign{a uno a modo de tres} \gloss{one in the mode of three} (\range{\measures}{24-25}, example~\ref{ex:Hidalgo-Cuando_el_alba_aplaude-1}), and this is fitting, since the pattern here divides one breve into three imperfect semibreve stresses.
Building to a climax at the end of the estribillo (example~\ref{ex:Hidalgo-Cuando_el_alba_aplaude-2}), Hidalgo creates overlapping sesquialtera groups in the voices as they imitate each other (\range{\measures}{35-37}), like a \soCalled{sesquialtera stretto}.
He follows this with a long passage of coloration (\range{\measures}{38-39}) which does not divide evenly; instead the stresses fall in irregular groups of two and three minims.
All this settles down to a perfect breve on the word \foreign{uno} in the two Tiple voices and accompaniment.

%%*******************
%\begin{example}
%\includegraphics[width=\linewidth]{scores-examples/Hidalgo-Cuando_el_alba_aplaude-ex2}
%\caption{%
%\soCalled{Sesquialtera stretto} in Hidalgo, \wtitle{Cuando el Alba aplaude alguno}, \range{\measures}{35-41}
%}
%\label{ex:Hidalgo-Cuando_el_alba_aplaude-2}
%\end{example}
%%*******************
%
There is much more musical subtlety here than this brief analysis conveys.
The work might even be seen as a demonstration of rhythmic techniques in triple meter, and a deeper study would reveal much more than could be gleaned from the theory treatises alone.
Hidalgo's piece, then, is a musical discourse on music itself.
%}}}3
%}}}2

%{{{2 section conclusions: sign and signified
\subsection{Sign and Signified}

The previous discussion has distinguished between ways of referring to music (imitative or abstract), and between distinct kinds of music referred to (dance styles, birdsong).
These are semiotic distinctions, and they articulate different relationships between sign (the music performed) and signified (the other music to which the performed music refers).
The relationships between sign and signified in villancicos about music may be distinguished by a loose application of C. S. Peirce's semiotic terms \term{icon}, \term{index}, and \term{symbol}.%
	%
	\footnote{%
The terms are used here as Peirce defined them in 1903, according to the synthesis of Albert Atkin, as a trichotomy of ways that the object of a sign worked in the process of signification.
\autocite{Atkin:Peirce}.
Our primary goal is not to further Peircian semiotic theory, but to use Peirce's distinctions to clarify the function of musical signs in metamusical villancicos.
See also \autocite{Turino:Signs}.
	}
	%
To be an icon, the sign must \quoted{reflect qualitative features of the object}, as in \quoted{portraits and paintings}.
An index utilizes \quoted{some existential or physical connection between it and its object}, such as \quoted{natural and causal signs} (smoke as a sign of fire) and also \quoted{pointing fingers and proper names}.
As a symbol, the sign utilizes \quoted{some convention, habit, or social rule or law that connects it with its object}, as in most speech acts.%
	%
	\autocite[§3.2]{Atkin:Peirce}
	%

A bird-like trill gesture in vocal music functions as an icon, then, to the extent that the listener connects it with the actual sound of birdsong; likewise for imitations of the \term{clarín} or castanets. 
The same gesture always also functions as a symbol, especially the more stylized and conventional it is; in other words, the trill reminds the listener of birdsong because it sounds like similar gestures in other pieces of music that also imitate birdsong.

In the case of the \term{clarín} pieces, though, the piece refers not only to the sound of that instrument but to the type of music that instrument usually plays and thus would be an index.
A clearer example is a piece that names a specific type of dance music and seems to quote or allude to that style of music.
In the few cases when we can actually identify the stylistic referent in contemporary collections of dance music, we can verify that there is a \soCalled{factual} correspondence between the sound of the villancico and the sound of the dance style mentioned in the villancico.
This type of reference is closest to Peirce's notion of index.

In the abstract type of metamusical villancico, these relationships become much harder to track.
In a villancico that literally refers to itself, as in the many pieces that begin by inviting the audience to listen to the piece currently being performed, the piece itself is both sign and signified.
In a villancico that refers to human music as an abstract category, as in the solmization example mentioned above, the music heard symbolizes the notion of music in the abstract.
%}}}3



%}}}2
%}}}1

%{{{1 theology of music

%{{{2 vcs about music as key to theological understanding of music
%**************************************
\section{%
Villancicos about Music as a Key to Theological Understanding of Music
}

Villancicos on the subject of music encapsulate theological understandings of music within musical performance.
These pieces offer a modern interpreter more than just verbal explanations of music, such as may be found in music-theory treatises or doctrinal statements; rather, they provide the opportunity to hear how early modern musicians created a true \term{musical theology}---a form of music that embodies the beliefs it proclaims.

These villancicos flourished during a time in which understandings of faith, hearing, and the power of music were rapidly changing.
The Renaissance and Reformation had brought new attention to human perception and feeling, a new concern with rediscovering the power of music over the human body and over society about which the Greeks had written so much.
Music was being employed not only as a beautifully ordered adornment, but as a means to moving the affects of listeners.
And new methods of musical rhetoric were transforming musical meaning from being understood as primarily symbolic and objective (where the meaning was inherent in the musical structures) toward a more dynamic, experiential model based on associations of figures, gestures, and stylistic topics, and dependent on communication between musicians and listeners through conventions.

The new discoveries in physiology and astronomy that were contributing to the nascent Scientific Revolution changed understandings of music as well.%
	%
	\autocite{Gouk:Sciences}
	%
The new scientific empiricism did not support the traditional theory of music based on harmonies between celestial spheres, the four elements, and the four humors of the body.
At the same time, though, Spanish Catholic poets and musicians continued to represent music according to the old system, even emphasizing the traditional cosmology more strongly.

In the Hispanic world, the contexts for hearing sacred music were changing, especially in the case of the villancico, as this courtly genre became part of church festivities, and developed from a form of intimate, aristocratic chamber music into a dramatic and expansive public genre.%
	%
	\footnote{%
	Corresponding to this development was a parallel growth of sacred chamber songs (\term{tonos divinos}).
	}
	%
In Spain, musicians toiled to keep up appearances of Habsburg splendor in the midst of continual wars, economic depression, famine, and plague.
In colonial America, the Spaniards' task shifted from evangelization and conquest toward civilization-building, in which the economic and social structures of music pedagogy, in cathedral schools and seminaries, played an important role.

Recent scholarship is increasingly demonstrating how central music was to Spain's imperial and colonial project.
Bernardo Illari, Geoffrey Baker, and David Irving have all interpreted Spanish colonial music as both reflecting and enacting hierarchical, Catholic, colonial society.%
	%
	\autocites{Illari:Polychoral}{Baker:Harmony}{Irving:Colonial}
	%
As these scholars have shown, Hispanic Catholics ritually performed their changing identities as members of the Church and subjects of the Spanish crown. 
Through sacred villancicos, these subjects also gave sounding expression to their faith, using music as a way not only to form earthly identities but also to establish connections with the divine.

Charles Seeger theorized that music and language were two distinct forms of discourse and ways of knowing.%
	%
	\autocite{Seeger:Unitary}
	%
Just as one could speak about making music (that is, create verbal discourse that referred to a musical form of discourse), one might also make music about speaking.
More intriguingly, if one could speak about speaking (which would constitute much of academic discourse), would it be possible to \emph{music} about music? 
The villancicos studied in this dissertation do just that. 
They reflect on the nature of music through the medium of music.
Musical experience can be shared across time more readily than religious experience; or to put it another way, it is easier to know through hearing how Spanish Catholics made music (from the testimony of the notated music) than it is to share in their religious experiences.
So this form of religious music offers insights into the world of historical subjects that no other form of historical document or art form can provide.

If inquirers today wish to know what early modern Christians believed, then, we must listen carefully to how they made their faith heard.
The sound of early modern sacred music---the way voices move in relationship to each other, the characteristic stylistic features of common chordal progressions, rhythmic gestures, the dramatic experience of musical forms unfolding in time---all of this provides a window for us to glimpse something of the religious experience of historical believers.
Put the other way, if we wish to understand not only what music meant to early modern people but even the details of how music worked, we must contemplate what the makers and hearers of that music believed about its sacred power. 
%}}}2

%{{{2 neoplatonic theology of music
\section{%
Listening for Unhearable Music:
The Power of Music in the Neoplatonic Tradition
}

The villancicos on sensation and faith elevate hearing above the other senses, and use listening to music as the paradigm of faithful listening.
But these pieces, like Calderón's \wtitle{Nuevo palacio}, also cast doubt on the ability of any sense to perceive spiritual matters unless tempered by faith.
These pieces, then, model a practice of listening to music in which the immediate object of hearing is not the primary goal of perception.
By explicitly drawing attention to the imperfections of music and its listeners, these villancicos challenge hearers to go beyond mere sound and listen for a higher, unhearable music of faith.

This gesture toward a higher, more perfect form of music is rooted in Christian Neoplatonism.
Villancicos on the subject of music, particularly those that will be discussed in part~\ref{part:Singing}, consistently manifest a Neoplatonic theological worldview.
The treatises used to teach musical composition in seventeenth-century Spain, most notably Pedro Cerone's \wtitle{El melopeo y maestro} (1613) and Andrés Lorente's \wtitle{El porqué de la música} (1672), present music within a Boethian cosmology of music, which has its roots in a Neoplatonic-Augustinian tradition.
Augustine was by far the most influential theologian for early modern Catholics, not only in Spain: his works were directly available in printed editions starting early in the sixteenth century and reissued and re-edited many times after, and through many compendia and digests of patristic theology; and his ideas infused every genre of theological writing.%
	%
	\footnote{%
	Chapter~\ref{ch:Padilla-Voces} includes a more detailed discussion of the kinds of theological literature that were most influential for poets and composers of villancicos.
	}
	%

One of the foremost proponents of Christian Neoplatonism in the Augustinian tradition was the Dominican Fray Luis de Granada, especially in his \wtitle{Introducción del Símbolo de la Fe} \gloss{Introduction to the Creed} of 1589.
Fray Luis's writings were widely read across the Hispanic world through the eighteenth century. 
Because his work is a self-acknowledged synthesis of patristic and Classical sources, his writings may be taken as both representative of widely held beliefs of his own time and after, as well as a guide to how earlier sources were read and understood by early modern Catholics.

	%
	%\footnote{%
	%William Christian recounts how a rural family in northern Spain in the 1970s showed him their copy of a book by Fray Luis de Granada which had been handed down for more than two centuries. cite \X
	%}
Fray Luis's introduction to the first article of the Apostle's Creed, \quoted{I believe in God, the Father almighty, Creator of heaven and earth}, is really a fulsome exposition of a theology of the created world.
In the Neoplatonic tradition, Fray Luis teaches that the natural world is a reflection of a higher truth---God's own nature---and that the creation was given so that by reflecting on it people would come to know its Creator.
Fray Luis frequently uses musical metaphors to describe the harmonious workings of the created world, and he includes a discussion of the physiology and theology of the human voice that applies directly to a historical understanding of music.

A second key source for Neoplatonic theology, in this case specific to music, is the encyclopedic \wtitle{Musurgia universalis} by another great synthesist of received wisdom, Athanasius Kircher.
% cite \X
The Jesuit polymath's 1650 work was disseminated through Jesuit networks across the globe: a copy was sent as far as Manila, and two copies are preserved today in Puebla.
Kircher describes in detail the latest scientific knowledge about the anatomy of hearing and vocal production and the physiology of bodily humors and affects; and lays out specific examples of how particular musical structures work through these bodily systems.
Kircher presents a cosmic view of music according to Neoplatonic traditions of theology and music theory, in which the whole universe is encompassed in the \quoted{working of music}---a rough translation of his inventive Greek-and-Latin title.%
%	\X secondary Kircher lit

The writings of Fray Luis de Granada and Athanasius Kircher provide the basis for a provisional historical theology of music within the Neoplatonic tradition. 
The fundamental concepts of this theology of music are the Neoplatonic chain of being and the Boethian three-fold division of music.
In brief, Christian Neoplatonists followed Augustine in viewing the material world as a reflection of a higher spiritual reality which ultimately had its source in the Supreme Good which was the Godhead.%
	%
	\footnote{%
	An important later source for this concept is the \wtitle{Spiritual Hierarchy} attributed to Dionysius the Areopagite. %\X
	}
	%
The material world reflected higher truths only imperfectly, but nevertheless this world was also the only means through which those truths could be reached.
In connection with Catholic sacramental theology, material objects and physical actions became means through which humans could encounter divine grace.
Neoplatonic contemplation could be understood as a dialectical process of discerning the degree both of similarity and of dissimilarity between earthly objects and heavenly truth.

Augustine's writings on music in the \emph{Confessions} (10:23) are sometimes interpreted today in an excessively negative way, as though Augustine only accepted music's value when it could serve as a neutral medium for sacred words.
When Augustine chastizes himself for being seduced by the song, rather than \quoted{that which is sung}, he is not condemning music or the voice as un-sacred.
Instead it is probably more accurate to interpret this statement as a confession---like those throughout the book---that he failed at the task of distinguished the created thing from its Creator.
He was captivated by the song as an object, that is, as an idol, rather than by the \quoted{holy thoughts} or \quoted{sentiments} that the song was meant to communicate to him and move him towards. 
He failed at the task of Neoplatonic-Christian listening by getting stuck at the lowest level of sensory experience, by letting carnal pleasure lead his mind rather than the reverse.

In short, he failed to rise from what he was hearing to contemplate the higher truths to which the singing was meant to move him.
He is not questioning music's Platonic value as a reflection of the divine.
Augustine values music's power over the affections, and acknowledges that it was his own weakness that made this a problem.  

While Augustine as a Neoplatonist does emphasize ideas over material forms, he certainly does not negate or reject the material world.
This would have been Manichaeism, which Augustine left behind for Christianity and spent much of his career refuting.
Christian Neoplatonists after Augustine, then, would see material forms as imperfect reflections of higher realities, but as necessary ones, for a person could only rise to contemplate higher things through the lower things.
Music reflects the structure of the universe and thereby points to God; this is why it is one of the higher liberal arts, leading on to philosophy and theology.
Music for Neoplatonists was not a neutral medium, but was sacred in and of itself because it embodied number and therefore truth.

The definition of music in Boethius's \wtitle{De musica} provided the classic formulation of how music fit into the Neoplatonic chain of being.%
The three Boethian types of music are arranged hierarchically and each one points beyond itself to a higher level.
At the lowest level is \term{musica instrumentalis}---music played and sounded, music that humans can hear.
Higher up is \term{musica humana}---the harmony of body and soul, and of one human being with another in society.
Still higher is \term{musica mundana}---the harmonies created by the perpetual movement of the planetary spheres.

Villancicos on the subject of music embody the notion that even these three levels of music are subordinate to the supernatural forms of music in Heaven---the chorus of angels and saints, and above them, the mysterious harmonies of three in one in the Trinity and two in one in the divine-and-human nature of Christ.
The three Boethian musics in this system would all be \soCalled{worldly} music, and it is important to clarify the distinction between the music of the \foreign{cielos} or heavens---that is, the planetary spheres---and the \soCalled{heavenly} music of the \foreign{cielo Empyreo} or Heaven, the supernatural realm beyond the material world.
Table~\ref{table:Neoplatonic-hierarchy-music} presents a synthesis of the Neoplatonic hierarchy of  music.

%%*******************
%\begin{table}
%	\caption{Neoplatonic hierarchy of music}
%	\label{table:Neoplatonic-hierarchy-music}
%	% Neoplatonic hierarchy of music
% table:neoplatonic-hierarchy-music

\begin{tabular}{lll}
\toprule
               & Harmony of Trinity           & \\
Heavenly music & Chorus of saints, angels     & \\
\midrule
               & \term{Musica mundana}        & Spheres\\
               & \term{Musica humana}         & Bodies\\
Worldly music  & \term{Musica instrumentalis} & Sounding music \\
\bottomrule
\end{tabular}
%\end{table}
%%*******************


\term{Musica instrumentalis}, then, though the lowest form of music in the chain of being, was the only form of music to which humans had direct access through the sense of hearing.
Metamusical villancicos explicitly emphasize the challenge that was central to all music-making in the Christian Neoplatonic tradition, to use the imperfect medium of sounding music to evoke all the higher forms of music, to lead listeners in contemplation up the chain of being beyond simply what was heard.

Vocal music played a special role in this system because for Neoplatonists, the human body was the microcosm of the whole created world. 
The voice, then, is the physical expression of the microcosm, and vocal music thus doubly reflects the order of nature: in its musical ratios and proportions, which reflect those of the spheres, and as an expression of the human body as the microcosm.
\emph{Musica instrumentalis} is the finite expression of \emph{musica humana} and reflects and leads to the contemplation of the \emph{musica mundana}, and to the higher Music of the Triune God who created all these lower forms of music.

%******************** FRAY LUIS ********************
\subsection{%
Hearing the Book of Nature Read Aloud
}

This Neoplatonic worldview was disseminated through the post-Tridentine Hispanic world through books like Fray Luis de Granada's \wtitle{Introduction to the Creed}.
Fray Luis begins his exposition of the Creed in the traditional manner of catechists (as modeled by the Roman Catechism), by using the first article to teach the theology of creation, through which, according to St.~Paul and Thomas Aquinas people can come to the natural knowledge of God.

\quoted{The ultimate and highest good of man}, Fray Luis states at the outset, \quoted{consists in the exercise and use of the most excellent work of man, which is the knowledge and contemplation of God}.%
	%
	\footnote{%
	\Autocite[182]{LuisdeGranada:Simbolo}: \quoted{El último y summo bien del hombre consistia en el ejercicio y uso de la mas excelent obra del hombre, qu es el conoscimiento y contemplación de Dios}.
	}
Fray Luis teaches that the created world is a \quoted{book of nature} in which is written the grandeur, love, wisdom, and faithfulness of its Creator.
The first goal of humankind, then, is to learn to read this \quoted{book of nature} in order to come through it to the knowledge of God. 
The goal of contemplating creation is \quoted{ascending by the staircase of the creatures to the contemplation of the wisdom and beauty of the Maker}.%
	%
	\footnote{%
	\Autocite[184]{LuisdeGranada:Simbolo}: \quoted{subiendo por la escalera de las criaturas á la contemplación de la sabiduría y hermosura del Hacedor}.
	}
	%

The reason one can \quoted{read} God through nature, Fray Luis teaches, is that the created world is a reflection of God's perfect order---a concept the friar repeatedly expresses using musical metaphors.
Fray Luis compares the perfect order of nature to a harmonious musical composition in which everything fits together \foreign{con sumo concierto} \gloss{with the most perfect concord}.%
	%
	\footnote{%
	Or harmony, agreement; the same word is used for a musical \quoted{concerto} or \quoted{concerted} music.
	This seems to be Fray Luis's Castilian equivalent for the Latin \emph{concordia}, the word most frequently used in this context by Augustine (as in the \emph{City of God}).
	}
	%
All the created things in this world, Fray Luis writes, \quoted{like concerted music for diverse voices, harmonize together [concuerdan] in the service of man, for whom they were created}.%	
	%
	\footnote{%
	\Autocite[191]{LuisdeGranada:Simbolo}: \quoted{Mas entre todas ellas es mucho para considerar, de la manera que todas (como una música concertada de diversas voces) concuerdan en el servicio del hombre, para quien fuéron criadas \Dots}.
	}
	%
The movement of the heavenly spheres, and their effects on the earth, are like a great \quoted{chain, or, it can be said, this dance, so well ordered, of the creatures, and like music for diverse voices \Dots.
Because things so diverse could not be reduced to a single end with a single order, if there were not one who was like a chapelmaster [maestro de capilla], who reduces them to this unity and consonance}.%
	%
	\footnote{%
	\Autocite[191]{LuisdeGranada:Simbolo}: \quoted{Asimismo los otros planetas y estrellas, segun los diversos aspectos que tienen entre sí y con el sol, son causa de diversos efectos acá en la tierra, como son lluvias, serenidad, vientos, frio, y calor y cosas semejantes. Esta cadena, ó, si se puede decir, esta danza tan ordenada de las criaturas, y como música de diversas voces, convenció á Averrois para creer que no habia mas que un solo Dios.
	Porque no se pueden reducir á un fin con una órden cosas tan diversas, si no hubiere uno que sea como maestro de capilla, que las reduzga á esta unidad y consonancia}.
	}
	%

In the Neoplatonic tradition, these references to music are more than just metaphors. 
The universe is not only like music, it acutally is in some sense musical.%
	%
	\footnote{%
	It should be said, though, that the tradition is not always clear on whether the music of the spheres is actual music that someone could hear or is only \quoted{music} in the sense of movement in perfect proportions.
	}
	%
While some might think of Neoplatonists as ignoring actual sounding music for the sake of abstracted \quoted{higher music}, it is not possible to compare something to music without having some kind of earhly music in mind.
When Fray Luis compares the world to music \quoted{in diverse voices} he obviously has in his \quoted{mind's ear} polyphonic music of his own time, such as he would have heard at the Portuguese Royal Chapel as confessor to the queen.
Likewise, when he compares God to a \foreign{maestro de capilla}, that has all the implications of that office in the Hispanic context, which included composition, teaching, and leading the choir in some form of conducting.%
	%
	\footnote{%
	The trope of Christ as a chapelmaster is discussed in chapter~\ref{ch:Padilla-Voces}.
	}
	%
Thus God for Fray Luis is creator, prime mover, and sovereign ruler over creation, actively and intimately involved in its ongoing progress.

For Fray Luis, not only does creation reflect God's order; it actively proclaims that fact.
It speaks or sings with its own voice to communicate God's glory to the human who knows how to listen.
Fray Luis glosses Augustine's commentary on Psalm 26 to say, \quoted{Look around at all these many things from the heaven to the earth, and you will see that they all sing and preach their Creator; because all types of creatures are voices [or perhaps, utterances] that sing his praises}.%
	%
	\footnote{%
	\Autocite[185]{LuisdeGranada:Simbolo}: \quoted{Rodea cuantas cosas hay dende el cielo hasta la tierra, y verás que todas canta y predican á su criador; proque todas las especies de las criaturas voces son que cantan sus alabanzas}.
	}
	%
While the full knowledge of God can only come with the aid of divine revelation through the Scriptures and the Church, Fray Luis praises God that humans can study his nature in \quoted{the university of created things, which declare to us [literally, \quoted{give us voices}] that you love us, and teach us why we should love you}.%
	%
	\footnote{%
	\Autocite[186]{LuisdeGranada:Simbolo}: \quoted{Ayúdanos tambien la universidad de las criaturas, las cuales nos dan voces que os amemos, y nos enseñan por que os habemos de amar}.
	}
	%

Fray Luis acknowledges, however, that apart from angels and birds, most of creation is mute and does not literally have its own voice with which to communicate its message of divine glory.
This \quoted{message} is not a linguistic one, but rather, their message is simply themselves: in the created world, the medium is the message.
\quoted{Now these admirable works do not speak or testify this with human voices \Dots}, Fray Luis writes, \quoted{rather their speech and testimony is their invariable order and their beauty, and the artifice with which they are so perfectly made, as though they were made with a ruler and plumb line}.%
	%
	\footnote{%
	\Autocite[192]{LuisdeGranada:Simbolo}: \quoted{Mas estas obras admirables no hablan ni testifican esto con voces humanas (las cuales no pudieran llegar al cabo del mundo); mas su habla y testimonio es la órden invariable, y la hermosura dellas, y el artificio con que están hechas tan perfectamente, como si se hicieran con regla y plomada}.
	}
	%

In this theological system, music has unique value because it actually provides a voice through which creation can make audible its message-of-being.
As Margit Frenk has documented, books in this period were not read silently, but required someone to give them voice, and were written with that intention.%
	%
	\footnote{%
	\autocite{Frenk:Voz}.
One edition of Fray Luis's own book \wtitle{Doctrina Cristiana} bore at the beginning a notice from the Archibishop of Toledo granting a certain number of days of indulgence for each paragraph that anyone \quoted{read or heard read} (that is, had read to them).
	}
	%

To read the \quoted{book of nature}, therefore, someone must perform it vocally---and this is what music could do.
In the Christian Neoplatonic tradition, human music unlocks the musical voice contained within the substance of created things.
Through metal pipes, horns, and bells; through wood viol cases, gut strings, and skin drums; even through reverberant stone church walls, the very matter of creation is made to resound with the perfectly ordered mathematical-harmonic proportions placed within it by the Creator---proportions which themselves reflect God's own perfect order.%
	%
	\footnote{%
	This idea recalls \bibleverse{Lk}(19:40): \quoted{If these [Christ's disciples] were silent, the stones would shout out} (dico vobis quia si hii tacuerint lapides clamabunt).
	}
	%

If pipes and strings testify to the order of creation, then the human body as the microcosm of creation is the ultimate instrument through which nature is given voice.
Fray Luis concludes his exposition of the six days of creation (based largely on the \emph{Hexameron} of St.~Basil) by saying that God's creation of man on the sixth day was like the conclusion of an oration, when the speaker draws together all his themes into a final epitome.
Thus man is the summation of all that God had created in the previous five days and encompasses them all within himself.%
	%
	\footnote{%
	\Autocite[243]{LuisdeGranada:Simbolo}
	}
	%

%*******************
\subsection{%
Voice as Expression of Man, the Microcosm
}

When Athanasius Kircher (in the tenth book of the \emph{Musurgia}) continues this hexameral tradition with his own treatment of the six days of creation, he replaces the rhetorical metaphor with a musical one.
Instead of creation being God's oration, Kircher presents it as a musical improvisation (a \quoted{Praeludium}) on God's cosmic organ.%
	%
	\footnote{%
	\Autocite[Vol. 2, 366--367]{Kircher:Musurgia}.
	Kircher's illustration of this is used as an interpretive key in chapter~\ref{ch:Cererols}.
	}
	%
On the sixth day, Kircher says, God recapitulates all his themes and pulls out all the stops by creating man.
Here again, the comparison to music must be based on some actual music known to Kircher; his description closely resembles the structure of a Praeludium by the likes of Dieterich Buxtehude, which develops a motivic kernel through various sections and culminates in a fugue for the full organ.
In Kircher's worldview, all the systems and elements of creation (stars, planets, humors, rocks, animals, and so on) intersect in the individual human body.%
	%
	\footnote{%
	See the illustration in \autocite[vol. 2, 402]{Kircher:Musurgia}.
	}
	%

For Kircher, the human voice is the unique expression of the individual, reflecting each person's unique temperament and blend of the four humors.%
	%
	\footnote{%
	In \headlesscite[vol. 1, 23--24]{Kircher:Musurgia}, among other places, Kircher discusses why different voices have unique qualities.
	}
	%
Kircher defines the voice thus: \quoted{The voice is a living sound [or, sound of the soul], produced by the glottis through the percussion of respired breaths that serve to express the affects of the soul}.%
	%
	\footnote{%
	\Autocite[vol. 1, 20]{Kircher:Musurgia}: \quoted{Vox est sonus animalis à glottide ex percussione respirati aeris ad affectus animi explicandos productus}.
	}
	%
Since each voice is unique, only in concert do voices fully reflect nature and nature's God.
Cantus, Altus, Tenor, and Bass parts provide a place for all types of human voices, Kircher explains, and correspond respectively to fire, air, water, and earth.
Thus they form a choral microcosm both of humanity and of all creation.%
	%
	\Autocite[vol. 1, 217--219]{Kircher:Musurgia}
	%

Fray Luis says that most of creation does not proclaim its message \quoted{with human voices}, but he also presents the human being as the \quoted{mundo menor} or microcosm;%
	%
	\footnote{%
	\Autocite[243]{LuisdeGranada:Simbolo}: El \quoted{mundo menor, que es el hombre}.
	}
	%
  he exalts the voice as the audible expression of the human body and vocal music as the most perfect kind of music. 
If man is the microcosm and human voices in concert are even more so, than polyphonic choral music could actually give voice to the spheres and all below them.
Fray Luis praises the human voice as the highest of all musical instruments (indeed, as the paradigm for them), as a means of forming social relationships between people, and as a form of communication between human and divine:

%********************
\begin{quote}
%
The lungs also serve to create the voice, because, when the air that they exhale leaves them with a great impetus, and touches the voicebox or \quoted{little bell} that we have at the entrance of the lungs, the voice is formed. \Dots\
But here it is to be noted that the mouth of the pipe coming out of the lungs is neither large nor round, but is drawn tight [hendida] just like the slot of an alms box. 
Which opening serves to form the voice; this is why the mouths of flutes and dulcians are constructed in this fashion, because in this manner, the compressed air entering through them, the voice is caused.%
	%
	\footnote{%
	\Autocite[252]{LuisdeGranada:Simbolo}: \quoted{Sirve también el pulmon para la voz, porque saliendo el aire que él despide de sí con algun ímpteu, y tocando en el calillo ó campanilla que tenemos á la entrada dél, se forma la voz. \Dots\
	Mas aquí es de notar que la boca de la caña deste pulmon, ni es grande ni redonda, ántes es hendida, así como la abertura de una alcancía. 
	Lo cual sirve para formar la voz; porque deste modo están fabricadas las bocas de las flautas y dulzainas; porque desta manera entrando por ellas el aire colado se causa la voz.}
	}
	%
%
\end{quote}
%********************

We may note again that the friar's reference to music corresponds to contemporary practice: flutes and dulcians were played in Iberian church music, both as independent instruments and in the form of organ pipes of those names (and we may also observe that as in Kircher, the organ seems to represent the highest of musical instruments and the closest analogue to the voice).
Fray Luis praises the flexibility of the voice, which unlike a wooden flute can take on any shape needed, and which is unique because each person's body is unique.
The voice therefore expresses human individuality, and voices of different types in concert enact harmony between people:

%********************
\begin{quotation}
%
Moreover, here is a thing worthy of much consideration, to see the omnipotence and wisdom of the Creator, who was able to form something like a flute from flesh, which serves for singing.
For to make a flute or trumpet from a solid material such as wood or some metal, is not much; for the hardness of the material serves for the resonance of the voice.
But to  make this out of flesh (such as is the windpipe of the lungs), and such that through it some voices are formed of women and of men, so sweet that they seem more like those of angels than of humans, and these with such variety of notes [punctos], without having the finger holes of flutes that provide this variety, this is something that declares the power and the wisdom of that sovereign artisan, who in such a manner forged the flesh of this windpipe so that in it could be formed a voice sweeter and milder than that of all the flutes and instruments that human industry has invented.

And there is no end of admiration for the variety that there is in this for the service of harmonious music [música acordada]. 
For some throats are narrow, in which are formed the trebles [tiples], and others in which are formed voices so full and resonant that they seem to thunder through an entire church, without which there could not be perfect music.

All of which that divine presider [presidente] traced and ordained, so that with this mildness and melody the divine offices and their praises should be celebrated, with which to awaken the devotion of the faithful.%
	%
	\footnote{%
	\Autocite[252]{LuisdeGranada:Simbolo}: \quoted{Mas aquí es cosa digna de mucha consideración, ver la omnipotencia y sabiduría del Criador, que pudo formar una como flauta de carne, la cual sirve para cantar.
	Porque hacer una flauta ó trompeta de materia sólida, como es de madera ó de algun metal, no es mucho; porque la dureza de la materia sirve para la resonancia de la voz.
	Mas hacer esto de carne (cual es la caña del pulmon), y que en ella se formen algunas voces de mujeres y de hombres, tan suaves, que mas parecen de ángeles que de hombres, y estas con tanta variedad de punctos, sin tener los agujeros de las flautas que sirven para esta variedad, esto es cosa que declara el poder y la sabiduría de aquel artifice soberano, que de tal manera fraguó la carne desta caña que se pudiese en ella formar una voz mas dulce y mas suave que la de todas las flautas y instrumentos que la industria humana ha inventado.
	Y aun no carece de admiracion la variedad que en esto hay para servicio de la música acordada. 
	Porque unas cañales hay delgadas, en las cuales se forman los tiples, y otras en que se forman voces tan llenas y tan resonantes, que parecen atronar toda una iglesia, sin las cuales no podia haber música perfecta.
	Lo cual todo trazó y ordenó así aquel divino presidente, para que con esta suavidad y melodía se celebrasen los divinos oficios y sus alabanzas, con que se despertare la devoción de los fieles}.
	}
	%
%
\end{quotation}
%********************

Fray Luis wants his readers to hear God's glory reflected most fully in the concerted harmony of diverse human voices, which he says were created for the purpose of singing of singing in divine worship.
The voice in church is the definitive example of vocal music for Fray Luis.
Sacred polyphony glorifies God, then, simply by realizing the potential for which the voice (and the body) was made.
In the above passage, the sound of the voice alone proclaims God's power and wisdom just in itself, apart from whatever words or musical figures it might articulate.

Fray Luis sees speech as something \quoted{added} to the voice, which makes it possible for the voice to communicate and form social relationships:

%********************
\begin{quote}
%
Now here it is to be noted that when to the voice which proceeds from this place is added the instrument of the tongue, we come to articulate and make distinctions with this voice, and thus is formed speech, serving us by this instrument and punctuating [hiriendo] with it sometimes in the teeth and other times in the interior of the mouth.
And just as the flute produces different sounds by touching on different holes, likewise the tongue, touching in different parts of the mouth, forms different words.
By this manner the Creator gave us the faculty to speak and communicate our thoughts and concepts to other men.
	%
	\footnote{%
	\Autocite[252]{LuisdeGranada:Simbolo}: \quoted{Mas aquí es de notar que cuando á la voz, que por aquí sale, se añade el instrumento de la lengua, venimas á articular y distinguir esa voz, y así se forma la habla, sirviéndonos deste instrumento, y hiriendo con él unas veces en los dientes y otras en lo interior de nuestra boca. \Dots\
	Y así como la flauta hace diversos sonidos tocando en diversos agujeros, así la lengua, tocando en diversas partes de nuestra boca, forma diversas palabras.
	Desta manera nos dió el Criador facultad para hablar y comunicar nuestros pensamientos y conceptos á otros hombres.}
	}
	%
%
\end{quote}
%********************

Fray Luis might see music---with its own system of articulations and distinctions---as another way to \quoted{communicate our thoughts and concepts} just as well as spoken language, but he also presents music as a product of the voice before any articulation is added.
This definition of voice would mean that in vocal music there are always two layers---the articulated \quoted{speech} aspects, and behind these the wordless sustained voice.
In a polyphonic vocal piece like a villancico, the bulk of musical structure is borne by the sustained tones of the voice, singing vowels.
Apart from the words being sung, musical elements like mode, meter, motivic development, and stylistic or topical allusions are all communicated by these musical voices, and not simply by the voice as the bearer of words.
Music could thus reflect the divine through its sonic structure, apart from any sacred linguistic meaning that may be attached as well.

This would mean that the hearer of music could and should seek out this level of musical structure while listening.
If music's value and sacredness are not comprised solely in the words being sung, then one must know how to hear the musical structure in order to receive the full benefit.
Citing Augustine's \wtitle{De doctrina christiana} (the classic exposition of Christian preaching and teaching), Fray Luis---who was himself the author of six volumes about \wtitle{Rhetorica ecclesiastica}---says that the main task of the student of rhetoric is to hear and identify the rhetorical tropes and techniques used by another orator.
In the same way, he says, the first task of humankind is to be a student of the natural world, and to learn to recognize in creation the signs of God's artifice as the Creator, which manifest his glory.
%}}}2

%{{{2 theory of three functions of VCs
\section{%
Three Theological Functions of Villancicos
}

The basic Neoplatonic logic traced above may be summarized as follows: 

\begin{enumerate}
\item Music is a reflection of the natural order.
\item The natural order is itself a reflection of God.
\item By paying attention to nature one can come to know and believe in its Maker.
\item Therefore listening to music may be a primary way of \quoted{reading the book of nature} and coming to faith in nature's Creator.
\end{enumerate}

In light of these theological sources we may point towards a tentive answer for the central question, How did Catholics understand music's role in the relationship between faith and hearing?
First, the Faith of the Church, according to the Roman Catechism and traditional teachings, was not only a collection of precepts but a dynamic encounter with the Word of God, which was the Incarnate Christ himself revealing himself through his body, the Church.
So if faith came through hearing, and hearing, by the Word of Christ, then it was the church that made Christ the Word audible through preaching, liturgy, community life---and music.
And though the Church by necessity attempted to accommodate the Faith to peoples' sensory and intellectual capacities, the Church also worked to train those capacities for faithful listening, which included understanding, belief, and obedience.
To put it simply if obscurely, one needed faith to hear the Faith with faith, and thereby to grow in faithfulness.
The ultimate origin of faith in Catholic theology was the grace of God, bestowed in the sacrament of Baptism and renewed in the other sacraments of the Church---and therefore denied to those like Calderón's \soCalled{Judaísmo} who rejected the Church and were in turn rejected.
For those given the grace of faith and welcomed into the Church, however, music contributed to the ongoing proclamation and celebration of the Faith, and learning to participate in music through listening or performance (which should include listening) could be a way to grow in faith and come closer to the object of faith, the Word that was Christ.

We may distinguish at least three theological functions of villancicos as part of this theological agenda of linking faith and hearing: mnemonic, contemplative, and affective. 
The three functions in this model are outlined in \cref{tab:three-functions}.
Each of the three categories overlaps with and includes the others; all are present to some degree in any villancico, though we may distinguish certain pieces and subgenres in which one function predominates, and we may observe shifts in the emphasis on the different functions across time. 
These functions include both a definition of the role of music and a task for listeners (the latter shown in italics in the table).

%%*******************
%\begin{table}
%\caption{Three theological funtions of villancicos: The role of music and the task of listeners}
%\label{table:three-functions}
%	\input{tables/three-functions}
%\end{table}
%%*******************
%
%*******************
\subsection{The Mnemonic Function}

At a simple level, vocal music could make faith appeal to hearing simply by pairing words about Christ with music that pleased the ears of hearers.
Whether the music pleased, however, would depend on the individual temperaments and cultural conditioning of the listeners.
This may be considered a \term{mnemonic function} of music, in which the music serves primarily as an aid (in Spanish, \term{mnemotecnia}, a tool for memory) for remembering a verbal text.
Villancicos could serve a mnemonic function when they were based on set memory texts, such as \wtitle{Señor mío Jesucristo}, an official Act of Contrition text set by Joan Cererols and likely used to teach the choirboys at the Abbey of Montserrat this important prayer.%
	%
	\footnote{%
	Edition in \autocite{Cererols:MEM-VC}. % \X
	}
	%
The strophic coplas of villancicos, which may be related to oral traditions of reciting \term{romances}, also contribute to the genre's mnemonic power by associating the words with a simple, memorable, and frequently repeated melody.

%*******************
\subsection{The Contemplative Function}

In the contemplative function, the structure of music itself worked within a Neoplatonic framework to reflect and proclaim the nature of God, as discussed in the previous section.
Since music was conceived of in rhetorical terms in this period (one classic source for that being Kircher), the listener was challenged to learn to recognize the rhetorical structure of the music.
The properly disposed listener could learn to perceive the artifice of the composer within a piece of music, and thereby hear how the piece reflects the artifice of God who created the numerical ratios and physical objects through which the music can be produced.
In this function the music works somewhat independently of accompanying verbal texts as an object for Neoplatonic contemplation, to point beyond itself to higher truths.
It is contingent on the listeners' intellectual capacity to perceive musical artifice, as well as their disposition toward \quoted{spiritual listening}.

The contemplative function is primary in most of the metamusical villancicos in chapter~\ref{ch:intro} and the pieces about heavenly music studied in part~\ref{part:Singing}.
Enigma pieces and game pieces could also function as objects of intellectual contemplation.
For instance, in Mateo de Villavieja's \wtitle{Jácara en anagramas}, not only are the poetic lines combinatorial, but so are the musical phrases and even the individual voice parts, in an example of algorithmic composition.%
	%
	\footnote{%
	E-MO: AMM.4261, from the Convento de la Encarnación in Madrid, preserved at the Abbey of Montserrat.
	}
	%

If the mnemonic function was emphasized in educating choirboys, then the contemplative function was favored among the lettered elite.
One venue for this would be in meetings or sponsored services of religious confraternities.
Don Antonio de Salazar (late-seventeenth-century chapelmaster of Mexico City Cathedral, composer of \wtitle{Angélicos coros}, discussed in chapter~\ref{ch:intro}) belonged to the Confraternity of St.~Michael the Archangel.
A collection of printed sermons by the Carmelite preacher Fray Andrés de San Miguel of Puebla includes a sermon Fray Andrés was invited to preach at a gathering of Salazar's confraternity.
The title of the sermon explicitly mentions Salazar as having commissioned the sermon, and this suggests that Salazar may have been head of the confraternity as well. 
In the opening of the sermon, the preacher mentions that they are gathered in the \quoted{church of the Encarnación}.
Both Salazar and Fray Andrés were from Puebla, and the devotion to the angels and St.~Michael particularly were characteristic of that city. 
As such, this church may be that of the Augustinian monastery in Puebla, which had that dedication. 
The monk's self-deprecating introduction makes it clear that he is addressing an elite congregation of highly learned and accomplished men.
The cleric expresses his concern that he does not really know enough music to be addressing such a group, which suggests these were men specifically educated in music.
%  \X cite source
The preacher's praise for Salazar (though expected if Salazar was responsible for the paid commission) indicates a high level of respect and appreciation for this chapelmaster.
He goes on to imply that there will be a musical performance at the same liturgy, in which the audience will be able to hear this for themselves.
In fact, Fray Andrés says that Don Salazar could compose a better sermon in music than he himself could preach in words. 

The sermon that follows is an exposition of the identity and deeds of St.~Michael the Archangel.
The friar structures this discussion not according to the verses or portions of a Scriptural citation (as was more common), but according to each syllable of the Guidonian hexachord.
Since Michael's name in Hebrew means \quoted{he who is like God}, or \quoted{Quis ut Deus} in Latin, the friar uses UT to discuss Michael's name.
Since Michael is the chief of all the angels, he is their \quoted{king} or RE; and so on.
As the preacher had warned, his musical knowledge does seem to have been rather thin, since he does not use any other musical terminology or musical metaphors in the sermon. 
The sermon is about angels, not about music; but it uses the terminology of music as a framework to discuss the angels.

We do not know what music may have been performed at this service (or even what kind of liturgy it was), but Salazar's \wtitle{Angélicos coros} would seem to be the type of music that might have been chosen.
The version surviving in the Colección Jesús Sánchez Garza is arranged for the sisters of the Convento de la Santísima Trinidad in Puebla---another semi-private music venue with elite, high-level musical performance.
This is not Salazar's most learned composition, or even his most metamusical, but it does allow us to imagine how themes of music were treated in villancicos performed in closed and private spaces for circles of musical and theological connoisseurs.
Villancicos about angels have a strong Neoplatonic charge to them: the singers turn their attention heavenwards to address the angels directly (since angels were believed to be present at every liturgy, \bibleverse{ICor}(11:10)), while they also stand in for, or sing along with, their heavenly counterparts.
Angel pieces exhort the audience to lift their ears upwards as well, and ascend in the chain of contemplation beyond even \emph{musica mundana} to the music of the \emph{cielo Empyreo} in the highest Heaven.

Salazar's confraternity is similar to the private societies that sponsored the culturally important poetry competitions of imperial Spain.
Martha Tenorio sees these competitions as the defining venue for poetry in New Spain, and demonstrates that for most of the seventeenth century, these events (memorialized and disseminated in special imprints) were centers for highly cultivated poetic virtuosity after the model of Góngora.% 
	%
	\autocite{Tenorio:PoesiaNovohispana}
	%
A musical counterpart to these competitions could be the \emph{oposiciones} that religious institutions held to select new chapelmasters and organists.
Composers had every incentive to use these pieces to demonstrate their virtuosity.
Though the metamusical pieces in part~\ref{part:Singing} were probably not used as official \emph{oposición} pieces, they serve a similar function, allowing a composer to demonstrate a particular kind of skill in music and theology, and to establish a place in a tradition of such pieces.

%*******************
\subsection{The Affective Function}

A third way for music to work in the relationship between hearing and faith would be through the affects.
Music with a primarily affective function would go beyond simply projecting the verbal text (the mnemonic function), and do more than serve as a passive object of contemplation; music in its affective function would exercise direct physical power over hearers by means of sympathetic vibration and the humoral system.
The affective function may have been the closest to the body (and therefore in a sense universal), but it was also the most conditioned by culture and even by individual personality.
Music moved the affects through a developing set of associative, inter-musical relationships---that is, through musical topics and tropes.

The affect of joy is predominant in Christmas villancicos, affects of wonder and awe are emphasized in Corpus Christi villancicos, and affects of love are cultivated in more intimate villancicos for Eucharistic devotion. 
Less common are the affects of grief and pain in villancicos intended for Passion contemplation (such as \wtitle{Ay, que dolor} by Joan Cererols), or pieces dedicated to the Virgin of Solitude.%
	%
	\autocite{Cererols:MEM-VC}
	%

The affective function of villancicos had a pedagogical and formative component.
The setting of the Act of Contrition \wtitle{Señor mío Jesucristo} by Cererols mentioned above under the mnemonic function would serve not only to teach the boys the words of the prayer, but also to model for the boys how to feel and express contrition, which was necessary according to canon law for a valid confession in the sacrament of Penance. 
Affective pedagogy could also mean training the body internally and externally to move in certain ways: this could include the movements of weeping lament, quiet reverence, or vigorous dancing.
In each case, these are actions that the whole assembly must do together, in harmony, with dance as the best example (even if the \quoted{dancing} happened mostly in the interplay between musicians or was only spiritual or conceptual).

The task for the listener in the affective function of villancicos would be to let oneself be moved in sympathy with the performers and with the subject of the song to holy affections--contrition, sorrow for Christ in his passion, joy for Christ in his grace and glory, and above all, love for God and neighbor.
The goal was to be changed by this experience and formed into a community with other listeners, all attuned in sympathetic vibration (which was understood as a real physical action uniting people bodily in their affections).

Affective villancicos suggested a way that a person might be moved through the sense of hearing to believe, not just through rational proofs, but through the affective faculty, by principles of sympathetic vibration.
The affective function united \quoted{sensation} (the external sense of hearing) and \quoted{feeling} (the internal affects in response to what was heard), and could produce a physical response---a bodily reaction to the feeling, such as provoked weeping, or more commonly for villancicos, provoked laughter, gladness, and even dancing---and not alone, but together with the whole listening church.
%}}}2
%}}}1

%{{{1 conclusions
\section{%
Conclusions
}

Vernacular villancicos could evoke a response in their hearers in a way that the rest of the liturgy did not.
Villancicos offered listeners an opportunity to think about (and \quoted{feel about}) the content of the faith in their own language (or at least in one they understood better than Latin), an opening otherwise only offered in the sermon, when one was preached.
As Azevedo, following the Roman Catechism, taught, the faith had to be made \quoted{pleasing to the ear}, and villancicos did just that.
Further, Azevedo taught that the \quoted{hearer of the faith} must remain completely silent and be fully attentive to the master's teaching; so likewise the many villancicos beginning \quoted{Listen!} demand just that.
And being quiet was not only a prerequisite for receiving religious teaching; it was also the expected response to the miracles and mysteries of the faith. 
Thus a villancico by Joan Cererols celebrates the logic-defying doctrine of the Assumption of the Virgin Mary and concludes with the reiterated phrase, \quoted{Callar y creer}---hush, and believe.%
	%
	\footnote{%
	Cererols, \wtitle{Serrana, tú que en los valles}, edition in \autocite{Cererols:MEM-VC}. % \X check dedication 
	}
	%

Villancicos invite listeners not simply to hear, but to \quoted{take heed}, to both discern deeper meanings in what they hear and to put what they hear into practice.
Moreover, while many villancicos do begin with an exhortation to listen, most pieces also include imperatives to actively respond to what is heard.
The command is often affective and devotional---\quoted{llorad}, \quoted{sentid}, \quoted{arde} \gloss{weep, feel, burn}, or (more commonly with Christmas pieces) \quoted{Cantad}, \quoted{Alegren}, \quoted{Repican} \gloss{sing, be joyful, repeat the angels' song}.
Other pieces call on listeners to dance and play instruments.
In other words, villancicos ask listeners to both contemplate and obey---in short, to hear the Faith with faith and to respond in faithfulness. 

From this perspective we may cautiously accept some aspects of the conventional wisdom about the function of villancicos.
The current exotic stereotype of villancicos is as a popular form of devotion that was \quoted{allowed} in the liturgy by the church authorities because they hoped it would attract the common people to Church and to faith---and that it retained some element of impious subversion from its popular roots.
But villancicos were not just passively \quoted{permitted}; they were actively cultivated by cathedral chapters and paid for with generous sums.
The poetry imprints celebrating the performances at a particular church must have been a point of pride for those who endowed the festival, and for this reason were eagerly disseminated far and wide (such as the many Seville imprints in Puebla).
And villancicos may have had a lower social register, and may have been influenced by oral traditions now lost, but the surviving written repertoire was largely composed and performed by highly educated professionals in elite settings.

Nevertheless---we can affirm that villancicos did \quoted{appeal to the people} in an active sense, imploring them to be quiet, to listen, to take heed; even as they appealed in the \quoted{entertainment} sense as well, allowing people to find humor, delight, and wonder in religious mysteries that might otherwise have remained inaccessible and uninteresting to them.
Not only because of the words, but because of the lively, diverse musical styles used, which probably were associated with a lower social register, villancicos \quoted{appealed to the ear} both in the sense of being \quoted{pleasing} and in the sense of \quoted{reaching out to} or \quoted{making a claim upon}.

There are so many \quoted{Listen!} openings that this gesture deserves deeper reflection than dismissing it as simply a practical way of getting attention, or as a generic convention. 
The recurrence of this kind of exordium in villancicos may indicate that the genre itself was fundamentally about getting people to listen.
The rest of the liturgy may have passed through the lay people's ears like the incense wafted by their noses, creating a general atmosphere of devotion but not evoking any specific sentiments (thoughts, ideas, images) in the mind and not provoking any direct response.
But the vernacular villancicos demanded attention.
Many of them presented hearers with bold, striking images at their openings, projecting an intriguing poetic conceit or scenario through musical structures and styles that made people take notice. 
In other words, villancicos made faith appeal to hearing.

But did this effort really work?
Did people understood the riddles or get the jokes, and if so, which people were they? 
Did the musical rhetoric and symbolism that is unpacked in such detail in this study really serve as objects of contemplation to anyone at the time, including the performers?

It is not possible to answer these questions fully without more documentation about how Spanish subjects at different levels of society heard and thought about music.
But from the poetic and musical texts themselves, it is possible to affirm that church leaders made a hearty effort to reach out to a wide public, both cultivated and common, and to get them to listen to the faith in a new way.

It is certainly possible that the whole culture of producing and consuming villancicos was the domain of a cultivated elite and did little actually to propagate faith among a broader range of hearers.
In the late sixteenth century Antonio de Azevedo lamented that the church continued celebrating its doctrines through festive ceremonies, even while lay people had hardly any idea what they were about:
\quoted{For this [teaching] is such an important business, especially given that it is a mortal sin that they should not know what day it is, what it means that our Lord was born, what it means that he died and rose to the heavens: and all this which the Church celebrates with such great festivities}.%
	%
	\footnote{%
	\autocite[27]{Azevedo:Catecismo}:
	\foreign{Pues es negocio tan importante, tanto que es pecado mortal, que no sepan el dia de oy, que cosa es nacer nuestro Señor, ni que cosa es morir, y subir a los cielos: y esto que la Iglesia celebra, con tan grandes festiuidades, como lo vemos.}
	}
	%
The added difficulties of teaching in the colonial context and the increasing aesthetic of complexity in learned Spanish poetry and music of the seventeenth century would suggest that many commoners remained unformed in such basic matters of faith.
As Azevedo suggests, the church's ceremonies, including the dozens of poetic verses and musical figures of villancicos in festival Matins, did actively celebrate the Church's faith---but just because cathedrals echoed with these words and music does not mean that everyone understood them on the same level.

And even for literate listeners, the complexity of music across a whole cycle of eight or more villancicos would present a challenge for any listener seeking \quoted{to hear Faith with faith} through music.
This would be especially true in metamusical pieces, which required a sophisticated knowledge of music to even make sense of the text.
As Padre Daniel Codina, monk at the present-day Abbey of Monsterrat, said of a villancico by Montserrat's seventeenth-century chapelmaster Joan Cererols (\wtitle{Suspended, cielos}, the subject of chapter~\ref{ch:Cererols}), such a piece is an explanation which itself needs to be explained.%
	%
	\footnote{%
	Personal conversation, November 2012.
	}
	%

Catholic devotional music provided a practical medium for both appealing to the
ear and training it, though music amplified the challenges of acquiring faith
through hearing.
Catholic listeners were encouraged to doubt their senses as much as to trust
them; and church leaders struggled with the frightening possibility that some
people might simply lack the capacity for hearing with faith.
Religious ear training required individual discipline to avoid the danger of
over-reliance on subjective sensory experience and to learn to discern the
spiritual truth communicated through musical patterns.
This training would also need to discipline the whole community to overcome
misunderstandings based on cultural conditioning.

Propagating faith, then, meant trying to establish not just individual
Christians, but also building a Christian society as the body of Christ.
Faithful Catholics had to learn to submit their sensory experience to the
authority of the Church as the source of certainty, as the living, communal
embodiment of Christ the Word in the world.
For Roman Catholics, the Church \emph{was} the gospel, and the task of building
the Church could not be separated from the work of building an empire.

As Catholics worked to create Christian communities, music was a potent tool for
creating harmony, for instituting social discipline as a reflection of the
heavenly hierarchy.%
    \Autocites{Baker:Harmony}{Irving:Colonial}{Illari:Polychoral}
The virtue of man as Neoplatonic microcosm was reflected in the broader society
and in turn depended on it.
Spanish political thinkers conceived of the colonial project in terms of
establishing harmony in society.%
    \Autocite[22--31]{Baker:Harmony}
Most educated Spaniards were familiar with the medieval philosopher Boethius
(either directly or through expositions of his ideas in contemporary music
treatises like that of Pedro Cerone) and his concept that there were three kinds
of music: \emph{musica instrumentalis}, sounding, playing music; \emph{musica
humana}, the harmony of the individual in body and soul, reason and passion, and
the concord of human society; and \emph{musica mundana}, the music of the
celestial spheres.%
    \Autocites
    [\range{bk}{2}, \pagenums{187--189}]{Cerone:Melopeo}
    [203--208]{Boethius:Musica}
The proper performance of \emph{musica instrumentalis}, they believed, could
actually attune the \emph{musica humana} on individual and social levels,
bringing human society in concord with the order of the cosmos, and beyond it,
with the mysterious harmonious of the triune God.

Catholic music, then, was not \emph{about} society; it \emph{was} society.
This is why the Franciscan friars in New Spain and the Jesuit priests in
California not only started parishes, but also trained choirs.
Forming choirs of boys and training ensembles of village musicians in colonial
cities were practical means of establishing the Church and propagating faith on
individual and communal levels.
The musical ritual of the seventeenth-century Church involved a large number of
community participants, for whom performing music with the body and hearing it
were inextricably linked.
The musical efforts of the colonizing church concretely built social
relationships through musical training.%
    \Autocite{RamosKittrell:PlayingCathedral}
For this reason, we cannot fully understand the faith of early modern Catholics
on the basis of verbal formulations alone; we need to see and hear how
communities practiced their faith through coordinated action---such as in
devotional music.%
\begin{Footnote}
    The Lutheran hymn composer Johann Crüger advocated a similar concept of
    \quoted{the musical practice of piety} (\wtitle{Praxis pietatis melica},
    1647 and many later editions), coming out of the Lutheran \quoted{new piety}
    movement of the seventeenth century, whose proponents (Martin Moller, Johann
    Arndt) were inspired by much of the same medieval devotional literature as
    their Catholic counterparts.
\end{Footnote}

%}}}1

\endinput
