% 2-scratch.tex

The catechism teaches that God communicated his own nature to humanity by taking on human flesh in Christ.
Therefore the true Word of God was not confined to Scripture or doctrines alone---the Word was a person, Jesus Christ, to whom the prophetic scriptures testified and from whom the Church's doctrine flowed.
In the words of John's gospel, Jesus Christ himself was the \term{logos} or \term{verbum} (\scripture{Jn 1:1}), the \quoted{Word made flesh}.
Where God had spoken through the law and the prophets, now he had spoken through his Son (\scripture{Heb 1:1}).
Christ called apostles and formed the church, and after his death, resurrection, and ascension this church was anointed by the Holy Spirit to continue communicating the divine Word to the world across the ages, through the legitimate succession of the church's ordained leaders.
The Word, then, was not a verbal formula to memorize or a concept to understand, but rather a person, made known through a community.
The church existed not only to proclaim the Word but to embody it, and new disciples of Christ, or catechumens, must seek to encounter Christ the Word through his embodied, sacramental presence in the community of the church.
For Roman Catholics, the church \emph{was} the gospel.

%***
The Church's task in promulgating faith, then, was not merely to teach rote verbal formulas to its catechumens or to persuade them to agree with certain concepts.
Rather, the church represented Christ to the world, and the catechumen's role was ultimately to enter into communion with the triune God through participation in the church.
How, then, could the church communicate Christ the Word through the sense of hearing?

% word, text, person, community, performance
% challenge of faithful hearing, connection of faith and ethics and community; types of faith (informe, formata); music as both way to increase appeal to hearing as well as form of building social relationships; but cultural conditioning and subjective experience are problems for both endeavors.
% XXX

Villancicos seem to have been created and heard primarily by those already within the fold of the church.
Whatever spiritual work this genre of poetry and music was meant to do---whether teaching doctrine, instilling awe and wonder, or simply delighting the senses---it reinforced beliefs and attitudes rather than presenting them for the first time.
We will want to know whether villancicos that treat subjects of hearing and faith reflect any concern for individual subjectivity.

%**********************************************************
% new intro to ``The Challenge of Making Faith Appeal to Hearing''
To say that faith came through hearing was to acknowledge that individual sensory experience was needed for faith, as well as to emphasize the need for both teachers to make faith heard and for disciples to listen.
In other words, hearing linked individual subjects, in community, with the transcendent object of faith---the triune God who was made known through Christ, \quoted{the Word made flesh} (\scripture{Jn 1}).
Music would seem an obvious choice of medium for making faith appeal to hearing in this way.
But the central problem for using music to make faith appeal to hearing was that early modern Catholics did not trust the sense of hearing.
Most Christians of any confession in this period agreed that music had tremendous power to shape individual experience and to join people together in community; but Catholics after the Reformation grew increasingly concerned that this power was dangerous.
Villancicos, we will see, provided a way to make faith appeal to hearing while also shaping thoughts and attitudes about the nature of hearing and fatih.
They embody a central tension of early modern Catholicism between accommodating the ear and training it.
This tension is manifested clearly in doctrinal literature of the post-Tridentine period, and exploring Latin and Spanish catechisms provides a good initial foundation for understanding what Catholics in this period believed about the relationships between faith, hearing, and the power of music.

Let us begin with faith.
Luther's reform, from a Protestant perspective, restored a true biblical theology of faith; from a Roman Catholic perspective, Luther redefined the traditional, orthodox, doctrine of faith completely.
Post-Reformation Catholics insisted on continuity with medieval theology of faith as formulated by Thomas Aquinas and others, in which faith was one of three virtues or capacities, along with hope and charity.
Simple belief in intellectual propositions was \quoted{unformed faith} (\term{fides informe}); fully \quoted{formed faith} (\term{fides formata}) \quoted{worked through} the two higher virtues to result in what might best be translated as \quoted{faithfulness}.
A man with this kind of faith so completely believed in God as made known through Christ that he committed himself to God, to live in conformity to God's will together with the rest of the Christian community.
A man of faith was a man of virtue (the root of \term{virtus} being \term{vir}), and a virtuous society was built from virtuous men.%
\Autocite[8]{Catholic:Catechismus1614}

For Luther, faith meant trusting in Christ alone for salvation, something that could only happen through God's grace and did not depend on a person's actions.
Catholics who polemicized against Luther argued that Luther was turning his followers away from the trustworthy, institutional church, and leaving them with only a subjective experience as proof of salvation.
Moreover, they saw Luther's theology as separating faith from ethics.
The new, subjectively defined \quoted{faith without works} undermined the Catholic Humanist project of training men of virtue to form a just society.


It will be important to bear in mind throughout this study that Hispanic Catholics held differing views on faith, hearing, and the power of music.
Neither Athanasius Kircher nor Juan de la Cruz are likely to have reached anywhere near the size of audience that villancicos did.
Vernacular devotional poetry and music shaped and expressed the faith of a much broader spectrum of people, and we will see that villancicos do not reflect a single, consistent position---they are not simple projections of the Roman Catechism, for example, through a different medium.
Many villancicos seem designed to create the kind of awestruck affective experience that Kircher praises, and that Juan de la Cruz critiques; but most pieces communal and individual perspectives through the interplay of chorus and soloists, and sets of villancicos include pieces meant to please the crowd through broad humor or virtuoso display, as well as pieces that seem intended for more personal contemplation.

% XXX ******************* START ***************************
% Revised section on Kircher 2017/01/07
This tension between accommodating the ear and training it may be seen in the explanations of music's power by Athanasius Kircher.
Kircher's 1650 exhaustive compendium of musical knowledge, \worktitle{Musurgia universalis}, was disseminated throughout the Hispanic world, with copies sent to centers of colonial culture including Puebla and Manila.%
\begin{Footnote}
  On the worldwide distribution of the book as far as Manila, see \autocite[\XXX]{Irving:Colonial}.
  The book may be found in historical collections in Madrid, Barcelona, Mexico City, and Puebla (two copies).
  On Kircher, see \XXX[cites].
\end{Footnote}

% start XXX

Drawing on a wide range of writings about music from classical Greek and Roman texts through Renaissance music theory, and responding to reports sent back to Rome from his fellow Jesuits working on missions across the globe, Kircher 


the Jesuit writer argues

Kircher discusses the power of music several times throughout his ten-volume treatise, including a detailed analysis of \quoted{whether, why, and what kind of power music might have to move people's souls, and whether the stories are true that were written about the miraculous effects of ancient music}.%
  \Autocite[bk.~VII, Erothema VI, 549]{Kircher:Musurgia}
Kircher's contribution to this favorite controversy of the Renaissance is to defend the superiority of modern music on the basis of, among other factors, its increased ability to move listeners through varieties of musical structure and style.
To begin with, Kircher argues with conviction that music is capable of dramatically enhancing the power of spoken words to move listeners.
% quotes from 1st version and discussion

How did music achieve such power to move listeners?


Kircher's argument is a fascinating mixture of a universal conception of music, with a new acknowledgment of a certain degree of relativity in music's power.
On the universal side, Kircher argues that if in fact the ancients like Alexander the Great and King Saul were moved in spirit by music, then the mechanism of that movement must be the same mechanism that made music moving in Kircher's own day.
Modern music was superior to ancient music not because it operated according to different principles, but because modern musicians had found ways to extend and amplify the same principles that made ancient music great.
% quotes

These principles are physical in nature: they are based on an interaction between the way music moves the air---determined by the numerical proportions of the music---and the way this effects the movement of a person's \term{spiritus animales}, an ethereal substance that circulated through the body and connected exterior sensation to interior understanding. % see below

On the relative side, Kircher makes a number of claims that might seem to contradict the universal workings of music.
First of all, Kircher acknowledges that different nations are moved by different types of music, and therefore have cultivated differing national traditions of music.
People of one nation prefer their music to that of other nations.
Kircher posits that one reason for this difference is the physical environment in which these nations have developed. % examples
% details

Second, Kircher recognizes that the same kind of music does not move every person the same way, because of variations in individual temperaments.
% details

Third, Kircher says that even within a national tradition, there are many different styles that move people in different ways.
% details

Music in Kircher's theory moves people through sympathetic vibration.
The structure of the music's movements must correspond to the movements of the body's humors.
Kircher theorizes four conditions that are necessary for music to achieve an effect; without any one, music will fail to move the listener in the intended way:
\begin{quote}
  The first is harmony itself.
  Second, number, and proportion.
  Third, the power and efficacy of the words to be pronounced in music itself; or, the oration.
  Fourth is the disposition of the hearers, or the subject's capacity to remember things. 
  % original p. 550
\end{quote}
Music, Kircher concludes, does not move just anyone, but \quoted{only the one whose natural humor is congruent with the music}; \quoted{unless the spirits of the subject respond exactly} to the \quoted{number and proportion of motion} in the music, \quoted{it will effect nothing}. % original p. 550
\quoted{Therefore, if our musicians wish to renew the miracles of ancient music, they should devote their attention, first to explring the inclination and natural way of being [habit] of a particular subject, so that from this they may adapt a congruent theme of harmonic numbers and words, and they should not doubt that through these, like the ancients, the effect is caused}. % p 551

In some ways Kircher is constructing a musical corollary to the catechism's instruction of accommodating teaching to the sense and intelligence.
Music in his view is capable of physically transforming listeners through their senses, of creating non-metaphorical harmony between the subject of the music (that is, what the music is about), the performer and the listeners.
In other words, in this high concept of music, sympathy creates community; harmony physically binds together those who participate in music, including listeners; music unites the listening subject with the subject of the music; and in the case of sacred music, where the subject is Christ the Word, music then would directly unite people with the person at the heart of the Church's proclamation.
% wow, hold on

But all along the way, seemingly without recognizing it, Kircher is acknowledging problems with music's power.
Despite Kircher's confidence in modern musicians' ability to make music music move people, the conditions he names may not be as easy to fulfill as he suggests.
There must be congruence, first of all, between the structure of the music and subject of that music: the music must move in the same way as the affective movements it seeks to incite.
Harmonic ratios, metrical proportions, verbal rhetoric---all of these must align, but they are still not enough without the fourth condition, the disposition of the hearer.
The listener must have a humoral temperament that is moved in the desired way by the music. 
This is like a chemical reaction: without the proper makeup on both sides, no reaction will happen.
If, as Kircher acknowledges, people of different nations are moved by different kinds of music, and if individual people respond differently depending on their temperament as well as their intellectual capacity, how could any musician be sure of the effect of a composition?

In other words, the tension between acccommodating hearing and training it is multiplied vastly by the addition of music.
While it might seem that music would allow for greater accommodation, the number of potential obstacles is increased.
Individual subjectivity and cultural conditioning pose particular challenges for the universal efficacy of music.


Si 
QUIS 
Deo deuotum HOMINUM 
rerumque cœlestium, meditationi deditum in exoticos affectus raptusque mentis 
COMMOUERE VELLET 

IS 
supra insigne aliquod verborum 
THEMA, 
  quod 
  rerum cælestium dulcedinem, \& suauitatem 
  auditori in memoriam 
  reuocaret, 
MODULO DORIO per clausulas interuallaque aptè 
ADAPTET,
      
\& 
EXPERIETUR 
[QUOD dixi verum esse], 
statim extra se 
FACTOS dulcedine harmonica 
EÒ, 
vbi 
 VERA SUNT 
 gaudi rapi.





If someone
were wishing
to move
the kind of man who is devoted to God
and given in meditation of heavenly things
[to move him?]
in otherworldly affects
and rapture of mind

[then] he
should/would aptly adapt 
over/upon some notable theme of words,
which theme would recall to the hearer in the memory
the sweetness and mildness of heavenly things,
in the dorian mode upon/with/by cadences/phrases and intervals

and the truth of what was said would be experienced/
( and [the man] would experience the truth of what was said/
 and he would experience that what was said was actually true )

suddenly beyond/outside himself
[those cadences and intervals] having been made by the harmonic sweetness

to/by/for him/it,
where 
[those things] are true
to be carried away/raptured
of joy

suddenly to be carried away outside himself
[of/]with joy to where those things are true

If there was the kind of man who was devoted to God
and dedicated to meditating on heavenly things,
and if someone should to move this kind of man 

Just as, on the other hand, 
if, 
wishing to move the sort of man 
    who was devoted to God and dedicated to meditation on heavenly things
  in otherworldly affects and rapture of the mind,

he [Timotheus] would take up 
some notable theme expressed in words---
  a theme that would recall the sweetness and mildness of heavenly things
  to the listener's memory---
and [he would] fittingly adapt it 
  in the dorian mode through cadences/phrases and intervals,

[the king] would experience the truth of what was said,
[he would experience]
[experience] those [heavenly] things that were made by harmonic sweetness
and suddenly be carried away beyond himself with joy 
to that place where those things are true. 

If the musician addressed the sort of man who was devoted to God and dedicated to meditation on heavenly things,
and wished to move him in otherworldly affects and rapture of the mind,
he would take up some notable theme expressed in words---a theme that would recall to the listener's memory the sweetness and mildness of heavenly things---
and he would fittingly adapt it in the Dorian mode through cadences and intervals,
then the listener would experience that what was said was actually true,
those heavenly things that were made by harmonic sweetness,
and he would suddenly be carried away beyond himself to that place where those joyful things are true.


%*****************************

Kircher does not seem to resolve the tension between the notion that music must be congruent to the pre-existing temperament of listeners and the notion that music could move these listeners to experience something outside themselves, something alien to their temperament.
For someone to be \quoted{carried away} by music to experience heavenly truth, for music to make it possible for faith to come through hearing, music would need to move listeners to new experiences.



%*********************************




To understand the role of villancicos in the dynamics of hearing and faith,
then, we must consider listeners as active participants in the process, and we
must recognize that the function of music went far beyond dogmatic teaching.
Approaching villancicos as simplistic teaching pieces makes it difficult to
understand much of the repertoire.





 
