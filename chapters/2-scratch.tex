% 2-scratch.tex

The catechism teaches that God communicated his own nature to humanity by taking on human flesh in Christ.
Therefore the true Word of God was not confined to Scripture or doctrines alone---the Word was a person, Jesus Christ, to whom the prophetic scriptures testified and from whom the Church's doctrine flowed.
In the words of John's gospel, Jesus Christ himself was the \term{logos} or \term{verbum} (\scripture{Jn 1:1}), the \quoted{Word made flesh}.
Where God had spoken through the law and the prophets, now he had spoken through his Son (\scripture{Heb 1:1}).
Christ called apostles and formed the church, and after his death, resurrection, and ascension this church was anointed by the Holy Spirit to continue communicating the divine Word to the world across the ages, through the legitimate succession of the church's ordained leaders.
The Word, then, was not a verbal formula to memorize or a concept to understand, but rather a person, made known through a community.
The church existed not only to proclaim the Word but to embody it, and new disciples of Christ, or catechumens, must seek to encounter Christ the Word through his embodied, sacramental presence in the community of the church.
For Roman Catholics, the church \emph{was} the gospel.

%***
The Church's task in promulgating faith, then, was not merely to teach rote verbal formulas to its catechumens or to persuade them to agree with certain concepts.
Rather, the church represented Christ to the world, and the catechumen's role was ultimately to enter into communion with the triune God through participation in the church.
How, then, could the church communicate Christ the Word through the sense of hearing?

% word, text, person, community, performance
% challenge of faithful hearing, connection of faith and ethics and community; types of faith (informe, formata); music as both way to increase appeal to hearing as well as form of building social relationships; but cultural conditioning and subjective experience are problems for both endeavors.
% XXX

Villancicos seem to have been created and heard primarily by those already within the fold of the church.
Whatever spiritual work this genre of poetry and music was meant to do---whether teaching doctrine, instilling awe and wonder, or simply delighting the senses---it reinforced beliefs and attitudes rather than presenting them for the first time.
We will want to know whether villancicos that treat subjects of hearing and faith reflect any concern for individual subjectivity.

