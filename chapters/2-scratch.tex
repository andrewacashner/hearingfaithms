% 2-scratch.tex

The catechism teaches that God communicated his own nature to humanity by taking on human flesh in Christ.
Therefore the true Word of God was not confined to Scripture or doctrines alone---the Word was a person, Jesus Christ, to whom the prophetic scriptures testified and from whom the Church's doctrine flowed.
In the words of John's gospel, Jesus Christ himself was the \term{logos} or \term{verbum} (\scripture{Jn 1:1}), the \quoted{Word made flesh}.
Where God had spoken through the law and the prophets, now he had spoken through his Son (\scripture{Heb 1:1}).
Christ called apostles and formed the church, and after his death, resurrection, and ascension this church was anointed by the Holy Spirit to continue communicating the divine Word to the world across the ages, through the legitimate succession of the church's ordained leaders.
The Word, then, was not a verbal formula to memorize or a concept to understand, but rather a person, made known through a community.
The church existed not only to proclaim the Word but to embody it, and new disciples of Christ, or catechumens, must seek to encounter Christ the Word through his embodied, sacramental presence in the community of the church.
For Roman Catholics, the church \emph{was} the gospel.

%***
The Church's task in promulgating faith, then, was not merely to teach rote verbal formulas to its catechumens or to persuade them to agree with certain concepts.
Rather, the church represented Christ to the world, and the catechumen's role was ultimately to enter into communion with the triune God through participation in the church.
How, then, could the church communicate Christ the Word through the sense of hearing?

% word, text, person, community, performance
% challenge of faithful hearing, connection of faith and ethics and community; types of faith (informe, formata); music as both way to increase appeal to hearing as well as form of building social relationships; but cultural conditioning and subjective experience are problems for both endeavors.
% XXX

Villancicos seem to have been created and heard primarily by those already within the fold of the church.
Whatever spiritual work this genre of poetry and music was meant to do---whether teaching doctrine, instilling awe and wonder, or simply delighting the senses---it reinforced beliefs and attitudes rather than presenting them for the first time.
We will want to know whether villancicos that treat subjects of hearing and faith reflect any concern for individual subjectivity.

%**********************************************************
% new intro to ``The Challenge of Making Faith Appeal to Hearing''
To say that faith came through hearing was to acknowledge that individual sensory experience was needed for faith, as well as to emphasize the need for both teachers to make faith heard and for disciples to listen.
In other words, hearing linked individual subjects, in community, with the transcendent object of faith---the triune God who was made known through Christ, \quoted{the Word made flesh} (\scripture{Jn 1}).
Music would seem an obvious choice of medium for making faith appeal to hearing in this way.
But the central problem for using music to make faith appeal to hearing was that early modern Catholics did not trust the sense of hearing.
Most Christians of any confession in this period agreed that music had tremendous power to shape individual experience and to join people together in community; but Catholics after the Reformation grew increasingly concerned that this power was dangerous.
Villancicos, we will see, provided a way to make faith appeal to hearing while also shaping thoughts and attitudes about the nature of hearing and fatih.
They embody a central tension of early modern Catholicism between accommodating the ear and training it.
This tension is manifested clearly in doctrinal literature of the post-Tridentine period, and exploring Latin and Spanish catechisms provides a good initial foundation for understanding what Catholics in this period believed about the relationships between faith, hearing, and the power of music.

Let us begin with faith.
Luther's reform, from a Protestant perspective, restored a true biblical theology of faith; from a Roman Catholic perspective, Luther redefined the traditional, orthodox, doctrine of faith completely.
Post-Reformation Catholics insisted on continuity with medieval theology of faith as formulated by Thomas Aquinas and others, in which faith was one of three virtues or capacities, along with hope and charity.
Simple belief in intellectual propositions was \quoted{unformed faith} (\term{fides informe}); fully \quoted{formed faith} (\term{fides formata}) \quoted{worked through} the two higher virtues to result in what might best be translated as \quoted{faithfulness}.
A man with this kind of faith so completely believed in God as made known through Christ that he committed himself to God, to live in conformity to God's will together with the rest of the Christian community.
A man of faith was a man of virtue (the root of \term{virtus} being \term{vir}), and a virtuous society was built from virtuous men.%
\Autocite[8]{Catholic:Catechismus1614}

For Luther, faith meant trusting in Christ alone for salvation, something that could only happen through God's grace and did not depend on a person's actions.
Catholics who polemicized against Luther argued that Luther was turning his followers away from the trustworthy, institutional church, and leaving them with only a subjective experience as proof of salvation.
Moreover, they saw Luther's theology as separating faith from ethics.
The new, subjectively defined \quoted{faith without works} undermined the Catholic Humanist project of training men of virtue to form a just society.


It will be important to bear in mind throughout this study that Hispanic Catholics held differing views on faith, hearing, and the power of music.
Neither Athanasius Kircher nor Juan de la Cruz are likely to have reached anywhere near the size of audience that villancicos did.
Vernacular devotional poetry and music shaped and expressed the faith of a much broader spectrum of people, and we will see that villancicos do not reflect a single, consistent position---they are not simple projections of the Roman Catechism, for example, through a different medium.
Many villancicos seem designed to create the kind of awestruck affective experience that Kircher praises, and that Juan de la Cruz critiques; but most pieces communal and individual perspectives through the interplay of chorus and soloists, and sets of villancicos include pieces meant to please the crowd through broad humor or virtuoso display, as well as pieces that seem intended for more personal contemplation.



