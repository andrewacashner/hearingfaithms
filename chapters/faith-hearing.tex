% Cashner, Villancico monograph
% chapter 2: Making Faith Appeal to Hearing

% 2018-05-16      Converted back to LaTeX!
% 2018-02-14      Complete draft used for UR Humanities Center work-in-progress
%                   session
% 2017-11-13      Converted from LaTeX to Markdown, new revision in progress
%
% 2017-01-27      Version for book proposal to Univ. California press
% 2016-12-29      Compromise new book draft based on USCB with some
%                   additions from 9/16 version and new material (50pp)
%
% 2016-10-13      Abridged version for USCB presentation (12pp)
% 2016-09-16      First draft based on diss., expanded (60pp)
% 2016-08-31      Revision for book begun
%
% 2015-03-18      Dissertation ch. 2 defended
% 2014-06-30      Earliest version in computer


\chapter{Making Faith Appeal to Hearing}
\label{ch:faith-hearing}

\epigraphTranslation
{Yo no te alcanzo ni tu enigma sé, \\ 
porque a la Fe he escuchado sin la fe.}
{I cannot comprehend you nor solve your puzzle, \\
because I have listened to Faith without faith.}
{\quoted{Judaism}, in Calderón, \wtitle{El nuevo palacio del Retiro}}

Churchgoers in the Hispanic world of the seventeenth century were familiar with
many kinds of stock characters who appeared alongside the shepherds and kings in
the devotional music performed at Christmas.
One of the most intriguing of these recurring characters in villancicos is the
\emph{sordo}, a deaf or hard-of-hearing man.
Poets placed the \emph{sordo} character in dialogue scenes opposite a friar or
catechist, in which the religious teacher attempted in vain to overcome the
impediment of hearing and communicate his message.
In 1671, one such villancico was performed by King Philip IV's own ensemble, the
Royal Chapel, for the feast of Christmas at Madrid's Convent of the Incarnation.
The music was composed by Matías Ruiz, chapelmaster at the convent, to an
anonymous poem, \wtitle{Pues la fiesta del niño es}.%
    \footnote{\sig{E-E}{Mús. 83-12}.}
The singers and players of the royal ensemble presented a dialogue between
a catechist and \quoted{deaf} man.
This is how the two men meet:
\begin{quotepoem}
    \speaker{Sordo} Éntrome de hoz, y de coz. &
    \speaker{Deaf Man} Here I come, like it or not. \\

    \speaker{Preg.} ¿Quién llama con tanto estruendo? &
    \speaker{Catechist} Who's that making such a ruckus? \\

    \speaker{S} Hablen alto, que no entiendo, &
    \speaker{D} Speak up, for I don't understand \\

    sino levantan la voz. &
    unless you raise your voice. \\

    \speaker{P} Bajad la voz, &
    \speaker{C} Lower your voice, \\

    que a Dios gracias no soy sordo. &
    for I am not deaf, thank God. \\

    \speaker{S} ¿Dice que está el niño gordo? &
    \speaker{D} Are you saying the baby is fat? \\
    
    pues de eso me alegro mucho. &
    well, that sure makes me happy. \\
\end{quotepoem}
The deaf man is not the only object of humor here: the catechist completely
fails to accommodate the other man's impairment, and the garbled echoes of
misheard teaching threaten to make the churchman the more absurd of the two
characters.

In fact, the \emph{sordo} follows this introductory dialogue by declaring his
love for the baby Jesus. 
The full eight-voice chorus joins him in affirming a maxim that could be
interpreted as a critique of hearing and deaf people alike:
\begin{quotepoem}
    \speaker{Sordo} Pues vaya de viestas    
    & \speaker{Deaf Man} So on with the festivities \\
    al niño que adoro       & for the Christ-child I adore, \\
    que está como un oro,   & since he is like a gold coin, \\
    y el coro sonoro        & and let the resounding choir \\
    responda veloz,         & respond quickly, \\
    que sordos son          & for the deaf are those \\
    los que no escuchan     & who do not listen \\
    ni entienden el son.    & nor understand the sound. \\[1ex]

    \speaker{Coro} Que sordos son          
    & \speaker{Chorus} For the deaf are those \\
    los que no escuchan     & who do not listen \\
    ni entienden el son.    & nor understand the sound.
\end{quotepoem}
If, theologically speaking, the truly deaf are those who do not listen or
understand, then both characters in this scene are deaf to each other.
This gives added meaning to the title written in Ruíz's manuscript performing
parts, \wtitle{Villancico de los sordos}---using the plural for \emph{deaf} even
though there is only one character labelled \emph{sordo} in the piece.

This scene of mishearing and misunderstanding is emblematic of the central
problem of the Roman Catholic Church in the new religious landscape after
Columbus and Luther.
The Church faced the challenges on every front: in Europe, to defend its
teaching and regulate its practice in the face of Protestantism; and overseas,
to convert natives to Christianity and build a Catholic civilization in the
colonized lands.
    %  \citXXX[basic bibliography on Catholic Reform]
Whether the church's teachers were trying to speak to \quoted{heretics} or
\quoted{pagans}; whether their quest was to defend the traditional meaning of
\emph{faith} against Lutheran reinterpretation, or to find a suitable
translation for \emph{God} to communicate with Mexica peasants and Mandarin
philosophers in their own languages, Catholics found themselves, knowingly or
not, in \quoted{dialogues of the deaf}.%
\begin{Footnote}
    The English expression is borrowed from the French \emph{dialogue de sourds}
    to mean \quoted{a discussion, meeting, etc., in which neither side
    understands or makes allowance for the point of view of the other}
    (\autocite[\sv{dialogue}]{OED}).
    See \autocite{MacGaffey:DialoguesDeaf}. % XXX read
\end{Footnote}

If faith came through hearing, as Saint Paul wrote, then church leaders needed
to find a way to engage the auditory sense, some way not only to make faith
heard but actually to make it understood, and to enable people to put this faith
into practice.
For this purpose Catholics embraced music as a supernaturally powerful means of
making faith appeal to hearing.
Music was at the center of all these evangelizing and civilizing efforts:
Jesuits in Europe used music and theater to train the youth of the nobility to
follow in the Catholic flock, while their missionary brothers in Brazil and
Japan adapted Iberian folk songs to indigenous languages; the Franciscans in New
Spain formed choruses of indigenous boys and established the foundation of a
rich musical culture in the new colony.%
    \Autocites
    {Castagna:JesuitsConversionBrazil}
    {Waterhouse:EarliestJapaneseContacts}
    {Candelaria:Psalmodia}
    %    \citXXX[more on Mexican colonial music]
But the same problem of communication depicted \emph{within} the villancico
about deafness also affected the music of the villancico itself: what if the
music failed to communicate?
Given the potential for misunderstanding and confusion, how could anyone know
that \quoted{what was heard} in a musical performance actually led to faith?

The church's answer, explained in the official Roman Catechism and echoed across
the genres of theological literature and devotional art, was to accommodate
the sense of hearing and train it at the same time.%
    \autocite{Catholic:Catechismus1567}
Acccommodation meant adapting the method of communication---but not, ideally,
the message---to compensate for the weakness of the senses and the obstacles to
intelligibility caused by differing language, culture, education, and personal
temperament. 
Training meant exercising the senses, sensitizing and disciplining them to
create the capacity to perceive the church's teaching rightly and live
accordingly.
%    \citXXX[sound studies]
Devotional music might even be considered as a kind of prosthetic device to
overcome the disability of the spiritually deaf.
%    \citXXX[disability studies]
Hispanic villancicos, I argue, functioned both to make faith appeal to the
ears of a wide range of hearers, and to cultivate disciplines that would turn
hearers into listeners---people who heard the message of faith \emph{with}
faith, and who lived out their faith together as part of a harmonious society.
%    \citXXX[Dell'Antonio]

Villancicos were the most widespread form of religious music with words in
vernacular languages in the Catholic world after the Council of Trent, and they
provide evidence for a sustained endeavor by church leaders to establish
conventions of communication with ordinary people.
The creators of villancicos drew on common experiences of everyday life and
linked them to the sacred in inventive ways that met the devotional needs of
specific communities.
Each piece provides a new answer to Christ's question, \quoted{With what can we
compare the kingdom of God, or what parable will we use for it?}
(\scripture{Mk}{4:21}).
Villancicos thus represent a key component of the Hispanic church's effort
to use music to make faith appeal to hearing.
They are evidence of the church working to accommodate hearing and train it at
the same time.

This type of devotional music spoke to a variety of people at different levels
of understanding.
They were a central part of the celebrations of Christmas, Corpus Christi, the
Conception of Mary, and feasts of the saints across the Hispanic world,
performed both inside and outside the church, at Matins and Mass, in a multitude
of public and private contexts.
Villancicos were composed in sets of eight of more pieces, most commonly
interpersed between the readings of the Matins liturgy and replacing or
supplementing the Responsory chants according to local practice.
Each of these cycles includes an array of subgenres that would speak to
different portions of the congregation.
These range from silly dialogues of Christmas shepherds that would have
entertained children and their parents alike to sophisticated meditations on
metaphorical conceits, such as the pieces based on musical terminology that will
be studied in \cref{part:unhearable-music} of this book.

Even the structure of individual villancicos reflects the effort to communicate
on multiple levels.
The \emph{estribillo} section of a typical villancico was scored for full
ensemble and performed at the beginning and then repeated at the end of the
piece; composers usually set this in relatively complex polyphony similar to
what they would use for a motet.
In the center of the piece, the \emph{coplas} or verses were usually set
strophically for solo singers or a reduced ensemble with accompaniment.
As Bernardo Illari argues, the \emph{copla} settings are probably based closely
on oral traditions for singing poetry, especially in the \emph{romance} meter,
to stock melodic formulas; and it would have been easier for common listeners to
make sense of the words that were sung to the simple, repeating melodies.%
    \Autocite{Illari:Polychoral}
The \emph{estribillo}, by contrast, is often much more complex and draws on
traditions of learned counterpoint; composers often invoke a variety of
stylistic registers and styles to convey the meaning of the words and heighten
their rhetorical impact.

But though villancicos have these aspects that seem designed to engage a wide
popular audience, they differ from other dominant forms of vernacular religious
music in this period---Lutheran chorales and Reformed psalms---in that they were
not sung by ordinary parishioners.
Rather, more like Anglican anthems and German sacred concertos, they were
performed by professional church musicians for the benefit of the congregation.
The printed commemorative leaflets of villancico poetry, and the manuscript
performing parts of the musical settings preserve only one side of the church's
dialogue.
Hispanic Catholics did not, generally speaking, cultivate a society of literate,
self-advocating lay people who would have left behind traces of their personal
beliefs and devotional practices.
For the Spanish Empire, then, we know what people heard, but not what they
understood or how they responded.%
    \Autocite{Burstyn:PeriodEar} % + Did people listen? etc.
And when villancicos represent types of people---whether catechists, deaf men,
black slaves, or Indians---they leave us only with conventional caricatures, not
ethnohistorical descriptions.%
    \Autocites
    {Baker:EthnicVC}
    {Baker:PerformancePostColonial}
    {Davies:LocalContent}

All the same, the devotional music that survives from imperial Spain
can still open a fascinating window into the process of religious communication.
First, villancicos should not be understood as an exclusively top-down
communication, and certainly not as a simple mode of religious indoctrination.
The creators of villancicos were not always members of the most elite strata,
and their readers and hearers included commoners.
The cultivated poet Francisco de Quevedo was credited with mocking \quoted{the
whole caste of villancico poets} as hacks, saying that \quoted{the poor are
drowning in poets, continually hearing their braying}.%
    \Autocite
    [37: \quoted{Toda la casta de Poetas villanciqueros, que surtian de
    coplas de Gil y Menga las navidades; y los que escribian jacarandaynas para
    los ciegos se han arrimado á los cómicos, y se ahogan los pobres en Poetas,
    oyendo continuamente sus rebuznos; y si no los confundiera la grave y sonora
    armonía de la música moderna, fuera los mismo que escuchar los alaridos de
    la tortura}.]
    {Torres:SuenosMorales}
If there is any truth to the critique of villancico poets as stringing together
clichés to satisfy the tastes of a lower-class market, then the same low-class
elements that those poets disdained can provide us with insight into culture at
a more common level.
On the musical side as well, some villancico composers were not prestigious
cathedral chapelmasters and we know of at least one who was of indigenous
ancestry, Juan de Araújo in Boliiva.%
    \Autocite{Illari:Popular}
%    \citXXX[non-MC composers, Illari]
Besides, regardless of their personal background, villancio poets and composers
were following the demands of their local congregation, who according to
contemporary accounts turned out in droves to hear the annual villancico
performances.
%    \citXXX[audience turnout]
Somewhat like mass-mediated popular music today, this music was not typically
created by common people themselves, but it both reflected and shaped popular
tastes and attitudes.

Moreover, the poetic and musical texts of villancicos themselves may be read as
discourses on the process of using music to make faith appeal to hearing.
Many villancicos explicitly treat this theme, overtly asking hearers,
\quoted{Listen!} \quoted{Pay attention!} and even (in the words of a piece set
by Joan Cererols) \emph{Callar y creer}---literally, shut up and believe.%
    \begin{Footnote}
        \worktitle{Serrana, tú que en los valles}, in
        \autocite[205--212]{Cererols:MEM-VC}.
    \end{Footnote}
As demonstrated in \cref{ch:intro}, people used metamusical villancicos to
articulate their theological beliefs about music through the practice of music
itself.
A large proportion of the repertoire presents scenes of dialogue, such as most
of the examples in this chapter.%
\begin{Footnote}
    Villancicos may sometimes have been staged, even with costumes, but the
    practice varied locally and more research is needed into this question.
    % XXX Evidence?
    It is possible that the performance practice depended on the venue: for
    example, we might speculate that Juan Gutiérrez de Padilla, chapelmaster at
    Puebla Cathedral and also a priest of the Oratorian Society, presented his
    pieces unstaged at the cathedral, but might have repeated them at the
    \emph{Oratorio} in a more theatrical manner.
    If so, then villancico cycles would need to be reconsidered as an important
    early form of Spanish musical theater.
\end{Footnote}
These dramatized conversations paint colorful pictures of the sometimes bumbling
and misguided conversations between church leaders and their parishioners.

As the \quoted{villancicos of the deaf} demonstrate, problems could arise at
every stage of the effort to communicate.
Spanish devotional music manifests widespread anxieties about how people could
acquire the capacity to listen with faith.
Spaniards and their colonial subjects worried about how to listen faithfully,
about the role of subjective experience and cultural conditioning, and about the
possibility that some listeners might lack the capacity to hear music rightly.
Above all these is the danger that both the teacher and the pupil, performer and
audience, might be deaf to each other; that neither one could truly know whether
their communication had been successful.

This chapter explores the theological climate in which villancicos were created
and heard, analyzing and interpreting villancicos that focus on the link between
faith and hearing, in the context of theological literature, religious drama,
and music theory.
We will begin by discussing theological and literary sources that shaped and
reflected widespread notions about faith, hearing, and music, while also
revealing tensions in those understandings.
Then we will look at a group of related villancicos, never previously edited or
studied, that present musical discourses on the links between faith and hearing,
and manifest similar tensions.
The first two pieces stage allegorical contests of the senses in which hearing
is the favored sense of faith.
Other pieces deliberately confuse the senses to point to a higher truth that is
beyond sensation.
The last two pieces are \quoted{villancicos of the deaf}, the one already
discussed by Ruiz and an earlier example from Puebla by Juan Gutiérrez de
Padilla.
They represent characters whose impairments of hearing render them unable to
understand religious teaching, all the while poking fun at the futile
communication of some churchmen.

Villancicos on the subject of hearing and faith, I argue, provided a way for
church leaders to make faith appeal to the hearing of a broad range of
listeners.
They thus fulfilled one of the central prerogatives of Catholic teaching after
the Council of Trent, to accommodate the sense of hearing even while training
it.
If we listen closely to villancicos as historical sources for theological
understanding, it becomes clear that they functioned as much more than tools for
simplistic religious teaching or as banal, worldly entertainment.
These pieces often raise as many questions as they answer, but this tension, I
argue, was what enabled these to function as a kind of theological ear training.
The villancicos in this chapter appeal to the ear in order to give listeners an
opportunity to contemplate the challenges of faithful hearing.

%***********************************************************************
\section{Accommodating and Training the Ear}

Devotional music of the Spanish Empire should be understood as part the Catholic
Church's larger endeavor to make faith appeal to hearing.
Catholic theologians taught that since faith came through hearing, as Saint Paul
said, ministers must accommodate their teaching to the sense of hearing of their
parishioners.
They did not, though, view hearing alone as completely trustworthy.
Even if faith came through hearing, one could not believe everything one heard;
and thus Catholics emphasized not only accommodating hearing, but training it.

Though the Protestant reformers insisted that their teachings were a return to
true biblical orthodoxy, Catholic theologians believed that the reformers had
redefined faith completely.
In contrast, Catholics continued to develop the theology of faith they inherited
from medieval writers like Thomas Aquinas, in which faith was one of three
virtues or capacities, along with hope and charity.%
    \Autocite[130--132]{Schreiner:Certainty}
Simple belief in intellectual propositions was \quoted{unformed faith}
(\emph{fides informe}).%
    \Autocite[\sv{Credo in Deum Patrem, 15--20}]{Catholic:Catechismus1614}
It was essential that every Christian believe certain things, but this was not
the summit of Christian life.
The goal was that Christians would develop fully \quoted{formed faith}
(\emph{fides formata}), which \quoted{worked through} the two higher virtues to
result in  what might best be translated as \quoted{faithfulness}.
True faith was a conviction that manifested itself through ethical behavior in
fidelity to God's will.

For Catholics, then, faith encompassed beliefs, attitudes, and behaviors.
Faith connected individual experience with ethical action as part of the
community of the Church.
A man of faith was a man of virtue (the root of \emph{virtus} being \emph{vir}),
and thus faith was central to the Catholic Humanist goal of building a virtuous
society.
Christian disciples were required not only to \quoted{hear} the Church's
teaching and believe it (this would be unformed faith); they also had to
\quoted{listen} in the sense of obeying.

The Church emphasized this theme in its official guidance for religious
teachers, the new catechism \quoted{for parish priests} commissioned by the
Council of Trent and first published in 1566.%
    \Autocites[\sv{catechism}]{Catholic:Catechismus1614}{NewCatholic}
In responding to both the challenge of Protestantism and the underlying problems
that had allowed heresy to take root, the council's bishops sought to improve
the education of clergy and laity in matters of faith.%
    \Autocite[\sv{Trent, Council of}]{NewCatholic}
They required priests and bishops to preach on all Sundays and holy days, but
they also made provision to ensure that those preaching were themselves properly
grounded in the faith.
The Latin catechism of Trent was a model for teaching the clergy how to preach
and teach---a guide both to the content of Catholic faith and to the best ways
of making the faith heard and understood.

In the Roman Catechism, hearing is central to faith because the object of faith
is Christ, the Word of God, and the Church's mission is to make that Word
audible.
As the preface teaches, God communicated his own nature to humanity by taking on
human flesh in Christ.
Therefore the true Word of God was not confined to Scripture or doctrines
alone---the Word was a person, Jesus Christ, to whom the prophetic scriptures
testified and from whom the Church's doctrine flowed.%
    \Autocite
    [9: \quoted{Omnis autem doctrinae ratio, quae fidelibus tradenda sit, verbo
    Dei continetur, quod in scripturas, traditionesque distributum est}.]
    {Catholic:Catechismus1614}
In the words of John's gospel, Jesus Christ himself was the \emph{logos} or
\emph{verbum} (\scripture{Jn}{1:1}), the \quoted{Word made flesh}.
Where God had previously spoken through the law and the prophets, now he had
spoken through his Son (\scripture{Hb}{1:1}).
The catechism builds its ecclesiology, or its theology of the church, on this
theology of the Incarnation.
Christ, the incarnate Word, appointed apostles, chief among them St. Peter, to
be the custodians of the true faith.
The community of the Church, then, was the sacramental means through which
people would come to know Christ after his resurrection.
There was thus a close link in Catholic theology between hearing the Word of the
Church's teaching, obeying that Word, and encountering Christ who was the Word.
The absolute center of the Church's teaching, according to the catechism, was
found in Christ's own words in \scripture{John}{17:3}, \quoted{This is eternal
life, that you should know the only true God, and the one whom he sent, Jesus
Christ}.%
    \Autocite
    [6: \quoted{Haec est vita aeterna, vt cognoscant te solùm verum Deum, \et{}
    quem misisti, Iesum Christum}.]
    {Catholic:Catechismus1614}
All the Church's teachers should exert themselves to the end that
\begin{quoting}
    the faithful should know and love from the heart Jesus Christ, and him
    crucified (\scripture{1Co}{2:2}); and indeed to persuade them so that they
    believe with the faithfulness \addorig{pietas} of their inmost heart and
    with disciplined devotion \addorig{religio}, that there is no other name
    given to people under heaven, through whom they can be saved
    (\scripture{Ac}{4:12}), since Christ is the atonement for our sins
    (\scripture{1Jn}{2:2}).%
        \Autocite
        [6 (the scriptural citations are in marginal notes): \quoted{Quamobrem
        in eo praecipuè Ecclesiastici doctoris opera versabitur, vt fideles
        scire ex animo cupiant, Iesum Christum \et{} hunc crucifixum: sibíque
        certò persuadeant, atque intima cordis pietate, \et{} religione credant,
        aliud nomen non esse datum hominibus sub coelo, in quo oporteat nos
        saluos fieri: siquidem ipse propitiatio est pro peccatis nostris}.]
        {Catholic:Catechismus1614}
\end{quoting}
The catechism places a distinctly Catholic emphasis on obedience and faithful
living---that is, on \quoted{formed faith} that manifests in love---countering
the Protestant emphasis on salvation by faith alone.
Since \quoted{in this we know that we know him, if he keep his commandments}
(\scripture{1 Jn}{2:3}), teachers should model faithful living, \quoted{not in
leisure} but \quoted{in applying diligent effort to justice, piety, faith,
charity, mercy}, having been redeemed \quoted{to do good works} (\scripture{1
Tm}{2:12}); so that \quoted{whether one sets out to believe or hope or do
anything}, the love of God should be the summit of all Christian life.%
    \Autocite
    [6--7: \quoted{At verò quia in hoc scimus, quoniam cognouimus eum, si
    mandata eius obseruemus, proximum est, \et{} cum eo, quod diximus, maximè
    coniunctum, vt simul etiam ostendat vitam à fidelibus non in otio, \et{}
    desidia degendam esse, verùm oportere, \Dots{} sectemúrque omni studio
    iustitiam, pietatem, fidem, charitatem, mansuetudinem: dedit enim semetipsum
    pro nobis, vt nos redimeret ab omni iniquitate, \et{} mundaret sibi populum
    acceptabilem, sectatorem bonorum operum \Dots{}.
    Siue enim credendum, siue sperandum, siue agendum aliquid proponatur, ita in
    eo semper charitas Domini nostri commendari debet
    \Dots{}}.]
    {Catholic:Catechismus1614}
The church's mission was not only to instill belief but to build a community in
which people could live out their faith in love for one another.
Listening to Christ the Word meant joining oneself to the Church, which was
Christ's body, the physical and social manifestation of the Word in the world.

How, then, could the Church communicate the Word through the sense of hearing?
Drawing on Saint Paul's dictum from Romans, the catechism challenges the Church's
ministers to find a way to make Christ the Word audible:
\begin{quoting}
    Since, therefore, faith is conceived by means of hearing, it is apparent,
    how necessary it is for achieving eternal life to follow the works of the
    legitimate teachers and ministers of the faith. \Dots{} Those who are called
    to this ministry should understand that in passing along the mysteries of
    faith and the precepts of life, \emph{they must accommodate the teaching to
    the sense of hearing and intelligence}, so that by these \add{mysteries and
    precepts}, \emph{those who possess a well-trained sense} should be filled up
    by spiritual food.%
        \Autocite
        [2, 8--9 (emphasis added): \quoted{Cvm autem fides ex auditu
        concipiatur, perspicuum est, quàm necessaria semper fuerit ad aeternam
        vitam consequendam doctoris legitimi fidelis opera, ac ministerium
        \Dots{} vt videlicet intelligerent, que ad hoc ministerium vocati sunt,
        ita in tradendis fidei mysteriis, ac vitae praeceptis, doctrinam ad
        audientium sensum, atque intelligentiam accommodari oportere, vt cùm
        eorum animos, qui exercitatos sensus habent, spirituali cibo expleuerint
        \Dots{}}.]
        {Catholic:Catechismus1614}
\end{quoting}
To do this, pastors should consider \quoted{the age, intelligence
\addorig{ingenium}, customs, condition} of their charges, to give milk to
spiritual infants and solid food to the maturing, to raise up a \quoted{perfect
man \addorig{virum perfectum}, after the measure of the fullness of Christ}.%
    \Autocite
    [8: \quoted{Obseruanda est enim audientium aetas, ingenium, mores, conditio
    \Dots{}. sed cùm alij veluti modò geniti infantes sint, alij in Christo
    adolescere incipiant, nonnulli verò quodammodo confirmata sint aetate,
    necesse est diligenter considerare, quibus lacte, quibus solidiore cibo opus
    sit, ac singulis ea doctrinae alimenta praebere, quae spiritum augeant,
    donec occurramus omnes in vnitatem fidei, \et{} agnitiones filij Dei, in
    virum perfectum, in mensuram aetatis plenitudinis Christi}.]
    {Catholic:Catechismus1614}

The catechism's principle of accommodation was exemplified in vernacular
expositions of the catechism, like the Spanish versions of Antonio de Azevedo,
Juan Eusebio de Nieremberg, and Juan de Palafox y Mendoza.%
    \Autocites{Azevedo:Catecismo}{Nieremberg:PracticaCatecismo}{Palafox:Bocados}
The Spanish versions brought this teaching down to a more colloquial level,
aiming to help \quoted{laborers and simple folk} (in Palafox's words) with
earthy illustrations and paraphrases of Biblical stories.
These texts come alive when read aloud, and indeed the primary goal of this type
of literature was to prepare pastors to teach unlettered disciples through words
and voice, whether those disciples were the \quoted{Indians} of America or Asia,
or the peasants of Europe whose customs were still more pagan than Christian.%
    \Autocite
    [On Europe as a mission front after Trent, see][60--63]
    {Kamen:EarlyModernSociety}

The Augustinian friar Antonio de Azevedo begins his 1589 \wtitle{Catechism of
the Mysteries of the Faith} by declaring that communicating the faith requires
both trustworthy teachers and faithful listeners---a concept for which Azevedo
uses music as a key metaphor.%
    \Autocite{Azevedo:Catecismo}
He begins with the image taken from Pliny of a pupil listening at the feet of a
teacher who is holding a harp.
The teacher is shown, Azevedo says,
\begin{quoting}
    with a musical instrument which gives pleasure to the ear, so that we should
    understand that Faith enters through the ear \addorig{oído}, as Saint Paul
    says, and that the disciple should be like a child, simple, without malice
    or duplicity, without knowing even how to respond or argue, but only how to
    listen and learn.
    Thus this image depicts for us elegantly, what the hearer of the Faith
    should be like.%
        \Autocite
        [\range{f}{1b}:  \quoted{Para pintar los Romanos la Fe, lo primero que
        hizieron templo y altar \Dots{} Numa pompilio \Dots{} puso vn idolo: de
        forma de vn viejo cano, que tenia vna harpa en la mano, i estava
        enseñando vn niño echado a sus pies.
        En esta figura o geroglifica esta encerrada mucha filosofia, i aun
        cristiana.
        En el templo i ara denota, que la fe a de ser firme i fixa, no movediça,
        ni flaca, que a cada ayre de novedad se mueva I tambien que a menester
        maestros que la enseñen, i dicipulos que la oygan.
        I, que el maestro a de ser anciano, i maduro en edad y bondad.
        Porque la dotrina es grave, antigua, i de tomo i sustancia. No nueua, ni
        de pocos años sino antigue dende los Apostoles, i con instrumento musico
        que da gusto al oido.
        Paraque entendamos, que la Fe entra por el oido; como dize S. Pablo.
        I que el dicipulo sea como niño, sencillo, sin malicia ni doblez, sin
        saber ni replicar, ni arguir, mas de solo oir, y deprender.
        En lo qual nos dibuxa galanamente, qual a de ser el oiente de la Fe}.]
        {Azevedo:Catecismo}
\end{quoting}
The teacher's task, then, is not only to make the faith heard, but to make it
\quoted{appeal to the ear}, just as Azevedo says music does.%
\begin{Footnote}
    The Spanish word \emph{oído} could mean both \quoted{hearing} and \quoted{ear},
    much like the Greek word \emph{akoē} translated as \quoted{hearing} in
    \scripture{Romans}{10:17}.
\end{Footnote}
For Azevedo, proper teaching required a certain discipline from both the speaker
and the hearer.
Simply accommodating the ear was not enough, since the Roman Catechism speaks of
the need to train the senses as well.
With the threat of heresy all around, Catholic teachers did not want
parishioners believing everything they heard.
Instead, in line with the idea that faith was a virtue by which believers shaped
their lives in the image of Christ the word, Catholics sought to create a
community of faithful hearers---one in which both the message and the process of
teaching were controlled under the authority of the Church.
The hearer's role in this scenario is simply to listen: Azevedo does not make
any provision for the listener as a creative participant, as a reader whose
response might effect the teaching process.

How could the Church use music, then, in the effort to make Christ the Word
audible? 
The catechism's theology might suggest that music would contribute to
propagating faith because this medium could appeal in a special way to the sense
of hearing.
If teaching should appeal to the ear \emph{like} music, as Azevedo says, then
combining teaching with actual music would appeal all the more.
Villancicos, as an auditory medium based on poetry in the vernacular, would seem
like an ideal tool for accommodating the faith to Spanish-speaking listeners'
\quoted{sense and intelligence}.
This music had the capacity both to propagate faith in Christ and promote
faithful living in community, because of the medium's special appeal to the
sense of hearing, and because of its social dimensions.
At the same time, music added its own set of challenges to the task of making
faith appeal to the ear, since listeners must learn to understand not only
spoken language but musical structures as well.
In other words, music would seem to be a way of better accommodating faith to
the sense of hearing, but it also required more training to be heard in a
faithful way.

\subsection{Experiencing Spiritual Truth through Music}

This tension between accommodating the ear and training may be seen in one of
the most serious attempts to explain music's power in the seventeenth century,
by the Jesuit polymath Athanasius Kircher (1602--1680).
Kircher's compendium of musical knowledge, \wtitle{Musurgia universalis}, was
published in Rome in 1650 and thence disseminated throughout the Hispanic world,
with copies sent to centers of colonial culture including Puebla and Manila.%
\begin{Footnote}
    \Autocites{Findlen:Kircher}{Godwin:KircherTheater};
    on the worldwide distribution of the book as far as Manila, see
    \autocite[48--50]{Irving:Colonial}.  
    The book may be found in historical collections in Madrid, Barcelona, Mexico
    City, and Puebla (two copies).
% XXX  Revise if discussing Kircher in ch 1
\end{Footnote}
Kircher discusses the power of music several times throughout his ten-part
treatise, including a detailed analysis of \quoted{whether, why, and what kind
of power music might have to move people's souls, and whether the stories are
true that were written about the miraculous effects of ancient music}.%
    \Autocite
    [\range{bk}{7}, \pagenum{549}: \quoted{Vtrum, cur, \et{} quomodo Musica
    uim habeat ad animos hominum commouendos, \et{} vtrum vera sint, quae de
    mirificis Musicae Veteris effectibus scribuntur}.] 
    {Kircher:Musurgia}
Kircher's contribution to this favorite controversy of the Renaissance is to
defend the superiority of modern music on the basis of, among other factors, its
increased ability to move listeners through varieties of musical structure and
style.

For Kircher, music added such power to words, that it could move a listener not
only to understand the subject of the words, but to physically experience their
truth.
According to legend the famed \emph{aulos} player Timotheus aroused Alexander
the Great to the furor of war through music, and Kircher says he did this by
adapting his song both to the feeling of war and to the disposition of the
king.%
\begin{Footnote}
    Kircher is probably responding to \autocite[90]{Galilei:Dialogo}, in
    discussing the Classical source, Dio Chrysostom, \wtitle{Orationes} 1
    (\wtitle{Peri Basileias}); a later treatment of this subject is
    \autocite{Dryden:Alexander}.
\end{Footnote}
The same music, he says, would have had a different effect on someone else.
To illustrate this contrast, Kircher goes on to paint a remarkable picture of
how sacred music can move those who are disposed to it:
\begin{quoting}
    If, on the other hand, \add{the musician} addressed the sort of man who was
    devoted to God and dedicated to meditation on heavenly things, and wished to
    move him in otherworldly affects and rapture of the mind, he would take up
    some notable theme expressed in words---a theme that would recall to the
    listener's memory the sweetness and mildness of heavenly things---and he
    would fittingly adapt it in the Dorian mode through cadences and intervals,
    then \add{the listener} would experience that what was said was actually
    true, those heavenly things that were made by harmonic sweetness, and he
    would suddenly be carried away beyond himself to that place where those
    joyful things are true.%
        \Autocite[\range{bk}{7}, \pagenum{550}: \quoted{Sicuti contra, si quis
        Deo deuotum hominum rerumque coelestium, meditationi deditum in exoticos
        affectus raptusque mentis commouere vellet is supra insigne aliquod
        verborum thema, quod rerum caelestium dulcedinem, \et{} suauitatem
        auditori in memoriam reuocaret, modulo dorio per clausulas interuallaque
        aptè adaptet, \et{} experietur quod dixi verum esse, statim extra se
        factos dulcedine harmonica eò, vbi vera sunt gaudi rapi: vidi ego
        nonsemel in viris ordinis nostri sanctitate illustribus huiusmodi
        experimenta}.] 
        {Kircher:Musurgia}
\end{quoting}
Kircher says that the experience of Jesuit missionaries around the world
provides ample evidence of this miraculous power of music combined with
preaching.
But Kircher's depiction of music's power in this passage goes well beyond the
Jesuit formula of \quoted{teaching, pleasing, and persuading}.%
    \Autocite
    [On the Jesuit approach to religious arts, see] [35--51]
    {Bailey:Art}
Music not only makes the teaching of doctrinal truth appealing and persuasive;
it actually transforms listeners through affective experience.

In this conception, music links the objective truth with subjective experience
through the unique ways that music affects the human body.  Kircher posits that
music moves a listener's soul through the principle of sympathy.%
    \Autocites
    [On the links between this interest in the occult powers of music and early
    scientific research, see][]{Gouk:Sciences}
    {Gouk:Harmonics}
Just as a string plucked on one lute would cause a string tuned to the same
pitch on another lute to vibrate, music created physical harmony---sympathetic
vibration---between the affective content of the music and the humors of the
person listening.
When the music moved in the same way as the affections it expressed, this in
turn moved the listener to feel the same affections.
% \begin{Footnote}
%     Kircher's theory bears a striking resemblance to some recent theories of
%     music cognition, such as Lawrence Zbikowski's description of musical
%     structures as \quoted{sonic analogues for internal dynamic processes};
%     \autocite[\XXX{}]{Zbikowski:DanceTopoi}
% \end{Footnote}
Kircher explains:
\begin{quoting}
    Since harmony is nothing other than the concord, agreement, and mutually
    corresponding proportion between dissimilar voices, this proportion, then,
    of numbers, sets the air in motion; the motion, indeed, is to be varied by
    the ratios of various intervals, ascending and descending; so that the
    spirits \add{i.e., \emph{spiritus animales}}, or the implanted internal air
    \add{in the inner ear} \Dots{}, should be moved according to the proportions
    of the motion of the external air, so that the spirits' motion are effected
    in various ways; and through this affections can be engendered in the
    person.%
        \Autocite
        [552: \quoted{Cum harmonia nihil aliud sit, quam dissimilium vocum
        concordia, consensus, \et{} undequaque correspondens proportio;
        proportio autem numerorum in motu aeris elucescat; motus verò pro varia
        interuallorum, ascensus, descensusque ratione varius sit, spiritus
        quoque, siue aer internus implantatus, vti paulò ante ostensum fuit,
        iuxta proportionem motus aeris extrinseci moueatur, fit vt spiritus moti
        ope variae, indè in homini affectiones nascantur}.]
        {Kircher:Musurgia}
\end{quoting}
Kircher seems to assume that the affective properties of different modes and
styles of music are inherent in the numerical proportions of the music; in other
words, the affective character of the music he describes in the Dorian mode
would always be the same.
Music with different intervallic relationships, with the semitone placed
differently in different modes, had different affective content.

But because music's power depended on sympathetic resonance, the \emph{effect}
of music was dependent on the relationship between that objective affective
content and the subjective disposition of an individual listener.
The structure of the music's movements must correspond to the movements of the
body's humors.
Kircher theorizes that four conditions are necessary for music to achieve an
effect; without any one, music will fail to move the listener in the intended
way:
\begin{quoting}
    The first is harmony itself. Second, number, and proportion. Third, the
    power and efficacy of the words to be pronounced in music itself; or, the
    oration.
    Fourth is the disposition of the hearers, or the subject's capacity to
    remember things.%
        \Autocite
        [550: \quoted{Tertius purè naturalis est, per harmonicum, scilicet
        sonum, qui nisi quatuor conditiones annexas habeat, quarum vna
        deficiente, desideratus effectus minimé obtinebitur: Prima est ipsa
        harmonia.
        Secunda, numerus, \et{} proportio.
        Tertia, verborum in ipsa musica prouniciandorum vis, \et{} efficacia,
        siue ipsa oratio.
        Quarta audientis dispositio, siue subiectum memoratarum rerum capax}.]
        {Kircher:Musurgia}
\end{quoting}
Since there must be this kind of congruence between music and listener, Kircher
acknowledges that music affects different people in different ways.
First, Kircher concedes that geographic and cultural factors influence music
style and its effect, such that Italians and Germans are moved by different
styles and therefore compose differently.
These styles, he says, are the result of a national \quoted{genius} (that is,
the special gift of that people), as well as environmental factors, such that
Germans developed a grave style from living in a cold climate, contrasted with
the more moderate style of Italians.
People of the Orient who visit Rome, Kircher says, do not enjoy the highly
delicate music of that city, and prefer their own strident, clangorous music.
These differences of style and perception are caused by patriotism---the
inordinate love of things from one's own country, as Kircher describes it; and
by what each person is accustomed to hearing, which is shaped by the traditions
of each country.%
    \Autocites
    [543--544]{Kircher:Musurgia}
    [see partial translation in][707--711]{Strunk:SRMH}

Moreover, \quoted{just as different nations enjoy a different style of music,
likewise within each nation, people of different temperaments appreciate
different styles that conform the most to their natural inclinations}.%
    \Autocite
    [544: \quoted{Quod quemadmodum diversae nationes diuerso stylo musico
    gaudent, ita \et{} in vnaquaque natione diuersi temperamenti homines,
    diuersis stylis, vnusquisque suae naturali inclinationi maximè conformibus
    afficiuntur}.]
    {Kircher:Musurgia}
What delights a person with a sanguine temperament might enrage or madden a
melancholic listener; what has a strong effect on one person may have no effect
at all on another.%
    \Autocite[550]{Kircher:Musurgia}
\quoted{Music does not just move any subject, but the one with which the natural
humor of the music is congruent \Dots{} for unless the spirits of the receiving
subject correspond exactly, the music accomplishes nothing}.%
    \Autocite
    [550: \quoted{Musica igitur vt moueat, non qualecunque subiectum vult, sed
    illud cuius humor naturalis musicae congruit \Dots{} quae nisi recipientis
    subiecti spiritui extactè respondeant, nihil efficient}.]
    {Kircher:Musurgia}
Furthermore, Kircher suggests that not only humoral temperament but also
training and intelligence are a factor in individual listening, since he
includes the capacity of memory in his list of four conditions for effective
music.

Despite Kircher's confidence in modern musicians' ability to make music move
people, the conditions he names may not have been as easy to fulfill as he
suggests.
There must be congruence, first of all, between the structure of the music and
subject of that music: the music must move in the same way as the affective
movements it seeks to incite.
Harmonic ratios, metrical proportions, verbal rhetoric---all of these must
align, but they still are enough without the fourth condition, the disposition
of the hearer.
The listener must have a humoral temperament that is moved in the desired way by
the music.

This means that the tension between accommodating hearing and training it is
multiplied vastly by the addition of music.
While it might seem that music would allow for greater accommodation, the number
of potential obstacles is increased because the content of the music, the
performance, and the listeners must all be in harmony.
When Kircher compares music to preaching, he says that an effective preacher is
familiar with his audience and therefore \quoted{knows which strings to
pluck}---a phrase that recalls Antonio de Azevedo's image of the religious
teacher with harp in hand.%
    \Autocite
    [551: \quoted{Nouerat enim praedicator fuorum auditorum inclinationem;
    nouerat chordam, quam tangere debebat}.]
    {Kircher:Musurgia}
But Kircher never fully resolves the tension between the universal power of
music and the variables of individual subjectivity and cultural conditioning.
Music could, as Kircher describes, move someone to experience the truth of
religious teaching through affective experience, but only if the listener was
\quoted{the sort of man who was devoted to God}---that is, someone who already
had faith, whose temperament was already disposed to religious devotion of this
kind.
If, as Kircher acknowledges, people of different nations are moved by different
kinds of music, and if individual people respond differently depending on their
temperament as well as their intellectual capacity, how could the creators of
sacred music be assured of its power?
How could they know which strings to pluck?

Kircher's theory represents a highly learned, quasi-scientific attempt to
reconcile the challenges of accommodating the ear and training it.
But outside the erudite realm of theoretical speculation, this was a problem
that faced every Catholic church leader and musician who was serious about using
music to make faith appeal to hearing.
How could the Church use music to accommodate hearing, when individuals and
communities did not hear the same way?
The capacity to listen faithfully, and therefore music's power to make faith
appeal to hearing, would be impaired by both personal subjectivity and cultural
conditioning, and these limitations created anxiety and fear about the role of
hearing in faith.

%***********************************************************************
\subsection{Danger and Doubt}

Even as Catholic leaders emphasized that faith came through hearing, they also
feared that it was dangerous for individuals to trust too much in their
subjective sensory experience.
Reformation controversies had pushed Catholics into the awkward position of
simultaneously urging people to trust what they heard from the Roman
church---so as not to doubt their faith and fall away from salvation---while
also actively encouraging them to doubt the voices they heard coming from
heretical pulpits and pamphlets.
In other words, Catholics actually had to cultivate the right kind of doubt in
order to avoid the dangerous kind.

Catholic defenders like Thomas More accused Martin Luther of replacing the
trustworthy institutional church with a foolhardy reliance on individual
subjective experience.%
    \footnote{Thomas More, \worktitle{Responsio ad Lutherum} (1523).}
From the Catholic perspective, Luther was leading his flock into danger by
asking common people to listen to his voice alone and ignore the chorus of
church fathers who condemned his heresy.
It was the work of the Holy Spirit in the divinely sanctioned hierarchy of the
Roman church that gave its sacraments their objectively operating power and
allowed that faithful to have certainty of faith and salvation---not personal
interpretations of Scripture or some kind of inner conviction.%
    \Autocite[131--208]{Schreiner:Certainty}

The post-Reformation Catholic theology of faith and sensation, however, did not
completely remove the subjective element.
On the one hand, the catechism of Trent is careful to teach that the human
senses alone cannot reach the knowledge of God, according to a tradition of
Augustinian Neoplatonism:
\quoted{in order for our minds to reach God, since nothing is more sublime than
God, our mind needs to be pulled away from everything that pertains to the
senses---something that we, in this natural life, do not have the capacity to
do}.%
    \Autocite
    [18: \quoted{Nam vt mens nostra ad Deum, quo nihil est sublimius, perueniat,
    necesse est eam omnino à sensibus abstractam esse: cuius rei facultatem in
    hac vita naturaliter non habemus}.]
    {Catholic:Catechismus1614}
On the other hand, the catechism acknowledges that faith requires an individual
response: it is more than having an opinion or conception of something, but
rather, faith \quoted{has the strength of the most certain agreement, such that
the mind, having been opened by God to his mysteries, firmly and steadfastly
gives assent}.%
    \Autocite
    [15: \quoted{Igitvr credendi vox, hoc loco putare, existimare, opinari, non
    significat, sed vt docent sacrae litterae, certissimae assensionis vim
    habet, qua mens Deo sua mysteria aperienti, firmè constantérq;
    assentitur}.]
    {Catholic:Catechismus1614}
This theology of faith begins with God's grace opening the mind of the
passive person, but ends with an act of the individual will.
Though God is beyond sensation, sensation is the entrance to the path. 
Faith comes through \quoted{what is heard}, then, from the created world that
speaks of God's nature to those who can perceive it, from the Scriptures, and
chiefly through the authoritative Roman church.
Each person is still called to a process of seeking to know God, to move beyond
sensory experience, and ultimately to \quoted{firmly and steadfastly give
assent} to the truth of God.
Moreover, reacting against the perceived \quoted{fideism} of the Lutherans, who
taught that a person's good works could not suffice for their salvation,
Catholics emphasized that faith had to bear fruit in ethical action.

Spanish theological writers cultivated disciplines for regulating spiritual
experiences and submitting individual sensation to the Church's authority.
The Spanish Inquisition investigated Teresa of Ávila (1515--1582) and other
mystics who claimed authority only on the basis of spiritual experiences, and
most were not as successful as Teresa in avoiding punishment.%
    \Autocites{Ahlgren:TeresaPolitics}{Francisca:Inquisition}
Teresa's student John of the Cross (Juan de la Cruz, \circa{1542}--1591) taught
contemplatives to pursue union with God by weaning themselves of sensory
experiences in the \quoted{dark night of the soul}.
The Carmelite reformer defines that union not in sensual terms but in ethical
ones, as the total surrender and conformity of one's will to God.%
    \Autocite
    [\range{bk}{I}, \range{ch}{5--7}, \pagenums{226--248}]
    {JuandelaCruz:Subida} % Carmelite secondary
Similarly, Ignatius of Loyola (\circa{1491}--1556) provided Jesuits with his
\wtitle{Spiritual Exercises} for discerning the validity of their religious
sensations, and for using those experiences to make decisive changes in life, as
part of a radical commitment of the self to God.%
    \Autocite[\range{ch}{6}]{Schreiner:Certainty}

In such a climate, music might be used for the purposes of cultivating faith,
because of its power over the senses and affections, but this power could also
be dangerous.
Some Catholics such as the Jesuit missionaries, despite their founder's
suspicion of music, were eager to use this power to advance the cause of the
Church.
On the mission field, the Jesuits were involved with subjective experience to
such a degree that they even interpreted the dreams of native people in New
Spain.%
    \Autocite[40--41]{Bailey:Art}
Others, though, saw the use of music as a potential distraction or distortion.
For example, John of the Cross complains that even though churches would seem the
ideal place for prayer, their decorations, ceremonies, and music so engage a
person's senses that it can be impossible to worship God \quoted{in spirit and
in truth} (\scripture{Jn}{4:23--24}).%
    \Autocite
    [\range{bk}{3}, \range{ch}{39--45}, \pagenums{415--424}]
    {JuandelaCruz:Subida}

John warns that desiring \quoted{to feel some effect on oneself} in doing
elaborate ceremonies \quoted{is no less than to tempt God and provoke him
gravely; so much so, that sometimes it gives license to the devil to deceive
them, making them feel and understand things far removed from what is good for
their soul}.%
    \Autocite
    [\range{bk}{3}, \range{ch}{43}, \pagenum{420}: \quoted{Y lo que es peor es
    que algunos quieren sentir algún efecto en sí, o cumplirse lo que piden, o
    saber que se cumple el fin de aquellas sus oraciones ceremoniáticas; que no
    es menos que tentar a Dios y enojarle gravemente; tanto, que algunas veces
    da licencia al demonio para que los engañe, haciéndolos sentir y entender
    cosas harto ajenas del provecho de su alma, mereciéndolos ellos por la
    propiedad que llevan en sus oraciones, no deseando más que se haga lo que
    Dios quiere que lo que ellos pretenden}.] 
    {JuandelaCruz:Subida}
In a strong counterpoint to Kircher's statement about music's power to augment
preaching, John admonishes his readers to be wary of overly artful preaching
which, like music, only serves to stimulate \quoted{the sense and
understanding}---John uses the exact language of the Roman Catechism---but has
no impact on the hearer's will to live faithfully:
\begin{quoting}
    How commonly we see that \Dots{} if the preacher's life is better, greater
    is the fruit that he gains, though his style be low and his rhetoric scanty,
    and his teaching common, because the living spirit infuses him with ardor;
    but the other preacher gets very little gain, no matter how much more
    elevated his style and doctrine may be: because, though it is true that good
    style and actions and elevated doctrine and good language move and create
    more effect when accompanied by a good spirit, without the spirit, though
    the sermon may give the sense and understanding much to savor and enjoy, it
    infuses little or no sustenance to the spirit, because commonly it remains
    as lax and loath as before to labor, even though marvelous things were said
    in marvelous ways, which only serve to delight the ear \addorig{oído}, like
    some polyphonic music \addorig{una música concertada} or the clanging of
    bells \Dots{}.
    It matters little to hear someone perform one kind of music that is better
    than some other, if it does not move me more than the other to do works;
    because, although they have spoken marvels, then they are forgotten, as they
    do not infuse fire into the will.%
    \Autocite
    [\range{bk}{3}, \range{ch}{45}, \pagenum{425}: \quoted{Que comúnmente vemos
    que---cuanto acá podemos juzgar---cuanto el predicador es de mejor vida,
    mayor es el fruto que hace por bajo que sea su estilo, y poca su retórica, y
    su doctrina común, porque del espíritu vivo se pega el calor; pero el otro
    muy poco provecho hará, aunque más subido sea su estilo y doctrina; porque,
    aunque es verdad que el buen estilo y acciones y subida doctrina y buen
    lenguaje mueven y hacen más efecto acompañado de buen espíritu; pero sin él,
    aunque da sabor y gusto el sermón al sentido y al entendimiento, muy poco o
    nada de jugo pega a la voluntad, porque comúnmente se queda tan floja y
    remisa como antes para obrar, aunque haya dicho maravillosas cosas
    maravillosamente dichas, que sólo sirven para deleitar el oído, como una
    música concertada o sonido de campanas; mas el espíritu, como digo, no sale
    de sus quicios más que antes, no teniendo la voz virtud para resucitar al
    muerto de su sepultura.  
    Poco importa oír una música mejor que otra sonar, si no me mueve ésta más
    que aquélla a hacer obras; porque, aunque hayan dicho maravillas, luego se
    olvidan, como no pegaron fuego en la voluntad}.] 
    {JuandelaCruz:Subida}
\end{quoting}
This passage of warning comes at the end of John's encyclopedic treatment of
contemplative practice and mystical experience, \wtitle{The Ascent of Mount
Carmel}.
In concluding the work the monk seems to deliberately echo the \quoted{love}
chapter of 1 Corinthians in insisting that, as John says elsewhere, \quoted{a
single work or act of the will done in charity is more precious before God than
all the visions and revelations and communications from heaven that there can
be}.%
    \Autocite
    [\range{bk}{2}, \range{ch}{22}, \pagenum{306}: \quoted{\Dots{} es más
    preciosa delante de Dios una obra o acto de voluntad hecho en caridad que
    cantas visiones y revelaciones y comunicaciones pueden tener del cielo}.]
    {JuandelaCruz:Subida}

The contrast between this passage and the quotation from Kircher shows that
Catholics were not all of one mind about sensation and faith.
The difference between these two descriptions of music's power reflects  a
chronological change in Catholic attitudes between the late sixteenth and mid
seventeenth centuries, as well as a contrast between religious orders, and
differing emphases on parts of the Christian life.
John's \wtitle{Ascent of Mount Carmel} was first published in 1618, though it
was written some forty years earlier; his outlook reflects the hard ascetic
extreme of Catholic Reformation attitudes toward music.
Kircher, on the other hand, writes with the confidence of the Jesuit order at
the height of its global influence, when the Society was still trying to engage
with any aspect of culture that they saw as advancing their mission.
Kircher is interested in music's power to convert---not only to persuade
catechumens of the truth but to enable them to experience truth.
John, on the other hand, is concerned with the limiting effect of sonic
pleasures on the maturity of the already converted.
Both acknowledge music's power over the individual person's senses, but they
differ in how much they trust that power to produce godly effects.
Later Catholics who shared John's ascetic bent, or his pastoral concern with
spiritual growth, saw that power as a danger or even an opening for diabolical
influence.


\subsection{Cultural Conditioning in Hearing}

If reckoning with individual sensory experience was a struggle for church
leaders after Trent, their evangelistic and colonial efforts posed the even
greater challenge of training entire communities to hear faithfully.
Catholics at the boundaries of intercultural encounters around the globe found
themselves in what Ines Županov has termed the \quoted{missionary
tropics}---both a geographic zone and a process of troping or turning through
cultural transformation.%
    \Autocite{Zupanov:MissionaryTropics}
Some missionaries like the Jesuits in Brazil and Paraguay actively sought to
accommodate local customs and music; but everywhere that missionaries brought
Christian faith, the process of cultural translation inevitably transformed it
into something neither they nor their converts could necessarily predict.%
    \Autocites
    {Castagna:JesuitsConversionBrazil}
    {Bailey:Art}
    {Fromont:DancingKingCongo}
% XXX plus others
Half a century after the Aztec empire fell to the Spanish, the Franciscan
Bernardino de Sahagún was trying to provide Mexica natives with suitable
Christian songs in their own language.
But at the same time church leaders in New Spain complained that most places
lacked linguists sufficiently skilled in Náhuatl to verify that the songs and
dances of the natives were not in fact still invoking the demonic powers of the
Aztecs.%
    \Autocite[637]{Candelaria:Psalmodia}
As Lorenzo Candelaria has suggested, when the Council of Trent issued its
infamous decree that music should be preserved from anything \quoted{lascivious
or impure}, the bishops may have been motivated more by concern with such songs
in non-European languages, not to mention other syncretic musical and ritual
practices across the globe, than by any European church music.%
    \Autocite[637--638]{Candelaria:Psalmodia}
% XXX  on Trent
In the realm of visual art, even where Jesuits wrote to their superiors in Rome
praising the natives' uncanny ability to copy European models, the actual
surviving artifacts often bear strong traces of indigenous methods, aesthetics,
and religious understandings.%
    \Autocite[27--29, 34]{Bailey:Art}
Likewise, given the differing techniques and styles of vocal production between
places like central Africa and south India today, the actual performance of
music in mission churches and colonial cathedrals, even plainchant and polyphony
imported directly from Europe, probably sounded quite different the practice of
European chapels.
As the Church was adapting itself to native sensibilities, or being adapted by
colonial subjects in ways the Church could not fully control, how was the church
to accommodate its teaching to the \quoted{sense and intelligence} of all these
different people?
The church could proclaim \quoted{the Faith}, but how could leaders know that
people heard what they intended?
And even more challenging, once these processes of cultural adaptation and
exchange had begun to \quoted{turn}, what parts of Christianity constituted
\quoted{the Faith} that was supposed to come through hearing?
Did changing the musical style of worship, for example, mean changing the Faith
as well?
If ways of hearing music were culturally conditioned, then religious ear
training was required.
The cultural aspect of acquiring a \quoted{properly trained sense}, as the Roman
Catechism puts it, is plainly stated in a Latin dialogue published by the
leaders of the Jesuit mission in Japan in 1590.%
    \Autocite{Sande:DeMissioneLegatorum}
The missionaries had taken four Japanese noble youths on a grand tour of Spain
and Italy between 1582 and 1590, during which they trained in music and heard
the finest ensembles of Catholic Europe.
Their trip included most of the major Iberian musical centers discussed in this
study: on the outgoing trip, Lisbon, Évora, Toledo, Madrid, and Alcalá; and on
the return, Barcelona, Montserrat, Zaragoza and Daroca.
With this trip and with the subsequent publication, the leader of the Jesuit
mission to Japan, Alessandro Valignano, hoped to persuade the authorities of his
order and church that (as Derek Massarella puts it) \quoted{European Jesuits
must accommodate themselves to Japanese manners and customs}.%
    \Autocite[4]{Massarella:JapaneseTravellers}

In Colloquio XI of the dialogue, Michael, one of the \quoted{ambassadors}, tells
his friend who stayed home about European music:
\begin{quoting}
    You must remember \Dots{} how much we are swayed by longstanding custom, or
    on the other side, by unfamiliarity and inexperience, and the same is true
    of singing.
    You are not yet used to European singing and harmony, so you do not yet
    appreciate how sweet and pleasant it is, whereas we, since we are now
    accustomed to listening to it, feel that there is nothing more agreeable to
    the ear.

    But if we care to avert our minds from what is customary, and to consider
    the thing in itself, we find that European singing is in fact composed with
    remarkable skill; it does not always keep to the same note for all voices,
    as ours does, but some notes are higher, some lower, some intermediate, and
    when all of these are skillfully sung together, at the same time, they
    produce a certain remarkable harmony \Dots{} all of which, \Dots{} together
    with the sounds of the musical instruments, are wonderfully pleasing to the
    ear of the listener \Dots{}.

    With our singing, since there is no diversity in the notes, but one and the
    same way of producing the voice, we don't yet have any art or discipline in
    which the rules of harmony are contained; whereas the Europeans, with their
    great variety of sounds, their skillful construction of instruments, and
    their remarkable quantity of books on music and note shapes, have hugely
    enriched this art.%
        \Autocite[155-156]{Massarella:JapaneseTravellers}
\end{quoting}
Michael's friend Linus responds with a statement that surely understates the
attitude of many non-Europeans whose ears had not yet been trained for European
music:
\begin{quoting}
    I am sure all these things which you say are true; for the variety of the
    instruments and the books which you have brought back, as well as the
    singing and the modulation of harmony, testify to a remarkable artistic
    system.
    Nor do I doubt that our normal expectations in listening to singing are an
    impediment when it comes to appreciating the beauties of European harmony.%
        \Autocite[156]{Massarella:JapaneseTravellers}
\end{quoting}


\subsection{Obstacles to Faith and Mistrust of Hearing}

How, then, could the Church overcome such an impediment?
Would music's power be lost on a person who simply lacked the proper disposition
to hear the Word with faith?
These doubts were displayed vividly in a sacred drama performed before King
Philip IV, \wtitle{El nuevo palacio del Retiro} by court poet Pedro Calderón de
la Barca.%
    \Autocite{Calderon:Retiro}
This Corpus Christi mystery play (\emph{auto sacramental}), was staged in Madrid
in 1634 to inaugurate the Spanish monarch's new palace-retreat, the Buen Retiro.
Like most \emph{autos} this one was performed in the context of a whole day of
liturgical and processional music, probably including numerous villancicos; and
the drama includes texts to be sung, though the music is lost.
%    \citXXX[sources for Corpus Christi 1634 music; villancicos from 1630s about
%    Buen Retiro]
Literary scholars have considered Spanish \emph{autos sacramentales} to be
\quoted{dramas of conversion} with the primary goal of teaching
doctrine.
%    \citXXX[Wardropper, Parker, McKendrick]
Thus this literary and performative genre, which was so closely tied to
devotional music, allows us to evaluate how auditory arts were used to propagate
faith, and how that effort served the purposes of the Spanish state.
The plays typically present allegorical characters whose conflicts dramatize
theological debates in the scholastic tradition (as in Aquinas) of thesis,
antithesis, and synthesis.

Calderón's play for the politically significant occasion glorifies the king as
the defender of the true faith, since his name \emph{Felipe} cannot be said
without saying \emph{Fe} (faith)---a faith that centers on the doctrine of the
Eucharist.%
    \Autocite[\poemline{\XXX}]{Calderon:Retiro}
Since the Eucharist defies normal sensation and must be perceived through faith,
much of the play sets up debates about the relationship between faith and the
senses.
By staging a contest of the senses before Faith (all personified as characters),
Calderón promulgates the church's teaching that unless one possessed the
divinely implanted virtue of faith to begin with, the senses could actually
become obstacles to faith.

If the King was the defender of the faith, then Judaism was its enemy, as
dramatized by the long portion of the play that centers on Judaism's incapacity
for faith.
The allegorical character \emph{Judaísmo} becomes a chilling representation of
the incapacity to acquire faith, \emph{despite} the sense of hearing.
Judaism is forcefully excluded from the festivities celebrated within the play,
which culminate in the consecration of the Eucharist.
Instead Judaism stands to the side and asks the character Faith to explain each
event to him (\cref{poem:Calderon-Retiro-Judaismo}).
But despite trying to connect Faith's message with what he knows of the Hebrew
Scriptures, Judaism cannot accept any of these explanations.
In fact he is unable to believe what Faith has said, because, as he says in an
increasingly embittered refrain, \quoted{I have listened to Faith without
Faith}.

\begin{poemexample}
    \caption{Calderón, \wtitle{El nuevo palacio del Retiro}, \poemlines{1303--1336}:
    Judaism rejects faith}
    
    \label{poem:Calderon-Retiro-Judaismo}
    \includePoem{Calderon-Retiro-Judaismo}
\end{poemexample}

Judaism's eloquent confession of unbelief is immediately drowned out by music,
as clarion fanfares announce a royal procession.
For Calderón's listeners, who had been taught to regard Jews as the embodiment
of willful unbelief and worse, the entry of the musicians would clear away the
acrid sound of Judaism's speech.
The feeling of doubt about the senses, however, pervades the entire play.

Two other sections of the play dramatize a contest of the senses, in which
Hearing prevails---but only after confessing to his own incertitude.
Each personified sense competes for a laurel prize awarded by Faith
(\cref{poem:Calderon-Retiro-Hearing}).
Each sense in turn boasts of his powers, but Faith rejects each one.
Hearing is the last sense to present himself, and in contrast to the other
senses, he speaks of his weakness, and how easily he can be fooled by echoes or
feigned voices.
Since he cannot trust his own powers, he must rely on faith.
In response, Faith crowns Hearing precisely because of his
\emph{desconfianza}---a term that can mean humility as well as lack of
confidence and even mistrust.

\begin{poemexample}
    \caption{Calderón, \wtitle{El nuevo palacio}, \poemlines{593--602}: Faith
    crowns Hearing}

    \label{poem:Calderon-Retiro-Hearing}
    \includePoem{Calderon-Retiro-Hearing}
\end{poemexample}

What would it mean, then, for hearing to be the favored sense of faith not just
because of its humility, but because of its actual weakness?
Dominique Reyre interprets the play as a straightforward medium for
\quoted{transmission of dogma}, and Margaret Greer analyzes it anthropologically
as a ritual that reinforced the king's power; but neither scholar accounts
sufficiently for the high degree of uncertainty expressed on stage.%
    \Autocite{Reyre:Retiro}{Greer:Retiro}
Calderón presents one character, Judaism, who hears what Faith says but lacks
the faith to believe it; and another, Hearing, who admits that he cannot trust
his own sense but for that very reason receives Faith's favor.
In Calderón's play, the mysteries of the Eucharist are beyond physical
sensation: vision,
taste, touch, and smell would not lead to the truth of Christ's presence in the
host, but only hearing and believing Christ's words \quoted{This is my body} as
spoken by the priest.
Fully in line with the discourse on sensation and faith in the Roman Catechism,
Calderón advises his listeners to trust some senses while distrusting others; to
seek God through sensation while purging themselves of reliance on the senses.

The dialectic of trust and doubt in the senses urged Catholics to rely on the
Church community and not their own experience.
The senses were powerful, and thus sensory art was powerful; but this power had
to be harnessed to serve the purposes of the Church.
Simply put, faithful hearing required listeners to doubt their own sensation.


\subsection{Hearing the Word in Community}

Catholic devotional music provided a practical medium for both appealing to the
ear and training it, though music amplified the challenges of acquiring faith
through hearing.
Catholic listeners were encouraged to doubt their senses as much as to trust
them; and church leaders struggled with the frightening possibility that some
people might simply lack the capacity for hearing with faith.
Religious ear training required individual discipline to avoid the danger of
over-reliance on subjective sensory experience and to learn to discern the
spiritual truth communicated through musical patterns.
This training would also need to discipline the whole community to overcome
misunderstandings based on cultural conditioning.

Propagating faith, then, meant trying to establish not just individual
Christians, but also building a Christian society as the body of Christ.
Faithful Catholics had to learn to submit their sensory experience to the
authority of the Church as the source of certainty, as the living, communal
embodiment of Christ the Word in the world.
For Roman Catholics, the Church \emph{was} the gospel, and the task of building
the Church could not be separated from the work of building an empire.

As Catholics worked to create Christian communities, music was a potent tool for
creating harmony, for instituting social discipline as a reflection of the
heavenly hierarchy.%
    \Autocites{Baker:Harmony}{Irving:Colonial}{Illari:Polychoral}
The virtue of man as Neoplatonic microcosm was reflected in the broader society
and in turn depended on it.
Spanish political thinkers conceived of the colonial project in terms of
establishing harmony in society.%
    \Autocite[22--31]{Baker:Harmony}
Most educated Spaniards were familiar with the medieval philosopher Boethius
(either directly or through expositions of his ideas in contemporary music
treatises like that of Pedro Cerone) and his concept that there were three kinds
of music: \emph{musica instrumentalis}, sounding, playing music; \emph{musica
humana}, the harmony of the individual in body and soul, reason and passion, and
the concord of human society; and \emph{musica mundana}, the music of the
celestial spheres.%
    \Autocites
    [\range{bk}{2}, \pagenums{187--189}]{Cerone:Melopeo}
    [203--208]{Boethius:Musica}
The proper performance of \emph{musica instrumentalis}, they believed, could
actually attune the \emph{musica humana} on individual and social levels,
bringing human society in concord with the order of the cosmos, and beyond it,
with the mysterious harmonious of the triune God.

Catholic music, then, was not \emph{about} society; it was a form \emph{of}
society.
This is why the Franciscan friars in New Spain and the Jesuit priests in Brazil
not only started parishes, but also trained choirs.
Forming choirs of boys and training ensembles of village musicians in colonial
Mexico were practical means of establishing the Church and propagating faith on
individual and communal levels.
The musical ritual of the seventeenth-century Church involved a large number of
community participants, for whom performing music with the body and hearing it
were inextricably linked.
The musical efforts of the colonizing church concretely built social
relationships through musical training.%
    \Autocite{RamosKittrell:PlayingCathedral}
For this reason, we cannot fully understand the faith of early modern Catholics
on the basis of verbal formulations alone; we need to see and hear how
communities practiced their faith through coordinated action---such as in
devotional music.%
\begin{Footnote}
    The Lutheran hymn composer Johann Crüger advocated a similar concept of
    \quoted{the musical practice of piety} (\wtitle{Praxis pietatis melica},
    1647 and many later editions), coming out of the Lutheran \quoted{new piety}
    movement of the seventeenth century, whose proponents (Martin Moller, Johann
    Arndt) were inspired by much of the same medieval devotional literature as
    their Catholic counterparts.
\end{Footnote}

%******************************************************************************
\section{The Primacy of Hearing in Villancicos}

Villancicos provide evidence for a broad public discourse about sensation and
faith that manifests these same theological preoccupations and anxieties.
Through vernacular poetry and likely also through musical style, the creators
and performers of villancicos addressed hearers of multiple social stations and
levels of education.
These pieces often represented complex theological concepts, but they made these
concepts accessible in quite a different way than a theological treatise or even
a sermon.

Certainly it is legitimate to ask, as José María Díez Borque has done for
Calderonian drama, how much listeners could actually hear and understand of
these complex performances---a question that would also apply to other
vernacular music like Italian oratorios and German Lutheran sacred concertos.%
    \Autocite{DiezBorque:Publico}
How much of the poetic text could be perceived by listeners outside the walls of
the cathedral choir or on the other side of convent partitions? Factors of
acoustics, vocal performance practice, and instrumentation (such as the balance
between singers and players), would all come into play, in addition to the
problems of individual training already discussed.  
Educated people who possessed the means and connections to acquire printed
pamphlets of villancico poetry would doubtless have understood more in the
moment of performance and in reflection on the text before or after.
But even common people may have had the aural capacity to perceive the gist of
what these pieces were trying to communicate.
Margit Frenk argues that Spanish literature of this period was written for the
ear and that reading in most cases meant one person reading out loud to an
audience of illiterate family and friends, whose auditory comprehension and
memory would probably astound us today.%
    \Autocite{Frenk:Voz}  
As individual villancicos were repeated, and as conventional villancico types
were performed at multiple festivals each year in public settings where many
kinds of listeners were present, these pieces must have shaped attitudes and
beliefs in the broader community.


\subsection{Contests of the Senses at Segovia Cathedral (Irízar, Carrión)}

Two villancicos from Segovia in the later seventeenth century demonstrate that
devotional music could present a sophisticated discourse on sensation and faith,
one that not only educated hearers about doctrine but actually challenged them
to listen with new ears.  
The contest of the senses in Calderón's play is echoed in villancicos by
successive chapelmasters at Segovia Cathedral.
In the 1670s, Miguel de Irízar set the villancico \wtitle{Si los sentidos queja
forman del Pan divino}, later attributed to the Zaragozan poet Vicente Sánchez,
for eight voices and instruments in grand polychoral style (\sig{E-SE}{18/19,
5/32}).%
    \Autocite{LopezCalo:Segovia}
Irízar was born in 1634 and served at Segovia from 1671 until his death in 1684.
A few decades later Irízar's successor, Jerónimo de Carrión (1660--1721), set
the same text as a solo continuo song (\sig{E-SE}{28/25}).%
    \Autocite[133--152]{Cashner:WLSCM32}

The poem invites hearers to imagine the senses \quoted{filing a complaint}
against the bread of the Eucharist because \quoted{what they sense is not by
faith consented} (\emph{porque lo que ellos sienten no es de fe consentido},
\crefrange{poem:Si_los_sentidos-Sanchez-1}{poem:Si_los_sentidos-Sanchez-2}).%
    \Autocite[171--172]{Sanchez:LiraPoetica}
In a motto that is repeated after each copla, the poem admonishes, \quoted{let
the senses not resent it} (\emph{no se den por sentidos los sentidos})---playing
on the word for \emph{sense} in several ways.  
Each of the coplas treats a different sense, following nearly the same order as
in Calderón's play: Sight comes first, followed by Touch; next are Taste and
Smell, and Hearing comes last and wins the day (\cref{tab:senses-order}).

\begin{poemexample}
    \caption{\wtitle{Si los sentidos queja forman del Pan divino}, attr. Vicente
    Sánchez, \wtitle{Lyra Poética} (Zaragoza, 1688), 171--172, first portion}
    
    \label{poem:Si_los_sentidos-Sanchez-1}
    \includePoem{Si_los_sentidos-Sanchez-1}
\end{poemexample}

\begin{poemexample}
    \caption{\wtitle{Si los sentidos queja forman del Pan divino}, conclusion}

    \label{poem:Si_los_sentidos-Sanchez-2}
    \includePoem{Si_los_sentidos-Sanchez-2}
\end{poemexample}


\begin{table}
    \caption{Order of the senses in versions of \wtitle{Si los sentidos},
    correlated with Calderón, \wtitle{El nuevo palacio del Retiro}, and
    Veracruce, \wtitle{Phisica, speculatio}} 
    
    \label{tab:senses-order}
    \includeTable{senses-order}
\end{table}

The two surviving settings of \wtitle{Si los sentidos} by Irízar and Carrión
stage this contest of the senses in sound.
The Segovia Cathedral archive preserves manuscript performing parts for both
settings along with Irízar's draft score for his version.%
\begin{Footnote}
    The performing parts for Irízar's setting are in \sig{E-SE}{5/32} (in a
    copyist's hand), and \sig{E-SE}{18/19} contains the the score in the
    composer's own hand.
    Carrión's performing parts are in \sig{E-SE}{28/25}.
\end{Footnote}
As rare as composers' scores of villancicos are, Irízar's manuscripts are doubly
valuable because this economical composer drafted his music in makeshift
notebooks stitched together from piles of his received letters, fitting up to
twelve staves of music in the margins and unused sides of the paper.

The letters reflect a correspondence with a broad peninsular network of
musicians, often about exchanging villancico poetry (see \cref{ch:segovia}).%
    \Autocite{LopezCalo:IrizarLetters1}{Olarte:Irizar}{Rodriguez:Networks}
Irízar probably received this poem through such correspondence with a colleague
in Zaragoza.
Carrión likely knew (and probably performed) Irízar's setting from the archive,
but the differences between his text and that of Irízar suggest that Carrión had
a separate source for the Sánchez poem, one that was closer to the poem
published in Sánchez's 1688 posthumous works.%
    \Autocite[171--172]{Sanchez:LiraPoetica}

In the earlier setting, for Corpus Christi 1674 at Segovia Cathedral, Miguel de
Irízar creates a musical competition in festival style by pitting his two
four-voice choirs against each other in polychoral dialogue
(\cref{music:Irizar-Si_los_sentidos}).%
\begin{Footnote}
    \Autocite[133--148]{Cashner:WLSCM32}.  
    The score includes the heading, \quoted{Fiesta del SSantissimo de este año
    del 1674}.
\end{Footnote}
This kind of dialogue between separate choirs in one ensemble is typical of
large-scale villancicos, but Irízar has the choirs interrupt each other in ways
that create not just a dialogue, but a debate.
Like a film editor creating a fight scene, Irízar builds intensity by cutting
the text into shorter phrases to be tossed back and forth between the two
choirs: \emph{no se den por sentidos} becomes \emph{no se den} and then
\emph{no, no}.

\begin{musicexample}
    \caption{Miguel de Irízar, \wtitle{Si los sentidos queja forman del pan
    divino} (\sig{E-SE}{18/19, 5/32})}

    \label{music:Irizar-Si_los_sentidos}
    \includeMusic{Irizar-Si_los_sentidos}
\end{musicexample}

Irízar creates a steadily increasing sense of excitement through shifts of
rhythmic motion and style.
The setting of the opening phrase suggests a tone of hushed awe: the voices sing
low in their registers, with a slow harmonic rhythm, and pause for prominent
breaths (\measures{1--9}).
The harmonies here change less frequently than in the following sections,
creating a relatively static feeling for this introduction.
In \measure{10} Irízar has the ensemble switch to ternary meter and increases
the rate of harmonic motion.
The sense of antagonism is heightened when one choir interrupts the other with
exclamations of \emph{no} on normally weak beats (\measures{16--17}).
When Irízar returns to duple meter in \measure{25}, the voices move in smaller
note values (\emph{corcheas}) and exchange shorter phrases, so that the tempo
feels faster (and the actual tempo could certainly be increased here in
performance).
Each choir's entrances become more emphatic, repeating tones in simple triads,
and Irízar adds more offbeat accents and syncopations, particularly for \emph{no
se den por sentidos los sentidos} in \measures{32--53}.
The estribillo builds to a climactic \emph{peroratio} (the rousing conclusion of
an oration) with the voices breaking into imitative texture in descending
melodic lines.

The distinguishing stylistic characteristics of the setting suggest that Irízar
is evoking a musical battle topic, a style one may find in \emph{batallas} for
organ as well as other villancicos on military themes.%
\begin{Footnote}
    \Autocite[\sv{battle music}]{Grove}; 
    \autocite{Sutton:IberianBatalla} discusses examples including those in
    Martín y Coll's \wtitle{Huerto ameno de flores de música} (Madrid, 1709).
    Another villancico in this style is Antonio de Salazar's \wtitle{Al campo, a
    la batalla} (\sig{MEX-Mc}{A28}).
\end{Footnote}
Battle pieces typically feature a slow, peaceful introduction followed by
sections in contrasting meters and styles and a texture of dialogue between
opposing groups (as in between high and low registers on the keyboard).  
Typical of the style is the reiteration of \musfig{5}{3} (\quoted{root
position}) chord voicings, with the bass moving by fourths and fifths, and the
syncopated 3--3--2 groupings of \emph{corcheas} (eighth notes) on \emph{no se
den por sentidos los sentidos} as in \measures{31--33}.%
\begin{Footnote}
    Compare, for example, the anonymous \emph{Batalha} in \sig{P-BRad}{Ms.963},
    \range{fol}{56}; \autocite{Araujo:Batalla}.  
% XXX  specific points of comparison, with music example, to a batalla that
    % could be tied to Segovia or Irízar's network?  
\end{Footnote}

Irízar sets the coplas, by contrast, in a sober and deliberate style.
The melody moves more calmly in duple meter with melodic phrases that fit well
with the rhetorical structure of the poetic strophes.  
Irízar has the treble soloist sing the third and fourth lines of each strophe in
short paired phrases; each time the second phrase repeats the first, transposed
down a fifth.
This creates a feeling of a teacher saying \quoted{on the one one hand} and
\quoted{on the other hand} that suits the general philosophical tone of the
strophes and matches the specific poetic phrasing of these lines.
To recall the Jesuit formula, Irízar's estribillo seems more designed to
delight, while the coplas provide more of an opportunity to teach.

Irízar's villancico seems to speak to a large crowd through bold, unsubtle
gestures and sharp contrasts of bright colors.
Jerónimo de Carrión's later setting of the same poem invites a more personal
reflection (\cref{music:Carrion-Si_los_sentidos}).%
    \Autocite[149--152]{Cashner:WLSCM32}
Carrión was capable of the festival style, as in his \wtitle{Qué destemplada
armonía} (\sig{E-SE}{20/5}), which almost takes on the dimensions of a
\emph{cantada}.  
But this setting fits more in the subgenre of \emph{tono divino} or chamber
villancico, a continuo song used in more intimate settings like Eucharistic
devotion.%
    \Autocite[See, for example][]{Robledo:MadridTonos}
% XXX  article in Knighton/Torrente on chamber vs. tutti villancicos?  
The style is similar to the \quoted{high Baroque} music of contemporary Italy,
with a tonal harmonic language, a running bass part in the accompaniment, and a
single affective manner throughout.
% XXX  on \quoted{Italianization}?

\begin{musicexample}
    \caption{Jerónimo de Carrión, \wtitle{Si los sentidos queja forman del pan
    divino} (\sig{E-SE}{28/25})} % measure range?

    \label{music:Carrion-Si_los_sentidos}
    \includeMusic{Carrion-Si_los_sentidos}
\end{musicexample}

The dialogue and rivalry of the poetic text is embodied now not through
polychoral effects but through motivic exchanges between voice and
accompaniment.
Instead of metrical contrasts from one section to the next, Carrión creates
rhythmic contrasts between simultaneous voices.
Carrión dramatizes \emph{queja} (\measures{3--4}) with a metrical disagreement
between the two voices (normal ternary motion versus the voice's sesquialtera).
The descending pattern of leaps for \emph{porque lo que ellos sienten} perhaps
evokes the sense in tumult, and it creates a certain amount of rhythmic
confusion as it moves between voices.
Carrión creates a climax through a series of phrases in close imitation between
soloist and accompaniment in \measures{18--32} that leads the singer to the top
of his register in \measures{35--37}.
The upward leaps in the last line on \emph{no se den} (\measures{41--42})
contrast with the downward leaping motive of the opening gesture on
\emph{sentidos} (\measure{1}).
The solo setting allows the singer and instrumentalists much more expressive
freedom than is possible in a large-ensemble texture, and likewise the
individual voice of the singer might encourage a more subjective response in
individual listeners.


\subsection{Ranking the Senses in Early Modern Philosophy}

The treatment of the senses in this villancico family reflects a common
physiological model of sensation and perception, as educated Spaniards would
have learned from the kind of scientific and theological treatises available to
them in seminary and convent libraries in Spain and Mexico.%
\begin{Footnote}
    Some of these libraries are still preserved in their original location, such
    as the Monastery of the Escorial in Spain; others are now kept in the state
    libraries or form the core of smaller rare-book collections like those in
    Puebla (see \cref{tab:puebla-compendia}).
\end{Footnote}
% XXX  source on theological/philosophical education in Spanish Empire, libraries
Spanish philosophers taught that vision, not hearing, was the first and highest
of the five exterior senses; but hearing superseded the other senses in matters
of faith.
A typical example is the 1557 natural-philosophy textbook \wtitle{Phisica,
Speculatio} by an Augustinian friar in New Spain, Alphonsus à Veracruce.%
    \Autocite{Veracruce:Phisica}
Veracruce summarizes the traditional Catholic teaching, which drew on Aristotle
as interpreted by Thomas Aquinas.
% XXX   cite Aristotle early modern
Veracruce's Latin treatise accords with the Dominican friar Luis de Granada's
widely read Spanish \wtitle{Introduction to the Creed} of 1583.%
    \Autocites{LuisdeGranada:Simbolo}{LuisdeGranada-Balcells:SimboloPtI}

As Fray Luis explains this widespread understanding of perception, the five
exterior senses were linked to a set of interior senses or faculties, including
the affective faculty, in which the sensory stimuli interacted with the balance
of bodily humors (\cref{tab:senses-fray-luis}).%
    \Autocite
    [\range{ch}{27--35}, \pagenums{439--494}]
    {LuisdeGranada-Balcells:SimboloPtI}
The five exterior senses mediated between the outside world and the interior
senses by means of the ethereal \emph{spiritus animales}, which were something
like invisible beams of light that flowed through the nerves.  
The cerebrum housed the internal faculties, which \quoted{made sense} of what
the external senses told them---first the \quoted{common sense}, a kind of
reception area where the exterior senses met the interior faculties; and next
the imagination, the cogitative faculty, and memory.
All of these exterior and interior senses were part of the \emph{ánima
sensitiva}, the sensing, reasoning soul.
In addition to these senses the \emph{ánima sensitiva} possessed an affective
faculty, in which the balance of humors in the body interacted with the interior
and exterior senses to produce different \quoted{passions} or \quoted{affects}
(Fray Luis uses \emph{pasiones} and \emph{afectos} interchangeably).
Based on a fundamental dichotomy like magnetism between attraction and
repulsion, this \quoted{concupiscible} part of the soul experienced three
primary pairs of passions: love and hate, desire and fear, joy and sadness.
% XXX   secondary on physiology of sensation

\begin{table}
    \caption{The senses and faculties of the sensible soul (\emph{ánima
    sensitiva}), according to Fray Luis de Granada}
    
    \label{tab:senses-fray-luis}
    \includeTable{senses-fray-luis}
\end{table}

The act of sensation, then, involved the entire body and soul, in a holistic
model that Descartes would later challenge.  
The external senses differed, though, in how they connected the external world
to the internal faculties and passions.
The hierarchy of the senses was determined by the degree of mediation between
the object of sensation and the person sensing.
The most base sense was taste, because the person actually had to physically
consume the object of sensation.
The most powerful sense was sight, since it enabled a person to perceive objects
a great distance away without any direct contact.

Hearing stood out from the other senses because for it alone, the object of
perception was not identical with the thing sensed.
As Calderón's character Hearing says, \quoted{Sight sees, without doubting/ what
she sees; Smell smells/ what he smells; Touch touches/ what he touches, and
Taste tastes/ what he tastes, since the object/ is proximate \add{immediate} to
the action}.%
    \Autocite[\poemlines{577--582}]{Calderon:Retiro}
But Hearing hears a person's voice, not the person directly, as Calderón's text
continues: \quoted{But what Hearing hears/ is only a fleeting echo,/ born of a
distant voice/ without a known object}.%
    \Autocite[\poemlines{583--586}]{Calderon:Retiro}
While this feature of hearing may have made it \quoted{easily deceived}, it also
gave this sense a unique capability in spiritual matters, where the object of
perception was not immediately sensible at all.

The poetic contests of the senses thus rearrange the traditional scholastic
hierarchy by putting Hearing at the end for a dramatic climax.
Sight comes first, but Hearing, the underdog competitor, triumphs at the
last.
\begin{Footnote}
    Irízar alone (perhaps working from an earlier version of the text later
    attributed to Sánchez), puts Hearing in its traditional position, though
    this goes against the spirit of the poetic text.  
    Perhaps he was influenced by a scholastic training to \quoted{correct} the
    philosophical order.
\end{Footnote}
In the Sánchez villancico, each of the coplas highlights the failure of one of
the senses to rightly perceive the sacrament.
For example, the eyes \quoted{do not look at what they see}, and the Eucharist
reduces Sight to \quoted{blindness} (copla 1).
The \quoted{colors} and \quoted{rays of light} through which Sight normally
operates are \quoted{hidden} \quoted{beneath transparent veils} and
\quoted{transformed} so that \quoted{God Incarnate is not seen} (copla 2).
Similarly, Touch may make contact with the host, but not with the
\quoted{mystery} hidden within (copla 3).
Taste and Smell (coplas 5 and 6) are similarly hindered by their ability only to
perceive material accidents and not spiritual substance.

The Sánchez villancico presents hearing in the last copla, through the conceit
of music.
The senses are \quoted{five instruments} like a musical consort, which must be
\quoted{tempered} by faith.
Without Faith, sight is actually blind, and touch, taste, and smell are fooled;
but when properly attuned by Faith, the senses can be harmonized into a pleasing
concord.
Here Faith is not the object of sensation, but the subject, who delights in
hearing the music of properly tuned senses.

While the poem speaks of Faith listening to the music of the senses, the musical
settings allow humans to listen in as well.
The musical discourse adds its own layer of meaning to the poetic discourse,
both by helping listeners imagine the contest of the senses (each setting in its
own way), and by turning the poem into an object of sensory perception.
The similarities between the settings of \wtitle{Si los sentidos} by Irízar and
Carrión demonstrate the persistence of concerns about the hearing's role in
faith.
Meanwhile the differences between versions reflect different styles not only of
composition but of devotional practice in public and private settings.

Irízar and Carrión take a verbal discourse on sensation and faith, in which
music is the paradigm of something that pleases the ear, and bring it to life
through actual music.
Thus the pieces seem designed to teach listeners how to hear music even as they
are listening---as the catechism advises, they accommodate hearing while
training it.


\subsection{Sensory Confusion}

While the \wtitle{Si los sentidos} villancicos may not project as much
uncertainty about sensation and faith as does Calderón's \wtitle{El nuevo
palacio del Retiro}, they still emphasize the need for all the senses to submit
to faith, which means that listeners should not trust their senses alone.
Some villancico poets and composers go further than stating that senses can be
deceitful; they use paradox to deliberately confuse the senses for pious
purposes.
We have already seen in \cref{ch:intro} that many villancicos feature
auditory \quoted{special effects} like echoes, voices imitating instruments, and
voices imitating birdsong.
Such pieces might be compared to the contemporary rise of \emph{trompe l'oeil}
effects in visual art, like the illusion of the heavens opening in the
\emph{Transparente} of Toledo Cathedral (by Narciso Tomé, 1732) or the false
domes painted by Andrea Pozzo and his students on Jesuit church ceilings from
Rome to China.%
    \Autocites
    [\sv{Tomé, Narciso}]{GroveArt}
    [110]{Bailey:Art}

Villancicos with \quoted{synesthetic} topics mismatch the senses in the spirit
of paradox and enigma.%
\begin{Footnote}
    \Autocite{DoetschKraus:Sinestesia} explores connections between poetic
    \quoted{synthesis of the senses} in Spanish verse and the actual
    neurological phenomenon of synesthesia.  
    For example, sight and hearing are the principal objects in the anonymous
    Marian fragment \wtitle{Porque cuando las voces puedan pintarla} (If voices
    could only paint her, \sig{E-Mn}{M3881/44}).
\end{Footnote}
Cristóbal Galán juxtaposes hearing and vision in a villancico for the Conception
of Mary.%
\begin{Footnote}
    \sig{D-Mbs}{Mus. ms. 2893}, edition in \autocite[567--568]{CaberoPueyo:PhD}.
\end{Footnote}
Galán was master of the Royal Chapel from 1680 to 1684, and to judge from copies
of his works in multiple archives, his works were performed all across the
empire and likely served as models for provincial composers who wanted to stay
current with trends in Madrid.
He had actually preceded Irízar as chapelmaster at Segovia Cathedral from 1664
to 1667.%
    \Autocite{Baron-Sage:GalanC}
% XXX  dates, cite sources on bio; complete works edition??  
Galán's text exhorts listeners to \quoted{hear the bird} and \quoted{see the
voice}.
The visual language in this villancico evokes the common iconography of the Holy
Spirit as a dove surrounded by golden rays, such as may be seen in the Monastery
of the Encarnación in Madrid, where the Royal Chapel frequently performed.%
\begin{Footnote}
    \Autocite[69--70, 81]{Sanz:GuiaDescalzasEncarnacion}.
    The image was painted on the ceiling of the monastery's Capilla del Cordero
    and when a new church building was added later, this image was incorporated
    as the central element atop the high altar.
\end{Footnote}
The poem makes \emph{equivocación} (confusion) of sight and hearing, which is
projected partly through irregular poetic meter
(\cref{poem:Oigan_todos_del_ave-Galan}).%
\begin{Footnote}
    The division into lines is speculative, but the syllable counts and line
    groupings in this arrangement could be scanned as  10, 6 10, 8 7 6, 6, 6 10,
    10.
\end{Footnote}
In his musical setting for eleven voices in three choirs, Galán creates
\emph{equivocación} through rhythm, notation, and texture.  
Galán juxtaposes the three voices of Chorus I against the other two choirs by
having Chorus I sing primarily in a normal triple-meter motion (with dotted
figures intensifying the ternary feeling), while the other two choirs interject
\emph{¡Oigan!} and \emph{¡Miran!} in sesquialtera rhythm.

\begin{poemexample}
    \caption{\wtitle{Oigan todos del ave}, from setting by Cristóbal Galán,
    estribillo}

    \label{poem:Oigan_todos_del_ave-Galan}
    \includePoem{Oigan_todos_del_ave-Galan}
\end{poemexample}

To notate these rhythms, Galán must use white notes for the regular ternary
motion in Chorus I, but blackened noteheads (mensural coloration) to indicate
the hemiola pattern in the other choirs.
But when each voice in Chorus I sings the synesthetic phrase \quoted{the light
is heard to shine}, Galán turns the lights out---on the page at least---by
giving each voice a passage in all black-note sesquialtera
(\cref{fig:Galan-Oigan-coloratio}).
They return to white notation again for the following phrase, \quoted{while the
voice is seen in purity}.
Any attentive listener could hear these juxtapositions and abrupt shifts in
rhythmic patterns, though only the musicians themselves would likely have
recognized the dark--light symbolism in the notation.%
    \Autocite[36]{Kendrick:Jeremiah}

\begin{figure}
    \caption{Galán, \wtitle{Oigan todos del ave} (\sig{D-Mbs}{Mus. ms. 2893}),
    Tiple I-2, end of estribillo: Ironic play of coloration}

    \label{fig:Galan-Oigan-coloratio}
    \includeFigure{Galan-Oigan-coloratio}
\end{figure}

For \emph{qué equivocación} (\quoted{what confusion!}) Galán creates a sudden
flurry of contrapuntal motion: the sudden outburst of polyphonic texture in the
midst of primarily homophonic polychoral dialogue could create an affective
sense of confusion.
The fugato here also paints the word \emph{equivocación} in a literal way---by
having \quoted{equal voices} sing a long ascending scalar figure in imitation
(starting with the Tiple II's stepwise ascent \pitch{A}{4}--\pitch{G}{5}).
As the estribillo continues, Galán increasingly mixes up the music for
\emph{Oigan todos del ave}, the sesquialtera interjections, and the contrapuntal
texture of \emph{qué equivocación}, between the various choirs.

Pieces like this describe and seek to incite a condition of sensory overload, an
ecstasy in which all the senses blend together in the effort to grasp something
that is beyond them.
Such pieces provide further evidence that Catholic religious arts could not be
reduced to the function of simply teaching doctrine; these pieces train the
senses by appealing to them directly.
They seem intended to provoke listeners to a higher form of sensation, to a holy
dismay and wonder that would lead to true, faithful perception.

%***********************************************************************
\section{Impaired Hearers, Incompetent Teachers: \quoted{Villancicos of the
Deaf}}

In the terms of Athanasius Kircher, such pieces would incite
\quoted{otherworldly affects and rapture of the mind}, and they would seem to
share Kircher's ideal that a listener who was \quoted{carried away beyond
himself} in this way would actually move beyond sheer sensory overload and come
to \quoted{experience the truth of what was said}.%
    \Autocite[\range{bk}{7}, \pagenum{550}]{Kircher:Musurgia}  
It seems reasonable to assume that those who paid repeatedly for the creation
and performance of this kind of villancico believed that this music had the
power to overwhelm the senses with poetic and musical ingenuity, and that they
saw this kind of \quoted{dazzlement} as contributing positively to the goals of
church and state (to borrow Olivier Messiaen's term from a later stage of
Catholic musical evangelism).
At the same time, though, we might wonder how the creators of this devotional
music responded to the other strain of post-Tridentine thought about music, more
in the ascetic line of leaders like John of the Cross, who were suspicious of
music's powers over the senses and expressed concern that such music would
produce no actual fruit, no increase of faith working through hope and charity
to contribute to building a virtuous society.
As it turns out, villancico poets and composers addressed this topic as well.
Several villancicos represent characters who are deaf or hard-of-hearing,
creating comic dialogues between these figures and a religious teacher who fails
to find a way to make them hear the Word.

The final section of this chapter presents two of these \emph{villancicos de
sordos}---villancicos of the deaf---which dramatize the limitations of hearing,
and poke fun at the difficulty some religious teachers faced in making faith
appeal to this sense.
Similar pieces from both sides of the Atlantic use hearing disability as a
symbol of spiritual deafness: the first is by Juan Gutiérrez de Padilla from
Puebla Cathedral in New Spain, and the other is by Matías Ruíz for the Royal
Chapel in Madrid.
Both pieces present mock catechism scenes with a friar and a \quoted{deaf} man.


\subsection{Laughing at the Deaf (Gutiérrez de Padilla, Puebla Cathedral)}

% XXX  Add measure numbers for both after making editions and musical examples
In 1651, only two years after the reforming bishop Juan de Palafox y Mendoza had
consecrated a new, but still unfinished, cathedral in Puebla de los Ángeles, the
cathedral's chapelmaster included a \quoted{villancico of the deaf} among the
villancicos for Christmas Matins.
In a piece that begins \emph{Óyeme, Toribio} and is labeled \emph{sordo} in the
partbooks, Juan Gutiérrez de Padilla created a comic dialogue between a
religious teacher and a \quoted{deaf} man named Toribio
(\cref{poem:Oyeme_Toribio-Padilla}).%
\begin{Footnote}
    \sig{MEX-Pc}{Leg. 1/2}; \Autocites{Stanford:Catalog}{Puebla:Microfilm}.
\end{Footnote}
This villancico is part of Juan Gutiérrez de Padilla's earliest surviving
complete cycle of Christmas villancicos, and the first extant musical
compositions that is known to have been performed in the new space.%
    \Autocites{Cashner:Cards}{Mauleon:PadillaPalafox}
Since Padilla was a member of the Congregation of the Oratory of Philip Neri,
whose building in Puebla was consecrated in 1651, these villancicos may have
been performed at the \emph{oratorio} as well.
% XXX  ref to other chapters, Padilla-Puebla lit, Oratorians  
In that venue, dramatic elements like costumes, gestures, and even choreography
might have been more likely that in the cathedral; and the group of listeners
likely included a broader mixture of people of different social status, who
would have listened with somewhat different expectations than those
participating in the Christmas Eve liturgy before the new high altar, or those
straining to listen in from surrounding spaces.
Both the cathedral and the oratorio were designed as spaces in which colonial
residents could hear the Word through Scripture, liturgy, preaching, and music.
Padilla's comic \quoted{deaf} villancico explicitly plays with the challenges of
hearing the faith and making it heard.

\begin{poemexample}
    \caption{\wtitle{Óyeme, Toribio (El sordo)}, from setting by Juan Gutiérrez
    de Padilla, Puebla, 1651 (\sig{MEX-Pc}{Leg. 1/2}), excerpt}

    \label{poem:Oyeme_Toribio-Padilla}
    \includePoem{Oyeme_Toribio-Padilla}
\end{poemexample}

Padilla sets the anonymous poem as a dialogue between two soloists, each
accompanied by a bass reed instrument (Padilla calls the texture of soloist and
accompanying instrument a \emph{dúo con bajón}); followed by a five-voice choral
section.
Though two key partbooks are missing, including the Tenor I part who played the
deaf man, the dialogue can be reconstructed because the lyrics of the deaf man's
part were written in the surviving bass part.%
\begin{Footnote}
    In Padilla's manuscripts, \emph{dúo} means a single voice at a time with a
    single instrumental bass line, in this case specified to be played on
    \emph{bajón}.  
    The surviving parts for this section are Altus I and Bassus II; the Tenor I
    and Bassus I partbooks are lost.
    The Altus I includes the friar's part, which was probably accompanied by the
    Bassus I on \emph{bajón} (lost).  
    The deaf man was played by the Tenor I (lost), accompanied by Bassus II on
    \emph{bajón}.
    Though the Tenor I solo music is missing, the lyrics for the vocal part are
    written in above the accompaniment in the Bassus II part.  
    Thus we have the music for the friar without its accompaniment, and the
    accompaniment for the deaf man and most of the lyrics, but not the deaf
    man's music.
\end{Footnote}
The lyrics are preserved in one of the few imprints of villancico poetry from
Puebla to match up with surviving music, from Christmas 1651.%
\begin{Footnote}
    Thanks to Gustavo Mauleón for making available scans of a few pages of this
    imprint, held in an anonymous private collection in Puebla.
\end{Footnote}

Playing with a conventional villancico type of a dialogue between a \emph{docto}
and a \emph{simple}---a learned man and a simpleton---Padilla's villancico
stages a parody of catechism instruction.
% XXX   examples from catalog of this type, or from Leon Marchante  
The friar's attempts to communicate with the \quoted{deaf} man fail, and this
prompts the chorus to warn the congregation against spiritual deafness.

Padilla dramatizes the two characters' unsuccessful attempts to communicate
through disjunctions of rhythm and mode  (\cref{music:Padilla-Sordo-intro}).
Rhythmically, Padilla gives the friar relatively refined and elegant musical
speech, while the deaf man's speech is halting and clumsy, as in the offbeat
figure on \emph{¿Eh? ¿eh? que no te oigo} (Huh? huh? I can't hear you).
The two men interrupt each other and cannot follow each other's train of
thought.

\begin{musicexample}
    \caption{Juan Gutiérrez de Padilla, \wtitle{Óyeme, Toribio (El sordo)}
    (\sig{MEX-Pc}{Leg. 1/2}), introducción, extant parts (missing Tenor I,
    Bassus I)}

    \label{music:Padilla-Sordo-intro}
    \includeMusic{Padilla-Sordo-intro}
\end{musicexample}

Padilla illustrates the men's disagreement by having them fail to concur on
where to cadence.
Given the one-flat \emph{cantus mollis} signature, the cadence points
articulated by the extant bass part, and the final on F, Padilla probably would
have categorized this piece as mode 11 or 12.%
    \Autocites
    [873--882]{Cerone:Melopeo} 
    {Judd:RenaissanceModalTheory}{Barnett:TonalOrganization17C}  
Thus the friar appropriately sings an opening phrase that surely would have
cadenced on F; but the deaf man responds with a phrase that cadences on C.
At one point the friar moves to a cadence on A, but the deaf man, responding
that he can't hear out of that ear, cadences on D. 
The friar says he will try the other ear; as though trying to meet Toribio on
his level, he moves to cadence on the same note.
But no sooner has the friar moved to D, than Toribio, saying, \quoted{Out of
that ear I hear even less!} moves to a cadence on C.  
This pushes the friar over the edge.
He bursts out, \quoted{You are a sheer idiot!}---mimicking the deaf man's
halting short-long rhythms, and returning to his own final of F.

Next the Altus I singer, who has been representing the friar, addresses the
congregation like a preacher: \quoted{The laughter of the dawn will turn to
sobs}, he says---referring to the Virgin Mary---when, having given birth to the
Word Incarnate, \quoted{her eyes see deaf men}.  
Padilla sets the final phrase about deaf men with ten blackened notes in
mensural coloration.
Where Galán used musical coloration as a symbol of blindness, here Padilla would
seem to use the same symbol to point to a deficit in hearing.
When the rest of Chorus I joins in for the responsión, their repeated dotted
rhythm suggests vivacious laughter and comic offbeat sobs on \emph{sollozos}
(\cref{music:Padilla-Sordo-responsion}).
The descending figure is passed through all the voices in imitation, leading to
a harmonic form of descent when the Tiple I adds E flat---shifting further away
from the \quoted{natural} into the \quoted{weak} realm of flats.
% XXX  [@chafe?]  
The heavy syncopation in each voice creates rhythmic confusion that is not
sorted out until the final cadential flourish on F, validating the friar's
initial choice of mode.

\begin{musicexample}
    \caption{Gutiérrez de Padilla, \wtitle{Óyeme, Toribio (El sordo)},
    responsión a 5, extant voices}

    \label{music:Padilla-Sordo-responsion}
    \includeMusic{Padilla-Sordo-responsion}
\end{musicexample}

Padilla uses his characteristic mixture of sophisticated musical technique,
high-minded theology, and low caricature to exploit the deaf for the amusement
of the hearing, making them into a warning against spiritual deafness.
The \quoted{deaf men} that will make Mary weep are all people whose ears have
been stopped by sin and cannot hear the divine Word of Christ with faith.
This concept recalls the definition of \emph{sordo} in the 1611 Spanish
dictionary of Sebastián de Covarrubias, who defines the \emph{sordo}, not as one
who is \emph{unable} to hear, but rather as \quoted{he who does not hear}.
He adds, \quoted{There is no worse kind of deaf man than the one who is
unwilling to hear}.%
    \Autocite
    [\sv{sordo}: \quoted{SORDO, Lat. surdus, el que no oye. No ay peor sordo que
    le que no quiere oyr}.] 
    {Covarrubias:Tesoro}


\subsection{Learning from the Deaf (Ruiz, Madrid Royal Chapel)}

For the last example we return to the piece discussed at the outset of the
chapter, the \wtitle{Villancico de los sordos} by Matías Ruiz.
This piece invites more sympathy for the deaf and extends its parody to the
catechist as well
(\crefrange{poem:Pues_la_fiesta-Ruiz-1}{poem:Pues_la_fiesta-Ruiz-2}).%
    \footnote{\sig{E-E}{Mús. 83-12}.}
The poetry imprint survives from the first performance by the Royal Chapel at
Madrid's Convent of the Incarnation in 1671, where Ruiz was chapelmaster.%
   \Autocites{1671-Madrid-Enc-Nav}{Grove:Ruiz}
% XXX sources on Ruiz, Encarnacion: who was audience at Encarnacion?
Here the \emph{sordo} is a hard-of-hearing man, \quoted{very learned in humane
letters}---a doddering old university professor, or perhaps a street sage.
The piece mocks his impairment while contrasting true faith with the book
learning of this would-be humanistic scholar.
But the biggest laughs come at the expense of the friar, as the deaf man
mishears his rote teaching formulas in increasingly absurd ways.

\begin{poemexample}
    \caption{\wtitle{Pues la fiesta del Niño es (Villancico de los sordos)},
    from setting by Matías Ruiz, Madrid, 1671 (\sig{E-E}{Mús. 83-12},
    \sig{E-Mn}{R/34989/1}), first portion}

    \label{poem:Pues_la_fiesta-Ruiz-1}
    \includePoem{Pues_la_fiesta-Ruiz-1}
\end{poemexample}

\begin{poemexample}
    \caption{\wtitle{Pues la fiesta del Niño es (Villancico de los sordos)},
    conclusion}

    \label{poem:Pues_la_fiesta-Ruiz-2}
    \includePoem{Pues_la_fiesta-Ruiz-2}
\end{poemexample}

The piece begins with soloist and chorus gleefully crying \quoted{On with the
deaf man!} rather like a bunch of high-school bullies, telling everyone to speak
up so he can hear.
When the catechist and the \emph{sordo} enter, Ruiz illustrates their inability
to understand each other with contrasting styles.
The deaf man's musical speech is abrupt, uncouth, and loud, fitting with the
friar's mockery of the deaf man's unmodulated voice
(\cref{music:Ruiz-Sordos-dialogue}).
The deaf man bursts on the scene with a scale from the top of his register to
the bottom (\pitch{F}{4} to \pitch{G}{3}).
The descent across the registers or \emph{passaggi} of the voice would encourage
the singer to bawl the phrases in a coarse tone of voice, so that ironically, a
character who cannot hear is marked for the audience by the sound he makes.

\begin{musicexample}
    \caption{Matías Ruiz, \wtitle{Pues la fiesta del niño es (Villancico de los
    sordos)} (\sig{E-E}{Mús. 83-12}), estribillo} 

    \label{music:Ruiz-Sordos-dialogue}
    \includeMusic{Ruiz-Sordos-dialogue}
\end{musicexample}

Unlike in Padilla's villancico, here the deaf man has a lesson of his own to
teach.
He may not be able to hear well but he has come with love to adore the
Christ-child.
Acting as a kind of holy fool, and echoing Covarrubias's definition of deafness,
he reminds everyone within the sound of his voice that the truly deaf are
\quoted{those who neither listen nor understand the sound}.

In the parodied catechism lesson presented in the coplas, the friar quizzes his
pupil on the same key doctrines of Christmas emphasized by contemporary
theological writers both scholastic and pastoral.%
    \Autocite[\range{ch}{3}]{Cashner:PhD} % page num
Tell, \emph{sordo}, he asks, how did God fulfill his word to the prophet-king
David?  What motivated Christ to become incarnate? But the deaf man mishears
every statement: he mistakes \emph{sordo} for \emph{gordo} (chubby), and hears
\emph{profeta} (prophet) as \emph{estafeta} (mailman).

His supposed learning in the humanities leads only to confusion.
When the friar says \emph{el portal es nuestro alivio} (the stable is our
remedy), the deaf man thinks he is citing \emph{Tito Libio}.
Since the catechist has just been lauding the \emph{bailes} (dances) of
Christmas, the humanist is puzzled: he has read the Classical historian Livy, he
says, but Livy doesn't say anything about \emph{frailes} (friars).

Hearing that the child Jesus is shivering with cold, the deaf man suggests he
drink hot chocolate.
The friar reassures him, \quoted{the Queen}---the Virgin Mary---is keeping the
child bundled, such that he glows (\emph{arde}) with warmth.
The deaf man now seems to feel that at last he has figured out what they are
talking about, and sums up with satisfaction, \emph{Esta es, por la mañana y
tarde, la Reina de las bebidas} (Chocolate is, morning and evening, the Queen of
beverages).

We can imagine that Ruiz's deaf man would invite the sympathies of listeners.
He is an earthy, common character, keenly aware of the chill, as many older men
are.
The deaf man even explicitly asks not to be mocked: since he cannot hear the
nine choirs of Christmas angels, he asks them to \quoted{sing out
loudly}---\quoted{as long as they don't say anything bad about me}.
The deaf man's bumbling but endearing statements contrast strikingly with the
friar's abstract theology and clichéd poetic language.
The characters represent contrasting types of learning: the churchman who
repeats the same teaching points in every catechism class, versus an ersatz
humanist who has read Livy and perhaps Ovid but may not understand them at all.
Ruiz's hard-of-hearing humanities scholar demonstrates a central tenet of
Reformation-era Catholicism: that Classical learning alone is not enough to
understand Christianity.%
    \Autocite
    [206: \quoted{Erasmus perceived in the paganistic trends of the Renaissance
    a greater threat to religion than the theological squabbles he was so
    reluctant to participate in}.] 
    {Erasmus:Dolan}
What the man lacks in knowledge, though, he makes up in heartfelt devotion,
doting on the baby Jesus.

On the balance, though, the piece still uses deafness as a negative symbol.
When the chorus sings that \quoted{the deaf are those who neither listen nor
understand the sound}, they use the term \emph{son}, which the dictionary of
Covarrubias defines as a kind of dance.
Indeed, this villancico features a distinctive harmonic and rhythmic pattern
with alternating ternary and sesquialtera groupings, which are most clear on the
phrase \emph{los que no escuchan ni entienden el son}
(\cref{music:Ruiz-Sordos-son}).
This pattern bears a close resemblance to dance forms known as \emph{son}
today---most obviously, the Mexican \emph{huarache} familiar from the song
\quoted{America} in Leonard Bernstein's \emph{West Side Story}.%
    \Autocites
    [\sv{huarache}]{Grove}
    {Wells:WestSideStory}
Just as in the contests of the senses, then, music is the paradigm of faithful
hearing; those who can't pick up the tune, spiritually speaking, are the truly
deaf.

\begin{musicexample}
    \caption{Ruiz, \wtitle{Villancico de los sordos}, conclusion of estribillo,
    \measures{76--83}: Possible evocation of \emph{son} song or dance style}

    \label{music:Ruiz-Sordos-son}
    \includeMusic{Ruiz-Sordos-son}
\end{musicexample}

Where did this theology of hearing leave actual people with hearing
disabilities? 
At the same time as these caricatures of the deaf were performed, the Spanish
churchman Juan Pablo Bonet was laying the foundations of modern deaf education.%
\begin{Footnote}
    Juan Pablo Bonet, \wtitle{Reduction de las letras y Arte para enseñar a
    ablar a los mudos} (Madrid, 1620), cited and discussed in
    \autocite{Plann:DeafEducationSpain}.  
% XXX  more 
\end{Footnote}
These villancicos, though, do not offer much hope for actual people with
disabilities.
While Calderón's Judaism heard Faith without faith, the deaf men in villancicos
cannot even hear Faith to begin with.

Whether these pieces accomplished anything more than amusing hearers by
reinforcing their prejudices is a question that haunts the whole repertoire of
Spanish devotional music---and one that should give modern performers pause
before reviving some villancicos.%
    \Autocite[7]{Cashner:WLSCM32}
This problem is especially vexing in the \quoted{ethnic} villancicos, still
frequently performed by early-music groups today, which represent Africans and
other non-Castilian groups through caricatured deformations of speech and
music.%
    \Autocites{Baker:EthnicVC}{Baker:PerformancePostColonial}
These pieces appear to welcome people from the margins of society to Christ's
stable, or around his altar, and thereby raise their status; but the stereotyped
representation actually reinforces their position at the bottom of the social
hierarchy.

%***********************************************************************
\section{Failures of Faithful Hearing}

These villancicos depict failures of faithful hearing as a call for disciplined
listening.
The villancicos of the deaf fit in with a characteristic tendency of Roman
Catholics after Trent to critique the poor level of theological knowledge among
the lay people and the low quality of teaching among the clergy.%
    \Autocite[56--57]{Kamen:EarlyModernSociety}
In the vernacular catechism discussed at the beginning of this chapter, the
Augustinian friar Antonio de Azevedo describes real-life scenes of failed
catechesis: 
\begin{quoting}
    Some will say that the doctrine of the gospel has already been taught
    everywhere or almost everywhere (I am speaking of our Spain), and we
    concede; but there are so many parts that so badly lack anyone who could
    teach matters of faith, that indeed it is a shame to see it happen in many
    parts of Spain, and particularly in the mountains, where there are many so
    unlettered \addorig{bozales} in the matters of faith, that if you would ask
    them, how many are the persons of the Holy Trinity, some would say that they
    are seven, and others, fifteen; and others say about twenty---of this I am a
    good witness.
    And a principal friar of my order, I've heard that once he was asking a
    woman how many \add{persons in the Trinity} there were, and she said,
    \quoted{Fifteen}. 
    And he said, \quoted{\emph{Ay}, is that really your answer?} 
    And then she wanted to correct herself, and she said, \quoted{\emph{Ay
    Señor}, I think I was wrong---I'll say there are five hundred}.%
        \Autocite
        [26: \quoted{Diran, o que ya ay dotrina del Euangelio en todas partes, o
        casi todas (hablo de nuestra España) concedamoslo: Pero ay tanta falta
        en muchas, de quien enseñe las cosas de la fe; que cierto que es
        lastima, verlo que en muchas partes de España, y particularmente en
        montañas passa: a do estan muchos tan boçales en las cosas de la fe, que
        si les preguntays, quantas son las personas de la Santissima Trinidad,
        vnos dizen que son siete otros que quinze; y otros veynte desatinos, de
        los quales yo soy buen testigo.  Y a un frayle principal de mi orden le
        oy, que preguntando el a vna muger, quantas eran, que dixo ella que
        quinze, y diziendole el ay, y esso aueys de dezir?  y ella se quiso
        emendar, y dixo ay Señor, digo mi culpa, digo que son quinientas}.]
        {Azevedo:Catecismo}
\end{quoting}
Azevedo sees no humor in this lack of religious knowledge; and he faults not the
illiterate laypeople but the friars and clergy who have failed to teach the
basics of faith in a plain way, as Azevedo himself endeavors to do in his book:
\begin{quoting}
    It is a shame to see the ignorance that there is in many, in things of such
    importance \Dots{}.
    Because even though the religious orders and those who preach do declare the
    gospel, they do not explain the ABCs \addorig{b, a, ba} of Christianity;
    they do not want to deal with giving milk because this is the task of
    mothers, those lordly Curates or Orators, who are responsible for this task,
    and what I have described is their fault.%
        \Autocite
        [27: \quoted{Es lastima ver la ignorancia que ay en muchos, en cosas de
        tanta importancia: Y preguntados algunos qual de las tres personas
        encarno, el vno dize, que el Padre otros que el Espiritu Santo: y en muy
        buenos pueblos lo he oydo yo, hartas vezes con mis oydos; porque dado
        los religiosos y los que predican declaren el Euangelio, no tratan del
        b,a ba de cristiandad, no tratan de dar leche porque esse es officio de
        madres, de los señores Curas, o Retores, a cuyo cargo esta esso; y cuya
        culpa es lo dicho}.] 
        {Azevedo:Catecismo}
\end{quoting}
The friars of Padilla's and Ruiz's villancicos seem to fit with Azevedo's
description of \quoted{lordly orators} who delight in lofty language and fail to
adapt their disciples' capacity, rather than motherly teachers who spell out the
fundamentals of Christian faith.
Both pieces illustrate the difference in language and understanding through
appropriate contrasts of musical style.

What does it mean, then, that this music invites hearers to laugh at the
Church's incompetent teachers? 
After all, the friars are caricatured just as much as the deaf men are.
Padilla's piece was performed at the epicenter of religious reform in the New
World, and Ruíz's piece may have been heard by the royal defender of the faith
Charles II.  % really? 
The function of villancicos must be more complex than the imposition of dogma.
Few villancicos of the seventeenth century would satisfy Azevedo's call to teach
the \quoted{\emph{b, a, ba} of Christianity}.  
As the severity of the years after Trent gave way to Baroque aesthetics that
valued more elaborate forms of expression, even the comic villancicos involved
learned plays of language and music, like the Classical references in Ruiz's
poetic text, or the play on modal cadences and black notation of Padilla's
music.
Composers of villancicos in the seventeenth century seem to have followed an
ideal closer to Kircher's Jesuit ideal of affectively powerful sacred music
attuned to the varying desires of a broad audience.
The depictions of imperfect hearing in the villancicos of the deaf, in fact,
depended on the attention of listeners with well-trained ears.


\subsection{\quoted{Make a Hedge around Your Ears}}

To understand the role of villancicos in the dynamics of hearing and faith,
then, we must consider these performative texts as more than just one-way
transmissions of religious teaching from the Church to listening worshippers.
The creators of these pieces seem to expect listeners to be active, attentive,
and intelligent.
Indeed, Catholic listeners in this period had to pay attention, because as
multiple examples have demonstrated, hearing was \quoted{the sense most easily
deceived} and Catholics were not expected to believe everything they heard.
Hearing with faith meant questioning anything that contradicted true faith and
accepting the true faith even when it contradicted the senses.

It is clear from a rare listener's account of a villancico that not everyone
felt adequately trained to appreciate this music.
The chronicler of a Zaragoza festival in 1724 praises the performance by the
musicians of the city's two principal churches, El Pilar and La Seo, leaning
heavily on a stock vocabulary for musical encomium: 
\begin{quoting}
    During the whole event, the senses were enchanted with an imponderable spell
    by the sweet, solemn, and sonorous harmony, with which the two chapels,
    united to this end by the chapter, officiated the Mass, and since in the
    short time that was given to their Masters for composition, it was necessary
    that they employed, in competition, the most exquisite skill
    \addorig{primor} of the art, which should be credited mutually to the
    virtuosity \addorig{destreza} of the voices and the well-adjusted management
    of the instruments.  Each Master set to music one of the following
    villancicos \add{included in the chronicle}, which were part of the design
    of the chapter's order.
\end{quoting}
\quoted{The two villancicos were sung nobly, and were heard with pleasure}, he
writes, but, truth be told, \quoted{it would have been even better, if every ear
was intelligent in points of consonance}.%
\begin{Footnote}
    \Autocite
    [97: \quoted{Todo el rato que durò la Funcion, tuvo en imponderable embeleso
    à los sentidos, la dulce, grave, y sonora harmonìa, con que oficiaron la
    Missa las dos Capillas, unidas à este fin por el Cabildo, y prevenidas de
    que en el corto plazo que se diò à sus Maestros para la composicion, havia
    de emplearse, à competencia, el primor mas exquisito del Arte, que
    acreditasse mutuamente la destreza de las vozes, con el ajustado manejo de
    los instrumentos.  
    Cada Maestro puso en musica uno de los Villancicos siguientes, que fueron
    parte del desempeño de la orden del Cabildo: los dos se cantaron con gala, y
    se oyeron con gusto; pero aun huviera sido mayor, si todos los oìdos fuessen
    inteligentes, en puntos de consonancia}]
    {Zaragoza1724Relacion}.
    The language used to describe music reiterates a set of key vocabulary that
    appears in most of the texts of villancicos about music in this
    study---\emph{dulce, grave, y sonora armonía}, \emph{primor},
    \emph{destreza}, \emph{puntos de consonancia}.
\end{Footnote}
Moreover, in the anxious theological climate of early modern Spain, not even an
aural-skills course would meet the more pressing demand for spiritual discipline
in listening.
A moral emblem from 1610 by dictionary author Sebastián de Covarrubias shows two
fragile human ears, protected from the four winds by a crown of thorns
(\cref{fig:Covarrubias-Emblemas-202-ears-thorns}).
Its Latin motto, from \scripture{Ecclesiasticus}{28:28}, reads, \quoted{Make a
hedge around your ears with thorns}.%
    \Autocite
    [202: \quoted{SEPI AVRES TVAS SPINIS}.]
    {Covarrubias:Emblemas}
In the double explanation of the emblem, first in vernacular poetry, then in
prose, readers who contemplated this image and its possible interpretations were
advised to shield their ears \quoted{from hearing flattery, gossip, lies, and
false doctrines; and so that these things will not reach our ears we must put a
strong fence around them, and protect it with thorns}.%
    \Autocite
    [203: \quoted{Esse mesmo peligro tienen las orejas, sino las apartamos de
    oir lisonjas, murmuraciones, mentiras, y falsas dotrinas: y para que no
    lleguen a nuestros oydos deuemos hecharles van fuerta cerca, y vardarlas con
    espinas}.] 
    {Covarrubias:Emblemas}
\quoted{In this life, which is a battle}, he warns, \quoted{if you wish to keep
yourself safe, take refuge in Christ and his crown}.%
    \Autocite
    [202: \quoted{En esta vida, que es vna milicia,/ Si asegurar quisieres tu
    persona,/ Amparate de Christo, y su corona}.] 
    {Covarrubias:Emblemas}

\begin{figure}
    \caption{\quoted{Make a hedge around your ears with thorns}, from
    Covarrubias, \wtitle{Emblemas}, 202}

    \label{fig:Covarrubias-Emblemas-202-ears-thorns}
    \includeFigure{Covarrubias-Emblemas-202-ears-thorns}
\end{figure}

Spanish devotional music of the seventeenth century appealed to the ears of
diverse people at different levels; but it challenged all of them to
\quoted{temper} their own hearing, as the Segovia villancicos say, lest the
Church's musical proclamation of faith fall on \quoted{deaf ears}.
The fundamental Tridentine problem, of making faith appeal to hearing by both
accommodating the senses and training them, remained a challenge for Catholics,
both those who would teach through speech or song and those who would listen.


\endinput
