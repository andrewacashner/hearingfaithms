\chapter{Acknowledgments}
\label{ch:thanks}

I wish to express my gratitude to the many people and institutions who have
made this book possible.
For help and support throughout the project, I thank above all Robert Kendrick,
advisor for my PhD dissertation: Bob, I hung in there. 
Mary Frandsen guided me into the field and has continued to provide
level-headed support.

This research, which began with my PhD dissertation at the University of
Chicago, was funded by a Jacob K. Javits Fellowship from the United States
Department of Education, a Pre-Dissertation Research Fellowship from the
Center for European Studies at Columbia University, and a Dissertation
Completion Fellowship from the Mellon Foundation and the American Council of
Learned Societies.
Travel for archival research in Mexico and Spain was funded by grants from the
Center for Latin American Studies and the Department of Music at the
University of Chicago, and a Eugene K. Wolf Research Travel Grant from the
American Musicological Society.
Support for the final stages of publication was provided by an internal
fellowship from the Humanities Center at the University of Rochester.

I am grateful to numerous librarians, archivists, and institutional
caretakers at the following institutions:
\begin{itemize}
    \item In Chicago, the University of Chicago Regenstein Library and Special
        Collections (Scott Landvatter)
    \item The Newberry Library, Chicago
    \item The Capitular Archive of the Cathedral of Puebla de los Ángeles
        (P. Francisco Vázquez, rector, and the Illmo. Sr. Carlos
        Ordaz, \foreign{canónigo archivista})
    \item Biblioteca Palafoxiana, Puebla
    \item Biblioteca José María Lafragua, Universidad Autónoma de Puebla
    \item CENIDIM, the Mexican national center for music research, Mexico
        City
    \item Biblioteca de Catalunya, Barcelona
    \item Biblioteca Nacional, Madrid
    \item The Capitular Archive of Segovia Cathedral (P. Bonifacio Bartolomé)
    \item The monastery libary of San Lorenzo de El Escorial
    \item The Abbey of Our Lady of Montserrat (P. Daniel Codina)
    \item The Archdiocese of Girona, for providing digital scans from the
        parochial archive of the Church of Saints Peter and Paul, Canet de
        Mar, Barcelona province
    \item Bayerische Staatsbibliothek, Munich
    \item British Library, London
    \item Lilly Library of Indiana University, Bloomington
\end{itemize}

Personal thanks are due to Gustavo Mauleón Rodríguez for getting me in the
door of the Puebla archives and for sharing select sources from a private
collection in Puebla.
For hospitality, I thank Alfredo Amieva (Mexico City), Emilio Ros-Fabregas,
and María Gembero-Ustárroz, and Tess Knighton (Barcelona).

I am grateful to all the members of my dissertation committee, Anne Walters
Robertson, Martha Feldman, Frederick de Armas, María Gembero Ustárroz.
This project has benefitted from the interchange of ideas and sources with
colleagues in early modern Ibero-American studies, including (in alphabetical
order)
Ireri Chávez-Bárcenas,
Walter Clark,
Drew Davies,
Elizabeth Davis,
Cesar Favila,
Timothy Foster,
Bernardo Illari,
Paul Laird,
Dianne Lehmann Goldman,
Javier Marín López
Miguel Martínez, 
Jesús Ramos-Kittrell
José Rodríguez Garrido, 
Craig Russell,
John Swadley,
Martha Tenorio, 
Álvaro Torrente,
and 
Lorena Uribe Bracho.
Other scholars who shaped my thinking and skills include 
Gauvin Bailey, 
Philip Bohlman, 
Bruce Alan Brown,
Melvin Butler, 
William Christian, 
Cécile Fromont, 
Ryan Giles, 
Maxwell Johnson, 
Nathan Mitchell,
Clemens Risi,
David Yearsley, 
and
Lawrence Zbikowski.

Among my peers, Anita Damjanovic, James Nemiroff, and Ana Sánchez-Rojo helped
with Spanish; Drayton Benner, with Hebrew; and Miriam Tripaldi, with Latin.
Many other colleagues supported me through numerous challenges,
including 
Chelsea Burns,
Mary Channen Caldwell,
Lisa Cooper-Vest,
Erika Honish, 
Sarah Iker,
Abigail Fine,
John Romey,
August Sheehy, 
Martha Sprigge,
and
Mari Jo Velasco.


Thanks also to my colleagues in the University of Rochester
College Music Department,
Matt BaileyShea,
John Covach,
Cory Hunter,
Kim Kowalki,
Jennifer Kyker, 
and my chair, Honey Meconi.
I am grateful for feedback and support from the members of Grupo, the UR
working group on Latin American and Iberian culture, including
Molly Ball,
Beth Jorgensen,
Rachel O'Donnell.
Ryan Prendergast, 
and
Pablo Sierra Silva.
Thanks also to Dean Gloria Culver, Dean Jeffrey Runner, Joan Rubin, and
Jennifer Hadingham.

I am grateful to the many volunteers who created and maintain the free
software I used to create the book: the Vim editor, the \LaTeX{}
document-preparation system (based on the work of Donald Knuth and Leslie
Lamport), and the Lilypond music-notation system used to create all the music
examples.

Among many teachers of music who deserve thanks here I must single out
Charles Combopiano and Ralph Burkhart.
I thank David Schneider and the late Rusty Hollingsworth for teaching me
Spanish in high school, and Norma Veramallay and Mary Ann Thompson for
teaching me to write.
Derek Katz provided an inspiring first exposure to musicology.
I am grateful to the faith communities and ministers that have supported me:
Central United Methodist Church in my hometown of Richmond, IN; Intervarsity
Christian Fellowship at Lawrence University, Appleton, WI (Tim Webster);
Vineyard Church of Hyde Park, Chicago (Rand Tucker); St. Luke's Anglican
Church of La Crescenta, CA; and New Hope Free Methodist Church of Rochester,
NY (Scott Sittig).

I am grateful to my parents, Randall and Ann Cashner, and to my brother, Matt
Cashner (particularly for help with computing); and to my wife's family, Win
and Nancy Miller, Robert and Sarah Miller, Mary Walquist, Betty Walquist,
and Robert Walquist of cherished memory.

A special word of gratitude is due to Devin Burke, whom I met on the first day
of college in September 2019, and has been an unflagging source of
encouragement, humor, and insight across two decades---no matter how
burdensome the burden.

This book is dedicated with love to my wife Ann, the kindest, wisest, and most
beautiful person I know, for loving me and for making this all worthwhile.

I give thanks to God of Abraham, Isaac, and Jacob: 
to our Creator, the source of all good things; 
(in Spanish style) to the most holy baby Jesus, who came to save me; 
and to the Holy Spirit, for the power to change and the hope of new mercies
every morning.  

Having said that, I wish to stress that this book is not about my own faith.
I want this book to help all readers, religious or not, to gain a deeper
understanding of what seventeenth-century Catholics in the Spanish Empire
believed about music, and how they embodied those beliefs through music
itself.
I hope readers will engage sympathetically in the effort to listen to Spanish
villancicos within the theological framework of their first hearers.
For Christians like me, this effort poses a special challenge, because though
aspects of these historical musical-theological practices may seem quite
beautiful and ingenious, they are inseparable from the culture that formed
them, a culture in which Church leaders actively promulgated genocide and
slavery, and in which Christian believers used metaphors of musical harmony to
support a profoundly unjust society.
I pray that Christians today, especially those involved as I am with worship,
would not use music, as they did, to lie to ourselves and drown out the true
call of Christ.
That call continues to ring out in all its simplicity and power: \quoted{Love
the Lord your God with all your heart, mind, soul, and strength, and love your
neighbor as yourself.}
\quoted{If you have ears to hear, listen!}


\endinput
