\documentclass[12pt]{book}
\usepackage{newtxmath,newtxtext}
\usepackage[T1]{fontenc}
\usepackage[utf8]{inputenc}
\usepackage[endnotes]{semantic-markup}
\newcommand{\wtitle}{\worktitle}

\usepackage{biblatex-chicago}
\addbibresource{master.bib}

\usepackage[margin=1in]{geometry}
\usepackage{setspace}
\doublespacing
\usepackage[document]{ragged2e}
\frenchspacing
\raggedbottom
\setlength{\RaggedRightParindent}{0.5in}
\RenewDocumentEnvironment{quote}{}
    {\list{}{\leftmargin=0.25in \rightmargin=0.25in}%
    \item\relax}
    {\endlist}

\RequirePackage[defaultlines=2,all]{nowidow}

\makeatletter
\renewcommand{\enoteformat}{%
    \setlength{\parindent}{0.5in}%
    \@theenmark.\ }
\makeatother



\begin{document}

Written examples of teaching and preaching about Christ's birth demonstrate this
same devotional approach in their emphasis on wonder.
Even in the learned genre of a Latin Biblical commentary, Cornelius á Lapide
stresses that Christ's birth defies understanding: 
\quoted{The Word was made flesh, God was made man, the Son of God was made the
son of a Virgin.
This \Dots{} was of all God's works the greatest and best, such that it
stupefied and stupefies the angels and all the saints}.%
    \Autocite
    [50, on Mt 1:15: 
    \quoted{q. d. Verbum caro factum est, Deus factus est homo, Filius Dei
    factus est Filius Virginis.
    Hoc, \Dots{} fuit omnium Dei operum summum et maximum, ideoque illud stupuerunt
    et stupent Angeli, Sanctique omnes}.]
    {Lapide:Gospels19C}

In a model sermon for Christmas, Fray Luis de Granada draws on all his
rhetorical skills to exhort worshippers to marvel at the sight and sound of
Christ at his lowly birth:
\begin{quote}
    Come and see the Son of God, not in the bosom of the Father \add{Jn 1}, but
    in the arms of the Mother; not above choirs of angels, but among filthy
    animals; not seated at the right hand of the Majesty on high \add{Heb 1},
    but reclining in a stable for beasts; not thundering and casting lightning
    in Heaven, but crying and trembling from cold in a stable.
        \Autocite
        [37: \quoted{Venid á ver al Hijo de Dios, no en el seno del Padre, sino en
        los brazos de la Madre; no sobre los coros de los ángeles, sino entre
        viles animales; no asentado á la diestra de la Majestad en las alturas,
        sino reclinado en un pesebre de bestias; no tronando y relampagueando en
        el cielo, sino llorando y temblando de frio en un establo}.]
        {LuisdeGranada:Xmas}
\end{quote}
The result of contemplating this mystery, according to Fray Luis, is to be
\quoted{struck numb} with amazement:
\begin{quote}
    What theme, then, can cause any greater wonder?  Oh Lord our God, says Saint
    Cyprian, how admirable is your name in all the earth \add{Ps 8:1}! Truly you
    are the God who works miracles \add{Ps 77:14}.  \Dots{} I do not wonder at
    the figure of the world, nor the firmness of the earth \Dots{}; I marvel to
    see how the all-powerful one in the crib; I marvel to see how the word of
    God could take on flesh, and how, while God is a spiritual substance, he
    received the clothing of a body. \Dots{} In this mystery the greatness of
    the shock steals away all my senses, and with the prophet \add{Hb 3} it
    makes me cry out: Lord, I heard your words, and I feared: I considered your
    works, and I was struck numb.  With good reason, indeed, you are amazed,
    Prophet: for what thing could surprise anyone more, than that to which the
    Evangelist here refers in a few words, saying, \quoted{She gave birth to her
    only-begotten son, and she wrapped him in some rags, and laid him in a
    manger, because she did not find another place in that stable}?%
        \Autocite
        [38: \quoted{¿Pues qué cosa puede ser de mayor maravilla? 
        ¡Oh Señór Dios nuestro, dice Sant Cipriano, cuán admirable es vuestro
        nombre en toda la tierra!
        Verdaderamente vos sois Dios obrador de maravillas.
        Ya no me maravillo de la figura del mundo, ni de la firmeza de la tierra
        \Dots{}; sino maravíllome de ver al Todopoderoso en la cuna; maravíllome
        de ver cómo á la palabra de Dios se pudo pegar carne, y cómo siendo Dios
        substancia espiritual, recibió vestidura corporal.
        \Dots{} En este la grandeza del espanto roba todos mis sentidos, y con
        el Profeta me hace clamar: Señor, oi tus palabras, y temí: consideré tus
        obras, y quedé pasmado.
        Con mucha razon por cierto os espantais, Profeta; porque ¿qué cosa mas
        para espantar, que la que aquí en pocas palabras nos refiere el
        Evangelista, diciendo: Parió á su Unigénito, y envolvióle en unos
        pañales, y acostóle en un pesebre, porque no halló otro lugar en aquel
        establo?}]
        {LuisdeGranada:Xmas}
\end{quote}
Lapide's exegesis and Fray Luis's preaching guide the faithful to the right kind
of devotion at Christmas---to an affective response of awe at the mystery of
Christ's birth.

Spanish devotional music for Christmas seems designed primarily to cultivate
this same attitude of wonder.
Padilla's setting of \wtitle{Voces, las de la capilla} evokes wonder not only in
the words in the virtuoso composition and performance of the music as well.
The villancico aims less to instruct than to amaze. 
This supports Mary Gaylord's argument that the goal of elaborate Spanish poetry
is \quoted{to produce effects of astonishment and awe conveyed by the Latin term
\term{admiratio}}.%
\Autocite[227]{Gaylord:Poetry}
Indeed, Padilla's piece specifically asks listeners to imagine a song that is
\quoted{as much to hear as to admire \add{\term{admirar}},/ as much to admire as
to hear}.

The concept of \term{admiratio} is, in fact, central to the Christmas liturgy. 
It is encapsulated in the fourth Responsory of Matins, \wtitle{O magnum
mysterium et admirabile sacramentum}:
\begin{quote}
    \emph{Respond.} O great mystery and admirable sacrament, that the animals
    should see the newborn Lord, lying in the manger.
    Blessed Virgin, whose womb was worthy to bear the Lord Christ.\newline
    \emph{Versicle.} Greetings, Mary, full of grace: The Lord is with you.%
        \Autocite
        [175: \quoted{\emph{Respond.} O magnum mysterium, \& admirabile
        sacramentum, vt animalia vidêrent Dominum natum, iacentem in praesepio:
        Beata Virgo, cuius viscera meruerunt portare Dominum Christum.
        \emph{Versicle.} Ave Maria, gratia plena: Dominus tecum}.]
    {Catholic:Breviarium1631}
\end{quote}
In his sermon Fray Luis alludes to this Responsory in terms quite similar to
those in Padilla' villancico, when he cries out, \quoted{O venerable mystery,
more to be felt than to be spoken of; not to be explained with words but to be
adored with wonder in silence}.

This same Responsory was probably paired with Padilla's \wtitle{Voces, las de la
capilla} in the Puebla Cathedral liturgy on Christmas Eve 1657.
Based on the position of this villancico in the Puebla musical manuscripts, it
was most likely sung as the fourth villancico in the Matins cycle.
This means that in accord with a 1633 decree of the cathedral chapter the
villancico would have occupied the same liturgical time and space as the 
Responsory \wtitle{O magnum mysterium}.
The chapter mandated that while \quoted{the lessons should be sung in their
entirety}, \quoted{the \term{chanzoneta} shall serve for the Responsory, which
shall be prayed speaking while the singing is going on}.%
\begin{Footnote}
    MEX-Pc: AC 1633-12-30:
    \quoted{que a los maitines de nauidad deste año y de los venideros \Dots{}
    se canten todas las liçiones yn totum sin dejar cossa alguna dellas y que la
    chansoneta sirba de Responsorio el qual se diga resado mientrass se
    estubiere cantando}.
\end{Footnote}
This villancico stood in between the lessons of the second Nocturne, which were
taken from a Christmas sermon of Leo the Great.
This means that after a cantor chanted the first lesson, a reader spoke the
mandatory liturgical text of the fourth Responsory above *while* the chorus sang
this villancico.
The conjunction of texts may not have communicated much to the lay people
outside the walls of the architectural choir, who could not hear or understand
the words of the Latin liturgical texts; but for the learned cathedral canons,
the simultaneous performance of the Latin prayer and Spanish song would have
deepened the hidden connections between the two.

The Responsory evokes the scenario of the Nativity, with the animals gathered
around the manger, just as the villancico calls up the image of angels, beasts,
and humans joining together in song around the Christ-child.
In the quatrain (a \term{redondilla}) that closes the estribillo, the lines
\foreign{tan de oír y de admirar/ tan de admirar y de oír} actually seem like a
reply to the Responsory, as though to say that the mystery of the Incarnate
Christ certainly was an \quoted{admirable} sacrament---that is, one that can be
seen---but it is also an audible sacrament.
Listeners in Puebla could not visit the manger in Bethlehem; they could only look
at the retable paintings of the Adoration of the Shepherds and the Visitation of
the Kings and imagine.
But in the conception of this villancico, they could hear the \quoted{voices of the
chapel choir} reverberating in the space and through those voices they could
\quoted{hear the voice of the Father/ singing in tones of weeping}---that is,
they could hear the Christ the Word himself not merely speaking, but singing,
through all the other voices from the choirboys up to the angels.


\end{document}
