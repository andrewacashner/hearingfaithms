\documentclass{aac-poem}
\begin{document}

% Voces las de la capilla
% For float in book
% Based on WLSCM 32

% 2018/03/09

\numberlinetrue
\begin{poemtranslation}
\begin{original}

\StanzaSection{6}[\add{Introducción}]
1. Voces, las de la capilla, &
\critnote{cuenta}{Pay attention to.} con lo que se canta, &
que es músico el rey, y \critnote{nota}{Takes note of.} &
las más leves disonancias &
a lo de Jesús infante &
y a lo de David monarca.
\SectionBreak

\StanzaSection{4}[Respuesta]
Puntos ponen a sus letras &
los siglos de sus hazañas. &
La clave que sobre el hombro &
para el treinta y tres se aguarda.
\SectionBreak

\StanzaSection{6}[\add{Introducción} cont.]
2. Años antes la divisa, &
la destreza en la esperanza, &
por sol comienza una gloria, &
por mi se canta una gracia, &
y a medio compás la noche &
remeda quiebros del alba.
\SectionBreak[\add{Repeat Respuesta}]

\StanzaSection{15}[\add{Estribillo}]
Y a trechos las distancias &
en uno y otro coro, &
grave, suave y sonoro, &
hombres y brutos y Dios, &
tres a tres y dos a dos, &
uno a uno, &
y aguardan tiempo oportuno, &
quién antes del tiempo fue. &
Por el signo a la mi re, &
puestos los ojos en mi, &
a la voz del padre oí &
cantar por puntos de llanto. &
\hphantom{uno a uno,} ¡O qué canto! &
tan de oír y de admirar, &
tan de admirar y de oír. \&

\Stanza{2}
Todo en el hombre es subir &
y todo en Dios es bajar.
\SectionBreak

\StanzaSection{4}[Coplas]
1. Daba un niño peregrino &
tono al hombre y subió tanto &
que en sustenidos de llanto &
dió octava arriba en un trino. \&

\Stanza{4}
2. Hizo alto en lo divino &
y de la máxima y breve &
composición en que pruebe &
de un hombre y Dios consonancias. \&

\end{original}

\begin{translation}
\StanzaSection{6}
1. Voices, those of the chapel choir, &
keep count with what is sung, &
for the king is a musician, and he notes &
even the most venial dissonances, &
in the manner of Jesus \critnote{the infant prince}{\term{Infante} means both infant and prince.}, &
as in the manner of David the monarch. \&

\StanzaSection{4}
The centuries of his heroic exploits &
are putting notes to his lyrics. &
The \critnote{key}{Or clef.} that upon his shoulder &
awaits the thirty-three. \&

\StanzaSection{6}
2. Years before the sign, &
\critnote{dexterity in hope}
  {In Golden Age literature \term{destreza} connotes heroic skill in combat,
    particularly in \term{esgrima} or swordsmanship.  Musically, the term
    suggests virtuosity. 
    The whole phrase sounds like a heraldic device (\term{divisa}) or motto,
    summing up Christ's mission.}%
    , & 
\critnote{with the sun}
    {Here begins a series of musical plays on words: \term{sol} and \term{mi} are solmization syllables with double meanings; \term{gloria} and \term{gracia} probably refer to the songs of Christmas in both history and liturgy like the \term{Gloria in excelsis}.}
  [on \term{sol}] a \textquote{glory} begins, &
upon me [\term{mi}] a \textquote{grace} is sung, &
and at the half-measure, the night &
imitates the trills of the dawn. \&

\StanzaSection{15}
And from afar, the \critnote{intervals}
  {Both musical intervals and astronomical distances between planetary spheres.} &
in one choir and then the other, &
solemn, mild, and resonant, &
men, animals, and God, &
three by three and two by two, &
one by one, &
they all await the opportune time, &
the one who was before all time. &
Upon the sign of \term{A (la, mi, re)}, &
with eyes placed on me [\term{mi}] &
at the voice of the Father I heard &
singing in tones of weeping--- &
\hphantom{one by one,} Oh, what a song! &
as much to hear as to admire, &
as much to admire as to hear! \&

\Stanza{2}
Everything in Man is to ascend &
and everything in God is to descend. \&

% COPLAS
\StanzaSection{4}
1. A baby gave a \critnote{wandering song}
  {Or \textquote{pilgrim song}, or the musical \term{tonus peregrinus}.} &
to the Man, and ascended so high &
that in \critnote{sustained weeping}
  {Musically, \textquote{sharps of weeping}.} &
\critnote{he went up the eighth \add{day} into the triune.}
  {Musically, \textquote{he went up the octave in a trill.}} \&

\Stanza{4}
2. From \critnote{on high}
  {\term{Alto} also denotes the musical voice part.} in divinity, &
\critnote{of the greatest and least}
  {A play on the name of very long and short music notes.}, &
he made a composition in which to \critnote{prove}{Or \textquote{test}.} &
the consonances of a Man and God. \&
\end{translation}
\end{poemtranslation}
\end{document}
