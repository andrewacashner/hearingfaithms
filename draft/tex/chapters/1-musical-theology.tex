% Cashner Villancico Book
% Chapter 1: Villancicos as Musical Theology
%
% 2016-04-18  Begun

\part{Listening for Faith in Villancicos}

\chapter{Villancicos as Musical Theology}

\label{ch:musical-theology}

\epigraph
{Faith therefore \add{comes} through hearing, and hearing, by the Word of 
Christ.}
{Romans 10:17}

St.~Paul taught that faith came by means of hearing, and one of the distinctive 
effects of the sixteenth-century reformations of Western Christianity was that 
Christians discovered new ways to make their faith audible.
Voices raised in acrid contention or pious devotion boomed from pulpits, 
clamored in public squares, and were echoed in homes and schools.
In innovative forms of vernacular music, the voices of the newly distinct 
communities united to articulate their own vision of Christian faith.
Catholic reformers and missionaries enlisted music in their campaigns to 
educate, evangelize, and build Christian civilization, both in an increasingly 
divided Europe and in the new domains of the Spanish crown across the globe.
In these efforts to make \quoted{the word of Christ} to be heard and believed, 
then, what was the role of music?
What kind of power did Catholics believe music had over the dynamics of hearing 
and faith?  
This book is a study of how Christian in early modern Spain and Spanish America 
enacted religious beliefs about music through the medium of music itself.
It focuses on villancicos, a widespread genre of devotional poetry and musical 
performance, for two primary reasons.
First, these pieces were actively employed by the Spanish church and states as 
tools for propagating faith.
By the seventeenth century villancicos had grown into a complex, large-scale 
form of vocal and instrumental music based on poetic texts in the vernacular, 
and they were performed in and around liturgical celebrations on all the major 
feast days, across the Iberian world.

In their poetic themes and their musical content, villancicos combined elements 
of elite and common culture.
In subject matter as well as in the places and occasions of their performance, 
villancicos stood on the threshold between the world in and outside of church 
(which is not quite the same as a modern divide between sacred and secular).
Sets of villancicos features dramatic, often comic texts reminiscent of Spanish 
minor theater (\term{teatro menor}) alongside cultivated and even arcanely 
sophisticated theological reflections.
The music for villancicos covered a wide stylistic range from old-style 
polyphonic techniques to highly rhythmic music drawing on dance traditions.

The boundary-crossing nature of villancicos is expressed vividly in one of the 
more commonly performed pieces from colonial Mexico, a villancico set by Juan 
Gutiérrez de Padilla for the Cathedral of Puebla de los Ángeles in 1653 (in 
what is now Mexico).
As they perform a song in the \term{jácara} subgenre, the singers describe 
their own music-making as serving up \quoted{village feed} to a refined courtly 
table.
% \begin{poemtranslation}
% \begin{original}
% A la jácara, jacarilla, \\
% de buen garbo y lindo porte \\
% traigo por plato de corte \\
% siendo pasto de la villa.
% \end{original}
% \begin{translation}
% For the \term{jácara}, a little \term{jácara}, \\
% in good taste and with fair mien \\
% I bring as a dish of the court \\
% what is really feed from the village.
% \end{translation}
% \end{poemtranslation}

Villancicos are valuable, then, for assessing the interaction of these distinct 
elements that meet within the genre.
Were villancicos, according to one view of post-Tridentine religious arts, a 
form of top-down \soCalled{propaganda} intended to indoctrinate and control?
Or were they a grassroots expression of popular devotion?
Did they work on multiple levels, even contradictory ones?

The second reason for focusing on villancicos is that a large portion of the 
repertoire explicitly addresses theological beliefs about music.
The Spanish poetry of seventeenth-century villancicos frequently treats musical 
topics, sometimes using ingenious conceits that create rich and nuanced links 
between musical and theological ideas.
The musical settings turn a poetic discourse about music into a musical 
discourse about music.

Of all the musical forms of Catholic Spain, then, sacred villancicos address 
the theological nature and function of music most frequently and directly.
Among the thousands of surviving printed poetry leaflets and musical 
manuscripts, a great many pieces begin with direct invocations of the sense of 
hearing, exhorting the congregation to \gloss{oíd}{hear}, 
\gloss{escuchad}{listen}, and \gloss{atended}{pay attention}.
Villancico poets and composers favored themes of singing and dancing, as in the 
many Christmas pieces that represent the choirs of angels, singing and dancing 
shepherds, magi, and even animals in the Bethlehem stable.
There are also representations of instrumental performance and characteristic 
dances of different ethnic groups such as African slaves and \soCalled{gypsies}.
Early examples of villancicos about singing include \VCtitle{Gil pues a cantar} 
from Pedro Ruimonte's \worktitle{Parnaso español} (Antwerp, 1614) and 
\VCtitle{Sobre vuestro canto llano} by Gaspar Fernández (Puebla, 1610).%
  \autocites[296--309]{Ruimonte:Parnaso}[240--244]{Fernandez:Cancionero}
  % \X Note on Fernandez as Portuguese vs Spanish
A few \soCalled{black} villancicos or \term{negrillas}, which feature dancing 
and playing instruments as part of a caricatured representation of Africans, 
have become well known, including \VCtitle{A siolo Flasiquiyo} by Juan 
Gutiérrez de Padilla (Puebla, 1653) and \VCtitle{Los coflades de la estleya} by 
Juan de Araujo (Sucre, \circa{1700}).%
  \autocites{Padilla:Tello}
    [Araujo ed. Robert Stevenson (1985) in][I:~655--668]{Burkholder:Anthology}
These pieces constitute \quoted{music about music}.
If a play within a play in Spanish Golden Age drama may be termed 
metatheatrical, then these pieces are \quoted{metamusical}. % \X cite Begue

Understanding the theology of music articulated in villancicos can illuminate 
why and how villancicos were used to propagate faith.
Doing so will deepen our knowledge of how music in general fit into the 
religious worldview of early modern Catholics.
For Catholic believers in the Spanish Empire of the seventeenth century, what 
kind of power did music have to effect the relationship between faith and the 
sense of hearing?
If music had supernatural power, how was that power linked to the worldly 
powers of the church and state? 
By engaging interpretively with these villancicos we may gain a better 
understanding of how early modern Catholics used music for spiritual ends, and 
how the spiritual intersected with the worldly functions of this music within 
Spanish society.

This book is the first large-scale attempt to understand the theological aspect 
of seventeenth-century villancicos across the Hispanic world.
Previous musicological studies have contributed to our understanding of the 
formal development of the genre and on its function within ritual and social 
life. %\X examples
Literary studies of villancicos have concentrated primarily on the works of Sor 
Juana Inés de la Cruz, though since villancicos were written to be set to 
music, a full understanding of the genre requires a consideration of both 
literary and musical aspects. %\X examples
There has not been a study, however, that provides a global perspective on 
villancicos by collecting examples from across the Spanish Empire, or that 
combines detailed poetic and musical analysis with cultural interpretation in 
the context of contemporary theological literature, visual art, and social 
history.

The primary goal of this project is to combine musical and theological 
analysis, to understand how theological beliefs were expressed and shaped 
through the details of musical composition and performance.
The goal is to understand what may be termed the \term{musical theology} of 
villancicos.
This term does not refer to a mere verbal formulation of theological ideas 
about music; it also does not indicate a normative, personal interpretation of 
music in spiritual terms (as in \soCalled{theomusicology}). %\X cite
Instead, we may conceive of this historical form of devotional performance as a 
communal act in which religious ideas and values were enacted through musical 
structures.
To understand the theological content, we must understand the musical 
practices; and to make sense of the music, we must seek to hear it as a form of 
theological expression.


\section{Music about Music in the Villancico Genre}

The villancicos studied in this book refer to making or hearing music.
They are music that refers to itself, music about music.
The first \mentioned{music} in that formulation has at least three distinct 
layers of meaning referring to musical performance: 
\begin{enumerate}
\item the performance instructions encoded in notation, 
\item the music as it sounds when performed, generalizing from various possible 
interpretations and guessing about elements of performance not recorded in 
notation, and also 
\item the piece as it existed in history, such as at its first known 
performance in a particular place.
\end{enumerate}
By the second term \mentioned{music} we may mean several things depending on 
the piece in question: 
\begin{enumerate}
\item other sounding music (the \term{musica instrumentalis} of Boethius) that 
the villancico imitates or to which the piece alludes, quotes, or pays homage, 
as in musical topics and tropes; or 
\item music as an abstract concept, which can have increasing levels of 
abstractions along a Neoplatonic chain ascending to the \quoted{music} of the 
Triune Godhead itself.
\end{enumerate}
In other words, a metamusical villancico is a performance of music that evokes 
other types of audible, imagined, or conceptual music.

A global survey of villancico poems and music reveals nine main categories of 
metamusical villancicos.%
  \footnote{This non-exhaustive survey was drawn from archival musical and 
poetic sources and from listings in catalogs and published studies, covering a 
global range of sources.}
The survey found nearly nine hundred extant, cataloged villancicos that 
reference musical themes, a number that only hints at the original size of this 
repertoire (\tableref{metamusical-survey}).
The largest proportion, thirty-one percent, include explicit references to the 
sense of hearing, as well as echoes, applause, and exhortations to 
\quoted{listen}, \quoted{hear}, or \quoted{be quiet}.
Seventeen percent of the metamusical pieces invoke techniques and terms from 
musical performance, and specifically mention singing, voices, harmony, rhythm, 
counterpoint, and solmization.

Pieces in other categories represent music-making by birds, especially the 
\gloss{ruiseñor}{nightingale}; by angels of varying ranks like cherubim and 
seraphim; and by the heavenly spheres---that is, the stars and planets of 
Ptolemaic and Boethian cosmology.
Thirty-eight pieces mention the \gloss{clarín}{clarion or bugle}, and others 
mention bells, drums, castanets, tambourines, flute, violin, 
\gloss{gaitas}{bagpipes}, even theorbo.
Numerous pieces invoke dancing either in general or by referring to specific 
dances by name, though these references to oral and folk traditions can be 
difficult to corroborate.
Many of the dance pieces are also \soCalled{ethnic} villancicos parodying the 
distinctive speech, singing, and dancing of blacks and other non-Castilian 
groups.
Also included in the survey are pieces that directly address questions of faith 
and sensation, such as pieces that personify the senses or represent people 
with sensory impairments; as well as pieces that appeal to hearer's affections 
with exhortations to weep, cry, or rejoice.

%************
\begin{table}
\caption{Topics of metamusical villancicos in global survey}
\label{table:metamusical-survey}
\inputtable{metamusical-survey}
\end{table}
%***********

\section{Evoking the Festival: Metamusical Topics}

To understand how villancicos refer to music, we may begin with two pieces that 
invoke several of these topics at the same time---pieces that also stand as 
representative examples of the villancico genre in the mid-seventeenth century.
The first example is \VCtitle{En la gloria de un portalillo} by Juan Gutiérrez 
de Padilla (born near Málaga, \circa{1590}, died Puebla, 1664).
The piece was performed during the liturgy of Matins on Christmas Eve in 1652 
by the musical chapel of the cathedral of Puebla de los Ángeles.
Like many villancicos, it survives only in a set of manuscript performing parts 
(\signature{MEX-Pc}{Leg.~1/2}).
In just the first seven lines of this anonymous text, the villancico refers to 
sound, voices, singing, choirs, dancing, birds, and solmization:
% \begin{poemtranslation}
% \begin{original}
% En la gloria de un portalillo, \\
% los zagales se vuelven niños \\
% y en tonos sonoros \\
% repiten a coros \\
% en bailes lucidos. \\
% Canten las aves \\
% al Sol nacido. \\
% ¡Vaya de fiestas! \\
% pues Dios es niño.
% \end{original}
% \begin{translation}
% In the glory (\term{Gloria}) of a little stable, \\
% the shepherd boys become children \\
% and in resounding tones \\
% they repeat in chorus (or \quoted{in choirs}) \\
% in brilliant dances. \\
% Let the birds sing \\
% to the newborn Sun (\term{sol}). \\
% On with the festivities! \\
% for God is a baby boy.
% \end{translation}
% \end{poemtranslation}

Padilla's setting demonstrates several typical features of the genre.% 
(\exmusicref{Padilla-En_la_gloria}).
The piece begins with a soloist whose words present a striking poetic conceit, 
and whose music likewise lays out a central musical theme for the 
\gloss{estribillo}{refrain}.
The solo line is followed by a passage of dialogue between two four-voice 
choirs, concluding in typical polychoral style with an emphatic cadence for the 
full chorus.
(The bottom voice of each was played instrumentally on \term{bajón}---dulcian 
or bass curtal.)
Padilla's setting is in a lively triple meter, \term{tiempo imperfecto de 
proporción menor}, C\meter{3}{2}, notated with the cursive shorthand CZ. %\X 
cite
Padilla makes frequent use of syncopation as well as \term{sesquialtera} or 
hemiola (shifting from two groups of three minims to three groups of two).
The shifts of duple and triple stresses combine with accents on the second beat 
of the \gloss{compás}{\term{tactus}, measure} to create an energetic atmosphere 
with a rejoicing affect.
The polychoral dialogue, with the voices of each choir declaiming 
homorhythmically in the same highly accented, syncopated manner as the soloist, 
and with the \gloss{tiples}{boy sopranos} of both choirs singing at the top of 
their range, would have brilliantly seized the attention of listeners.

% %**********
% \begin{exmusic}
% \inputexmusic{Padilla-En_la_gloria}
% \caption{Padilla, \VCtitle{En la gloria de un portalillo}, Puebla Cathedral, 
% Christmas 1652 (\signature{MEX-Pc}{Leg.~1/2}), estribillo (\measurenums{6--17})}
% \exmusiclabel{exmusic:Padilla-En_la_gloria}
% \end{exmusic}
% %**********

After this introductory \term{exordium}, the Tiple I soloist continues to 
describe the scene at the manger.
As the shepherds \quoted{are turned to boys}, Padilla has the musicians 
\quoted{turn} modally by adding C sharps, accented in a sesquialtera ($3:2$) 
group. 
The passage that follows this moment is in evenly accented ternary patterns, in 
two-compás groups.
These groups emphasize the rhymes in \quotedsp{tonos sonoros, repiten a coros} 
and the clear triple meter evokes the dances of \quotedsp{en bailes lucidos}.

When the soloist refers to the newborn Sun, he sings the note identified in 
Guidonian terminology as \solfa{D}{la, sol, re}---\term{sol} in the hard (G) 
hexachord.
On the same word, the bass accompanist plays a different \term{sol}, 
\solfa{G}{sol, re, ut}.
(Note that \foreign{sol re} in Spanish means \quoted{sun king}.)%
  \begin{Footnote}
  The major Spanish music-theoretical treatises of the seventeenth century give 
full expositions of the techniques of Guidonian solmization: 
\autocites{Cerone:Melopeo}{Lorente:Porque}.
  The frequent symbolic use of Guido's syllables in villancicos suggests that 
these treatises do reflect how music was actually taught in practice.
  \end{Footnote}

Padilla's villancico may be understood as \quoted{singing about singing} on 
several levels.
The text, which is being performed through music, itself refers to musical 
performance.
The performance by the Puebla Cathedral chapel dramatizes the historical 
celebration of the first Christmas while also celebrating the festival in 
Padilla's present day.
The music is self-referential on a symbolic level (as in the plays on 
\term{sol}), but also functions on a more simple affective level to model and 
incite affections of exuberant joy and wonder, which contemporary theological 
writers emphasized were the appropriate affects for the feast of Christmas. % 
\XXX[ref to future chapter]

A similar example of a villancico that includes multiple metamusical topics is 
\VCtitle{Fuera, que va de invención} (\signature{E-Bbc}{M/760}) by Joan 
Cererols (1618--1680), monk and chapelmaster at the Benedictine Abbey of 
Montserrat near Barcelona.%
  \autocite[81--94]{Cererols:MEM-VC}
The piece summons up all the elements of a Christmas festival---masques, 
\term{zarabandas} and other dancing, lavish decorations and clothing, pipes, 
drums, and so on.
As in many villancicos, the chorus acts dramatically in the role of the 
festival crowd, shouting affirmations (\quotedsp{¡vaya!}) for each element of 
the celebration as the soloists name them.
Whereas Padilla's \VCtitle{En la gloria de un portalillo} focused primarily on 
the music of the historical Christmas day, the villancico of Cererols is 
unambiguously about celebrating \quoted{Christmas present}.
The piece seeks a theological meaning behind the Christmas customs: the masques 
of Christmas, the poem says, are appropriate because in the Incarnation of 
Christ, \quotedgloss{Dios se disfraza}{God masks himself}. 
The villancico allows performers and listeners to celebrate the festival in two 
senses: to sing the praises of the Christmas feast, while also singing the 
praises of Christ that are appropriate to the feast.
Cererols's original audience of pilgrims to the mountaintop shrine of 
Montserrat would not have sung along with this piece, but the piece still 
invites their wholehearted participation in the rituals of Christmas, both 
through enjoying the choral singing (and joining \quoted{in spirit}, perhaps), 
and in the many other common-culture customs that the piece celebrates.

In referring to singing within the song, villancico poets probably drew 
inspiration from the genre's close association with psalmody and liturgical 
chants.
The liturgical psalms are full of self-referential statements like \quoted{Sing 
to the Lord a new song} (\scripture{Ps 97}), or \quoted{Come, let us worship 
and bow down} (\scripture{Ps 95}).
The latter psalm was sung as the Invitatory hymn at every Matins liturgy, and 
thus preceded most villancico performances.
The first Responsory of Christmas Matins describes and enacts the angels' 
\worktitle{Gloria}, and the third Responsory is devoted to the shepherds' 
adoration.%
  \autocite[172--173]{Catholic:Breviarium1631}

Vernacular villancicos take this rhetorical posture beyond what was possible in 
the ancient Latin texts, by incorporating specific references to music-making 
of the listeners' own time and place.  
Where \scripture{Ps 150} enlists instruments of the Hebrew temple for God's 
praise (trumpet, cymbals), metamusical villancicos are able to call to mind 
instruments of the Spanish church and square, from vihuelas to bagpipes, 
clarions to African drums.

\section{Imitative References to Music: Birdsong, Instruments, and Dance}

In turning to the more specific categories of metamusical references in 
villancicos, we may distinguish between imitative and abstract references to 
music. 
Imitative pieces like those with instrumental topics refer to real human 
music-making (\term{musica instrumentalis}).
These pieces are highly intermusical, in the way a verbal text full of 
references to other texts is intertextual.

Abstract references, by contrast, refer to music more as a concept than a 
sounding reality.
These pieces evoke the higher Boethian levels of music (\term{musica humana} 
and \term{mundana}), music as a Neoplatonic ideal, and the music of Heaven, 
both angelic and divine.
These notions overlap in inconsistent ways in early modern thought.
Of course, the only way these pieces can refer to music in the abstract is 
through the medium of real sounding music.
In other words, to refer to music that cannot be heard, such as the music of 
the spheres, a composer must use some actual human music, such as old-style 
polyphony.
Some of these pieces with abstract references depend more on 
\soCalled{intramusical} relationships---that is, musical references internal to 
the individual piece itself, such as melodic or rhythmic motives or internal 
contrasts of musical style, without over references to pre-existing styles 
\soCalled{outside the piece}.

\subsection{Birdsong}

A frequent example of imitative musical reference in villancicos is when the 
ornamented vocal lines are used to represent birdsong.
In a piece called \VCtitle{Sagrado pajarillo} (Little sacred bird), Zaragoza
composer José de Cáseda sets the lyrics \quotedgloss{con gorgeos}{with trills}
to twittering melismas.%  (\exmusicref{exmusic:Caseda-Sagrado_pajarillo}).
Cáseda's trills may be compared to the Athanasius Kircher's transcriptions of 
birdsong in the \worktitle{Musurgia universalis}, which was disseminated widely 
in the Spanish Empire.%
  \autocite[30, Iconismus III]{Kircher:Musurgia}

% %**********
% \begin{exmusic}
% \inputexmusic{Caseda-Sagrado_pajarillo}
% \caption{Bird-like trills in Cáseda, \VCtitle{Sagrado pajarillo} 
% (\signature{MEX-Mcen}{CSG.155}), excerpt from the estribillo, Tiple I-1}
% \exmusiclabel{exmusic:Caseda-Sagrado_pajarillo}
% \end{exmusic}
% %**********

\subsection{\term{Clarines}}

Another common example of the imitative, intermusical type would be a piece 
that mentions \gloss{clarines}{clarions or bugles}, in which the singers 
perform patterns that are meant to sound like brass fanfares.
