% later version
Catholic defenders like Thomas More accused Martin Luther of replacing the
trustworthy institutional church with a foolhardy reliance on individual
subjective experience.%
    \footnote{Thomas More, \worktitle{Responsio ad Lutherum} (1523).}
From the Catholic perspective, Luther was leading his flock into danger by
asking common people to listen to his voice alone and ignore the chorus of
church fathers who condemned his heresy.
It was the work of the Holy Spirit in the divinely sanctioned hierarchy of the
Roman church that gave its sacraments their objectively operating power and
allowed that faithful to have certainty of faith and salvation---not personal
interpretations of Scripture or some kind of inner conviction.%
    \Autocite[131--208]{Schreiner:Certainty}

The post-Reformation Catholic theology of faith and sensation, however, did not
completely remove the subjective element.
On the one hand, the catechism of Trent is careful to teach that the human
senses alone cannot reach the knowledge of God, according to a tradition of
Augustinian Neoplatonism:
\quoted{in order for our minds to reach God, since nothing is more sublime than
God, our mind needs to be pulled away from everything that pertains to the
senses---something that we, in this natural life, do not have the capacity to
do}.%
    \Autocite[18]{Catholic:Catechismus1614}
On the other hand, the catechism acknowledges that faith requires an individual
response: it is more than having an opinion or conception of something, but
rather, faith \quoted{has the strength of the most certain agreement, such that
the mind, having been opened by God to his mysteries, firmly and steadfastly
gives assent}.%
    \Autocite[15]{Catholic:Catechismus1614}
This theology of faith begins with God's grace opening the mind of the
passive person, but ends with an act of the individual will.
Though God is beyond sensation, sensation is the entrance to the path. 
Faith comes through \quoted{what is heard}, then, from the created world that
speaks of God's nature to those who can perceive it, from the Scriptures, and
chiefly through the authoritative Roman church.
Each person is still called to a process of seeking to know God, to move beyond
sensory experience, and ultimately to \quoted{firmly and steadfastly give
assent} to the truth of God.
Moreover, reacting against the perceived \quoted{fideism} of the Lutherans, who
taught that a person's good works could not suffice for their salvation,
Catholics emphasized that faith had to bear fruit in ethical action.

% ...
Some Catholics such as the Jesuit missionaries, despite their founder's
suspicion of music, were eager to use this power to advance the cause of the
Church.
In their overseas missions, the Jesuits were involved with subjective experience
to such a degree that they even interpreted the dreams of native people in New
Spain.%
    \Autocite[40--41]{Bailey:Art}

%************
% earlier version
Regarding the problem of individual hearing, Reformation controversies had
pushed Catholics into an increasingly negative and anxious attitude toward
subjective sensation and experience.
Catholic polemicists like Thomas More accused Martin Luther of turning his
followers away from the trustworthy institutional church with its objectively
operating sacraments and leaving them with only a subjective experience as
assurance of salvation.


As Susan Schreiner argues, Catholic theologians rejected this reliance on
individual experience, teaching instead that the certainty of faith and
salvation came through the work of the Holy Spirit in the institutional church.%
    \Autocite[131--208]{Schreiner:Certainty}

Whereas for Martin Luther, the central plank of his platform---perhaps the only
real plank---was the gospel that human beings were saved by God's grace alone,
through Christ alone, by faith alone, as revealed in Scripture alone.
For Luther the Reformation was a controversy about the theology of salvation,
but Catholics saw it as a debate about authority.%
    \Autocite{Schreiner:Certainty}
Luther believed that his doctrine of salvation by faith gave comfort to
believers because it relieved them of the burden of trying to earn salvation by
good works.%
    \citXXX[Luther on Christian freedom?]
Catholic defenders like Thomas More, on the other hand, saw Luther as replacing
the trustworthy institutional church with a foolhardy reliance on individual
subjective experience.%
    \citXXX[More, Responsio ad Lutherum, 1523]
It was the work of the Holy Spirit in the divinely sanctioned hierarchy of the
Roman church that gave its sacraments their objectively operating power and
allowed that faithful to have certainty of faith and salvation---not personal
interpretations of Scripture or some kind of inner conviction.%
    \Autocite[131--208]{Schreiner:Certainty}
From the Catholic perspective, Luther was leading his flock into danger by
asking common people to listen to his voice alone and ignore the chorus of
church fathers who condemned his heresy.
Not every individual perception or experience was valid, but only those that
accorded with the revelation already given through the Church.


