% 2019/07/24
%{{{3 dance, ethnic vcs
\subsubsection{Dance and Difference: \term{Jácaras} and Social Class}

Dance topics in villancicos provided another way for the genre to point beyond
itself to other kinds of music in society, and like clarion topics these
references both reflected and reinforced Spanish attitudes toward social
structure.
Many dances are explicitly named and often the text proclaims the villancico
itself to \emph{be} a specific kind of dance, including \term{zarabanda},
\term{jácara}, \term{guarache}, \term{danza de espadas}, and
\term{papalotillo}.%
    \citXXX[sources]
Only a few of these dance references have been corroborated with other notated
sources of dance music, and while those other sources, byproducts of a chiefly
improvised tradition, are often sketchy and difficult to reconstruct, the
villancico arrangements of dances provide a complete musical texture.%
\begin{Footnote}
    Compare, for example, the elaborate instrumental dances recorded by Andrew
    Lawrence-King and the Harp Consort, \headlesscite{Lawrence-King:DancesCD}
    with the rudimentary notation of a few bars of chords and minimal strumming
    patterns in the source, \shortcite{Ruiz:Luz}.
\end{Footnote}
Gutiérrez de Padilla's sacred \term{jácaras} for Puebla Cathedral in 1651--53
and 1655 all play with variants of the same basic tune, harmonic pattern, and
rhythmic groove, and are closely similar to the music Álvaro Torrente has
reconstructed for the secular \term{jácara}.%
    \citXXX[padilla, torrente]
This type of song and dance, in the world outside the church, celebrated the
deeds of outlaws and lowlifes---the anti-hero \term{pícaros} celebrated in
popular novels---using the slang of criminals and in some examples making
sexual innuendo that still registers as obscene today.
Their braggodocio and their clever and provocative wordplay suggest a social
register similar to rap today.
In the sacred version of \term{jácaras}, the renegades whose heroic exploits
are recounted include the Christ-child and the Virgin Mary.
The \term{jácaras} from Puebla depict the baby Jesus as a gunslinger---in one
case, \quoted{from way up in Texas}---here to finish off a feud with the devil,
such a bad dude that Gentile kings come to pay their respects and throw gold
and incense at his feet.
The poetry mixes outlaw language with chivalric images and offbeat theological
references.
As in hip hop today, verbal virtuosity and inside references was prized in this
genre; in Gutiérrez de Padilla's best-known \term{jácara}, \wtitle{A la jácara,
jacarilla} (1655), every line of the coplas is built from \foreign{principios
de romances} (the first lines of traditional \term{romance} ballads), an
ingenious secret only hinted at in the final verse.
The core of the music is clearly drawn from the secular dance, but each year
Gutiérrez de Padilla made the setting more complex, contrapuntally and
rhythmically.
By 1655 he was playing with three-measure groups where the first measure was a
regular ternary pattern and the next two a sesquialtera group (i.e., like
\musFig{3}{2} followed by \musFig{3}{1}), and more complex, irregular rhythmic
patterns in the coplas.

In many cases the villancico settings may be our only remaining traces of these
dances.
The question of whether performers or listeners actually danced in church is
another problem here, related to the question of whether or to what degree
performers staged the dramatic villancicos in the sacred space.%
    \citXXX[?]
There certainly was ritual dance on Corpus Christi in Seville and Valencia
Cathedrals, performed by the boy choristers known as \term{seises}.%
    \Autocite{Comes:Danzas}
Perhaps like \term{clarín} pieces with no actual clarion, dance references in
villancicos should not be taken as evidence of actual dancing; their purpose is
to call to mind dancing that happened elsewhere and to make use of the symbolic
meanings of dance. % XXX
	
The \term{jácara} (also spelled \term{xácara} but always pronounced with a
guttural H sound) originated as a type of song and dance in Spanish theater and
street performances, typically recounting the deeds of ruffian outlaws in the
rough and sometimes bawdy language of the underworld (a comparison with rap
would not be inappropriate).%
    \Autocites{Torrente:Jacara}{XXX}
Juan Gutiérrez de Padilla included a sacred adaptation of the genre in every
one \XXX[check] of his Christmas villancico cycles for Puebla Cathedral from
1651--1659: these pieces herald the exploits of not a human \term{pícaro} but
the baby Jesus, adapting the outlaw language markers from the worldly genre to
sacred purposes.
Gutiérrez de Padilla's most well-known contribution to this subgenre is
\wtitle{A la jácara, jacarilla} from the 1655 cycle.
As he had done with his previous \term{jácaras}, the Puebla chapelmaster
borrowed the poetic text from the imprint of an earlier Royal Chapel
performance in Madrid (in this case, from the previous year).%
    \citXXX[pliego, Torrente book]
He puts this text to a variant of the same tune that he had used in his three 
preceding \term{jácaras} that survive and the same general style: the main tune
features a stepwise gesture ascending and descending motive
C\sh--D--E--F--E--D--C\sh{} harmonized with \musFig{5 3} chords on the first
and fifth degree of the first mode (to modern ears, this sounds like \term{i}
and \term{V} in D minor).
% TODO example (Ruiz y Ribayaz and transcription in jazz-chord notation?)
This matches the basic outlines of the improvised tune type reconstructed by
Álvaro Torrente for the secular \term{jácara} (secular as in worldly and
irreverent, in theme and performance venue).

Like the improvised model, Gutiérrez de Padilla's setting moves rhythmically in
triple meter (\meterCZ) with extraordinarily heavy use of syncopation and
sesquialtera.
As this composer developed this tune in each year's successive setting, he made
the rhythm and phrasing more complex each time.
% TODO examples
For a good portion of the estribillo in 1655, he creates what to current
knowledge is an unprecedented nine-minim pattern of three-measure groups: a
normal group of three minims is followed by a sesquialtera group with pulses in
three imperfect semiminims.
The melody in the coplas defies any attempt at regular rhythmic grouping. 
% XXX

Why would Gutiérrez de Padilla create such a complex rhythmic and polyphonic
setting to represent a dance with such common, even sordid origins?
First, the beginning of the text proclaims a specific intention to bring
contrasting worlds together.
% TODO quote
The contrast between \term{corte} and \term{villa} is between noble and common,
gentility and laborers, urban and rural, refined and crude---notably not sacred
and secular.
It is also a play on the term \term{villancico}, which comes from \term{villa},
and suggests an attempt to say something here about the function and meaning of
the genre as a whole.
Mixing the style and specific motives of a low-life ballad genre with the
techniques of learned counterpoint; in fact using compositional technique to
actually amplify the complexity of the oral source material perhaps
beyond what would more readily be improvised, certainly contributed to this
goal of mixing high and low elements of society.
Theologically the Christmas feast actually centered on the mixture of high and
low, as the infinite and all-powerful God had confined himself to the
vulnerable body of the tiny Christ-child (see \cref{ch:padilla-voces}).
Christ's birth in a feed-trough and his manifestation to lowly shepherds and
heathen magi were also understood in the Spanish context to elevate the dignity
of lower-class people, though typically in way that ultimately reinforced the
social hierarchy rather than challenging it.%
    \citXXX[al establo]
Compared to source material like the \soCalled{Frog \term{Jácara}}, which
catalogs sexual positions in explicit detail, Gutiérrez de Padilla's
representations of Christ as an outlaw---in one piece, a gunslinger from
\quoted{way up in Texas}---seem quite tame, but within the context of what New
Spanish worshippers could hear in church they must have brought some delight
and sense of play into the liturgy.%
    \citXXX[Torrente, playing cards; vc ex]
In the last copla of \term{A la jacara, jacarilla}, the singer tells the
Christ-child, \quoted{We will leave you here with these \term{principios de
romances}}, tipping off listeners who had not yet figured it out that the
preceding set of \XXX[no.] verses were all constructed from the first lines of
traditional \term{romance} ballads.
% TODO table
Here again is a popular practice (again comparable to hip hop) of rapid-fire
quotations riffing on existing texts and reshaping them into new meanings, but
written down and given a fully notated musical setting in a complex, highly
literate manner.

The trickster quality of the typical \term{jácara} hero may also explain the
cryptic texts and puzzling musical patterns: the \term{jaque} was often a
gambler, a swindler, and a quick draw, so the sacred \term{jácara} became a
site for poetic and musical trickery.
Later in the century, pieces called \term{jácara} did not always have the
distinct musical markers connected to the secular source traditions; but they
did retain this sense of playful ingenuity.
Mateo de Villavieja's \term{Jácara en anagramas} (\XXX[date, place], from the
Convento de la Encarnación in Madrid)
features a poetic text written in anagrams, such that the lines and phrases of
one stanza are shuffled to create the next.%
    \footnote{\sig{E-MO}{AMM.4261}.}
\XXX[details]
The music, too, is composed in anagrams: the voice parts are rotated for each
successive copla; as are the phrases. \XXX[details]

The reasons for Gutiérrez de Padilla's rhythmic play with triple meter may be
hinted in a \term{jácara} villancico poem by Manuel de León Marchante.
In \XXX[16XX], León Marchante wrote a set of villancicos for \XXX[Toledo]
Cathedral in which, after an introductory piece, each villancico represented
one of the seven liberal arts.
(The next year he would balance things out with a set on the \quoted{manual
arts}, including sailing, surgery, and blacksmithing.\XXX[check!])
It is fascinating how León Marchante pairs the conventional subgenres of
villancicos with each of the divisions of learning: geometry is a
\term{villancico de naciones} (an \quoted{ethnic} villancico, see below),
because one needs geometry to make maps and navigate\XXX[other examples].
Where does the \term{jácara} appear?
As \term{arithmetic}---because, León Marchante says in the villancico, it is
\quoted{governed by the rule of threes}.
Perhaps there is a connection here to Gutiérrez de Padilla's three-measure
groups of triple meter.%
    \footnote{Perhaps also to his use of the symbolic number 33 in his
    depiction of Christ as a card player, which I have proposed is a
    proto-jácara.}
The jácara, then, would be a game of numbers, celebrating the ultimate
trickster who hid divine identity inside a child's body\XXX[etc].

The cost of turning the jácara into a theologically signicant display of
wit and ingenuity, it would seem, is losing a connection to the lower-class
sources of the genre.
Sacred jácaras became yet another pious entertainment for the educated classes,
perhaps with a bit of added thrill by their association with ribald origins,
but increasingly losing any sense of crossing boundaries of \term{corte} and
\term{villa} in a way that would have had any meaning for residents of the
latter.
%}}}3

%{{{3 ethnic VCs
\subsubsection{Representing Ethnic Difference}

Metamusical references to traditional music-making of lower-class people
extended also to the depiction of ethnic difference.
There are villancicos that depict non-Castilian groups like Native Americans,
African people, Catalans, Frenchmen, even Irishmen, through caricatured
deformations of language and music.
What have come to be called \quoted{ethnic villancicos} were labeled in their
time as \term{villancicos de naciones} or by the name of the particular ethnic
type for that piece, like \term{gallego} (Galician), \term{gitano}
(\quoted{Gypsy}), \term{indio} (\quoted{Indian}), or \term{negro},
\term{guineo}, and similar terms for Africans.
Most of these pieces, and especially the \term{villancicos de negro}, refer
specifically to the characteristic music and dancing of these groups, often
naming their instruments and describing their motions.
The texts use some smatterings of foreign words but mostly ask the performers
to put on an accent in Spanish: in these caricatures the Gypsy ends all her
words with a Z (\foreign{Puez los trez zon Magoz,/ hombrez de ezfera}).
\begin{Footnote}
    \term{Vamos al portal gitanilla}, Imprint from Epiphany 1666, Zaragoza,
    Iglesia de El Pilar (\sig{E-Mn}{VE/1303/1}), later attributed to Vicente
    Sánchez, \headlesscite[203--204]{Sanchez:LiraPoetica}.
\end{Footnote}
The Black says when he should say R and J, drops ending S sounds, and
mismatches genders and cases (\foreign{Mi siñol Manuele, \Dots{} Sesu, \Dots{}
pluque son linda cosa}).%
\begin{Footnote}
    \term{Al establo más dichoso}, Music manuscripts by Juan Gutiérrez de
    Padilla of \term{ensaladilla} for Christmas 1652, Puebla Cathedral
    (\sig{MEX-Pc}{Leg. 1/3\XXX}), \XXX[WLSCM32].
\end{Footnote}
Villancicos about African characters also frequently feature nonsense
syllables, whether lullaby phonemes like \foreign{ro ro ro ro} and \foreign{le
le le le}, or apparent gibberish like \foreign{tumbucutú, cutú, cutú} and
\foreign{gulumbé, gulumbá} that tells us what African languages like Kikongo
sounded like to a Spanish ear.%
    \citXXX[al establo, other]
This type of piece represents Africans as always happily engaged in drumming
and dancing.%
    \citXXX[baker etc]

These pieces were created by Spaniards primarily for other Spaniards;
\quoted{black villancicos} are not really about depicting African identity but
are rather ways of constructing a Spanish one by reference to the Other.
Immediately after purging Iberia of both Moors and Jews, the Castilians had
been overwhelmed with encounters with new ethnic groups, languages, and
religions around the world; these pieces offered the potential to create an
imagined world in which all these groups were situated in their proper place
within a well-controlled social hierarchy.
These pieces may offer glimpses of the language and music of non-Spanish
groups, but only through a glass heavily darkened by racial prejudice and
deliberate caricature for the sake of humor and mockery; they further clouded
by the cultural distance from which modern observers must approach these
pieces.
With regard to the nature of musical references in \soCalled{ethnic}
villancicos, then, these pieces encompass a mixture of literal imitation (as of
percussion, and of the \soCalled{musical} sound of foreign languages) and
broader stylistic references (as, perhaps, to African musical styles, though no
one has yet demonstrated concrete evidence for a connection).
They also include more abstract references to music through the use of nonsense
words that, somewhat like solmization syllables (see below), symbolize and
enact music-making.
Like \term{jácaras}, ethnic villancicos grow increasingly conventionalized over
the years and more distant from the low-caste sources they grew from, so that
the \term{negro} character in one year's villancicos was much more similar to
the \term{negro} of the previous year's set that he probably was to any real
person of African descent.
And like \term{clarín} pieces, ethnic villancicos both reflected and reinforced
imperial Spain's power structure by projecting a theological vision of that
structure as divinely ordained and immobile.
That said, we must note that very few of these pieces have received any serious
scrutiny, especially their music; and that those that are known are not simply
racist caricatures like later minstrel shows in the United States.

Their discourse on racial difference must be understood within a Neoplatonic
theological concept of music and society, in which the lowliest elements in the
created world could lead a person to the knowledge of the highest.
Juan Gutiérrez de Padilla, who included a \quoted{black villancico} in most of
his Christmas cycles for Puebla, depicted the paradox of Neoplatonic thought
when in 1652 he had his black characters, described as Angolans in the piece
like Gutiérrez de Padilla's own slave Juan, say \quoted{Listen, for we are
singing like the angels}.
As the Angolans go on to sing a vernacular \term{Gloria in excelsis} in their
dancing triple meter, full of syncopations notated by coloring in the mensural
noteheads, above them suddenly enter the two boy soprano parts of the second
chorus, which have been silent until now, singing the same \term{Gloria} with
them---but in the white notes of duple meter, and quoting a plainchant
intonation (\cref{mux:Padilla-Al_establo-Gloria}).
The Angolans and the angels are brought together for a miraculous moment
through contrasting types of rhythmic movement in which the hidden harmony
between earthly and heavenly music is revealed.
The Angolans are in some ways depicted sympathetically, as instead
of the gold, frankincense, myrrh of the magi (one of whom was portrayed on
Puebla's high altar as a black African), bring the Christ-child the homelier
and more practical gifts: a potato, a toy, and diapers.
But nothing about the piece really exalts the African characters in any way
that would affect the lives of real Africans like Gutiérrez de Padilla's slave.

%{{{5 music GdP Al establo Gloria
\insertMusic{Padilla-Al_establo-Gloria}
{Gutiérrez de Padilla, \wtitle{Al establo más dichoso (Ensaladilla)} 
(\sig{MEX:Pc}{Leg. 2/1}, Puebla Cathedral, Christmas 1652), \term{Negrilla}
section: polymetrical \term{Gloria} of Angolans and angels}
%}}}5

It is possible, though more evidence is needed, that the vogue for black
villancicos at Christmas and Epiphany was linked to the practice across the
Spanish and Portuguese Empires of \quoted{Black Kings} festivals, in which
confraternities of enslaved and free people of African descent elected a mock
royal court and paraded them around their city with music and dancing, usually
with military elements with origins in the Christian Kingdom of Kongo before
the start of slavery.%
    \citXXX[sources]
This possible connection does not mean that these villancicos express any real
African voice or viewpoint; rather, they tell us about the insecurities, fears,
and prejudices of Spaniards and may help us understand how they justified their
place in an unjust society by appeal to theology and aesthetic beauty.
Gutiérrez de Padilla's polymetrical Gloria fits perfectly with the image of
angels singing and dancing on the round painted high the new Puebla
Cathedral's high altar, hovering over the images of shepherds and magi greeting
the newborn Christ at the altar's base (see \cref{ch:padilla-voces}), and we
can imagine the theological aesthetics of this were some comfort to this
chapelmaster-priest and his peers; but they were no help to enslaved men and
women and, when such pieces are revived uncritically today, they continue to do
their historic work of reinforcing racial prejudice and appeasing the
consciences of elite (and typically white) listeners.

With regard to hearing and faith, ethnic villancicos and black villancicos in
particular enabled Spaniards to envision themselves as the rightful masters of
a society in which other groups were naturally subordinate; in other words what
they heard helped them believe in the rightness of the social order as governed
by the church. %XXX
Though their representations are purposefully distorted, they do suggest that
the Spanish elite accepted the coexistence of multiple languages and types of
music within society, and enjoyed sampling these exotic sounds through the safe
filter of their own caricatures.
%}}}3


