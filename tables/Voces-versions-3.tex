% Voces-versions-3.tex
\begin{tabular}{lll}
    \toprule
    1642 Lisbon (Santiago?) & 
    1647 Seville (Jalón?) & 
    1657 Puebla (Padilla) \\
    \midrule 
 
    &
    &
    \textsc{Coplas} \\


    \strophe{} A suspensiones del cielo. &
    \strophe{} \uline{A Suspensiones} \uline{el Cielo} & 
    \\

    hacen sus esferas pausas, &
    \uline{hace} \uline{en sus esferas pausas,} &
    \\

    porque el Ángel, que las mueve &
    \uline{porque el Ángel que} los \uline{mueve} &
    \\

    cuenta le admira, se pasma. &
    \uline{cuanto le admira} le \uline{pasma.} &
    \\

    \strophe{} Los compases son del tiempo, &
    \strophe{} \uline{Los compases son del tiempo,} &
    \\

    que ya sus compases guarda, &
    \uline{que ya sus compases guarda} &
    \\

    quien al tiempo lleva siglos, &
    \uline{quien al tiempo lleva siglos,} &
    \\

    quien lleva al siglo distancias. &
    \uline{quien lleva al siglo distancias.} &
    \\

    \strophe{} Toda la solfa, la cifran, &
    & \\

    relieves de nieve, y grana, &
    & \\

    en dos labios que rubrica, &
    & \\

    y en dos mejillas que escarcha. &
    & \\

    \strophe{} Los puntos son cuantas perlas &
    & \\

    dos vivos diamantes passan, &
    & \\

    de si mismos, que las vierten &
    & \\

    a un pesebre que las guarda. 
    & \\

    & 
    \strophe{} Tambien se canta a ternario, 
    & \\

    & 
    pues entran, caben y passan 
    & \\

    & 
    tres Reyes en un compas, 
    & \\
    
    &
    de \uline{brutos} \uline{breve} morada. 
    & \\


    & 
    \strophe{} \uline{Por sol comienza una gloria,}  &
    \\

    & 
    \uline{por mi se canta una gracia,} & 
    \\
    
    & \uline{y a medio compás la Noche} & 
    \\

    & \uline{remeda quiebros del Alba.} &
    \\

    & 
    &
    \strophe{} Daba un niño peregrino \\
    
    &
    &
    tono al hombre y subió tanto \\
    
    &
    &
    que en sustenidos de llanto \\
    
    &
    &
    dió octava arriba en un trino. \\
    
    &
    &
    \strophe{} Hizo alto en lo divino \\
    
    &
    &
    y de la máxima y breve \\
    
    &
    &
    composición en que pruebe \\
    
    &
    &
    de un hombre y Dios consonancias. \\

    \strophe{} \emph{Y a trechos \et{}c.}
    & \strophe{} \emph{O que lindamente \et{}c.}
    & \strophe{} \emph{Y a trechos \et{}c.} \\

    \bottomrule
\end{tabular}
\endinput
