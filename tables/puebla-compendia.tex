\documentclass[table]{vcbook-float}
\renewcommand{\arraystretch}{1.5}
\begin{document}
\begin{tabulary}{\linewidth}{lLl}
    \toprule
    Author & Book & \emph{Ex libris} mark \\
    \midrule

    Augustine, St.
    & \wtitle{Tomus primus \add{--decimus} omnium operum D. Aurelii Augustini
    Hipponensis episcopi \Dots{}} 10 vols.
    Paris, 1555
    & Oratorio de San Felipe Neri \\

    Bigne, M. de
    & \wtitle{Magna bibliotheca veterum patrum et antiquorum scriptorum
    Ecclesiasticorum Opera \Dots{}: continens Scriptores saeculi II id est, ab Ann.
    Christi 100 usq; 200}
    Cologne, 1618
    & Colegio de San Juan \\

    Corderio, B.
    & \wtitle{Catena LXV patrum graecorum in sanctum Lucam}
    Antwerp, 1628
    & Colegio del Espíritu Santo \\

    Corderio, B.
    & \wtitle{Catena Patrvm Graecorum in Sanctvm Ioannem ex Antiqvissimo Graeco Codice
    MS. \Dots{}}
    Antwerp, 1630
    & Colegio de San Juan \\

    Feliciano, G. B.
    & \wtitle{Catena explanationvm veterum sanctorum patrum, in Acta Apostolorum,
    \add{et} Epistolas catholicas}
    Basel, 1552
    & Convento de Santo Domingo\\

    Lapide, C. à
    & \wtitle{Commentarium in IV. Evangelia}
    London, 1638
    & Colegio del Espíritu Santo \\

    Luis de Granada
    & \wtitle{Sylua locorum communium omnibus diuini verbi concionatoribus \Dots{}:
    in qua tum veterum Ecclesiae Patrum tum philosophorum, oratorum et poëtarum
    egregia dicta aureaeq\add{ue} sententiae \Dots{} leguntur}
    London, 1587
    & Oratorio de San Felipe Neri \\

    Murillo, D.
    & \wtitle{Discursos predicables sobre los evangelios que canta la Iglesia en los
    quatro Domingos de Aduiento, y fiestas principales que ocurren en este tiempo
    hasta la Septuagesima}
    Zaragoza, 1610
    & \quoted{Biblioteca del seminario} \\

    \bottomrule
\end{tabulary}
\end{document}
