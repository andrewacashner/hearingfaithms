% Voces-versions-2.tex
\documentclass{vcfloat}
\newcommand{\strophe}{\hspace{1em}}
\begin{document}

\begin{tabular}{lll}
    \toprule
    1642 Lisbon (Santiago?) & 
    1647 Seville (Jalón?) & 
    1657 Puebla (Padilla) \\
    \midrule 
 
    \textsc{Estribo} &
    \textsc{Estribo} &
    \\

    \strophe{} Y a trechos las estancias, &
    &
    \strophe{} \uline{Y a trechos las distancias} \\

    en uno, y otro coro, &
    &
    \uline{en uno y otro coro,} \\

    grave, suave, sonoro, &
    &
    \uline{grave, suave y sonoro,} \\

    hombres, y brutos, y Dios. &
    &
    \uline{hombres y brutos y Dios,} \\

    tres a tres, y dos a dos &
    &
    \uline{tres a tres y dos a dos,} \\

    uno a uno, &
    &
    \uline{uno a uno,} \\
    
    & 
    O que lindamente suenan! 
    & \\
    
    & 
    o que dulcemente cantan 
    & \\
    
    & 
    al compás que lleva \uline{el Infante,} 
    & \\
    
    & 
    Serafines que cruzan y passan! 
    & \\
    
    & 
    y de sus gemidos 
    & \\
    
    & 
    aprended trinados y sustenidos, 
    & \\
    
    & 
    y con mil primores 
    & \\ 
    
    & 
    responden los Reyes y los Pastores, 
    & \\
 
    & 
    después que \uline{aguardaron uno} 
    & \\

    y aguarda tiempo oportuno, &
    que llegó a \uline{tiempo oportuno}, &
    y aguardan \uline{tiempo oportuno,} \\ 
    
    quién antes del tiempo fue: &
    \uline{quien antes del tiempo fue} &
    \uline{quién antes del tiempo fue}. \\

    por el signo a la mi re, &
    \uline{por el signo a la mi re,} &
    \uline{Por el signo a la mi re,} \\

    puestos los ojos en mi &
    \uline{puestos los ojos en mi,} &
    \uline{puestos los ojos en mi,} \\

    a la voz del padre oí &
    con que mil maravillas vi &
    \uline{a la voz del padre oí} \\

    cantan por puntos de llanto, &
    &
    cantar \uline{por puntos de llanto.} \\

    O qué canto &
    &
    \uline{O qué canto} \\

    tan de oír, y de admirar! &
    \uline{tan de ir} [sic] \uline{y de admirar,} &
    \uline{tan de oír y de admirar,} \\

    & 
    que si lo acierto a dezir, &
    tan de admirar y de oír. \\

    todo en el hombre es subir, &
    \uline{todo en el hombre es subir,} &
    \uline{Todo en el hombre es subir} \\

    y todo en Dios es bajar, &
    \uline{y todo en Dios es bajar.} &
    \uline{y todo en Dios es bajar.} \\

    y con el favor usanos, &
    & \\

    los corazones humanos, &
    & \\

    vuelven guecos. &
    & \\


    \strophe{} \speaker{Eco} Ecos, &
    & \\

    pues zagalejos: &
    & \\

    \strophe{} \speaker{Eco} Lejos, &
    & \\

    desvalidos, &
    & \\

    \strophe{} \speaker{Eco} Idos, &
    & \\

    distraídos, &
    & \\

    \strophe{} \speaker{Eco} Traídos, &
    & \\

    son sustenidos, &
    & \\

    \strophe{} \speaker{Eco} tenidos, &
    & \\

    del niño hermoso, &
    & \\

    suena blando el arrullo, &
    & \\

    y en la tropa el orgullo &
    & \\

    travieso del viento &
    & \\

    por aquí por allí contento, &
    & \\

    inquieto, y bullicioso, &
    & \\

    se penetra blando, &
    & \\

    y discurre airoso.
    & \\
    \bottomrule
\end{tabular}
\end{document} 
