\begingroup
\footnotesize
\renewcommand{\arraystretch}{1.5}
\noindent\begin{tabulary}{\textwidth}{LLLLL}
    \toprule
    Composer & Villancico Incipit & Occasion, Place & Source Type & Source
    Location \\ \midrule
    
    Fray Francisco de Santiago & 
    \emph{Voces, las de la capilla} &
    Pre-1644 Christmas, Seville Cathedral (or Lisbon) &
    Text incipits, 1649 catalog &
    Lisbon, lost collection of João IV \\

    Luis Bernardo Jalón & 
    \emph{Cantores de la capilla} & 
    1647 Epiphany, Seville Cathedral & 
    Text, poetry imprint & 
    Sole exemplar in Puebla \\

    Juan Gutiérrez de Padilla & 
    \emph{Voces, las de la capilla} & 
    1657 Christmas, Puebla Cathedral & 
    Music, partbooks & 
    Puebla Cathedral archive \\
    \bottomrule
\end{tabulary}
\endgroup
\endinput
