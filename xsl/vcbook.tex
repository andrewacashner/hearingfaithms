
    % Converted from TEI-XML by Andrew's custom XSL teitolatex.xsl
    \documentclass{memoir}
    \usepackage{lmodern}
    \usepackage[T1]{fontenc}
    \usepackage[utf8]{inputenc}
    \usepackage[american]{babel}
    \usepackage{microtype}
    \usepackage{csquotes}
    
    \begin{document}
    
    \frontmatter
    
    \begin{titlingpage}
    
    \title{Faith, Hearing, and the Power of Music in Hispanic Villancicos, 1600–1700}
  
    \author{Andrew A. Cashner}
  
    \date{2015}
  
    \maketitle
    \end{titlingpage}
    \tableofcontents*
  
    \chapter{Preface}
    \label{chapter:preface}
    
    This preface aims to help readers who may approach this book from a variety of disciplines and backgrounds, by clarifying basic terms and concepts as they are used here.
    Villancico in this study refers broadly to a type of Spanish vernacular poetry and music prevalent across the Spanish Empire in the seventeenth century.\footnote
    The genre originated as a courtly entertainment in the fifteenth and sixteenth century, similar to formes fixes like the frottola and later more similar to the madrigal.
    It likely drew on roots in oral traditions of poetic recitation and popular song.
    In the last decades of the sixteenth century, villancicos, some of whic had always had religious content, came to be performed in and around church liturgies, especially in connection with sacred dramas.
    By the early seventeenth century the villancico had become a predominantly liturgical and paraliturgical genre.
  
    In function, villancicos were performed most frequently in sets of eight or more pieces, interspersed after the Lessons in the liturgy of Matins, and either following or substituting for the singing of the Latin Responsories, depending on local custom.
    At the Cathedral of Puebla de los Ángeles, for instance, the musical chapel performed villancicos in place of the Responsory chants while a cleric spoke the words of the Responsory aloud (see \ref{#appendix:PueblaDecree}).
    Sets of villancicos were performed at Matins for the highest feast days of the Hispanic church year in every major church, from Madrid to Manila—Christmas, Epiphany, Corpus Christi, and the Conception of Mary.
    Villancicos could also be performed at Mass (functioning similarly to motets) or as devotional pieces in Forty Hours’ Devotion and so-called siesta services of Eucharistic devotion.
  
    The villancico repertoire includes many subgenres within it, such as particular song and dance types and conventional topics.
    Ensaladas (salads) are medleys of song types probably based on popular sources.
    \enquote{Ethnic} villancicos make fun of non-Castilian people groups, especially through deformations of language.
    The most notorious of these are the negrillas or \enquote{black villancicos}, which represent people of African descent in denigrating, caricatured ways (see \ref{#chapter:Puebla}).
  
    \mainmatter
    
    \chapter{Villancicos as Musical Theology}
    \label{chapter:intro}
    
	  ergo fides ex auditu/ auditus autem per verbum Christi
	
	Thus faith [comes] from hearing; and hearing, by the word of Christ.
      
	St. Paul, Romans 10:17
      
    St. Paul taught that faith came by means of hearing, and one of the distinctive effects of the sixteenth-century reformations of Western Christianity was that Christians discovered new ways to make their faith audible.
    Voices raised in acrid contention or pious devotion boomed from pulpits, clamored in public squares, and were echoed in homes and schools.
    In new forms of vernacular music, the voices of the newly distinct communities united to articulate their own vision of Christian faith.
    Catholic reformers and missionaries enlisted music in their campaigns to educate, evangelize, and build Christian civilization, both in an increasingly divided Europe and in the new domains of the Spanish crown across the globe. 
    In these efforts to make the word of Christ to be heard and believed, then, what was the role of music?
    What kind of power did Catholics believe music had over the dynamics of hearing and faith?
  
    This book is a study of how Christians in early modern Spain and Spanish America enacted religious beliefs about music through the medium of music itself.
    It focuses on villancicos, a widespread genre of devotional poetry and musical performance, for two primary reasons.
    First, these pieces were actively employed by the Spanish church and state as tools for propagating faith.
    By the seventeenth century villancicos had grown in to a complex, large-scale form of vocal and instrumental music based on poetic texts in the vernacular, and they were performed in and around liturgical celebrations on all the major feast days, across the Spanish world.
  
    In their poetic themes and in their musical content, villancicos combined elements of elite and common culture. 
    In subject matter as well as in the places and occasions of their performance, villancicos stood on the threshold between the world in and outside of church (which is not quite the same as a modern divide between sacred and secular). 
    Sets of villancicos featured dramatic, often comic texts reminiscent of Spanish minor theater (teatro menor) alongside cultivated and even arcanely sophisticated theological reflections.
    The music for villancicos covered a wide stylistic range from old-style polyphonic techniques to highly rhythmic music drawing on dance traditions.
  
    The boundary-crossing nature of villancicos is expressed vividly in one of the more commonly performed villancicos from colonial Mexico, a piece set by Juan Gutiérrez de Padilla for Puebla Cathedral in 1653 (discussed further in chapter 7).
    Here the singers describe their own music-making as serving up village feed to a refined courtly table:
    A la jácara, jacarilla,de buen garbo y lindo portetraigo por plato de cortesiendo pasto de la villa.For the jácara, a little jácara,in good taste and with fair mienI bring as a dish of the courtwhat is really feed from the village.
    Villancicos are valuable, then, for assessing the interaction of these distinctive elements that meet within the genre.
    Were villancicos a form of top-down \enquote{propaganda} intended to indoctrinate and control, as some have claimed of post-Tridentine religious arts?
    Or were they a grassroots expression of popular devotion? 
    Did they work on multiple levels, even contradictory ones?
  
    The second reason for focusing on villancicos is that a large portion of the repertoire explicitly addresses theological beliefs about music.
    The Spanish poetry of seventeenth-century villancicos frequently treats musical topics, sometimes using ingenious conceits that create rich and nuances links between musical and theological ideas.
    The musical settings turn a poetic discourse about music into a musical discourse about music.
  
    Of all the musical forms of Catholic Spain, then, sacred villancicos address the theological nature and function of music most frequently and directly.
    Among the hundreds of surviving musical manuscripts and the even larger quantity of printed poetry leaflets, a great many pieces begin with direct invocations of the sense of hearing, exhorting hearers to hear, listen, and pay attention (oíd, escuchad, atended).
    Villancico poets and composers favored themes of singing and dancing, as in Christmas sets, for example, they represented the angelic choirs of Christmas, singing and dancing shepherds, Magi, and even animals in the Nativity stable.
    There are also representations of instrumental performance and characteristic dances of different ethnic groups (such as African slaves and \enquote{gypsies}).
    Early example of villancicos about singing include Gil pues a cantar from Pedro Ruimonte’s Parnoso español (Antwerp, 1614) and Sobre bro canto llano by Gaspar Fernández (Puebla, 1610).\footnote
    A few \enquote{black} villancicos or negrillas, which represent dark-skinned people dancing and playing instruments in what is supposed to be their characteristic style, have become well known, such as A siolo Flasiquiyo by Juan Gutiérrez de Padilla (Puebla, 1653) and Los coflades de la estleya by Juan de Araujo (Sucre, ca. 1700).\footnote
    These pieces constitute \enquote{music about music}.
    If a play within a play in Spanish Golden Age drama may be termed metatheatrical, then these pieces are \enquote{metamusical}.
  
    Understanding the theology of music articulated in villancicos can illuminate why and how villancicos were used to propagate faith.
    Doing so will deepen our knowledge of how music general fit into the religious worldview of early modern Catholics.
    For Catholic believers in the Spanish Empire of the seventeenth century, what kind of power did music have to affect the relationship between faith and the sense of hearing?
    If music had supernatural power, how was that power linked to the worldly powers of church and state?
    By engaging interpretively with these villancicos we may gain a better understanding of how early modern Catholics used music for spiritual ends, and how the spiritual intersected with the wordly functions of this music within Spanish society.
  
    This book is the first large-scale attempt to understand the theological aspect of seventeenth-century villancicos across the Hispanic world.
    It is also one of only a few studies to analyze villancicos musically in detail.
    Most importantly, the primary goal of the project is to combine these two modes of analysis, to understand how theological beliefs were expressed and shaped through the details of musical composition and performance.
  
    The goal is to understand the musical theology of villancicos.
    This term does not mean just a verbal formulation of theological ideas about music by historical figures, and it certainly does not connote a normative theological interpretation from a modern scholar’s own personal perspective.
    
    Instead, we may conceive of this historical form of devotional performance as a communal act in which religious ideas and values were performed (put into practice) through musical structures.
    To understand the theological content, we must understand the musical practices; and to make sense of the music, we must seek to hear it as a form of theological expression.
  
    \section{Overview}
    \label{}
    
      This book will explore the musical theology in villancicos in three parts.
      Part I (chapters 1 and 2) argues that villancicos on the subject of music may be interpreted as sources for historical theologies of music.
      It considers how certain conceptual problems regarding music’s role in the relationship between faith and hearing manifested in this genre, and proposes a historically grounded model for understanding these pieces theologically.
    
      The first chapter in part I introduces the category of metamusical villancico in its several subtypes, using examples by composers who will be discussed further in the rest of the book.
      The chapter traces the roots of the interpretive approach in this book within musicology and several other disciplines, and clarifies the project’s relationship to existing scholarship on villancos and early modern sacred music.
    
      The second chapter argues that the relationship between hearing and faith was theological problem in seventeenth-century Spain.
      Catholics had to balance the desire to make faith accommodate the sense of hearing with the need to train the sense of hearing.
      Hearing had to be shaped by faith in order to perceive the content of faith.
      To understand music’s role in connecting hearing and faith, the chapter examines how villancicos that represent the senses, sensory confusion, and sensory impairment manifest theological concepts of perception.
      The chapter situates villancicos within a Neoplatonic understanding of hearing and music and outlines three primary theological functions of villancicos, each of which requires a different kind of listening practice.
    
      Part II (chapters 3--6) presents detailed case studies of individual pieces, or pieces in specific traditions and places, on themes of heavenly music.
      These pieces constitute singing about singing.
      Chapters 3 and 4 each interpret a single villancico tradition that represents earthly music as a Neoplatonic reflection of heavenly music.
      Chapter 3 focuses on a piece from Puebla in colonial Mexico, and chapter 4 traces a web of settings connected to a villancico from Montserrat in Catalonia.
      Both pieces develop the theological trope of Christ as the verbum Infans—the unspeaking/infant Word whose incarnate body is itself the highest form of music, harmonizing divine and human.
    
      Chapters 5 and 6 present groups of pieces from specific locations (Segovia and Zaragoza region) that demonstrate a shift in theological understandings of music, where earthly music is seen as more an expression of human affects than as a reflection of heavenly order.
      
      All four chapters in part II also demonstrate that these metamusical villancicos functioned as a special subgenre in which composers could demonstrate their own mastery within the context of a lineage of composition and a tradition of treatments of this theme of heavenly music.
    
      Part III (chapters 7 and 8) focuses on how the musical theology of villancicos was developed in coordination with the Spanish projects of colonizing and civilizing in the New World.
      Chapter 7 looks at the relationship of Juan Gutiérrez de Padilla’s Christmas villancico cycles (extant from 1651--1659) and the building of Puebla Cathedral (consecrated 1649).
      It argues that Padilla’s villancico cycles construct a utopian microcosm of hierarchical colonial society.
      The chapter focuses on Padilla’s representations of people at the bottom of the social hierarchy, such as Indians and black slaves, in a piece from 1652.
      Especially through the interplay of language and musical rhythm, this composer and his ensemble constructed a world in which every member of colonial society was put into its proper hierarchical place, in a combination of Neoplatonic music theory and ethics.
      
      The final chapter draws general conclusions and points to directions for future work.
    
      Accompanying this text is an anthology of the scores most thoroughly discussed in the book, together with their poetic texts and annotated translations.
      These music and poetry editions are an integral part of this project of interpretation and communication; the book will not be coherent without constant reference to these sources.
      Most of these have never been edited before, and a few now receive their first critical, corrected editions.
      The English translations are among the first translations of seventeenth-century villancico poems into any language.
    
      To begin, then, we must consider what metamusical villancicos are and what they reveal about seventeenth-century theological concepts of music.
    
    \section{Music about Music in the Villancico Genre}
    \label{}
    
      The villancicos studied in this book refer in some way to music.
      Some focus on making music; others on hearing it.
      As such, these pieces constitute music that refers to itself.
    
      If we say that a villancico is music about music, the two instances of music in this label have multiple meanings.
      The first music refers to a specific villancico as a musical entity, which includes the performance instructions encoded in notation, the music as it sounds when performed (generalizing from various possible interpretations and guessing about elements of performance not recorded in notation), and also the piece as it existed in history, such as its first known performance in a particular place.
      By the second term music we may mean several things depending on the piece in question: it could mean other sounding music (the musica instrumentalis of Boethius) that the villancico imitates or to which the piece alludes, quotes, or pays homage, as in musical topics and tropes.
      It could also mean music as an abstract concept, which can hve increasing levels of abstractions along a Neoplatonic chain ascending to the \enquote{Music} of the Triune Godhead itself.
      The essential point is that metamusical villancicos create a link between two kinds or levels of music: the music performed and heard points beyond itself to other kinds of music, human or celestial.
    
      A global survey of villancico poems and music reveals nine main categories of metamusical villancicos.
      This non-exhaustive survey was drawn from archival musical and poetic sources and form listings in catalogs and published studies (see the bibliography under Primary Sources), covering a global range of sources.
      The survey found more than nine hundred extant, cataloged villancicos that reference metamusical themes, a number that only hints at the original size of this repertoire.
      Table \ref{#table:survey-metamusical-topics} lists the most common topics in order of frequency.
      We will consider select examples of pieces in each category, beginning with two exemplary villancicos in which several of these topics are found together.
    QuantityPercentTopicNotes26830.9Hearing, sound
	  Includes explicit references to the sense of hearing, as well as echoes, applause and exhortations to listen, hear, or be quiet.
	15017.3Music, singing
	  General references to music, singing, voices, harmony rhythm counterpoint, solmization
	13415.5Birdsong
	  Birds as musicians, their songs, specific birds like the ruiseñor (nightingale)
	11313.0Dance
	  Invitations to dance, specific dances such as the jácara; many of these are \enquote{ethnic} villancicos parodying blacks, Indians, \enquote{gypsies}, Catalans, etc., singing and dancing
	768.8InstrumentsClarín (clarion or bugle: 38 examples in survey), bells, drums, castanets, tambourines, flute, violin, even theorbo
	758.7Angels
	  Specifically musical references to angels (among a vast number of pieces about angeles in general): angel choirs, specific types of angels like cherubim, seraphim
	202.3Heavens or spheres
	  Usually not referring to English Heaven (in Spanish, this was cielo Empyreo), but to cielos as in the music of the heavenly spheres—the stars and planets of Ptolemaic and Boethian cosmology
	161.8Sensation and faith
	  Pieces that connect faith with the senses, especially hearing and sight
	151.7Affects
	  Exhortations to weep, cry, rejoice; or apostrophes to the affects themselves
	
    \section{Pieces with Multiple Topics: Padilla and Cererols}
    \label{}
    
      It is common to find references to several of these topics in a single piece, and looking at two typical examples of this sort will begin to make clear what is meant by metamusical villancicos.
      The first example is a villancico from the 1652 Christmas cycle (MEX-Pc: Leg. 1/2) written for the Cathedral of Puebla de los Ángeles by Juan Gutiérrez de Padilla (ca. 1590–1664).\footnote
      In just the first seven lines of this anonymous text, the villancico refers to sound, voices, singing, choirs, dancing, birds, and solmization (poem ).
    En la gloria de un portalillo,los zagales se vuelven niñosy en tonos sonorosrepiten a corosen bailes lucidos.Canten las avesal Sol nacido.¡Vaya de fiestas!pues Dios es niño.In the glory (Gloria) of a little stable,the shepherd boys become childrenand in resounding tonesthey repeat in chorus [or in choirs]in brilliant dances.Let the birds singto the newbord Sun [the note sol]for God is a baby boy.
      Padilla’s setting demonstrates several typical features of the genre (music example ).
      
      The piece begins with a soloist whose words present a striking poetic conceit, and whose music likewise lays out a central musical theme for the estribillo (refrain).
      The solo line is followed by a passage of polychoral dialogue between two four-voice choirs, concluding (typically for polychoral technique) with an emphatic cadence for the full chorus.
      Padilla’s setting is in a lively triple meter (tiempo menor de proporción menor, notated CZ in Spanish sources) that makes frequent use of sesquialtera or hemiola.\footnote
      The shifts of duple and triple stresses combine with stresses on the second beat of the compás (tactus, measure) to create an energetic atmosphere with a rejoicing affect.
      The polychoral dialogue, with the voices of each choir declaiming homorhythmically in the same highly rhythmic, syncopated manner as the soloist, and with the tiples (boy sopranos) of both choirs singing at the top of their range, would have brilliantly seized the attention of listeners.
    
      After this introductory exordium, the Tiple I soloist continues to describe the scene at the manger.
      As the shepherds are turned to boys, Padilla has the musicians turn modally by adding C sharps, accented in a sesquialtera (3 : 2) group.
      The passage that follows this moment is in evenly accented ternary patterns, in two-compás groups.
      These groups emphasize the rhymes in tonos sonoros, repiten a coros and the clear triple meter evokes the dances of en bailes lucidos.
    
      When the soloist refers to the newborn Sun, he sings the note identified in Guidonian terminology as D (la, sol, re)—sol in the hard (G) hexachord.
      On the wame word, the bass accompanist plays a different sol, G (sol, re, ut). (Note that sol re in Spanish means sun king.)
      \footnote
      Padilla's villancico may be understood as singing about singing on several levels.
      The poetic text, which being performed through music, itself refers to musical performance.
      The performance by the Puebla Cathedral chapel dramatizes the historical celebration of the first Christmas while also celebrating the festival in Padilla’s present day.
      The music is self-referential on a symbolic level (as in the plays on sol), but also functions on a more simple affective level to model and incite affections of exuberant joy and wonder, which contemporary theological writers emphasized were the appropriate affects for the feast of Christmas.
      \footnote
      A similar example of a villancico that includes multiple metamusical topics is Fuera, que va de invención (E-Bbc: M/760) by Joan Cererols (1618–1680), monk and chapelmaster at the Benedictine Abbey of Montserrat near Barcelona.
      \footnote
      Like the numerous catalog-style Christmas songs in English, from Deck the Halls with Boughs of Holly to Chestnuts Roasting on an Open Fire, this villancico summons up all the elements of a Christmas festival—masques, zarabandas (sarabandes) and other dancing, lavish decorations and clothing, pipes, drums, and so on.
      As in many villancicos, the chorus acts dramatically in the role of the festival crowd, shouting affirmations (¡vaya!) for each element of the celebration as the soloists name them.
      Whereas Padilla’s En la gloria de un portalillo focused primarily on the music of the historical Christmas day, the villancico of Cererols is unambiguously about celebrating \enquote{Christmas present}.
      The piece seeks a theological meaning behind the Christmas customs: the masques of Christmas, the poem says, are appropriate because in the Incarnation of Christ, Dios se disfraza (God masks himself).
      The villancicos allows performers and listeners to celebrate the festival in two senses: to sing the praises of the Christmas feast, while also singing the praises of Christ that are appropriate to that feast.
      Cererols’s original audience of pilgrims to the mountaintop shrine of Montserrat would not have sung along with this piece, but the piece still invites their wholehearted participation in the rituals of Christmas, both through enjoying the choral singing (and joining \enquote{in spirit}, perhaps), and in the many other common-culture customs that the piece celebrates.
    
    \chapter{Bibliography}
    \label{biblio}
    "Tis Nature's Voice": Music, Natural Philosophy and the Hidden World in Seventeenth-Century EnglandLinda PhyllisAusternMusic Theory and Natural Order from the Renaissance to the Early Twentieth CenturySuzannahClarkAlexanderRehdingCambridge30–67Cambridge University Press2001Art on the Jesuit Missions in Asia and Latin America, 1542–1773Gauvin A.BaileyToronto1999The "Ethnic" Villancico and Racial Politics in 17th-Century MexicoGeoffreyBaker399–408Bolognese Instrumental Music, 1660–1710: Spiritual Comfort, Courtly Delight, and Commercial TriumphGregory RichardBarnettAldershot, UKAshgate2008A Greek-English Lexicon of the New Testament and other Early Christian LiteratureWalterBauerFrederick WilliamDanker3ChicagoUniversity of Chicago Press2000A Literary and Typological Study of the Late 17th-Century VillancicoAlainBegue231–282"I Have Never Seen Your Equal": Agricola, the Virgin, and the CreedM. JenniferBloxamEarly Music34391–4082006"The Third Villancico Was a Motet": The Villancico and Related GenresAndreaBombi149–1882007"Suban las vozes al Cielo" Villancico polifónico de Miguel Ambiela prodia del homónimo de su maestro Pablo BrunaPedroCalahorra Martínez219–421986Breviarium Romanum ex decretum Sacros. Conc. Trid. Restitutum Pii V. Pont. Max. iuss editum, & Clementis VIII Primum, nunc denuo Vrbani Pp. VIII auctoritate recognitumCatholic Church[Rome?]1631Arte y naturaleza: El bodegón español en el Siglo de OroPeterCherryMadridEdiciones Doce Calles1999Apparitions in Late Medieval and Renaissance SpainWilliam A., Jr.ChristianPrincetonPrinceton University Press1981A Voice and Nothing MoreMladenDolarCambridge, MAMIT Press2006Agudeza y arte de ingenio, en qve se explican todos los modos y diferencias de concetos, con exemplares escogidos de todo lo mas bien dicho, assi sacro, como humanoBaltasarGraciánAntwerp1669Los dos cuerpos del rey en Calderón: <i>El nuevo palacio del Retiro</i> y <i>El mayor encanto, amor</i>MargaretGreerEl teatro clásico español a través de sus monarcasLucianoGarcía LorenzoMadrid181–202Fundamentos2006Beginnings in Ritual StudiesRonald L.Grimesn.p.n.d."Music Charms the Senses. . .": Devotional Music in the \mkbibemphTriunfos festivos of San Ginés, Madrid, 1656JanetHathaway219–230Baroque Music: Music in Western Europe, 1580–1750John WalterHillNew YorkNorton2005Aristotle's Physics and Its Medieval VarietiesHelen S.LangAlbanyState University of New York Press1992Boccherini's Body: An Essay in Carnal MusicologyElisabethLe GuinBerkeleyUniversity of California Press2006A History of Afro-Hispanic Language: Five Centuries, Five ContinentsJohn M.LipskiCambridgeCambridge University Press2005Breve historia de PueblaLeonardoLomelí VanegasMexico CityEl Colegio de México2001Juan Gutiérrez de Padilla desde el ámbito civil: Un <i>corpus</i> documentalGustavoMauleón Rodríguez179–242Breve, e svccinta Relatione del Viaggio nel regno dei Congo nell' Africa MeridionaleGirolamoMerollaNaples1692"... de Ángeles también el coro": Estética y simbolismo en la misa Ego flos campi de Juan Gutiérrez de PadillaRicardoMiranda131–153Anglo-American Postmodernity: Philosophical Perspectives on Science, Religion, and EthicsNanceyMurphyBoulder, COWestview Press1997Aportes metodológicos para una investigación sobre música colonial MexicanaBárbaraPérez Ruízyear 2321–792002A Silent Minority: Deaf Education in Spain, 1550–1835SusanPlannBerkeley, CAUniversity of California Press1997Archivo de música sacra de la catedral de PueblaLincoln B.SpiessThomsStanfordMexico City1967Arte métrica españolaJuan DíazRengifoSalamanca1592Transmisión poética y dramatúrgica del dogma en el auto <i>El nuevo palacio del Retiro</i> de Calderón: La teología eucarística de las metáforasDominiqueReyre102113–1222008<i>Todo el mundo en general</i>, ecos historiográficos desde Chile de una copla a la Inmaculada Concepción en la primera mitad del siglo XVIIVíctorRondón212009Music as an "Instrumentum Regni" in Spanish Seventeenth-Century DramaJack W.Sage613384–3901994A Select Library of the Nicene and Post-Nicene Fathers of the Christian Church. Second SeriesPhilipSchaffEdinburghT&T Clark1894Are You Alone Wise?: The Search for Certainty in the Early Modern EraSusan ElizabethSchreinerOxfordOxford University Press2011The "Distinguished Maestro of New Spain": Juan Gutiérrez de PadillaRobert MurrellStevenson353363–3731955¿Cómo se cantaba al "tono de jácara"?Álvaro JoséTorrenteLiteratura y música del hampa en los Siglos de OroMadridVisor Libros2014Música, propaganda y reforma religiosa en los siglos XVI y XVII: cánticos para la "gente del vulgo" (1520–1620)Alfonso deVicente12007\footnoteBiblia Sacra iuxta vulgatam versionemRobertWeberRogerGryson4StuttgartDeutsche Biblegesellschaft1994"Vor deinen Thron tret ich and the Art of Dying"DavidYearsleyBach and the Meanings of CounterpointCambridge1–41Cambridge University Press2002Magna bibliotheca veterum patrum et antiquorum scriptorum Ecclesiasticorum Opera et sturdio doctissimorum in alma universitate Colon. Agripp. theologorum ac profess. ; tomus secundus : continens Scriptores saeculi II id est, ab Ann. Christi 100 usq; 200Marguerin deBigneCologne1618Catena Patrvm Græcorum in Sanctvm Ioannem ex Antiqvissimo Græco Codice MS. Nvnc Prmvm in Lvcem editaBalthasarCorderioAntwerp1630Catena LXV patrum graecorum in sanctum LucamBalthasarCorderioAntwerp1628Catena explanationvm veterum sanctorum patrum, in Acta Apostolorum, [et] Epistolas catholicasGiovanni BernardinoFelicianoBasel1552Divinarum Scriptuarum iuxta sanctorum patrum sententias locupletissimus thesaurus in quo parabole, metaphorae, phrases, et difficiliora quaeq[ue] loca totius sacrae paginae declarantur: cum concordia utriusq[ue] TestamentiJuanFernándezMedina del Campo1594Decreta scriptorum ecclesiasticorum, conciliorvm, et romanorum pontificvm / Dn. Gratiani opera congesta; suasque in classes distributa: & succinctis Antonij Democharis ooaratítlois illustrata [sic in catalog]GratianAntonius?DemocharusLondon1555Concionum de temporeFrayLuis de GranadaSalamanca1577Sylua locorum communium omnibus diuini verbi concionatoribus ... : in qua tum veterum Ecclesiæ Patrum tum philosophorum, oratorum et poëtarum egregia dicta aureæq[ue] sententiæ ... legunturFrayLuis de GranadaLondon1587Discursos predicables sobre los evangelios que canta la Iglesia en los quatro Domingos de Aduiento, y fiestas principales que occurren en este tiempo hasta la SeptuagesimaFray DiegoMurilloZaragoza1610Villancicos qve se cantaron en la S. Iglesia Metropolytana de Seville, en los Maytines de los Santos Reyes. En este año de mil y seiscientos y quarenta y sieteSeville1647Letras de los villancicos de Navidad, Que se han de cantar en la Santa Iglesia de Toledo Primada de las Españas, este año de 1666Toledo1666Letras de los villancicos de Navidad, Que se han de cantar en la Santa Iglesia de Toledo Primada de las Españas, este año de 1667Toledo1666Villancicos que se han de cantar en el Real Convento de la Encarnación, la Noche de Navidad, Este Año de 1671Madrid1671Villancicos que se cantaron en la S. I. Catedral de la Puebla de los Ángeles, en los Maitines solemnes de la Purísima Concepción de Nuestra Señora, este año de 1689SorJuana Inés de la CruzPuebla de los Ángeles1689A History of Western MusicPeter J.BurkholderDonald JayGroutClaude V.Palisca9th editionNew YorkNorton2014Norton Anthology of Western MusicPeter J.BurkholderClaude V.Palisca7th editionNew York1: Ancient to BaroqueNorton2014Music: A Very Short IntroductionNicholasCookOxfordOxford University Press1998Anthology for Music in the BaroqueWendyHellerNew YorkNorton2014Music in the BaroqueWendyHellerNew YorkNorton2014Source Readings in Music HistoryOliverStrunkLeoTreitlerRevised editionNew YorkNorton1998Music in the Western World: A History in DocumentsPieroWeissRichardTaruskinNew YorkSchirmer1984\mkbibquoteTis Nature's Voice: Music, Natural Philosophy and the Hidden World in Seventeenth-Century EnglandLinda PhyllisAusternMusic Theory and Natural Order from the Renaissance to the Early Twentieth CenturySuzannahClarkAlexanderRehdingCambridge30–67Cambridge University Press2001The \mkbibquoteEthnic Villancico and Racial Politics in 17th-Century MexicoGeoffreyBaker399–408\mkbibquoteI Have Never Seen Your Equal: Agricola, the Virgin, and the CreedM. JenniferBloxamEarly Music34391–4082006\mkbibquoteThe Third Villancico Was a Motet: The Villancico and Related GenresAndreaBombi149–1882007\mkbibquoteSuban las vozes al Cielo Villancico polifónico de Miguel Ambiela prodia del homónimo de su maestro Pablo BrunaPedroCalahorra Martínez219–421986Los dos cuerpos del rey en Calderón: \mkbibemphEl nuevo palacio del Retiro y \mkbibemphEl mayor encanto, amorMargaretGreerEl teatro clásico español a través de sus monarcasLucianoGarcía LorenzoMadrid181–202Fundamentos2006\mkbibquoteMusic Charms the Senses. . .: Devotional Music in the \mkbibemphTriunfos festivos of San Ginés, Madrid, 1656JanetHathaway219–230Juan Gutiérrez de Padilla desde el ámbito civil: Un \mkbibemphcorpus documentalGustavoMauleón Rodríguez179–242\mkbibquote\dots de Ángeles también el coro: Estética y simbolismo en la misa \mkbibemphEgo flos campi de Juan Gutiérrez de PadillaRicardoMiranda131–153Transmisión poética y dramatúrgica del dogma en el auto \mkbibemphEl nuevo palacio del Retiro de Calderón: La teología eucarística de las metáforasDominiqueReyre102113–1222008\mkbibemphTodo el mundo en general, ecos historiográficos desde Chile de una copla a la Inmaculada Concepción en la primera mitad del siglo XVIIVíctorRondón212009Music as an \mkbibquoteInstrumentum Regni in Spanish Seventeenth-Century DramaJack W.Sage613384–3901994The \mkbibquoteDistinguished Maestro of New Spain: Juan Gutiérrez de PadillaRobert MurrellStevenson353363–3731955¿Cómo se cantaba al \mkbibquotetono de jácara?Álvaro JoséTorrenteLiteratura y música del hampa en los Siglos de OroMadridVisor Libros2014Música, propaganda y reforma religiosa en los siglos XVI y XVII: cánticos para la \mkbibquotegente del vulgo (1520–1620)Alfonso deVicente12007\footnote\mkbibemphVor deinen Thron tret ich and the Art of DyingDavidYearsleyBach and the Meanings of CounterpointCambridge1–41Cambridge University Press2002Die Musik der Natur- und Orientalischen KulturvölkerHandbuch der MusikgeschichteGuidoAdlerFrankfurt am MainFrankfurter Verlags-Anstalt1924Teresa of Ávila and the Politics of SanctityGillian T. W.AhlgrenIthaca, NYCornell University Press1996El maestro aragonés Miguel de Ambiela (1666–1733): su contribución al Barroco musicalCarmen MaríaÁlvarez EscuderoOviedo, SpainArte-Musicología, Servico de Publicaciones, Universidad de Oviedo1982Peirce's Theory of SignsAlbertAtkinThe Stanford Encyclopedia of PhilosophyEdward N.ZaltaStanford, CAStanford University2010\footnoteThe One Who Comes after Me: John the Baptist, Christian Time, and Symbolic Musical TechniquesMichael AlanAndersonJournal of the American Musicological Society663639–7082013Tomus primus [-decimus] omnium operum D. Aurelii Augustini Hipponensis episcopi, ad fidem vetustorum exemplarium summa vigilantia repurgatoru[m] à mendis innumeris ... : cui accesserunt libri epistolae, sermones, & fragmenta aliquot, hactenus nunquam impressa. Additus est & index, multo quám Basilensis fuerat copiosor.St.Augustine of HippoParis1555The City of GodSt.Augustine of HippotranslatorMarcusDodsNew YorkThe Modern Library1993De civitate DeiSt.Augustine of HippoJ. P.MigneParisMigne1841Patrologia latina41Sermones de temporeSt.Augustine of HippoJ. P.MigneParisMigne1841Patrologia latina38\mkbibquoteTis Nature's Voice: Music, Natural Philosophy and the Hidden World in Seventeenth-Century EnglandLinda PhyllisAusternMusic Theory and Natural Order from the Renaissance to the Early Twentieth CenturySuzannahClarkAlexanderRehdingCambridge30–67Cambridge University Press2001Catecismo de los misterios de la fe, con la esposicion del Simbolo de los Santos Apostoles. A do se enseña, todo lo que vn fiel Cristian esta obligado a creer, y vn cura de almas a saber, para enseñar a sus ouejasFray Antonio deAzevedoBarcelona1589Art on the Jesuit Missions in Asia and Latin America, 1542–1773Gauvin A.BaileyToronto1999Joan Cererols (1618–1680): L'entorn familiar; Regest dels documents de l'Arxiu Parroquial de Martorell; Notes inèdites obre Gabriel Manalt i Domènech (1657–1687)FerranBalanza i González25–75The \mkbibquoteEthnic Villancico and Racial Politics in 17th-Century MexicoGeoffreyBaker399–408Imposing Harmony: Music and Society in Colonial CuzcoGeoffreyBakerDurham, NCDuke University Press2008Bolognese Instrumental Music, 1660–1710: Spiritual Comfort, Courtly Delight, and Commercial TriumphGregory RichardBarnettAldershot, UKAshgate2008Tonal Organization in Seventeenth-Century Music TheoryGregoryBarnett407–455The Grain of the VoiceRolandBarthesImage—Music—TextNew YorkHill and Wang1977A Greek-English Lexicon of the New Testament and other Early Christian LiteratureWalterBauerFrederick WilliamDanker3ChicagoUniversity of Chicago Press2000A Literary and Typological Study of the Late 17th-Century VillancicoAlainBegue231–282Musica ficta: Theories of Accidental Inflections in Vocal Polyphony from Marchetto da Padova to Gioseffo ZarlinoKarolBergerCambridgeCambridge University Press1987La vihuela de la iglesia de la Compañia de Jesús de QuitoEgbertoBermúdezyear 4717969-771993[Declaración de instrumentos musicales] Comiença el libro llamado declaracio[n] de instrume[n]tos musicales [. . .] examinado y aprouado por los egregios musicos Bernardino de figueroa, y Christoual de moralesJuanBermudo[Madrid]1555In nativitate DominiSt.Bernard of Clairvaux [Bernardus Claraevallensis]J. P.MigneParisMigne1854Patrologia latina183Music in the Seventeenth CenturyLorenzoBianconitranslatorDavidBryantCambridgeCambridge University Press1987Spanish Romances of the Sixteenth CenturyThomasBinkleyMargitFrenkBloomington, INIndiana University Press1995Native American Song at the Frontiers of Early Modern MusicOlivia A.BloechlCambridgeCambridge University Press2008\mkbibquoteI Have Never Seen Your Equal: Agricola, the Virgin, and the CreedM. JenniferBloxamEarly Music34391–4082006Catálogo de villancicos de la Biblioteca Nacional, Siglo XVIIBiblioteca Nacional de EspañaMadridMinisterio de Cultura1992Catálogo de villancicos y oratorios en la Biblioteca Nacional, siglos XVIII–XIXBiblioteca Nacional de EspañaMadridMinisterio de Cultura1990\mkbibquoteThe Third Villancico Was a Motet: The Villancico and Related GenresAndreaBombi149–1882007Fons de l'Església Parroquial de Sant Pere i Sant Pau de Canet de MarFrancescBonastreJosep MariaGregoriAndreuGuinart i VerdaguerBarcelonaGeneralitat de Catalunya, Departament de Cultura i Mitjans de Comunicació2009Petri Bvngi Bergomatis Nvmerorvm MysteriaPietroBongo [Petrus Bungus]Paris1643Early Christian Worship: A Basic Introduction to Ideas and PracticePaulBradshawCollegeville, MNThe Liturgical Press1996Painting in Spain: 1500–1700JonathanBrownNew HavenYale University Press1998Kirchenmusik und Seelenmusik: Studien zu Frommigkeit und Musik im Luthertum des 17. JahrhundertsChristianBunnersGöttingenVandenhoeck & Ruprecht1966Singende Frommigket: Johann Crugers Widmungsvorreden zur \mkbibquotePraxis Pietatis MelicaChristianBunners529–241980In Quest of the Period EarShaiBurstyn254693–7011997Der Villancico des XVI. und XVII. Jahrhunderts in SpanienBernatCabero PueyoBerlinDissertation.de2000Religiosity, Power, and Aspects of Social Representation in the Villancicos of the Portuguese Royal ChapelRuiCabral Lopes199–218Historia de la música en Aragón (Siglos I–XVII)PedroCalahorra MartínezZaragozaLibrería General1977\mkbibquoteSuban las vozes al Cielo Villancico polifónico de Miguel Ambiela prodia del homónimo de su maestro Pablo BrunaPedroCalahorra Martínez219–421986La música en Zaragoza en los siglos XVI y XVII: II, Polifonistas y ministrilesPedroCalahorra MartínezZaragoza, SpainInstitución \mkbibquoteFernando el Católico1977Signifying Nothing: On the Aesthetics of Pure Voice in Early Venetian OperaMauroCalcagno204461–4972003El nuevo palacio del RetiroPedroCalderón de la BarcaAlan K. G.PatersonPamplonaUniversidad de Navarra1998Autos sacramentales completos de Calderón19Playing Cards at the Eucharistic Table: Music, Theology, and Society in a Corpus Christi Villancico from Colonial Mexico, 1628Andrew A.Cashner184383–4192014Breviarium Romanum ex decretum Sacros. Conc. Trid. Restitutum Pii V. Pont. Max. iuss editum, & Clementis VIII Primum, nunc denuo Vrbani Pp. VIII auctoritate recognitumCatholic Church[Rome?]1631Catechismvs Ex Decreto Concilii Tridentini, Ad Parochos Pii V. Pont. Max. iussu editusCatholic ChurchCologne1567Catechismus ex decreto sacrosancti Concilii Tridentini: iussu Pij V. pontif. max. editus [. . .]Catholic ChurchLondon1614The Liber Usualis: With Introduction and Rubrics in EnglishCatholic ChurchTournaiDesclée & Cie.1956For More than One Voice: Toward a Philosophy of Vocal ExpressionAdrianaCavarerotranslatorPaul A.KottmanStanford, CAStanford University Press2005Istorica Descrizione de tre Regni Congo, Matmba et AngolaGiovannio AntonioCavazziBologna1687Joan Cererols IIIJoanCererolsDavidPujolsMonestir de Montserrat1932Mestres de l'Escolania de Montserrat3El melopeo y maestroPedroCeroneNaples1613Tonal Allegory in the Vocal Music of J.S. BachEricChafeBerkeleyUniversity of California Press1991Distorting Reality: Christmas Villancicos and the Culture of Sacred Immanence in Early Seventeenth-Century Puebla de los ÁngelesIreri E.Chávez-Bárcenas2014Arte y naturaleza: El bodegón español en el Siglo de OroPeterCherryMadridEdiciones Doce Calles1999Apparitions in Late Medieval and Renaissance SpainWilliam A., Jr.ChristianPrincetonPrinceton University Press1981Local Religion in Sixteenth-Century SpainWilliam A., Jr.ChristianPrincetonPrinceton University Press1981Person and God in a Spanish ValleyWilliam A., Jr.Christian2PrincetonPrinceton University Press1988Catàleg dels villancicos i oratoris impresos de la Biblioteca de Montserrat, segles XVII–XIXDanielCodina i GiolMontserratPublicacions de l'Abadia de Montserrat2003Notes, Scales, and Modes in the Earlier Middle AgesDavid E.Cohen307–363Historical CalendarA. R.Collins\footnoteDanzas del Santísimo Corpus ChristiJuan BautistaComesVicenteGarcía JulbeManuelPalauValenciaInstituto Valenciano de Musicologia, Institución Alfonso el Magnánimo, Diputación Provincial de Valencia1952Obras en lengua romanceJuan BautistaComesJoséClimentValenciaInstituto Valenciano de Musicología, Institución Alfonso el Magnanimo1977Colección de entremeses, loas, bailes, jácaras y mojigangas desde fines del siglo XVI a mediados del XVIII ordenadaEmilioCotarelo y MoriMadridBailly-Bailliére1911Emblemas moralesSebastián deCovarrubias OrozcoMadrid1610Tesoro de la lengua castellana, o españolaSebastián deCovarrubias OrozcoMadrid1611Puebla de los Ángeles: Una ciudad en la historiaMiguelÁngel CuenyaCarlosContreras CruzPuebla de los ÁngelesOcéano2012Wonders and the Order of Nature, 1150–1750LorraineDastonKatharineParkNew YorkZone Books1998St. Peter and the Triumph of the Church in Music from New SpainDrew EdwardDavies667–892009The Return of Astraea: An Astral-Imperial Myth in CalderónFrederick A.De ArmasLexington, KYUniversity Press of Kentucky1986Inka Bodies and the Body of Christ: Corpus Christi in Colonial Cusco, PeruCarolynDeanDurham, NCDuke University Press1999Listening to Sacred Polyphony c. 1500JeffreyDeanEarly Music254611–6361997Of Dancing Cardinals and Mestizo Madonnas: Reconfiguring the History of Roman Catholicism in the Early Modern PeriodSimonDitchfield83–4386–4082002A Voice and Nothing MoreMladenDolarCambridge, MAMIT Press2006La sinestesia en la poesía española: Desde La Edad Media hasta mediados del siglo XIX; Un enfoque semánticoUrsulaDoetsch KrausPamplonaEdiciones Universidad de Navarra1992Imperial Spain 1469–1716JohnElliottLondonPenguin2002Entremés de los romancesDanielEisenbergGeoffreyStagg222151–1742002Santa Mariana de Jesús, hija de la Compañía de Jesús: Estudio histórico-ascético de su espiritualidadAurelioEspinosa PólitQuitoLa Prensa Católica1956Esbós per a un estudi de l'obra de Joan Cererols (1618–1680)GrigoriEstrada7–23Villancicos aragoneses del siglo XVII de una a ocho vocesAntonioEzquerro EstébanBarcelonaConsejo Superior de Investigaciones Científicas1998Monumentos de la música española55Villancicos policorales aragoneses del siglo XVIIAntonioEzquerro EstébanBarcelonaConsejor Superior de Investigaciones Científicas2000Monumentos de la música española59Tonos humanos, letras y villancicos catalanes del siglo XVIIAntonioEzquerro EstébanBarcelonaConsejo Superior de Investigaciones Científicas2002Monumentos de la música española65The Virgin of Chartres: Making History through Liturgy and the ArtsMargot E.FasslerNew HavenYale University Press2010Music and the Order of the PassionsMarthaFeldmanRepresenting the Passions: Histories, Bodies, VisionsRichardMeyerLos Angeles37–67Getty Research Institute2003Opera and Sovereignty: Transforming Myths in Eighteenth- Century ItalyMarthaFeldmanChicagoUniversity of Chicago Press2007Cancionero musical de Gaspar Fernandes: Tomo primeroGasparFernandesAurelioTelloMexico CityCentro Nacional de Investigación, Documentación e Información Musical Carlos Chávez2001Utriusque cosmi maioris scilicet et minoris metaphysica, physica atqve technica historia, in duo volumina secundum cosmi differentiam diuisaRobertFluddOppenheim1621The Order of Things: An Archaeology of the Human SciencesMichelFoucaultNew YorkPantheon Books1971The Inquisition of Francisca: A Sixteenth-Century Visionary on TrialFrancisca de los ApóstolesGillian T. W.AhlgrenChicagoUniversity of Chicago Press2005Crossing Confessional Boundaries: The Patronage of Italian Sacred Music in Seventeenth-Century DresdenMary E.FrandsenOxfordOxford University Press2006Entre la voz y el silencio: La lectura en tiempos de CervantesMargitFrenkMadridFondo de Cultura Económica2005Pedro García Ferrer: Un artista aragonés del siglo XVII en la Nueva EspañaMontserratGalíTeruelAyuntamiento de Alcorisa: Instituto de Estudios Turolenses, Excma. Diputación Provincial de Teruel1996The Making of Baroque PoetryMary MalcolmGaylord222-237Muy amigo de música: El obispo Juan de Palafox (1600–1659) y su entorno musical en el Virreinato de Nueva EspañaMaríaGembero-Ustárroz55–130Inhabiting the Cruciform God: Kenosis, Justification, and Theosis in Paul's Narrative SoteriologyMichael J.GormanGrand Rapids, MIEerdmans2009The Role of Harmonics in the Scientific RevolutionPenelopeGoukThe Cambridge History of Western Music TheoryCambridge223–245Cambridge University Press2002Music, Science, and Natural Magic in Seventeenth-Century EnglandPenelopeGoukNew HavenYale University Press1999Representing Emotions: New Connections in the Histories of Art, Music, and MedicinePenelopeGoukHellenHillsAshgate2005Music and the SciencesPenelopeGoukThe Cambridge History of Seventeenth-Century MusicTimed. CarterJohnButtCambridge132-157Cambridge University Press2005Agudeza y arte de ingenio, en qve se explican todos los modos y diferencias de concetos, con exemplares escogidos de todo lo mas bien dicho, assi sacro, como humanoBaltasarGraciánAntwerp1669Los dos cuerpos del rey en Calderón: \mkbibemphEl nuevo palacio del Retiro y \mkbibemphEl mayor encanto, amorMargaretGreerEl teatro clásico español a través de sus monarcasLucianoGarcía LorenzoMadrid181–202Fundamentos2006Beginnings in Ritual StudiesRonald L.Grimesn.p.n.d.Grove Music OnlineDianaPoultonAntonioCorona AlcaldeMaitines de Navidad, 1652JuanGutiérrez de PadillaMexico CityUrtext (CD UMA-2011)1999México Barroco/Puebla VIINavidad del año de 1657JuanGutiérrez de PadillaMEX-Pc: Leg. 3/1Missa Ego flos campiJuanGutiérrez de PadillaMartynImrieIvanMoodyBrunoTurnerIsle of Lewis, ScotlandMapa Mundi (Vanderbeek & Imbrie)1992Tres cuadernos de Navidad: 1653, 1655 y 1657JuanGutiérrez de PadillaMariantoniaPalaciosAurelioTelloCaracasFundación Vicente Emilio Sojo: Consejo Nacional de la Cultura1998Música de la Catedral de Puebla de los ÁngelesJuanGutiérrez de PadillaSevilleAlmaviva (CD DS-0142)2005Spanish Genre Painting in the Seventeenth CenturyMariannaHaraszti-TakácsBudapestAkadémiai Kiadó1983New Evidence for Musica Ficta: The Cautionary SignDonHarrán29177–981976More Evidence for Cautionary SignsDonHarrán313490–4941978\mkbibquoteMusic Charms the Senses. . .: Devotional Music in the \mkbibemphTriunfos festivos of San Ginés, Madrid, 1656JanetHathaway219–230Libro de Agricultura, que tracta de labrança y criança, y de muchas otras particularidades y prouechos del campoGabriel Alonso deHerreraMedina del Campo1569Baroque Music: Music in Western Europe, 1580–1750John WalterHillNew YorkNorton2005Fire in My Bones: Transcendence and the Holy Spirit in African American GospelGlennHinsonPhiladelphiaUniversity of Pennsylvania Press2000Early Historical Illustrations of West and Central African MusicWalterHirschberg436–181969The Untuning of the Sky: Ideas of Music in English Poetry 1500–1700JohnHollanderPrincetonPrinceton University Press1961Poems of Gerard Manley HopkinsGerard ManleyHopkinsRobertBridgesW. H.GardnerNew YorkOxford University Press1948Juan Gutiérrez de Padilla: El insigne maestro de la catedral de Puebla de los Ángeles (Málaga, c. 1590; Puebla de los Ángeles, 8-IV-1664)NelsonHurtado138–13929–672008Colonial Counterpoint: Music in Early Modern ManilaDavidIrvingOxfordOxford University Press2008Lex orandi, lex credendi: Origins and Meaning; State of the QuestionKevin W.Irwin1157–692002Neither Voice Nor Heart Alone: German Lutheran Theology of Music in the Age of the BaroqueJoyceIrwinNew YorkP. Lang1993Obras completasSanJuan de la CruzJoséVicente RodríguezFedericoRuiz Salvador6MadridEditorial de Espiritualidad2009Primeira parte do index da livraria de mvsica do mvyto alto, e poderoso rey Dom João o IV. nosso senhorPauloCraesbeck[Lisbon?]1649Rhetorical Personification of the Dead in 17th-Century German Funeral Music: Heinrich Schutz's \mkbibemphMusikalische Exequien (1636) and Three Works by Michael Wiedemann (1693)Gregory S.Johnston92186-2131991Renaissance Modal Theory: Theoretical, Compositional, and Editorial PerspectivesCristle CollinsJudd364–406Handbook of Patristic Exegesis: The Bible in Ancient ChristianityCharlesKannengiesserLeidenBrill2004On Liturgical TheologyAidanKavanaghNew YorkPueblo1984Singing Jeremiah: Music and Meaning in Holy WeekRobert L.KendrickBloomington, INIndiana University Press2014Devotion, Piety, and Commemoration: Sacred Songs and OratoriosRobert L.Kendrick324–377Intent and Intertextuality in Barbara Strozzi's Sacred MusicRobert L.Kendrick142002Mvsvrgia vniversalis, sive Ars magna consoni et dissoni in X. libros digestaAthanasiusKircherRome1650Song Migrations: The Case of Adoramoste, SeñorTessKnightonDevotional Music in the Iberian World, 1450–180052–76Towards a History of the Spanish VillancicoPaul R.LairdWarren, MIHarmonie Park Press1997Mémoires d'Olivier de la MarcheOlivier deLa MarcheParis2Librairie Renouard1888Aristotle's Physics and Its Medieval VarietiesHelen S.LangAlbanyState University of New York Press1992Commentaria In Acta Apostolorum Epistolas Canonicas Et ApocalypsinCornelius aLapideAntwerp1627Commentaria in scripturam sacramCornelius aLapideLondonJ. P. Pelagaud\circa1868Commentarium in IV. EvangeliaCornelius aLapideLondon1638Museu de Montserrat: La sorpresa de l'artJosep de C.LaplanaAbadia de MontserratMuseu de Montserrat2011Spanish MetrificationA. RobertLauer2002\footnoteDistinctions among Canaanite, Philistine, and Israelite Lyres, and Their Global Lyrical ContextsBoLawergren30941–681998Spanish Dances: Selections from Ruiz de Ribayaz's \mkbibemphLuz y norteAndrewLawrence-King[Germany]Deutsche Harmonia Mundi CD 05472-77340-21995Boccherini's Body: An Essay in Carnal MusicologyElisabethLe GuinBerkeleyUniversity of California Press2006SermonesPope St.Leo IJ. P.MigneParisMigne1846Patrologia latina54Obras poeticas posthumas: Poesias sagradas, tomo segundoManuel deLeón MarchanteMadrid1733A History of Afro-Hispanic Language: Five Centuries, Five ContinentsJohn M.LipskiCambridgeCambridge University Press2005Breve historia de PueblaLeonardoLomelí VanegasMexico CityEl Colegio de México2001Corresponsales de Miguel de IrízarJoséLópez-Calo18197–2221963La música en la Catedral de SegoviaJoséLópez-CaloSegoviaDiputación Provincial de Segovia1988El Porqué de la Música, en que se contiene los quatro artes de ella, canto llano, canto de organo, contrapvnto, y composicionAndrésLorenteAlcalá de Henares1672Pleasure and Meaning in the Classical SymphonyMelanieLoweBloomingtonIndiana University Press2007Compendio de Dotrina ChristianaFrayLuis de GranadaMadrid1595Compendio y explicacion de la doctrina CristianaFrayLuis de GranadaMadrid1945Biblioteca de autores españoles11Sermon en la fiesta del nascimiento de nuestro señorFrayLuis de GranadaMadrid1945Biblioteca de autores españoles11Introducción del símbolo de la feFrayLuis de GranadaMadridM. Rivadeneyra1871Biblioteca de autores españoles6Historical Handbook of Major Biblical InterpretersDonald K.McKimDowners Grove, ILInterVarsity Press1998Huerto ameno de varias flores de músicaFray AntonioMartín y CollMadrid1609Japanese Travellers in Sixteenth-Century Europe: A Dialogue concerning the Mission of the Japanese Ambassadors to the Roman Curia (1590)DerekMassarellatranslatorJ. F.MoranLondonThe Hakluyt Society/Ashgate2012Juan Gutiérrez de Padilla desde el ámbito civil: Un \mkbibemphcorpus documentalGustavoMauleón Rodríguez179–242The Vox Dei: Communication in the Middle AgesSophiaMenacheNew YorkOxford University Press1990La Catedral Basilica de la Puebla de los ÁngelesEduardoMerlo JuárezJosé AntonioQuintana FernándezMiguelPavón RiveroPuebla de los ÁngelesUniversidad Popular Autónoma del Estado de Puebla2006Breve, e svccinta Relatione del Viaggio nel regno dei Congo nell' Africa MeridionaleGirolamoMerollaNaples1692\mkbibquote\dots de Ángeles también el coro: Estética y simbolismo en la misa \mkbibemphEgo flos campi de Juan Gutiérrez de PadillaRicardoMiranda131–153Divas in the Convent: Nuns, Music, and Defiance in Seventeenth-Century ItalyCraigMonsonChicagoUniversity of Chicago Press2012Calculation of the Ecclesiastical CalendarMarcos J.Montes\footnoteOperas for the Papal Court, 1631-1668MargaretMurata391981Studies in MusicologySinging about Singing, or, The Power of Music Sixty Years AfterMargaretMurataIn cantu et in sermone: For Nino Pirrotta on His 80th BirthdayFirenze363–384Olschki1989Italian Medieval and Renaissance Studies2Anglo-American Postmodernity: Philosophical Perspectives on Science, Religion, and EthicsNanceyMurphyBoulder, COWestview Press1997Catalan Voicing Phenomena: Final Obstruent DevoicingScottMyersMeganCrowhurstPhonology: Case StudiesAustin, TXThe University of Texas at Austin2006\footnoteMétrica española: Reseña histórica y descriptivaTomásNavarro TomásNew YorkLas Americas Pub. Co.1966The Study of Ethnomusicology, Thirty-One Issues and ConceptsBrunoNettl2Urbana, ILUniversity of Illinois Press2005You Will Never Understand This Music: Insiders and Outsiders149–160Practica del catecismo romano, y doctrina christiana, sacada principalmente de los catecismos de Pio V. y Clemente VIII. compuestos conforme al Decreto del santo Concilio Tridentino: Con las divisiones, y adiciones necesarias al cumplimiento de las obligaciones Christianas, para que se pueda leer cada Domingo, y dia de fiestaJuan EusebioNierembergMadrid1640Oculta filosofia de la sympatia y antipatia de las cosas, artificio de la naturaleza, y noticia natural del mundoJuan EusebioNierembergMadrid1633The Street Has Its Masters: Caravaggio and the Socially MarginalTodd P.OlsonCaravaggio: Realism, Rebellion, ReceptionGenevieveWarwickCranbury, NJ69–81Rosemont Publishing2006El Pastor de Noche Buena: Practica Breve de las Virtudes; Conocimiento Facil de los ViciosJuan dePalafox y MendozaBarcelona1730Virtudes del indioJuan dePalafox y MendozaValladolidQuirón Ediciones1998Catàlech de la Biblioteca musical de la Diputació de BarcelonaFelipPedrellBarcelonaPalau de la Diputació1908Aportes metodológicos para una investigación sobre música colonial MexicanaBárbaraPérez Ruízyear 2321–792002Musical Patronage at the Royal Court of France under Charles VII and Louis XI (1422–83)Leeman L.PerkinsJournal of the American Musicological Society37507–661984A Silent Minority: Deaf Education in Spain, 1550–1835SusanPlannBerkeley, CAUniversity of California Press1997Archivo de música sacra de la catedral de PueblaLincoln B.SpiessThomsStanfordMexico City1967La musica en el teatro de CalderonMiguelQuerol GavaldaBarcelonaInstitut del Teatre1981How Early America SoundedRichard CullenRathIthaca, NYCornell University Press2003Relacion historica, y panegyrica de las fiestas, qve la civdad de Zaragoza dispvso, con motivo del decreto, en qve la Santidad de Inocencio XIII. concediò para todo este Arzobispadio, el OFICIO proprio de la APARICION de Nuestra Señora del PILAR, en el de la Dedicacion de los dos Santos Templos del Salvador, y del Pilar \DotsZaragozaAyuntamiento de Zaragoza1990Arte métrica españolaJuan DíazRengifoSalamanca1592Poemas vihuelísticosPepeRey2011\footnoteTransmisión poética y dramatúrgica del dogma en el auto \mkbibemphEl nuevo palacio del Retiro de Calderón: La teología eucarística de las metáforasDominiqueReyre102113–1222008Livraria de música de el-rei D. João IV; estudo musical, histórico e bibliográficoMário de SampayoRibeiroLisbonAcademia Portuguesa da História1967Francisci Riberæ Villacastinensis Presbyteri Societatis Iesu, Doctorisque Theologi, In sacram beati Ioannis Apostoli & Euangelistæ Apocalypsin CommentarijFranciscoRiberaAntwerp1603Commentaria SymbolicaAntonioRicciardo [Antonius Ricciardus]Venice1591El pequeño mundo del hombre: Varia fortuna de una idea en la cultura españolaFranciscoRico2MadridAlianza Editorial1986Power and Religion in Baroque Rome: Barberini Cultural PoliciesPeterRietbergenLeidenBrill2006Guillaume de Machaut and Reims: Context and Meaning in His Musical WorksAnne WaltersRobertsonCambridgeCambridge University Press2002Villancicos and Personal Networks in 17th-Century SpainPablo-LorenzoRodríguez879–891998The Villancico as Music of State in 17th-Century SpainPablo-L.Rodriguez189–198The Origins of Christmas: The State of the QuestionSusan K.Roll273–290\mkbibemphTodo el mundo en general, ecos historiográficos desde Chile de una copla a la Inmaculada Concepción en la primera mitad del siglo XVIIVíctorRondón212009Forma del villancico polifonico desde el siglo XV hasta el XVIIISamuelRubioCuencaInstituto de Musica Religiosa de la Excma. Diputación Provincial de Cuenca1979Parnaso español de madrigales y villancicos a cuatro, cinco y seysPedroRuimonte [Rimonte]PedroCalahorraZaragozaExcma. Diputación Provincial de Zaragoza, Institución \mkbibquoteFernando el Católico (C. S. I. C)1980Luz, y norte musical para caminar por las cifras de la Guitarra Española, y Arpa, tañer, y cantar á compás por canto de Organo; y breue explicacion del Arte, con preceptos faciles, indubitables, y explicados con claras reglas por teorica, y practicaLucasRuiz de RibayazMadrid1677Villancicos (de dos a dieciséis voces)JosephRuiz SamaniegoLuis AntonioGonzález MarínBarcelonaConsejo Superior de Investigaciones Científicas: Institución \mkbibquoteMilà i Fontanals, Dept. de Musicología2001Monumentos de la música española63Music as an \mkbibquoteInstrumentum Regni in Spanish Seventeenth-Century DramaJack W.Sage613384–3901994ZurbaránMaría IsabelSánchez QuevedoTres Cantos, MadridAkal Ediciones2000De missione legatorvm Iaponensium ad Romanam curiam, rebusq; in Europa, ac toto itiner adimaduersis DIALOGVS ex ephemeride ipsorvm legatorvm colectvs, & in sermonem Latinvm versvs ab Eduardo de Sande Sacerdote Societatis IESV.Duarte deSandeMacao1590Instruccion de musica sobre la guitarra española y metodo de sus primeros rudimentos hasta tañerla con destreza: con dos laberintos ingeniosos, variedad de sones [. . .] con vn breve tratado para acompañar con perfeccion sobre la parte muy essencial para la guitarra, arpa y organo [. . .]GasparSanzZaragoza1674Lyra Poética de Vicente Sanchez, natvral de la Imperial Civdad de Zaragoza. Obras PosthvmasVicenteSánchezZaragoza1688A Select Library of the Nicene and Post-Nicene Fathers of the Christian Church. Second SeriesPhilipSchaffEdinburghT&T Clark1894For the Life of the World: Sacraments and OrthodoxyAlexanderSchmemann2Crestwood, NYSt. Vladimir's Seminary Press1973Are You Alone Wise?: The Search for Certainty in the Early Modern EraSusan ElizabethSchreinerOxfordOxford University Press2011Toward a Unitary Field Theory for Musicology102–138New World Symphonies: From Araujo to Zipoli; An A to Z of Latin American BaroqueJeffreySkidmoreLondonHyperion CD A673802003Musicking: The Meanings of Performing and ListeningChristopherSmallHanover, NHUniversity Press of New England1998Let the Church Sing!: Music and Worship in a Black Mississippi CommunityThérèseSmithRochester, NYUniversity of Rochester Press2004Villancicos y letras sacrasSorJuana Inés de la CruzAlfonsoMéndez PlancarteMexico CityInstituto Mexiquense de Cultura, Fondo de Cultura Económica1952Obras completas de Sor Juana Inés de la Cruz2Don Quijote and the “Entremés de los romances”: A RetrospectiveGeoffreyStagg222129–1502002Catálogo de los acervos musicales de las catedrales metropolitanas de México y Puebla y de la Biblioteca Nacional de Antropología e Historia y otras colecciones menoresE. ThomasStanfordMexico CityInstituto Nacional de Antropología e Historia2002Songs of Mortals, Dialogues of the Gods: Music and Theatre in Seventeenth-Century SpainLouise K.SteinOxfordClarendon Press1993Santiago, fray Francisco de (born ca. 1578 at Lisbon; died October 5, 1644, at Seville)Robert MurrellStevenson1–111970Christmas Music from Baroque MexicoRobert MurrellStevensonBerkeleyUniversity of California Press1974The \mkbibquoteDistinguished Maestro of New Spain: Juan Gutiérrez de PadillaRobert MurrellStevenson353363–3731955Sor Juana Inés de la Cruz's Musical Rapports: A Tercentenary RemembranceRobert MurrellStevenson1511–211996The Liturgical Year: Studies, Prospects, ReflectionsRobert F.Taft3–24The Origins of the Liturgical YearThomas J.Talley2Collegeville, MNPueblo (The Liturgical Press)1991El Gongorismo en Nueva España: Ensayo de restituciónMartha LiliaTenorioMexico CityEl Colegio de México2013Poesía novohispana: AntologíaMartha LiliaTenorioMexico CityColegio de México: Fundación para las Letras Mexicanas2010Los villancicos de Sor JuanaMartha LiliaTenorioMexico CityEl Colegio de México1999Reading the Bible with the Dead: What You Can Learn from the History of Exegesis that You Can't Learn from Exegesis AloneJohn L.ThompsonGrand Rapids, MIEerdmans2007Powerhouse for God: Speech, Chant, and Song in an Appalachian Baptist ChurchJeff ToddTitonAustinUniversity of Texas Press1988Music in Renaissance Magic: Toward a Historiography of OthersGaryTomlinsonChicagoUniversity of Chicago Press1993Villancicos de Tomás de Torrejón y VelascoTomás deTorrejón y VelascoOmarMorales AbrilGuatemala CityUniversidad de San Carlos de Guatemala, Centro de Estudios Folklóricos2005El repertorio de la Catedral de GuatemalaCuando un estribillo no es un estribillo: Las formas del villancico en el siglo XVIIÁlvaro JoséTorrente2014¿Cómo se cantaba al \mkbibquotetono de jácara?Álvaro JoséTorrenteLiteratura y música del hampa en los Siglos de OroMadridVisor Libros2014Grove Music OnlineÁlvaro JoséTorrenteThe Historiography of Music: Issues of Past and PresentLeoTreitlerSigns of Imagination, Identity, and Experience: A Peircian Semiotic Theory for MusicThomas R.Turino432221–2551999Pliegos de villancicos en la Hispanic Society of America y la New York Public LibraryÁlvaroTorrenteJanetHathawayKasselEdition Reichenberger2007Pliegos de villancicos en la British Library (Londres) y la University Library (Cambridge)ÁlvaroTorrenteMiguelÁngel MarínKasselEdition Reichenberger2000Historia de la literatura españolaÁngelValbuena Prat2BarcelonaEditorial Gustavo Gili1946Phisica, SpecvlatioAlphonsus aVeracruceMexico City1557Música, propaganda y reforma religiosa en los siglos XVI y XVII: cánticos para la \mkbibquotegente del vulgo (1520–1620)Alfonso deVicente12007\footnoteProcessions for the Dead, the Senses, and Ritual Identity in Colonial MexicoGraysonWagstaffMusic, Sensation, and SensualityLinda PhyllisAusternNew York167–180Routledge2002Critical and Cultural Musicology5Religious Literature in Early Modern SpainAlison P.Weber149–158Biblia Sacra iuxta vulgatam versionemRobertWeberRogerGryson4StuttgartDeutsche Biblegesellschaft1994The Maze and the Warrior: Symbols in Architecture, Theology, and MusicCraigWrightCambridge, MAHarvard University Press2001\mkbibemphVor deinen Thron tret ich and the Art of DyingDavidYearsleyBach and the Meanings of CounterpointCambridge1–41Cambridge University Press2002Towards an Allegorical Interpretation of Buxtehude's Funerary CounterpointsDavidYearsley802183–2061999Conceptualizing Music: Cognitive Structure, Theory, and AnalysisLawrence M.ZbikowskiNew YorkOxford University Press2002Devotional Music in the Iberian World, 1450–1800: The Villancico and Related GenresTessKnightonÁlvaro JoséTorrenteAldershot, UKAshgate2007The Popular, the Sacred, the Colonial and the Local: The Performance of Identities in the Villancicos from Sucre (Bolivia)BernardoIllariDevotional Music in the Iberian World, 1450–1800: The Villancico and Related GenresTessKnightonÁlvaro JoséTorrenteAldershot, UK409–440Ashgate2007Polychoral Culture: Cathedral Music in La Plata (Bolivia), 1680–1730BernardoIllariUniversity of Chicago2001\footnoteThe Sacred Villancico in Early Eighteenth-Century Spain: The Repertory of Salamanca CathedralÁlvaro JoséTorrenteUniversity of Cambridge1997\footnote
    \end{document}
  