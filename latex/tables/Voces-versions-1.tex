% Voces-versions-1.tex
\newcommand{\str}{\hspace{1em}}
\begin{tabular}{lll}
    \toprule
    1642 Lisbon (Santiago?) & 
    1647 Seville (Jalón?) & 
    1657 Puebla (Padilla) \\
    \midrule 
    
    \str{} Voces las de la capilla,    &
    \str{} Cantores \uline{de la Capilla,} &
    \str{} \uline{Voces las de la capilla,} \\

    cuenta con lo que se canta, &
    \uline{cuenta con lo que se canta,} &
    \uline{cuenta con lo que se canta} \\

    que es músico el Rey, y nota &
    \uline{que es Músico el} Niño, \uline{y nota} &
    \uline{que es músico el Rey, y nota} \\

    las más leves disonancias. &
    \uline{las más leves disonancias.} &
    \uline{las más leves disonancias} \\

    & 
    \str{} La música que componó 
    & \\

    & 
    de vozes altas y bajas, 
    & \\

    & 
    a compás mayor las rige, 
    & \\

    & 
    y es proporción abreviada. 
    & \\

 
    \str{} A lo de Jesús infante & 
    \str{} Una \uline{clave} con \uline{tres} tiempos &
    \uline{a lo de Jesús infante} \\

    y a lo de David monarca, & 
    \uline{pone} con \uline{destreza} tanta, &
    \uline{y a lo de David monarca}. \\

    puntos ponen a sus letras,
    & que el pasado y el futuro &
    \str{} \uline{Puntos ponen a sus letras} \\

    los siglos de sus hazañas.
    & al \uline{compás} presente iguala. &
    \uline{los siglos de sus hazañas.} \\

    \str{} La clave, que sobre el hombre
    & \str{} Un Coro errado enmendó &
    \uline{La clave que sobre el} hombro \\

    para el treinta y tres se guarda,
    & con un \uline{medio}, que a la entrada &
    \uline{para el treinta y tres se aguarda.} \\

    años antes la divisa,
    & puso, y una espiración & 
    \str{} \uline{Años antes la divisa,} \\

    la destreza, en la esperanza.
    & que \uline{para el} Calvario \uline{guarda.} & 
    \uline{la destreza en la esperanza} \\

    \str{} Por sol comienza una gloria. & 
    (cf. Seville copla 4) & 
    \uline{por sol comienza una gloria,} \\

    por mi se canta una gracia, &
    &
    \uline{por mi se canta una gracia,} \\

    y a medio compás la noche, &
    &
    \uline{y a medio compás la noche} \\

    remeda quiebros del alba. &
    &
    \uline{remeda quiebros del alba.} \\
  
    & &
    \str{} Puntos ponen a sus letras \\
   
    & & 
    los siglos de sus hazañas. \\

    & & 
    La clave que sobre el hombre \\

    & & 
    para el treinta y tres se aguarda. \\
    \bottomrule
\end{tabular}
 
