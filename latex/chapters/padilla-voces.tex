% padilla-voces.tex

% Cashner, *Faith, Hearing, and the Power of Music*,
% chapter 3: Hearing Christ in the "Voices of the Chapel Choir"
% 
% 2017-05-01      New version for book begun in Montrose
% 2017-08-04      Work resumed in Rochester
% 2017-12-14      Converted to Markdown
% 2018-01-22      Work resumed after ch2 revision
% 2018-02-08      Complete draft 
% 
% 2018-02-26      Begin revision 1
% 2018-04-04      Begin revision 2 after making floats
% 2018-04-10      Submitted to Grupo
% 2018-04-23      Revision 3 after Grupo comments
% 2017-05-09      Revision 3 completed, converted to LaTeX

\part{Listening for Unhearable Music}
\label{part:unhearable-music}

\chapter[Hearing Christ]
{Hearing Christ in the \quoted{Voices of the Chapel Choir} 
(Puebla, 1657)}
\label{ch:padilla-voces}

On Christmas Eve 1657 in colonial Puebla, the cathedral's tower bells had been
ringing for an hour when the first voices of the Christmas liturgy began to sing
at 11 p.m.
Having heard the summons, the cathedral chapter was joined in a space glowing
with luminaries by other clerics, professors, landowners with their slaves, and
common worshippers of every caste, from Spaniards and their descendents down to
indigenous people, enslaved and free Africans, and many degrees of mixture in
between.
Whether they came out of habit or obligation, or in sincere devotion, there was
little else for them to do but listen.
Fortunately the chapter had ensured that its chapelmaster, Juan Gutiérrez de
Padilla, had prepared yet another sumptuous banquet of music, including chant
and Latin-texted polyphony both old and new.
The main attraction for many listeners, though, was the new set of Spanish
villancicos.

As a cantor in Puebla Cathedral was intoning a Latin sermon of Pope Saint Leo
the Great---the first reading in the second Nocturne of Matins---the musical
chapel was preparing to raise their own voices.
Holding handwritten notebooks with their individual performing parts for this
year's new cycle of villancicos, they looked to Father Padilla for their cue.
When the chanting concluded with a descending cadence, Padilla made sure the
chorus had the right starting pitches in their ears: he may have sung an
intonation, or had someone play a chord or short improvised cadence on the
organ.
He raised his hand and indicated the downbeat, the start of the metrical
measure, and on one side of the double-choir ensemble, the three
voice parts (possibly three individual singers) of Chorus I entered on the
second beat, singing the word \wtitle{Voces}.
The boy treble, adolescent alto, and tenor all sang this word high in their
registers, and the soft harmony of the opening G-minor (\emph{mollis}) chord
hung mysteriously in the vaulted space between the columns of the new
cathedral's architectural choir.
Moving in the same rhythm, following the natural accents of the poetry, the
voices declaimed this text like a solemn choral recitative
(\cref{music:Padilla-Voces-opening}):
\begin{quotepoem}
    Voces, las de la capilla,   & Voices of the chapel choir, \\
    cuenta con lo que se canta, & keep count with what is sung,  \\
    que es músico el Rey, y nota & for the King is a musician, and notes \\
    las más leves disonancias   & even the most venial dissonances, \\
    a lo de Jesús infante       & after the manner of Jesus the infant prince \\
    y a lo de David monarca.    & just as in the manner of David the monarch.
\end{quotepoem}
The other choir remained silent for these lines, as their notated parts
instructed them to heed the other chorus's admonition and \quoted{keep count} of
twenty-seven measures of rests until their entrance.

\begin{musicexample}
    \caption{Gutiérrez de Padilla, \wtitle{Voces, las de la capilla}, opening}
    
    \label{music:Padilla-Voces-opening}
   
    \includeWideGraphic{Padilla-Voces-opening}
\end{musicexample}

Some of those closest to the voices had already seen the poem in the published
commemorative pamphlet; some of the seminary professors, perhaps, had even spent
some time with their students puzzling over its complex wordplay.
A few of them recognized the poem from an earlier, but slightly different,
version they had seen in an imported print of villancicos from Seville.
One of them may have even heard the Seville version in person.
These educated worshippers listened closely to hear how the chapelmaster, now
approaching his seventieth birthday and showing signs of age, would demonstrate
his mastery by realizing the poem's musical conceits in actual music.

The rest of the crowd heard and understood less, but recognized the opening
conceit of \emph{voces}, and heard distinctly the phrase \emph{que es músico el
Rey} and the references to David and \emph{Jesús infante}.
As they looked past the choir to the newly decorated Altar of the Kings, filled
with images of Christ's birth and of celestial music, and as they heard the
solemn dialogue between choirs give way to more lively music, they worked to
imagine what sort of music the choir was singing about---the music of King
David with his lyre, the song of the Christmas angels, or the music they were
hearing this night in the heart of New Spain?

Thus begins a tour-de-force of music about music, in which the composer and his
ensemble take a verbal discourse about music and turn it into a musical
discourse about music.
This meta-musical discourse uses musical means to communicate something about
music itself---and here the subject is not only that made by human performance.
When the first chorus refers to \quoted{what is sung}, they are referring to
more than this literal level of human performance.
Instead, they point to \emph{Jesús infante}---the child who is both the
\emph{infante} or heir of the musician-king David and who is also God made
flesh as an infant, a child too young to be able to speak.

The next two chapters (forming part 2) analyze and interpret two families of
villancicos that represent Christ as both singer and song, inviting hearers to
listen for the harmonies between earthly, heavenly, and divine forms of music.
The first is \wtitle{Voces, las de la capilla}, set by Juan Gutiérrez de Padilla
for Puebla Cathedral in 1657; this is one version of a family with evidence for
at least two previous settings.
The second is \wtitle{Suspended, cielos, vuestro dulce canto}, one of the
best-attested families of villancicos, with eight documented settings of variant
poetic texts in the later seventeenth century; the one extant complete musical
setting is by Joan Cererols of the Abbey of Montserrat, around 1660.
These villancicos for Christmas connect incarnation, voice, and creation, as
they invite hearers to contemplate human music-making as a reflection of
Christ's nature as the divine \quoted{Word made flesh} (Jn 1).
Both textual traditions build on an ancient theological trope of Christ as
\emph{Verbum infans}---the infant, or unspeaking Word.
Christ the Word does not need to speak because God is already communicating
himself to humankind through the Christ-child's incarnate body.
The villancicos set by Padilla and Cererols turn this into musical theology by
imagining the baby Jesus not speaking, but singing; and by considering Christ
himself as the song being sung.

In this chapter we will listen closely to the words and music of Padilla's
\wtitle{Voces, las de la capilla} to understand how the Puebla chapel choir put
their beliefs about music's power into practice through musical performance, and
to learn what this piece tells us about how Hispanic Catholics listened to
music.
This villancico and other metamusical pieces related to it, I argue, trained
worshippers to listen past human music-making, to strain to hear the unhearable
higher music of the divine, a music that defies human imagining.
By representing the trope of Christ as singer and song through this genre of
sung poetry, the piece challenges listeners to hear the divine voice through the
voices of the chorus.

Within a Neoplatonic theological tradition, these pieces connected faith and
hearing by making Christ the Word audible through poetic and musical structures.
In a sense they \quoted{incarnated} the poetry and give it material form through
musical performance in a specific place and time.
At the same time they point beyond sounding music to higher forms of music.
In the terms of Boethius, they use \emph{musica instrumentalis} (musical
performance) as a way of tuning \emph{musica humana} (the harmony of body and
soul, and of people in society) in accord with the order of nature, reflected in
\emph{musica mundana} (the music of the cosmic spheres).%
    \Autocite{Boethius:Musica}
Even beyond these three levels of music, metamusical pieces point to the
ethereal harmonies of supernatural Heaven (\emph{cielo Empyreo}, the Empyrean),
including the chorus of saints and angels, the mysterious unity of the Trinity,
and the dual nature of Christ as both human and divine.

Catholics believed that singing and listening to singing in this way could unite
the human community in harmony with each other and with Christ, through the
ritual of the church.
The ingenuity of the structures suggests that the creators and performers of
these villancicos sought primarily not to teach doctrine, but to promote
doxology---glorifying God through contemplative devotion.
They invite listeners to respond to the mystery of the Incarnation in awe,
wonder, and adoration. 
They do this by first inspiring listeners to marvel at the musical virtuosity of
the \quoted{voices of the chapel choir}.

This chapter is a detailed study of the poetry, music, and theological context
of one villancico---Padilla' \wtitle{Voces, las de la capilla}---and to a lesser
degree, two other texts related to it.
Some might question whether this level of attention is appropriate for a genre
that many scholars have assumed functioned as little more than secular
entertainment in church, a parade of stock minor-theater characters and
dogmatic clichés.
The more villancicos one comes to know in depth, however, the less that
stereotype will seem to apply.
There are just as many sophisticated villancicos as there are silly ones, and
they did not serve the same functions.
On the contrary, this anonymous poem and Padilla's musical setting both demand
and reward detailed analysis.

In fact, that kind of close study is an extension of the same kind of
contemplative listening practice the piece was designed to inculcate.
The poem is cryptic even by seventeenth-century standards, but it was printed
and disseminated publicly, with the intent to communicate with some audience of
readers.
Most likely, the poem was written specifically to be set by music, and was
designed to give a composer as many musical concepts to play with as possible.
Likewise, much of Padilla's musical ingenuity would have only registered with
the most well-trained listeners, but the piece was performed as part of a public
liturgy that probably drew a large and varied congregation.
Many people heard the piece, and even if many were puzzled by it, someone
understood it---and it is possible to understand it today if we take the effort
to listen closely.
Modern people, though, cannot count on sharing the same conventions and
understandings that the poet and composer took for granted in their readers and
hearers, and so the task of the modern interpreter is also to recover the lost
context on which the meaning of this performative text depended in its time.

Theology was a major intellectual pursuit of the Spanish and New Spanish elite,
and as such it was a creative activity---not merely reciting dogmas
approved by the church, but playfully seeking out ever-new ways of connecting
revealed truth to observed experience.
Thinking theologically in an early modern Catholic sense meant building 
endless chains of association and allusion among Biblical texts, writings of
church fathers (patristics), medieval theologians, and the liturgy. 
It meant interpreting new texts in light of these old ones, and reinterpeting
the old ones in light of the new.
And it grew out of and reinforced a view of the world as a book waiting to be
read (see \cref{ch:intro}).
One had to apply oneself to the effort of discerning how the sacred was
imminent in the mundane and common.%
    \Autocite{Chavez:DistortingReality}
% XXX cite diss when available 

It is the central argument of this book that devotional music provided Spanish
Catholics with a way of performing theology: making and hearing music was a
creative pursuit in which people sought to forge connections to God and to each
other through musical structures.
These same Spanish intellectuals who studied Augustine and Aquinas also learned
the fundamentals of music on both theoretical and practical levels.
They had read Boethius on the three levels of music described above, and they
had some working knowledge of music notation---pitches and staff notation,
rhythm and meter, and at least the general concept of consonance and dissonance
in counterpoint.
Those with more performing experience had been trained as boys in solmization in
the tradition of Guido of Arezzo: they could sing the notes of the scale from
\emph{ut--re--mi} up to \emph{la} on the three different hexachords on C
(natural), F (soft/\emph{mollis}), and G (hard/\emph{durus}); they could name
every note that could be sung by these syllables as laid out in an imagined
spiral on the upraised palm of the Guidonian hand, and they could go
\quoted{outside the hand} for sharp and flat notes using \emph{musica ficta}.%
    \Autocite{Cohen:NotesMiddleAges}

Metamusical villancicos brought these two domains of knowledge into a mutually
illuminating relationship.
Even in simply reading the poetic texts of these pieces, or hearing them read,
a person must know a fair amount of music theory in order to understand the
theological concepts, and vice versa.
When someone performed the musical setting of this kind of text, or heard it
performed, they faced an even greater challenge to understand the words as
projected through the music and perceive the ways the music depicted the sense
and affect of the text. 

The reward for taking this music seriously today as a source for historical
theology about music is a richer understanding of the intellectual culture of
imperial Spain and a holistic sense of how devotional music served theological
and social functions in Hispanic communities, even creating relationships across
the Atlantic between poets, musicians, and institutions in the mainland and in
the colonial viceroyalties.
In the colonial context of seventeenth-century Puebla, villancicos like this
catered to the religious and social demands of a highly literate and
theologically adept circle of listeners, while also making a broader appeal to
the larger public.
As the case studies in \cref{part:unhearable-music} will demonstrate, Spanish
chapelmasters used metamusical villancicos socially to prove their craft as
master musicians.
In this way they established links of kinship to teachers and fellow musicians
who set the same or similar texts and developed the same kinds of
musical-theological tropes.
On the artistic level composers developed tropes for using music to represent
itself, vying with each other for the most overt, symbolically meaningful, and
moving displays of musical artifice.
As part of a theological tradition, the pieces on themes of heavenly music
manifest changing ways of thinking about the relationship between earthly music
and heavenly or divine music, in the midst of shifting early modern
understandings of the cosmos, the human body, and society.
Even as those conceptions were changing, villancicos of the Spanish Empire
continued to promote and embody a Neoplatonic listening practice in which music
points beyond itself to a higher, unhearable music of heavenly truth.


\section[\quoted{Voices of the Chapel Choir}]
{\quoted{Voices of the Chapel Choir} and the \quoted{Unspeaking Word}}

Villancicos circulated in print and manuscript as poems, independently of their
musical settings; and in the case of villancicos on the subject of music, it is
necessary to understand the discourse about music that is presented in the poem
alone before we can recognize the ways the composer has added meaning through
actual music.
The author of \wtitle{Voces, las de la capilla}, as is the case for most
villancicos, is unknown.
As we will see there are multiple versions of this textual tradition, but since
are primary goal is to understand Padilla's setting from 1657 Puebla, it will be
most productive to concentrate on the version of the poem that Padilla set to
music.

The poem is so elaborately contrived that it may seem completely unintelligible
at the first encounter (\cref{poem:Voces-Padilla}).%
    \Autocites
    [37--38, 119--132]{Cashner:WLSCM32}
    [133--203]{Cashner:PhD}
The piece demands a high level of intellectual engagement to tease out the
intricate conceit, and may thus be compared with what Bernardo Illari describes
as \quoted{enigma} villancicos.%
    \Autocite[vol. 2, 304--308]{Illari:Polychoral}
Part of its difficulty comes from the influence of the poetry of Luis de Góngora
(1561--1627), who cultivated a new aesthetic that emphasized learned artifice
and highly wrought dramatic effects, a style referred to as \emph{barroco} by
Spanish literary scholars.%
    \Autocites
    [222--235]{Gaylord:Poetry}
    [vol. 1, 1014--1061]{Valbuena:Literatura}
Góngora's role in Spanish literature is similar to that of Giambattista Marino
in Italian letters, with the difference being that Góngora had a much more
widespread global influence, especially in New Spain.%
    \Autocite{Tenorio:Gongorismo}
Poets writing after the manner of Góngora reveled in arcane plays on words,
contorted Latinate syntax, and multiple levels of meanings.
This poem exemplifies the tradition of Spanish \emph{conceptismo}, in which the
poet creates a sustained analogy between at least two different things such that
the understanding of each one informs the other.
Here the two elements in the conceit are music---specifically the voices of
choral singing---and the Incarnate Christ at his birth.

\begin{poemexample}
    \caption{\wtitle{Voces, las de la capilla}, from setting by Juan Gutiérrez
    de Padilla, Puebla, 1657 (\sig{MEX-PC}{Leg. 3/3})}
    
    \label{poem:Voces-Padilla}
   
    \inputpoem{Voces-Padilla}
\end{poemexample}

The first of the two coplas provides a clear example of this technique:
\begin{quotepoem}
    Daba un niño peregrino       & A baby gave a wandering song \\
    tono al hombre y subió tanto & to the Man, and ascended so high \\
    que en sustenidos de llanto  & that in sustained weeping \\
    dió octava arriba en un trino. 
    & he went up the eighth \add{day} into the triune.
\end{quotepoem}
One can read this strophe solely on the one plane referring to Christ's
Incarnation and Passion, however elliptically.
In the first copla, the Christ-child gave a \quoted{wandering song} to the
Man---referring to the first man, Adam, being cast out of Paradise.
Christ \quoted{went up so high} in \quoted{sustained tones of
weeping}---suffering on the cross for human redemption.
The poem says Christ \quoted{arose on the eighth} day, a traditional way of
referring to the the Resurrection on Easter Sunday.
He ascended \quoted{into the triune}, the Godhead of three persons in one
being.%
\begin{Footnote} 
    See the entries for the numbers eight and three in
    \autocite{Bongo:NumerorumMysteria} and
    \autocite{Ricciardo:CommentariaSymbolica}.
\end{Footnote}
Reading this copla according to the other side of the conceit, the strophe
describes a musical performance: the child intoned the \emph{tonus peregrinus}
chant formula, and, as a virtuoso singer, \quoted{he went up so high} that
\quoted{in a cry of sharps}, he \quoted{went up the octave in a trill}.

The poet has selected musical terms with double meanings that allow listeners
with musical knowledge to think about theological concepts in a new way, and
vice versa.
For example, the words \emph{peregrino tono} could have called up for educated
listeners a tradition of using the chant \emph{tonus peregrinus} to symbolize
the expectant wandering of sinful humanity waiting for the coming of Christ, as
well as concepts of the Christian life as a pilgrimage.%
\begin{Footnote}
    This trope was developed through medieval sources like the allegorical
    commentary on the liturgy of Guillelmus Durandus, whose works were available
    in Puebla: \autocite[234]{Wright:Maze}.
\end{Footnote}
The seventeenth-century Biblical interpreter Cornelius à Lapide comments that
Christ was born like a \quoted{pilgrim} \addorig{peregrinus} on a journey in
a borrowed stable, \quoted{in order to teach us to be pilgrims on earth, though
actually citizens of heaven}.
    \Autocites
    [884, on Jn 1:14:
    \quoted{Hoc est tentorium vel tabernaculum fixit in nobis, id est, inter
    nos, ad modicum tempus, quasi hospes et peregrinus in terra aliena: erat
    enim ipse civis, incola et dominus coeli ac paradisi}.]
    {Lapide:Gospels19C}
    [669, on Lk 2:5: 
    \quoted{ut doceret nos in terra esse peregrinos, cives vero coeli, ut ab hoc
    exilio magnis virtutum passibus tendamis in coelum, ceu patriam et civitatem
    nostram}.] 
    {Lapide:Gospels19C}
The composer and theorist Andrés Lorente in his 1672 music treatise takes up the
\quoted{pilgrim song} trope as a moral exhortation to aspiring musicians.
The musician of virtue, he says, should match the music of his compositions with
\quoted{the spiritual Music of his person, cleansing his conscience, and
rejoicing his soul with Divine Music, so that he may say with David \add{Vulgate
Ps 118:54}, \quoted{Your right precepts have served as songs for me in the place
of my wanderings \addorig{in loco peregrinationis meae}}}.%
    \Autocite
    [609: \quoted{la Musica espiritual de su persona, limpiando su conciencia, y
    alegrando su Anima con Musica diuina, para que pueda dezir con Dauid,
    \emph{Cantabiles mihi erant iustificationes tuae in loco peregrinationis
    meae}}.] 
    {Lorente:Porque}
Within this tradition, then, the villancico poem uses the name of the chant tone
to present Christ himself as the song given to sinful Man in his pilgrimage. 
In the musical setting, Padilla quotes the chant formula literally, so that the
symbol is present to the ear in both word and tone.

With this preliminary understanding of the poetic technique and its rich
symbolic potential, we may recognize that the central conceit of the poem is to
link the \quoted{voices of the chapel choir} with a higher, theological Music
with a capital M, namely, the Christ-child himself.
The poem evokes the musical voices of human singers, angelic choirs, ancient
prophets, and even the crying voice of the newborn Christ.
The overall conceit is clearest in the second copla: Christ is a divine musical
\quoted{composition}, in which the divine chapelmaster \quoted{proves} his
mastery at creating \quoted{consonances of a Man and God}:
\begin{quotepoem}
    Hizo alto en lo divino      & From on high in divinity, \\
    y de la máxima y breve      & of the greatest and least \\
    composición en que pruebe   & he made a composition in which to prove \\
    de un hombre y Dios consonancias. & the consonances of a Man and God. \\
\end{quotepoem}
Christ, the poem says, is a composition through which the divine chapelmaster
proves his craft.
Like Spanish composers who established their superior musicianship in
competition with other applicants for a position through the audition process
known as \emph{oposición}, God demonstrates his mastery by creating concord
between opposed elements.
Christ brings together infinite and finite (\quoted{maxima and breve}), and
creates a consonance to restore the discordant relationship between sinful Man
and the holy God by reconciling both in his own body.

The poem begins with the image of a \quoted{chapel}---that is, a musical
ensemble---performing before the king, like the Spanish \emph{Capilla Real}
(quoted at the beginning of the chapter).
This king \quoted{is a musician}, listening carefully for any defect in the
composition or performance: \quoted{he notes even the most venial dissonances}.
On the theological side of the conceit, who exactly is this king, who listens so
carefully to the chapel choir's voices?
The poet explicitly connects the king to \quoted{David the monarch}, the paragon
of Biblical musicians as both the traditional author of the psalms and as the
founder of the first musical ensemble for worship in the ancient Hebrew temple
(1 Chr 25).
The poem explicitly identifies Jesus as the royal \emph{infante}---heir to the
throne.
As a human, Jesus was the Messiah, heir to David's throne; as divine, he was the
Son of God, second person of the Trinity.
The phrase \emph{a lo de} (in the manner or style of) suggests that this child
will be no less exacting a musical taskmaster than his ancestor.

It only becomes clear later in the poem that the word \emph{infante} also points
to another theological trope based on the double meaning of the Latin word
\emph{infans} as both \quoted{infant} and \quoted{unable to speak}.
In this tradition, the Christ-child is \emph{Verbum infans}---the
\quoted{unspeaking Word}, who does not need to speak because he himself
\emph{is} the Word.
This villancico's conceit treats both the Word and the child in musical terms,
so that Christ as the incarnate Word is a musical composition.
The child, then is depicted not as speaking but, through his cries, as
singing---making Christ both singer and song.

Through this musical metaphor, the poem uses a puzzling series of music terms
that are also Biblical allusions to present Christ as the theological
fulfillment of the prophecies made to and through David, especially the psalms:
\begin{quotepoem}
    Puntos ponen a sus letras          & The centuries of his heroic exploits \\
    los siglos de sus hazañas.         & are putting notes to his lyrics. \\
    La clave que sobre el hombro       & The key that upon his shoulder \\
    para el treinta y tres se aguarda. & awaits the thirty-three.
\end{quotepoem}
In musical terms, Christ's life \quoted{is putting notes to his lyrics}
(\poemlines{8--9}), and thus his life is recounted with the technical vocabulary
for describing a musical composition or performance.
Theologically, God had promised to David an heir to sit on his throne forever
and deliver his people (2 Sm 7), and through Isaiah the prophet he renewed this
promise by saying that a child would be born \quoted{upon whose shoulder} would
rest the \quoted{key} of divine, eternal authority (Is 22:22).
As Biblical interpreters of the time all agreed, the complete fulfillment of
these prophecies, the culmination of all God's \quoted{centuries of heroic
exploits} (\poemline{8}) came not at Christ's birth, but at his death and
resurrection, traditionally thirty-three years (and three months and three days)
later.%
\begin{Footnote}
    \Autocite
    [17: \quoted{Christus enim vixit 33 annos et tres meses, qui a natali ejus
    exerunt usque ad Pascha}.]
    {Lapide:Gospels19C}
    See also the entries for the number thirty-three in
    \autocite{Ricciardo:CommentariaSymbolica}, and
    \autocite{Bongo:NumerorumMysteria}.
    Padilla's 1628 villancico \wtitle{A que\dots{} el juego es visto admirable}
    uses \emph{treinta y tres} (as the amount of a bet in a card game) to refer
    to Christ's passion: \autocite{Cashner:Cards}.
\end{Footnote}
In musical terms, the words of David and the prophets are just the lyrics;
Christ's life is the song.
The key of authority---\emph{clave}---is the same word for clef; and it awaits
\quoted{the thirty-three}, suggesting some kind of musical measure.

The second section of the introduction (\poemlines{11--16}) depicts the moment
of Christ's birth as a musical performance, in an extraordinarily cryptic
passage combining music terms with theological as well as heraldic references:
\begin{quotepoem}
    Años antes la divisa,        & Years before the sign, \\
    la destreza en la esperanza, & dexterity in hope, \\
    por sol comienza una gloria, 
    & with the sun \add{on \emph{sol}} a \quoted{glory} begins, \\
    por mi se canta una gracia,  
    & upon me \add{\emph{mi}} a \quoted{grace} is sung, \\
    y a medio compás la noche    & and at the half-measure, the night \\
    remeda quiebros del alba.    & imitates the trills of the dawn.
\end{quotepoem}
Christ was born, the poem says, \quoted{years before the sign, \quoted{dexterity
in hope}} (\emph{años antes la divisa/ la destreza en la esperanza}).
A \emph{divisa} could be a sign of any kind but typically meant a heraldic
device or motto, such as would appear on a crest or flag.%
    \Autocite
    [\sv{divisa}: \quoted{La señal que el cauallero trae para ser conocido
    \Dots{}. Y deuisa tanto quiere dezir como heredad que viene al hombre de
    parte de su padre, o de su madre, o de sus abuelos, \et{}c}.]
    {Covarrubias:Tesoro}
Theologically the sign may refer at one level to Christ's death on the cross and
on another level to Christ himself.
This is similar to the way the poet has just used \quoted{the thirty-three} to
stand for Christ's passion, and thus to connect Christ's birth to his death.
The infancy narrative in the Gospel of Luke (in the Latin Vulgate) uses the word
\emph{signum} twice: both about Christ, the first referring to his birth and the
second to his death.
The angel who appears to the shepherds at Christ's birth says, \quoted{And this
shall be a sign for you: you will see a baby wrapped in swaddling clothes and
lying in a manger} (Lk 2:12).
When Christ's mother presents him at the temple, the prophet Simeon tells Mary
that the child will be \quoted{a sign that will be opposed} (Lk 2:34).
In an influential twelfth-century Christmas sermon, Saint Bernard of Clairvaux
interprets these two passages together to say that Christ himself is a sign
(\emph{signum}).%
    \Autocite[Sermo 4, 126C]{Bernard:Nativitate}

If the \emph{divisa} means a heraldic device, then \emph{la destreza en la
esperanza} would be the text of the motto.
\quoted{Dexterity in hope} sounds like a phrase from the Roman historian
Tacitus, \emph{spes in virtute, salus ex victoria}; the phrase describes a
desperate moment in Caesar's battle against the Germanic tribes on the Elbe, in
which \quoted{valour was their only hope, victory their only safety}.%
\begin{Footnote}
    Tacitus, \wtitle{Annals} II:20, translation from
    \autocite[49]{Tacitus:Annales-English}.
\end{Footnote}
This \emph{divisa} would be fitting for Christ in his struggle to save humanity.
The vocabulary here (\emph{hazañas}, \emph{destreza}, \emph{divisa}) marks
Christ as a heroic warrior-king in a style that resonated with the
military-influenced culture of early modern Spain. 

Here again he follows after his ancestor David the giant-killer (1 Sm 16).%
\begin{Footnote}
    The preface to \autocite{Azevedo:Catecismo} uses the same kind of language
    to compare Charles I of Spain to King David in his battles to protect faith
    against heresy. 
    Azevedo says of the Creed, \quoted{This is the sign and standard that we who
    are of the Lord, and vassals of the faith, are to bear} (\quoted{Y
    particularmente en la declaracion del Simbolo, que la Iglesia nos manda
    creer, que es la señal y diuisa, que los que somos del Señor, y vassallos de
    la Fe, emos de traer}).
\end{Footnote}
This description also fits with the performance context of this villancico among
the liturgical lessons of Christmas Matins, since the sermons of Leo the Great
characterize Christ's birth as the beginning of a battle with the devil.
The word \emph{destreza} was used for musical heroes as well, to signify
virtuosity, especially compositional ingenuity.%
\begin{Footnote}
    \Autocite
    [\sv{destreza}: \quoted{La agilidad con que se haze alguna cosa,
    atribuyendolo a la mano diestra}]
    {Covarrubias:Tesoro}
    The musical sense may be seen in the title of the guitar manual,
    \autocite{Sanz:Guitarra}.
    Padilla used \emph{destreza} and \emph{hazañas} together to characterize the
    baby Jesus as a heroic rogue in two other villancicos in the \emph{jácara}
    subgenre, in the cycles for Christmas 1651 (\sig{MEX-Pc}{Leg. 1/2}) and 1659
    (poetry imprint only, \sig{US-BL}{PQ7296.A1V8}).
\end{Footnote}
On the musical side of the conceit, the \emph{divisa} could be a musical sign as
well such as a meter signature.

In the next lines we begin to hear this musician's song, and the theological and
musical come too close to cleanly separate.
The solmization syllables \emph{sol} and \emph{mi} here indicate musical
pitches, as well as the symbolic puns on \quoted{sun} and \quoted{me}.
The \quoted{Gloria} of the angelic choirs begins \quoted{on \emph{sol}}, as many
Gregorian \emph{Gloria in excelsis} chants do. 
Their music also begins \quoted{with the sun}: this refers both to the tradition
that Christ was born at midnight, \quoted{at the half-measure of night}
(\poemline{15}, a traditional interpretation of Ws 18:14--15).%
    \Autocite
    [37: \quoted{Era la media noche, mas clara que el mediodía}.]
    {LuisdeGranada:Xmas}
The sun is also a symbol of Christ's royalty, the same one used by Spain's own
king Philip IV.%
\begin{Footnote}
    For the light imagery, see also Is 60:1, Jn 1:14, and Ti 2:11, all featured
    in the liturgies of Christmas.
\end{Footnote}

The linked terms \emph{grace} and \emph{glory} in these verses also refer to
a pervasive tradition in theological literature such as Christmas sermons and
commentaries on the Gospel infancy narratives.
This tradition links the \emph{Gloria} of the Christmas angels to the grace
offered to humanity through Christ's Incarnation, and to the glory awaiting the
redeemed when they join the angels in heaven.
As summarized by the interpreter Cornelius á Lapide, \quoted{Grace, therefore,
is the seed of glory, and in turn glory is the consummation of grace}.%
    \Autocites
    [878, on Jn 1:4:
    \quoted{In verbo quasi in fonte et cause primiginia erat vita nostra
    supernaturalis, puta gratiae et gloriae, ideoque ut hanc vitam nobis
    impetraret, incarnatus est et factus homo, ut initio dixi.  Supernaturalis
    enim vita est duplex: inchoata per gratiam, qua homo justus per fidem, spem
    et charitatem servit Deo, vivitque vitam supernaturalem, in Deum
    supernaturaliter credendo, sperando, eumque super omnia amando: altera vita
    supernaturalis est consummata per gloriam, qua Beati Deo fruntur,
    deliciantur et beantur in aeternum}.]
    {Lapide:Gospels19C}
    [Cf.] [98, Sermo 185, In Natali Domini 2, in connection with Rom 5:1--2]
    {Augustine:SermonesPL}

Thus in the poem, as \quoted{glory is sung}, a \quoted{grace begins}: Christ's
birth is the beginning of God's decisive action to redeem humanity, to extend
his grace to them and elevate them to share in his glory.
The \quoted{thirty-three} marks the completion of that saving work in Christ's
death and resurrection.

The introduction and response have used musical terms with theological
connotations, and vice versa, to establish the concept of Christ as a
musician-king, heir to David.
The voices in this section have been those of the chapel choir singing in the
present about Christ's birth, and the voices of David and the prophets pointing
to Christ the \emph{infante}, who will fulfill their words by \emph{being} the
Word.
Christ as son of David and son of God (that is, both human and divine) will
complete God's heroic work and save humanity through his life and death.

Now, in the estribillo (\poemlines{17--33}), the poem evokes the musical voices
at the moment of Christ's birth:
\begin{quotepoem}
    Y a trechos las distancias      & And from afar, the intervals \\
    en uno y otro coro,             & in one choir and then the other, \\
    grave, suave y sonoro,          & solemn, mild, and resonant, \\
    hombres y brutos y Dios,        & men, animals, and God, \\
    tres a tres y dos a dos,        & three by three and two by two, \\
    uno a uno,                      & one by one, \\
    y aguardan tiempo oportuno,     & they all await the opportune time, \\
    quién antes del tiempo fue.     & the one who was before all time. \\
    Por el signo a la mi re,        & Upon the sign of \emph{A (la, mi, re)} \\
    puestos los ojos en mi,         & with eyes placed on me \add{\emph{mi}} \\
    a la voz del padre oí           & at the voice of the Father I heard \\
    cantar por puntos de llanto.    & singing in tones of weeping--- \\
    ¡O qué canto!                   & Oh, what a song! \\
    tan de oír y de admirar,        & as much to hear as to admire, \\
    tan de admirar y de oír.        & as much to admire as to hear! \\
    Todo en el hombre es subir      & Everything in Man is to ascend \\
    y todo en Dios es bajar.        & and everything in God is to descend.
\end{quotepoem}
Here the musical terms are used less as metaphors for theological concepts (as
was the case for \emph{divisa} and \emph{destreza}) and more to imagine actual
music-making.
In other words, \emph{destreza} referred to Christ's virtuosity in
\quoted{putting notes to \add{David's} lyrics}, but all this was a way of
referring to Christ's saving life and death as the fulfillment of prophecy.
In the estribillo, by contrast, the text uses musical terms to describe musical
performance: it brings together the whole creation in praise of Christ, panning
down from the celestial music of the spheres (\emph{las distancias} or
intervals, a technical term in both astronomy and music), to the polychoral
ensemble of \quoted{men and beasts} (\emph{hombres y brutos}, l. 20) joining the
angels.
Like all poetry about music, this representation of music draws on the poet's
experience of actual, contemporary music---thus the spheres sing \quoted{in one
choir and the other} (\emph{en uno y otro coro}, \poemline{18}), like Spain's
polychoral ensembles.

The numbers here (\quoted{three by three, two by two, one by one}) at the most
literal level would seem to refer to the number of voices in a musical texture,
and indeed Padilla picks up on this cue in arranging the voices for his musical
setting of this passage.
These numbers, like \quoted{the thirty-three}, also have theological
significance in the traditions of interpretation around Christ's nativity, with
a wide range of possible meanings.
\quoted{Two by two} would seem to be a reference to the animals on Noah's Ark
(Gn 6:9), here referring not only to the animals in the stable, but also to the
whole scene as a picture of the Christian church, a symbolic connection going
back to the first century (1 Pt 3:18--22).%
    \Autocite[15:25]{Augustine:CivitateDei}
\quoted{Three by three} probably refers to the traditional nine ranks of angelic
choirs.%
    \begin{Footnote}
        See the entry for the number nine in \autocite{Bongo:NumerorumMysteria};
        an example of the trope of the nine-rank angelic choir is the angelic
        canon for nine choirs of angels on the frontispiece of
        \autocite{Kircher:Musurgia}.
    \end{Footnote}
\quoted{One by one} could refer to humans or to Christ himself, particularly his
union of divine and human in a single body.
It is also possible that these lines form a chiastic or ring structure, such
that \quoted{three by three} refers to the triune God, \quoted{two by two}
refers to animals, and \quoted{one by one} refers to humans, who must enter the
kingdom of God single-file:
\begin{center}
    \begin{tabular}{ll}
        \toprule
        Who?    & How Many? \\
        \midrule
        Dios    & tres a tres \\
        brutos  & dos a dos \\
        hombres & uno a uno \\
        \bottomrule
    \end{tabular}
\end{center}

Thus far the poet has directed the listener's ear from attending to the chapel
choir singing in the present, to the ancient temple choir of David and the
voices of prophets through the centuries, and up to the moment when the angels
led the song of Gloria at Christ's birth.
But all these voices, the poem now says, have been \quoted{awaiting the
opportune time, the one who was before all time} (\poemlines{24--25}).  
The true music of Christmas is Christ himself, and thus the next lines represent
the voice of the baby Jesus.
The musical imagery here continues the conceit of the King as musician, a
\emph{padre} (father, priest) like most Spanish chapelmasters including Padilla.
He either sounds the pitch \emph{A (la, mi, re)} in Guidonian solmization with
his voice as a tuning note (the same one used today), or sings an intonation on
that note as a cantor would do.

In this section, the poem emphasizes the act of hearing voices through its
grammatical structure.
In the first fifteen lines of the estribillo there are only two simple verbs
that are not participles or part of a dependent clause: \emph{aguardan} in
line~23 and \emph{oí} (I heard) in line~27.  
The first, \emph{aguardan}, follows six verses describing the spheres, angels,
men, and beasts, who all \quoted{await} the time of Christ's birth.
After this, three more verses build up to \emph{a la voz del padre oí} (at the
voice of the Father I heard).
This is the only use of the first person in the poem.
From the opening invocation of \emph{voces}, the reader has to wait all this
time for a simple reference to hearing.
The word \emph{oí} makes the reader a hearer, and situates the act of listening
imaginatively in the middle of the nativity scene, surrounded by the chorus of
creation.

What did the speaker hear? 
Singing, he says, in a song (\emph{cantar por puntos de llanto/ O qúe canto},
\poemlines{29--30}).
\emph{Cantar} can mean both the act of singing and the thing sung, surely a
purposeful ambiguity.
But this is not the song of the creation chorus; rather this is what they were
awaiting---the voice of Christ.
That voice sings in \emph{puntos de llanto} (tones of weeping); musically this
seems to play on \emph{puntos de canto llano} (plainchant), while theologically
the reference to tears, like other elements of the poem, points to Christ's
suffering.
This connects Christ's cries at his birth with the \quoted{loud voice} with
which he died (Mk 15:37).
Another translation of the contorted syntax here might suggest that the poetic
speaker actually \quoted{heard the voice of the Father singing}, that is,
\emph{through} the voice of the child.
This could also be a reference to the heavenly voice heard at Christ's baptism.%
    \footnote{Mt 3:13--17; Mk 1:9--11; Lk 3:21--22.}
This song is \quoted{as much to be seen (or admired) as to be heard}---because
the song and the singer are one and the same.

The song at the center of this poem's concentric circles of voices is the
Christ-child himself.
If the song is Christ, then, \quoted{the sign of \emph{A (la, mi, re)}} is not
just a reference to musical tuning; it connects to the \emph{divisa} of the
introduction to present Christ himself as a sign.
The use of the letter A here calls up rich theological resonances, since Christ
uses the first and last letters of the Greek alphabet to describe himself in the
Revelation to John: \quoted{I am the Alpha and the Omega, the first and the
last, the beginning and the end} (Rv 22:13).%
    \footnote{See also Rv 1:8, 21:6.}
The Latin Vulgate, following the original Greek, simply uses the Greek letter
$\alpha$ here, not spelled out.
In Catholic theology, Christ is \quoted{first} and \quoted{the beginning}
because, as the Nicene Creed declares, he was \quoted{begotten of the father
before all ages}.%
    \Autocite
    [42: \quoted{ex patre natum ante omnia secula}.]
    {Catholic:Catechismus1614}
Thus in the villancico the reference to the \quoted{sign of A} follows the
description of Christ as \quoted{the one who was before all time}
(\poemline{24}).
The time symbolism works on a musical level as well, but the \quoted{sign of A}
also suggests that the musical pitch A is meant to represent Christ himself.
As will be argued below, the Spanish phoneme \emph{a} (pronounced \emph{ah}) may
also evoke the wordless cry of the baby Jesus as a form of music.

The estribillo concludes with the couplet, \quoted{Everything in Man is to
ascend/ and everything in God is to descend} (\poemlines{32--33}).  
Because the estribillo is repeated after the coplas, this line also ends the
whole text in performance.
These verses uses the musical structure of rising and falling musical lines (in
modern theoretical terms, a voice exchange) to epitomize the theology of
incarnation as an exchange between God and humanity.
This concept was repeated in every theological text on Christ's birth.
As Lapide's commentary expresses the idea, Christ \quoted{lowered himself to the
earth and flesh, in order to lift us up to heaven. \quoted{Therefore}, says
Saint Anselm, \quoted{God was made man, in order that man might be made God}.}%
    \Autocite
    [670, on Lk 2:7: \quoted{Depressit se in terram et carnem, ut nos eveheret
    in coelum. \emph{Ideo}, ait S.  Anselm, \emph{Deus factus est homo, ut homo
    fieret Deus}}.]
    {Lapide:Gospels19C}
To sum up this reading of the poem, then, the villancico began by drawing
listeners' attention to the voices of Christmas, and exhorting the singing
voices of the chapel choir to take note of their own singing while also
listening for \quoted{what is sung} on a higher level.
The piece connects Christ and David as musician-kings, with Christ as the song
that puts the prophetic \quoted{lyrics} of David and other Scriptural authors to
music.
After long waiting, at the \quoted{opportune time}, Christ was born into the
world to begin a battle \quoted{in hope}, a virtuoso performance fulfilled in
his death and resurrection at \quoted{the thirty-three}, upon the \quoted{sign}
of the cross.  
Christ himself is the incarnate Word, and his infant cries are the true
\quoted{sign of A}, the \quoted{song} that sets the tone for all the other
voices, \quoted{in one choir and another} of the Christmas manger, and at the
Christmas liturgy in the time of the villancico's performance.

%***********************************************************************
\section[Music about Music]
{Music about Music in the Voices of Puebla's Chapel Choir}

The poem sets up a chain of echoes, in which what God spoke through the voices
of David and the other prophets reverberates in the song of the angels at the
first Christmas and especially the voice of the Christ-child. 
Ultimately all this resounds through the actual \quoted{voices of the chapel
choir} singing in the present.
When Juan Gutiérrez de Padilla \quoted{puts notes to his lyrics}, he uses his
compositional ingenuity, and calls upon the virtuosity of his performers, to
turn a series of poetic conceits \emph{about} music into actual sounding music
that worshippers could hear.
Listeners whose ears were well trained spiritually and musically could discern
the higher levels of music through the sounding forms.

Padilla crafts musical structures that clearly project the formal structure of
the poem at the levels of grammar, phrasing, and metrical patterns.
He presents the words clearly according to their prosody and grammatical
structure, and sets them to memorable melodic and rhythmic patterns.
This aspect of vocal music may be termed text projection.
This approach fits with the aesthetic that prevailed in Catholic music after
Trent, which valorized clarity in projecting the words through music.
The text-driven approach also shows the influence of Spanish popular and
theatrical traditions of singing poetry, especially practices of adapting stock
melodic formulas for singing \emph{romance} poetry.
% XXX more research needed 

In addition to projecting the text in a way that made it intelligible, Padilla
also uses two other text-setting techniques---text depiction and text
expression.%
\begin{Footnote}
    \Autocite[207]{Burkholder:History} defines \emph{text depiction} as
    \quoted{using musical gestures to reinforce visual images in the text}, and
    \emph{text expression} as \quoted{conveying through music the emotions or
    overall mood suggested by the text}, and identifies these as \quoted{two
    principal ways that music can reflect the meaning of the words, both of
    which became common in the sixteenth century}.
    I add to these terms the concept of text projection, and my conceptions of
    depiction and expression are somewhat broader: in particular, depiction need
    not only be limited to \quoted{visual images} but includes numerological
    symbols, puns, and other such figures.
\end{Footnote}
The composer depicts the meaning of the words through musical symbols and
figures that correspond to concepts and imagery in the text.
These include the same kind of \quoted{madrigalisms} favored in
sixteenth-century Italy and Spain, as well as more arcane devices like
numerological symbols of an even older vintage.

In the technique of text expression, the composer goes beyond illustrating the
text and uses different stylistic registers and topics (that is, allusions to
other pre-existing types of music) to convey the meaning and feeling of the
text.
The composer dramatizes the text and uses music to heighten its rhetorical
power.
Text expression instills an affective experience in listeners that matches with
the goals of the poem.
Any vocal piece contains some element of text projection, depiction, and
expression; and these aspects often overlap.

In the case of this villancico, Padilla \emph{projects} the text at the large
scale through the formal structure of sections and harmonic motion; and at the
small scale through nuances of phrasing and rhythmic emphasis.
He \emph{depicts} the text by matching the musical conceits of the poem with
musical figures that correspond literally---essentially, puns.
The level of text painting seems to be Padilla's main focus in this piece, but
he also \emph{expresses} the text through contrasting styles with different
affective associations, shaping the piece to build to a dramatic climax.

\subsection{Projecting the Words}

Padilla projects the structure of the text through the distinct sections of his
setting.
The poem begins with an introductory section, which will be useful to label
\emph{introducción} as many poetry imprints of other villancicos do.
This section consists of two six-line strophes, each followed by the same
\emph{respuesta} or response of a four-line strophe.
The placement of rests and repeat markings in the performing parts makes clear
that the response is sung after each of the six-line strophes.
An \emph{estribillo} follows, which is repeated after the two \emph{coplas}.
Each of these sections is demarcated in the music with silence and a change of
texture, style, and rhythmic movement to match each part of the text.

The piece's harmonic structure further helps articulate the form of the poem.
The piece is in mode I, in \emph{cantus mollis}---that is, the one flat in the
key signature transposes the mode up a fourth; the final is on G and the Tenor
parts have a mostly authentic ambitus.
The internal cadences in the introduction are on G (\measure{19}), D
(\measure{27}), and G (\measure{44}); and the cadences at the end of the
estribillo and of both coplas are also on G.
These cadence points are in line with the prescriptions of contemporary
theorists for modal counterpoint.%
    \Autocites
    [873--882, 883--885, 907--912]{Cerone:Melopeo}
    [364--406]{Judd:RenaissanceModalTheory}
    {Barnett:TonalOrganization17C}
Padilla uses those conventions to punctuate the sections of the poem and make
its grammatical and rhetorical structure clear to listeners.

The composer has paid close attention to both metrical patterns and details of
accentuation and diction, at the levels of strophes, verses, and individual
words.
Padilla begins the piece with the voices of the first choir moving together
rhythmically to declaim the words in something like choral recitative.
Padilla sets the first two lines of poetry (\measures{1--10}) with relatively
long note values on the stressed syllables, creating a deliberate, careful tone
that embodies the poetic exhortation to \quoted{pay attention} to what is sung.
For this poetry in \emph{romance} meter, Padilla has the singers declaim the
eight-syllable lines in pairs, with emphasis on the assonant even-numbered
lines.
He has the singers pause briefly between verse pairs, and punctuates the
assonant lines with clear points of harmonic arrival.
He articulates the end of the strophes in the introduction with full cadences.
Padilla's setting thus aurally projects the text as though it were arranged like
this:
\begin{quotepoem}
    Voces, las de la capilla,       & cuenta con lo que se canta, \\
    que es músico el rey, y nota    & las más breves disonancias \\
    a lo de Jesús infante           & y a lo de David monarca.
\end{quotepoem}
This structure mirrors the pattern of \emph{romance} poetry as heard, rather
than as written in the narrow columns of villancico imprints.
In the national epic of medieval Spain, the \wtitle{Cantar de mío Cid}, the
\emph{romance} meter is arranged the same way, in sixteen-syllable lines, with a
\emph{caesura} in the middle and assonance at the end.%
\begin{Footnote}
    \Autocites
    {Navarro:Metrica} % XXX Navarro page nos. 
    [32--50]{MenendezPidal:Crestomatia}; 
    cf. also Old English and Germanic poetry.
\end{Footnote}

The opening phrase demonstrates Padilla's subtle attention to the sound and
stress of the words (\cref{music:Padilla-Voces-opening} above).
The first chorus sings the first word, \wtitle{Voces}, beginning on the second
subdivision of the ternary measure, with three minims for the first syllable and
two for the second.
Since the word ends on S and is an invocation, it would make sense to have the
singers make a small break between \measures{2--3}, placing the S sound on the
last minim of \measure{2} or just after.
This allows the word to be sung in a way that matches the natural quantities of
its two syllables and that projects its grammatical structure in the opening
sentence of the poem.
That first word, in a common device of Padilla's villancicos, is sung on the
second minim of the measure: thus the leader could conduct the downbeat, cueing
the singers to breathe, and then the chorus would sing their entrance. 
This first phrase, because of its high tessitura and irregular, offbeat rhythms,
seems suspended in the air in a way that would attract listeners' attention to
the ethereal \quoted{voices of the chapel choir}. 
The word \emph{cuenta} (\measure{6}) is also sung on the second minim of the
measure and is then held for three minims, syncopated across the downbeat.
After this long note, like pulling back a spring, the metrical pattern is
released and the voices flow in even, regularly accented minims on \emph{con lo
que se canta}.
For the next phrase (\emph{que es músico el rey}) Padilla creates the effect of
an interjection, breaking the rhythmic pattern and beginning this phrase, like
the others, on the second minim of the measure.

The rest of Padilla's setting is as meticulously crafted as this opening phrase
on the level of text projection.
Another example is the interjection \emph{¡O qué canto!}, which Padilla starts
on the last minim of the measure before the previous phrase has finished,
exactly as an interjection should.
Even as Padilla's setting of the estribillo emphasizes text depiction to a
greater degree, the rhythm and phrasing still closely follows the natural
accentuation patterns of the poetic text.
The two meter changes (\measure{45 and 60}) match the stress patterns implicit
in these sections of poetry, in addition to from the symbolic meanings they also
seem to have.


\subsection{Depicting the Words}

It is on the level of text depiction---representing the meaning of the text
through musical figures and symbols---that Padilla demonstrates his full mastery
of his craft by presenting an intricate discourse about music through music
itself.
As already noted, the opening invocation to the \quoted{Voices of the chapel
choir, keep count with what is sung} is voiced by one division of the polychoral
ensemble while the other chorus literally counts its rests until its entrance in
the \emph{respuesta}.
The first chorus sings the word \emph{cuenta} on a long, offbeat note that
audibly captures the idea of \quoted{counting}.
It is notated as a blackened, dotted semibreve (that is, artificially perfected)
that leaps off the page as a visual indication to the singer to \quoted{keep
count}.
Padilla sets \quoted{the lightest dissonance} in \measures{14--19} by creating
just that: he has the Altus I suspend across the first minim of \measure{18},
making a dissonant seventh against the Tenor's A that quickly resolves to F
sharp and then to a cadence on G in \measure{19}.

In the \emph{respuesta}, Padilla continues this literal approach.
Where Chorus II sings about awaiting \quoted{the thirty-three}, Padilla writes
precisely thirty-three pitches for both of the sung vocal parts.
Just after the chorus sings that the whole world was \quoted{waiting} for
\quoted{the sign}, Padilla uses the C meter sign to indicate a shift to duple
meter (\measure{45}).  
After this the musicians shift from the declamatory style of the introduction to
a more regular rhythmic pattern, moving more quickly together in \emph{corcheas}
(modern eighth notes).
Here Padilla depicts what the words say by building up a point of imitation
\quoted{from one choir to the other} (\measures{45--50}) and then creating
polychoral dialogue (\measures{51--59}).
Padilla sets the numbers in the poem literally: three performers for \emph{tres
a tres}, two for \emph{dos a dos}, and one for \emph{uno a uno}.

At the phrase \emph{y aguardan tiempo oportuno} (\quoted{and they await the
opportune time}, \measure{60}), Padilla shifts meter signature again, returning
to ternary meter---in Spanish terminology, a new \emph{tiempo}.
The 1672 music treatise of Lorente says the term \emph{tiempo} can denote both
the meter and the symbol that sets the meter.%
    \Autocite[bk.~2, 149]{Lorente:Porque}
After the time signature C3 (or CZ), then, Padilla begins a new lilting rhythmic
pattern that creates a sense of arrival in a new \quoted{time}.
He abruptly halts this movement at the end of the phrase, \emph{quien antes del
tiempo fue} (\quoted{the one who was before time}, \measures{63--65}).
As Christ is \quoted{the first and the last} (Rv 1:17), this halt is fitting.
Since Christ existed before all time theologically, Padilla puts this phrase
\quoted{before the time signature} musically.

The musical conceits in the next lines of poetry shift the focus from rhythm to
melody, as the poem uses solmization symbols: \emph{por el signo a la mi re,
puestos los ojos en mi}.
Likewise, Padilla's metamusical conceits play on the musical terms in the most
literal sense, realizing the solmization syllables in several ways at the same
time (\cref{music:Padilla-Voces-alamire}).
Both Altus I and Tenor I sing the word \emph{a} (\measure{67}) on the pitch
known by its Guidonian syllables as \emph{A (la, mi, re)}.
On the words \emph{la mi re} (\measures{68--69}), the Tiple I (boy treble) sings
the pitches D--C\sh{}--D, which could plausibly be sung to those syllables.
In the soft hexachord (which starts on F) the D would in fact be \emph{la}.
The written sharp on C would alter it to a \emph{mi} in \emph{musica ficta}. 
The final D could be \emph{re} in the natural hexachord (which starts on C);
thus, Padilla has spelled out \emph{A la mi re}.
On the same words, the Tenor sings D--A--D: this would be \emph{la--mi} in the
soft hexachord, then \emph{re} in the natural hexachord.
At the end of this phrase (\measure{72}), all three voices sing the word
\emph{mi} by literally \quoted{putting their eyes on \emph{mi}}: the Tiple and
Tenor sing \emph{mi} on A (in the soft hexachord) and the Altus sings \emph{mi}
on E (in the natural hexachord).

\begin{musicexample}
    \caption{Gutiérrez de Padilla, \emph{Voces, las de la capilla}, estribillo
    (\measures{66--72}): \quoted{The sign of \emph{A (la, mi, re)}}}
    
    \label{music:Padilla-Voces-alamire}
   
    \includeWideGraphic{Padilla-Voces-alamire}
\end{musicexample}

Guidonian solmization was still a fundamental part of Spanish theory treatises
through the eighteenth century, and were used frequently enough that they later
came to be used as the Spanish names for pitch-classes in the seven-note scale
(for example, \emph{sol} today is always the note G, which was historically
\emph{sol} in the natural hexachord).
The prevalence of solmization practice is evidenced by books intended for
specialists (Cerone) and those for beginners, such as the 1677 guitar primer of
Lucas Ruiz de Ribayaz, as well as student notebooks in manuscript.%
\begin{Footnote}
    \Autocite{Ruiz:Luz}; examples of manuscript student notebooks are in
    \sig{E-Bbc}{M732/13--16}; see also \autocite{Cohen:NotesMiddleAges}.
\end{Footnote}
Even if fully trained singers did not often resort to the actual syllables in
reading music, they were trained in the system and could certainly have
recognized the Guidonian puns in this passage.

In the final couplet of the estribillo (\emph{Todo en el hombre es subir/ y todo
en Dios es bajar}, \measures{100--126}), Padilla matches the theological concept
of interchange between Man and God by creating an exchange of musical gestures.
One gesture ascends in ternary rhythm and the other descends in duple rhythm.
In the first of these gestures, for Man ascending, the voices ascend stepwise in
minims, in a lilting dotted rhythm with a strong ternary feel.
This is first heard in Altus I and Tenor I, \measures{98--100}, with the ascent
highlighted by having the Tenor move through F sharp.
In the second gesture, for God descending, all the voices move downwards in
emphatic duple rhythm with blackened semibreves.
The Tenor I has the highest number of blackened notes, singing a sequence of
descending intervals of decreasing size: first fourths (\measures{100--103}),
then thirds (\measures{108--112}), and finally seconds (\measures{118--122})
(figure \ref{fig:Padilla-Voces-TI-bajar}).
Just as the ascent pushed up into sharps, so the descent sinks down into added E
flats (Chorus I, \measures{100--104}).

% Footnote: check? XXX
% The appropriate rhetorical terms for these gestures would be \emph{anabasis}
% (ascent) and \emph{catabasis} (descent).
% cf Chafe

\begin{figure}
    \caption{Gutiérrez de Padilla, \wtitle{Voces, las de la capilla}, end of
    estribillo in Tenor I partbook: Coloration on figure for \quoted{God
    descending}}
  
    \label{fig:Padilla-Voces-TI-bajar}
   
    \includeWideGraphic{Padilla-Voces-TI-bajar}
\end{figure}

When Padilla juxtaposes these ideas in the full polychoral texture, listeners
can hear the fusion of both rising and falling melodic lines, and two different
rhythmic systems (\cref{music:Padilla-Voces-subir_bajar}).
There is a clearly audible contrast between the rhythm of \quoted{God
descending} (moving in a triple-simple feel) and that of \quoted{Man ascending}
(duple-compound feel resulting from \emph{sesquialtera}).
In the final cadence, the Altus I combines the two gestures at once by singing
the words \quoted{everything in God is to descend} to the music associated with
\quoted{Man ascending} (\measures{124--126}).  
The passage musically embodies the central theology of the Incarnation: through
God's descent to become Man in Christ, Man may ascend to share in God's nature.

\begin{musicexample}
    \caption{Gutiérrez de Padilla, \wtitle{Voces, las de la capilla}, estribillo
    (\measures{106--126}): Contrasting motives and rhythmic patterns for
    \quoted{Man ascending} and \quoted{God descending}}

    \label{music:Padilla-Voces-subir_bajar}
   
    \includeWideGraphic{Padilla-Voces-subir_bajar}
\end{musicexample}

Padilla's literal approach to text depiction continues in the two coplas.
Each of the two poetic coplas centers on a concept from music theory: the first
plays with the notion of \emph{peregrino tono} (\quoted{wandering song}, or the
plainchant \emph{tonus peregrinus}); the second, on the contrasting rhythmic
values of \emph{la máxima y breve}.
True to form, Padilla sets \emph{peregrino tono} to a fragment of the actual
plainchant tone.
The last psalm tone was known in Latin sources as the \emph{tonus peregrinus}
and in Spanish sources as \emph{tono irregular} or \emph{tono mixto}.
Its final cadence is a rising minor third followed by a stepwise descent to the
final (\cref{fig:Cerone-tonus_peregrinus}).%
    \Autocite[354]{Cerone:Melopeo}
In its normal, untransposed, form (that is, in \emph{cantus durus}), this is
D--F--E--D.
If transposed up a fourth to \emph{cantus mollis} to match the mode of this
villancico, the pitches would be G--B\fl{}--A--G. 
Those are exactly the notes that the Altus I sings on these words
(\measures{128--129}, \cref{music:Padilla-Voces-peregrino_tono}).
The same motive of the a leap up followed by a descending third is then imitated
in the Tenor I and Tiple I on F--E--D, the \emph{durus} version.%
\begin{Footnote}
    In other versions of the \emph{tonus peregrinus}, the medial cadence matches
    exactly with the music of the Altus I (G--B\fl{}--A--G--F):
    \autocite[160]{Catholic:LiberUsualis1956}.
% XXX confirm chant dialect in Puebla 
\end{Footnote}

\begin{figure}
    \caption{The \emph{tonus peregrinus} (\quoted{tono irregular o mixto,}
    \quoted{octavo irregular}) in Cerone, \wtitle{El melopeo y maestro}, 354}

    \label{fig:Cerone-tonus_peregrinus}

    \includeWideGraphic{Cerone-tonus_peregrinus}
\end{figure}

\begin{musicexample}
    \caption{Gutiérrez de Padilla, \wtitle{Voces, las de la capilla}, copla 1
    (\measures{127--132}): Point of imitation quoting cadences chant \emph{tonus
    peregrinus} on words \emph{peregrino tono}}

    \label{music:Padilla-Voces-peregrino_tono}

    \includeWideGraphic{Padilla-Voces-peregrino_tono}
\end{musicexample}

In the last phrase of this copla, when the poem speaks of Christ \quoted{going
up the octave} or theologically \quoted{ascending on the eighth day}, Padilla
creates an octave ascent across the voices, with the Tenor leaping
\pitch{D}{4}--\pitch{G}{4} and the Tiple continuing, \pitch{G}{4}--\pitch{D}{5}.
This octave also plays into the symbolism of the \emph{tonus peregrinus}, since
Cerone says that Spanish writers call this \quoted{eighth} psalm tone
\emph{octavo irregular}.%
    \Autocite[354]{Cerone:Melopeo}

The second copla emphasizes rhythm, using the note values of the \emph{máxima}
and \emph{breve}, to point to the union of eternal and temporal, infinite and
finite in the incarnate Christ as the divine chapelmaster's
\quoted{composition}.
In medieval theory these were the longest and shortest note values: the maxima
was worth eight breves (the breve corresponding to a modern double-whole note).
Padilla presents the basic concept of long versus short note values through the
lengthened note on \emph{máxima} in the Tenor (\measures{153--154}) over a long
held note in the Bassus (\measures{154--155}).
Ironically, each of these long notes is actually a breve
(\cref{music:Padilla-Voces-maxima}).  
How better, though, to express the unity between these opposites than by
vocalizing the name for one while singing the value of the other?
At a more arcane level, the whole first phrase of this copla
(\measures{147--156}) is ten measures (\emph{compases}) long, which is precisely
equal to the length of a true maxima plus a breve (that is, eight measures in C3
plus two measures).

\begin{musicexample}
    \caption{Gutiérrez de Padilla, \wtitle{Voces, las de la capilla}, copla 2
    (\measures{152--156}): The word \emph{máxima} sung on a breve (original note
    values shown without bar lines)}
    
    \label{music:Padilla-Voces-maxima}

    \includeWideGraphic{Padilla-Voces-maxima}
\end{musicexample}


\subsection{Expression and Madrigal Style}

This analysis shows that Padilla's setting as a whole is intimately connected to
the sound and meaning of the poetry.
Padilla has his chapel choir present the poem in a way that allows the words to
be heard clearly, reflecting their grammatical structure and the poem's dramatic
shape, while also embodying the conceits of the poetry in appropriate musical
symbols and even adding some musical puzzles of his own.
The contrasting styles in the piece would also work on a less intellectual, more
experiential level---that of text expression---to move the affections of
hearers.

The style of the estribillo, contrasting with the other sections, is like that
of a madrigal (\cref{music:Padilla-Voces-madrigal}).
This section, then, is \quoted{music about music} not only in the way it uses
musical figures to represent music, but also by referring to multiple existing
genres and style of music within one villancico.
As the poem depicts the actual singing performed at the first Christmas, Padilla
uses the style of a genre used for convivial group singing.
The angels, planets, shepherds, and animals around the créche are represented
not only as singing in the abstract---they are singing a madrigal.

In fact, the play on numbers, \quoted{three by three} and so on, seems like a
direct reference to the madrigal \wtitle{As Vesta Was from Latmos Hill Descending}
by Thomas Weelkes (published in Thomas Morley's collection \wtitle{The Triumphes
of Oriana} in 1601).
Weelkes's music, and that of other English madrigalists, did circulate in
Iberia: his madrigals, and possibly this specific collection, were included
along with Italian madrigals in the 1649 catalog of the music collection of
Portuguese King John (João) IV in Lisbon.%
    \Autocite[no.~559, 584]{JohnIV:Catalog}
It therefore seems plausible that this repertoire was also known in Spain (where
Padilla lived into his thirties) and possibly Spanish America.
There may be a more direct connection to Lisbon, since a text closely related to
Padilla's \wtitle{Voces} was performed at the Royal Chapel in Lisbon and a
setting with the same incipits was also part of John IV's collection (see
below).
Even without this specific connection, Padilla's approach to matching music to
words in the estribillo depicts the words' meaning in a manner that any musician
of the time would have identified with madrigals.

\begin{musicexample}
    \caption{Gutiérrez de Padilla, \wtitle{Voces, las de la capilla}, estribillo
    (\measures{48--59}): Evocation of madrigal style}

    \label{music:Padilla-Voces-madrigal}

    \includeWideGraphic{Padilla-Voces-madrigal}
\end{musicexample}

By referencing different levels of musical style, Padilla maps contrasting types
of human music onto the contrast of earthly and heavenly music.
The phrase \emph{grave, suave y sonoro} seems to have been a stock description
of sacred music appropriate for liturgical worship.%
    \Autocite
    [On \emph{suave} and other common vocabulary used to evoke music in Spanish
    poetry of the period, see][]{UribeBracho:OrfeoPhD}
The same words are used to describe the music of Christ as a musician (in fact,
as a musical instrument) in José de Cáseda's \wtitle{Qué música divina} from a
half-century later, as discussed in \cref{ch:Zaragoza}.
In 1682 a nobleman used the same adjectives---\emph{gravísima y suavísima
música}---to characterize the music he endowed at Mexico City Cathedral for
Matins for Holy Trinity.%
    \Autocite[140--141]{Goldman:Responsory}
The music he requested, though, was specifically \emph{not} villancicos, but
rather only Latin-texted responsory settings.
This suggests that in this nobleman's mind the vernacular genre was not suitably
\quoted{solemn}.
In Padilla's piece, though, the term is used within a villancico to refer to a
higher form of music-making. 
This gesture fits within the Neoplatonic tradition of using that which is lowly
to point toward higher truths.

This stylistic reference also echoes the way the music of Christmas is portrayed
in contemporary Nativity images.
Padilla's Andalusian compatriot Francisco de Zurbarán depicted two levels of
music in his \wtitle{Adoration of the Shepherds}, painted for the high altar of
the Carthusian monastery of Jerez de la Frontera in 1638--1639
(\cref{fig:Zurbaran-Jerez-Adoracion}).
(Padilla had been chapelmaster of the cathedral of Jerez de la Frontera in
1612--1616.)%
    \Autocite{Gembero:Padilla}
In heaven above, angels sing to the accompaniment of the harp, while below,
another angelic consort joins the company of worshippers in the stable, and they
sing to the accompaniment of the lute.
In Spain the harp was associated with both heavenly music and earthly church
music (see \cref{ch:Zaragoza}), and the lute, with musical genres
performed outside of church, such as the madrigal.
Villancicos crossed both domains, and therefore could incorporate references to
both styles within them.
This is analogous to the way Zurbarán's painting incorporates aspects of genre
painting---the representation of mundane details from everyday life---into a
representation of sacred history.%
    \Autocites[31]{Sanchez:Zurbaran}{Cherry:Bodegon}{Haraszti-Takacs:Genre}
Both visual and musical forms of crossover were especially fitting to represent
the Christmas moment when the \quoted{maxima} and \quoted{breve}, high and low,
were brought together, and the music of the heavenly chorus broke through to be
heard by humble shepherds.%
\begin{Footnote}
    Additionally, Zurbarán's image includes a bound lamb next to the manger as a
    symbol of Christ's fate as the paschal lamb, much as \quoted{the
    thirty-three} and other references in this villancico connect Christ's birth
    to his sacrificial death.
    The same trope may be seen in the Adoration of the Shepherds on the retable
    of Puebla Cathedral (\cref{fig:Puebla-Ferrer-Pastores}, discussed below).
    It also appears in the manger scene in
    \autocite[168]{Catholic:Breviarium1631}.
\end{Footnote}

\begin{figure}
    \caption{Francisco de Zurbarán, \emph{Adoración de los pastores}, retable of
    the Carthusian monastery of Jerez de la Frontera, 1638--39: The consort of
    heavenly and earthly music, harp above and lute below}

    \label{fig:Zurbaran-Jerez-Adoracion}

    \includeWideGraphic{Zurbaran-Jerez-Adoracion}
\end{figure}

\endinput
%***********************************************************************
\section{Hearing the Christ-Child's Voice in Devotional Theology }

By inviting hearers to listen for the voice of \emph{Jesús infante}, Padilla's
villancico presents a new, musical twist on an ancient theological trope.
The concept of devotion to Christ as \quoted{unspeaking Word} is first attested in New
Testament texts like the beginning of John's gospel, which describes Christ as
\quoted{the Word made flesh}, and extends through the preaching of Saint Augustine in
the fourth century and of Saint Bernard of Clairvaux in the twelfth.
This trope continued to be turned and refined through the sixteenth and
seventeenth centuries, and can be found in Christmas sermons like a model sermon
by Fray Luis de Granada and exegetical commentaries on the Gospel accounts of
Christ's birth like those of Cornelius á Lapide.
These texts were widely available in the Spanish Empire of the seventeenth
century and would have been familiar to a university-educated priest like
Padilla.

This piece is the product of theological creativity in reflecting on the
relationship between the incarnation and the voices of Christmas. 
It links the theology of \quoted{the Word} to theology of music in a way that positions
listening to music as a way of encountering Christ himself through the sense of
hearing.
This villancico conceives of \quoted{the Word} in musical terms and then represents
that concept through actual music; in this way it draws on listeners' knowledge
of both theology and music and creates an opportunity for them to contemplate
the relationship between these two domains of learning.

Considering this devotional music in its theological context helps us understand
what this music meant and how it functioned within its original interpretive
community.
This piece is devotional because its primary function is not as a tool for
indoctrination, but as a object used to cultivate spiritual attitudes of
worship.
The piece does embody certain Catholic beliefs about the incarnation of Christ,
but its complex poetry and music would not make the most effective teaching
tools.
Instead, the piece directs listener's attention to a certain aspect of
Christ---that is, it invites hearers to think of Christ in a certain way; and it
provides an opportunity for them to respond to this concept of Christ.
Instead, through its central metamusical trope, the piece directs listeners'
attention to a certain aspect of Christ---the Christ-child as \emph{Verbum
infans}.
Through verbal and musical virtuosity, the poet, composer, and performers invite
the congregation to respond to the music with the same sense of wonder that they
should cherish before the mystery of Christ's birth.  

Villancicos developed as part of the church's annual cycle of feasts and
seasons, and the Catholic liturgical year functioned not so much to teach people
doctrine about Christ as to allow them to participate communally in the body of
Christ.
All of the church's festivals provided occasions to remember God's deeds in the
past for human salvation (\emph{los siglos de sus hazañas} in the language of the
villancico), and to act creatively to relive and share in those deeds in the
present.
This is the theological concept of \emph{anamnesis}, which is closely connected to
the sacramental character of Catholic worship: through human actions and
physical means, Christ is revealed anew within each community and the community
is transformed by the experience.
[For an influential modern Catholic theology of the function of liturgical
feasts, see @Taft:LiturgicalYear, 3--4, 12--23.]
The function of early modern Catholic feasts was not so much to teach people
that Christ \emph{is} something, as to cultivate devotion to Christ \emph{as} something.
The primary goal was not indocrination but doxology---not so much believing
correctly as worshipping properly.[^feasts-and-seasons]

[^feasts-and-seasons]:
This concept of devotion brings together the distinct but interlinked ritual
modes of \quoted{liturgy} and \quoted{celebration} theorized by @Grimes:Beginnings, 54--62.
Scholars of ritual studies and Christian liturgical theology have consistently
argued that liturgy does more than promulgate doctrines, a perspective that
musicological and philological studies of post-Tridentine religious arts would
do well to consider more seriously.


The feast of Christmas focuses not only on remembering the event of Christ's
miraculous birth, but on cultivating devotion to the Christ-child and the
particular aspect of God's saving work that the child represents---namely, that
God took on human flesh, that the almighty humbled himself to become a helpless
child, that God descended to the lowly estate of humankind in order to raise
humans to share in his divine nature.
Christmas had first developed from regionally varying annual celebrations of
Christ's birth and beginnings, and solidified as a formal observance among the
church communities of fourth-century Rome.
[@Talley:LiturgicalYear, 85--140; @Bradshaw:EarlyWorship, 86--89;
@Roll:OriginsChristmas]
The customary readings and chants from Scripture developed alongside the other
customs of Christmas throughout the medieval period, so that the festival
encompassed a wide range of official and unofficial, formal and informal beliefs
and practices.
Catholics over the centuries had developed what might be called a Christmas
imaginary, the contents of which one may see restated and endlessly varied not
only in villancicos but also painted and sculpted on Spanish church walls and
printed in the text and illustrations of contemporary theological books.
The visual and performing arts preserved in such historic sources represent only
a portion of the lively variety of customary social practices connected with the
feast---gift-giving, traditional foods, popular songs from oral tradition.
Both official sources like the liturgy, creeds, catechisms, commentaries, and
homilies as well as unofficial popular devotion drew from and contributed to a
commmon fund of Christmas tropes, which by the seventeenth century was filled to
overflowing.

Though oral tradition was probably still the most important force in cultivating
the Christmas imaginary in early modern Spain, printed theological literature
disseminated and reinforced these tropes in Padilla's day and makes them
accessible today.
The historic libraries of Puebla's seminaries and convents are preserved today
in the city's rare-book archives (the Biblioteca Palafoxiana and Biblioteca
LaFragua), and these collections include a large number of compendia of
patristic commentaries on Scripture, model sermons, and both vernacular
devotional books and learned Latin theological treatises.
[On patristic exegesis and its influence, see @Kannengiesser:PatristicExegesis;
@McKim:BiblicalInterpreters; @Thompson:ReadingwDead.]
\Cref{tab:puebla-compendia} includes two books from the library of
Puebla's Oratory of Saint Philip of Neri, the priestly order to which Padilla
belonged.
[@Mauleon:PadillaCivil]
These collections provide evidence for the high level of learning among New
Spanish clergy like Juan Gutiérrez de Padilla and documents the currents that
formed them spiritually.

(insert \cref{tab:puebla-compendia})
\label{tab:puebla-compendia}

Hispanic Catholics focused their devotion at Christmas on the baby Jesus as God
made flesh to restore humankind to right relationship with God. 
Their traditions emphasized the affects of awe and wonder in response to
Christ's miraculous incarnation.
The object of Christmas devotion was, in the words of the Apostle's Creed,
\quoted{Jesus Christ, \add{God's} only Son, our Lord, who was conceived by the Holy Spirit, born of the Virgin Mary}.
[@Catholic:Catechismus1614, 34: \quoted{Et in Iesum Christum Filium eius vnicum, Dominum nostrum}; 
@Catholic:Catechismus1614, 46, \quoted{Qui conceptus est de Spiritu sancto, natus ex Maria virgine}.]
The Catechism of Trent instructs pastors to teach the \quoted{admirable mystery} 
of this article of faith by having \quoted{the faithful repeat by memory \Dots{} that
he \add{Christ} is God, who took on human flesh, and thereby was truly \enquote{made
man}---which cannot be grasped by our mind, nor explained through words: that he
should wish to become a human, to the end that we humans should be reborn as
children of God}.
[@Catholic:Catechismus1614, 50: \quoted{Haec sunt quae de admirabili conceptionis
mysterio explicanda visa sunt, ex quibus, vt salutaris fructus ad nos redundare
possit, illa in primis fideles memoria repetere, ac saepiùs cogitare cum animis
suis debent, Deum esse, qui humanam carnem assumpsit: ea verò ratione hominem
factum, quam mente nobis assequi non licet, ne dum verbis explicare: ob eum
denique finem hominem fieri voluisse, vt nos homines filij Dei renasceremur.
Haec cùm attentè considerauerint, tum verò omnia mysteria, quae hoc articulo
continentur, humili ac fideli animo credant, \et{} adorent: nec curiosè \et{} quod sine
periculo vix vnquam fieri potest, illa inuestigare, ac perscrutari velint}.]
The incarnation of Christ, this passage suggests, was not so much a concept to
be understood as a miracle to be marvelled at.
While the Incarnation was certainly a dogma, a \quoted{rule of belief}, even the
offical catechism moves beyond simply defining a theological concept, to include
an affective, devotional emphasis---a \quoted{rule of prayer} as well.[^lex-orandi]
The proper response to meditating on this \quoted{all of these mysteries}, the
catechism says, would be \quoted{that with a humble and faithful spirit they should
believe, and adore}.
[@Catholic:Catechismus1614, 50: \quoted{Haec cùm attentè considerauerint, tum verò omnia
mysteria, quae hoc articulo continentur, humili ac fideli animo credant, \et{}
adorent: nec curiosè \et{} quod sine periculo vix vnquam fieri potest, illa
inuestigare, ac perscrutari velint}.]

[^lex-orandi]:
The idea of a \quoted{rule of prayer} comes from a Latin motto, drawn from Prosper of
Aquitaine, that has become a standby in modern liturgical studies, \emph{lex orandi,
lex credendi}.
Rather than use the motto to create a dichotomy between belief and prayer---in
which \quoted{the rule of prayer establishes the rule of faith} (see
@Kavanagh:LiturgicalTheology)---the intent here is simply to emphasize that both
faith and worship may be seen as integral and interrelated parts of religious
life in Christianity.
For critical discussion of the motto, with attentions to its actual origin and
its changing meaning in recent scholarship, see @Irwin:LexOrandi.

Written examples of teaching and preaching about Christ's birth demonstrate this
same devotional approach in their emphasis on wonder.
Even in the learned genre of a Latin Biblical commentary, Cornelius á Lapide
stresses that Christ's birth defies understanding: 
\quoted{The Word was made flesh, God was made man, the Son of God was made the son of a
Virgin.
This \Dots{} was of all God's works the greatest and best, such that it
stupefied and stupefies the angels and all the saints}.
[@Lapide:Gospels19C, 50, on Mt 1:16:
\quoted{q. d. Verbum caro factum est, Deus factus est homo, Filius Dei factus est
Filius Virginis.
Hoc, \Dots{} fuit omnium Dei operum summum et maximum, ideoque illud stupuerunt
et stupent Angeli, Sanctique omnes}.]
In a model sermon for Christmas, Fray Luis de Granada draws on all his
rhetorical skills to exhort worshippers to marvel at the sight and sound of
Christ at his lowly birth:

> Come and see the Son of God, not in the bosom of the Father \add{Jn 1}, but in
> the arms of the Mother; not above choirs of angels, but among filthy animals;
> not seated at the right hand of the Majesty on high \add{Heb 1}, but
> reclining in a stable for beasts; not thundering and casting lightning in
> Heaven, but crying and trembling from cold in a stable. \Dots{}
[@LuisdeGranada:Xmas, 37:
\quoted{Venid a ver al hijo de Dios, no en el seno del Padre, sino el los brazos de la
Madre; no sobre los coros de los Ángeles, sino entre viles animales; no asentado
en la diestra de la Majestad en las alturas, sino reclinado en un pesebre de
bestias; no tronando y relampagueando en el cielo, sino llorando y temblando de
frío en un establo}.]

\noindent
Contemplating Christ's birth, Luis preaches, will cause anyone to be \quoted{struck
numb} with awe:

> What theme, then, can cause any greater wonder? \Dots{}
> \add{As Saint Cyprian says}, I do not wonder at the figure of the world, nor the
> firmness of the earth \Dots{}; I marvel to see how the word of God could take
> on flesh. \Dots{} In this mystery the greatness of the shock steals away all
> my senses, and with the prophet \add{Hab. 3} it makes me cry out: Lord, I heard
> your words, and I feared: I considered your works, and I was struck numb.
> With good reason, indeed, you are amazed, Prophet: for what thing could
> surprise anyone more, than that to which the Evangelist here refers in a few
> words, saying, \quoted{She gave birth to her only-begotten son, and she wrapped him
> in some rags, and laid him in a manger, because she did not find another place
> in that stable}?
[@LuisdeGranada:Xmas, 38:
\quoted{¿Pues qué cosa puede ser de mayor maravilla? 
¡O Señor Dios nuestro (dice san Cipriano), cuán admirable es vuestro nombre en
toda la tierra!
Verdaderamente vos sois Dios obrador de maravillas; ya no me maravillo de la
figura del mundo, ni de la firmeza de la tierra (estando cercada de un cielo tan
movible), no de la sucesión de los días, ni de la mudanza de los tiempos (en los
cuales unas cosas se secan, otras reverdecen; unas mueren, y otras viven), de
nada desto me maravillo; sino maravíllome de ver a Dios en el vientre de una
doncella; maravíllome de ver al todopoderoso en la cuna; maravíllome de ver como
a la palabra de Dios se pudo pegar carne, y cómo siendo Dios sustancia
espiritual, recibió vestidura corporal.
\Dots{}
Con mucha razón, por cierto, os espantáis Profeta; porque, ¿que cosa más para
espantar, que la que aquí en pocas palabras nos refiere el Evangelista,
diciendo: Parió a su unigénito, y envolvióle en unos pañales; 
y acostóle en un pesebre, porque no halló otro lugar en aquel establo?}]

\noindent
Lapide's exegesis and Fray Luis's preaching guide the faithful to the right kind
of devotion at Christmas---to an affective response of awe at the mystery of
Christ's birth.

Spanish devotional music for Christmas seems designed primarily to cultivate
this same attitude of wonder.
Padilla's setting of \wtitle{Voces, las de la capilla} evokes wonder not only in the
words in the virtuoso composition and performance of the music as well.
The villancico aims less to instruct than to amaze. 
This supports Mary Gaylord's argument that the goal of elaborate Spanish poetry
is \quoted{to produce effects of astonishment and awe conveyed by the Latin term \emph{admiratio}}.
[@Gaylord:Poetry, 227]
Indeed, Padilla's piece specifically asks listeners to imagine a song that is
\quoted{as much to hear as to admire \add{\emph{admirar}},/ as much to admire as to hear}.

The concept of \emph{admiratio} is, in fact, central to the Christmas liturgy. 
It is encapsulated in the fourth Responsory of Matins, \emph{O magnum mysterium et
admirabile sacramentum}:

> \emph{Respond}. O great mystery and admirable sacrament, that the animals should
> see the newborn Lord, lying in the manger.
> Blessed Virgin, whose womb was worthy to bear the Lord Christ.\
> \emph{Versicle}. Greetings, Mary, full of grace: The Lord is with you.
[@Catholic:Breviarium1631, 175:
\quoted{\emph{Respond}. O magnum mysterium, \et{} admirabile sacramentum, vt
animalia vidêrent Dominum natum, iacentem in praesepio: Beata Virgo, cuius
viscera meruêrunt portare Dominum Christum. \emph{Versicle}. Ave Maria, gratia
plena: Dominus tecum}.] \noindent
In his sermon Fray Luis alludes to this Responsory in terms quite similar to
those in Padilla' villancico, when he cries out, \quoted{O venerable mystery, more to be felt than to be spoken of; not to be explained with words but to be adored with wonder in silence}.

This same Responsory was probably paired with Padilla's \wtitle{Voces, las de la
capilla} in the Puebla Cathedral liturgy on Christmas Eve 1657.
Based on the position of this villancico in the Puebla musical manuscripts, it
was most likely sung as the fourth villancico in the Matins cycle.
This means that in accord with a 1633 decree of the cathedral chapter the
villancico would have occupied the same liturgical time and space as the 
Responsory \emph{O magnum mysterium}.
The chapter mandated that while \quoted{the lessons should be sung in their entirety},
\quoted{the \emph{chanzoneta} shall serve for the Responsory, which shall be
prayed speaking while the singing is going on}.[^Puebla-Matins-AC]
This villancico stood in between the lessons of the second Nocturne, which were
taken from a Christmas sermon of Leo the Great.
This means that after a cantor chanted the first lesson, a reader spoke the
mandatory liturgical text of the fourth Responsory above \emph{while} the chorus sang
this villancico.
The conjunction of texts may not have communicated much to the lay people
outside the walls of the architectural choir, who could not hear or understand
the words of the Latin liturgical texts; but for the learned cathedral canons,
the simultaneous performance of the Latin prayer and Spanish song would have
deepened the hidden connections between the two.

[^Puebla-Matins-AC]:
\sig{MEX-Pc}{AC 1633-12-30}:
\quoted{que a los maitines de nauidad deste año y de los venideros \Dots{} se canten
todas las liçiones yn totum sin dejar cossa alguna dellas y que la chansoneta
sirba de Responsorio el qual se diga resado mientrass se estubiere cantando}.

The Responsory evokes the scenario of the Nativity, with the animals gathered
around the manger, just as the villancico calls up the image of angels, beasts,
and humans joining together in song around the Christ-child.
In the quatrain (a \emph{redondilla}) that closes the estribillo, the lines
\emph{tan de
oír y de admirar/ tan de admirar y de oír} actually seem like a reply to the
Responsory, as though to say that the mystery of the Incarnate Christ certainly
was an \quoted{admirable} sacrament---that is, one that can be seen---but it is also an
audible sacrament.
Listeners in Puebla could not visit the manger in Bethlehem; they could only look
at the retable paintings of the Adoration of the Shepherds and the Visitation of
the Kings and imagine.
But in the conception of this villancico, they could hear the \quoted{voices of the
chapel choir} reverberating in the space and through those voices they could
\quoted{hear the voice of the Father/ singing in tones of weeping}---that is, they
could hear the Christ the Word himself not merely speaking, but singing, through
all the other voices from the choirboys up to the angels.

\subsection{The Infant's Voice}

The source of wonder at Christmas in the particular trope of this villancico was
the voices heard at Christ's birth, and the voice of the baby Jesus in particular.
Fray Luis appeals to the sense of hearing throughout his Christmas sermon,
starting by asking the faithful to imagine the conversation of Mary and Joseph
on their way to Bethlehem.
After a detailed visual meditation on the scene in Bethlehem, he exhorts
worshippers to turn from sight toward hearing:

> After the devout sight of the manger we open our ears to hear the music of the
> angels, of whom the Evangelist says, that when one of them had finished giving
> these very glad tidings to the shepherds, there were joined with him a crowd
> of the heavenly army, and that they all in one voice sang upon the airs
> praises to God, saying, Glory be to God in the heights, and on the earth peace
> to men of good will.
[@LuisdeGranada:Xmas, 40:
\quoted{Despues de la vista devota del pesebre, abramos los oídos para oír las músicas
de los Ángeles: de los cuales dice el Evangelista, que acabando uno dellos de
dar estas tan alegres nuevas a los pastores, se juntó con él una muchedumbre de
ejército celestial, y que todos a una voz cantaban por aquellos aires alabanzas
a Dios, diciendo: Gloria sea a Dios en las alturas; y en la tierra paz a los
hombres de buena voluntad}.]

\noindent
Even more important than these angelic voices, Fray Luis preaches, is the voice
of the newborn Christ himself, \quoted{crying and trembling with cold in the stable}.
Fray Luis follows an ancient tradition of reading Wisdom 7---in which Christ's
ancestor King Solomon speaks about his own infancy---as a Messianic prophecy:
\quoted{I
too am a mortal man like others, \Dots{} and the first sound \add{\emph{voz}} that I made
was crying like other children, because not one of the kings had any different
origin in their birth}.
[@LuisdeGranada:Xmas, 37--38:
\quoted{De suerte, que ya puede él decir por sí aquellas palabras del Sabio: Soy yo
también hombre mortal como los otros, del linaje terreno de aquél que primero
que yo fue formado; y en el vientre de mi madre tomé sustancia de carne; y
después de nacido recibí este aire común a todos, y caí en la misma tierra que
todos; y la primera voz que di, fue llorando, como todos los otros niños; porque
ninguno de los Reyes tuvo otro orígen en su nacimiento; todos tienen una misma
manera de entrar en la vida, y una manera de salir della}.
Cf. @Lapide:Gospels19C, 670, on Lk 2:7]

Fray Luis explicitly compares the voice of this incarnate Word with music.
The Dominican friar, as an avid student and teacher of Catholic Humanism,
presents Christ as an orator and philosophical teacher, a \quoted{Master of
Heaven}---\emph{Maestro del cielo}, the same term used for a musical master:

> Oh fortunate house!
> Oh stable, more precious than all the royal palaces, where God sat upon the
> chair \add{\emph{cátedra}} of the philosophy of heaven, where the word of God, though
> made mute, speaks so much more clearly, all the more silently it admonishes
> us!
> Look, then, brother, if you wish to be a true philosopher, do not remove
> yourself from this stable where the word of God cries while keeping silent;
> but this cry is greater eloquence than that of Tully [Cicero], and even than
> the music of the angels of heaven.
[@LuisdeGranada:Xmas, 39: 
\quoted{¡O dichosa casa!  O establo más precioso, que todos los palacios reales,
donde Dios asentó la cátedra de la Filosofía del cielo, donde la palabra de Dios
enmudecida, tanto más claramente habla, cuanto más calladamente nos avisa.  Mira
pues, hermano (si quieres ser verdadero Filósofo) no te apartes deste establo;
donde la palabra de Dios callando llora; mas este lloro es mayor elocuencia que
la de Tulio, y aun que las músicas de los Ángeles del cielo}.]

\noindent
It is important to note that here again, the preacher's concern is not simply to
explain some fact about Christ's birth; instead he wants to prompt the faithful
to \emph{listen} to Christ's voice.

Fray Luis's image of the infant Christ as an orator encapsulates the trope of
Christ as the \emph{Verbum infans}.
For this passage Fray Luis cites \quoted{a doctor}, and in fact the whole passage is a
paraphrase from Saint Bernard of Clairvaux.
When the friar calls Jesus \emph{la palabra de Dios enmudecida} he is glossing the
term \emph{Verbum infans} in Bernard's fifth Christmas sermon.
Bernard expresses the trope in this way:

> But what kind of mediator is this, you ask, who is born in a stable, placed in
> a manger, wrapped in cloths like all others, cries like all others, in sum,
> who lies unspeaking as an infant \add{\emph{infans}}, just as others are accustomed to
> do?
> A great mediator he is indeed, even in this seeking all the things that are
> for peace, not just going through the motions but working effectively.
> He is an infant, but he is the infant Word \add{\emph{Verbum infans}}, and not even in
> his infancy does he keep silent.
[@Bernard:Nativitate, 128A, Sermo V:
\quoted{Sed qualis mediator est, inquies, qui in stabulo nascitur, in praesepio
ponitur, pannis involvitur sicut caeteri, plorat ut caeteri, denique infans
jacet, ut caeteri consueverunt?  Magnus plane mediator est, etiam in his
omnibus, quae ad pacem sunt, non perfunctorie, sed efficaciter quaerens.  Infans
quidem est, sed Verbum infans, cujus ne ipsa quidem infantia tacet}.]

\noindent
The term refers to Christ as the Word who is both \quoted{infant} and \quoted{unspeaking},
since like the Spanish \emph{infante} in Padilla's villancico, the Latin word
\emph{infans} can have either meaning.

Bernard is himself drawing on an older tradition of the theology of the Word
going back to Saint Augustine, who reiterates this trope throughout his dozens of
Christmas sermons.
Augustine's editors even title one sermon \emph{Verbum infans doctor humilitatis}---the
infant Word, teacher of humility.
[@Augustine:SermonesPL, 1004, heading for sermon 187.]
In another, Augustine preaches,

> \add{This day} is called the Nativity of the Lord, when the Wisdom of God
> manifested itself unspeaking \add{or, as an infant}, and the Word of God without
> words sent forth a voice of flesh.
> That divinity which was thus hidden, was both signified to the Magi by the
> witness of Heaven, and announced to the shepherds by an angelic voice.
> This, therefore, is the day whose anniversary we celebrate in our ritual.
[@Augustine:SermonesPL, 997, Sermo 185, In Natali Domini 2:
\quoted{Natalis Domini dicitur, quando Dei Sapientia se demonstravit infantem,
et Dei Verbum sine verbis vocem carnis emisit.  Illa tamen occulta divinitas, et
Magis coelo teste significate, et pastoribus angelica voce nuntiata est.  Hanc
igitur anniversaria solemnitate celebramus diem \Dots{}}.] 

\noindent
True to his background in Neoplatonic philosophy, Augustine builds a chain of
signification from the highest level of being to the lowly realm of human
experience.
He moves from the Godhead and its incarnation as the Christ-child to the signs
that pointed to this reality in the material world;
moving from divine Wisdom to the child in whom God manifested himself.
He then moves to the worldly signs that pointed to Christ at his birth, and
finally links this to the present-day ritual of the Mass of the Nativity.
As the master of a theology of communication through sign and signified (in
\emph{De
doctrina christiana}), Augustine is teaching the faithful to ascend in
contemplation through that same chain---to hear the words proclaimed in their
midst through Scripture, preaching, and liturgy as signs pointing to the angels'
song in Bethlehem; to hear and heed the angel's exhortation to the shepherds to
seek out the \quoted{sign} of \quoted{a child wrapped in swaddling clothes and lying in a manger} (Lk 2:12).
The Christ child they found was at this \quoted{infant} stage of life, not a speaker of
divine words, but was the very Word himself---as Augustine says, \quoted{a voice of flesh}.

The \emph{Verbum infans} trope fundamentally provides a way of thinking about
communication between God and humankind.
As the first-century author of the New Testament letter to the Hebrews explained
it, \quoted{Long ago God spoke to our ancestors in many and various ways by the prophets, but in these last days he has spoken to us by a Son \Dots{}.  He is the reflection of God's glory and the exact imprint of God's very being, and he sustains all things by his powerful word} (Heb. 1:1--3, NRSV). 
Augustine creates an extended metaphor of the voice to explore how exactly Christ
could be the Word incarnate, using human communication to understand divine
communication.
The peculiar features of the voice allow Augustine to defend the doctrine that
Christ was not \quoted{changed} into flesh, but \quoted{remained the Son of God} even \quoted{having been made the Son of Man}.
[@Augustine:SermonesPL, 1002, Sermo 187, In Natali Domini 4:
\quoted{manentem Dei filium, factum hominis filium}.]
The Word of God existed from eternity, Augustine teaches, drawing on his
learning in rhetoric and language (as also expressed in his treatise \emph{De
musica}).
In its eternal state it \quoted{was not varied by punctuation marks whether
short or long \add{\emph{nec morulis brevibus longisque}, a concept echoed in
Padilla's \emph{máxima y breve}}, nor drawn together by the voice, nor ended by silence}.
[@Augustine:SermonesPL, 1001, Sermo 187, In Natali Domini 4:
\quoted{Quanto magis Verbum Dei, per quod facta sunt omnia, et quod in se manens
innovat omnia; quod nec locis concluditur; nec temporibus tenditur, nec morulis
brevibus longisque variatur, nec vocibus texitur, nec silentio terminatur;
quanto magis hoc tantum et tale Verbum potuit matris uterum assumpto corpore
fecundare, et de sinu Patris non emigrare?}] 
But Christ as the Word took on flesh and became known to humans, Augustine
teaches, just an idea becomes a spoken word, without ceasing to be an idea.
In the form of a spoken word it can be communicated to those who hear it and
enter into their minds as an idea.
This is because for Augustine, a word first exists as pure thought before it is
spoken, but when uttered is \quoted{clothed in the voice}:

> A word \add{\emph{verbum}; or, thought} that we carry in the heart, when joined with a
> voice \add{\emph{vox}; or, speech, spoken word}, we bring forth to the ear, is not
> changed into the voice, but the whole word is assumed into the voice in which
> it proceeds, so that internally the idea the word makes intelligible remains,
> while externally the voice produces the sound that is heard.
> This word, then, brings forth in sound, what previously resounded in silence.
> The word, upon being made a voice \add{or, upon being spoken}, is not changed into
> the voice itself, but rather, remaining in the mind's light, and having
> assumed the voice \add{or speech} of flesh, it proceeds to the hearer, and does
> not leave the thinker.
> The word in silence is not thought by means of this voice \add{spoken word},
> whether it is Greek or Latin or whatever other tongue: but rather, the thing
> itself which is to be said, before all other differentiations of tongues, is
> understood in some naked manner in the chambers of the heart, from whence it
> proceeds, being spoken, to be clothed in the voice.
[@Augustine:SermonesPL, 1002, Sermo 187, In Natali Domini 4:
\quoted{Sicut verbum quod corde gestamus, fit vox cum id ore proferimus, non
tamen illud in hanc commutatur, sed illo integro ista in qua procedat assumitur,
ut et intus maneat quod intelligatur, et foris sonet quod adiatur: hoc idem
tamen profertur in sono, quod ante sonuerat in silentio; atque ita verbum cum
fit vox, non mutatur in vocem; sed manens in mentis luce, et assumpta carnis
voce procedit ad audientem, et non deserti cogitantem.  Non cum ipsa vox in
silentio cogitatur, quae vel graecae est, vel latinae, vel linguae alterius
cujuslibet: sed cum ante omnem linguarum diversitatem res ipsa quae dicenda est,
adhuc in cubili cordis quodam modo nuda est intelligenti, quae ut inde procedat
loquentis voce vestitur}.] 

\noindent
Augustine is reflecting on language as a way to understand how Christ can be in
his body the form of communication between God and
humankind.[^Augustine-further-word]
The spoken voice serves as the external medium through which human thought is
transferred from one person's inner heart and mind to that of another.
The body thus converts an inarticulate, abstract idea into a form that others
can perceive through the physical act of hearing.
The voice itself, then, is independent of the body; it transfers the idea
to the hearer without the idea leaving the speaker's mind.
Theologically, in Augustine's conception, Christ as the eternal Word is like
the inner thought before being expressed in speech; Christ as the Word made
flesh is like the human spoken word.
Thus the Christ-child embodies divine communication not through spoken words,
but through his very body.[^Augustine-signum-res]

[^Augustine-further-word]:
For further on Augustine's concept of the Word and its relationship to language
and rhetoric, see the sermons \emph{De verbis Domini} and the treatise \emph{De doctrina
christiana} (which connects voice and incarnation in the same way in I 13:26).
The theology of Christ as the Word is also exhaustively treated in Lapide's
commentary on the first chapter of John, @Lapide:Gospels19C, 872--889.

Augustine's theology of voice opens up rich possibilities for later interpreters
in the tradition to consider the Christ-child in specifically musical terms.
If Christ communicates God through his body, then one might imagine the 
Christ-child as an oration given by a master speaker, as Lapide does in his
commentary on Luke:
\quoted{We hear God teaching and preaching from the chair \add{\emph{cathedra}} of this manger, not by a word but by a deed: \Dots{} I have been made a little one, of your bone and your flesh, I am made man, in order to make you God}. [@Lapide:Gospels19C, 673, on Lk 2:
\quoted{Quid fecit tantus Deus, in tantilla carne, jacens in praesepio?
Audiamus ipsummet in praesepii cathedra, non verbo sed facto docentem et
praedicantem: \Dots{} Parvulus factus sum, os tuum et caro tua, factus sum homo,
ut te Deum efficerem}.] 
If the voice is an apt metaphor for Christ as the divine Word, conveying the
divine nature to humanity, then Christ's actual voice would communicate doubly.
And if Christ can be both orator and oration, then surely he can be both singer
and song as well.

As the villancico \wtitle{Voces, las de la capilla} presents these tropes, the
Christ-child is the masterwork that proves the craft of the divine
craftsman, imagined as a chapelmaster in the Spanish fashion.
Bringing together \emph{maxima} and \emph{breve}, \quoted{high} and low in the incarnate Christ,
God the Father \quoted{proves} that he can form \quoted{consonances between a man and God}
(copla 2).
Thus on one level Christ is God's song, and \quoted{the Word} is envisioned not as
speech but as music.
Christ, then, takes the \quoted{lyrics} of his ancestor David, royal chapelmaster, and
\quoted{sets them to notes} (poem l. 7) through his heroic, sacrificial life.
He himself holds the \emph{clave} (musical clef/key of authority); he himself is the
\emph{divisa} (sign); and it is Christ himself who is \emph{el signo a la mi
re}, \quoted{the sign of A}---the first note of a new song in which can be heard \quoted{the voice of the Father}. (ll.~9, 11, 24, 27).
This connects to a larger theological trope of Christ as the \quoted{new song} of
the psalms and the Apocalypse.^[Ps 33:3, 40:3, 96:1, 98:1, 144:9, 149:1; Is
42:10.; Rv 5:9, 14:3.]

This theological context can help us recognize that one of the most puzzling
lines in the poem is also a key to its meaning. 
When the poem refers to \quoted{the sign of \emph{A (la, mi, re)}} it makes Christ himself
the sign, and it imagines the sound of Christ's infant voice not as words but as
music. 
The \emph{signo} in this line of the villancico's estribillo connects back to the
\emph{divisa} of the poem's introduction.
Both words draw on theological traditions that see Christ the Word as a \quoted{sign},
and therefore even as a letter.
In his 1611 Spanish dictionary Sebastián de Covarrubias glosses the Spanish
\emph{divisa} with the Latin \emph{signum}, from the Greek \emph{sēma}, sign. 
Likewise, Lapide calls Christ \quoted{the sign of reconciliation of the human race to God}.
[@Covarrubias:Tesoro, \sv{divisa}; @Lapide:Gospels19C, 685--686, on Lk 2:
XXX]

What kind of a sign is Christ, then?
There is a strong tradition, extending back to words reportedly spoken by Christ
himself in the Revelation to John, of thinking of Christ as a letter.
In a Christmas sermon Augustine connects the concept of the Word in John 1 to
Christ's statement in Revelation, \quoted{I am \emph{alpha} and \emph{omega}, first and last, beginning and end} (Rv 22:13).
Augustine uses the first and last letters of the Greek alphabet to explain the
doctrine of the two conceptions of Christ, namely, that the Son of God was
begotten eternally of the Father, but born temporally of the Virgin.
Christ's status as the Word had no beginning, but there was a temporal beginning
to his life as a man.
\quoted{And just as no letter comes before \emph{alpha}}, Augustine preaches, nothing
precedes Christ or follows after him, \quoted{for he is God}.
[@Augustine:Opera1555, vol. 10, 118r, In Natali Domini 2:
\quoted{Si fides assit, aperta est ratio qua Christus nunc minor, nunc aequalis
patri sacris voluminibus asseratur, sicut ipse de se dicit, Ego primus \et{}
novissimus.  Indubitanter agnosce quòd priorem non potest habere qui primus est.
Item, Ego sum $\alpha$ et $\omega$. Et sicut alpha litera nulla praecedit, ita \et{} filium dei
nulli secundum constat esse, quia deus est}. See also @Lapide:Apocalypse1627, on
Rv 1.]
In the villancico's terms, he is \quoted{the one who is before time} (l. 24).
The \emph{divisa}, then, signifies the start of a new \emph{tiempo} (l.  11, 23).
To use Augustine's concept of voice and word discussed earlier, then, the
beginning of Christ's human life is the moment when idea is \quoted{clothed in the voice} and communication becomes possible.

In the villancico, the reference to Christ as the letter \emph{A} has more than just
the typical symbolic associations of \emph{alpha} and \emph{omega}, since the piece makes
explicit that the \quoted{sign} is a musical pitch, \emph{A (la, mi, re)}---the note at the
center of the Guidonian hand.[^Guidonian-hand]
Padilla makes this symbolism even stronger through his literal puns of
solmization on the note \emph{A} and the syllables \emph{la}, \emph{mi}, and \emph{re}.
If Christ himself is the sign, and if he is both singer and song, then the \quoted{sign of A} is his singing voice---the cries of the infant considered as music.
The theological sources of the \emph{Verbum infans} trope do not all depict the
Christ-child's crying voice.
Much like in the English carol that says \quoted{little Lord Jesus, no crying he makes}, Bernard and Fray Luis say the Word speaks even while silent; Luis
and Lapide describe Christ as an orator but make clear that Christ himself \emph{is}
the oration; and Augustine's treatment of the voice is really a metaphor for
Incarnation.
But the concept of \quoted{the sign of A} in the villancico, understood within the
\emph{Verbum infans} theological trope, opens up the possibility of interpreting the
newborn Christ's inarticulate cries in musical terms as an expression, not of
verbal concepts, but of his physical being.
In fact, there is evidence that Spanish Catholics heard a special theological
significance in the sound of a baby's first cry, and that they believed a boy's
first cry to be literally the sound \emph{A} (pronounced like \emph{ah}).
Covarrubias begins his Spanish dictionary with this definition of the letter A:

> The first letter in order according to all the nations that used characters,
> \Dots{} and this because of its being so very simple in its pronunciation
> \add{\emph{prolación}}. \Dots{} And thus it is the first thing that man pronounces in
> being born, except that the male (since he has more strength) says A, and the
> female E \add{pronounced \emph{eh}}, in which man seems to enter into the world,
> lamenting his first parents Adam and Eve.
[@Covarrubias:Tesoro, \sv{A}:
\quoted{Primera letra en orden cerca de todas las naciones que usaron
caracteres, \Dots{} y esto por ser simplicissima en su prolacion. \Dots{} Y assi
es la primera que el hombre pronuncia en naciendo, saluo que el varon como tiene
mas fuerça dize A, y la hembra E, en que parece entrar en el mundo, lamentandose
de sus primeros padres Adan y Eva}. 
On symbolic alphabets in early modern devotional music, see @Kendrick:Jeremiah,
38--40.]

\noindent
By this account, keeping in mind Fray Luis's description of Christ's newborn
cries, the baby Jesus first cried out with the inarticulate vowel \emph{A},
expressing in this sound his essenced as \emph{alpha} and \emph{omega}, as incarnate Word.
Of course, A is also the first of the musical tones alphabetically, and may have
been used as a common tuning pitch, as today.
% XXX  evidence  
The sound of his voice in this context expressed Christ's identity as the son of
Adam, while also serving as a tuning note, or intonation, for a new song to
replace the \quoted{wandering song} given to \quoted{the first man} (l. 34--35).
All the other voices of Christmas follow after and echo the voice of Christ.
Thus Christ in his cries performs the song that he himself is.
What the worshippers heard in the baby's cries was \quoted{the voice of the Father/ singing in tones of weeping}, a \quoted{song/ as much to hear as to admire} (ll.
27--30).
This was not just an \emph{admirabile sacramentum} as in the Responsory, but an
audible sacrament as well---the sign of the bodily presence of the divine
entering into their ears through sounds that were more like music than speech.

[^Guidonian-hand]:
In Guido's mnemonic system, all the notes that could be sung from \emph{Gamma-ut}
upwards (\pitch{G}{2} to \pitch{E}{5}) were laid out in a spiral on the upraised palm of the
left hand.
The \emph{A (la, mi, re)} above \quoted{middle C} (that is, \pitch{A}{4}) was indicated by pointing
to the inside of the second knuckle of the middle finger.
Speculatively, there is a common type of image of the infant Jesus, a visual
expression of the \emph{Verbum infans} trope, in which the child sits on his mother's
lap, mouth closed, with his hand raised in a gesture of blessing: typically this
means the index and middle fingers are raised and the fourth and fifth fingers
are curled down.
Could the \quoted{sign of A} be a reference to this specific type of image, a Guidonian
reading of that hand position?


Musical performance provided an apt medium to realize this concept, as the
singing and playing gave material form to the concepts of the poetry and
illustrated those concepts through the artifice of the music.
Augustine, Fray Luis, and Lapide all connect the present-day liturgy of
Christmas to the historical music of the angels at Christmas and ultimately to
Christ himself as the divine Word, \emph{Verbum infans}.
Likewise, the villancico poem begins by invoking not angelic choirs or the
musical menagerie of the manger, but rather calls listeners' attention to the
\quoted{voices of the chapel choir}, asking them to heed \quoted{what is sung} (\poemlines{1--2}).
Only through \quoted{putting notes to his lyrics} can the composer fully enact the
concept of theological hearing central to the text.
Listeners are asked to pay attention not only to the words, but to \quoted{what
is sung}---which includes the sonic whole of the composition with its plays of
solmization, rhythm, stylistic allusion, and contrapuntal devices.
Through the voices of the Puebla chapel choir, the faithful could listen for the
unhearable higher music of Christmas, which ultimately meant encountering Christ
himself.


\section{Establishing a Pedigree in a Lineage of Metamusical Composition}

The high level of ingenuity, both theological and musical, in this villancico
makes \wtitle{Voces, las de la capilla} one of Juan Gutiérrez de Padilla's
master-works, in the early modern sense of a piece that proves the artisan's
mastery of his craft.
As such, the piece served a social function in addition to being an object of
devotion.
For Padilla's fellow musicians, chapelmasters, and the educated elite of Puebla,
the piece demonstrated Padilla's skill and established his place in a tradition
of composition.
Padilla's setting is one link in a chain of homage and emulation, within a
specific family of villancicos as well as within the broader subgenre of
metamusical pieces.
In the following chapters we will see more examples of the same patterns of
adaptation.
In the case of \wtitle{Voces, las de la capilla}, evidence survives for two previous
villancicos based on the same or similar poems, though the music for both is
apparently lost.
These pieces allowed Spanish composers to prove both their compositional craft
and their acuity as literary and theological interpreters in a tradition of
\quoted{music about music}.

Prior to Padilla's 1657 piece, there are sources for at least two earlier
versions. 
As shown in \cref{tab:Voces-Cantores-settings}, there are two
sources for a pre-1644 version beginning \wtitle{Voces, las de la capilla}: a 1649
catalog entry for a setting by Francisco de Santiago (then chapelmaster of
Seville Cathedral) from the collection of Portuguese King John IV, and a
1642 poetry imprint of a performance, possibly of Santiago's setting, by the
royal chapel in Lisbon.
There is also a source for a variant version of the text, \wtitle{Cantores de la
capilla}, performed for Epiphany 1647 at Seville Cathedral and probably composed
by Santiago's successor, Luis Bernardo Jalón.
This 1647 Seville imprint survives in a single copy as part of a binder's
collection in Puebla.

(include \cref{tab:Voces-Cantores-settings})
\label{tab:Voces-Cantores-settings}

The earliest known musical setting of this textual tradition is by Francisco de
Santiago.
This Carmelite friar was born with the surname Veiga in Portugal and
served as chapelmaster at Seville Cathedral from 1617 (succeeding Alonso Lobo)
to 1643 (succeeded by Luis Bernardo Jalón).
[@Stevenson:SantiagoF; @Perez:DMEH-Santiago]
% XXX DMEH sv Sevilla? 
Santiago had maintained a lifelong association and correspondence with the Duke
of Braganza, who after 1640 reigned as King John IV of Portugal.
Santiago obtained permission from the Seville Cathedral chapter to visit Lisbon
every five years (1625, 1630, 1635, and 1640).
Whether Santiago wrote specifically for the royal chapel in Lisbon, or
simply brought John IV copies of his music from Seville, the Portuguese monarch
acquired a collection of five hundred thirty-eight villancicos by Santiago, not
to mention other musical genres.
In 1649 John had a catalog printed of his collection, which is now all that
survives after the fires that followed the Lisbon earthquake of 1755.
Among the \quoted{Christmas Villancicos of Fray Francisco de Santiago} in the catalog
appears the following entry:
\quoted{Vozes las de la capilla. solo. Ya trechos las distancias. a 9}. 
[@JohnIV:Catalog, caixão 26, no. 674; see also @Ribeiro:JohnIV.]
Unfortunately only this description remains; the music parts probably perished
along with the rest of the collection in the fires that followed the Lisbon
earthquake of 1755.

Unfortunately only this description remains of Santiago's setting, but an
imprint from the Portuguese Royal Chapel, Christmas 1642, may document a
performance of the same piece.[^Lisbon-1642-pliego] 
The imprint, as was typical, does not list the composer or poet's names, so it
may also be a different setting of a variant texts with the same incipits.
In support of the argument that the imprint corresponds to Santiago's setting is
the evidence that in 1640, the year Santiago last visited Lisbon, an imprint
records the performance by John IV's chapel of another villancico, \emph{Antón
Llorente y Bartolo}, which would also later be included in the 1649 catalog
among the works of Santiago.[^Anton-Llorente]
On the other hand, Santiago was probably in Seville in 1642, beginning to suffer
the paralyzing illness which would soon end his life.
In 1643 the chapter hired Luis Bernardo Jalón to substitute for Santiago that
Christmas, and Santiago died in October 1644.
[@Ezquerro:JalonLB]
% XXX confirm that Santiago was in Seville in 1642 

[^Lisbon-1642-pliego]:
\emph{Villancicos qve se cantarão na real capella do muyto alto, \et{} poderoso Rey Dom
Ioamo IIII. nosso senhor. Nas matina da noite do Natal da era de 1642} (Lisbon,
1642), P:Ln RES-189-3-P (no. 2)

[^Anton-Llorente]:
L:Pn RES-189-1-p (no. 2); @JohnIV:Catalog, caixão 26, no. 675.
The Sánchez Garza collection, originally from the Conceptionist Convento de la
Santísima Trinidad in Puebla, includes an anonymous musical setting of \wtitle{Antón
Llorente y Bartolo} (\sig{MEX-Mcen}{CSG.014}; see also \cref{ch:intro}).
The piece matches the incipits of a Santiago setting in the catalog of John IV;
the words are an abridged and adapted version of the text performed by the
Lisbon royal chapel in 1640 when Santiago was visiting (as noted above).
The music in the Puebla collection is thus probably not by Santiago, but by a
New Spanish composer who adapted the same text for use in Puebla.
The same collection features several other villancicos by Padilla, and it is
likely that he knew this setting (if he was not its composer).

% XXX new catalog no. 


Jalón, as chapelmaster of Seville Cathedral, was probably the composer for the
next known villancico in this family, the poem \wtitle{Cantores de la capilla},
performed there for Epiphany 1647.
[^Jalon-Cantores-pliego]
Jalón's position at Seville was the culmination of a restlessly ambitious
career.
[@Ezquerro:JalonLB]
He had left his post at the Convento de la Encarnación in 1623 to be
chapelmaster of the cathedrals of Burgos (1623--1634), Cuenca (1634--1642), and
Toledo (1642--1643).
On November 10, 1643, he was appointed chapelmaster at the cathedral of Santiago
de Compostela, but dropped everything a month later when he was invited to
assist the ailing Santiago at Seville Cathedral for the Christmas season of
1643--44.
The two composers must have had close contact during that time, and in any case,
Jalón could have had access to Santiago's music in the cathedral archive.
Jalón's setting of \wtitle{Cantores} was performed only two years after Santiago's
death (considering Epiphany 1647 as part of the 1646--47 liturgical season), and
it is plausible that Jalón composed the piece deliberately in homage to his
predecessor.

[^Jalon-Cantores-pliego]:
\emph{Villancicos qve se cantaron en la S. Iglesia Metropolytana de Seville, en los
Maytines de los Santos Reyes.
En este año de mil y seiscientos y quarenta y siete} (Seville, 1647), Puebla,
private collection, courtesy of Gustavo Mauleón Rodríguez.

\subsection{Adaptation and Homage}

Compared to \wtitle{Cantores}, Padilla's 1657 text \wtitle{Voces, las de la capilla} is much
closer to that in the 1642 Lisbon imprint and matches the incipits of Santiago's
pre-1644 setting in the John IV catalog (\cref{tab:Voces-versions}).
Padilla's use of this earlier text tradition demonstrates that some source
crossed the Atlantic to Puebla, whether Padilla acquired it before emigrating in
1622 or after.
The Puebla chapelmaster may have had access to the text from a source like the 1642
Lisbon print, or he may have known an even earlier source.
It is possible he knew Santiago's version specifically.
Padilla almost certainly knew Jalón's later \wtitle{Cantores} text, since the only
extant copy of this Seville print survives in a binder's collection in Puebla.

(insert \cref{tab:Voces-versions})
\label{tab:Voces-versions}

The 1647 \wtitle{Cantores} text corresponds closely to the text both versions of
\wtitle{Voces, las de la capilla}.
The first four lines are the same, except that Jalón's text has \emph{Cantores}
(singers) instead of \emph{Voces} (voices) and \emph{Niño} (child) instead of \emph{Rey}
(king).
The first two coplas of \wtitle{Cantores} are identical to those in the 1642 \wtitle{Voces}.

At the same time, \wtitle{Cantores} also significantly modifies the text to suit a
different performance context and aesthetic goals.
Verses are added in the estribillo and a new copla is included that explicitly
reference the Three Kings, suitable for the performance of \wtitle{Cantores} at
Epiphany in Seville.
The end of the introduction in \wtitle{Voces} (\emph{Por sol comienza una Gloria}) is
moved to serve as the final copla in \wtitle{Cantores}.
The whole \emph{eco} section at the end of the 1642 \wtitle{Voces} is omitted, so that the
estribillo ends with the couplet \emph{Todo en el hombre es subir/ y todo en Dios es
bajar}.

In terms of theological and musical conceits, \wtitle{Cantores} reads like an attempt
to take the dense \emph{conceptismo} of \wtitle{Voces} and both simplify and explain it.
The connection of Christ's voice (\quoted{the sign of A}) and the \quoted{voices of the chapel choir} is obscured, as the opening is changed to \quoted{singers of the chapel choir}.
The crucial lines from \wtitle{Voces}, \emph{a la voz del padre oí/ cantan por puntos de
llanto}, are missing as well, which makes it harder to make sense of \emph{el signo a
la mi re}.
The connection between David and Christ as musician-kings is absent as well.
\quoted{The King is a musician} in \wtitle{Voces} is changed to \quoted{the child is a musician}---so
that Christ is now explicitly the creator of the music rather than himself being
the Music \emph{and} the musician.
By removing the \emph{divisa} passage, the connection between \emph{divisa} and
the \emph{signo A} is lost.
Throughout the poems, where \wtitle{Voces} has an ambiguous or cryptic line, \wtitle{Cantores}
has a less multivalent one. 
Instead of \emph{y aguarda tiempo oportuno/ quien antes del tiempo fue} \wtitle{Cantores}
has \emph{después que aguardaron uno/ que llegó a tiempo oportuno/ quién antes
del tiempo fue}.
The \wtitle{Voces} version of these lines is pithy but obscure; the \wtitle{Cantores}
version is crystal-clear but requires an extra line to say the same thing.
More significantly, when the compositor of \wtitle{Cantores} replaces \emph{a la voz del
padre oí} with \emph{con que mil maravillas vi}, he has replaced the central reference
to the act of hearing (\emph{oí})---the only first-person active verb in the
poem---with seeing (\emph{vi}), thus obscuring the poem's central concept. 
Instead of listening to voices (\emph{oí}, \emph{voces}), the speaker of \wtitle{Cantores} is
looking at singers (\emph{vi}, \emph{cantores}).

The \emph{conceptismo} in \wtitle{Cantores} is not as tightly bound as in \wtitle{Voces}: in many
cases, \wtitle{Cantores} makes sense on the musical side but not on the theological
side.
Some of the musical terminology is deployed innaccurately, as when \wtitle{Cantores}
has Christ the composer writing in \emph{compás mayor} in a \emph{proporción abreviada}
using a \emph{clave con tres tiempos}---it is hard to know which actual meters might
be indicated here.
The poet of \wtitle{Cantores} writes multiple lines like \emph{O que lindamente suenan!/ o
que dulcemente cantan} that do not advance the conceit, where the poet of
\wtitle{Voces} restrained such effusions to the four-syllable \emph{O qué canto}.
The new poet has retained the technical terms and other key words used in the
first version of the poem, but has attempted to explain the metaphors, sometimes
in ways that change the meaning from the first poem.

Padilla's 1657 text, by contrast, is much closer to the 1642 \wtitle{Voces}.
There are only a few major differences.
First, Padilla groups the verses in the introduction as a six-line strophe
followed by a four-line \emph{respuesta}, which is then repeated after the rest of
the introduction verses. 
The 1642 text confirms the argument advanced above that the text of Padilla's
\emph{respuesta} (the first time through) is meant to follow the sixth verse, just as
it is notated in the musical manuscripts.
Second, Padilla, like Jalón, omits the \emph{eco} portion of the 1642 text, ending
with \emph{Todo en Dios es bajar}. 
Padilla does not use the same coplas as in the 1642 Lisbon or 1647 Seville
prints (beginning \emph{A suspensiones el cielo}), but rather includes two
completely different coplas (beginning \emph{Daba un niño peregrino tono}).

It is possible that Padilla (or whoever produced Padilla's poetic text), being
familiar with Jalón's text as well as at least one earlier version, made a
composite version that drew on both earlier variants.
On the other hand, Padilla's text is so conceptually consistent, and tightly
patterned metrically, that an argument can be made that his version of \wtitle{Voces}
actually reflects an earlier version of the text than that in \emph{either} of the
two previous imprints.
We have already noted that \wtitle{Cantores} modifies both the style and content of the
earlier \wtitle{Voces}.
But there are also elements of that 1642 version that do not seem to fit with
the core of the text preserved Padilla's later setting.
Padilla's reading \emph{la clave que sobre el hombro}, quoting Isaiah 22:12, is
certainly superior to the 1642 reading \emph{la clave que sobre el hombre} (though
this could be the publisher's mistake).
Similarly, the astronomical and musical term \emph{distancias} in Padilla's text
makes more sense in the conceit of the poem than the Lisbon \emph{estancias}.
\wtitle{Cantores} has these same readings, and this suggests not that Padilla copied them
from \wtitle{Cantores} but that both versions are based on an earlier source than the
1642 print.

The metrical patterning of the three texts also suggests that Padilla's version
reflects an earlier stage of the tradition.
First, Padilla's text features a \emph{respuesta} section: this structure was more
commonly used in villancicos before 1640, though Padilla, now in his senior
years, continued to use the form in the 1650s.[^respuesta]
Second and most crucially, the line \emph{Y a trechos las distancias} at the
beginning of the estribillo serves as a \emph{linea de vuelta} (hinge line) in
Padilla's text: it connects to the end of the coplas (\emph{de un hombre y Dios
consonancias}) and rhymes when the estribillo is repeated after that verse.
The \emph{linea de vuelta} was a holdover from the rather different structure of the
courtly villancicos of the sixteenth century, such as those set by Juan del
Encina.
[@Navarro:Metrica] % XXX page no 
The 1642 Lisbon \wtitle{Voces}, and the catalog entry for Santiago's \wtitle{Voces}, both
contain this same line (\emph{Y a trechos las distancias}), but the Lisbon version
has different coplas from Padilla, and the end of the coplas does not rhyme with
\emph{distancias}, so that the \emph{linea de vuelta} structure is absent.
The \quoted{new} coplas in Padilla's version are much more thematically consistent with
the conceit of God as a musician and the newborn Christ as his masterwork, and
the connection to the \emph{linea de vuelta} seems hard to explain as a later
addition.

[^respuesta]:
Gutiérrez de Padilla, villancicos for Corpus Christi 1628, \wtitle{Salir primero de ti}
(\sig{MEX-Pc}{Leg. 1/1}); Christmas 1653, \wtitle{A siolo Flasiquiyo}
(\sig{MEX-Pc}{Leg. 2/1}).

The rest of the estribillo in Padilla's version, up to \emph{Todo en Dios es bajar},
is highly patterned in a series of fully rhyming verse pairs, all in
eight-syllable lines.
The half-line \quoted{uno a uno} forms a bracket with the other half-line \quoted{O qué canto}, marking off this section of paired verses from the quatrain that
follows, a \emph{redondilla abrazada}.
The line \emph{Y a trechos} does not fit in this pattern; but it makes sense as a
\emph{linea de vuelta} only in Padilla's version, and it is absent from \wtitle{Cantores}.

By contrast, the estribillo of \wtitle{Cantores} begins with two lines with neither
assonance nor rhyme, followed by a line that rhymes with nothing and can only
scan as eight syllables if \emph{que} and \emph{lleva} are elided; the fourth line can
only be read as ten syllables.
The next four lines group in pairs with full rhymes, alternating lines of six
and eleven syllables.
These metrical irregularities are confined to this first portion of the
estribillo; after this the remainder is almost identical to \wtitle{Voces}.
Thus the first section appears \quoted{tacked on} to the more refined pre-existing
material in the second section.

Similarly, the 1642 Lisbon \wtitle{Voces} ends with an \emph{eco} section in a completely
irregular meter, thematically almost unrelated to concept of the rest of the
poem.
Starting with \emph{y con el favor usanos} there is are two eight-syllable rhyming
couplets plus a three-line rhyming group (with their echoes), then lines of
five, seven, and nine syllables, in no clear pattern, with an \emph{abbccada} rhyme
scheme.
This sloppy section also seems like a superflous addition compared to the
concise \emph{conceptismo} and elegant metrics of Padilla's version.

Until more sources come to light it is not possible to know for certain whether
Padilla's text preserves a pure earlier source or whether it is an
ingenious improvement on the previous variants.
It resolves certain metrical irregulaties and conceptual inconsistencies so
elegantly that it seems to have more integrity as a source than the others.
There is of course the danger of confirmation bias, especially since Padilla's
musical setting articulates its poetic structure and interprets its theological
conceits in a way that makes that version seem authoritative.

\subsection{Demonstrating Musical Kinship}

In sum, the 1642 and 1657 villancicos \wtitle{Voces, las de la capilla}, though not
identical, form one primary branch of this textual family, and the 1647
\wtitle{Cantores de la capilla} forms another.
Compared to the older \wtitle{Voces} branch, the \wtitle{Cantores} branch is more simplistic,
more irregular metrically, and less coherent.
These differences suggest that \wtitle{Cantores} was adapted from \wtitle{Voces} with the goal
of tempering its Góngora-like difficulty, and that is is not an especially
skillful adaptation.
There is reason to believe that the three composers known to have set texts in
this family---Francisco de Santiago (\wtitle{Voces}, before 1644), Luis Bernardo Jalón
(\wtitle{Cantores}, 1647), and Juan Gutiérrez de Padilla (\wtitle{Voces}, 1657)---were
personally connected in a network of musicians, and that Jalón and Padilla chose
intentionally to set a version of this text in order to establish kinship with
Santiago.
Jalón was Santiago's successor at Seville Cathedral and composed his \wtitle{Cantores}
setting only two Christmas--Epiphany seasons after Santiago died.
Padilla probably knew Santiago personally from his early career in Andalusia;
his 1657 setting of \wtitle{Voces} has the same incipits as Santiago's
setting in in the 1649 catalog of John IV, and the text is closely related to
the version performed in Lisbon in 1642.
These chapelmaster deliberately chose to set their particular versions in order
to situate themselves differently in relation to the earlier master (table
\ref{tab:Voces-connections}).

(insert \cref{tab:Voces-connections})
\label{tab:Voces-connections}

Padilla almost certainly knew Jalón's adapted version of this text because the
Seville imprint circulated to Puebla, where the only exemplar survives today.
Padilla regularly used texts from such imprints: for example, he took
the text for his 1653 \wtitle{A la jacara, jacarilla} from a Madrid Royal Chapel
imprint of a year earlier.^[\sig{E-Mn}{VE/88/55}.]
A binder's collection in Puebla contains numerous peninsular villancico imprints
with other correspondences to Padilla's settings.^[\sig{MEX-Plf}{80070-42010404}]
Padilla likely kept his own collection of imprints (if not this actual
collection) as sources for composition, keeping current with the latest mainland
trends.
If Jalón's villancico was a modernization of Santiago's, then, and the text was
available to Padilla in New Spain through the imprint, why did Padilla choose to
set Santiago's original text rather than the new one set by Jalón?
One possibility is that Padilla specifically wanted to differentiate himself
from Jalón and associate himself more directly with Santiago.
With his setting of \wtitle{Voces, las de la capilla}, Padilla may have been trying to
show that he, not Jalón, was the true successor to Santiago.

There are several ways that Padilla might have known Santiago's setting of
\wtitle{Voces} or the poem on which it was based.
It is quite likely that the two chapelmasters knew each other personally from
Padilla's early career in Andalusia.
Padilla, baptized in Málaga in 1590, climbed his way rapidly up the ladder of
prestigious positions in the region, moving from positions at Ronda (1608--1612)
and Jerez de la Frontera (1612--1616) to the cathedral of Cádiz, where he served
as chapelmaster from March 17, 1616 until he emigrated to New Spain in 1622.
[@Gembero:Padilla]
Padilla's years in Cádiz overlap with Santiago's tenure in Seville (1617--1643),
leaving about six years when the two chapelmasters could easily have interacted
either personally or through correspondence.
The two port cities, about seventy-five miles apart (a few days' journey by
horse), shared close economic and social links.
Both cities were among the first to print leaflets of villancico poems, and
Padilla could have had access to the texts set by Santiago through this medium
alone.
[@BNE:VCs17C, \sv{Cádiz} and \emph{Sevilla}]

Even if there was no personal connection between Padilla and Santiago, the
Seville composer's position at the helm of the flagship music program in the
Hispanic world (in fact, the mother church for all of the Indies), would have
made him a prime target for emulation, homage, or competition.
In fact, Santiago's appointment at Seville may have even precipitated Padilla's
departure for the New World.
Santiago was hired at Seville without any public competition for the position,
which would have been the pinnacle of achievement for any Spanish composer,
particularly one from Andalusia.
Padilla had just started his position in Cádiz---not a lowly post---but the
appointment of the relatively young Santiago would have deprived him of any
hopes of further advancement on the Spanish main.
Other prominent posts in Madrid were taken, with Mateo Romero at the Royal Chapel
since 1598, and at the prestigious Convento de la Encarnación, none other than
Luis Bernardo Jalón.
[@Ezquerro:JalonLB] % XXX Capilla real? 
This may have been part of the reason why in about 1622, Padilla sought better
opportunities in America.

Santiago, Padilla, and Jalón were not only part of a professional network; their
relationships with other chapelmasters may be best understood in terms of
kinship.
Most Hispanic chapelmasters were priests or men of religious orders, under vows
of celibacy; so in many cases they established genealogical lineages through
their musical relationships.
All of these men learned their craft through apprenticeship, and all of them
passed on their learning in the same way once they became the \emph{maestros}.
The links between these men and their elders began in boyhood: Padilla, for one,
fulfilled a common requirement of housing and educating his choirboys in his own
home.
^[More research is needed into likely high prevalence and impact of sexual abuse
in this environment.]
Chapelmasters were linked to their predecessors by a common practice of
assisting the incumbent in his final years, before succeeding to the position.
Parallel lines of succession with such interim assistantships in Puebla and
Sevilla are shown in \cref{tab:Puebla-Seville-MCs}.

(insert \cref{tab:Puebla-Seville-MCs})
\label{tab:Puebla-Seville-MCs}

In Padilla's case, his apprenticeship began by serving as choirboy and cantor at
Málaga Cathedral and then assistant to the local chapelmaster Francisco Vásquez
(ca.  1602--1608); unfortunately he was bested by a more experienced composer in
the competition to succeed his teacher.
[The Portuguese composer Estevão de Brito won the \emph{oposición} at Málaga.
@Gembero:Padilla; @Stevenson:BritoE]
A new stage of apprenticeship began after Padilla emigrated to New Spain, when
he was selected in 1628 to assist the ailing chapelmaster of Puebla Cathedral,
Gaspar Fernández.
[@Morales:Fernandez]
He composed the villancicos for Corpus Christi in that year, probably both as a
way to fill in for Fernández and to prove his own mastery.
[@Cashner:Cards]
He then succeeded Fernández after he died.
Now the master, Padilla cultivated his own apprentice in Puebla, Juan García de
Céspedes.
When Padilla's health began to fail in 1660, he signed a \quoted{power of attorney}
document giving legal rights to García, a member of the Puebla ensemble whose
name appears throughout Padilla's partbooks.
García then succeeded Padilla after his death in 1664.
[@Mauleon:PadillaCivil, 237--238]

The Seville chapelmasters established a kinship-like lineage in the same way.
Francisco de Santiago first served as assistant in Alonso Lobo during his old
age in 1616. 
[@Stevenson:SantiagoF]
Santiago composed the \emph{chanzonetas} (villancicos) for Christmas that year, and
then succeeded Lobo after his death in 1617.
When Santiago's time came and he was debilitated by a paralyzing medical
condition, the chapter called on Luis Bernardo Jalón to provide the music for
Christmas 1643, and Jalón inherited Santiago's position the following year.

In these cases the lines of succession were established either by the composers
themselves or by the cathedral chapters, surely with the older master's
approval, since all these composers had at least a year to orient and train
their successors.
But musicians could also voluntarily demonstrate kinship even when they were not
made direct heirs. 
Younger musicians could deliberately affiliate themselves with teachers and
paragons through composing musical homages (\cref{tab:Padilla-homages}).
Francisco Vidales, organist in Padilla's Puebla chapel, demonstrated his
affiliation to Padilla by writing a parody mass based on a motet by Padilla.
[@Koegel:Padilla: 
\quoted{The Puebla organist Francisco de Vidales used Padilla's \wtitle{Exultate
justi in Domino} as the model for his parody \wtitle{Missa super Exultate}, and another
connection between the two men is seen in Vidales's addition of a tenor part to
Padilla's \wtitle{O Domine Jesu Christe}}.] % XXX confirm 
Dianne Goldman has documented a similar chain of homage in several stages of
reworking a Victoria motet at Mexico City Cathedral through the eighteenth
century.
[@Goldman:StileAntico]
Homage composition of this kind was a widespread way for composers to
demonstrate both real and aspirational kinship with a mentor, teacher, or
paragon.

(insert \cref{tab:Padilla-homages})
\label{tab:Padilla-homages}

It makes sense, then, that Jalón would adapt a text set by Santiago in a way
that presented him as Santiago's heir, acknowledging influence while also
setting himself apart.
The modifications in \wtitle{Cantores de la capilla} suggest that Jalón represented a
new generation and a new style, and his music was likely in a more modern style.
For the sixty-seven-year-old Padilla, who as a young man had not been able to
capture the coveted position of Seville chapelmaster in 1616, setting the older
text may have established Padilla's affinity for the older generation and its
style.
Padilla's madrigalesque text setting and strict counterpoint suggest more
old-fashioned musical tastes compared to the new sounds of music from Madrid in
the 1650s, like the experimental music-drama collaborations of Juan Hidalgo
and Pedro Calderón de la Barca.
[@Stein:Songs]

This practice of affiliating oneself through musical homage may also shed light
on the origins and meaning of Padilla's most performed piece of Latin-texted
music today, the \wtitle{Missa \quoted{Ego flos campi}} for eight voices.
[\sig{MEX-Pc}{LiPol XV}; @Padilla:MissaEgoFlosCampi; recent recordings in
@Mauleon:PadillaPalafox; @Skidmore:NewWorldCD; along with many recent
performances.]
The piece would appear be a parody mass, and one might assume it is based on a
lost motet \wtitle{Ego flos campi} by Padilla.
But John IV's catalog lists another \wtitle{Missa \quoted{Ego flos campi} a 8}---by Francisco de
Santiago.
Moreover, the catalog specifies that Santiago's mass was based on the motet
\wtitle{Ego flos campi a 8} by Nicolas Dupont, a Flemish composer in the Spanish Royal
Chapel.
[@JohnIV:Catalog, 417, caixão 34, no. 787: 
\quoted{Missas \Dots{} Ego flos campi, a 8. \emph{Ferta sobre hum Motette de
Niculas du Pont}}; @JohnIV:Catalog, 381, caixão 32, no. 767: \quoted{Ego flos campi, a 8, Niculas du Pont. \emph{De Nossa Senhora}}.
(Neither piece has been found.)
Based on this connection, Robert Stevenson speculated that Santiago, like his
peer Diego de Pontac, studied with Nicolas Dupont during Santiago's time in
Madrid; -@Stevenson:SantiagoF.]
For Santiago to use this music by Dupont, who may have been one of his teachers
in Madrid, was to connect himself to the august lineage of Hapsburg
Franco-Flemish composers going back to Ockeghem and Dufay.
It is possible, then, that Juan Gutiérrez de Padilla's \wtitle{Missa \quoted{Ego flos campi}}
is not based on his own motet, but on Santiago's mass of the same name, or even
on the same Dupont source that Santiago had used.
This would be another instance, then, of Padilla establishing his musical
pedigree through homage composition, specifically in connection to Santiago.


\section{\quoted{All Who Heard It Were Amazed}}

The metamusical villancico provided composers an ideal opportunity to establish
their own position in a tradition of compositional and theological
ingenuity---that is, a tradition of musical theology.
The next chapter will show another example of this, in eight known settings of
villancicos from the same textual family by a network of interrelated composers
across Spain and into the New World.
The music's social function of establishing kinship with other composers through
an impressive display of craft overlaps with the social and theological function
of inspiring listeners to awe and wonder at Christmas.

For this study, Padilla's \wtitle{Voces} serves as the starting point for a trajectory
of \quoted{singing about singing} in seventeenth-century Spanish devotional music.
The poem and music represent sounding music---including the \quoted{present} music in
Puebla as well as the historical music of the angels and shepherds at
Bethlehem---as \emph{musica instrumentalis}, fully in accord with the Neoplatonic
system known from Boethius and early modern sources.
The musical performance invites listeners to contemplate the higher forms of
music, and above all the theological \quoted{Music} of the incarnate Christ, bringing
God and Man into consonance and forming in his own body and voice the perfect
\emph{musica humana}.

In singing this villancico about singing, the Puebla chapel choir drew its
listeners' attention to the fact that they were singing.
The musicians emphasized the musical artifice, and so encouraged listeners to
contemplate the higher levels of music.
The listeners would have the opportunity to listen for the voice of the spheres,
the angels, and most importantly, of Christ himself through the voices of the
chapel choir.
These higher forms of music would be audible only by faith, through the present
cathedral music which was its echo.
The exhortation in the opening line means that the Puebla \quoted{chapel choir} is to
set its time and tuning based on these higher forms of music, and that listeners
are therefore invited to \quoted{pay attention} as well to the verbal and musical
riddles presented to them.
Far from being a one-off occasional piece or crowd-pleasing entertainment, and
rather than being a simple didactic piece explaining doctrine, the complex
poetic and musical structure provides an object for aesthetic reflection as a
spiritual discipline, even inviting detailed analysis.
% respond to @DellAntonio:Listening here XXX 

This type of piece, then, encourages faithful hearing---hearing that is tempered
by faith and which serves as a tool for exercising faith.
But as with the pieces about faith and sensation discussed in chapter
\ref{ch:faith-hearing}, this kind of hearing must be trained and carefully
cultivated. 
In order to contemplate the piece in the most spiritual way, the listener must
be equipped with both faith and knowledge: faith to seek the higher theological
meaning behind the words and music, and knowledge to understand it.
While Padilla wrote many villancicos that would have been easy for common people
to appreciate, this one seems to address the most sophisticated listeners in the
congregation.
The poem's double and triple conceits, some of them based on references to Latin
Scriptural and liturgical texts, require extensive knowledge not only of
theology and literature, but also of technical music theory.
Likewise, many of Padilla's musical puns---such as singing \emph{máxima} on a breve,
the thirty-three notes of the \emph{respuesta}, and the quotation of the \emph{tonus
peregrinus}---would probably be accessible only to the performers themselves, if
even they were paying close enough attention.
Given the realities of hurried rehearsal schedules and the practical mindset of
the cathedral performers, much of this artistry would remain hidden, known
only to the composer himself.
Because of its clear text declamation and distinct contrasts of styles, the
piece would still leave an overall impression on the less attentive and educated
listeners, inspiring them to connect the voices of the chapel choir with the
voices of the Christmas angels and the mystery of Christmas.

In this way, the piece functions not only for intellectual contemplation but
also makes an affective appeal to bodily, communal reaction and participation.
This function fits with the instructions in the Roman Catechism that the mystery
of the Incarnation is more to be marvelled at than explained in words.
Similarly, Fray Luis concludes his model Christmas sermon with a \emph{peroratio}
that invites his listeners to join their voices and hearts with the worshippers
at Bethelehem:

> But consider, that if the angels on that day sang and solemnized this mystery
> with \emph{Glorias} and praises, giving thanks \add{\emph{gracias}} for the redemption
> that came to us from heaven, even though they themselves were not the ones
> redeemed, what should we do who are redeemed?
> If they thus give thanks for the grace \add{\emph{gracia}} and mercy given to
> strangers, what should those do who were redeemed and restored by
> it?
[@LuisdeGranada:Xmas, 41:
\quoted{Considera mas, que si los Ángeles en tal día cantaron, y solemnizaron este
misterio con glorias y alabanzas, dando gracias por la redencion que nos vino
del cielo, no siendo ellos los redemidos, ¿qué deben hacer los redemidos?  Si
ellos así dan gracias por la gracia, y misericordia ajena, ¿qué deben hacer los
que fueron redimidos, y reparados por ella?}]

\noindent
The devotee's response of awe, which as we have seen was a central element of
Christmas devotion, is meant to surpass private experience and motivate people
to share their response with others.
Music could both motivate this kind of awe-filled experience and provide a way
to share it.

The theology of Christ as the \emph{Verbum infans} that underlies this villancico is
central for understanding how Catholics believed sacred music to function.
In the Catholic theological tradition of Christmas, Christ as the eternal Word
of God became incarnate, not simply to communicate \quoted{words} from God, but to be
in his body the medium of communication itself, as well as the message.
The point of the Christmas liturgy, then, would not be simply to \quoted{hear about}
God, but to encounter the incarnate Christ in his flesh, through the sacrament
of the Eucharist.
Lapide cites Saint Cyril to say that \quoted{symbolically, the manger is the altar, on
which Christ in the Mass by consecration is as though born and sacrificed}.
Paraphrasing Saint John Chrysostom, Lapide writes, \quoted{That which the Magi saw in
the manger, in a little hut, with much veneration and fear approached and
adored, you perceive the same thing not in the manger, but on the altar}.
[@Lapide:Gospels19C, 672, on Lk 2:
\quoted{Sic et S. Cyrillus in Caten. Symbolice praesepe est altare, in quo Christus in
Missa per consecrationem quasi nascitur et immolatur.
Unde S. Chyrs. in Catena: \emph{Quod in praesepio videntes atque tugurio Magi, cum
multa veneratione ac timore accesserunt et adorarunt: tu idipsum non in
praesepio, sed in altari cernens, majorem istis brabaris exhibe pietatem}}.]
This is the Christmas version of the central theological concept for Roman
Catholics: that the resurrected Christ made himself present and conferred his
power and grace through the sacraments of the Church.
As Leo the Great epitomized the theology, \quoted{that which was plainly
visible in our Savior, passed into the sacraments \add{or signs, or mysteries}}.
[@Leo:SermonesPL, 398, Sermo II de Ascensione Domini:
\quoted{Quod itaque redemptoris nostri conspicuum fuit, in sacramenta transiuit}.]

Anyone wishing to encounter the Christ-child, then, would find him on the altar
in the Christ-Mass.
All the auditory elements of the service, from the reading of Scripture to the
singing of villancicos, could only point toward this sacramental encounter.
But because of music's ability to work on the external sense of hearing and the
internal faculties of the memory via the ethereal medium of the voice,
devotional music could still play a special role for pious listeners---helping
them experience intellectually and affectively something of the wonder and
mystery of the Incarnation at Christmas.
The primary audience of this cultivated subgenre of metamusical villancico would
likely have been the cathedral chapter, who from their privileged position
within the architectural choir would have been best able to hear the verbal and
musical conceits, who had the education to understand the theological
background, and who participated fully in the liturgy by communing.
% XXX  source 
By contrast, since the laity in Puebla and across the Hispanic world would have
been unlikely to eat the Eucharistic wafer or drink the consecrated wine, the
common parishioner's relationship to the \emph{admirabile sacramentum} was primarily
to look (\emph{admirar}), to listen, and to adore.

Worshippers in Puebla who looked toward the altar while listening were presented
with a resplendent vista of images to contemplate.
Padilla's 1657 villancico would have been performed facing a resplendent
high-altar retable on which the actual altar was flanked by an image of the
Adoration of the Shepherds on the left and the Visitation of the Magi on the
right (\cref{fig:Puebla-Ferrer-Retablo}).
This \quoted{Altar of the Kings} was completed by Pedro García Ferrer and consecrated
by bishop Juan de Palafox y Mendoza in 1649.
It places images of human encounters with the incarnate Christ at the closest
visual proximity to the altar and central Eucharistic tabernacle.
[@Gali:GarciaFerrer]
Above these images, the eye is drawn upward to ascend into heaven along with
the Virgin Mary, for Catholics the paragon of true faith and devotion, and the
central focus of the Puebla community's intense devotion to the Immaculate
Conception.
Mary is ascending to heaven through a celestial hierarchy  into a realm
of baby angels playing musical instruments like harp and organ, through a
group of cherubim dancing in the round, to the Holy Trinity at the highest
point (\cref{fig:Puebla-Ferrer-BMV}).[^celestial-hierarchy]
These figures drew devotees' attention to the other central object of devotion
in this \quoted{city of the angels}, which residents believed was built on a site
revealed in a dream to the bishop of Tlaxcala by angels.
[@Lomeli:Puebla, 67--68; @AngelContreras:Puebla, 21]

[^celestial-hierarchy]:
The \emph{Celestial Hierarchy} attributed to Saint Dionysius the Areopagite was
widely disseminated across the Hispanic world, including a copy in Puebla's
Biblioteca Palafoxiana.

(insert \cref{fig:Puebla-Ferrer-Retablo})
\label{fig:Puebla-Ferrer-Retablo}

(insert \cref{fig:Puebla-Ferrer-BMV})
\label{fig:Puebla-Ferrer-BMV}

Like Bishop Palafox's own devotional book \emph{El Pastor de Nochebuena}, these
images invite worshippers to imagine themselves journeying with the shepherds
and Gentiles to meet Christ in his lowly incarnation as an infant.
[@Palafox:Nochebuena]
In honor of Palafox's book, Ferrer painted Palafox's portrait as one of the
shepherds on the Puebla retable (\cref{fig:Puebla-Ferrer-Pastores}).
[@Merlo:PueblaCat, 188--189]
Journeying with the shepherds meant imitating the shepherds' faith: the goal of
contemplating Christ's birth was to cultivate faith in Christ, including both
faith as belief and faith as action or faithfulness.
Lapide holds up the shepherds to his readers as an example of faith: they heard
the angel chorus and saw the sign of Christ in his manger, and then they went
and told everyone, and \quoted{they went forth praising and glorifying God for everything they had seen and heard} (Lk 2:29).
Lapide says Christmas shepherds were chosen as the first outside Holy Family and
the stable animals to see the newborn Christ because in their own humble poverty
they would not be put off by Christ's lowly birth.
For \quoted{however many might have approached the manger, and seen Christ, but
only those could have believed in Christ, whose hearts God had effectively
moved; while the others, taking offense at Christ's poverty, would have spurned
him}. 
[@Lapide:Gospels19C, 677, on Lk 2: 
\quoted{Hinc patet pastores multis narrase eq quae de Christo nato ab angelo adierant et
viderant: quare multos praesepe adiise, Christumque vidisse; sed eos solos in
Christo credidisse, quorum Deus corda efficaciter tangebat, ceteros offensos
paupertate Christi eum sprevisse}.]
The shepherds not only heard and saw, but believed because they freely received
the gift of faith, as Lapide explains in his commentary on John 1:12:
\quoted{God gives the power to become sons of God to those who freely receive Christ by faith and obedience, excluding those who do not wish to receive him}.
[@Lapide:Gospels19C, 882: 
\quoted{Hoc est enim quot ait hic St. Joanes, Deum dedisse potestatem filios Dei his qui libere Christum per fidem et obedientiam receperunt, exclusis iis, qui eum recipere noluerunt}. Lapide seems to have a polemical eye here on his
Calvinist compatriots in the Low Countries.]
If the listeners were to imitate the shepherds, then, they would not only listen
for the angelic choirs and look for Christ's presence sacramentally on the
altar, but they would allow God to move their hearts to receive him in
faith---a faith that could motivate them to witness their faith to others, as
the shepherds did.
Metamusical Christmas villancicos appealed to the hearing of worshippers on
several levels, and opened the possibility for them to listen in faith for the
voices of angels and the voice of Christ.
Through compositional craft, the musical performance could also move listeners'
hearts to wonder and adoration.

(insert \cref{fig:Puebla-Ferrer-Pastores})
\label{fig:Puebla-Ferrer-Pastores}

We have identified two main functions for this villancico, and others like it:
first, it served the congregation as an object of devotion; and second, it
enabled the composer to establish a musical pedigree.
These contrasting functions may be harmonized by remembering the theological
emphasis on wonder, amazement, even \quoted{stupefaction} (to use Lapide's term) in
response to the voices of Christmas and the voice of the Word made
flesh.
In Luke's account, the shepherds are struck with terror when the angel speaks to
them; when the shepherds tell what they heard (presumably to Mary and Joseph
in the stable), \quoted{all who heard were amazed}, both at the shepherds' words and
those the shepherds had heard from the angels.
The Virgin Mary, in turn, \quoted{treasured all these things \add{\emph{verba}} within her heart} (Lk 2:19).
If the goal of Christmas music in seventeenth-century Puebla and Seville was to
echo those historical voices of Christmas and provoke the same kind of awestruck
wonder, then we can understand the efforts of poets and chapelmasters (and the
institutions that paid them) to impress and even confound hearers through
ingenious auditory artistry.
Listeners' relationship to a villancico like \wtitle{Voces, las de la capilla} would
mirror their relationship to the mysterious and logic-defying theology of the
Incarnation: those who understood little could still be amazed greatly; those
who strained to understand more would only find themselves more in awe.

