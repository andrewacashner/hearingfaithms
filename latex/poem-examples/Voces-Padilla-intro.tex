% Voces las de la capilla
% Introducción
\documentclass[class=vcbook,preview]{standalone}
\begin{document}
\floatfontsize
\numberlinetrue
\begin{poemtranslation}
\begin{original}

\StanzaSection{6}[\add{Introducción}]
1. Voces, las de la capilla, &
cuenta con lo que se canta, &
que es músico el rey, y nota &
las más leves disonancias &
a lo de Jesús infante &
y a lo de David monarca.
\SectionBreak

\StanzaSection{4}[Respuesta]
Puntos ponen a sus letras &
los siglos de sus hazañas. &
La clave que sobre el hombro &
para el treinta y tres se aguarda.
\SectionBreak

\StanzaSection{6}[\add{Introducción} cont.]
2. Años antes la divisa, &
la destreza en la esperanza, &
por sol comienza una gloria, &
por mi se canta una gracia, &
y a medio compás la noche &
remeda quiebros del alba.
\SectionBreak[\add{Repeat Respuesta}]

\end{original}

\begin{translation}
\StanzaSection{6}
1. Voices, those of the chapel choir, &
keep count with what is sung, &
for the king is a musician, and he notes &
even the most venial dissonances, &
in the manner of Jesus the infant prince, &
as in the manner of David the monarch. \&

\StanzaSection{4}
The centuries of his heroic exploits &
are putting notes to his lyrics. &
The key that upon his shoulder &
awaits the thirty-three. \&

\StanzaSection{6}
2. Years before the sign, &
dexterity in hope, & 
with the sun \add{on \term{sol}} a \textquote{glory} begins, &
upon me \add{\term{mi}} a \textquote{grace} is sung, &
and at the half-measure, the night &
imitates the trills of the dawn. \&

\end{translation}
\end{poemtranslation}
\end{document}
